\dicLetter{ó}{letter19}
\dicEntry[ó] \dicTerm{ó\smash{\textsuperscript{1}}} \dicsymFrequent\  \dicIPA{{ou}{\textlengthmark}} \dicPos{inter} \textbf{1.} \dicDirectTranslationCS{ó, oh, och} \dicIndirectTranslationCS{(vyjadřuje různé pocity)} \dicExampleIS{ó nei} \dicExampleCS{oh ne}  \textbf{2.} \dicDirectTranslationCS{au} \dicIndirectTranslationCS{(vyjadřuje bolest)}
\dicEntry[ó] \dicTerm{ó-\smash{\textsuperscript{2}}} \dicIPA{{ou}{\textlengthmark}} \dicPos{predp} \dicDirectTranslationCS{ne-} \dicIndirectTranslationCS{(předpona vyjadřující zápor)} \dicExampleIS{óviss} \dicExampleCS{nejistý}
\dicEntry[óa] \dicTerm{ó|a} \dicIPA{{ou}{\textlengthmark}{a}} \dicPos{v}[1] \dicFlx{(‑aði)}[81] \dicPhraseIS{e‑n\,/\addthin e‑m óar} \dicFlx{impers} \dicSynonym*{e‑r óttast} \dicDirectTranslationCS{(kdo) se bojí, (kdo) se děsí} \dicExampleIS{Mig óar við því.} \dicExampleCS{Děsím se toho.}
\dicEntry[óaðfinnanlega] \dicTerm{ó··að·finnan·lega} \dicIPA{{ou}{\textlengthmark}{a}{\texttheta}{f}{\textsci}{n}{a}{n}{l}{\textepsilon}{\textbabygamma}{a}} \dicPos{adv} \dicDirectTranslationCS{bezvadně, dokonale, perfektně}
\dicEntry[óaðfinnanlegur] \dicTerm{ó··að·finnan·legur} \dicIPA{{ou}{\textlengthmark}{a}{\texttheta}{f}{\textsci}{n}{a}{n}{l}{\textepsilon}{\textbabygamma}{\textscy}{\textsubring{r}}} \dicPos{adj}[1]\dicFlx{}[-8] \dicSynonym{gallalaus} \dicDirectTranslationCS{bezvadný, dokonalý} \dicExampleIS{Frágangurinn á bókinni er óaðfinnanlegur.} \dicExampleCS{Konec knížky je dokonalý.}
\dicEntry[óaðgengilegur] \dicTerm{ó··að·gengi·legur} \dicIPA{{ou}{\textlengthmark}{a}{ð}{\r{\textObardotlessj}}{ei}{\textltailn}{\r{\textObardotlessj}}{\textsci}{l}{\textepsilon}{\textbabygamma}{\textscy}{\textsubring{r}}} \dicPos{adj}[1]\dicFlx{}[-8] \textbf{1.} \dicSynonym{ótækur} \dicDirectTranslationCS{nepřijatelný, nepřípustný} \dicExampleIS{óaðgengileg skýring} \dicExampleCS{nepřijatelné vysvětlení} \dicAntonym{aðgengilegur}  \textbf{2.} \dicDirectTranslationCS{nepřístupný, nedostupný (vchod ap.)}  \textbf{3.} \dicSynonym*{illskiljanlegur} \dicDirectTranslationCS{nedostupný, nesrozumitelný (článek ap.)}
\dicEntry[óaðgætinn] \dicTerm{ó··að·gætinn} \dicIPA{{ou}{\textlengthmark}{a}{ð}{\r{\textObardotlessj}}{a}{i}{\textsubring{d}}{\textsci}{\textsubring{n}}} \dicPos{adj}[6]\dicFlx{}[-2] \dicSynonym{gálaus} \dicDirectTranslationCS{neopatrný, neobezřetný} \dicAntonym{aðgætinn}
\dicEntry[óaðskiljanlegur] \dicTerm{ó··að·skiljan·legur} \dicIPA{{ou}{\textlengthmark}{a}{\texttheta}{s}{\r{\textObardotlessj}}{\textsci}{l}{j}{a}{n}{l}{\textepsilon}{\textbabygamma}{\textscy}{\textsubring{r}}} \dicPos{adj}[1]\dicFlx{}[-8] \textbf{1.} \dicDirectTranslationCS{neoddělitelný, nedílný} \dicExampleIS{Þetta svæði er óaðskiljanlegur hluti ríkisins.} \dicExampleCS{Toto území je neoddělitelnou částí státu.}  \textbf{2.} \dicDirectTranslationCS{nerozlučný, trvalý} \dicExampleIS{Þeir eru óaðskiljanlegir vinir.} \dicExampleCS{Jsou to nerozluční kamarádi.}
\dicEntry[óafmáanlegur] \dicTerm{ó··af·máan·legur} \dicIPA{{ou}{\textlengthmark}{a}{v}{m}{au}{a}{n}{l}{\textepsilon}{\textbabygamma}{\textscy}{\textsubring{r}}} \dicPos{adj}[1]\dicFlx{}[-8] \dicDirectTranslationCS{nesmazatelný, neodstranitelný} \dicExampleIS{óafmáanleg ummerki} \dicExampleCS{nesmazatelné znamení}
\dicEntry[óafsakanlegur] \dicTerm{ó··af·sakan·legur} \dicIPA{{ou}{\textlengthmark}{a}{f}{s}{a}{\r{g}}{a}{n}{l}{\textepsilon}{\textbabygamma}{\textscy}{\textsubring{r}}} \dicPos{adj}[1]\dicFlx{}[-8] \dicSynonym{óforsvaranlegur} \dicDirectTranslationCS{neomluvitelný, neprominutelný} \dicExampleIS{óafsakanleg framkoma} \dicExampleCS{neomluvitelné chování}
\dicEntry[óafturkallanlegur] \dicTerm{ó··aftur·kallan·legur} \dicIPA{{ou}\-{\textlengthmark}\-{a}\-{f}\-{\textsubring{d}}\-{\textscy}\-{\textsubring{r}}\-{k\smash{\textsuperscript{h}}}\-{a}\-{\textsubring{d}}\-{l}\-{a}\-{n}\-{l}\-{\textepsilon}\-{\textbabygamma}\-{\textscy}\-{\textsubring{r}}\-} \dicPos{adj}[1]\dicFlx{}[-8] \dicDirectTranslationCS{neodvolatelný, nezrušitelný} \dicExampleIS{óafturkallanleg ákvörðun} \dicExampleCS{neodvolatelné rozhodnutí}
\dicEntry[óafturkræfur] \dicTerm{ó··aftur·kræfur} \dicIPA{{ou}{\textlengthmark}{a}{f}{\textsubring{d}}{\textscy}{\textsubring{r}}{k\smash{\textsuperscript{h}}}{r}{a}{i}{v}{\textscy}{\textsubring{r}}} \dicPos{adj}[1]\dicFlx{}[-6] \dicDirectTranslationCS{nereklamovatelný, nevymahatelný} \dicExampleIS{óafturkræft lán} \dicExampleCS{nevymahatelná půjčka} \dicAntonym{afturkræfur}
\dicEntry[óaldarflokkur] \dicTerm{ó·aldar··flokk|ur} \dicIPA{{ou}{\textlengthmark}{a}{l}{\textsubring{d}}{a}{\textsubring{r}}{f}{l}{\textopeno}{h}{\r{g}}{\textscy}{\textsubring{r}}} \dicPos{m}[6] \dicFlx{(‑s, ‑ar)}[8] \dicLink{óaldarlýður}
\dicEntry[óaldarlýður] \dicTerm{ó·aldar··lýð|ur}\dicTerm{, óaldarflokkur} \dicIPA{{ou}\-{\textlengthmark}\-{a}\-{l}\-{\textsubring{d}}\-{a}\-{r}\-{l}\-{i}\-{ð}\-{\textscy}\-{\textsubring{r}}\-} \dicPos{m}[6] \dicFlx{(‑s)}[17] \dicSynonym*{spellvirkjar} \dicDirectTranslationCS{banda, gang (chuligánů ap.)}
\dicEntry[óalgengur] \dicTerm{ó··al·gengur} \dicIPA{{ou}{\textlengthmark}{a}{l}{\r{\textObardotlessj}}{ei}{\ng}{\r{g}}{\textscy}{\textsubring{r}}} \dicPos{adj}[1]\dicFlx{}[-1] \dicSynonym{sjaldgæfur} \dicDirectTranslationCS{nezvyklý, nevšední, neobyčejný} \dicAntonym{algengur}
\dicEntry[óarðbær] \dicTerm{ó··arð·bær} \dicIPA{{ou}{\textlengthmark}{a}{r}{ð}{\textsubring{b}}{a}{i}{\textsubring{r}}} \dicPos{adj}[5] \dicFlx{(f ‑)}[8] \dicSynonym{gagnslaus} \dicDirectTranslationCS{neziskový, nerentabilní} \dicAntonym{arðbær}
\dicEntry[óáfengur] \dicTerm{ó··á·fengur} \dicIPA{{ou}{\textlengthmark}{au}{f}{ei}{\ng}{\r{g}}{\textscy}{\textsubring{r}}} \dicPos{adj}[1]\dicFlx{}[-1] \dicDirectTranslationCS{nealkoholický} \dicAntonym{áfengur}
\dicEntry[óákv.] \dicTerm{óákv.} \dicPos{zkr} \dicPhraseIS{óákveðinn} \dicDirectTranslationCS{neurčitý}
\dicEntry[óákveðinn] \dicTerm{ó··á·kveðinn} \dicIPA{{ou}{\textlengthmark}{au}{\r{g}}{v}{\textepsilon}{ð}{\textsci}{\textsubring{n}}} \dicPos{adj}[6]\dicFlx{}[-2] \textbf{1.} \dicSynonym{ótiltekinn} \dicDirectTranslationCS{neurčitý, nejasný} \dicAntonym{ákveðinn};  \dicPhraseIS{óákveðinn greinir} \dicDirectTranslationCS{neurčitý člen};  \dicPhraseIS{óákveðið fornafn} \dicFieldCat{jaz.} \dicDirectTranslationCS{zájmeno neurčité}  \textbf{2.} \dicDirectTranslationCS{nerozhodný, váhavý} \dicExampleIS{Kjósendur eru óákveðnir.} \dicExampleCS{Voliči jsou nerozhodní.}
\dicEntry[óálitlegur] \dicTerm{ó··á·lit·legur} \dicIPA{{ou}{\textlengthmark}{au}{l}{\textsci}{\textsubring{d}}{l}{\textepsilon}{\textbabygamma}{\textscy}{\textsubring{r}}} \dicPos{adj}[1]\dicFlx{}[-8] \dicSynonym{óefnilegur} \dicDirectTranslationCS{nevýrazný, neatraktivní} \dicAntonym{álitlegur}
\dicEntry[óánægður] \dicTerm{ó··á·nægður} \dicIPA{{ou}{\textlengthmark}{au}{n}{a}{i}{\textbabygamma}{ð}{\textscy}{\textsubring{r}}} \dicPos{adj}[2]\dicFlx{}[-1] \dicSynonym{óhamingjusamur} \dicDirectTranslationCS{nespokojený, nešťastný} \dicAntonym{ánægður}
\dicEntry[óánægja] \dicTerm{ó··á·nægj|a} \dicIPA{{ou}{\textlengthmark}{au}{n}{a}{i}{j}{a}} \dicPos{f}[1] \dicFlx{(‑u)}[5] \dicSynonym*{vansæla} \dicDirectTranslationCS{nespokojenost, nelibost} \dicExampleIS{Það er óánægja með þessi málalok.} \dicExampleCS{Lidé jsou nespokojení s~tímto závěrem.} \dicAntonym{ánægja}
\dicEntry[óáran] \dicTerm{ó··áran} \dicIPA{{ou}{\textlengthmark}{au}{r}{a}{\textsubring{n}}} \dicPos{f}[4] \dicFlx{(‑ar)}[3] \dicSynonym{harðindi} \dicDirectTranslationCS{špatný rok, špatná úroda\,/\addthin sezóna}
\dicEntry[óáreiðanlegur] \dicTerm{ó··á·reiðan·legur} \dicIPA{{ou}{\textlengthmark}{au}{r}{ei}{ð}{a}{n}{l}{\textepsilon}{\textbabygamma}{\textscy}{\textsubring{r}}} \dicPos{adj}[1]\dicFlx{}[-8] \dicSynonym{ótraustur} \dicDirectTranslationCS{nespolehlivý, nedůvěryhodný} \dicAntonym{áreiðanlegur}
\dicEntry[óáreiðanleiki] \dicTerm{ó··á·reiðan·leik|i} \dicIPA{{ou}{\textlengthmark}{au}{r}{ei}{ð}{a}{n}{l}{ei}{\r{\textObardotlessj}}{\textsci}} \dicPos{m}[1] \dicFlx{(‑a)}[3] \dicDirectTranslationCS{nespolehlivost, nedůvěryhodnost}
\dicEntry[óáreitinn] \dicTerm{ó··á·reitinn} \dicIPA{{ou}{\textlengthmark}{au}{r}{ei}{\textsubring{d}}{\textsci}{\textsubring{n}}} \dicPos{adj}[6]\dicFlx{}[-2] \dicSynonym{friðsamur} \dicDirectTranslationCS{nenásilný, pokojný} \dicAntonym{áreitinn}
\dicEntry[óáreittur] \dicTerm{ó··á·reittur} \dicIPA{{ou}{\textlengthmark}{au}{r}{ei}{h}{\textsubring{d}}{\textscy}{\textsubring{r}}} \dicPos{adj}[1]\dicFlx{}[-13] \dicSynonym*{ótruflaður} \dicDirectTranslationCS{ponechaný v~klidu, nerušený} \dicExampleIS{láta hann óáreittan} \dicExampleCS{nechat ho v~klidu}
\dicEntry[óásjálegur] \dicTerm{ó··á·sjá·legur} \dicIPA{{ou}{\textlengthmark}{au}{s}{j}{au}{l}{\textepsilon}{\textbabygamma}{\textscy}{\textsubring{r}}} \dicPos{adj}[1]\dicFlx{}[-8] \dicSynonym{lítilfjörlegur} \dicDirectTranslationCS{nevzhledný, nepěkný} \dicAntonym{ásjálegur}
\dicEntry[ób.] \dicTerm{ób.} \dicPos{zkr} \dicPhraseIS{óbeygjanlegur} \dicDirectTranslationCS{nesklonný}
\dicEntry[óbanginn] \dicTerm{ó··banginn} \dicIPA{{ou}{\textlengthmark}{\textsubring{b}}{au}{\textltailn}{\r{\textObardotlessj}}{\textsci}{\textsubring{n}}} \dicPos{adj}[6]\dicFlx{}[-3] \dicSynonym{óhræddur} \dicDirectTranslationCS{nebojácný, smělý} \dicAntonym{banginn}
\dicEntry[óbeðinn] \dicTerm{ó··beðinn} \dicIPA{{ou}{\textlengthmark}{\textsubring{b}}{\textepsilon}{ð}{\textsci}{\textsubring{n}}} \dicPos{adj}[6]\dicFlx{}[-6] \dicDirectTranslationCS{nevyžádaný, nepožádaný}
\dicEntry[óbeinn] \dicTerm{ó··beinn} \dicIPA{{ou}{\textlengthmark}{\textsubring{b}}{ei}{\textsubring{d}}{\textsubring{n}}} \dicPos{adj}[7]\dicFlx{}[-1] \dicDirectTranslationCS{nepřímý} \dicAntonym{beinn};  \dicPhraseIS{óbein ræða} \dicFieldCat{jaz.} \dicDirectTranslationCS{nepřímá řeč}
\dicEntry[óbeit] \dicTerm{ó··beit} \dicIPA{{ou}{\textlengthmark}{\textsubring{b}}{ei}{\textsubring{d}}} \dicPos{f}[7] \dicFlx{(‑ar)}[3] \dicSynonym{andstyggð} \dicDirectTranslationCS{odpor, antipatie, averze} \dicExampleIS{hafa óbeit á e‑u} \dicExampleCS{mít k~(čemu) odpor}
\dicEntry[óbetranlegur] \dicTerm{ó··betran·legur} \dicIPA{{ou}{\textlengthmark}{\textsubring{b}}{\textepsilon}{\textsubring{d}}{r}{a}{n}{l}{\textepsilon}{\textbabygamma}{\textscy}{\textsubring{r}}} \dicPos{adj}[1]\dicFlx{}[-8] \dicSynonym{óbætanlegur} \dicDirectTranslationCS{nepolepšitelný, nenapravitelný, notorický}
\dicEntry[óbeygjanlegur] \dicTerm{ó··beygjan·legur} \dicIPA{{ou}{\textlengthmark}{\textsubring{b}}{ei}{j}{a}{n}{l}{\textepsilon}{\textbabygamma}{\textscy}{\textsubring{r}}} \dicPos{adj}[1]\dicFlx{}[-8] \textbf{1.} \dicSynonym{ósveigjanlegur} \dicDirectTranslationCS{neohebný}  \textbf{2.} \dicFieldCat{jaz.} \dicDirectTranslationCS{neohebný} \dicAntonym{beygjanlegur}
\dicEntry[óbifandi] \dicTerm{ó··bif·andi} \dicIPA{{ou}{\textlengthmark}{\textsubring{b}}{\textsci}{v}{a}{n}{\textsubring{d}}{\textsci}} \dicPos{adj}[13] \dicFlx{indecl}[1] \dicSynonym{bjargfastur} \dicDirectTranslationCS{neochvějný} \dicExampleIS{óbifandi trú} \dicExampleCS{neochvějná víra}
\dicEntry[óbifanlegur] \dicTerm{ó··bifan·legur} \dicIPA{{ou}{\textlengthmark}{\textsubring{b}}{\textsci}{v}{a}{n}{l}{\textepsilon}{\textbabygamma}{\textscy}{\textsubring{r}}} \dicPos{adj}[1]\dicFlx{}[-8] \dicSynonym*{óraskanlegur} \dicDirectTranslationCS{neochvějný, zatvrzelý} \dicExampleIS{óbifanleg tryggð} \dicExampleCS{neochvějná věrnost}
\dicEntry[óbilandi] \dicTerm{ó··bil·andi} \dicIPA{{ou}{\textlengthmark}{\textsubring{b}}{\textsci}{l}{a}{n}{\textsubring{d}}{\textsci}} \dicPos{adj}[13] \dicFlx{indecl}[1] \dicSynonym{óbifandi} \dicDirectTranslationCS{neochvějný, neotřesitelný} \dicExampleIS{óbilandi starfsþrek} \dicExampleCS{neochvějná pracovní výdrž}
\dicEntry[óbilgirni] \dicTerm{ó··bil·girn|i} \dicIPA{{ou}{\textlengthmark}{\textsubring{b}}{\textsci}{l}{\r{\textObardotlessj}}{\textsci}{r}{\textsubring{d}}{n}{\textsci}} \dicPos{f}[3] \dicFlx{(‑i)}[3] \dicSynonym{frekja} \dicDirectTranslationCS{neústupnost, nesmlouvavost} \dicAntonym{bilgirni}
\dicEntry[óbilgjarn] \dicTerm{ó··bil·|gjarn} \dicIPA{{ou}{\textlengthmark}{\textsubring{b}}{\textsci}{l}{\r{\textObardotlessj}}{a}{r}{\textsubring{d}}{\textsubring{n}}} \dicPos{adj}[5] \dicFlx{(f ‑gjörn)}[6] \dicSynonym{ósanngjarn} \dicDirectTranslationCS{neústupný, nekompromisní, nesmlouvavý} \dicExampleIS{vera frekur og óbilgjarn} \dicExampleCS{být drzý a~nesmlouvavý}
\dicEntry[óbjörgulegur] \dicTerm{ó··björgu·legur} \dicIPA{{ou}{\textlengthmark}{\textsubring{b}}{j}{\oe}{r}{\r{g}}{\textscy}{l}{\textepsilon}{\textbabygamma}{\textscy}{\textsubring{r}}} \dicPos{adj}[1]\dicFlx{}[-8] \dicSynonym{vandræðalegur} \dicDirectTranslationCS{beznadějný, zoufalý} \dicAntonym{björgulegur}
\dicEntry[óblandaður] \dicTerm{ó··|bland·aður} \dicIPA{{ou}{\textlengthmark}{\textsubring{b}}{l}{a}{n}{\textsubring{d}}{a}{ð}{\textscy}{\textsubring{r}}} \dicPos{adj}[3] \dicFlx{(f ‑blönduð)}[2] \dicSynonym{hreinn} \dicDirectTranslationCS{nemíchaný, neředěný} \dicAntonym{blandaður}
\dicEntry[óblandinn] \dicTerm{ó··blandinn} \dicIPA{{ou}{\textlengthmark}{\textsubring{b}}{l}{a}{n}{\textsubring{d}}{\textsci}{\textsubring{n}}} \dicPos{adj}[6]\dicFlx{}[-4] \dicDirectTranslationCS{čistý, ryzí} \dicExampleIS{óblandin ánægja} \dicExampleCS{čistá radost} \dicAntonym{blandinn}
\dicEntry[óblíður] \dicTerm{ó··blíður} \dicIPA{{ou}{\textlengthmark}{\textsubring{b}}{l}{i}{ð}{\textscy}{\textsubring{r}}} \dicPos{adj}[2]\dicFlx{}[-6] \dicSynonym{byrstur} \dicDirectTranslationCS{nevlídný, nepříjemný} \dicAntonym{blíður}
\dicEntry[óboðinn] \dicTerm{ó··boðinn} \dicIPA{{ou}{\textlengthmark}{\textsubring{b}}{\textopeno}{ð}{\textsci}{\textsubring{n}}} \dicPos{adj}[6]\dicFlx{}[-6] \dicDirectTranslationCS{nezvaný, nepozvaný} \dicExampleIS{óboðinn gestur} \dicExampleCS{nezvaný host}
\dicEntry[óboðlegur] \dicTerm{ó··boð·legur} \dicIPA{{ou}{\textlengthmark}{\textsubring{b}}{\textopeno}{ð}{l}{\textepsilon}{\textbabygamma}{\textscy}{\textsubring{r}}} \dicPos{adj}[1]\dicFlx{}[-8] \dicDirectTranslationCS{nereprezentativní, nevhodný k~nabídce} \dicAntonym{boðlegur}
\dicEntry[óborganlegur] \dicTerm{ó··borgan·legur} \dicIPA{{ou}{\textlengthmark}{\textsubring{b}}{\textopeno}{r}{\r{g}}{a}{n}{l}{\textepsilon}{\textbabygamma}{\textscy}{\textsubring{r}}} \dicPos{adj}[1]\dicFlx{}[-8] \textbf{1.} \dicDirectTranslationCS{nezaplatitelný}  \textbf{2.} \dicSynonym{ómetanlegur} \dicDirectTranslationCS{neocenitelný, (jsoucí) k~nezaplacení}
\dicEntry[óbó] \dicTerm{óbó} \dicIPA{{ou}{\textlengthmark}{\textsubring{b}}{ou}} \dicPos{n}[2] \dicFlx{(‑s, ‑)}[34] \dicFieldCat{hud.} \dicSynonym*{hápípa} \dicDirectTranslationCS{hoboj}
\dicEntry[óbótamaður] \dicTerm{ó·bóta··|maður} \dicIPA{{ou}{\textlengthmark}{\textsubring{b}}{ou}{\textsubring{d}}{a}{m}{a}{ð}{\textscy}{\textsubring{r}}} \dicPos{m}[13] \dicFlx{(‑manns, ‑menn)}[2] \dicSynonym{illvirki\smash{\textsuperscript{1}}} \dicDirectTranslationCS{zločinec, zločinkyně, zlosyn(ka)}
\dicEntry[óbragð] \dicTerm{ó··bragð} \dicIPA{{ou}{\textlengthmark}{\textsubring{b}}{r}{a}{\textbabygamma}{\texttheta}} \dicPos{n}[2] \dicFlx{(‑s)}[2] \dicDirectTranslationCS{pachuť, mdlá chuť}
\dicEntry[óbreytanlegur] \dicTerm{ó··breytan·legur} \dicIPA{{ou}{\textlengthmark}{\textsubring{b}}{r}{ei}{\textsubring{d}}{a}{n}{l}{\textepsilon}{\textbabygamma}{\textscy}{\textsubring{r}}} \dicPos{adj}[1]\dicFlx{}[-8] \dicSynonym*{óbreytilegur} \dicDirectTranslationCS{neměnný, stálý} \dicAntonym{breytanlegur}
\dicEntry[óbreyttur] \dicTerm{ó··breyttur} \dicsymFrequent\  \dicIPA{{ou}{\textlengthmark}{\textsubring{b}}{r}{ei}{h}{\textsubring{d}}{\textscy}{\textsubring{r}}} \dicPos{adj}[1]\dicFlx{}[-10] \textbf{1.} \dicSynonym{samur} \dicDirectTranslationCS{nezměněný, stejný, původní} \dicAntonym{breyttur}  \textbf{2.} \dicSynonym{hversdagslegur} \dicDirectTranslationCS{obyčejný, řadový} \dicExampleIS{óbreyttur kjósandi} \dicExampleCS{řadový volič};  \dicPhraseIS{óbreyttur borgari} \dicDirectTranslationCS{civilista, civilistka, řadový občan, řadová občanka};  \dicPhraseIS{óbreyttur hermaður} \dicDirectTranslationCS{vojín}
\dicEntry[óbrigðull] \dicTerm{ó··brigðull} \dicIPA{{ou}{\textlengthmark}{\textsubring{b}}{r}{\textsci}{\textbabygamma}{ð}{\textscy}{\textsubring{d}}{\textsubring{l}}} \dicPos{adj}[8]\dicFlx{}[-4] \dicSynonym{áreiðanlegur} \dicDirectTranslationCS{hodnověrný, jistý, spolehlivý} \dicExampleIS{óbrigðult ráð} \dicExampleCS{spolehlivá rada} \dicAntonym{brigðull}
\dicEntry[óbrjótandi] \dicTerm{ó··brjót·andi} \dicIPA{{ou}{\textlengthmark}{\textsubring{b}}{r}{j}{ou}{\textsubring{d}}{a}{n}{\textsubring{d}}{\textsci}} \dicPos{adj}[13] \dicFlx{indecl}[1] \dicDirectTranslationCS{nerozbitný, nezničitelný}
\dicEntry[óbrotinn] \dicTerm{ó··brotinn} \dicIPA{{ou}{\textlengthmark}{\textsubring{b}}{r}{\textopeno}{\textsubring{d}}{\textsci}{\textsubring{n}}} \dicPos{adj}[6]\dicFlx{}[-6] \textbf{1.} \dicSynonym{heill\smash{\textsuperscript{2}}} \dicDirectTranslationCS{nerozbitý, neporušený}  \textbf{2.} \dicSynonym{venjulegur} \dicDirectTranslationCS{obyčejný, prostý, normální} \dicExampleIS{óbrotinn sjómaður} \dicExampleCS{obyčejný rybář}
\dicEntry[óbugaður] \dicTerm{ó··bug·|aður} \dicIPA{{ou}{\textlengthmark}{\textsubring{b}}{\textscy}{\textbabygamma}{a}{ð}{\textscy}{\textsubring{r}}} \dicPos{adj}[3] \dicFlx{(f ‑uð)}[4] \dicDirectTranslationCS{nezlomený, nezdrcený}
\dicEntry[óbundinn] \dicTerm{ó··bundinn} \dicIPA{{ou}{\textlengthmark}{\textsubring{b}}{\textscy}{n}{\textsubring{d}}{\textsci}{\textsubring{n}}} \dicPos{adj}[6]\dicFlx{}[-2] \dicSynonym{frjáls} \dicDirectTranslationCS{nespoutaný, nevázaný} \dicExampleIS{hafa óbundnar hendur} \dicExampleCS{mít volné ruce} \dicAntonym{bundinn};  \dicPhraseIS{óbundið mál} \dicFieldCat{lit.} \dicDirectTranslationCS{próza};  \dicPhraseIS{óbundið ljóð} \dicFieldCat{lit.} \dicDirectTranslationCS{volný rým}
\dicEntry[óbyggð] \dicTerm{ó··byggð} \dicIPA{{ou}{\textlengthmark}{\textsubring{b}}{\textsci}{\textbabygamma}{\texttheta}} \dicPos{f}[7] \dicFlx{(‑ar, ‑ir)}[1] \dicSynonym{auðn} \dicDirectTranslationCS{divočina, pustina} \dicAntonym{byggð}
\dicEntry[óbyggður] \dicTerm{ó··byggður} \dicIPA{{ou}{\textlengthmark}{\textsubring{b}}{\textsci}{\textbabygamma}{ð}{\textscy}{\textsubring{r}}} \dicPos{adj}[2]\dicFlx{}[-1] \textbf{1.} \dicDirectTranslationCS{neobydlený} \dicExampleIS{óbyggt svæði} \dicExampleCS{neobydlené území}  \textbf{2.} \dicDirectTranslationCS{nezastavěný}
\dicEntry[óbyggilegur] \dicTerm{ó··byggi·legur} \dicIPA{{ou}{\textlengthmark}{\textsubring{b}}{\textsci}{\r{\textObardotlessj}}{\textsci}{l}{\textepsilon}{\textbabygamma}{\textscy}{\textsubring{r}}} \dicPos{adj}[1]\dicFlx{}[-8] \dicDirectTranslationCS{neobyvatelný} \dicExampleIS{Landið er óbyggilegt.} \dicExampleCS{Země je neobyvatelná.}
\dicEntry[óbærilegur] \dicTerm{ó··bæri·legur} \dicIPA{{ou}{\textlengthmark}{\textsubring{b}}{a}{i}{r}{\textsci}{l}{\textepsilon}{\textbabygamma}{\textscy}{\textsubring{r}}} \dicPos{adj}[1]\dicFlx{}[-8] \dicSynonym{óþolandi} \dicDirectTranslationCS{nesnesitelný, neúnosný} \dicAntonym{bærilegur}
\dicEntry[óbætanlegur] \dicTerm{ó··bætan·legur} \dicIPA{{ou}{\textlengthmark}{\textsubring{b}}{a}{i}{\textsubring{d}}{a}{n}{l}{\textepsilon}{\textbabygamma}{\textscy}{\textsubring{r}}} \dicPos{adj}[1]\dicFlx{}[-8] \dicSynonym*{bótlaus} \dicDirectTranslationCS{nenahraditelný, nekompenzovatelný} \dicExampleIS{óbætanlegur skaði} \dicExampleCS{nenahraditelná škoda}
\dicEntry[ódauðlegur] \dicTerm{ó··dauð·legur} \dicIPA{{ou}{\textlengthmark}{\textsubring{d}}{\oe i}{ð}{l}{\textepsilon}{\textbabygamma}{\textscy}{\textsubring{r}}} \dicPos{adj}[1]\dicFlx{}[-8] \dicSynonym{eilífur} \dicDirectTranslationCS{nesmrtelný, nehynoucí} \dicExampleIS{ódauðlegt listaverk} \dicExampleCS{nesmrtelné umělecké dílo} \dicAntonym{dauðlegur}
\dicEntry[ódauðleiki] \dicTerm{ó··dauð·leik|i} \dicIPA{{ou}{\textlengthmark}{\textsubring{d}}{\oe i}{ð}{l}{ei}{\r{\textObardotlessj}}{\textsci}} \dicPos{m}[1] \dicFlx{(‑a)}[3] \dicSynonym{eilífð} \dicDirectTranslationCS{nesmrtelnost}
\dicEntry[ódaunn] \dicTerm{ó··daun|n} \dicIPA{{ou}{\textlengthmark}{\textsubring{d}}{\oe i}{\textsubring{d}}{\textsubring{n}}} \dicPos{m}[6] \dicFlx{(‑s)}[43] \dicSynonym{óþefur} \dicDirectTranslationCS{(odporný) puch, zápach, smrad}
\dicEntry[ódrengilegur] \dicTerm{ó··drengi·legur} \dicIPA{{ou}{\textlengthmark}{\textsubring{d}}{r}{ei}{\textltailn}{\r{\textObardotlessj}}{\textsci}{l}{\textepsilon}{\textbabygamma}{\textscy}{\textsubring{r}}} \dicPos{adj}[1]\dicFlx{}[-8] \dicSynonym{ruddalegur} \dicDirectTranslationCS{hanebný, nečestný, neférový} \dicAntonym{drengilegur}
\dicEntry[ódrepandi] \dicTerm{ó··drep·andi} \dicIPA{{ou}{\textlengthmark}{\textsubring{d}}{r}{\textepsilon}{\textsubring{b}}{a}{n}{\textsubring{d}}{\textsci}} \dicPos{adj}[13] \dicFlx{indecl}[1] \textbf{1.} \dicSynonym{lífseigur} \dicDirectTranslationCS{houževnatý, nezdolný}  \textbf{2.} \dicDirectTranslationCS{nezničitelný, věčný}
\dicEntry[ódrukkinn] \dicTerm{ó··drukkinn} \dicIPA{{ou}{\textlengthmark}{\textsubring{d}}{r}{\textscy}{h}{\r{\textObardotlessj}}{\textsci}{\textsubring{n}}} \dicPos{adj}[6]\dicFlx{}[-6] \dicSynonym{algáður} \dicDirectTranslationCS{střízlivý} \dicAntonym{drukkinn}
\dicEntry[óduglegur] \dicTerm{ó··dug·legur} \dicIPA{{ou}{\textlengthmark}{\textsubring{d}}{\textscy}{\textbabygamma}{l}{\textepsilon}{\textbabygamma}{\textscy}{\textsubring{r}}} \dicPos{adj}[1]\dicFlx{}[-8] \dicSynonym{framtakslaus} \dicDirectTranslationCS{nesnaživý, neusilovný} \dicAntonym{duglegur}
\dicEntry[ódugnaður] \dicTerm{ó··dug·nað|ur} \dicIPA{{ou}{\textlengthmark}{\textsubring{d}}{\textscy}{\r{g}}{n}{a}{ð}{\textscy}{\textsubring{r}}} \dicPos{m}[10] \dicFlx{(‑ar)}[9] \dicSynonym{slappleiki} \dicDirectTranslationCS{nevýkonnost, neefektivita} \dicAntonym{dugnaður}
\dicEntry[ódýr] \dicTerm{ó··dýr} \dicsymFrequent\  \dicIPA{{ou}{\textlengthmark}{\textsubring{d}}{i}{\textsubring{r}}} \dicPos{adj}[5] \dicFlx{(f ‑)}[8] \textbf{1.} \dicSynonym*{verðlítill} \dicDirectTranslationCS{levný, laciný} \dicExampleIS{ódýrt vinnuafl} \dicExampleCS{levná pracovní síla} \dicAntonym{dýr\smash{\textsuperscript{2}}}  \textbf{2.} \dicSynonym{einfaldur} \dicDirectTranslationCS{jednoduchý, prostý}  \textbf{3.} \dicSynonym{ómerkilegur} \dicDirectTranslationCS{nevýrazný, nijaký, laciný} \dicExampleIS{ódýr brandari} \dicExampleCS{laciný vtip}
\dicEntry[ódæði] \dicTerm{ó··dæði} \dicIPA{{ou}{\textlengthmark}{\textsubring{d}}{a}{i}{ð}{\textsci}} \dicPos{n}[2] \dicFlx{(‑s, ‑)}[14] \dicSynonym{níðingsverk} \dicDirectTranslationCS{ukrutnost, zvěrstvo} \dicExampleIS{fremja ódæði} \dicExampleCS{spáchat zvěrstva}
\dicEntry[ódæðismaður] \dicTerm{ó·dæðis··|maður} \dicIPA{{ou}{\textlengthmark}{\textsubring{d}}{a}{i}{ð}{\textsci}{s}{m}{a}{ð}{\textscy}{\textsubring{r}}} \dicPos{m}[13] \dicFlx{(‑manns, ‑menn)}[2] \dicDirectTranslationCS{zlosyn(ka), zločinec, zločinkyně}
\dicEntry[ódæðisverk] \dicTerm{ó·dæðis··verk} \dicIPA{{ou}{\textlengthmark}{\textsubring{d}}{a}{i}{ð}{\textsci}{s}{v}{\textepsilon}{\textsubring{r}}{\r{g}}} \dicPos{n}[2] \dicFlx{(‑s, ‑)}[5] \dicDirectTranslationCS{ukrutný\,/\addthin krutý čin}
\dicEntry[ódæll] \dicTerm{ó··dæll} \dicIPA{{ou}{\textlengthmark}{\textsubring{d}}{a}{i}{\textsubring{d}}{\textsubring{l}}} \dicPos{adj}[8]\dicFlx{}[-1] \dicSynonym{baldinn} \dicDirectTranslationCS{neposlušný, neukázněný} \dicAntonym{dæll}
\dicEntry[óð] \dicTerm{óð} \dicIPA{{ou}{\textlengthmark}{\texttheta}} \dicPos{v} \dicFlx{ind pf sg 1 pers} \dicLink{vaða\smash{\textsuperscript{2}}}
\dicEntry[óðagot] \dicTerm{óða··got} \dicIPA{{ou}{\textlengthmark}{ð}{a}{\r{g}}{\textopeno}{\textsubring{d}}} \dicPos{n}[2] \dicFlx{(‑s)}[2] \dicSynonym{flýtir} \dicDirectTranslationCS{zbrklost, ukvapenost}
\dicEntry[óðal] \dicTerm{óð|al} \dicIPA{{ou}{\textlengthmark}{ð}{a}{\textsubring{l}}} \dicPos{n}[3] \dicFlx{(‑als, ‑ul\,/\addthin ‑öl)}[4] \textbf{1.} \dicDirectTranslationCS{(rodinný) majetek\,/\addthin pozemek} \dicExampleIS{vera staddur heima á óðalinu} \dicExampleCS{být na rodinném majetku}  \textbf{2.} \dicSynonym*{óðal dýrs} \dicDirectTranslationCS{teritorium, revír}  \textbf{3.} \dicFieldCat{mat.} \dicDirectTranslationCS{definiční obor}
\dicEntry[óðalsbóndi] \dicTerm{óðals··|bóndi} \dicIPA{{ou}{\textlengthmark}{ð}{a}{l}{s}{\textsubring{b}}{ou}{n}{\textsubring{d}}{\textsci}} \dicPos{m}[2] \dicFlx{(‑bónda, ‑bændur)}[4] \dicDirectTranslationCS{statkář(ka), farmář(ka)}
\dicEntry[óðamála] \dicTerm{óða··mála} \dicIPA{{ou}{\textlengthmark}{ð}{a}{m}{au}{l}{a}} \dicPos{adj}[13] \dicFlx{indecl}[1] \dicDirectTranslationCS{mluvící rychle\,/\addthin kvapně\,/\addthin zbrkle}
\dicEntry[óðar] \dicTerm{óðar}\dicTerm{, óðara} \dicIPA{{ou}\-{\textlengthmark}\-{ð}\-{a}\-{\textsubring{r}}\-} \dicPos{adv} \dicFlx{comp (pos ótt, sup óðast)} \dicDirectTranslationCS{okamžitě, ihned};  \dicPhraseIS{óðar en} \dicFlx{conj} \dicDirectTranslationCS{jakmile; v~okamžiku, kdy}
\dicEntry[óðara] \dicTerm{óðara} \dicIPA{{ou}{\textlengthmark}{ð}{a}{r}{a}} \dicPos{adv} \dicFlx{comp (pos ótt, sup óðast)} \dicLink{óðar}
\dicEntry[óðfús] \dicTerm{óð··fús} \dicIPA{{ou}{ð}{f}{u}{s}} \dicPos{adj}[5]\dicFlx{}[-1] \dicSynonym{albúinn} \dicDirectTranslationCS{dychtivý, nedočkavý}
\dicEntry[Óðinn] \dicTerm{Óð|inn} \dicIPA{{ou}{\textlengthmark}{ð}{\textsci}{\textsubring{n}}} \dicPos{m}[6] \dicFlx{(‑ins)}[38] \dicFlx{prop} \dicFieldCat{myt.} \dicDirectTranslationCS{Ódin} \dicIndirectTranslationCS{(nejvyšší bůh severské mytologie)}
\dicEntry[óðinshani] \dicTerm{óðins··han|i} \dicIPA{{ou}{\textlengthmark}{ð}{\textsci}{n}{s}{h}{a}{n}{\textsci}} \dicPos{m}[1] \dicFlx{(‑a, ‑ar)}[8] \dicFieldCat{zool.} \dicDirectTranslationCS{lyskonoh úzkozobý} \textit{(l.~{\textLA{Phalaropus lobatus}})}  \dicsymPhoto\ 
\dicFigure{ds_image_odinshani_0_1.jpg}{Óðinshani}{Óðinshani - Dave Menke, PD}
\dicEntry[óðum] \dicTerm{óðum\smash{\textsuperscript{1}}} \dicIPA{{ou}{\textlengthmark}{ð}{\textscy}{\textsubring{m}}} \dicPos{v} \dicFlx{ind pf pl 1 pers} \dicLink{vaða\smash{\textsuperscript{2}}}
\dicEntry[óðum] \dicTerm{óðum\smash{\textsuperscript{2}}} \dicIPA{{ou}{\textlengthmark}{ð}{\textscy}{\textsubring{m}}} \dicPos{adv} \dicSynonym{hratt\smash{\textsuperscript{2}}} \dicDirectTranslationCS{kvapem, chvatem}
\dicEntry[óður] \dicTerm{óð|ur\smash{\textsuperscript{1}}} \dicIPA{{ou}{\textlengthmark}{ð}{\textscy}{\textsubring{r}}} \dicPos{m}[10] \dicFlx{(‑s\,/\addthin ‑ar)}[39] \dicDirectTranslationCS{óda, chvalozpěv}
\dicEntry[óður] \dicTerm{óður\smash{\textsuperscript{2}}} \dicsymFrequent\  \dicIPA{{ou}{\textlengthmark}{ð}{\textscy}{\textsubring{r}}} \dicPos{adj}[2]\dicFlx{}[-6] \textbf{1.} \dicSynonym{geðbilaður} \dicDirectTranslationCS{šílený, ztřeštěný}  \textbf{2.} \dicSynonym*{ofsareiður} \dicDirectTranslationCS{zuřivý, nepříčetný} \dicExampleIS{Hundur geltir sem óður.} \dicExampleCS{Pes zuřivě štěká.}  \textbf{3.} \dicSynonym{hraður} \dicDirectTranslationCS{chvatný, spěšný}
\dicEntry[óeðli] \dicTerm{ó··eðli} \dicIPA{{ou}{\textlengthmark}{\textepsilon}{ð}{l}{\textsci}} \dicPos{n}[2] \dicFlx{(‑s)}[20] \dicSynonym{ónáttúra} \dicDirectTranslationCS{nepřirozenost, nenormálnost} \dicAntonym{eðli}
\dicEntry[óeðlilegur] \dicTerm{ó··eðli·legur} \dicsymFrequent\  \dicIPA{{ou}{\textlengthmark}{\textepsilon}{ð}{l}{\textsci}{l}{\textepsilon}{\textbabygamma}{\textscy}{\textsubring{r}}} \dicPos{adj}[1]\dicFlx{}[-8] \dicSynonym{afbrigðilegur} \dicDirectTranslationCS{nepřirozený, nenormální, neobvyklý} \dicExampleIS{óeðlileg viðbrögð} \dicExampleCS{nepřirozená reakce} \dicAntonym{eðlilegur}
\dicEntry[óefni] \dicTerm{ó··efni} \dicIPA{{ou}{\textlengthmark}{\textepsilon}{\textsubring{b}}{n}{\textsci}} \dicPos{n}[2] \dicFlx{(‑s, ‑)}[14] \dicSynonym{klípa\smash{\textsuperscript{1}}} \dicDirectTranslationCS{potíž, obtížná situace} \dicExampleIS{e‑að er komið í óefni} \dicExampleCS{(co) se dostává do potíží}
\dicEntry[óefnilegur] \dicTerm{ó··efni·legur} \dicIPA{{ou}{\textlengthmark}{\textepsilon}{\textsubring{b}}{n}{\textsci}{l}{\textepsilon}{\textbabygamma}{\textscy}{\textsubring{r}}} \dicPos{adj}[1]\dicFlx{}[-8] \dicSynonym{óbjörgulegur} \dicDirectTranslationCS{nevěštící nic dobrého, neblahý} \dicAntonym{efnilegur}
\dicEntry[óeigingirni] \dicTerm{ó··eigin·girn|i} \dicIPA{{ou}{\textlengthmark}{ei}{j}{\textsci}{n}{\r{\textObardotlessj}}{\textsci}{r}{\textsubring{d}}{n}{\textsci}} \dicPos{f}[3] \dicFlx{(‑i)}[3] \dicDirectTranslationCS{nesobeckost, nezištnost, altruismus}
\dicEntry[óeigingjarn] \dicTerm{ó··eigin·|gjarn} \dicIPA{{ou}{\textlengthmark}{ei}{j}{\textsci}{n}{\r{\textObardotlessj}}{a}{r}{\textsubring{d}}{\textsubring{n}}} \dicPos{adj}[5] \dicFlx{(f ‑gjörn)}[6] \dicSynonym{ósérplæginn} \dicDirectTranslationCS{nesobecký, nezištný, altruistický} \dicAntonym{eigingjarn}
\dicEntry[óeiginlegur] \dicTerm{ó··eigin·legur} \dicIPA{{ou}{\textlengthmark}{ei}{j}{\textsci}{n}{l}{\textepsilon}{\textbabygamma}{\textscy}{\textsubring{r}}} \dicPos{adj}[1]\dicFlx{}[-8] \textbf{1.} \dicSynonym{yfirfærður} \dicDirectTranslationCS{přenesený, obrazný} \dicExampleIS{í óeiginlegri merkingu} \dicExampleCS{v~přeneseném významu} \dicAntonym{eiginlegur}  \textbf{2.} \dicSynonym{óeðlilegur} \dicDirectTranslationCS{nepřirozený, nenormální}
\dicEntry[óeining] \dicTerm{ó··ein·ing} \dicIPA{{ou}{\textlengthmark}{ei}{n}{i}{\ng}{\r{g}}} \dicPos{f}[4] \dicFlx{(‑ar)}[7] \dicDirectTranslationCS{nejednotnost, nesvornost}
\dicEntry[óeinlægur] \dicTerm{ó··ein·lægur} \dicIPA{{ou}{\textlengthmark}{ei}{n}{l}{a}{i}{\textbabygamma}{\textscy}{\textsubring{r}}} \dicPos{adj}[1]\dicFlx{}[-1] \dicSynonym{falskur} \dicDirectTranslationCS{neupřímný, pokrytecký} \dicAntonym{einlægur}
\dicEntry[óeirð] \dicTerm{ó··eirð} \dicIPA{{ou}{\textlengthmark}{ei}{r}{\texttheta}} \dicPos{f}[7] \dicFlx{(‑ar, ‑ir)}[1] \dicSynonym{órói} \dicDirectTranslationCS{neklid, nepokoj}
\dicEntry[óekta] \dicTerm{ó··ekta} \dicIPA{{ou}{\textlengthmark}{\textepsilon}{x}{\textsubring{d}}{a}} \dicPos{adj}[13] \dicFlx{indecl}[1] \dicDirectTranslationCS{falešný, falšovaný} \dicAntonym{ekta}
\dicEntry[óendanlegur] \dicTerm{ó··endan·legur} \dicIPA{{ou}{\textlengthmark}{\textepsilon}{n}{\textsubring{d}}{a}{n}{l}{\textepsilon}{\textbabygamma}{\textscy}{\textsubring{r}}} \dicPos{adj}[1]\dicFlx{}[-8] \dicSynonym{endalaus} \dicDirectTranslationCS{nekonečný, nekončící} \dicAntonym{endanlegur}
\dicEntry[óendurkræfur] \dicTerm{ó··endur·kræfur} \dicIPA{{ou}{\textlengthmark}{\textepsilon}{n}{\textsubring{d}}{\textscy}{\textsubring{r}}{k\smash{\textsuperscript{h}}}{r}{a}{i}{v}{\textscy}{\textsubring{r}}} \dicPos{adj}[1]\dicFlx{}[-6] \dicDirectTranslationCS{nevymahatelný, nedobytný (pohledávka ap.)}
\dicEntry[óf] \dicTerm{óf} \dicIPA{{ou}{\textlengthmark}{f}} \dicPos{v} \dicFlx{ind pf sg 1 pers} \dicLink{vefa}
\dicEntry[ófaglærður] \dicTerm{ó··fag·lærður} \dicIPA{{ou}{\textlengthmark}{f}{a}{\textbabygamma}{l}{a}{i}{r}{ð}{\textscy}{\textsubring{r}}} \dicPos{adj}[2]\dicFlx{}[-4] \dicDirectTranslationCS{nekvalifikovaný} \dicExampleIS{ófaglærður starfsmaður} \dicExampleCS{nekvalifikovaný pracovník}
\dicEntry[ófalsaður] \dicTerm{ó··|fals·aður} \dicIPA{{ou}{\textlengthmark}{f}{a}{l}{s}{a}{ð}{\textscy}{\textsubring{r}}} \dicPos{adj}[3] \dicFlx{(f ‑fölsuð)}[2] \dicSynonym{ósvikinn} \dicDirectTranslationCS{nefalšovaný, skutečný} \dicAntonym{falsaður}
\dicEntry[ófarinn] \dicTerm{ó··farinn} \dicIPA{{ou}{\textlengthmark}{f}{a}{r}{\textsci}{\textsubring{n}}} \dicPos{adj}[6]\dicFlx{}[-4] \textbf{1.} \dicDirectTranslationCS{(stále) přítomný, (ještě) neodešlý (o~lidech ap.)}  \textbf{2.} \dicDirectTranslationCS{(ještě) neprošlý (cesta ap.)}
\dicEntry[ófarir] \dicTerm{ó··farir} \dicIPA{{ou}{\textlengthmark}{f}{a}{r}{\textsci}{\textsubring{r}}} \dicPos{f}[7] \dicFlx{pl}[18] \dicSynonym{ósigur} \dicDirectTranslationCS{neúspěch, nezdar};  \dicPhraseIS{gleðjast yfir óförum e‑rs} \dicFlx{refl} \dicDirectTranslationCS{těšit se z~neúspěchu (koho)};  \dicPhraseIS{hefna ófaranna} \dicDirectTranslationCS{revanšovat se, odvděčit se}
\dicEntry[ófarnaður] \dicTerm{ó··far·nað|ur} \dicIPA{{ou}{\textlengthmark}{f}{a}{r}{\textsubring{d}}{n}{a}{ð}{\textscy}{\textsubring{r}}} \dicPos{m}[10] \dicFlx{(‑ar)}[9] \dicSynonym{óhapp} \dicDirectTranslationCS{neštěstí, nezdar} \dicAntonym{farnaður}
\dicEntry[ófáanlegur] \dicTerm{ó··fáan·legur} \dicIPA{{ou}{\textlengthmark}{f}{au}{a}{n}{l}{\textepsilon}{\textbabygamma}{\textscy}{\textsubring{r}}} \dicPos{adj}[1]\dicFlx{}[-6] \textbf{1.} \dicDirectTranslationCS{nesehnatelný, nedající se sehnat (zboží ap.)} \dicExampleIS{Þessi bók er alveg ófáanleg.} \dicExampleCS{Ta knížka se vůbec nedá sehnat.} \dicAntonym{fáanlegur}  \textbf{2.} \dicSynonym{tregur} \dicDirectTranslationCS{liknavý, zdráhavý}
\dicEntry[ófeiminn] \dicTerm{ó··feiminn} \dicIPA{{ou}{\textlengthmark}{f}{ei}{m}{\textsci}{\textsubring{n}}} \dicPos{adj}[6]\dicFlx{}[-2] \dicDirectTranslationCS{neostýchavý, ostřílený} \dicAntonym{feiminn}
\dicEntry[ófeiti] \dicTerm{ó··feit|i} \dicIPA{{ou}{\textlengthmark}{f}{ei}{\textsubring{d}}{\textsci}} \dicPos{f}[3] \dicFlx{(‑i)}[3] \dicSynonym{megurð} \dicDirectTranslationCS{vyzáblost, hubenost}
\dicEntry[ófélagslyndur] \dicTerm{ó··fé·lags·lyndur} \dicIPA{{ou}{\textlengthmark}{f}{j}{\textepsilon}{l}{a}{x}{s}{l}{\textsci}{n}{\textsubring{d}}{\textscy}{\textsubring{r}}} \dicPos{adj}[2]\dicFlx{}[-14] \dicDirectTranslationCS{nespolečenský}
\dicEntry[óféti] \dicTerm{ó··féti} \dicIPA{{ou}{\textlengthmark}{f}{j}{\textepsilon}{\textsubring{d}}{\textsci}} \dicPos{n}[2] \dicFlx{(‑s, ‑)}[14] \dicDirectTranslationCS{ničema, darebák, darebačka, lotr(yně)} \dicExampleIS{ófétið þitt!} \dicExampleCS{ty ničemo!}
\dicEntry[óflekkaður] \dicTerm{ó··flekk·|aður} \dicIPA{{ou}{\textlengthmark}{f}{l}{\textepsilon}{h}{\r{g}}{a}{ð}{\textscy}{\textsubring{r}}} \dicPos{adj}[3] \dicFlx{(f ‑uð)}[4] \dicSynonym{hreinn} \dicDirectTranslationCS{neposkvrněný, (jsoucí) bez poskvrny};  \dicPhraseIS{hafa óflekkað mannorð} \dicDirectTranslationCS{být bezúhonný, mít neposkvrněnou pověst}
\dicEntry[ófn.] \dicTerm{ófn.} \dicPos{zkr} \dicPhraseIS{óákveðið fornafn} \dicFieldCat{jaz.} \dicDirectTranslationCS{neurčité zájmeno}
\dicEntry[óforbetranlegur] \dicTerm{ó··for·betran·legur} \dicIPA{{ou}{\textlengthmark}{f}{\textopeno}{r}{\textsubring{b}}{\textepsilon}{\textsubring{d}}{r}{a}{n}{l}{\textepsilon}{\textbabygamma}{\textscy}{\textsubring{r}}} \dicPos{adj}[1]\dicFlx{}[-8] \dicSynonym{forhertur} \dicDirectTranslationCS{nenapravitelný, nepolepšitelný}
\dicEntry[óformlegur] \dicTerm{ó··form·legur} \dicIPA{{ou}{\textlengthmark}{f}{\textopeno}{r}{m}{l}{\textepsilon}{\textbabygamma}{\textscy}{\textsubring{r}}} \dicPos{adj}[1]\dicFlx{}[-8] \dicDirectTranslationCS{neformální, nenucený} \dicAntonym{formlegur}
\dicEntry[óforsjáll] \dicTerm{ó··for·sjáll} \dicIPA{{ou}{\textlengthmark}{f}{\textopeno}{\textsubring{r}}{s}{j}{au}{\textsubring{d}}{\textsubring{l}}} \dicPos{adj}[8]\dicFlx{}[-1] \dicSynonym{fyrirhyggjulaus} \dicDirectTranslationCS{neprozíravý, krátkozraký} \dicAntonym{forsjáll}
\dicEntry[óforsjálni] \dicTerm{ó··for·sjáln|i} \dicIPA{{ou}{\textlengthmark}{f}{\textopeno}{\textsubring{r}}{s}{j}{au}{l}{n}{\textsci}} \dicPos{f}[3] \dicFlx{(‑i)}[3] \dicDirectTranslationCS{neprozíravost, krátkozrakost} \dicAntonym{forsjálni}
\dicEntry[óforskammaður] \dicTerm{ó··for·|skamm·aður} \dicIPA{{ou}{\textlengthmark}{f}{\textopeno}{\textsubring{r}}{s}{\r{g}}{a}{m}{a}{ð}{\textscy}{\textsubring{r}}} \dicPos{adj}[3] \dicFlx{(f ‑skömmuð)}[1] \dicSynonym{ósvífinn} \dicDirectTranslationCS{nestydatý, neomalený} \dicExampleIS{óforskömmuð framkoma} \dicExampleCS{neomalené chování}
\dicEntry[óforsvaranlegur] \dicTerm{ó··for·svaran·legur} \dicIPA{{ou}{\textlengthmark}{f}{\textopeno}{\textsubring{r}}{s}{v}{a}{r}{a}{n}{l}{\textepsilon}{\textbabygamma}{\textscy}{\textsubring{r}}} \dicPos{adj}[1]\dicFlx{}[-8] \dicSynonym{óafsakanlegur} \dicDirectTranslationCS{neomluvitelný} \dicExampleIS{óforsvaranlegt kæruleysi} \dicExampleCS{neomluvitelná nedbalost} \dicAntonym{forsvaranlegur}
\dicEntry[óframfærinn] \dicTerm{ó··fram·færinn} \dicIPA{{ou}{\textlengthmark}{f}{r}{a}{m}{f}{a}{i}{r}{\textsci}{\textsubring{n}}} \dicPos{adj}[6]\dicFlx{}[-2] \dicSynonym{feiminn} \dicDirectTranslationCS{zdrženlivý, ostýchavý}
\dicEntry[óframkvæmanlegur] \dicTerm{ó··fram·kvæman·legur} \dicIPA{{ou}{\textlengthmark}{f}{r}{a}{m}{k\smash{\textsuperscript{h}}}{v}{a}{i}{m}{a}{n}{l}{\textepsilon}{\textbabygamma}{\textscy}{\textsubring{r}}} \dicPos{adj}[1]\dicFlx{}[-8] \dicSynonym{ómögulegur} \dicDirectTranslationCS{neproveditelný, neuskutečnitelný, nerealizovatelný, nevykonatelný} \dicAntonym{framkvæmanlegur}
\dicEntry[ófrágenginn] \dicTerm{ó··frá·genginn} \dicIPA{{ou}{\textlengthmark}{f}{r}{au}{\r{\textObardotlessj}}{ei}{\textltailn}{\r{\textObardotlessj}}{\textsci}{\textsubring{n}}} \dicPos{adj}[6]\dicFlx{}[-6] \dicDirectTranslationCS{neprovedený, nedokončený}
\dicEntry[ófrávíkjanlegur] \dicTerm{ó··frá·víkjan·legur} \dicIPA{{ou}{\textlengthmark}{f}{r}{au}{v}{i}{\r{\textObardotlessj}}{a}{n}{l}{\textepsilon}{\textbabygamma}{\textscy}{\textsubring{r}}} \dicPos{adj}[1]\dicFlx{}[-8] \dicSynonym{óhjákvæmilegur} \dicDirectTranslationCS{závazný, nezbytný} \dicExampleIS{ófrávíkjanleg regla} \dicExampleCS{nezbytné pravidlo}
\dicEntry[ófrelsi] \dicTerm{ó··frelsi} \dicIPA{{ou}{\textlengthmark}{f}{r}{\textepsilon}{l}{s}{\textsci}} \dicPos{n}[2] \dicFlx{(‑s)}[20] \dicSynonym{kúgun} \dicDirectTranslationCS{nesvoboda, útlak, oprese} \dicAntonym{frelsi}
\dicEntry[ófreskja] \dicTerm{ó··freskj|a} \dicIPA{{ou}{\textlengthmark}{f}{r}{\textepsilon}{s}{\r{\textObardotlessj}}{a}} \dicPos{f}[1] \dicFlx{(‑u, ‑ur)}[17] \dicSynonym{skrímsli} \dicDirectTranslationCS{nestvůra, netvor}
\dicEntry[ófriðarseggur] \dicTerm{ó·friðar··segg|ur} \dicIPA{{ou}{\textlengthmark}{f}{r}{\textsci}{ð}{a}{\textsubring{r}}{s}{\textepsilon}{\r{g}}{\textscy}{\textsubring{r}}} \dicPos{m}[9] \dicFlx{(‑s, ‑ir)}[15] \dicSynonym*{friðarspillir} \dicDirectTranslationCS{problémový člověk, rebel(ka)}
\dicEntry[ófriður] \dicTerm{ó··frið|ur} \dicIPA{{ou}{\textlengthmark}{f}{r}{\textsci}{ð}{\textscy}{\textsubring{r}}} \dicPos{m}[10] \dicFlx{(‑ar)}[7] \textbf{1.} \dicSynonym{stríð} \dicDirectTranslationCS{válka, (válečný) konflikt} \dicExampleIS{Þetta getur kveikt ófrið.} \dicExampleCS{To může vyvolat válku.} \dicAntonym{friður}  \textbf{2.} \dicSynonym*{ósamkomulag} \dicDirectTranslationCS{konflikt, různice, neshoda}
\dicEntry[ófríður] \dicTerm{ó··fríður} \dicIPA{{ou}{\textlengthmark}{f}{r}{i}{ð}{\textscy}{\textsubring{r}}} \dicPos{adj}[2]\dicFlx{}[-6] \dicSynonym{ólaglegur} \dicDirectTranslationCS{nevzhledný, nehezký} \dicAntonym{fríður}
\dicEntry[ófríkka] \dicTerm{ó··fríkk|a} \dicIPA{{ou}{\textlengthmark}{f}{r}{i}{h}{\r{g}}{a}} \dicPos{v}[1] \dicFlx{(‑aði)}[34] \dicDirectTranslationCS{(z)ošklivět, stát\,/\addthin stávat se ošklivým} \dicAntonym{fríkka}
\dicEntry[ófrískur] \dicTerm{ó··frískur} \dicIPA{{ou}{\textlengthmark}{f}{r}{i}{s}{\r{g}}{\textscy}{\textsubring{r}}} \dicPos{adj}[1]\dicFlx{}[-1] \textbf{1.} \dicSynonym{lasinn} \dicDirectTranslationCS{nemocný, nezdravý}  \textbf{2.} \dicPhraseIS{ófrísk} \dicFlx{f} \dicSynonym*{þunguð} \dicDirectTranslationCS{těhotná}
\dicEntry[ófrjáls] \dicTerm{ó··frjáls} \dicIPA{{ou}{\textlengthmark}{f}{r}{j}{au}{l}{s}} \dicPos{adj}[5]\dicFlx{}[-1] \textbf{1.} \dicSynonym{kúgaður} \dicDirectTranslationCS{nesvobodný, utiskovaný}  \textbf{2.} \dicSynonym{óleyfilegur} \dicDirectTranslationCS{nepovolený, zakázaný} \dicExampleIS{Þetta er þér ófrjálst.} \dicExampleCS{To nesmíš.}
\dicEntry[ófrjór] \dicTerm{ó··frjór} \dicIPA{{ou}{\textlengthmark}{f}{r}{j}{ou}{\textsubring{r}}} \dicPos{adj}[4]\dicFlx{}[-1] \textbf{1.} \dicDirectTranslationCS{neplodný, sterilní} \dicExampleIS{ófrjó kona} \dicExampleCS{neplodná žena} \dicAntonym{frjór}  \textbf{2.} \dicSynonym*{óræktanlegur} \dicDirectTranslationCS{neúrodný} \dicExampleIS{ófrjór jarðvegur} \dicExampleCS{neúrodná půda}
\dicEntry[ófrjósemi] \dicTerm{ó··frjó·sem|i} \dicIPA{{ou}{\textlengthmark}{f}{r}{j}{ou}{s}{\textepsilon}{m}{\textsci}} \dicPos{f}[2] \dicFlx{(‑i)}[2] \textbf{1.} \dicDirectTranslationCS{neplodnost, sterilita}  \textbf{2.} \dicDirectTranslationCS{neúrodnost}
\dicEntry[ófróðlegur] \dicTerm{ó··fróð·legur} \dicIPA{{ou}{\textlengthmark}{f}{r}{ou}{ð}{l}{\textepsilon}{\textbabygamma}{\textscy}{\textsubring{r}}} \dicPos{adj}[1]\dicFlx{}[-6] \dicDirectTranslationCS{nezajímavý, nepoučný} \dicAntonym{fróðlegur}
\dicEntry[ófróður] \dicTerm{ó··fróður} \dicIPA{{ou}{\textlengthmark}{f}{r}{ou}{ð}{\textscy}{\textsubring{r}}} \dicPos{adj}[2]\dicFlx{}[-6] \dicSynonym{fáfróður} \dicDirectTranslationCS{neznalý, nepoučený, nevědomý} \dicAntonym{fróður}
\dicEntry[ófrumbjarga] \dicTerm{ó··frum·bjarga} \dicIPA{{ou}{\textlengthmark}{f}{r}{\textscy}{m}{\textsubring{b}}{j}{a}{r}{\r{g}}{a}} \dicPos{adj}[13] \dicFlx{indecl}[1] \dicFieldCat{biol.} \dicDirectTranslationCS{heterotrofní} \dicAntonym{frumbjarga}
\dicEntry[ófrýnilegur] \dicTerm{ó··frýni·legur} \dicIPA{{ou}{\textlengthmark}{f}{r}{i}{n}{\textsci}{l}{\textepsilon}{\textbabygamma}{\textscy}{\textsubring{r}}} \dicPos{adj}[1]\dicFlx{}[-8] \dicSynonym{ljótur} \dicDirectTranslationCS{škaredý, šeredný} \dicAntonym{frýnilegur}
\dicEntry[ófrægja] \dicTerm{ó··fræg|ja} \dicIPA{{ou}{\textlengthmark}{f}{r}{a}{i}{j}{a}} \dicPos{v}[2] \dicFlx{(‑ði, ‑t)}[92] \dicFlx{acc} \dicSynonym{baktala} \dicDirectTranslationCS{hanobit, pomluvit} \dicAntonym{frægja}
\dicEntry[ófullburða] \dicTerm{ó··full·burða} \dicIPA{{ou}{\textlengthmark}{f}{\textscy}{\textsubring{d}}{\textsubring{l}}{\textsubring{b}}{\textscy}{r}{ð}{a}} \dicPos{adj}[13] \dicFlx{indecl}[1] \dicDirectTranslationCS{nedonošený (dítě ap.)} \dicExampleIS{ófullburða barn} \dicExampleCS{nedonošené dítě}
\dicEntry[ófullgerður] \dicTerm{ó··full·gerður} \dicIPA{{ou}{\textlengthmark}{f}{\textscy}{\textsubring{d}}{\textsubring{l}}{\r{\textObardotlessj}}{\textepsilon}{r}{ð}{\textscy}{\textsubring{r}}} \dicPos{adj}[2]\dicFlx{}[-1] \dicDirectTranslationCS{nedokončený, nedodělaný} \dicExampleIS{ófullgert verk} \dicExampleCS{nedokončená práce} \dicAntonym{fullgerður}
\dicEntry[ófullkominn] \dicTerm{ó··full·kominn} \dicIPA{{ou}{\textlengthmark}{f}{\textscy}{\textsubring{d}}{\textsubring{l}}{k\smash{\textsuperscript{h}}}{\textopeno}{m}{\textsci}{\textsubring{n}}} \dicPos{adj}[6]\dicFlx{}[-2] \textbf{1.} \dicSynonym{óheill\smash{\textsuperscript{2}}} \dicDirectTranslationCS{nedodělaný, nekompletní} \dicAntonym{fullkominn}  \textbf{2.} \dicSynonym{takmarkaður} \dicDirectTranslationCS{nedokonalý}
\dicEntry[ófullnægjandi] \dicTerm{ó··full·nægj·andi} \dicIPA{{ou}{\textlengthmark}{f}{\textscy}{\textsubring{d}}{l}{n}{a}{i}{j}{a}{n}{\textsubring{d}}{\textsci}} \dicPos{adj}[13] \dicFlx{indecl}[1] \dicSynonym{ónógur} \dicDirectTranslationCS{neuspokojivý, nedostačující} \dicAntonym{fullnægjandi}
\dicEntry[ófullur] \dicTerm{ó··|fullur} \dicIPA{{ou}{\textlengthmark}{f}{\textscy}{\textsubring{d}}{l}{\textscy}{\textsubring{r}}} \dicPos{adj}[10] \dicFlx{(comp ‑fyllri, sup ‑fyllstur)}[7] \dicSynonym{ódrukkinn} \dicDirectTranslationCS{střízlivý} \dicAntonym{fullur}
\dicEntry[ófullveðja] \dicTerm{ó··full·veðja} \dicIPA{{ou}{\textlengthmark}{f}{\textscy}{\textsubring{d}}{l}{v}{\textepsilon}{ð}{j}{a}} \dicPos{adj}[13] \dicFlx{indecl}[1] \dicDirectTranslationCS{nezletilý} \dicExampleIS{ófullveðja aldur} \dicExampleCS{nezletilost} \dicAntonym{fullveðja}
\dicEntry[ófum] \dicTerm{ófum} \dicIPA{{ou}{\textlengthmark}{v}{\textscy}{\textsubring{m}}} \dicPos{v} \dicFlx{ind pf pl 1 pers} \dicLink{vefa}
\dicEntry[ófús] \dicTerm{ó··fús} \dicIPA{{ou}{\textlengthmark}{f}{u}{s}} \dicPos{adj}[5]\dicFlx{}[-1] \dicSynonym{tregur} \dicDirectTranslationCS{zdráhavý, neochotný} \dicExampleIS{ófús til e‑s} \dicExampleCS{neochotný k~(čemu)} \dicAntonym{fús}
\dicEntry[ófyrirgefanlegur] \dicTerm{ó··fyrir·gefan·legur} \dicIPA{{ou}{\textlengthmark}{f}{\textsci}{r}{\textsci}{r}{\r{\textObardotlessj}}{\textepsilon}{v}{a}{n}{l}{\textepsilon}{\textbabygamma}{\textscy}{\textsubring{r}}} \dicPos{adj}[1]\dicFlx{}[-8] \dicDirectTranslationCS{neodpustitelný, neprominutelný} \dicExampleIS{ófyrirgefanlegt kæruleysi} \dicExampleCS{neodpustitelné pochybení} \dicAntonym{fyrirgefanlegur}
\dicEntry[ófyrirleitinn] \dicTerm{ó··fyrir·leitinn} \dicIPA{{ou}{\textlengthmark}{f}{\textsci}{r}{\textsci}{r}{l}{ei}{\textsubring{d}}{\textsci}{\textsubring{n}}} \dicPos{adj}[6]\dicFlx{}[-2] \dicSynonym{ósvífinn} \dicDirectTranslationCS{drzý, troufalý, nestoudný}
\dicEntry[ófyrirsjáanlegur] \dicTerm{ó··fyrir·sjáan·legur} \dicIPA{{ou}{\textlengthmark}{f}{\textsci}{r}{\textsci}{\textsubring{r}}{s}{j}{au}{a}{n}{l}{\textepsilon}{\textbabygamma}{\textscy}{\textsubring{r}}} \dicPos{adj}[1]\dicFlx{}[-8] \dicDirectTranslationCS{nepředvídatelný} \dicExampleIS{ófyrirsjáanlegar afleiðingar} \dicExampleCS{nepředvídatelné následky} \dicAntonym{fyrirsjáanlegur}
\dicEntry[ófyrirsynja] \dicTerm{ó··fyrir·synj|a} \dicIPA{{ou}{\textlengthmark}{f}{\textsci}{r}{\textsci}{\textsubring{r}}{s}{\textsci}{n}{j}{a}} \dicPos{f}[1] \dicFlx{(‑u)}[6] \dicSynonym{vanhyggja} \dicDirectTranslationCS{neopatrnost};  \dicPhraseIS{að ófyrirsynju} \dicFlx{adv} \dicSynonym*{án tilefnis} \dicDirectTranslationCS{bezdůvodně, jen tak}
\dicEntry[ófæddur] \dicTerm{ó··fæddur} \dicIPA{{ou}{\textlengthmark}{f}{a}{i}{\textsubring{d}}{\textscy}{\textsubring{r}}} \dicPos{adj}[2]\dicFlx{}[-21] \dicSynonym*{óborinn} \dicDirectTranslationCS{nenarozený} \dicAntonym{fæddur}
\dicEntry[ófær] \dicTerm{ó··fær} \dicIPA{{ou}{\textlengthmark}{f}{a}{i}{\textsubring{r}}} \dicPos{adj}[5] \dicFlx{(f ‑)}[8] \textbf{1.} \dicSynonym*{ógengur} \dicDirectTranslationCS{nesjízdný, nesplavný} \dicExampleIS{ófær vegur} \dicExampleCS{nesjízdná cesta} \dicAntonym{fær}  \textbf{2.} \dicSynonym{ómögulegur} \dicDirectTranslationCS{nemožný, nepovolený}  \textbf{3.} \dicSynonym{hjálparlaus} \dicDirectTranslationCS{bezradný, bezmocný}  \textbf{4.} \dicSynonym{ótækur} \dicDirectTranslationCS{nepřijatelný, nepřípustný}
\dicEntry[ófæra] \dicTerm{ó··fær|a} \dicIPA{{ou}{\textlengthmark}{f}{a}{i}{r}{a}} \dicPos{f}[1] \dicFlx{(‑u, ‑ur)}[19] \textbf{1.} \dicDirectTranslationCS{neschůdná cesta, nesplavný jez}  \textbf{2.} \dicSynonym{vandræði} \dicDirectTranslationCS{obtížná\,/\addthin bezvýchodná situace, problémy}
\dicEntry[ófærð] \dicTerm{ó··færð} \dicIPA{{ou}{\textlengthmark}{f}{a}{i}{r}{\texttheta}} \dicPos{f}[7] \dicFlx{(‑ar)}[3] \dicSynonym{umbrot} \dicDirectTranslationCS{nesjízdnost, neschůdnost, nesplavnost} \dicExampleIS{Það er mikil ófærð á heiðinni.} \dicExampleCS{Vysočina je velmi špatně sjízdná.}
\dicEntry[ógagn] \dicTerm{ó··gagn} \dicIPA{{ou}{\textlengthmark}{\r{g}}{a}{\r{g}}{\textsubring{n}}} \dicPos{n}[2] \dicFlx{(‑s)}[2] \dicSynonym{tjón} \dicDirectTranslationCS{škoda, neprospěch} \dicAntonym{gagn\smash{\textsuperscript{1}}}
\dicEntry[ógagnsær] \dicTerm{ó··gagn·sær} \dicIPA{{ou}{\textlengthmark}{\r{g}}{a}{\r{g}}{\textsubring{n}}{s}{a}{i}{\textsubring{r}}} \dicPos{adj}[4]\dicFlx{}[-7] \dicSynonym{mattur} \dicDirectTranslationCS{neprůhledný, neprůsvitný} \dicAntonym{gagnsær}
\dicEntry[ógát] \dicTerm{ó··gát} \dicIPA{{ou}{\textlengthmark}{\r{g}}{au}{\textsubring{d}}} \dicPos{f}[4] \dicFlx{(‑ar)}[3] \dicSynonym{gáleysi} \dicDirectTranslationCS{nedbalost, nepozornost} \dicAntonym{gát};  \dicPhraseIS{gera e‑ð í ógáti} \dicDirectTranslationCS{udělat (co) nechtěně\,/\addthin bezděčně}
\dicEntry[ógeð] \dicTerm{ó··geð} \dicIPA{{ou}{\textlengthmark}{\r{\textObardotlessj}}{\textepsilon}{\texttheta}} \dicPos{n}[2] \dicFlx{(‑s, ‑)}[5] \textbf{1.} \dicSynonym{andúð} \dicDirectTranslationCS{odpor, hnus};  \dicPhraseIS{hafa ógeð í e‑u} \dicDirectTranslationCS{mít odpor k~(čemu), hnusit si (co)}  \textbf{2.} \dicSynonym{vanþóknun} \dicDirectTranslationCS{nelibost, nechuť}
\dicEntry[ógeðfelldur] \dicTerm{ó··geð·felldur} \dicIPA{{ou}{\textlengthmark}{\r{\textObardotlessj}}{\textepsilon}{ð}{f}{\textepsilon}{l}{\textsubring{d}}{\textscy}{\textsubring{r}}} \dicPos{adj}[2]\dicFlx{}[-14] \dicSynonym{óþægilegur} \dicDirectTranslationCS{nepříjemný, nesympatický} \dicExampleIS{ógeðfelld tilfinning} \dicExampleCS{nepříjemný pocit} \dicAntonym{geðfelldur}
\dicEntry[ógeðslegur] \dicTerm{ó··geðs·legur} \dicIPA{{ou}{\textlengthmark}{\r{\textObardotlessj}}{\textepsilon}{ð}{s}{l}{\textepsilon}{\textbabygamma}{\textscy}{\textsubring{r}}} \dicPos{adj}[1]\dicFlx{}[-8] \dicSynonym{andstyggilegur} \dicDirectTranslationCS{odporný, hnusný} \dicExampleIS{ógeðsleg sjón} \dicExampleCS{hnusný pohled} \dicAntonym{geðslegur}
\dicEntry[ógerningur] \dicTerm{ó··ger·ning|ur} \dicIPA{{ou}{\textlengthmark}{\r{\textObardotlessj}}{\textepsilon}{r}{\textsubring{d}}{n}{i}{\ng}{\r{g}}{\textscy}{\textsubring{r}}} \dicPos{m}[6] \dicFlx{(‑s, ‑ar)}[8] \dicDirectTranslationCS{neproveditelnost, neuskutečnitelnost}
\dicEntry[ógestrisinn] \dicTerm{ó··gest·risinn} \dicIPA{{ou}{\textlengthmark}{\r{\textObardotlessj}}{\textepsilon}{s}{\textsubring{d}}{r}{\textsci}{s}{\textsci}{\textsubring{n}}} \dicPos{adj}[6]\dicFlx{}[-2] \dicSynonym*{gestfúll} \dicDirectTranslationCS{nepohostinný, nehostinný} \dicAntonym{gestrisinn}
\dicEntry[ógiftur] \dicTerm{ó··giftur} \dicIPA{{ou}{\textlengthmark}{\r{\textObardotlessj}}{\textsci}{f}{\textsubring{d}}{\textscy}{\textsubring{r}}} \dicPos{adj}[1]\dicFlx{}[-13] \dicSynonym*{konulaus} \dicDirectTranslationCS{svobodný, neprovdaný, neženatý} \dicAntonym{giftur}
\dicEntry[ógilda] \dicTerm{ó··gil|da} \dicIPA{{ou}{\textlengthmark}{\r{\textObardotlessj}}{\textsci}{l}{\textsubring{d}}{a}} \dicPos{v}[2] \dicFlx{(‑ti, ‑t)}[37] \dicFlx{acc} \dicSynonym{afnema} \dicDirectTranslationCS{anulovat, stornovat, prohlásit neplatným} \dicExampleIS{ógilda samninginn} \dicExampleCS{prohlásit smlouvu za neplatnou}
\dicEntry[ógilding] \dicTerm{ó··gild·ing} \dicIPA{{ou}{\textlengthmark}{\r{\textObardotlessj}}{\textsci}{l}{\textsubring{d}}{i}{\ng}{\r{g}}} \dicPos{f}[4] \dicFlx{(‑ar, ‑ar)}[5] \dicSynonym{ónýting} \dicDirectTranslationCS{anulování, stornování, zrušení}
\dicEntry[ógildur] \dicTerm{ó··gildur} \dicIPA{{ou}{\textlengthmark}{\r{\textObardotlessj}}{\textsci}{l}{\textsubring{d}}{\textscy}{\textsubring{r}}} \dicPos{adj}[2]\dicFlx{}[-17] \dicDirectTranslationCS{neplatný, anulovaný, stornovaný (nařízení ap.)} \dicAntonym{gildur}
\dicEntry[ógleði] \dicTerm{ó··gleð|i} \dicIPA{{ou}{\textlengthmark}{\r{g}}{l}{\textepsilon}{ð}{\textsci}} \dicPos{f}[3] \dicFlx{(‑i)}[3] \textbf{1.} \dicSynonym{þunglyndi} \dicDirectTranslationCS{melancholie, smutná nálada} \dicExampleIS{taka ógleði} \dicExampleCS{mít smutnou náladu}  \textbf{2.} \dicSynonym{velgja\smash{\textsuperscript{1}}} \dicDirectTranslationCS{nevolnost, zvedání žaludku}
\dicEntry[ógleymanlegur] \dicTerm{ó··gleyman·legur} \dicIPA{{ou}{\textlengthmark}{\r{g}}{l}{ei}{m}{a}{n}{l}{\textepsilon}{\textbabygamma}{\textscy}{\textsubring{r}}} \dicPos{adj}[1]\dicFlx{}[-8] \dicDirectTranslationCS{nezapomenutelný} \dicExampleIS{ógleymanleg sjón} \dicExampleCS{nezapomenutelný pohled}
\dicEntry[óglöggur] \dicTerm{ó··|glöggur} \dicIPA{{ou}{\textlengthmark}{\r{g}}{l}{\oe}{\r{g}}{\textscy}{\textsubring{r}}} \dicPos{adj}[10] \dicFlx{(comp ‑gleggri, sup ‑gleggstur)}[1] \textbf{1.} \dicSynonym{ógreinilegur} \dicDirectTranslationCS{nejasný, neurčitý} \dicAntonym{glöggur}  \textbf{2.} \dicDirectTranslationCS{zapomnětlivý, nemající paměť (na tváře ap.)} \dicExampleIS{óglöggur á fólk} \dicExampleCS{nemající paměť na lidi}
\dicEntry[ógn] \dicTerm{ógn} \dicIPA{{ou}{\r{g}}{\textsubring{n}}} \dicPos{f}[7] \dicFlx{(‑ar, ‑ir)}[1] \dicSynonym{skelfing} \dicDirectTranslationCS{hrozba, hrůza};  \dicPhraseIS{e‑m stendur ógn af e‑u} \dicFlx{impers} \dicSynonym{óttast} \dicDirectTranslationCS{(kdo) se (čeho) hrozí}
\dicEntry[ógna] \dicTerm{ógn|a} \dicIPA{{ou}{\r{g}}{n}{a}} \dicPos{v}[1] \dicFlx{(‑aði)}[45] \dicFlx{dat} \textbf{1.} \dicSynonym{hóta} \dicDirectTranslationCS{(po)hrozit, ohrožovat, strašit} \dicExampleIS{ógna e‑m með e‑u} \dicExampleCS{hrozit (komu čím)}  \textbf{2.} \dicPhraseIS{e‑m ógnar e‑að} \dicFlx{impers} \dicDirectTranslationCS{(kdo) se hrozí (čeho), (kdo) má hrůzu z~(čeho)}
\dicEntry[ógnun] \dicTerm{ógn|un} \dicIPA{{ou}{\r{g}}{n}{\textscy}{\textsubring{n}}} \dicPos{f}[7] \dicFlx{(‑unar, ‑anir)}[8] \dicSynonym{hótun} \dicDirectTranslationCS{hrozba, ohrožení}
\dicEntry[ógnvekjandi] \dicTerm{ógn··vekj·andi} \dicIPA{{ou}{\r{g}}{\textsubring{n}}{v}{\textepsilon}{\r{\textObardotlessj}}{a}{n}{\textsubring{d}}{\textsci}} \dicPos{adj}[13] \dicFlx{indecl}[1] \dicSynonym{hræðilegur} \dicDirectTranslationCS{hrůzostrašný, hrůzný, vzbuzující hrůzu}
\dicEntry[ógreiddur] \dicTerm{ó··greiddur} \dicIPA{{ou}{\textlengthmark}{\r{g}}{r}{ei}{\textsubring{d}}{\textscy}{\textsubring{r}}} \dicPos{adj}[2]\dicFlx{}[-18] \textbf{1.} \dicSynonym*{óborgaður} \dicDirectTranslationCS{neuhrazený, nezaplacený} \dicAntonym{greiddur}  \textbf{2.} \dicSynonym{úfinn} \dicDirectTranslationCS{neučesaný, rozcuchaný}
\dicEntry[ógreiðfær] \dicTerm{ó··greið·fær} \dicIPA{{ou}{\textlengthmark}{\r{g}}{r}{ei}{ð}{f}{a}{i}{\textsubring{r}}} \dicPos{adj}[5] \dicFlx{(f ‑)}[8] \dicSynonym{seinfarinn} \dicDirectTranslationCS{nesjízdný, neschůdný, nesplavný} \dicAntonym{greiðfær}
\dicEntry[ógreiði] \dicTerm{ó··greið|i} \dicIPA{{ou}{\textlengthmark}{\r{g}}{r}{ei}{ð}{\textsci}} \dicPos{m}[1] \dicFlx{(‑a, ‑ar)}[1] \dicSynonym{ógagn} \dicDirectTranslationCS{škoda, medvědí služba} \dicExampleIS{gera e‑m ógreiða} \dicExampleCS{prokázat (komu) medvědí službu}
\dicEntry[ógreinilegur] \dicTerm{ó··greini·legur} \dicIPA{{ou}{\textlengthmark}{\r{g}}{r}{ei}{n}{\textsci}{l}{\textepsilon}{\textbabygamma}{\textscy}{\textsubring{r}}} \dicPos{adj}[1]\dicFlx{}[-8] \dicSynonym{óskýr} \dicDirectTranslationCS{nejasný, nezřetelný, neurčitý} \dicExampleIS{ógreinileg spor} \dicExampleCS{nezřetelné stopy} \dicAntonym{greinilegur}
\dicEntry[ógrynni] \dicTerm{ó··grynni} \dicIPA{{ou}{\textlengthmark}{\r{g}}{r}{\textsci}{n}{\textsci}} \dicPos{n}[2] \dicFlx{(‑s, ‑)}[14] \dicDirectTranslationCS{nesčetné množství, spousta} \dicExampleIS{ógrynni fjár} \dicExampleCS{spousta ovcí}
\dicEntry[óguðlegur] \dicTerm{ó··guð·legur} \dicIPA{{ou}{\textlengthmark}{\r{g}}{v}{\textscy}{ð}{l}{\textepsilon}{\textbabygamma}{\textscy}{\textsubring{r}}} \dicPos{adj}[1]\dicFlx{}[-8] \textbf{1.} \dicSynonym{guðlaus} \dicDirectTranslationCS{bezbožný, bohaprázdný} \dicAntonym{guðlegur}  \textbf{2.} \dicSynonym{illur} \dicDirectTranslationCS{zlý, špatný}
\dicEntry[ógurlegur] \dicTerm{ógur··legur} \dicIPA{{ou}{\textlengthmark}{\textscy}{r}{l}{\textepsilon}{\textbabygamma}{\textscy}{\textsubring{r}}} \dicPos{adj}[1]\dicFlx{}[-8] \textbf{1.} \dicSynonym{ægilegur} \dicDirectTranslationCS{hrozivý, vzbuzující hrůzu}  \textbf{2.} \dicSynonym*{afar mikill} \dicDirectTranslationCS{nesmírný, obrovský}
\dicEntry[ógæfa] \dicTerm{ó··gæf|a} \dicIPA{{ou}{\textlengthmark}{\r{\textObardotlessj}}{a}{i}{v}{a}} \dicPos{f}[1] \dicFlx{(‑u)}[5] \dicSynonym{óhamingja} \dicDirectTranslationCS{neštěstí, smůla} \dicAntonym{gæfa};  \dicPhraseIS{til allrar ógæfu} \dicFlx{adv} \dicSynonym*{því miður} \dicDirectTranslationCS{naneštěstí, ke vší smůle}
\dicEntry[ógæftir] \dicTerm{ó··gæftir} \dicIPA{{ou}{\textlengthmark}{\r{\textObardotlessj}}{a}{i}{f}{\textsubring{d}}{\textsci}{\textsubring{r}}} \dicPos{f}[12] \dicFlx{pl}[10] \dicDirectTranslationCS{špatné počasí na rybolov} \dicAntonym{gæftir}
\dicEntry[ógæfumaður] \dicTerm{ó·gæfu··|maður} \dicIPA{{ou}{\textlengthmark}{\r{\textObardotlessj}}{a}{i}{v}{\textscy}{m}{a}{ð}{\textscy}{\textsubring{r}}} \dicPos{m}[13] \dicFlx{(‑manns, ‑menn)}[2] \dicDirectTranslationCS{smolař(ka), člověk mající smůlu}
\dicEntry[ógæfusamur] \dicTerm{ó·gæfu··|samur} \dicIPA{{ou}{\textlengthmark}{\r{\textObardotlessj}}{a}{i}{v}{\textscy}{s}{a}{m}{\textscy}{\textsubring{r}}} \dicPos{adj}[1] \dicFlx{(f ‑söm)}[2] \dicSynonym{óhamingjusamur} \dicDirectTranslationCS{smolný, smolařský, mající smůlu} \dicAntonym{gæfusamur}
\dicEntry[ógætinn] \dicTerm{ó··gætinn} \dicIPA{{ou}{\textlengthmark}{\r{\textObardotlessj}}{a}{i}{\textsubring{d}}{\textsci}{\textsubring{n}}} \dicPos{adj}[6]\dicFlx{}[-2] \dicDirectTranslationCS{neopatrný, neobezřetný}
\dicEntry[ógætni] \dicTerm{ó··gætn|i} \dicIPA{{ou}{\textlengthmark}{\r{\textObardotlessj}}{a}{i}{h}{\textsubring{d}}{n}{\textsci}} \dicPos{f}[3] \dicFlx{(‑i)}[3] \dicDirectTranslationCS{neopatrnost, neobezřetnost}
\dicEntry[óhaggaður] \dicTerm{ó··|hagg·aður} \dicIPA{{ou}{\textlengthmark}{h}{a}{\r{g}}{a}{ð}{\textscy}{\textsubring{r}}} \dicPos{adj}[3] \dicFlx{(f ‑högguð)}[2] \dicSynonym*{óhreyfður} \dicDirectTranslationCS{neměnný, nezměněný}
\dicEntry[óhagganlegur] \dicTerm{ó··haggan·legur} \dicIPA{{ou}{\textlengthmark}{h}{a}{\r{g}}{a}{n}{l}{\textepsilon}{\textbabygamma}{\textscy}{\textsubring{r}}} \dicPos{adj}[1]\dicFlx{}[-8] \dicSynonym{óbifanlegur} \dicDirectTranslationCS{neochvějný, neotřesitelný} \dicExampleIS{Ákvörðun er óhagganleg.} \dicExampleCS{Rozhodnutí je neochvějné.}
\dicEntry[óhagkvæmur] \dicTerm{ó··hag·kvæmur} \dicIPA{{ou}{\textlengthmark}{h}{a}{\textbabygamma}{k\smash{\textsuperscript{h}}}{v}{a}{i}{m}{\textscy}{\textsubring{r}}} \dicPos{adj}[1]\dicFlx{}[-1] \dicDirectTranslationCS{neefektivní, nevýnosný} \dicExampleIS{óhagkvæm viðskipti} \dicExampleCS{nevýnosný obchod} \dicAntonym{hagkvæmur}
\dicEntry[óhagstæður] \dicTerm{ó··hag·stæður} \dicIPA{{ou}{\textlengthmark}{h}{a}{x}{s}{\textsubring{d}}{a}{i}{ð}{\textscy}{\textsubring{r}}} \dicPos{adj}[2]\dicFlx{}[-6] \dicSynonym{óhagkvæmur} \dicDirectTranslationCS{nepříznivý, nepříhodný} \dicAntonym{hagstæður}
\dicEntry[óhagsýnn] \dicTerm{ó··hag·sýnn} \dicIPA{{ou}{\textlengthmark}{h}{a}{x}{s}{i}{\textsubring{d}}{\textsubring{n}}} \dicPos{adj}[7]\dicFlx{}[-1] \dicDirectTranslationCS{neekonomický, nevýnosný} \dicAntonym{hagsýnn}
\dicEntry[óhagur] \dicTerm{ó··hag|ur} \dicIPA{{ou}{\textlengthmark}{h}{a}{\textbabygamma}{\textscy}{\textsubring{r}}} \dicPos{m}[9] \dicFlx{(‑s, ‑ir)}[9] \dicSynonym{skaði} \dicDirectTranslationCS{škoda, újma} \dicAntonym{hagur\smash{\textsuperscript{1}}};  \dicPhraseIS{e‑að er e‑m í óhag} \dicDirectTranslationCS{(co) je (komu) ke škodě}
\dicEntry[óhamingja] \dicTerm{ó··hamingj|a} \dicIPA{{ou}{\textlengthmark}{h}{a}{m}{i}{\textltailn}{\r{\textObardotlessj}}{a}} \dicPos{f}[1] \dicFlx{(‑u)}[5] \dicSynonym{ógæfa} \dicDirectTranslationCS{smůla, pech} \dicAntonym{hamingja};  \dicPhraseIS{til allrar óhamingju} \dicFlx{adv} \dicSynonym*{því miður} \dicDirectTranslationCS{ke vší smůle, naneštěstí, bohužel} \dicExampleIS{Til allrar óhamingju bilaði bíllinn á leiðinni.} \dicExampleCS{Ke vší smůle se během cesty rozbilo auto.}
\dicEntry[óhamingjusamur] \dicTerm{ó·hamingju··|samur} \dicIPA{{ou}{\textlengthmark}{h}{a}{m}{i}{\textltailn}{\r{\textObardotlessj}}{\textscy}{s}{a}{m}{\textscy}{\textsubring{r}}} \dicPos{adj}[1] \dicFlx{(f ‑söm)}[2] \dicSynonym{ógæfusamur} \dicDirectTranslationCS{nešťastný, mající smůlu} \dicAntonym{hamingjusamur}
\dicEntry[óhapp] \dicTerm{ó··|happ} \dicIPA{{ou}{\textlengthmark}{h}{a}{h}{\textsubring{b}}} \dicPos{n}[2] \dicFlx{(‑happs, ‑höpp)}[8] \dicSynonym{óheppni} \dicDirectTranslationCS{nehoda, nešťastná\,/\addthin nepříjemná příhoda} \dicExampleIS{verða fyrir óhappi} \dicExampleCS{zažít nepříjemnou příhodu} \dicAntonym{happ}
\dicEntry[óharðnaður] \dicTerm{ó··|harð·naður} \dicIPA{{ou}{\textlengthmark}{h}{a}{r}{\textsubring{d}}{n}{a}{ð}{\textscy}{\textsubring{r}}} \dicPos{adj}[3] \dicFlx{(f ‑hörðnuð)}[1] \dicDirectTranslationCS{neostřílený, nezocelený} \dicExampleIS{óharðnaður unglingur} \dicExampleCS{neostřílený mladý člověk}
\dicEntry[óháður] \dicTerm{ó··háður} \dicsymFrequent\  \dicIPA{{ou}{\textlengthmark}{h}{au}{ð}{\textscy}{\textsubring{r}}} \dicPos{adj}[2]\dicFlx{}[-12] \textbf{1.} \dicSynonym{frjáls} \dicDirectTranslationCS{nezávislý} \dicExampleIS{Þessi ákvörðun er óháð pólitískum hagsmunum.} \dicExampleCS{Toto rozhodnutí je nezávislé na politických zájmech.}  \textbf{2.} \dicSynonym{hlutlaus} \dicDirectTranslationCS{nezaujatý, neangažovaný} \dicAntonym{háður}
\dicEntry[óheflaður] \dicTerm{ó··hefl·|aður} \dicIPA{{ou}{\textlengthmark}{h}{\textepsilon}{\textsubring{b}}{l}{a}{ð}{\textscy}{\textsubring{r}}} \dicPos{adj}[3] \dicFlx{(f ‑uð)}[3] \textbf{1.} \dicSynonym*{ófágaður} \dicDirectTranslationCS{neotesaný, nehoblovaný (dřevo ap.)}  \textbf{2.} \dicSynonym{hranalegur} \dicDirectTranslationCS{obhroublý, neotesaný (člověk ap.)}
\dicEntry[óheiðarlegur] \dicTerm{ó··heiðar·legur} \dicIPA{{ou}{\textlengthmark}{h}{ei}{ð}{a}{r}{l}{\textepsilon}{\textbabygamma}{\textscy}{\textsubring{r}}} \dicPos{adj}[1]\dicFlx{}[-8] \dicSynonym{óráðvandur} \dicDirectTranslationCS{nečestný, nepoctivý} \dicAntonym{heiðarlegur}
\dicEntry[óheilbrigður] \dicTerm{ó··heil·brigður} \dicIPA{{ou}{\textlengthmark}{h}{ei}{l}{\textsubring{b}}{r}{\textsci}{\textbabygamma}{ð}{\textscy}{\textsubring{r}}} \dicPos{adj}[2]\dicFlx{}[-1] \dicSynonym{sjúkur} \dicDirectTranslationCS{nezdravý, zdraví škodlivý} \dicAntonym{heilbrigður}
\dicEntry[óheilindi] \dicTerm{ó··heil·indi} \dicIPA{{ou}{\textlengthmark}{h}{ei}{l}{\textsci}{n}{\textsubring{d}}{\textsci}} \dicPos{n}[2] \dicFlx{pl}[19] \textbf{1.} \dicSynonym{fals} \dicDirectTranslationCS{faleš, neupřímnost}  \textbf{2.} \dicSynonym{vanheilsa} \dicDirectTranslationCS{slabé zdraví, churavost} \dicAntonym{heilindi}
\dicEntry[óheill] \dicTerm{ó··heill\smash{\textsuperscript{1}}} \dicIPA{{ou}{\textlengthmark}{h}{ei}{\textsubring{d}}{\textsubring{l}}} \dicPos{f}[7] \dicFlx{(‑ar, ‑ir)}[1] \dicSynonym{ógæfa} \dicDirectTranslationCS{smůla, pech, nezdar} \dicAntonym{heill\smash{\textsuperscript{2}}}
\dicEntry[óheill] \dicTerm{ó··heill\smash{\textsuperscript{2}}} \dicIPA{{ou}{\textlengthmark}{h}{ei}{\textsubring{d}}{\textsubring{l}}} \dicPos{adj}[8]\dicFlx{}[-1] \dicSynonym{falskur} \dicDirectTranslationCS{falešný, neupřímný}
\dicEntry[óheillaþróun] \dicTerm{ó··heilla·þró|un} \dicIPA{{ou}{\textlengthmark}{h}{ei}{\textsubring{d}}{l}{a}{\texttheta}{r}{ou}{\textscy}{\textsubring{n}}} \dicPos{f}[7] \dicFlx{(‑unar)}[9] \dicDirectTranslationCS{nepříznivý vývoj}
\dicEntry[óheilnæmur] \dicTerm{ó··heil·næmur} \dicIPA{{ou}{\textlengthmark}{h}{ei}{l}{n}{a}{i}{m}{\textscy}{\textsubring{r}}} \dicPos{adj}[1]\dicFlx{}[-1] \dicSynonym{óhollur} \dicDirectTranslationCS{nezdravý, škodlivý\,/\addthin neprospívající zdraví} \dicAntonym{heilnæmur}
\dicEntry[óheimill] \dicTerm{ó··heimill} \dicIPA{{ou}{\textlengthmark}{h}{ei}{m}{\textsci}{\textsubring{d}}{\textsubring{l}}} \dicPos{adj}[8]\dicFlx{}[-1] \dicSynonym{óleyfilegur} \dicDirectTranslationCS{neoprávněný, nelegální, nezákonný} \dicExampleIS{óheimill aðgangur} \dicExampleCS{neoprávněný vstup} \dicAntonym{heimill}
\dicEntry[óheimskur] \dicTerm{ó··heimskur} \dicIPA{{ou}{\textlengthmark}{h}{ei}{m}{s}{\r{g}}{\textscy}{\textsubring{r}}} \dicPos{adj}[1]\dicFlx{}[-1] \dicSynonym{greindur} \dicDirectTranslationCS{chytrý, důvtipný} \dicAntonym{heimskur}
\dicEntry[óhemja] \dicTerm{ó··hemj|a} \dicIPA{{ou}{\textlengthmark}{h}{\textepsilon}{m}{j}{a}} \dicPos{f}[1] \dicFlx{(‑u, ‑ur)}[7] \textbf{1.} \dicSynonym*{frenja} \dicDirectTranslationCS{bestie, divoch}  \textbf{2.} \dicSynonym{kynstur} \dicDirectTranslationCS{strašně\,/\addthin šíleně moc} \dicIndirectTranslationCS{(k~zdůraznění)} \dicExampleIS{óhemja af peningum} \dicExampleCS{šíleně moc peněz}
\dicEntry[óhemjuskapur] \dicTerm{ó·hemju··skap|ur} \dicIPA{{ou}{\textlengthmark}{h}{\textepsilon}{m}{j}{\textscy}{s}{\r{g}}{a}{\textsubring{b}}{\textscy}{\textsubring{r}}} \dicPos{m}[10] \dicFlx{(‑ar)}[15] \dicSynonym{ofsi} \dicDirectTranslationCS{divokost, prudkost}
\dicEntry[óhentugur] \dicTerm{ó··hent·ugur} \dicIPA{{ou}{\textlengthmark}{h}{\textepsilon}{\textsubring{n}}{\textsubring{d}}{\textscy}{\textbabygamma}{\textscy}{\textsubring{r}}} \dicPos{adj}[1]\dicFlx{}[-8] \dicDirectTranslationCS{nevhodný, nepříhodný} \dicExampleIS{á óhentugum tíma} \dicExampleCS{v~nevhodný čas} \dicAntonym{hentugur}
\dicEntry[óheppilegur] \dicTerm{ó··heppi·legur} \dicIPA{{ou}{\textlengthmark}{h}{\textepsilon}{h}{\textsubring{b}}{\textsci}{l}{\textepsilon}{\textbabygamma}{\textscy}{\textsubring{r}}} \dicPos{adj}[1]\dicFlx{}[-8] \dicSynonym{óþægilegur} \dicDirectTranslationCS{nevhodný, nešťastný (čas ap.)} \dicAntonym{heppilegur}
\dicEntry[óheppinn] \dicTerm{ó··heppinn} \dicIPA{{ou}{\textlengthmark}{h}{\textepsilon}{h}{\textsubring{b}}{\textsci}{\textsubring{n}}} \dicPos{adj}[6]\dicFlx{}[-2] \dicSynonym{lánlaus} \dicDirectTranslationCS{mající smůlu\,/\addthin pech, smolařský} \dicAntonym{heppinn}
\dicEntry[óheppni] \dicTerm{ó··heppn|i} \dicIPA{{ou}{\textlengthmark}{h}{\textepsilon}{h}{\textsubring{b}}{n}{\textsci}} \dicPos{f}[3] \dicFlx{(‑i)}[3] \dicSynonym{ólán} \dicDirectTranslationCS{smůla, pech} \dicAntonym{heppni}
\dicEntry[óheyrilegur] \dicTerm{ó··heyri·legur} \dicIPA{{ou}{\textlengthmark}{h}{ei}{r}{\textsci}{l}{\textepsilon}{\textbabygamma}{\textscy}{\textsubring{r}}} \dicPos{adj}[1]\dicFlx{}[-8] \textbf{1.} \dicSynonym{fáheyrður} \dicDirectTranslationCS{nebývalý, neslýchaný} \dicExampleIS{óheyrilegur kostnaður} \dicExampleCS{neslýchané výdaje}  \textbf{2.} \dicDirectTranslationCS{neslyšitelný}
\dicEntry[óhirða] \dicTerm{ó··hirð|a} \dicIPA{{ou}{\textlengthmark}{h}{\textsci}{r}{ð}{a}} \dicPos{f}[1] \dicFlx{(‑u)}[5] \dicSynonym{hirðuleysi} \dicDirectTranslationCS{zanedbání, zanedbanost} \dicExampleIS{Safnið er í mikilli óhirðu.} \dicExampleCS{Sbírka je velmi zanedbaná.} \dicAntonym{hirða\smash{\textsuperscript{2}}}
\dicEntry[óhjákvæmilegur] \dicTerm{ó··hjá·kvæmi·legur} \dicsymFrequent\  \dicIPA{{ou}{\textlengthmark}{\c{c}}{au}{k\smash{\textsuperscript{h}}}{v}{a}{i}{m}{\textsci}{l}{\textepsilon}{\textbabygamma}{\textscy}{\textsubring{r}}} \dicPos{adj}[1]\dicFlx{}[-8] \dicSynonym{óumflýjanlegur} \dicDirectTranslationCS{nevyhnutelný, neodvratný} \dicExampleIS{óhjákvæmileg spurning} \dicExampleCS{nevyhnutelná otázka}
\dicEntry[óhljóð] \dicTerm{ó··hljóð} \dicIPA{{ou}{\textlengthmark}{\textsubring{l}}{j}{ou}{\texttheta}} \dicPos{n}[2] \dicFlx{(‑s, ‑)}[5] \dicSynonym{óp} \dicDirectTranslationCS{jek, jekot, řev}
\dicEntry[óhlutbundinn] \dicTerm{ó··hlut·bundinn} \dicIPA{{ou}{\textlengthmark}{\textsubring{l}}{\textscy}{\textsubring{d}}{\textsubring{b}}{\textscy}{n}{\textsubring{d}}{\textsci}{\textsubring{n}}} \dicPos{adj}[6]\dicFlx{}[-2] \textbf{1.} \dicSynonym{sérstæður} \dicDirectTranslationCS{nezávislý} \dicIndirectTranslationCS{(nepodléhající vlivu někoho nebo něčeho)} \dicAntonym{hlutbundinn}  \textbf{2.} \dicSynonym{afstraktur} \dicDirectTranslationCS{abstraktní, nefigurativní (obraz ap.)}
\dicEntry[óhlutdrægni] \dicTerm{ó··hlut·drægn|i} \dicIPA{{ou}{\textlengthmark}{\textsubring{l}}{\textscy}{\textsubring{d}}{r}{a}{i}{\r{g}}{n}{\textsci}} \dicPos{f}[3] \dicFlx{(‑i)}[3] \dicSynonym{réttsýni} \dicDirectTranslationCS{nestrannost, nezaujatost, objektivita} \dicAntonym{hlutdrægni}
\dicEntry[óhlutdrægur] \dicTerm{ó··hlut·drægur} \dicIPA{{ou}{\textlengthmark}{\textsubring{l}}{\textscy}{\textsubring{d}}{r}{a}{i}{\textbabygamma}{\textscy}{\textsubring{r}}} \dicPos{adj}[1]\dicFlx{}[-1] \dicSynonym{réttsýnn} \dicDirectTranslationCS{nestranný, nezaujatý, objektivní} \dicAntonym{hlutdrægur}
\dicEntry[óhlutstæður] \dicTerm{ó··hlut·stæður} \dicIPA{{ou}{\textlengthmark}{\textsubring{l}}{\textscy}{\textsubring{d}}{s}{\textsubring{d}}{a}{i}{ð}{\textscy}{\textsubring{r}}} \dicPos{adj}[2]\dicFlx{}[-6] \dicSynonym{abstrakt} \dicDirectTranslationCS{abstraktní} \dicExampleIS{óhlutstæð hugmynd} \dicExampleCS{abstraktní myšlenka} \dicAntonym{hlutstæður}
\dicEntry[óhlýðinn] \dicTerm{ó··hlýðinn} \dicIPA{{ou}{\textlengthmark}{\textsubring{l}}{i}{ð}{\textsci}{\textsubring{n}}} \dicPos{adj}[6]\dicFlx{}[-2] \dicSynonym{óþekkur} \dicDirectTranslationCS{neposlušný, nezbedný} \dicExampleIS{óhlýðinn krakki} \dicExampleCS{neposlušné dítě} \dicAntonym{hlýðinn}
\dicEntry[óhlýðni] \dicTerm{ó··hlýðn|i} \dicIPA{{ou}{\textlengthmark}{\textsubring{l}}{i}{ð}{n}{\textsci}} \dicPos{f}[3] \dicFlx{(‑i)}[3] \dicSynonym{óþekkt} \dicDirectTranslationCS{neposlušnost, nezbednost} \dicAntonym{hlýðni}
\dicEntry[óhollur] \dicTerm{ó··hollur} \dicIPA{{ou}{\textlengthmark}{h}{\textopeno}{\textsubring{d}}{l}{\textscy}{\textsubring{r}}} \dicPos{adj}[1]\dicFlx{}[-1] \dicSynonym{óheilnæmur} \dicDirectTranslationCS{nezdravý, zdraví škodlivý} \dicExampleIS{óhollur matur} \dicExampleCS{nezdravé jídlo} \dicAntonym{hollur}
\dicEntry[óhollusta] \dicTerm{ó··holl·ust|a} \dicIPA{{ou}{\textlengthmark}{h}{\textopeno}{\textsubring{d}}{l}{\textscy}{s}{\textsubring{d}}{a}} \dicPos{f}[1] \dicFlx{(‑u)}[5] \textbf{1.} \dicSynonym*{óheilnæmi} \dicDirectTranslationCS{nezdravost, škodlivost (jídla ap.)} \dicAntonym{hollusta}  \textbf{2.} \dicSynonym{ótryggð} \dicDirectTranslationCS{nevěrnost, neloajalita}
\dicEntry[óhóf] \dicTerm{ó··hóf} \dicIPA{{ou}{\textlengthmark}{h}{ou}{f}} \dicPos{n}[2] \dicFlx{(‑s)}[2] \dicSynonym*{hófleysi} \dicDirectTranslationCS{nestřídmost, přemíra, nadbytek} \dicExampleIS{skemmta sér í óhófi} \dicExampleCS{bavit se nestřídmě}
\dicEntry[óhóflegur] \dicTerm{ó··hóf·legur} \dicIPA{{ou}{\textlengthmark}{h}{ou}{v}{l}{\textepsilon}{\textbabygamma}{\textscy}{\textsubring{r}}} \dicPos{adj}[1]\dicFlx{}[-8] \dicSynonym{hóflaus} \dicDirectTranslationCS{nepřiměřený, nadměrný} \dicExampleIS{óhófleg notkun einkabílsins} \dicExampleCS{nadužívání osobního auta} \dicAntonym{hóflegur}
\dicEntry[óhraustur] \dicTerm{ó··hraustur} \dicIPA{{ou}{\textlengthmark}{\textsubring{r}}{\oe i}{s}{\textsubring{d}}{\textscy}{\textsubring{r}}} \dicPos{adj}[1]\dicFlx{}[-13] \textbf{1.} \dicSynonym{sjúkur} \dicDirectTranslationCS{neduživý, churavý} \dicAntonym{hraustur}  \textbf{2.} \dicSynonym{blauður} \dicDirectTranslationCS{rozpačitý, bázlivý}  \textbf{3.} \dicPhraseIS{óhraust} \dicFlx{f} \dicSynonym{vanfær} \dicDirectTranslationCS{těhotná, gravidní}
\dicEntry[óhreinindi] \dicTerm{ó··hrein·indi} \dicIPA{{ou}{\textlengthmark}{\textsubring{r}}{ei}{n}{\textsci}{n}{\textsubring{d}}{\textsci}} \dicPos{n}[2] \dicFlx{pl}[19] \textbf{1.} \dicSynonym{saur} \dicDirectTranslationCS{špína, nečistota}  \textbf{2.} \dicSynonym{sóðaskapur} \dicDirectTranslationCS{špinavost, ušpiněnost}
\dicEntry[óhreinka] \dicTerm{ó··hreink|a} \dicIPA{{ou}{\textlengthmark}{\textsubring{r}}{ei}{\r{\ng}}{\r{g}}{a}} \dicPos{v}[1] \dicFlx{(‑aði)}[1] \dicFlx{acc} \dicSynonym{saurga} \dicDirectTranslationCS{(za)špinit, znečistit} \dicExampleIS{óhreinka sig} \dicExampleCS{zašpinit se}
\dicEntry[óhreinn] \dicTerm{ó··hreinn} \dicIPA{{ou}{\textlengthmark}{\textsubring{r}}{ei}{\textsubring{d}}{\textsubring{n}}} \dicPos{adj}[7]\dicFlx{}[-1] \textbf{1.} \dicSynonym{saurugur} \dicDirectTranslationCS{špinavý, nečistý} \dicExampleIS{óhreinn þvottur} \dicExampleCS{špinavé prádlo} \dicAntonym{hreinn}  \textbf{2.} \dicSynonym{ótrúr} \dicDirectTranslationCS{nevěrný, neloajální}  \textbf{3.} \dicFieldCat{hud.} \dicSynonym{falskur} \dicDirectTranslationCS{nečistý, falešný} \dicExampleIS{óhreinn söngur} \dicExampleCS{falešné zpívání}
\dicEntry[óhreinskilinn] \dicTerm{ó··hrein·skilinn} \dicIPA{{ou}{\textlengthmark}{\textsubring{r}}{ei}{n}{s}{\r{\textObardotlessj}}{\textsci}{l}{\textsci}{\textsubring{n}}} \dicPos{adj}[6]\dicFlx{}[-2] \dicSynonym{falskur} \dicDirectTranslationCS{falešný, neupřímný} \dicAntonym{hreinskilinn}
\dicEntry[óhreinskilni] \dicTerm{ó··hrein·skiln|i} \dicIPA{{ou}{\textlengthmark}{\textsubring{r}}{ei}{n}{s}{\r{\textObardotlessj}}{\textsci}{l}{n}{\textsci}} \dicPos{f}[3] \dicFlx{(‑i)}[3] \dicSynonym{fals} \dicDirectTranslationCS{falešnost, neupřímnost} \dicAntonym{hreinskilni}
\dicEntry[óhrekjandi] \dicTerm{ó··hrekj·andi} \dicIPA{{ou}{\textlengthmark}{\textsubring{r}}{\textepsilon}{\r{\textObardotlessj}}{a}{n}{\textsubring{d}}{\textsci}} \dicPos{adj}[13] \dicFlx{indecl}[1] \dicSynonym{órækur} \dicDirectTranslationCS{nezvratný, nevyvratitelný} \dicExampleIS{óhrekjandi rök} \dicExampleCS{nezvratný důkaz}
\dicEntry[óhress] \dicTerm{ó··hress} \dicIPA{{ou}{\textlengthmark}{\textsubring{r}}{\textepsilon}{s}} \dicPos{adj}[5]\dicFlx{}[-1] \textbf{1.} \dicSynonym{lasinn} \dicDirectTranslationCS{nemající sílu, (jsoucí) bez energie, malátný} \dicAntonym{hress}  \textbf{2.} \dicSynonym{óánægður} \dicDirectTranslationCS{rozčarovaný, nespokojený} \dicExampleIS{vera óhress með e‑ð} \dicExampleCS{být rozčarovaný z~(čeho)}
\dicEntry[óhreyfanlegur] \dicTerm{ó··hreyfan·legur} \dicIPA{{ou}{\textlengthmark}{\textsubring{r}}{ei}{v}{a}{n}{l}{\textepsilon}{\textbabygamma}{\textscy}{\textsubring{r}}} \dicPos{adj}[1]\dicFlx{}[-8] \dicSynonym{óbifandi} \dicDirectTranslationCS{nepřemístitelný, nehybný} \dicAntonym{hreyfanlegur}
\dicEntry[óhrjálegur] \dicTerm{ó··hrjá·legur} \dicIPA{{ou}{\textlengthmark}{\textsubring{r}}{j}{au}{l}{\textepsilon}{\textbabygamma}{\textscy}{\textsubring{r}}} \dicPos{adj}[1]\dicFlx{}[-8] \dicDirectTranslationCS{zchátralý, sešlý, ošumělý, zpustlý}
\dicEntry[óhróður] \dicTerm{ó··hróð|ur} \dicIPA{{ou}{\textlengthmark}{\textsubring{r}}{ou}{ð}{\textscy}{\textsubring{r}}} \dicPos{m}[5] \dicFlx{(‑urs\,/\addthin ‑rar)}[10] \dicSynonym{níð} \dicDirectTranslationCS{pomluva, pomlouvání, urážka na cti};  \dicPhraseIS{bera út óhróður um e‑n} \dicDirectTranslationCS{šířit o~(kom) pomluvy}
\dicEntry[óhræddur] \dicTerm{ó··hræddur} \dicIPA{{ou}{\textlengthmark}{\textsubring{r}}{a}{i}{\textsubring{d}}{\textscy}{\textsubring{r}}} \dicPos{adj}[2]\dicFlx{}[-18] \dicSynonym{óbanginn} \dicDirectTranslationCS{nemající strach, neohrožený, nebojácný} \dicAntonym{hræddur}
\dicEntry[óhuggandi] \dicTerm{ó··hugg·andi} \dicIPA{{ou}{\textlengthmark}{h}{\textscy}{\r{g}}{a}{n}{\textsubring{d}}{\textsci}} \dicPos{adj}[13] \dicFlx{indecl}[1] \dicDirectTranslationCS{neutěšitelný, bezútěšný}
\dicEntry[óhuggulegur] \dicTerm{ó··huggu·legur} \dicIPA{{ou}{\textlengthmark}{h}{\textscy}{\r{g}}{\textscy}{l}{\textepsilon}{\textbabygamma}{\textscy}{\textsubring{r}}} \dicPos{adj}[1]\dicFlx{}[-8] \dicSynonym{óhugnanlegur} \dicDirectTranslationCS{hrůzný, strašlivý}
\dicEntry[óhugnaður] \dicTerm{ó··hug·nað|ur} \dicIPA{{ou}{\textlengthmark}{h}{\textscy}{\r{g}}{n}{a}{ð}{\textscy}{\textsubring{r}}} \dicPos{m}[10] \dicFlx{(‑ar)}[9] \textbf{1.} \dicSynonym{óþægindi} \dicDirectTranslationCS{nesnáz, potíž}  \textbf{2.} \dicSynonym{böl} \dicDirectTranslationCS{zlo}  \textbf{3.} \dicSynonym{andstyggð} \dicDirectTranslationCS{znechucení, odpor}
\dicEntry[óhugnanlegur] \dicTerm{ó··hugnan·legur} \dicIPA{{ou}{\textlengthmark}{h}{\textscy}{\r{g}}{n}{a}{n}{l}{\textepsilon}{\textbabygamma}{\textscy}{\textsubring{r}}} \dicPos{adj}[1]\dicFlx{}[-8] \dicSynonym{ískyggilegur} \dicDirectTranslationCS{děsivý, úděsný} \dicExampleIS{óhugnanleg sjón} \dicExampleCS{děsivý pohled}
\dicEntry[óhugsandi] \dicTerm{ó··hugs·andi} \dicsymFrequent\  \dicIPA{{ou}{\textlengthmark}{h}{\textscy}{x}{s}{a}{n}{\textsubring{d}}{\textsci}} \dicPos{adj}[13] \dicFlx{indecl}[1] \dicSynonym*{óhugsanlegur} \dicDirectTranslationCS{nemyslitelný, nemožný} \dicExampleIS{það er óhugsandi að (gera e‑ð)} \dicExampleCS{je nemyslitelné ((co) udělat)}
\dicEntry[óhugur] \dicTerm{ó··hug|ur} \dicIPA{{ou}{\textlengthmark}{h}{\textscy}{\textbabygamma}{\textscy}{\textsubring{r}}} \dicPos{m}[10] \dicFlx{(‑ar)}[12] \dicSynonym{hræðsla} \dicDirectTranslationCS{obava, znepokojení};  \dicPhraseIS{óhug slær á e‑n við e‑ð} \dicFlx{impers} \dicDirectTranslationCS{(kdo) se znepokojuje kvůli (čemu)}
\dicEntry[óhultur] \dicTerm{ó··hultur} \dicIPA{{ou}{\textlengthmark}{h}{\textscy}{\textsubring{l}}{\textsubring{d}}{\textscy}{\textsubring{r}}} \dicPos{adj}[1]\dicFlx{}[-10] \dicSynonym{öruggur} \dicDirectTranslationCS{bezpečný, (jsoucí) mimo nebezpečí} \dicExampleIS{óhultur um líf sitt} \dicExampleCS{mimo nebezpečí života}
\dicEntry[óhyggilegur] \dicTerm{ó··hyggi·legur} \dicIPA{{ou}{\textlengthmark}{h}{\textsci}{\r{\textObardotlessj}}{\textsci}{l}{\textepsilon}{\textbabygamma}{\textscy}{\textsubring{r}}} \dicPos{adj}[1]\dicFlx{}[-8] \dicDirectTranslationCS{nerozumný, nerozvážný}
\dicEntry[óhæfa] \dicTerm{ó··hæf|a} \dicIPA{{ou}{\textlengthmark}{h}{a}{i}{v}{a}} \dicPos{f}[1] \dicFlx{(‑u)}[5] \textbf{1.} \dicSynonym{ósvinna} \dicDirectTranslationCS{nevhodnost, nepatřičnost, nemístnost} \dicExampleIS{Það er alger óhæfa að heilsa ekki forsetanum.} \dicExampleCS{Je nemístné nepozdravit prezidenta.}  \textbf{2.} \dicSynonym{ódæði} \dicDirectTranslationCS{zvěrstvo, krutost} \dicExampleIS{óhæfuverk} \dicExampleCS{krutý čin}  \textbf{3.} \dicSynonym{ógæfa} \dicDirectTranslationCS{pech, neštěstí}  \textbf{4.} \dicSynonym{ógrynni} \dicDirectTranslationCS{značný počet, spousta}
\dicEntry[óhæfilegur] \dicTerm{ó··hæfi·legur} \dicIPA{{ou}{\textlengthmark}{h}{a}{i}{v}{\textsci}{l}{\textepsilon}{\textbabygamma}{\textscy}{\textsubring{r}}} \dicPos{adj}[1]\dicFlx{}[-8] \dicSynonym{óhóflegur} \dicDirectTranslationCS{nepatřičný, nemístný, nevhodný} \dicAntonym{hæfilegur}
\dicEntry[óhæfur] \dicTerm{ó··hæfur} \dicIPA{{ou}{\textlengthmark}{h}{a}{i}{v}{\textscy}{\textsubring{r}}} \dicPos{adj}[1]\dicFlx{}[-1] \textbf{1.} \dicSynonym{ónothæfur} \dicDirectTranslationCS{nevhodný, nehodící se}  \textbf{2.} \dicSynonym{ósæmilegur} \dicDirectTranslationCS{nezpůsobilý, nekvalifikovaný} \dicExampleIS{vera óhæfur í starfið} \dicExampleCS{nemít kvalifikaci na vykonávání práce} \dicAntonym{hæfur}
\dicEntry[óhættur] \dicTerm{ó··hættur} \dicIPA{{ou}{\textlengthmark}{h}{a}{i}{h}{\textsubring{d}}{\textscy}{\textsubring{r}}} \dicPos{adj}[1]\dicFlx{}[-13] \dicLangCat{zast.} \dicSynonym{hættulaus} \dicDirectTranslationCS{(jsoucí) mimo nebezpečí};  \dicPhraseIS{e‑m er óhætt} \dicFlx{impers} \dicDirectTranslationCS{(kdo) je mimo nebezpečí}
\dicEntry[ójafn] \dicTerm{ó··|jafn} \dicIPA{{ou}{\textlengthmark}{j}{a}{\textsubring{b}}{\textsubring{n}}} \dicPos{adj}[5] \dicFlx{(f ‑jöfn)}[6] \textbf{1.} \dicSynonym{misjafn} \dicDirectTranslationCS{nerovný, nestejný} \dicAntonym{jafn}  \textbf{2.} \dicFieldCat{mat.} \dicDirectTranslationCS{lichý};  \dicPhraseIS{ójöfn tala} \dicFieldCat{mat.} \dicSynonym{oddatala} \dicDirectTranslationCS{liché číslo}
\dicEntry[ójafna] \dicTerm{ó··|jafna} \dicIPA{{ou}{\textlengthmark}{j}{a}{\textsubring{b}}{n}{a}} \dicPos{f}[1] \dicFlx{(‑jöfnu, ‑jöfnur)}[8] \textbf{1.} \dicSynonym*{misjafna} \dicDirectTranslationCS{nerovnost, hrbol} \dicAntonym{jafna\smash{\textsuperscript{1}}}  \textbf{2.} \dicFieldCat{mat.} \dicDirectTranslationCS{nerovnice}
\dicEntry[ójafnaðarmaður] \dicTerm{ó·jafnaðar··|maður} \dicIPA{{ou}{\textlengthmark}{j}{a}{\textsubring{b}}{n}{a}{ð}{a}{r}{m}{a}{ð}{\textscy}{\textsubring{r}}} \dicPos{m}[13] \dicFlx{(‑manns, ‑menn)}[2] \dicDirectTranslationCS{agresivní člověk}
\dicEntry[ójafnaður] \dicTerm{ó··|jafn·aður} \dicIPA{{ou}{\textlengthmark}{j}{a}{\textsubring{b}}{n}{a}{ð}{\textscy}{\textsubring{r}}} \dicPos{adj}[3] \dicFlx{(f ‑jöfnuð)}[2] \dicDirectTranslationCS{nezarovnaný, nevyrovnaný (text ap.)}
\dicEntry[ójafnvægi] \dicTerm{ó··jafn·vægi} \dicIPA{{ou}{\textlengthmark}{j}{a}{\textsubring{b}}{n}{v}{a}{i}{j}{\textsci}} \dicPos{n}[2] \dicFlx{(‑s)}[20] \textbf{1.} \dicDirectTranslationCS{nerovnováha} \dicAntonym{jafnvægi}  \textbf{2.} \dicSynonym*{tilfinningarót} \dicDirectTranslationCS{(duševní) nerovnováha\,/\addthin zmatek}
\dicEntry[ójöfnuður] \dicTerm{ó··|jöfn·uður} \dicIPA{{ou}{\textlengthmark}{j}{\oe}{\textsubring{b}}{n}{\textscy}{ð}{\textscy}{\textsubring{r}}} \dicPos{m}[10] \dicFlx{(‑jafnaðar\,/\addthin ‑jöfnuðar)}[43] \textbf{1.} \dicSynonym{mismunur} \dicDirectTranslationCS{rozdíl (číselný ap.)}  \textbf{2.} \dicSynonym{ranglæti} \dicDirectTranslationCS{nespravedlnost, křivda} \dicExampleIS{gera henni ójöfnuð} \dicExampleCS{ukřivdit jí}
\dicEntry[ók] \dicTerm{ók} \dicIPA{{ou}{\textlengthmark}{\r{g}}} \dicPos{v} \dicFlx{ind pf sg 1 pers} \dicLink{aka}
\dicEntry[ókannaður] \dicTerm{ó··|kann·aður} \dicIPA{{ou}{\textlengthmark}{k\smash{\textsuperscript{h}}}{a}{n}{a}{ð}{\textscy}{\textsubring{r}}} \dicPos{adj}[3] \dicFlx{(f ‑könnuð)}[1] \dicDirectTranslationCS{neprozkoumaný, neprověřený}
\dicEntry[ókei] \dicTerm{ókei} \dicIPA{{ou}{\textlengthmark}{\r{\textObardotlessj}}{ei}} \dicPos{inter} \dicLangCat{hovor.} \dicDirectTranslationCS{v~pořádku, fajn, prima, ok}
\dicEntry[ókennilegur] \dicTerm{ó··kenni·legur} \dicIPA{{ou}{\textlengthmark}{c\smash{\textsuperscript{h}}}{\textepsilon}{n}{\textsci}{l}{\textepsilon}{\textbabygamma}{\textscy}{\textsubring{r}}} \dicPos{adj}[1]\dicFlx{}[-8] \textbf{1.} \dicSynonym{óþekkjanlegur} \dicDirectTranslationCS{nerozpoznatelný, nepoznatelný}  \textbf{2.} \dicSynonym{óskiljanlegur} \dicDirectTranslationCS{tajemný, neobjasněný}
\dicEntry[ókeypis] \dicTerm{ó··keypis\smash{\textsuperscript{1}}} \dicIPA{{ou}{\textlengthmark}{c\smash{\textsuperscript{h}}}{ei}{\textsubring{b}}{\textsci}{s}} \dicPos{adj}[13] \dicFlx{indecl}[1] \dicDirectTranslationCS{bezplatný, (jsoucí) zadarmo\,/\addthin zdarma} \dicExampleIS{ókeypis læknisþjónusta} \dicExampleCS{bezplatná zdravotní služba}
\dicEntry[ókeypis] \dicTerm{ó··keypis\smash{\textsuperscript{2}}} \dicIPA{{ou}{\textlengthmark}{c\smash{\textsuperscript{h}}}{ei}{\textsubring{b}}{\textsci}{s}} \dicPos{adv} \dicSynonym{gefins} \dicDirectTranslationCS{zadarmo, zdarma, bezplatně}
\dicEntry[ókjör] \dicTerm{ó··kjör} \dicIPA{{ou}{\textlengthmark}{c\smash{\textsuperscript{h}}}{\oe}{\textsubring{r}}} \dicPos{n}[2] \dicFlx{pl}[9] \dicSynonym{ósköp\smash{\textsuperscript{1}}} \dicDirectTranslationCS{veliké množství, masa} \dicExampleIS{ókjör af e‑u} \dicExampleCS{masa (čeho)}
\dicEntry[óklár] \dicTerm{ó··klár} \dicIPA{{ou}{\textlengthmark}{k\smash{\textsuperscript{h}}}{l}{au}{\textsubring{r}}} \dicPos{adj}[5] \dicFlx{(f ‑)}[8] \dicSynonym{óskýr} \dicDirectTranslationCS{nejasný, nejistý}
\dicEntry[ókláraður] \dicTerm{ó··klár·|aður} \dicIPA{{ou}{\textlengthmark}{k\smash{\textsuperscript{h}}}{l}{au}{r}{a}{ð}{\textscy}{\textsubring{r}}} \dicPos{adj}[3] \dicFlx{(f ‑uð)}[4] \dicDirectTranslationCS{neukončený, nedokončený, nedodělaný} \dicExampleIS{óklárað verkefni} \dicExampleCS{nedokončený úkol}
\dicEntry[ókleifur] \dicTerm{ó··kleifur} \dicIPA{{ou}{\textlengthmark}{k\smash{\textsuperscript{h}}}{l}{ei}{v}{\textscy}{\textsubring{r}}} \dicPos{adj}[1]\dicFlx{}[-1] \textbf{1.} \dicSynonym{ófær} \dicDirectTranslationCS{neschůdný, nepřístupný (hora ap.)} \dicExampleIS{ókleift fjall} \dicExampleCS{neschůdná hora} \dicAntonym{kleifur}  \textbf{2.} \dicSynonym{ómögulegur} \dicDirectTranslationCS{nemožný, neproveditelný}
\dicEntry[óknyttir] \dicTerm{ó··knyttir} \dicIPA{{ou}{\textlengthmark}{k\smash{\textsuperscript{h}}}{n}{\textsci}{h}{\textsubring{d}}{\textsci}{\textsubring{r}}} \dicPos{m}[9] \dicFlx{pl}[2] \dicSynonym*{strákastrik} \dicDirectTranslationCS{rošťárny, darebnosti}
\dicEntry[ókominn] \dicTerm{ó··kominn} \dicsymFrequent\  \dicIPA{{ou}{\textlengthmark}{k\smash{\textsuperscript{h}}}{\textopeno}{m}{\textsci}{\textsubring{n}}} \dicPos{adj}[6]\dicFlx{}[-6] \textbf{1.} \dicDirectTranslationCS{nepřišlý} \dicExampleIS{Hann er ennþá ókominn.} \dicExampleCS{Ještě nepřišel.}  \textbf{2.} \dicSynonym*{óorðinn} \dicDirectTranslationCS{budoucí, nastávající} \dicExampleIS{ókominn tími} \dicExampleCS{budoucí doba}
\dicEntry[ókostur] \dicTerm{ó··kost|ur} \dicIPA{{ou}{\textlengthmark}{k\smash{\textsuperscript{h}}}{\textopeno}{s}{\textsubring{d}}{\textscy}{\textsubring{r}}} \dicPos{m}[10] \dicFlx{(‑ar, ‑ir)}[4] \dicSynonym{galli\smash{\textsuperscript{2}}} \dicDirectTranslationCS{nevýhoda, nedostatek, minus} \dicExampleIS{ókosturinn við þessa aðferð} \dicExampleCS{nevýhoda této metody} \dicAntonym{kostur}
\dicEntry[ókringdur] \dicTerm{ó··kringdur} \dicIPA{{ou}{\textlengthmark}{k\smash{\textsuperscript{h}}}{r}{\textsci}{\ng}{\textsubring{d}}{\textscy}{\textsubring{r}}} \dicPos{adj}[2]\dicFlx{}[-17] \dicFieldCat{jaz.} \dicDirectTranslationCS{nezaokrouhlený (samohláska ap.)}
\dicEntry[ókristilegur] \dicTerm{ó··kristi·legur} \dicIPA{{ou}{\textlengthmark}{k\smash{\textsuperscript{h}}}{r}{\textsci}{s}{\textsubring{d}}{\textsci}{l}{\textepsilon}{\textbabygamma}{\textscy}{\textsubring{r}}} \dicPos{adj}[1]\dicFlx{}[-8] \dicSynonym{óguðlegur} \dicDirectTranslationCS{nekřesťanský} \dicAntonym{kristilegur};  \dicPhraseIS{á ókristilegum tíma} \dicFlx{adv} \dicLangCat{přen.} \dicDirectTranslationCS{v~nekřesťanskou hodinu}
\dicEntry[ókræsilegur] \dicTerm{ó··kræsi·legur} \dicIPA{{ou}{\textlengthmark}{k\smash{\textsuperscript{h}}}{r}{a}{i}{s}{\textsci}{l}{\textepsilon}{\textbabygamma}{\textscy}{\textsubring{r}}} \dicPos{adj}[1]\dicFlx{}[-8] \dicSynonym{óálitlegur} \dicDirectTranslationCS{nelákavý, nevábný, nepřitažlivý} \dicExampleIS{ókræsileg atvinna} \dicExampleCS{nevábný způsob obživy}
\dicEntry[ókum] \dicTerm{ókum} \dicIPA{{ou}{\textlengthmark}{\r{g}}{\textscy}{\textsubring{m}}} \dicPos{v} \dicFlx{ind pf pl 1 pers} \dicLink{aka}
\dicEntry[ókunnugur] \dicTerm{ó··kunn·ugur} \dicsymFrequent\  \dicIPA{{ou}{\textlengthmark}{k\smash{\textsuperscript{h}}}{\textscy}{n}{\textscy}{\textbabygamma}{\textscy}{\textsubring{r}}} \dicPos{adj}[1]\dicFlx{}[-8] \textbf{1.} \dicSynonym{ókunnur} \dicDirectTranslationCS{neznalý, neseznámený} \dicExampleIS{Ég er því ókunnugur.} \dicExampleCS{S~tím jsem se neseznámil.} \dicAntonym{kunnugur}  \textbf{2.} \dicSynonym{framandi} \dicDirectTranslationCS{neznámý, neznalý (člověk ap.)}
\dicEntry[ókunnur] \dicTerm{ó··kunnur} \dicsymFrequent\  \dicIPA{{ou}{\textlengthmark}{k\smash{\textsuperscript{h}}}{\textscy}{n}{\textscy}{\textsubring{r}}} \dicPos{adj}[1]\dicFlx{}[-1] \textbf{1.} \dicSynonym{ókunnugur} \dicDirectTranslationCS{neznámý} \dicExampleIS{vera ókunnur í hópi listamanna} \dicExampleCS{být neznámý v~kruhu umělců} \dicAntonym{kunnur}  \textbf{2.} \dicSynonym{nafnlaus} \dicDirectTranslationCS{bezejmenný, anonymní}
\dicEntry[ókurteis] \dicTerm{ó··kurteis} \dicIPA{{ou}{\textlengthmark}{k\smash{\textsuperscript{h}}}{\textscy}{\textsubring{r}}{\textsubring{d}}{ei}{s}} \dicPos{adj}[5]\dicFlx{}[-1] \dicSynonym{dónalegur} \dicDirectTranslationCS{nezdvořilý, nevychovaný} \dicAntonym{kurteis}
\dicEntry[ókurteisi] \dicTerm{ó··kurteis|i} \dicIPA{{ou}{\textlengthmark}{k\smash{\textsuperscript{h}}}{\textscy}{\textsubring{r}}{\textsubring{d}}{ei}{s}{\textsci}} \dicPos{f}[3] \dicFlx{(‑i)}[3] \dicDirectTranslationCS{nezdvořilost, nevychovanost} \dicAntonym{kurteisi}
\dicEntry[ókvíðinn] \dicTerm{ó··kvíðinn} \dicIPA{{ou}{\textlengthmark}{k\smash{\textsuperscript{h}}}{v}{i}{ð}{\textsci}{\textsubring{n}}} \dicPos{adj}[6]\dicFlx{}[-2] \dicSynonym{djarfur} \dicDirectTranslationCS{klidný, (jsoucí) bez obav} \dicAntonym{kvíðinn}
\dicEntry[ókvæða] \dicTerm{ó··kvæða} \dicIPA{{ou}{\textlengthmark}{k\smash{\textsuperscript{h}}}{v}{a}{i}{ð}{a}} \dicPos{adj}[13] \dicFlx{indecl}[1] \dicPhraseIS{bregðast ókvæða við e‑u} \dicFlx{refl} \dicLangCat{přen.} \dicDirectTranslationCS{prudce na (co) zareagovat}
\dicEntry[ókvæðisorð] \dicTerm{ó·kvæðis··orð} \dicIPA{{ou}{\textlengthmark}{k\smash{\textsuperscript{h}}}{v}{a}{i}{ð}{\textsci}{s}{\textopeno}{r}{\texttheta}} \dicPos{n}[2] \dicFlx{(‑s, ‑)}[5] \dicDirectTranslationCS{urážka}
\dicEntry[ókvæntur] \dicTerm{ó··kvæntur} \dicIPA{{ou}{\textlengthmark}{k\smash{\textsuperscript{h}}}{v}{a}{i}{\textsubring{n}}{\textsubring{d}}{\textscy}{\textsubring{r}}} \dicPos{adj}[1]\dicFlx{}[-13] \dicSynonym{einhleypur} \dicDirectTranslationCS{neoženěný, nesezdaný, svobodný} \dicIndirectTranslationCS{(o~muži)} \dicExampleIS{vera ókvæntur og barnlaus} \dicExampleCS{být svobodný a~bez dětí} \dicAntonym{kvæntur}
\dicEntry[ókyrr] \dicTerm{ó··kyrr} \dicIPA{{ou}{\textlengthmark}{c\smash{\textsuperscript{h}}}{\textsci}{r}} \dicPos{adj}[5] \dicFlx{(f ‑)}[8] \textbf{1.} \dicSynonym{órólegur} \dicDirectTranslationCS{nepokojný, neklidný} \dicExampleIS{sofa ókyrrum svefni} \dicExampleCS{spát nepokojným spánkem}  \textbf{2.} \dicSynonym{óstöðugur} \dicDirectTranslationCS{turbulentní, nestabilní}
\dicEntry[ókyrrast] \dicTerm{ó··kyrr|ast} \dicIPA{{ou}{\textlengthmark}{c\smash{\textsuperscript{h}}}{\textsci}{r}{a}{s}{\textsubring{d}}} \dicPos{v}[2] \dicFlx{(‑ðist, ‑st)}[202] \dicFlx{refl} \dicDirectTranslationCS{zneklidnět, znepokojit se}
\dicEntry[ókyrrð] \dicTerm{ó··kyrrð} \dicIPA{{ou}{\textlengthmark}{c\smash{\textsuperscript{h}}}{\textsci}{r}{\texttheta}} \dicPos{f}[4] \dicFlx{(‑ar)}[3] \dicSynonym{órói} \dicDirectTranslationCS{nepokoj, neklid} \dicAntonym{kyrrð}
\dicEntry[ól] \dicTerm{ól\smash{\textsuperscript{1}}} \dicIPA{{ou}{\textlengthmark}{\textsubring{l}}} \dicPos{f}[4] \dicFlx{(‑ar, ‑ar)}[1] \dicSynonym{þvengur} \dicDirectTranslationCS{řemínek, popruh, pásek} \dicExampleIS{Hún var með svarta skjalatösku í ól á öxlinni.} \dicExampleCS{Na rameni měla na řemínku zavěšenou černou aktovku.};  \dicPhraseIS{elta ólar við e‑ð} \dicLangCat{přen.} \dicSynonym*{eltast við e‑ð} \dicDirectTranslationCS{soustředit se na (co), brát si na mušku (co)}
\dicEntry[ól] \dicTerm{ól\smash{\textsuperscript{2}}} \dicIPA{{ou}{\textlengthmark}{\textsubring{l}}} \dicPos{v} \dicFlx{ind pf sg 1 pers} \dicLink{ala}
\dicEntry[ólafssúra] \dicTerm{ólafs··súr|a} \dicIPA{{ou}{\textlengthmark}{l}{a}{f}{s}{u}{r}{a}} \dicPos{f}[1] \dicFlx{(‑u, ‑ur)}[7] \dicFieldCat{bot.} \dicDirectTranslationCS{šťovíček dvoublizný} \textit{(l.~{\textLA{Oxyria digyna}})}  \dicsymPhoto\ 
\dicFigure{77784.jpg}{Ólafssúra}{Ólafssúra - Hroneš Michal, Biolib, Copyright/CC-BY-NC}
\dicEntry[ólag] \dicTerm{ó··|lag} \dicIPA{{ou}{\textlengthmark}{l}{a}{x}} \dicPos{n}[2] \dicFlx{(‑lags, ‑lög)}[8] \textbf{1.} \dicSynonym{óreiða} \dicDirectTranslationCS{nepořádek, chaos}  \textbf{2.} \dicSynonym{bilun} \dicDirectTranslationCS{porucha, závada} \dicExampleIS{Það er ólag á vélinni.} \dicExampleCS{Stroj je porouchaný.};  \dicPhraseIS{í ólagi} \dicFlx{adv} \dicDirectTranslationCS{mimo provoz}  \textbf{3.} \dicSynonym*{hættuleg bylgja} \dicDirectTranslationCS{velká, životu nebezpečná vlna (na moři)}
\dicEntry[ólaginn] \dicTerm{ó··laginn} \dicIPA{{ou}{\textlengthmark}{l}{a}{i}{j}{\textsci}{\textsubring{n}}} \dicPos{adj}[6]\dicFlx{}[-3] \dicSynonym{klaufskur} \dicDirectTranslationCS{neobratný, nešikovný} \dicAntonym{laginn}
\dicEntry[ólaglegur] \dicTerm{ó··lag·legur} \dicIPA{{ou}{\textlengthmark}{l}{a}{\textbabygamma}{l}{\textepsilon}{\textbabygamma}{\textscy}{\textsubring{r}}} \dicPos{adj}[1]\dicFlx{}[-8] \dicSynonym{ófríður} \dicDirectTranslationCS{nepohledný, nehezký, nepěkný} \dicAntonym{laglegur}
\dicEntry[ólán] \dicTerm{ó··lán} \dicIPA{{ou}{\textlengthmark}{l}{au}{\textsubring{n}}} \dicPos{n}[2] \dicFlx{(‑s)}[2] \dicSynonym{óheppni} \dicDirectTranslationCS{smůla, neštěstí, pech} \dicAntonym{lán};  \dicPhraseIS{það var lán í óláni að} \dicLangCat{přen.} \dicDirectTranslationCS{to bylo štěstí v~neštěstí, že}
\dicEntry[ólánsamur] \dicTerm{ó··lán·|samur} \dicIPA{{ou}{\textlengthmark}{l}{au}{n}{s}{a}{m}{\textscy}{\textsubring{r}}} \dicPos{adj}[1] \dicFlx{(f ‑söm)}[2] \dicSynonym{ógæfusamur} \dicDirectTranslationCS{smolařský, nešťastný, mající pech} \dicAntonym{lánsamur}
\dicEntry[óleikur] \dicTerm{ó··leik|ur} \dicIPA{{ou}{\textlengthmark}{l}{ei}{\r{g}}{\textscy}{\textsubring{r}}} \dicPos{m}[6] \dicFlx{(‑s)}[17] \dicSynonym{hrekkur} \dicDirectTranslationCS{neplecha, rošťárna};  \dicPhraseIS{gera e‑m óleik} \dicDirectTranslationCS{provést (komu) neplechu}
\dicEntry[ólestur] \dicTerm{ó··lest|ur} \dicIPA{{ou}{\textlengthmark}{l}{\textepsilon}{s}{\textsubring{d}}{\textscy}{\textsubring{r}}} \dicPos{m}[5] \dicFlx{(‑rar\,/\addthin ‑urs, ‑rar)}[9] \dicSynonym{óreiða} \dicDirectTranslationCS{nepořádek, chaos};  \dicPhraseIS{í ólestri} \dicFlx{adv} \dicDirectTranslationCS{v~nepořádku}
\dicEntry[óleyfi] \dicTerm{ó··leyfi} \dicIPA{{ou}{\textlengthmark}{l}{ei}{v}{\textsci}} \dicPos{n}[2] \dicFlx{(‑s)}[20] \dicSynonym{leyfisleysi} \dicDirectTranslationCS{nepovolení, nedovolení} \dicAntonym{leyfi};  \dicPhraseIS{í óleyfi} \dicFlx{adv} \dicDirectTranslationCS{bez dovolení}
\dicEntry[óleyfilegur] \dicTerm{ó··leyfi·legur} \dicIPA{{ou}{\textlengthmark}{l}{ei}{v}{\textsci}{l}{\textepsilon}{\textbabygamma}{\textscy}{\textsubring{r}}} \dicPos{adj}[1]\dicFlx{}[-8] \dicSynonym{óheimill} \dicDirectTranslationCS{nepovolený, nedovolený} \dicAntonym{leyfilegur}
\dicEntry[óleysanlegur] \dicTerm{ó··leysan·legur} \dicIPA{{ou}{\textlengthmark}{l}{ei}{s}{a}{n}{l}{\textepsilon}{\textbabygamma}{\textscy}{\textsubring{r}}} \dicPos{adj}[1]\dicFlx{}[-8] \textbf{1.} \dicFieldCat{chem.} \dicDirectTranslationCS{nerozpustný}  \textbf{2.} \dicDirectTranslationCS{nevyřešitelný, neřešitelný}
\dicEntry[ólétta] \dicTerm{ó··létt|a} \dicIPA{{ou}{\textlengthmark}{l}{j}{\textepsilon}{h}{\textsubring{d}}{a}} \dicPos{f}[1] \dicFlx{(‑u, ‑ur)}[7] \dicDirectTranslationCS{těhotenství}
\dicEntry[óléttukjóll] \dicTerm{ó·léttu··kjól|l} \dicIPA{{ou}{\textlengthmark}{l}{j}{\textepsilon}{h}{\textsubring{d}}{\textscy}{c\smash{\textsuperscript{h}}}{ou}{\textsubring{d}}{\textsubring{l}}} \dicPos{m}[6] \dicFlx{(‑s, ‑ar)}[48] \dicDirectTranslationCS{těhotenské šaty}
\dicEntry[óléttur] \dicTerm{ó··léttur} \dicIPA{{ou}{\textlengthmark}{l}{j}{\textepsilon}{h}{\textsubring{d}}{\textscy}{\textsubring{r}}} \dicPos{adj}[1]\dicFlx{}[-10] \textbf{1.} \dicPhraseIS{ólétt} \dicFlx{f} \dicSynonym{vanfær} \dicDirectTranslationCS{těhotná, gravidní}  \textbf{2.} \dicSynonym{þungur} \dicDirectTranslationCS{těžký, obtížný} \dicAntonym{léttur}
\dicEntry[ólga] \dicTerm{ólg|a\smash{\textsuperscript{1}}} \dicIPA{{ou}{l}{\r{g}}{a}} \dicPos{f}[1] \dicFlx{(‑u)}[5] \textbf{1.} \dicSynonym{öldugangur} \dicDirectTranslationCS{vzdouvání (vln ap.)} \dicExampleIS{ólga hafsins} \dicExampleCS{vzdouvání moře}  \textbf{2.} \dicSynonym{gerjun} \dicDirectTranslationCS{fermentace, kvašení}  \textbf{3.} \dicSynonym{æsingur} \dicDirectTranslationCS{neklid, vření (v~národě ap.)} \dicExampleIS{Það var mikil ólga í landinu á þessum árum.} \dicExampleCS{V~těch letech panoval v~zemi velký neklid.}
\dicEntry[ólga] \dicTerm{ólg|a\smash{\textsuperscript{2}}} \dicIPA{{ou}{l}{\r{g}}{a}} \dicPos{v}[1] \dicFlx{(‑aði)}[44] \textbf{1.} \dicSynonym{freyða} \dicDirectTranslationCS{vzdouvat se, vlnit se (o~moři ap.)} \dicExampleIS{Sjórinn ólgar.} \dicExampleCS{Moře se vzdouvá.}  \textbf{2.} \dicSynonym{sjóða} \dicDirectTranslationCS{vařit se, bublat} \dicExampleIS{Það ólgar í katlinum.} \dicExampleCS{V~konvici už to bublá.}  \textbf{3.} \dicSynonym*{fá í sig gerjun} \dicDirectTranslationCS{kvasit, fermentovat}  \textbf{4.} \dicSynonym*{vera í uppnámi} \dicDirectTranslationCS{vřít, kypět (krev ap.)} \dicExampleIS{Blóðið ólgar í æðum mér.} \dicExampleCS{Krev mi vře v~žilách.}
\dicEntry[óliðlegur] \dicTerm{ó··lið·legur} \dicIPA{{ou}{\textlengthmark}{l}{\textsci}{ð}{l}{\textepsilon}{\textbabygamma}{\textscy}{\textsubring{r}}} \dicPos{adj}[1]\dicFlx{}[-8] \dicDirectTranslationCS{neochotný, nenápomocný}
\dicEntry[ólifnaður] \dicTerm{ó··lif·nað|ur} \dicIPA{{ou}{\textlengthmark}{l}{\textsci}{\textsubring{b}}{n}{a}{ð}{\textscy}{\textsubring{r}}} \dicPos{m}[10] \dicFlx{(‑ar)}[7] \dicSynonym*{saurlífi} \dicDirectTranslationCS{hýření, flámování}
\dicEntry[ólitaður] \dicTerm{ó··lit·|aður} \dicIPA{{ou}{\textlengthmark}{l}{\textsci}{\textsubring{d}}{a}{ð}{\textscy}{\textsubring{r}}} \dicPos{adj}[3] \dicFlx{(f ‑uð)}[3] \dicDirectTranslationCS{neobarvený, nezabarvený}
\dicEntry[ólífa] \dicTerm{ólíf|a}\dicTerm{, ólíva} \dicIPA{{ou}\-{\textlengthmark}\-{l}\-{i}\-{v}\-{a}\-} \dicPos{f}[1] \dicFlx{(‑u, ‑ur)}[7] \dicDirectTranslationCS{oliva}
\dicEntry[ólífi] \dicTerm{ó··lífi} \dicIPA{{ou}{\textlengthmark}{l}{i}{v}{\textsci}} \dicPos{n}[2] \dicFlx{(‑s)}[20] \dicSynonym{bani} \dicDirectTranslationCS{smrt};  \dicPhraseIS{særa e‑n til ólífis} \dicDirectTranslationCS{smrtelně (koho) zranit}
\dicEntry[ólífrænn] \dicTerm{ó··líf·rænn} \dicIPA{{ou}{\textlengthmark}{l}{i}{v}{r}{a}{i}{\textsubring{d}}{\textsubring{n}}} \dicPos{adj}[7]\dicFlx{}[-1] \dicFieldCat{biol.} \dicDirectTranslationCS{anorganický} \dicAntonym{lífrænn}
\dicEntry[ólífutré] \dicTerm{ólífu··tré} \dicIPA{{ou}{\textlengthmark}{l}{i}{v}{\textscy}{t\smash{\textsuperscript{h}}}{r}{j}{\textepsilon}} \dicPos{n}[2] \dicFlx{(‑s, ‑)}[36] \dicFieldCat{bot.} \dicDirectTranslationCS{olivovník} \textit{(l.~{\textLA{Olea}})}  \dicsymPhoto\ 
\dicFigure{1319.jpg}{Ólífutré}{Ólífutré - Zicha Ondřej, Biolib, Copyright/CC-BY-NC}
\dicEntry[ólíkindalæti] \dicTerm{ó·líkinda··læti} \dicIPA{{ou}{\textlengthmark}{l}{i}{\r{\textObardotlessj}}{\textsci}{n}{\textsubring{d}}{a}{l}{a}{i}{\textsubring{d}}{\textsci}} \dicPos{n}[2] \dicFlx{pl}[19] \dicSynonym{látalæti} \dicDirectTranslationCS{přetvařování, předstírání};  \dicPhraseIS{vera með ólíkindalæti} \dicSynonym*{látast} \dicDirectTranslationCS{předstírat, přetvařovat se}
\dicEntry[ólíkindatól] \dicTerm{ó·líkinda··tól} \dicIPA{{ou}{\textlengthmark}{l}{i}{\r{\textObardotlessj}}{\textsci}{n}{\textsubring{d}}{a}{t\smash{\textsuperscript{h}}}{ou}{\textlengthmark}{\textsubring{l}}} \dicPos{n}[2] \dicFlx{(‑s, ‑)}[5] \dicDirectTranslationCS{pokrytec, pokrytkyně}
\dicEntry[ólíkindi] \dicTerm{ó··lík·indi} \dicIPA{{ou}{\textlengthmark}{l}{i}{\r{\textObardotlessj}}{\textsci}{n}{\textsubring{d}}{\textsci}} \dicPos{n}[2] \dicFlx{pl}[19] \textbf{1.} \dicSynonym{látalæti} \dicDirectTranslationCS{přetvářka}  \textbf{2.} \dicDirectTranslationCS{nepravděpodobnost};  \dicPhraseIS{e‑að er með ólíkindum} \dicDirectTranslationCS{(co) je nepravděpodobné} \dicExampleIS{Það er með ólíkindum að þetta skuli ganga svo hægt.} \dicExampleCS{Je nepravděpodobné, že by to šlo tak pomalu.}  \textbf{3.} \dicSynonym{ósanngirni} \dicDirectTranslationCS{neférovost, nečestnost}
\dicEntry[ólíklegur] \dicTerm{ó··lík·legur} \dicIPA{{ou}{\textlengthmark}{l}{i}{\r{g}}{l}{\textepsilon}{\textbabygamma}{\textscy}{\textsubring{r}}} \dicPos{adj}[1]\dicFlx{}[-8] \dicSynonym{ósennilegur} \dicDirectTranslationCS{nepravděpodobný} \dicExampleIS{ólíkleg kenning} \dicExampleCS{nepravděpodobná teorie} \dicAntonym{líklegur}
\dicEntry[ólíkur] \dicTerm{ó··líkur} \dicsymFrequent\  \dicIPA{{ou}{\textlengthmark}{l}{i}{\r{g}}{\textscy}{\textsubring{r}}} \dicPos{adj}[1]\dicFlx{}[-1] \dicSynonym{mismunandi\smash{\textsuperscript{1}}} \dicDirectTranslationCS{rozdílný, odlišný} \dicExampleIS{Þær eru ólíkar í útliti.} \dicExampleCS{Liší se vzhledem.} \dicAntonym{líkur\smash{\textsuperscript{2}}}
\dicEntry[ólíva] \dicTerm{ólív|a} \dicIPA{{ou}{\textlengthmark}{l}{i}{v}{a}} \dicPos{f}[1] \dicFlx{(‑u, ‑ur)}[7] \dicLink{ólífa}
\dicEntry[óljós] \dicTerm{ó··ljós} \dicsymFrequent\  \dicIPA{{ou}{\textlengthmark}{l}{j}{ou}{s}} \dicPos{adj}[5]\dicFlx{}[-1] \dicSynonym{óskýr} \dicDirectTranslationCS{nejasný, mlhavý, vágní} \dicExampleIS{óljós merking} \dicExampleCS{nejasný význam} \dicAntonym{ljós\smash{\textsuperscript{2}}}
\dicEntry[ólmur] \dicTerm{ólmur} \dicIPA{{ou}{l}{m}{\textscy}{\textsubring{r}}} \dicPos{adj}[1]\dicFlx{}[-1] \textbf{1.} \dicSynonym{ákafur} \dicDirectTranslationCS{ohnivý, zapálený, zanícený} \dicExampleIS{ólmur í e‑ð} \dicExampleCS{zapálený do (čeho)}  \textbf{2.} \dicSynonym{hvass} \dicDirectTranslationCS{běsnící, prudký (vítr ap.)}
\dicEntry[ólokinn] \dicTerm{ó··lokinn} \dicIPA{{ou}{\textlengthmark}{l}{\textopeno}{\r{\textObardotlessj}}{\textsci}{\textsubring{n}}} \dicPos{adj}[6]\dicFlx{}[-6] \textbf{1.} \dicSynonym*{ekki búinn} \dicDirectTranslationCS{nedokončený, neuzavřený}  \textbf{2.} \dicSynonym*{óborgaður} \dicDirectTranslationCS{nesplacený, neuhrazený} \dicExampleIS{ólokin skuld} \dicExampleCS{nesplacený dluh}
\dicEntry[ólum] \dicTerm{ólum} \dicIPA{{ou}{\textlengthmark}{l}{\textscy}{\textsubring{m}}} \dicPos{v} \dicFlx{ind pf pl 1 pers} \dicLink{ala}
\dicEntry[ólund] \dicTerm{ó··lund} \dicIPA{{ou}{\textlengthmark}{l}{\textscy}{n}{\textsubring{d}}} \dicPos{f}[7] \dicFlx{(‑ar)}[3] \dicSynonym{fýla} \dicDirectTranslationCS{nedůtklivost, rozmrzelost, špatná nálada} \dicExampleIS{Það er ólund í honum.} \dicExampleCS{Je nedůtklivý.}
\dicEntry[ólundarlegur] \dicTerm{ó··lundar·legur}\dicTerm{, ólundlegur} \dicIPA{{ou}\-{\textlengthmark}\-{l}\-{\textscy}\-{n}\-{\textsubring{d}}\-{a}\-{r}\-{l}\-{\textepsilon}\-{\textbabygamma}\-{\textscy}\-{\textsubring{r}}\-} \dicPos{adj}[1]\dicFlx{}[-8] \dicSynonym{fýlulegur} \dicDirectTranslationCS{podrážděný, rozmrzelý, (jsoucí) ve špatné náladě}
\dicEntry[ólundlegur] \dicTerm{ó··lund·legur} \dicIPA{{ou}{\textlengthmark}{l}{\textscy}{n}{\textsubring{d}}{l}{\textepsilon}{\textbabygamma}{\textscy}{\textsubring{r}}} \dicPos{adj}[1]\dicFlx{}[-8] \dicLink{ólundarlegur}
\dicEntry[ólyfjan] \dicTerm{ó··lyfjan} \dicIPA{{ou}{\textlengthmark}{l}{\textsci}{v}{j}{a}{\textsubring{n}}} \dicPos{f}[4] \dicFlx{(‑ar)}[3] \dicSynonym{eitur} \dicDirectTranslationCS{jed}
\dicEntry[ólyginn] \dicTerm{ó··lyginn} \dicIPA{{ou}{\textlengthmark}{l}{i}{j}{\textsci}{\textsubring{n}}} \dicPos{adj}[6]\dicFlx{}[-2] \dicSynonym{sannsögull} \dicDirectTranslationCS{pravdomluvný} \dicIndirectTranslationCS{(často ironicky)}
\dicEntry[ólykt] \dicTerm{ó··lykt} \dicIPA{{ou}{\textlengthmark}{l}{\textsci}{x}{\textsubring{d}}} \dicPos{f}[7] \dicFlx{(‑ar)}[3] \dicSynonym{óþefur} \dicDirectTranslationCS{(odporný) puch, smrad}
\dicEntry[ólympískur] \dicTerm{ólympískur} \dicIPA{{ou}{\textlengthmark}{l}{\textsci}{\textsubring{m}}{\textsubring{b}}{i}{s}{\r{g}}{\textscy}{\textsubring{r}}} \dicPos{adj}[1]\dicFlx{}[-6] \dicDirectTranslationCS{olympijský}
\dicEntry[Ólympíuleikar] \dicTerm{Ólympíu··leikar} \dicIPA{{ou}{\textlengthmark}{l}{\textsci}{\textsubring{m}}{\textsubring{b}}{i}{j}{\textscy}{l}{ei}{\r{g}}{a}{\textsubring{r}}} \dicPos{m}[6] \dicFlx{pl}[2] \dicFieldCat{sport.} \dicDirectTranslationCS{olympijské hry}
\dicEntry[ólympíumet] \dicTerm{ólympíu··met} \dicIPA{{ou}{\textlengthmark}{l}{\textsci}{\textsubring{m}}{\textsubring{b}}{i}{j}{\textscy}{m}{\textepsilon}{\textsubring{d}}} \dicPos{n}[2] \dicFlx{(‑s, ‑)}[5] \dicFieldCat{sport.} \dicDirectTranslationCS{olympijský rekord}
\dicEntry[ólyst] \dicTerm{ó··lyst} \dicIPA{{ou}{\textlengthmark}{l}{\textsci}{s}{\textsubring{d}}} \dicPos{f}[7] \dicFlx{(‑ar)}[3] \textbf{1.} \dicSynonym{lystarleysi} \dicDirectTranslationCS{nechuť, nechutenství}  \textbf{2.} \dicSynonym{leiðindi} \dicDirectTranslationCS{nuda, dlouhá chvíle}
\dicEntry[ólystugur] \dicTerm{ó··lyst·ugur} \dicIPA{{ou}{\textlengthmark}{l}{\textsci}{s}{\textsubring{d}}{\textscy}{\textbabygamma}{\textscy}{\textsubring{r}}} \dicPos{adj}[1]\dicFlx{}[-8] \textbf{1.} \dicDirectTranslationCS{nechutný, nevábný (jídlo ap.)} \dicExampleIS{ólystugur matur} \dicExampleCS{nechutné jídlo}  \textbf{2.} \dicDirectTranslationCS{nemající apetit, (jsoucí) bez apetitu}
\dicEntry[ólýsanlegur] \dicTerm{ó··lýsan·legur} \dicIPA{{ou}{\textlengthmark}{l}{i}{s}{a}{n}{l}{\textepsilon}{\textbabygamma}{\textscy}{\textsubring{r}}} \dicPos{adj}[1]\dicFlx{}[-8] \dicSynonym{óumræðilegur} \dicDirectTranslationCS{nepopsatelný, nevylíčitelný} \dicExampleIS{ólýsanleg tilfinning} \dicExampleCS{nepopsatelný pocit}
\dicEntry[ólæknandi] \dicTerm{ó··lækn·andi} \dicIPA{{ou}{\textlengthmark}{l}{a}{i}{h}{\r{g}}{n}{a}{n}{\textsubring{d}}{\textsci}} \dicPos{adj}[13] \dicFlx{indecl}[1] \dicSynonym{banvænn} \dicDirectTranslationCS{nevyléčitelný, neléčitelný} \dicExampleIS{ólæknandi sjúkdómur} \dicExampleCS{nevyléčitelné onemocnění}
\dicEntry[ólærður] \dicTerm{ó··lærður} \dicIPA{{ou}{\textlengthmark}{l}{a}{i}{r}{ð}{\textscy}{\textsubring{r}}} \dicPos{adj}[2]\dicFlx{}[-1] \dicSynonym{fáfróður} \dicDirectTranslationCS{nevzdělaný, (jsoucí) bez vzdělání} \dicAntonym{lærður};  \dicPhraseIS{eiga e‑ð eftir ólært} \dicDirectTranslationCS{muset se (co) ještě naučit}
\dicEntry[ólæs] \dicTerm{ó··læs} \dicIPA{{ou}{\textlengthmark}{l}{a}{i}{s}} \dicPos{adj}[5]\dicFlx{}[-1] \dicDirectTranslationCS{negramotný, neumějící číst} \dicAntonym{læs}
\dicEntry[ólæsi] \dicTerm{ó··læsi} \dicIPA{{ou}{\textlengthmark}{l}{a}{i}{s}{\textsci}} \dicPos{n}[2] \dicFlx{(‑s)}[20] \dicDirectTranslationCS{negramotnost, analfabetismus}
\dicEntry[ólæsilegur] \dicTerm{ó··læsi·legur} \dicIPA{{ou}{\textlengthmark}{l}{a}{i}{s}{\textsci}{l}{\textepsilon}{\textbabygamma}{\textscy}{\textsubring{r}}} \dicPos{adj}[1]\dicFlx{}[-8] \dicDirectTranslationCS{nečitelný (rukopis ap.)} \dicAntonym{læsilegur}
\dicEntry[ólæti] \dicTerm{ó··læti} \dicIPA{{ou}{\textlengthmark}{l}{a}{i}{\textsubring{d}}{\textsci}} \dicPos{n}[2] \dicFlx{pl}[19] \dicSynonym{hávaði} \dicDirectTranslationCS{rámus, kravál, randál} \dicExampleIS{Það eru ólæti í krökkunum.} \dicExampleCS{Děti dělají rámus.}
\dicEntry[ólöglegur] \dicTerm{ó··lög·legur} \dicIPA{{ou}{\textlengthmark}{l}{\oe}{\textbabygamma}{l}{\textepsilon}{\textbabygamma}{\textscy}{\textsubring{r}}} \dicPos{adj}[1]\dicFlx{}[-8] \dicSynonym{réttlaus} \dicDirectTranslationCS{nezákonný, protiprávní, ilegální, nelegální} \dicAntonym{löglegur}
\dicEntry[ólögráða] \dicTerm{ó··lög·ráða} \dicIPA{{ou}{\textlengthmark}{l}{\oe}{\textbabygamma}{r}{au}{ð}{a}} \dicPos{adj}[13] \dicFlx{indecl}[1] \dicFieldCat{práv.} \dicDirectTranslationCS{nesvéprávný}
\dicEntry[ólögulegur] \dicTerm{ó··lögu·legur} \dicIPA{{ou}{\textlengthmark}{l}{\oe}{\textbabygamma}{\textscy}{l}{\textepsilon}{\textbabygamma}{\textscy}{\textsubring{r}}} \dicPos{adj}[1]\dicFlx{}[-8] \dicSynonym*{ólánlegur} \dicDirectTranslationCS{beztvarý, neforemný}
\dicEntry[óm] \dicTerm{óm} \dicIPA{{ou}{\textlengthmark}{\textsubring{m}}} \dicPos{n}[2] \dicFlx{(‑s, ‑)}[5] \dicFieldCat{fyz.} \dicDirectTranslationCS{ohm}
\dicEntry[óma] \dicTerm{óm|a} \dicIPA{{ou}{\textlengthmark}{m}{a}} \dicPos{v}[1] \dicFlx{(‑aði)}[44] \dicSynonym{hljóma} \dicDirectTranslationCS{znít, zvučet} \dicExampleIS{Skógurinn ómar af fuglasöng.} \dicExampleCS{V~lese zní ptačí zpěv.};  \dicIdiom{óma}[í\,/\addthin á]{ \dicPhraseIS{óma í\,/\addthin á e‑n}} \dicDirectTranslationCS{zavolat na (koho)}
\dicEntry[ómagi] \dicTerm{ó··mag|i} \dicIPA{{ou}{\textlengthmark}{m}{a}{i}{j}{\textsci}} \dicPos{m}[1] \dicFlx{(‑a, ‑ar)}[8] \dicIndirectTranslationCS{osoba, která je závislá na pomoci druhých (dítě, staří lidé ap.)}
\dicEntry[ómak] \dicTerm{ó··mak} \dicIPA{{ou}{\textlengthmark}{m}{a}{\r{g}}} \dicPos{n}[2] \dicFlx{(‑s)}[2] \dicSynonym{fyrirhöfn} \dicDirectTranslationCS{námaha, obtíž} \dicExampleIS{spara e‑m ómakið með e‑u} \dicExampleCS{ušetřit (komu) obtíže (čím)};  \dicPhraseIS{e‑að er ómaksins vert} \dicLangCat{přen.} \dicDirectTranslationCS{(co) stojí za obtíže}
\dicEntry[ómaka] \dicTerm{ó··mak|a} \dicIPA{{ou}{\textlengthmark}{m}{a}{\r{g}}{a}} \dicPos{v}[1] \dicFlx{(‑aði)}[13] \dicFlx{acc} \dicPhraseIS{ómaka sig á e‑u} \dicSynonym{ónáða} \dicDirectTranslationCS{obtěžovat se s~(čím)}
\dicEntry[ómaklegur] \dicTerm{ó··mak·legur} \dicIPA{{ou}{\textlengthmark}{m}{a}{\r{g}}{l}{\textepsilon}{\textbabygamma}{\textscy}{\textsubring{r}}} \dicPos{adj}[1]\dicFlx{}[-8] \dicSynonym{óverðskuldaður} \dicDirectTranslationCS{nezasloužený} \dicExampleIS{ómakleg hegning} \dicExampleCS{nezasloužený trest} \dicAntonym{maklegur}
\dicEntry[Óman] \dicTerm{Óman} \dicIPA{{ou}{\textlengthmark}{m}{a}{\textsubring{n}}} \dicPos{n}[2] \dicFlx{(‑\,/\addthin ‑s)}[35] \dicFieldCat{geog.} \dicDirectTranslationCS{Omán}
\dicEntry[Ómani] \dicTerm{Óman|i} \dicIPA{{ou}{\textlengthmark}{m}{a}{n}{\textsci}} \dicPos{m}[1] \dicFlx{(‑a, ‑ar)}[8] \dicDirectTranslationCS{Ománec, Ománka}
\dicEntry[ómannblendinn] \dicTerm{ó··mann·blendinn} \dicIPA{{ou}{\textlengthmark}{m}{a}{n}{\textsubring{b}}{l}{\textepsilon}{n}{\textsubring{d}}{\textsci}{\textsubring{n}}} \dicPos{adj}[6]\dicFlx{}[-2] \dicSynonym*{einlyndur} \dicDirectTranslationCS{nespolečenský, nedružný} \dicAntonym{mannblendinn}
\dicEntry[ómannúðlegur] \dicTerm{ó··mannúð·legur} \dicIPA{{ou}{\textlengthmark}{m}{a}{n}{u}{ð}{l}{\textepsilon}{\textbabygamma}{\textscy}{\textsubring{r}}} \dicPos{adj}[1]\dicFlx{}[-8] \dicSynonym*{ómildur} \dicDirectTranslationCS{nelidský, nehumánní} \dicAntonym{mannúðlegur}
\dicEntry[ómanskur] \dicTerm{ómanskur} \dicIPA{{ou}{\textlengthmark}{m}{a}{n}{s}{\r{g}}{\textscy}{\textsubring{r}}} \dicPos{adj}[1] \dicFlx{(f ómönsk)}[3] \dicDirectTranslationCS{ománský}
\dicEntry[ómeðfærilegur] \dicTerm{ó··með·færi·legur} \dicIPA{{ou}{\textlengthmark}{m}{\textepsilon}{ð}{f}{a}{i}{r}{\textsci}{l}{\textepsilon}{\textbabygamma}{\textscy}{\textsubring{r}}} \dicPos{adj}[1]\dicFlx{}[-8] \dicSynonym{baldinn} \dicDirectTranslationCS{nezvladatelný, neposlušný (dítě ap.)} \dicAntonym{meðfærilegur}
\dicEntry[ómeðvitaður] \dicTerm{ó··með·vit·|aður} \dicIPA{{ou}{\textlengthmark}{m}{\textepsilon}{ð}{v}{\textsci}{\textsubring{d}}{a}{ð}{\textscy}{\textsubring{r}}} \dicPos{adj}[3] \dicFlx{(f ‑uð)}[3] \dicDirectTranslationCS{neúmyslný, nevědomý}
\dicEntry[ómegð] \dicTerm{ó··megð} \dicIPA{{ou}{\textlengthmark}{m}{\textepsilon}{\textbabygamma}{\texttheta}} \dicPos{f}[4] \dicFlx{(‑ar)}[3] \textbf{1.} \dicSynonym*{ómagaaldur} \dicDirectTranslationCS{nesamostatnost, závislost} \dicExampleIS{börn í ómegð} \dicExampleCS{děti, které se o~sebe nejsou schopné postarat}  \textbf{2.} \dicSynonym*{barnahópur} \dicDirectTranslationCS{děti, hlouček dětí}
\dicEntry[ómegin] \dicTerm{ó··megin} \dicIPA{{ou}{\textlengthmark}{m}{ei}{\textsci}{\textsubring{n}}} \dicPos{n}[2] \dicFlx{(‑s)}[2] \dicSynonym{magnleysi} \dicDirectTranslationCS{mdloba};  \dicPhraseIS{líða\,/\addthin falla í ómegin} \dicDirectTranslationCS{omdlít, upadnout do mdlob}
\dicEntry[ómeiddur] \dicTerm{ó··meiddur} \dicIPA{{ou}{\textlengthmark}{m}{ei}{\textsubring{d}}{\textscy}{\textsubring{r}}} \dicPos{adj}[2]\dicFlx{}[-18] \dicSynonym{óskaddaður} \dicDirectTranslationCS{nezraněný, neporaněný} \dicAntonym{meiddur};  \dicPhraseIS{sleppa ómeiddur} \dicDirectTranslationCS{vyváznout bez zranění}
\dicEntry[ómeltur] \dicTerm{ó··meltur} \dicIPA{{ou}{\textlengthmark}{m}{\textepsilon}{\textsubring{l}}{\textsubring{d}}{\textscy}{\textsubring{r}}} \dicPos{adj}[1]\dicFlx{}[-13] \dicDirectTranslationCS{nestrávený (jídlo ap.)}
\dicEntry[ómengaður] \dicTerm{ó··meng·|aður} \dicIPA{{ou}{\textlengthmark}{m}{ei}{\ng}{\r{g}}{a}{ð}{\textscy}{\textsubring{r}}} \dicPos{adj}[3] \dicFlx{(f ‑uð)}[3] \textbf{1.} \dicDirectTranslationCS{neznečištěný, nezamořený (ovzduší ap.)}  \textbf{2.} \dicSynonym{óskemmdur} \dicDirectTranslationCS{čistý, ryzí} \dicExampleIS{ómengaður sannleikur} \dicExampleCS{ryzí pravda}
\dicEntry[ómenni] \dicTerm{ó··menni} \dicIPA{{ou}{\textlengthmark}{m}{\textepsilon}{n}{\textsci}} \dicPos{n}[2] \dicFlx{(‑s, ‑)}[14] \dicDirectTranslationCS{ničema, lotr(yně), bídák, bídačka}
\dicEntry[ómenning] \dicTerm{ó··menn·ing} \dicIPA{{ou}{\textlengthmark}{m}{\textepsilon}{n}{i}{\ng}{\r{g}}} \dicPos{f}[4] \dicFlx{(‑ar)}[7] \dicDirectTranslationCS{barbarství, dekadence} \dicAntonym{menning}
\dicEntry[ómennska] \dicTerm{ó··mennsk|a} \dicIPA{{ou}{\textlengthmark}{m}{\textepsilon}{n}{s}{\r{g}}{a}} \dicPos{f}[1] \dicFlx{(‑u)}[5] \textbf{1.} \dicSynonym*{ræfildómur} \dicDirectTranslationCS{nelidskost, ničemnost}  \textbf{2.} \dicSynonym{dugleysi} \dicDirectTranslationCS{nedostatek vůle, lenost, neschopnost}
\dicEntry[ómenntaður] \dicTerm{ó··mennt·|aður} \dicIPA{{ou}{\textlengthmark}{m}{\textepsilon}{\textsubring{n}}{\textsubring{d}}{a}{ð}{\textscy}{\textsubring{r}}} \dicPos{adj}[3] \dicFlx{(f ‑uð)}[3] \dicSynonym{óupplýstur} \dicDirectTranslationCS{nevzdělaný, nekultivovaný} \dicAntonym{menntaður}
\dicEntry[ómerkilegur] \dicTerm{ó··merki·legur} \dicIPA{{ou}{\textlengthmark}{m}{\textepsilon}{\textsubring{r}}{\r{\textObardotlessj}}{\textsci}{l}{\textepsilon}{\textbabygamma}{\textscy}{\textsubring{r}}} \dicPos{adj}[1]\dicFlx{}[-8] \textbf{1.} \dicSynonym{lélegur} \dicDirectTranslationCS{nekvalitní, podřadný (výrobek ap.)}  \textbf{2.} \dicSynonym{smásálarlegur} \dicDirectTranslationCS{bezvýznamný, druhořadý, banální} \dicAntonym{merkilegur}  \textbf{3.} \dicSynonym{óáreiðanlegur} \dicDirectTranslationCS{nespolehlivý}
\dicEntry[ómerkja] \dicTerm{ó··merk|ja} \dicIPA{{ou}{\textlengthmark}{m}{\textepsilon}{\textsubring{r}}{\r{\textObardotlessj}}{a}} \dicPos{v}[2] \dicFlx{(‑ti, ‑t)}[25] \dicFlx{acc} \dicDirectTranslationCS{zrušit platnost, zneplatnit, anulovat}
\dicEntry[ómerktur] \dicTerm{ó··merktur} \dicIPA{{ou}{\textlengthmark}{m}{\textepsilon}{\textsubring{r}}{\textsubring{d}}{\textscy}{\textsubring{r}}} \dicPos{adj}[1]\dicFlx{}[-13] \textbf{1.} \dicDirectTranslationCS{neoznačený, nevyznačený} \dicExampleIS{ómerktur lögreglubíll} \dicExampleCS{neoznačené policejní auto}  \textbf{2.} \dicDirectTranslationCS{neplatný, anulovaný} \dicExampleIS{ómerktur dómur} \dicExampleCS{neplatný rozsudek}
\dicEntry[ómetanlegur] \dicTerm{ó··metan·legur} \dicIPA{{ou}{\textlengthmark}{m}{\textepsilon}{\textsubring{d}}{a}{n}{l}{\textepsilon}{\textbabygamma}{\textscy}{\textsubring{r}}} \dicPos{adj}[1]\dicFlx{}[-8] \dicSynonym{dýr\smash{\textsuperscript{2}}} \dicDirectTranslationCS{neocenitelný, nedocenitelný (rada ap.)}
\dicEntry[ómettaður] \dicTerm{ó··mett·|aður} \dicIPA{{ou}{\textlengthmark}{m}{\textepsilon}{h}{\textsubring{d}}{a}{ð}{\textscy}{\textsubring{r}}} \dicPos{adj}[3] \dicFlx{(f ‑uð)}[3] \dicDirectTranslationCS{nenasycený};  \dicPhraseIS{ómettuð fita} \dicFieldCat{med.} \dicDirectTranslationCS{nenasycený tuk}
\dicEntry[óminni] \dicTerm{ó··minni} \dicIPA{{ou}{\textlengthmark}{m}{\textsci}{n}{\textsci}} \dicPos{n}[2] \dicFlx{(‑s, ‑)}[14] \dicSynonym{gleymska} \dicDirectTranslationCS{výpadek\,/\addthin ztráta paměti, zapomnětlivost};  \dicProverb\  \dicPhraseIS{Oft gjalda fætur óminnis.} \dicLangCat{přís.} \dicDirectTranslationCS{Co není v~hlavě, musí být v~nohách.}
\dicEntry[óminnugur] \dicTerm{ó··minn·ugur} \dicIPA{{ou}{\textlengthmark}{m}{\textsci}{n}{\textscy}{\textbabygamma}{\textscy}{\textsubring{r}}} \dicPos{adj}[1]\dicFlx{}[-8] \dicDirectTranslationCS{zapomnětlivý}
\dicEntry[ómissandi] \dicTerm{ó··miss·andi} \dicIPA{{ou}{\textlengthmark}{m}{\textsci}{s}{a}{n}{\textsubring{d}}{\textsci}} \dicPos{adj}[13] \dicFlx{indecl}[1] \dicSynonym{nauðsynlegur} \dicDirectTranslationCS{nepostradatelný, nezbytný}
\dicEntry[ómótaður] \dicTerm{ó··mót·|aður} \dicIPA{{ou}{\textlengthmark}{m}{ou}{\textsubring{d}}{a}{ð}{\textscy}{\textsubring{r}}} \dicPos{adj}[3] \dicFlx{(f ‑uð)}[3] \dicDirectTranslationCS{nezralý, nezformovaný, nerozvinutý} \dicExampleIS{ómótaður unglingur} \dicExampleCS{nezralý mladík} \dicAntonym{mótaður}
\dicEntry[ómótmælanlegur] \dicTerm{ó··mót·mælan·legur} \dicIPA{{ou}{\textlengthmark}{m}{ou}{\textsubring{d}}{m}{a}{i}{l}{a}{n}{l}{\textepsilon}{\textbabygamma}{\textscy}{\textsubring{r}}} \dicPos{adj}[1]\dicFlx{}[-8] \dicSynonym{vafalaus} \dicDirectTranslationCS{nepopiratelný, neoddiskutovatelný, nesporný}
\dicEntry[ómótstæðilegur] \dicTerm{ó··mót·stæði·legur} \dicIPA{{ou}{\textlengthmark}{m}{ou}{\textsubring{d}}{s}{\textsubring{d}}{a}{i}{ð}{\textsci}{l}{\textepsilon}{\textbabygamma}{\textscy}{\textsubring{r}}} \dicPos{adj}[1]\dicFlx{}[-8] \dicSynonym{óhjákvæmilegur} \dicDirectTranslationCS{neodolatelný, silný} \dicExampleIS{ómótstæðileg fegurð} \dicExampleCS{neodolatelná krása}
\dicEntry[ómóttækilegur] \dicTerm{ó··mót·tæki·legur} \dicIPA{{ou}{\textlengthmark}{m}{ou}{\textsubring{d}}{t\smash{\textsuperscript{h}}}{a}{i}{\r{\textObardotlessj}}{\textsci}{l}{\textepsilon}{\textbabygamma}{\textscy}{\textsubring{r}}} \dicPos{adj}[1]\dicFlx{}[-8] \dicSynonym{ónæmur} \dicDirectTranslationCS{nevnímavý, nenáchylný} \dicExampleIS{ómóttækilegur fyrir sýkingu} \dicExampleCS{nenáchylný k~nákaze} \dicAntonym{móttækilegur}
\dicEntry[ómsjá] \dicTerm{óm··sjá} \dicIPA{{ou}{\textlengthmark}{m}{s}{j}{au}} \dicPos{f}[4] \dicFlx{(‑r, ‑r)}[18] \dicFieldCat{med. } \dicDirectTranslationCS{sonograf}
\dicEntry[ómskoðun] \dicTerm{óm··skoð|un} \dicIPA{{ou}{\textlengthmark}{m}{s}{\r{g}}{\textopeno}{ð}{\textscy}{\textsubring{n}}} \dicPos{f}[7] \dicFlx{(‑unar)}[9] \dicFieldCat{med.} \dicDirectTranslationCS{ultrazvukové\,/\addthin sonografické vyšetření, sonografie, ultrazvuk}
\dicEntry[ómunatíð] \dicTerm{ó·muna··tíð} \dicIPA{{ou}{\textlengthmark}{m}{\textscy}{n}{a}{t\smash{\textsuperscript{h}}}{i}{\texttheta}} \dicPos{f}[7] \dicFlx{(‑ar)}[3] \dicPhraseIS{frá ómunatíð} \dicFlx{adv} \dicDirectTranslationCS{od nepaměti}
\dicEntry[ómur] \dicTerm{óm|ur} \dicsymFrequent\  \dicIPA{{ou}{\textlengthmark}{m}{\textscy}{\textsubring{r}}} \dicPos{m}[6] \dicFlx{(‑s, ‑ar)}[24] \dicSynonym{hljómur} \dicDirectTranslationCS{(vzdálený) zvuk, znění} \dicExampleIS{ómur af tali} \dicExampleCS{vzdálený zvuk hlasů}
\dicEntry[ómynd] \dicTerm{ó··mynd} \dicIPA{{ou}{\textlengthmark}{m}{\textsci}{n}{\textsubring{d}}} \dicPos{f}[7] \dicFlx{(‑ar, ‑ir)}[1] \textbf{1.} \dicSynonym*{hrákasmíði} \dicDirectTranslationCS{levota, fušeřina}  \textbf{2.} \dicSynonym{smán} \dicDirectTranslationCS{ostuda, hanba}
\dicEntry[ómyrkur] \dicTerm{ó··myrkur} \dicIPA{{ou}{\textlengthmark}{m}{\textsci}{\textsubring{r}}{\r{g}}{\textscy}{\textsubring{r}}} \dicPos{adj}[1]\dicFlx{}[-1] \dicSynonym{skýr} \dicDirectTranslationCS{jasný, zřetelný};  \dicPhraseIS{vera ómyrkur í máli} \dicDirectTranslationCS{mluvit jasně a~zřetelně, mluvit přímo k~věci}
\dicEntry[ómögulegur] \dicTerm{ó··mögu·legur} \dicsymFrequent\  \dicIPA{{ou}{\textlengthmark}{m}{\oe}{\textbabygamma}{\textscy}{l}{\textepsilon}{\textbabygamma}{\textscy}{\textsubring{r}}} \dicPos{adj}[1]\dicFlx{}[-8] \textbf{1.} \dicSynonym{óvinnandi} \dicDirectTranslationCS{nemožný, neproveditelný} \dicExampleIS{Það er ómögulegt að hann sé enn á lífi.} \dicExampleCS{Není možné, aby byl ještě naživu.} \dicAntonym{mögulegur}  \textbf{2.} \dicSynonym*{stórgallaður} \dicDirectTranslationCS{nemožný, nesnesitelný (člověk ap.)}  \textbf{3.} \dicSynonym{ónothæfur} \dicDirectTranslationCS{nepoužitelný}
\dicEntry[ómöguleiki] \dicTerm{ó··mögu·leik|i} \dicIPA{{ou}{\textlengthmark}{m}{\oe}{\textbabygamma}{\textscy}{l}{ei}{\r{\textObardotlessj}}{\textsci}} \dicPos{m}[1] \dicFlx{(‑a, ‑ar)}[1] \dicSynonym{ógerningur} \dicDirectTranslationCS{nemožnost, neproveditelnost} \dicAntonym{möguleiki}
\dicEntry[ónafngreiddur] \dicTerm{ó··nafn·greiddur} \dicIPA{{ou}{\textlengthmark}{n}{a}{\textsubring{b}}{\textsubring{n}}{\r{g}}{r}{ei}{\textsubring{d}}{\textscy}{\textsubring{r}}} \dicPos{adj}[2]\dicFlx{}[-21] \dicDirectTranslationCS{bezejmenný, anonymní, neznámý}
\dicEntry[ónauðsynlegur] \dicTerm{ó··nauð·syn·legur} \dicIPA{{ou}{\textlengthmark}{n}{\oe i}{ð}{s}{\textsci}{n}{l}{\textepsilon}{\textbabygamma}{\textscy}{\textsubring{r}}} \dicPos{adj}[1]\dicFlx{}[-8] \dicSynonym{óþarfur} \dicDirectTranslationCS{zbytečný, nadbytečný} \dicAntonym{nauðsynlegur}
\dicEntry[ónáð] \dicTerm{ó··náð} \dicIPA{{ou}{\textlengthmark}{n}{au}{\texttheta}} \dicPos{f}[7] \dicFlx{(‑ar)}[3] \dicSynonym{vanþóknun} \dicDirectTranslationCS{nemilost, nepřízeň};  \dicPhraseIS{falla í ónáð hjá e‑m} \dicSynonym{óvild} \dicDirectTranslationCS{upadnout v~nemilost u~(koho)}
\dicEntry[ónáða] \dicTerm{ó··náð|a} \dicIPA{{ou}{\textlengthmark}{n}{au}{ð}{a}} \dicPos{v}[1] \dicFlx{(‑aði)}[1] \dicFlx{acc} \dicSynonym{erta\smash{\textsuperscript{2}}} \dicDirectTranslationCS{obtěžovat, zlobit} \dicExampleIS{ónáða e‑n með sífelldu kvabbi} \dicExampleCS{obtěžovat (koho) neustálými dotěrnostmi}
\dicEntry[ónákvæmni] \dicTerm{ó··ná·kvæmn|i} \dicIPA{{ou}{\textlengthmark}{n}{au}{k\smash{\textsuperscript{h}}}{v}{a}{i}{m}{n}{\textsci}} \dicPos{f}[3] \dicFlx{(‑i)}[3] \dicSynonym{óáreiðanleiki} \dicDirectTranslationCS{nepřesnost} \dicAntonym{nákvæmni}
\dicEntry[ónákvæmur] \dicTerm{ó··ná·kvæmur} \dicIPA{{ou}{\textlengthmark}{n}{au}{k\smash{\textsuperscript{h}}}{v}{a}{i}{m}{\textscy}{\textsubring{r}}} \dicPos{adj}[1]\dicFlx{}[-1] \dicSynonym{óáreiðanlegur} \dicDirectTranslationCS{nepřesný} \dicExampleIS{ónákvæmar tölur} \dicExampleCS{nepřesná čísla} \dicAntonym{nákvæmur}
\dicEntry[ónáttúra] \dicTerm{ó··náttúr|a} \dicIPA{{ou}{\textlengthmark}{n}{au}{h}{\textsubring{d}}{u}{r}{a}} \dicPos{f}[1] \dicFlx{(‑u)}[5] \textbf{1.} \dicSynonym{óeðli} \dicDirectTranslationCS{nepřirozenost}  \textbf{2.} \dicDirectTranslationCS{zvrácenost, perverze}
\dicEntry[ónefndur] \dicTerm{ó··nefndur} \dicIPA{{ou}{\textlengthmark}{n}{\textepsilon}{m}{\textsubring{d}}{\textscy}{\textsubring{r}}} \dicPos{adj}[2]\dicFlx{}[-17] \textbf{1.} \dicSynonym{nafnlaus} \dicDirectTranslationCS{anonymní, bezejmenný}  \textbf{2.} \dicSynonym*{óupptalinn} \dicDirectTranslationCS{nejmenovaný}
\dicEntry[óneitanlega] \dicTerm{ó··neitan·lega} \dicsymFrequent\  \dicIPA{{ou}{\textlengthmark}{n}{ei}{\textsubring{d}}{a}{n}{l}{\textepsilon}{\textbabygamma}{a}} \dicPos{adv} \dicSynonym{vissulega} \dicDirectTranslationCS{nesporně, nepopiratelně} \dicExampleIS{Þetta var óneitanlega skrýtin reynsla.} \dicExampleCS{Toto byla nesporně divná zkušenost.}
\dicEntry[ónot] \dicTerm{ó··not} \dicIPA{{ou}{\textlengthmark}{n}{\textopeno}{\textsubring{d}}} \dicPos{n}[2] \dicFlx{pl}[1] \textbf{1.} \dicSynonym*{hryssingsleg orð} \dicDirectTranslationCS{hrubost, hrubá řeč\,/\addthin poznámka};  \dicPhraseIS{svara ónotum} \dicDirectTranslationCS{hrubě odpovědět};  \dicPhraseIS{hreyta\,/\addthin kasta ónotum í e‑n} \dicDirectTranslationCS{zasypat (koho) hrubými slovy}  \textbf{2.} \dicSynonym{óþægindi} \dicDirectTranslationCS{nepříjemný pocit, diskomfort} \dicExampleIS{ónot í maganum} \dicExampleCS{nepříjemný pocit v~žaludku}
\dicEntry[ónotaður] \dicTerm{ó··not·|aður} \dicIPA{{ou}{\textlengthmark}{n}{\textopeno}{\textsubring{d}}{a}{ð}{\textscy}{\textsubring{r}}} \dicPos{adj}[3] \dicFlx{(f ‑uð)}[4] \dicSynonym{nýr} \dicDirectTranslationCS{nepoužívaný, nepoužitý} \dicAntonym{notaður}
\dicEntry[ónotalegur] \dicTerm{ó··nota·legur} \dicIPA{{ou}{\textlengthmark}{n}{\textopeno}{\textsubring{d}}{a}{l}{\textepsilon}{\textbabygamma}{\textscy}{\textsubring{r}}} \dicPos{adj}[1]\dicFlx{}[-8] \textbf{1.} \dicSynonym{óvingjarnlegur} \dicDirectTranslationCS{nesnášenlivý, svárlivý} \dicAntonym{notalegur}  \textbf{2.} \dicSynonym{óþægilegur} \dicDirectTranslationCS{nepříjemný, protivný};  \dicPhraseIS{það er ónotalegt að (gera e‑ð)} \dicDirectTranslationCS{je nepříjemné (dělat (co))}
\dicEntry[ónothæfur] \dicTerm{ó··not·hæfur} \dicIPA{{ou}{\textlengthmark}{n}{\textopeno}{\textsubring{d}}{h}{a}{i}{v}{\textscy}{\textsubring{r}}} \dicPos{adj}[1]\dicFlx{}[-1] \dicSynonym{gagnslaus} \dicDirectTranslationCS{nepoužitelný, nevyužitelný} \dicExampleIS{ónothæf þekking} \dicExampleCS{nepoužitelná znalost} \dicAntonym{nothæfur}
\dicEntry[ónógur] \dicTerm{ó··nógur} \dicIPA{{ou}{\textlengthmark}{n}{ou}{\textscy}{\textsubring{r}}} \dicPos{adj}[1]\dicFlx{}[-14] \dicSynonym{ófullnægjandi} \dicDirectTranslationCS{nedostatečný, nepostačující} \dicAntonym{nógur}
\dicEntry[ónytjungur] \dicTerm{ó··nytj·ung|ur} \dicIPA{{ou}{\textlengthmark}{n}{\textsci}{\textsubring{d}}{j}{u}{\ng}{\r{g}}{\textscy}{\textsubring{r}}} \dicPos{m}[6] \dicFlx{(‑s, ‑ar)}[8] \dicSynonym{slæpingi} \dicDirectTranslationCS{zaháleč, povaleč(ka), flink}
\dicEntry[ónýta] \dicTerm{ó··nýt|a} \dicIPA{{ou}{\textlengthmark}{n}{i}{\textsubring{d}}{a}} \dicPos{v}[2] \dicFlx{(‑ti, ‑t)}[54] \dicFlx{acc} \dicSynonym{ógilda} \dicDirectTranslationCS{zničit, znehodnotit} \dicExampleIS{ónýta frímerki} \dicExampleCS{znehodnotit známku}
\dicEntry[ónýting] \dicTerm{ó··nýt·ing} \dicIPA{{ou}{\textlengthmark}{n}{i}{\textsubring{d}}{i}{\ng}{\r{g}}} \dicPos{f}[4] \dicFlx{(‑ar)}[7] \dicSynonym*{það að ónýta} \dicDirectTranslationCS{zničení, znehodnocení}
\dicEntry[ónýtur] \dicTerm{ó··nýtur} \dicIPA{{ou}{\textlengthmark}{n}{i}{\textsubring{d}}{\textscy}{\textsubring{r}}} \dicPos{adj}[1]\dicFlx{}[-1] \textbf{1.} \dicSynonym{ónothæfur} \dicDirectTranslationCS{nepoužitelný, nefungující} \dicAntonym{nýtur}  \textbf{2.} \dicSynonym{duglaus} \dicDirectTranslationCS{neschopný, nepodnikavý (zaměstnanec ap.)} \dicExampleIS{ónýtur starfsmaður} \dicExampleCS{neschopný zaměstnanec}  \textbf{3.} \dicSynonym*{ófær til getnaðar} \dicDirectTranslationCS{impotentní}
\dicEntry[ónæði] \dicTerm{ó··næði} \dicIPA{{ou}{\textlengthmark}{n}{a}{i}{ð}{\textsci}} \dicPos{n}[2] \dicFlx{(‑s)}[20] \dicSynonym{truflun} \dicDirectTranslationCS{(vy)rušení, vyrušování} \dicAntonym{næði\smash{\textsuperscript{1}}};  \dicPhraseIS{gera e‑m ónæði} \dicDirectTranslationCS{rušit (koho)}
\dicEntry[ónæmi] \dicTerm{ó··næmi} \dicIPA{{ou}{\textlengthmark}{n}{a}{i}{m}{\textsci}} \dicPos{n}[2] \dicFlx{(‑s)}[20] \dicFieldCat{med.} \dicDirectTranslationCS{imunita, obranyschopnost}
\dicEntry[ónæmisfræði] \dicTerm{ó·næmis··fræð|i} \dicIPA{{ou}{\textlengthmark}{n}{a}{i}{m}{\textsci}{s}{f}{r}{a}{i}{ð}{\textsci}} \dicPos{f}[3] \dicFlx{(‑i)}[3] \dicFieldCat{med.} \dicDirectTranslationCS{imunologie}
\dicEntry[ónæmisfræðingur] \dicTerm{ó·næmis·fræð··ing|ur} \dicIPA{{ou}{\textlengthmark}{n}{a}{i}{m}{\textsci}{s}{f}{r}{a}{i}{ð}{i}{\ng}{\r{g}}{\textscy}{\textsubring{r}}} \dicPos{m}[6] \dicFlx{(‑s, ‑ar)}[8] \dicDirectTranslationCS{imunolog, imunoložka}
\dicEntry[ónæmiskerfi] \dicTerm{ó·næmis··kerfi} \dicIPA{{ou}{\textlengthmark}{n}{a}{i}{m}{\textsci}{s}{c\smash{\textsuperscript{h}}}{\textepsilon}{r}{v}{\textsci}} \dicPos{n}[2] \dicFlx{(‑s, ‑)}[14] \dicFieldCat{med.} \dicDirectTranslationCS{imunitní systém}
\dicEntry[ónæmur] \dicTerm{ó··næmur} \dicIPA{{ou}{\textlengthmark}{n}{a}{i}{m}{\textscy}{\textsubring{r}}} \dicPos{adj}[1]\dicFlx{}[-1] \textbf{1.} \dicFieldCat{med.} \dicDirectTranslationCS{imunní, odolný (vůči nemoci ap.)}  \textbf{2.} \dicSynonym{ómóttækilegur} \dicDirectTranslationCS{imunní, odolný (vůči výtkám ap.)} \dicAntonym{næmur}
\dicEntry[ónærgætinn] \dicTerm{ó··nær·gætinn} \dicIPA{{ou}{\textlengthmark}{n}{a}{i}{r}{\r{\textObardotlessj}}{a}{i}{\textsubring{d}}{\textsci}{\textsubring{n}}} \dicPos{adj}[6]\dicFlx{}[-2] \dicSynonym{tillitslaus} \dicDirectTranslationCS{neohleduplný, necitlivý} \dicAntonym{nærgætinn}
\dicEntry[ónærgætni] \dicTerm{ó··nær·gætn|i} \dicIPA{{ou}{\textlengthmark}{n}{a}{i}{r}{\r{\textObardotlessj}}{a}{i}{h}{\textsubring{d}}{n}{\textsci}} \dicPos{f}[3] \dicFlx{(‑i)}[3] \dicDirectTranslationCS{neohleduplnost, necitlivost} \dicAntonym{nærgætni}
\dicEntry[óorð] \dicTerm{ó··orð} \dicIPA{{ou}{\textlengthmark}{\textopeno}{r}{\texttheta}} \dicPos{n}[2] \dicFlx{(‑s, ‑)}[5] \dicSynonym{vansæmd} \dicDirectTranslationCS{špatná pověst};  \dicPhraseIS{koma óorði á e‑n} \dicSynonym*{ófrægja e‑n} \dicDirectTranslationCS{znevážit (koho)};  \dicPhraseIS{fá óorð af e‑m} \dicDirectTranslationCS{být zkompromitovaný (kým)}
\dicEntry[óp] \dicTerm{óp} \dicIPA{{ou}{\textlengthmark}{\textsubring{b}}} \dicPos{n}[2] \dicFlx{(‑s, ‑)}[5] \dicSynonym{hróp} \dicDirectTranslationCS{(vý)křik, řev} \dicExampleIS{Það heyrist óp innan úr húsinu.} \dicExampleCS{Z~domu je slyšet křik.}
\dicEntry[óp.] \dicTerm{óp.} \dicPos{zkr} \dicPhraseIS{ópersónulegur} \dicFieldCat{jaz.} \dicDirectTranslationCS{neosobní}
\dicEntry[ópall] \dicTerm{ópal|l} \dicIPA{{ou}{\textlengthmark}{\textsubring{b}}{a}{\textsubring{d}}{\textsubring{l}}} \dicPos{m}[6] \dicFlx{(‑s, ‑ar)}[49] \dicFieldCat{geol.} \dicDirectTranslationCS{opál}
\dicEntry[ópera] \dicTerm{óper|a} \dicIPA{{ou}{\textlengthmark}{\textsubring{b}}{\textepsilon}{r}{a}} \dicPos{f}[1] \dicFlx{(‑u, ‑ur)}[7] \dicFieldCat{hud.} \dicDirectTranslationCS{opera}
\dicEntry[óperetta] \dicTerm{óperett|a} \dicIPA{{ou}{\textlengthmark}{\textsubring{b}}{\textepsilon}{r}{\textepsilon}{h}{\textsubring{d}}{a}} \dicPos{f}[1] \dicFlx{(‑u, ‑ur)}[19] \dicFieldCat{hud.} \dicDirectTranslationCS{opereta}
\dicEntry[ópersónulegur] \dicTerm{ó··persónu·legur} \dicIPA{{ou}{\textlengthmark}{p\smash{\textsuperscript{h}}}{\textepsilon}{\textsubring{r}}{s}{ou}{n}{\textscy}{l}{\textepsilon}{\textbabygamma}{\textscy}{\textsubring{r}}} \dicPos{adj}[1]\dicFlx{}[-8] \textbf{1.} \dicDirectTranslationCS{neosobní, odměřený} \dicExampleIS{Hún er köld og ópersónuleg.} \dicExampleCS{Je chladná a~neosobní.} \dicAntonym{persónulegur}  \textbf{2.} \dicDirectTranslationCS{objektivní, nezúčastněný} \dicExampleIS{Dómar hans eru ópersónulegir.} \dicExampleCS{Jeho posudky jsou objektivní.}  \textbf{3.} \dicFieldCat{jaz.} \dicDirectTranslationCS{neosobní};  \dicPhraseIS{ópersónuleg setning} \dicFieldCat{jaz.} \dicDirectTranslationCS{neosobní věta};  \dicPhraseIS{ópersónuleg sögn} \dicFieldCat{jaz.} \dicDirectTranslationCS{neosobní sloveso}
\dicEntry[óperuhús] \dicTerm{óperu··hús} \dicIPA{{ou}{\textlengthmark}{\textsubring{b}}{\textepsilon}{r}{\textscy}{h}{u}{s}} \dicPos{n}[2] \dicFlx{(‑s, ‑)}[5] \dicDirectTranslationCS{opera} \dicIndirectTranslationCS{(operní budova)}
\dicEntry[ópíum] \dicTerm{ópíum} \dicIPA{{ou}{\textlengthmark}{\textsubring{b}}{i}{j}{\textscy}{\textsubring{m}}} \dicPos{n}[2] \dicFlx{(‑s)}[31] \dicDirectTranslationCS{opium}
\dicEntry[ópólitískur] \dicTerm{ó··pólitískur} \dicIPA{{ou}{\textlengthmark}{p\smash{\textsuperscript{h}}}{ou}{l}{\textsci}{\textsubring{d}}{i}{s}{\r{g}}{\textscy}{\textsubring{r}}} \dicPos{adj}[1]\dicFlx{}[-1] \dicDirectTranslationCS{nepolitický, apolitický}
\dicEntry[óprúttinn] \dicTerm{ó··prúttinn} \dicIPA{{ou}{\textlengthmark}{p\smash{\textsuperscript{h}}}{r}{u}{h}{\textsubring{d}}{\textsci}{\textsubring{n}}} \dicPos{adj}[6]\dicFlx{}[-2] \dicSynonym{ófyrirleitinn} \dicDirectTranslationCS{bezohledný, nevybíravý}
\dicEntry[óprýði] \dicTerm{ó··prýð|i} \dicIPA{{ou}{\textlengthmark}{p\smash{\textsuperscript{h}}}{r}{i}{ð}{\textsci}} \dicPos{f}[3] \dicFlx{(‑i)}[3] \dicSynonym{lýti} \dicDirectTranslationCS{kaz, vada} \dicExampleIS{vera óprýði á e‑u} \dicExampleCS{být vadou na (čem)} \dicAntonym{prýði}
\dicEntry[óra] \dicTerm{ór|a} \dicIPA{{ou}{\textlengthmark}{r}{a}} \dicPos{v}[1] \dicFlx{(‑aði)}[66] \dicFlx{impers} \dicPhraseIS{e‑n órar fyrir e‑u} \dicDirectTranslationCS{(kdo co) tuší, (kdo) má tušení o~(čem)}
\dicEntry[óraddaður] \dicTerm{ó··|radd·aður} \dicIPA{{ou}{\textlengthmark}{r}{a}{\textsubring{d}}{a}{ð}{\textscy}{\textsubring{r}}} \dicPos{adj}[3] \dicFlx{(f ‑rödduð)}[1] \dicFieldCat{jaz.} \dicDirectTranslationCS{neznělý} \dicAntonym{raddaður}
\dicEntry[óragur] \dicTerm{ó··|ragur} \dicIPA{{ou}{\textlengthmark}{r}{a}{\textbabygamma}{\textscy}{\textsubring{r}}} \dicPos{adj}[1] \dicFlx{(f ‑rög)}[2] \dicSynonym{óhræddur} \dicDirectTranslationCS{nebojácný, neohrožený} \dicAntonym{ragur}
\dicEntry[órakaður] \dicTerm{ó··|rak·aður} \dicIPA{{ou}{\textlengthmark}{r}{a}{\r{g}}{a}{ð}{\textscy}{\textsubring{r}}} \dicPos{adj}[3] \dicFlx{(f ‑rökuð)}[2] \dicDirectTranslationCS{neoholený}
\dicEntry[óraleið] \dicTerm{óra··leið} \dicIPA{{ou}{\textlengthmark}{r}{a}{l}{ei}{\texttheta}} \dicPos{f}[7] \dicFlx{(‑ar, ‑ir)}[1] \dicDirectTranslationCS{dálka, velmi dlouhá cesta}
\dicEntry[órangútan] \dicTerm{órangútan} \dicIPA{{ou}{\textlengthmark}{r}{a}{\ng}{\r{g}}{u}{\textsubring{d}}{a}{\textsubring{n}}} \dicPos{m}[4] \dicFlx{(‑s, ‑ar)}[11] \dicFieldCat{zool.} \dicDirectTranslationCS{orangutan} \textit{(l.~{\textLA{Pongo}})}  \dicsymPhoto\ 
\dicFigure{ds_image_orangutan_0_1.jpg}{Órangútan}{Órangútan - Samuel Luna, CC BY 2.0}
\dicEntry[órannsakanlegur] \dicTerm{ó··rann·sakan·legur} \dicIPA{{ou}{\textlengthmark}{r}{a}{n}{s}{a}{\r{g}}{a}{n}{l}{\textepsilon}{\textbabygamma}{\textscy}{\textsubring{r}}} \dicPos{adj}[1]\dicFlx{}[-8] \dicSynonym{órækur} \dicDirectTranslationCS{nevyzpytatelný, záhadný}
\dicEntry[órar] \dicTerm{órar} \dicIPA{{ou}{\textlengthmark}{r}{a}{\textsubring{r}}} \dicPos{m}[1] \dicFlx{pl}[2] \dicDirectTranslationCS{výmysly, fantasmagorie}
\dicEntry[óraunhæfur] \dicTerm{ó··raun·hæfur} \dicIPA{{ou}{\textlengthmark}{r}{\oe i}{n}{h}{a}{i}{v}{\textscy}{\textsubring{r}}} \dicPos{adj}[1]\dicFlx{}[-1] \dicSynonym{óframkvæmanlegur} \dicDirectTranslationCS{nereálný, neproveditelný, neuskutečnitelný}
\dicEntry[óraunsæi] \dicTerm{ó··raun·sæi} \dicIPA{{ou}{\textlengthmark}{r}{\oe i}{n}{s}{a}{i}{j}{\textsci}} \dicPos{n}[2] \dicFlx{(‑s)}[20] \dicDirectTranslationCS{utopie, fantazie, nereálná představa}
\dicEntry[óraunsær] \dicTerm{ó··raun·sær} \dicIPA{{ou}{\textlengthmark}{r}{\oe i}{n}{s}{a}{i}{\textsubring{r}}} \dicPos{adj}[4]\dicFlx{}[-7] \dicDirectTranslationCS{nerealistický} \dicAntonym{raunsær}
\dicEntry[óraunverulegur] \dicTerm{ó··raun·veru·legur} \dicIPA{{ou}{\textlengthmark}{r}{\oe i}{n}{v}{\textepsilon}{r}{\textscy}{l}{\textepsilon}{\textbabygamma}{\textscy}{\textsubring{r}}} \dicPos{adj}[1]\dicFlx{}[-8] \dicSynonym{ímyndaður} \dicDirectTranslationCS{nereálný, neskutečný, imaginární} \dicAntonym{raunverulegur}
\dicEntry[óráð] \dicTerm{ó··ráð} \dicIPA{{ou}{\textlengthmark}{r}{au}{\texttheta}} \dicPos{n}[2] \dicFlx{(‑s)}[2] \textbf{1.} \dicSynonym*{vont ráð} \dicDirectTranslationCS{špatná rada} \dicAntonym{ráð}  \textbf{2.} \dicSynonym*{höfuðórar} \dicDirectTranslationCS{blouznění, třeštění (z~horečky ap.)} \dicExampleIS{tala í óráði} \dicExampleCS{blouznit}
\dicEntry[óráðinn] \dicTerm{ó··ráðinn} \dicIPA{{ou}{\textlengthmark}{r}{au}{ð}{\textsci}{\textsubring{n}}} \dicPos{adj}[6]\dicFlx{}[-2] \textbf{1.} \dicSynonym{óákveðinn} \dicDirectTranslationCS{nerozhodnutý, váhavý};  \dicPhraseIS{ég er óráðinn í því} \dicDirectTranslationCS{ještě jsem se nerozhodl}  \textbf{2.} \dicSynonym*{óleystur} \dicDirectTranslationCS{nevyřešený} \dicExampleIS{óráðin gáta} \dicExampleCS{nevyřešená hádanka}
\dicEntry[óráðlegur] \dicTerm{ó··ráð·legur} \dicIPA{{ou}{\textlengthmark}{r}{au}{ð}{l}{\textepsilon}{\textbabygamma}{\textscy}{\textsubring{r}}} \dicPos{adj}[1]\dicFlx{}[-8] \dicSynonym{heimskulegur} \dicDirectTranslationCS{nevhodný, nerozumný, neuvážený} \dicAntonym{ráðlegur}
\dicEntry[óráðvandur] \dicTerm{ó··ráð·|vandur} \dicIPA{{ou}{\textlengthmark}{r}{au}{ð}{v}{a}{n}{\textsubring{d}}{\textscy}{\textsubring{r}}} \dicPos{adj}[2] \dicFlx{(f ‑vönd)}[15] \dicSynonym{óheiðarlegur} \dicDirectTranslationCS{nepoctivý, nečestný} \dicAntonym{ráðvandur}
\dicEntry[óregla] \dicTerm{ó··regl|a} \dicIPA{{ou}{\textlengthmark}{r}{\textepsilon}{\r{g}}{l}{a}} \dicPos{f}[1] \dicFlx{(‑u)}[5] \textbf{1.} \dicSynonym*{regluleysi} \dicDirectTranslationCS{nepravidelnost}  \textbf{2.} \dicSynonym{óreiða} \dicDirectTranslationCS{zmatek, nepořádek} \dicExampleIS{Það er óregla á starfseminni.} \dicExampleCS{V~té činnosti je zmatek.}  \textbf{3.} \dicSynonym{ofdrykkja} \dicDirectTranslationCS{nestřídmost (v~pití alkoholu), opilství}
\dicEntry[óreglulegur] \dicTerm{ó··reglu·legur} \dicIPA{{ou}{\textlengthmark}{r}{\textepsilon}{\r{g}}{l}{\textscy}{l}{\textepsilon}{\textbabygamma}{\textscy}{\textsubring{r}}} \dicPos{adj}[1]\dicFlx{}[-8] \dicSynonym{óskipulegur} \dicDirectTranslationCS{nepravidelný} \dicExampleIS{óreglulegur andardráttur} \dicExampleCS{nepravidelný dech} \dicAntonym{reglulegur};  \dicPhraseIS{óregluleg beyging} \dicFieldCat{jaz.} \dicDirectTranslationCS{nepravidelné skloňování\,/\addthin časování};  \dicPhraseIS{óregluleg stigbreyting} \dicFieldCat{jaz.} \dicDirectTranslationCS{nepravidelné stupňování}
\dicEntry[óreglumaður] \dicTerm{ó·reglu··|maður} \dicIPA{{ou}{\textlengthmark}{r}{\textepsilon}{\r{g}}{l}{\textscy}{m}{a}{ð}{\textscy}{\textsubring{r}}} \dicPos{m}[13] \dicFlx{(‑manns, ‑menn)}[2] \dicSynonym{ofdrykkjumaður} \dicDirectTranslationCS{alkoholik, alkoholička, opilec, opilkyně} \dicAntonym{reglumaður}
\dicEntry[óreiða] \dicTerm{ó··reið|a} \dicIPA{{ou}{\textlengthmark}{r}{ei}{ð}{a}} \dicPos{f}[1] \dicFlx{(‑u)}[5] \dicSynonym{ringulreið} \dicDirectTranslationCS{chaos, zmatek, nepořádek} \dicExampleIS{Það er óreiða á bókhaldinu.} \dicExampleCS{V~účetnictví je nepořádek.}
\dicEntry[óreyndur] \dicTerm{ó··reyndur} \dicIPA{{ou}{\textlengthmark}{r}{ei}{n}{\textsubring{d}}{\textscy}{\textsubring{r}}} \dicPos{adj}[2]\dicFlx{}[-14] \textbf{1.} \dicSynonym{reynslulaus} \dicDirectTranslationCS{nezkušený, (jsoucí) bez zkušeností} \dicExampleIS{óreyndur kennari} \dicExampleCS{učitel bez zkušeností}  \textbf{2.} \dicSynonym*{óprófaður} \dicDirectTranslationCS{netestovaný, nevyzkoušený} \dicExampleIS{óreyndur bíll} \dicExampleCS{netestované auto}
\dicEntry[órétti] \dicTerm{ó··rétti} \dicIPA{{ou}{\textlengthmark}{r}{j}{\textepsilon}{h}{\textsubring{d}}{\textsci}} \dicPos{n}[2] \dicFlx{(‑s)}[20] \dicSynonym{ranglæti} \dicDirectTranslationCS{křivda, nespravedlnost}
\dicEntry[óréttlátur] \dicTerm{ó··rétt·látur} \dicIPA{{ou}{\textlengthmark}{r}{j}{\textepsilon}{h}{\textsubring{d}}{l}{au}{\textsubring{d}}{\textscy}{\textsubring{r}}} \dicPos{adj}[1]\dicFlx{}[-1] \dicSynonym{ranglátur} \dicDirectTranslationCS{nespravedlivý, předpojatý} \dicAntonym{réttlátur}
\dicEntry[óréttlæti] \dicTerm{ó··rétt·læti} \dicIPA{{ou}{\textlengthmark}{r}{j}{\textepsilon}{h}{\textsubring{d}}{l}{a}{i}{\textsubring{d}}{\textsci}} \dicPos{n}[2] \dicFlx{(‑s)}[20] \dicSynonym{ranglæti} \dicDirectTranslationCS{nespravedlnost, předpojatost} \dicAntonym{réttlæti}
\dicEntry[óréttmætur] \dicTerm{ó··rétt·mætur} \dicIPA{{ou}{\textlengthmark}{r}{j}{\textepsilon}{h}{\textsubring{d}}{m}{a}{i}{\textsubring{d}}{\textscy}{\textsubring{r}}} \dicPos{adj}[1]\dicFlx{}[-1] \textbf{1.} \dicSynonym{ósanngjarn} \dicDirectTranslationCS{neodůvodněný, neopodstatněný, neoprávněný} \dicAntonym{réttmætur}  \textbf{2.} \dicSynonym{ólöglegur} \dicDirectTranslationCS{nezákonný, neprávoplatný}
\dicEntry[óréttur] \dicTerm{ó··rétt|ur} \dicIPA{{ou}{\textlengthmark}{r}{j}{\textepsilon}{h}{\textsubring{d}}{\textscy}{\textsubring{r}}} \dicPos{m}[10] \dicFlx{(‑ar)}[7] \textbf{1.} \dicSynonym{óréttlæti} \dicDirectTranslationCS{křivda, nespravedlnost};  \dicPhraseIS{beita e‑n órétti} \dicDirectTranslationCS{křivdit (komu)}  \textbf{2.} \dicDirectTranslationCS{neprávo};  \dicPhraseIS{vera í órétti} \dicDirectTranslationCS{dopustit se přestupku, provinit se} \dicIndirectTranslationCS{(nedodržovat pravidla dopravního provozu ap.)}
\dicEntry[órjúfandi] \dicTerm{ó··rjúf·andi} \dicIPA{{ou}{\textlengthmark}{r}{j}{u}{v}{a}{n}{\textsubring{d}}{\textsci}} \dicPos{adj}[13] \dicFlx{indecl}[1] \dicSynonym{óbrigðull} \dicDirectTranslationCS{nerozlučný, neoddělitelný} \dicExampleIS{órjúfandi vinátta} \dicExampleCS{nerozlučné kamarádství}
\dicEntry[órjúfanlegur] \dicTerm{ó··rjúfan·legur} \dicIPA{{ou}{\textlengthmark}{r}{j}{u}{v}{a}{n}{l}{\textepsilon}{\textbabygamma}{\textscy}{\textsubring{r}}} \dicPos{adj}[1]\dicFlx{}[-8] \dicSynonym{helgur} \dicDirectTranslationCS{nerozdělitelný, neoddělitelný}
\dicEntry[órofa] \dicTerm{ó··rofa} \dicIPA{{ou}{\textlengthmark}{r}{\textopeno}{v}{a}} \dicPos{adj}[13] \dicFlx{indecl}[1] \textbf{1.} \dicSynonym{órjúfandi} \dicDirectTranslationCS{nerozdělitelný, nerozlučný} \dicExampleIS{órofa heild} \dicExampleCS{nerozdělitelná jednota}  \textbf{2.} \dicSynonym{sífelldur} \dicDirectTranslationCS{nepřetržitý, nepřerušený}
\dicEntry[óró] \dicTerm{ó··ró} \dicIPA{{ou}{\textlengthmark}{r}{ou}} \dicPos{f}[9] \dicFlx{(‑ar)}[3] \dicSynonym{ókyrrð} \dicDirectTranslationCS{neklid, nepokoj} \dicAntonym{ró\smash{\textsuperscript{2}}}
\dicEntry[óróaseggur] \dicTerm{ó·róa··segg|ur} \dicIPA{{ou}{\textlengthmark}{r}{ou}{a}{s}{\textepsilon}{\r{g}}{\textscy}{\textsubring{r}}} \dicPos{m}[9] \dicFlx{(‑s, ‑ir)}[15] \dicSynonym*{friðarspillir} \dicDirectTranslationCS{výtržník, výtržnice, buřič(ka)}
\dicEntry[órói] \dicTerm{ó··ró|i} \dicIPA{{ou}{\textlengthmark}{r}{ou}{\textsci}} \dicPos{m}[1] \dicFlx{(‑a)}[3] \textbf{1.} \dicSynonym{óró} \dicDirectTranslationCS{neklid, nepokoj} \dicExampleIS{tilfinning sem vakti honum óróa} \dicExampleCS{pocit, který v~něm budil nepokoj}  \textbf{2.} \dicIndirectTranslationCS{ozdoba k~zavěšení, která se hýbe při každém závanu větru}
\dicEntry[órólegur] \dicTerm{ó··ró·legur} \dicIPA{{ou}{\textlengthmark}{r}{ou}{l}{\textepsilon}{\textbabygamma}{\textscy}{\textsubring{r}}} \dicPos{adj}[1]\dicFlx{}[-8] \textbf{1.} \dicSynonym{flóttalegur} \dicDirectTranslationCS{neklidný, nepokojný} \dicAntonym{rólegur}  \textbf{2.} \dicSynonym{kvíðinn} \dicDirectTranslationCS{znepokojený, mající obavy}
\dicEntry[óróleiki] \dicTerm{ó·ró··leik|i} \dicIPA{{ou}{\textlengthmark}{r}{ou}{l}{ei}{\r{\textObardotlessj}}{\textsci}} \dicPos{m}[1] \dicFlx{(‑a)}[3] \dicSynonym{órói} \dicDirectTranslationCS{neklidnost, nepokojnost}
\dicEntry[órór] \dicTerm{ó··rór} \dicIPA{{ou}{\textlengthmark}{r}{ou}{\textsubring{r}}} \dicPos{adj}[4]\dicFlx{}[-4] \dicDirectTranslationCS{neklidný, nepokojný}
\dicEntry[óræður] \dicTerm{ó··ræður} \dicIPA{{ou}{\textlengthmark}{r}{a}{i}{ð}{\textscy}{\textsubring{r}}} \dicPos{adj}[2]\dicFlx{}[-6] \textbf{1.} \dicSynonym{leyndardómsfullur} \dicDirectTranslationCS{nevyzpytatelný, neproniknutelný} \dicExampleIS{órætt bros} \dicExampleCS{nevyzpytatelný úsměv}  \textbf{2.} \dicFieldCat{mat.} \dicDirectTranslationCS{iracionální};  \dicPhraseIS{óræð tala} \dicFieldCat{mat.} \dicDirectTranslationCS{iracionální číslo}
\dicEntry[órækt] \dicTerm{ó··rækt} \dicIPA{{ou}{\textlengthmark}{r}{a}{i}{x}{\textsubring{d}}} \dicPos{f}[4] \dicFlx{(‑ar)}[3] \textbf{1.} \dicSynonym{hirðuleysi} \dicDirectTranslationCS{zanedbání, nedbalost}  \textbf{2.} \dicDirectTranslationCS{neobdělaná půda\,/\addthin země} \dicExampleIS{Garðurinn er í órækt.} \dicExampleCS{Zahrada je neobdělaná.}
\dicEntry[órækur] \dicTerm{ó··rækur} \dicIPA{{ou}{\textlengthmark}{r}{a}{i}{\r{g}}{\textscy}{\textsubring{r}}} \dicPos{adj}[1]\dicFlx{}[-1] \textbf{1.} \dicSynonym*{óhrekjanlegur} \dicDirectTranslationCS{nesporný, nepopiratelný, nezvratný} \dicExampleIS{órækur vitnisburður} \dicExampleCS{nesporné svědectví}  \textbf{2.} \dicSynonym{öruggur} \dicDirectTranslationCS{jistý, bezpečný} \dicExampleIS{órækt meðal} \dicExampleCS{bezpečný lék}
\dicEntry[órökréttur] \dicTerm{ó··rök·réttur} \dicIPA{{ou}{\textlengthmark}{r}{\oe}{\r{g}}{r}{j}{\textepsilon}{h}{\textsubring{d}}{\textscy}{\textsubring{r}}} \dicPos{adj}[1]\dicFlx{}[-10] \dicDirectTranslationCS{nelogický, iracionální} \dicExampleIS{órökrétt hræðsla} \dicExampleCS{iracionální strach} \dicAntonym{rökréttur}
\dicEntry[órökstuddur] \dicTerm{ó··rök·studdur} \dicIPA{{ou}{\textlengthmark}{r}{\oe}{\r{g}}{s}{\textsubring{d}}{\textscy}{\textsubring{d}}{\textscy}{\textsubring{r}}} \dicPos{adj}[2]\dicFlx{}[-18] \dicSynonym*{heimildarlaus} \dicDirectTranslationCS{nepodložený, neopodstatněný} \dicExampleIS{órökstudd gagnrýni} \dicExampleCS{nepodložená kritika} \dicAntonym{rökstuddur}
\dicEntry[ós] \dicTerm{ós} \dicIPA{{ou}{\textlengthmark}{s}} \dicPos{m}[4] \dicFlx{(‑s, ‑ar)}[6] \dicSynonym{mynni} \dicDirectTranslationCS{ústí (řeky ap.)} \dicExampleIS{sigla út úr ósnum} \dicExampleCS{vyplout z~ústí}
\dicEntry[ósa] \dicTerm{ós|a} \dicIPA{{ou}{\textlengthmark}{s}{a}} \dicPos{v}[1] \dicFlx{(‑aði)}[34] \dicSynonym{reykja} \dicDirectTranslationCS{kouřit, čoudit, čadit (kamna ap.)} \dicExampleIS{Lampinn ósar.} \dicExampleCS{Lampa čadí.}
\dicEntry[ósagður] \dicTerm{ó··|sagður} \dicIPA{{ou}{\textlengthmark}{s}{a}{\textbabygamma}{ð}{\textscy}{\textsubring{r}}} \dicPos{adj}[2] \dicFlx{(f ‑sögð)}[3] \dicDirectTranslationCS{nevyslovený, nevyřčený};  \dicPhraseIS{láta e‑ð ósagt} \dicDirectTranslationCS{nechat (co) nevyřčeného}
\dicEntry[ósaltaður] \dicTerm{ó··|salt·aður} \dicIPA{{ou}{\textlengthmark}{s}{a}{\textsubring{l}}{\textsubring{d}}{a}{ð}{\textscy}{\textsubring{r}}} \dicPos{adj}[3] \dicFlx{(f ‑söltuð)}[2] \dicDirectTranslationCS{nesolený, neosolený} \dicExampleIS{ósaltað smjör} \dicExampleCS{nesolené máslo}
\dicEntry[ósamboðinn] \dicTerm{ó··sam·boðinn} \dicIPA{{ou}{\textlengthmark}{s}{a}{m}{\textsubring{b}}{\textopeno}{ð}{\textsci}{\textsubring{n}}} \dicPos{adj}[6]\dicFlx{}[-2] \dicDirectTranslationCS{nehodný (lásky ap.)};  \dicPhraseIS{vera ósamboðinn e‑m} \dicDirectTranslationCS{nebýt hodný (koho)}
\dicEntry[ósambærilegur] \dicTerm{ó··sam·bæri·legur} \dicIPA{{ou}{\textlengthmark}{s}{a}{m}{\textsubring{b}}{a}{i}{r}{\textsci}{l}{\textepsilon}{\textbabygamma}{\textscy}{\textsubring{r}}} \dicPos{adj}[1]\dicFlx{}[-8] \dicSynonym*{ósamjafnanlegur} \dicDirectTranslationCS{nesrovnatelný, neporovnatelný} \dicAntonym{sambærilegur}
\dicEntry[ósamdóma] \dicTerm{ó··sam·dóma} \dicIPA{{ou}{\textlengthmark}{s}{a}{m}{\textsubring{d}}{ou}{m}{a}} \dicPos{adj}[13] \dicFlx{indecl}[1] \dicDirectTranslationCS{(jsoucí) jiného názoru}
\dicEntry[ósamhljóða] \dicTerm{ó··sam·hljóða} \dicIPA{{ou}{\textlengthmark}{s}{a}{m}{\textsubring{l}}{j}{ou}{ð}{a}} \dicPos{adj}[13] \dicFlx{indecl}[1] \dicSynonym{ósamkvæmur} \dicDirectTranslationCS{neidentický, nesouhlasný} \dicAntonym{samhljóða}
\dicEntry[ósamkvæmur] \dicTerm{ó··sam·kvæmur} \dicIPA{{ou}{\textlengthmark}{s}{a}{m}{k\smash{\textsuperscript{h}}}{v}{a}{i}{m}{\textscy}{\textsubring{r}}} \dicPos{adj}[1]\dicFlx{}[-1] \dicSynonym{ósamhljóða} \dicDirectTranslationCS{rozporuplný, (jsoucí) v~rozporu} \dicExampleIS{Þessi krafa er ósamkvæm lögunum.} \dicExampleCS{Tento požadavek je v~rozporu se zákonem.}
\dicEntry[ósamlyndi] \dicTerm{ó··sam·lyndi} \dicIPA{{ou}{\textlengthmark}{s}{a}{m}{l}{\textsci}{n}{\textsubring{d}}{\textsci}} \dicPos{n}[2] \dicFlx{(‑s)}[20] \dicSynonym{ágreiningur} \dicDirectTranslationCS{nesoulad, neshoda} \dicAntonym{samlyndi}
\dicEntry[ósammála] \dicTerm{ó··sam·mála} \dicIPA{{ou}{\textlengthmark}{s}{a}{m}{au}{l}{a}} \dicPos{adj}[13] \dicFlx{indecl}[1] \dicDirectTranslationCS{nesouhlasný, nesouhlasící, neshodující se};  \dicPhraseIS{vera ósammála} \dicDirectTranslationCS{neshodnout se, neshodovat se} \dicExampleIS{Við erum ósammála um þetta.} \dicExampleCS{V~tom se neshodneme.}
\dicEntry[ósamrýmanlegur] \dicTerm{ó··sam·rýman·legur} \dicIPA{{ou}{\textlengthmark}{s}{a}{m}{r}{i}{m}{a}{n}{l}{\textepsilon}{\textbabygamma}{\textscy}{\textsubring{r}}} \dicPos{adj}[1]\dicFlx{}[-8] \dicSynonym{andstæður} \dicDirectTranslationCS{neslučitelný, odporující si (vzájemně)}
\dicEntry[ósamræmi] \dicTerm{ó··sam·ræmi} \dicIPA{{ou}{\textlengthmark}{s}{a}{m}{r}{a}{i}{m}{\textsci}} \dicPos{n}[2] \dicFlx{(‑s)}[20] \dicSynonym{mótsögn} \dicDirectTranslationCS{protiklad, rozpor, rozporuplnost} \dicExampleIS{Stíllinn er í ósamræmi við efnið.} \dicExampleCS{Styl je v~rozporu s~látkou vyprávění.} \dicAntonym{samræmi}
\dicEntry[ósamstæður] \dicTerm{ó··sam·stæður} \dicIPA{{ou}{\textlengthmark}{s}{a}{m}{s}{\textsubring{d}}{a}{i}{ð}{\textscy}{\textsubring{r}}} \dicPos{adj}[2]\dicFlx{}[-6] \dicSynonym*{ósamkynja} \dicDirectTranslationCS{různorodý, nesourodý, heterogenní} \dicAntonym{samstæður}
\dicEntry[ósamþykkur] \dicTerm{ó··sam·þykkur} \dicIPA{{ou}{\textlengthmark}{s}{a}{m}{\texttheta}{\textsci}{h}{\r{g}}{\textscy}{\textsubring{r}}} \dicPos{adj}[1]\dicFlx{}[-1] \dicSynonym{ósammála} \dicDirectTranslationCS{nesouhlasný, nesouhlasící} \dicExampleIS{Ég er því ósamþykkur.} \dicExampleCS{Nesouhlasím s~tím.} \dicAntonym{samþykkur}
\dicEntry[ósanngirni] \dicTerm{ó··sann·girn|i} \dicIPA{{ou}{\textlengthmark}{s}{a}{n}{\r{\textObardotlessj}}{\textsci}{r}{\textsubring{d}}{n}{\textsci}} \dicPos{f}[3] \dicFlx{(‑i)}[3] \dicSynonym{ranglæti} \dicDirectTranslationCS{nespravedlnost, neférovost} \dicAntonym{sanngirni}
\dicEntry[ósanngjarn] \dicTerm{ó··sann·|gjarn} \dicIPA{{ou}{\textlengthmark}{s}{a}{n}{\r{\textObardotlessj}}{a}{r}{\textsubring{d}}{\textsubring{n}}} \dicPos{adj}[5] \dicFlx{(f ‑gjörn)}[6] \dicSynonym{ranglátur} \dicDirectTranslationCS{nespravedlivý, neférový} \dicAntonym{sanngjarn}
\dicEntry[ósannindi] \dicTerm{ó··sann·indi} \dicIPA{{ou}{\textlengthmark}{s}{a}{n}{\textsci}{n}{\textsubring{d}}{\textsci}} \dicPos{n}[2] \dicFlx{pl}[19] \dicSynonym{skrök} \dicDirectTranslationCS{nepravda, lež};  \dicPhraseIS{fara með ósannindi} \dicSynonym{ljúga} \dicDirectTranslationCS{lhát}
\dicEntry[ósannsögli] \dicTerm{ó··sann·sögl|i} \dicIPA{{ou}{\textlengthmark}{s}{a}{n}{s}{\oe}{\r{g}}{l}{\textsci}} \dicPos{f}[3] \dicFlx{(‑i)}[3] \dicSynonym{skreytni} \dicDirectTranslationCS{lživost, prolhanost} \dicAntonym{sannsögli}
\dicEntry[ósannsögull] \dicTerm{ó··sann·sögull} \dicIPA{{ou}{\textlengthmark}{s}{a}{n}{s}{\oe}{\textbabygamma}{\textscy}{\textsubring{d}}{\textsubring{l}}} \dicPos{adj}[8]\dicFlx{}[-5] \dicSynonym{lyginn} \dicDirectTranslationCS{lživý, prolhaný} \dicAntonym{sannsögull}
\dicEntry[ósannur] \dicTerm{ó··|sannur} \dicIPA{{ou}{\textlengthmark}{s}{a}{n}{\textscy}{\textsubring{r}}} \dicPos{adj}[1] \dicFlx{(f ‑sönn)}[9] \dicSynonym{rangur} \dicDirectTranslationCS{nepravdivý, falešný, lživý} \dicExampleIS{Það er ósatt.} \dicExampleCS{To není pravda.} \dicAntonym{sannur\smash{\textsuperscript{1}}};  \dicPhraseIS{segja ósatt} \dicSynonym{ljúga} \dicDirectTranslationCS{lhát, vymýšlet si}
\dicEntry[ósatt] \dicTerm{ó··satt} \dicIPA{{ou}{\textlengthmark}{s}{a}{h}{\textsubring{d}}} \dicPos{adj} \dicFlx{n sg nom pos} \dicLink{ósannur}
\dicEntry[ósáinn] \dicTerm{ó··sáinn} \dicIPA{{ou}{\textlengthmark}{s}{au}{\textsci}{\textsubring{n}}} \dicPos{adj}[6]\dicFlx{}[-6] \dicDirectTranslationCS{neosetý} \dicExampleIS{ósáinn akur} \dicExampleCS{neoseté pole}
\dicEntry[ósátt] \dicTerm{ó··|sátt} \dicIPA{{ou}{\textlengthmark}{s}{au}{h}{\textsubring{d}}} \dicPos{f}[7] \dicFlx{(‑sáttar, ‑sáttir\,/\addthin ‑sættir)}[25] \dicSynonym{missætti} \dicDirectTranslationCS{neshoda, rozkol, rozpor}
\dicEntry[ósáttfús] \dicTerm{ó··sátt·fús} \dicIPA{{ou}{\textlengthmark}{s}{au}{h}{\textsubring{d}}{f}{u}{s}} \dicPos{adj}[5]\dicFlx{}[-1] \dicSynonym*{undanlátslaus} \dicDirectTranslationCS{nesmlouvavý, neústupný, nesmiřitelný} \dicAntonym{sáttfús}
\dicEntry[ósáttur] \dicTerm{ó··sáttur} \dicIPA{{ou}{\textlengthmark}{s}{au}{h}{\textsubring{d}}{\textscy}{\textsubring{r}}} \dicPos{adj}[1]\dicFlx{}[-10] \dicSynonym{ósammála} \dicDirectTranslationCS{nesouhlasící, nesouhlasný} \dicAntonym{sáttur}
\dicEntry[óseðjandi] \dicTerm{ó··seðj·andi} \dicIPA{{ou}{\textlengthmark}{s}{\textepsilon}{ð}{j}{a}{n}{\textsubring{d}}{\textsci}} \dicPos{adj}[13] \dicFlx{indecl}[1] \dicSynonym{gráðugur} \dicDirectTranslationCS{nenasytný, neukojitelný} \dicAntonym{seðjandi}
\dicEntry[ósegjanlegur] \dicTerm{ó··segjan·legur} \dicIPA{{ou}{\textlengthmark}{s}{\textepsilon}{j}{a}{n}{l}{\textepsilon}{\textbabygamma}{\textscy}{\textsubring{r}}} \dicPos{adj}[1]\dicFlx{}[-8] \dicDirectTranslationCS{nepopsatelný, nevýslovný, neslýchaný}
\dicEntry[ósekja] \dicTerm{ó··sekj|a} \dicIPA{{ou}{\textlengthmark}{s}{\textepsilon}{\r{\textObardotlessj}}{a}} \dicPos{f}[1] \dicFlx{(‑u)}[5] \dicPhraseIS{að ósekju} \dicFlx{adv} \dicSynonym*{að vítalausu} \dicDirectTranslationCS{bezdůvodně}
\dicEntry[ósennilegur] \dicTerm{ó··senni·legur} \dicIPA{{ou}{\textlengthmark}{s}{\textepsilon}{n}{\textsci}{l}{\textepsilon}{\textbabygamma}{\textscy}{\textsubring{r}}} \dicPos{adj}[1]\dicFlx{}[-8] \dicDirectTranslationCS{nepravděpodobný, pochybný} \dicAntonym{sennilegur}
\dicEntry[óséður] \dicTerm{ó··séður} \dicIPA{{ou}{\textlengthmark}{s}{j}{\textepsilon}{ð}{\textscy}{\textsubring{r}}} \dicPos{adj}[2]\dicFlx{}[-13] \dicDirectTranslationCS{neviděný, nespatřený};  \dicPhraseIS{kaupa e‑ð óséð, kaupa e‑ð að óséðu\,/\addthin ósénu} \dicLangCat{přen.} \dicDirectTranslationCS{koupit zajíce v~pytli} \dicIndirectTranslationCS{(koupit (co), aniž by si to člověk nejdříve prohlédl)}
\dicEntry[ósérhlífinn] \dicTerm{ó··sér·hlífinn} \dicIPA{{ou}{\textlengthmark}{s}{j}{\textepsilon}{\textsubring{r}}{\textsubring{l}}{i}{v}{\textsci}{\textsubring{n}}} \dicPos{adj}[6]\dicFlx{}[-2] \dicSynonym{ótrauður} \dicDirectTranslationCS{obětavý, nesobecký} \dicAntonym{sérhlífinn}
\dicEntry[ósérhlífni] \dicTerm{ó··sér·hlífn|i} \dicIPA{{ou}{\textlengthmark}{s}{j}{\textepsilon}{\textsubring{r}}{\textsubring{l}}{i}{\textsubring{b}}{n}{\textsci}} \dicPos{f}[3] \dicFlx{(‑i)}[3] \dicDirectTranslationCS{obětavost, nesobeckost}
\dicEntry[ósérplæginn] \dicTerm{ó··sér·plæginn} \dicIPA{{ou}{\textlengthmark}{s}{j}{\textepsilon}{\textsubring{r}}{p\smash{\textsuperscript{h}}}{l}{a}{i}{j}{\textsci}{\textsubring{n}}} \dicPos{adj}[6]\dicFlx{}[-2] \dicSynonym{óeigingjarn} \dicDirectTranslationCS{nesobecký, nezištný}
\dicEntry[ósérplægni] \dicTerm{ó··sér·plægn|i} \dicIPA{{ou}{\textlengthmark}{s}{j}{\textepsilon}{\textsubring{r}}{p\smash{\textsuperscript{h}}}{l}{a}{i}{\r{g}}{n}{\textsci}} \dicPos{f}[3] \dicFlx{(‑i)}[3] \dicSynonym{óeigingirni} \dicDirectTranslationCS{nesobeckost, nezištnost} \dicAntonym{sérplægni}
\dicEntry[ósiðaður] \dicTerm{ó··sið·|aður} \dicIPA{{ou}{\textlengthmark}{s}{\textsci}{ð}{a}{ð}{\textscy}{\textsubring{r}}} \dicPos{adj}[3] \dicFlx{(f ‑uð)}[3] \dicSynonym{ósvífinn} \dicDirectTranslationCS{nekultivovaný, necivilizovaný} \dicAntonym{siðaður}
\dicEntry[ósiðlegur] \dicTerm{ó··sið·legur} \dicIPA{{ou}{\textlengthmark}{s}{\textsci}{ð}{l}{\textepsilon}{\textbabygamma}{\textscy}{\textsubring{r}}} \dicPos{adj}[1]\dicFlx{}[-8] \dicSynonym{siðlaus} \dicDirectTranslationCS{nemravný, nemorální, neetický} \dicAntonym{siðlegur}
\dicEntry[ósiðsamur] \dicTerm{ó··sið·|samur} \dicIPA{{ou}{\textlengthmark}{s}{\textsci}{ð}{s}{a}{m}{\textscy}{\textsubring{r}}} \dicPos{adj}[1] \dicFlx{(f ‑söm)}[2] \dicSynonym{ósiðlegur} \dicDirectTranslationCS{nemravný, nemorální} \dicAntonym{siðsamur}
\dicEntry[ósiður] \dicTerm{ó··sið|ur} \dicIPA{{ou}{\textlengthmark}{s}{\textsci}{ð}{\textscy}{\textsubring{r}}} \dicPos{m}[10] \dicFlx{(‑ar, ‑ir)}[16] \dicSynonym{kækur} \dicDirectTranslationCS{zlozvyk, nešvar} \dicExampleIS{kenna börnunum ýmsa ósiði} \dicExampleCS{naučit děti různé zlozvyky}
\dicEntry[ósigrandi] \dicTerm{ó··sigr·andi} \dicIPA{{ou}{\textlengthmark}{s}{\textsci}{\textbabygamma}{r}{a}{n}{\textsubring{d}}{\textsci}} \dicPos{adj}[13] \dicFlx{indecl}[1] \dicSynonym{óvinnandi} \dicDirectTranslationCS{neporazitelný, nepřemožitelný}
\dicEntry[ósigur] \dicTerm{ó··sig|ur} \dicIPA{{ou}{\textlengthmark}{s}{\textsci}{\textbabygamma}{\textscy}{\textsubring{r}}} \dicPos{m}[5] \dicFlx{(‑urs, ‑rar)}[1] \dicSynonym{ófarir} \dicDirectTranslationCS{porážka, prohra} \dicExampleIS{bregðast karlmannlega við ósigrinum} \dicExampleCS{vyrovnat se mužně s~prohrou} \dicAntonym{sigur}
\dicEntry[ósjáanlegur] \dicTerm{ó··sjáan·legur} \dicIPA{{ou}{\textlengthmark}{s}{j}{au}{a}{n}{l}{\textepsilon}{\textbabygamma}{\textscy}{\textsubring{r}}} \dicPos{adj}[1]\dicFlx{}[-8] \dicSynonym{ósýnilegur} \dicDirectTranslationCS{neviditelný} \dicAntonym{sjáanlegur}
\dicEntry[ósjálfbjarga] \dicTerm{ó··sjálf·bjarga} \dicIPA{{ou}{\textlengthmark}{s}{j}{au}{l}{v}{\textsubring{b}}{j}{a}{r}{\r{g}}{a}} \dicPos{adj}[13] \dicFlx{indecl}[1] \textbf{1.} \dicSynonym{hjálparlaus} \dicDirectTranslationCS{nesamostatný, potřebující pomoci\,/\addthin rady} \dicAntonym{sjálfbjarga}  \textbf{2.} \dicDirectTranslationCS{bezradný, rozpačitý}
\dicEntry[ósjálfráða] \dicTerm{ó··sjálf·ráða} \dicIPA{{ou}{\textlengthmark}{s}{j}{au}{l}{v}{r}{au}{ð}{a}} \dicPos{adj}[13] \dicFlx{indecl}[1] \dicFieldCat{práv.} \dicDirectTranslationCS{nezletilý}
\dicEntry[ósjálfráður] \dicTerm{ó··sjálf·ráður} \dicIPA{{ou}{\textlengthmark}{s}{j}{au}{l}{v}{r}{au}{ð}{\textscy}{\textsubring{r}}} \dicPos{adj}[2]\dicFlx{}[-6] \textbf{1.} \dicSynonym{vélgengur} \dicDirectTranslationCS{bezděčný, samovolný, automatický (reakce ap.)} \dicAntonym{sjálfráður};  \dicPhraseIS{ósjálfrátt viðbragð} \dicDirectTranslationCS{reflex};  \dicPhraseIS{gera e‑ð ósjálfrátt} \dicDirectTranslationCS{dělat (co) bezděčně\,/\addthin bezmyšlenkovitě}  \textbf{2.} \dicFieldCat{práv.} \dicSynonym{ósjálfráða} \dicDirectTranslationCS{nezletilý}
\dicEntry[ósjálfstæði] \dicTerm{ó··sjálf·stæði} \dicIPA{{ou}{\textlengthmark}{s}{j}{au}{l}{v}{s}{\textsubring{d}}{a}{i}{ð}{\textsci}} \dicPos{n}[2] \dicFlx{(‑s)}[20] \dicDirectTranslationCS{nesamostatnost, závislost (na druhých lidech)} \dicAntonym{sjálfstæði}
\dicEntry[ósjálfstæður] \dicTerm{ó··sjálf·stæður} \dicIPA{{ou}{\textlengthmark}{s}{j}{au}{l}{v}{s}{\textsubring{d}}{a}{i}{ð}{\textscy}{\textsubring{r}}} \dicPos{adj}[2]\dicFlx{}[-6] \dicSynonym{háður} \dicDirectTranslationCS{nesamostatný, závislý (na druhých lidech)} \dicAntonym{sjálfstæður}
\dicEntry[ósk] \dicTerm{ósk} \dicsymFrequent\  \dicIPA{{ou}{s}{\r{g}}} \dicPos{f}[7] \dicFlx{(‑ar, ‑ir)}[1] \dicSynonym{löngun} \dicDirectTranslationCS{přání, touha, tužba} \dicExampleIS{Við sendum ykkur bestu óskir um gleðileg jól.} \dicExampleCS{Posíláme Vám srdečná přání veselých Vánoc.};  \dicPhraseIS{bera fram ósk} \dicDirectTranslationCS{(po)přát};  \dicPhraseIS{fara\,/\addthin ganga að óskum} \dicSynonym*{heppnast vel} \dicDirectTranslationCS{splnit se, vyjít podle přání};  \dicPhraseIS{e‑m verður að ósk sinni} \dicFlx{impers} \dicDirectTranslationCS{(komu) se splní přání}
\dicEntry[óska] \dicTerm{ósk|a} \dicIPA{{ou}{s}{\r{g}}{a}} \dicPos{v}[1] \dicFlx{(‑aði)}[1] \dicFlx{dat + gen} \textbf{1.} \dicSynonym*{bera fram ósk} \dicDirectTranslationCS{(po)přát};  \dicPhraseIS{óska e‑m e‑s} \dicDirectTranslationCS{přát (komu co)} \dicExampleIS{óska honum alls góðs} \dicExampleCS{přát mu vše dobré};  \dicPhraseIS{óska e‑m til hamingju} \dicSynonym{árna} \dicDirectTranslationCS{(po)přát (komu) vše nejlepší, (po)gratulovat (komu), (po)přát (komu) hodně štěstí}  \textbf{2.} \dicSynonym{æskja} \dicDirectTranslationCS{přát si, toužit};  \dicPhraseIS{óska sér e‑s} \dicDirectTranslationCS{přát si (co), toužit po (čem)};  \dicPhraseIS{óska eftir e‑u} \dicDirectTranslationCS{přát si (co), toužit po (čem)};  \dicIdiom{óskast}{ \dicPhraseIS{e‑r\,/\addthin e‑að óskast}} \dicFlx{refl} \dicDirectTranslationCS{(kdo\,/\addthin co) se hledá (hospodyně ap.), po (kom\,/\addthin čem) je poptávka} \dicExampleIS{Barnapía óskast.} \dicExampleCS{Hledá se chůva.}
\dicEntry[óskabarn] \dicTerm{óska··|barn} \dicIPA{{ou}{s}{\r{g}}{a}{\textsubring{b}}{a}{r}{\textsubring{d}}{\textsubring{n}}} \dicPos{n}[2] \dicFlx{(‑barns, ‑börn)}[8] \dicDirectTranslationCS{oblíbenec, oblíbenkyně, miláček, favorit(ka)}
\dicEntry[óskaddaður] \dicTerm{ó··|skadd·aður} \dicIPA{{ou}{\textlengthmark}{s}{\r{g}}{a}{\textsubring{d}}{a}{ð}{\textscy}{\textsubring{r}}} \dicPos{adj}[3] \dicFlx{(f ‑sködduð)}[2] \dicSynonym{ómeiddur} \dicDirectTranslationCS{nezraněný, neporaněný} \dicAntonym{skaddaður};  \dicPhraseIS{sleppa óskaddaður} \dicDirectTranslationCS{vyváznout bez zranění}
\dicEntry[óskaðlegur] \dicTerm{ó··skað·legur} \dicIPA{{ou}{\textlengthmark}{s}{\r{g}}{a}{ð}{l}{\textepsilon}{\textbabygamma}{\textscy}{\textsubring{r}}} \dicPos{adj}[1]\dicFlx{}[-8] \dicSynonym{meinlaus} \dicDirectTranslationCS{neškodný, neškodlivý} \dicAntonym{skaðlegur}
\dicEntry[óskalag] \dicTerm{óska··|lag} \dicIPA{{ou}{s}{\r{g}}{a}{l}{a}{x}} \dicPos{n}[2] \dicFlx{(‑lags, ‑lög)}[8] \dicDirectTranslationCS{píseň\,/\addthin písnička na přání (v~rádiu ap.)}
\dicEntry[óskaland] \dicTerm{óska··|land} \dicIPA{{ou}{s}{\r{g}}{a}{l}{a}{n}{\textsubring{d}}} \dicPos{n}[2] \dicFlx{(‑lands, ‑lönd)}[8] \dicDirectTranslationCS{vysněná\,/\addthin zaslíbená země}
\dicEntry[óskammfeilinn] \dicTerm{ó··skamm·feilinn} \dicIPA{{ou}{\textlengthmark}{s}{\r{g}}{a}{m}{f}{ei}{l}{\textsci}{\textsubring{n}}} \dicPos{adj}[6]\dicFlx{}[-2] \dicSynonym{ósvífinn} \dicDirectTranslationCS{nestoudný, nestydatý}
\dicEntry[óskammfeilni] \dicTerm{ó··skamm·feiln|i} \dicIPA{{ou}{\textlengthmark}{s}{\r{g}}{a}{m}{f}{ei}{l}{n}{\textsci}} \dicPos{f}[3] \dicFlx{(‑i)}[3] \dicSynonym{ósvífni} \dicDirectTranslationCS{nestoudnost, nestydatost}
\dicEntry[óskapast] \dicTerm{óskap|ast} \dicIPA{{ou}{\textlengthmark}{s}{\r{g}}{a}{\textsubring{b}}{a}{s}{\textsubring{d}}} \dicPos{v}[1] \dicFlx{(‑aðist)}[93] \dicFlx{refl} \dicSynonym*{fárast} \dicDirectTranslationCS{vyvádět, povykovat};  \dicPhraseIS{óskapast út af e‑u, óskapast yfir e‑u} \dicDirectTranslationCS{vyvádět kvůli (čemu)}
\dicEntry[óskaplega] \dicTerm{ó·skap··lega} \dicsymFrequent\  \dicIPA{{ou}{\textlengthmark}{s}{\r{g}}{a}{\textsubring{b}}{l}{\textepsilon}{\textbabygamma}{a}} \dicPos{adv} \dicSynonym*{gífurlega} \dicDirectTranslationCS{nesmírně, příšerně} \dicExampleIS{óskaplega flókið} \dicExampleCS{nesmírně složité}
\dicEntry[óskaplegur] \dicTerm{ó·skap··legur} \dicIPA{{ou}{\textlengthmark}{s}{\r{g}}{a}{\textsubring{b}}{l}{\textepsilon}{\textbabygamma}{\textscy}{\textsubring{r}}} \dicPos{adj}[1]\dicFlx{}[-8] \dicSynonym{afskaplegur} \dicDirectTranslationCS{nesmírný, příšerný (chyba ap.)}
\dicEntry[óskapnaður] \dicTerm{ó··skap·nað|ur} \dicIPA{{ou}{\textlengthmark}{s}{\r{g}}{a}{h}{\textsubring{b}}{n}{a}{ð}{\textscy}{\textsubring{r}}} \dicPos{m}[10] \dicFlx{(‑ar)}[9] \textbf{1.} \dicSynonym{ringulreið} \dicDirectTranslationCS{chaos, zmatek}  \textbf{2.} \dicSynonym{afskræmi} \dicDirectTranslationCS{zrůda, monstrum}
\dicEntry[óskasteinn] \dicTerm{óska··stein|n} \dicIPA{{ou}{s}{\r{g}}{a}{s}{\textsubring{d}}{ei}{\textsubring{d}}{\textsubring{n}}} \dicPos{m}[6] \dicFlx{(‑s, ‑ar)}[42] \dicFieldCat{pov.} \dicIndirectTranslationCS{kámen, který plní přání}
\dicEntry[óskáldlegur] \dicTerm{ó··skáld·legur} \dicIPA{{ou}{\textlengthmark}{s}{\r{g}}{au}{l}{\textsubring{d}}{l}{\textepsilon}{\textbabygamma}{\textscy}{\textsubring{r}}} \dicPos{adj}[1]\dicFlx{}[-8] \dicSynonym{hversdagslegur} \dicDirectTranslationCS{prozaický} \dicAntonym{skáldlegur}
\dicEntry[óskeikull] \dicTerm{ó··skeikull} \dicIPA{{ou}{\textlengthmark}{s}{\r{\textObardotlessj}}{ei}{\r{g}}{\textscy}{\textsubring{d}}{\textsubring{l}}} \dicPos{adj}[8]\dicFlx{}[-4] \dicSynonym{óbrigðull} \dicDirectTranslationCS{neomylný, neselhávající}
\dicEntry[óskemmdur] \dicTerm{ó··skemmdur} \dicIPA{{ou}{\textlengthmark}{s}{\r{\textObardotlessj}}{\textepsilon}{m}{\textsubring{d}}{\textscy}{\textsubring{r}}} \dicPos{adj}[2]\dicFlx{}[-14] \textbf{1.} \dicSynonym{ferskur} \dicDirectTranslationCS{nezkažený, nepokažený (jídlo ap.)} \dicAntonym{skemmdur}  \textbf{2.} \dicSynonym{alheill} \dicDirectTranslationCS{nepoškozený, nedotčený (věc ap.)}
\dicEntry[óskemmtilegur] \dicTerm{ó··skemmti·legur} \dicIPA{{ou}{\textlengthmark}{s}{\r{\textObardotlessj}}{\textepsilon}{\textsubring{m}}{\textsubring{d}}{\textsci}{l}{\textepsilon}{\textbabygamma}{\textscy}{\textsubring{r}}} \dicPos{adj}[1]\dicFlx{}[-8] \textbf{1.} \dicSynonym*{spauglaus} \dicDirectTranslationCS{nudný, nezábavný} \dicAntonym{skemmtilegur}  \textbf{2.} \dicSynonym{óþægilegur} \dicDirectTranslationCS{nepříjemný, špatný}
\dicEntry[óskertur] \dicTerm{ó··skertur} \dicIPA{{ou}{\textlengthmark}{s}{\r{\textObardotlessj}}{\textepsilon}{\textsubring{r}}{\textsubring{d}}{\textscy}{\textsubring{r}}} \dicPos{adj}[1]\dicFlx{}[-10] \dicSynonym{heill\smash{\textsuperscript{2}}} \dicDirectTranslationCS{neporušený, nedotčený} \dicAntonym{skertur}
\dicEntry[óskhyggja] \dicTerm{ósk··hyggj|a} \dicIPA{{ou}{s}{\r{g}}{h}{\textsci}{\r{\textObardotlessj}}{a}} \dicPos{f}[1] \dicFlx{(‑u)}[5] \dicSynonym*{óskahugsun} \dicDirectTranslationCS{zbožné\,/\addthin toužebné přání}
\dicEntry[óskil] \dicTerm{ó··skil} \dicIPA{{ou}{\textlengthmark}{s}{\r{\textObardotlessj}}{\textsci}{\textsubring{l}}} \dicPos{n}[2] \dicFlx{pl}[1] \dicPhraseIS{e‑að er í óskilum} \dicDirectTranslationCS{hledá se majitel (čeho)} \dicExampleIS{Það er hundur í óskilum.} \dicExampleCS{Hledá se majitel psa.}
\dicEntry[óskilgetinn] \dicTerm{ó··skil·getinn} \dicIPA{{ou}{\textlengthmark}{s}{\r{\textObardotlessj}}{\textsci}{l}{\r{\textObardotlessj}}{\textepsilon}{\textsubring{d}}{\textsci}{\textsubring{n}}} \dicPos{adj}[6]\dicFlx{}[-6] \dicSynonym*{hjágetinn} \dicDirectTranslationCS{nemanželský, nelegitimní} \dicExampleIS{óskilgetið barn} \dicExampleCS{nemanželské dítě} \dicAntonym{skilgetinn}
\dicEntry[óskiljanlegur] \dicTerm{ó··skiljan·legur} \dicsymFrequent\  \dicIPA{{ou}{\textlengthmark}{s}{\r{\textObardotlessj}}{\textsci}{l}{j}{a}{n}{l}{\textepsilon}{\textbabygamma}{\textscy}{\textsubring{r}}} \dicPos{adj}[1]\dicFlx{}[-8] \dicSynonym*{óskýranlegur} \dicDirectTranslationCS{nesrozumitelný, nepochopitelný} \dicExampleIS{óskiljanlegur texti} \dicExampleCS{nesrozumitelný text} \dicAntonym{skiljanlegur}
\dicEntry[óskiptur] \dicTerm{ó··skiptur} \dicIPA{{ou}{\textlengthmark}{s}{\r{\textObardotlessj}}{\textsci}{f}{\textsubring{d}}{\textscy}{\textsubring{r}}} \dicPos{adj}[1]\dicFlx{}[-13] \textbf{1.} \dicSynonym*{óskilinn} \dicDirectTranslationCS{plný, nerozdělený} \dicExampleIS{hlusta með óskiptri athygli} \dicExampleCS{poslouchat s~plnou pozorností}  \textbf{2.} \dicSynonym{óbreyttur} \dicDirectTranslationCS{nezměněný, nevyměněný}
\dicEntry[óskipulegur] \dicTerm{ó··skipu·legur} \dicIPA{{ou}{\textlengthmark}{s}{\r{\textObardotlessj}}{\textsci}{\textsubring{b}}{\textscy}{l}{\textepsilon}{\textbabygamma}{\textscy}{\textsubring{r}}} \dicPos{adj}[1]\dicFlx{}[-8] \dicSynonym{óreglulegur} \dicDirectTranslationCS{neuspořádaný, neorganizovaný, nesystematický} \dicAntonym{skipulegur}
\dicEntry[óskírður] \dicTerm{ó··skírður} \dicIPA{{ou}{\textlengthmark}{s}{\r{\textObardotlessj}}{i}{r}{ð}{\textscy}{\textsubring{r}}} \dicPos{adj}[2]\dicFlx{}[-4] \dicDirectTranslationCS{nepokřtěný}
\dicEntry[óskoraður] \dicTerm{ó··skor·|aður} \dicIPA{{ou}{\textlengthmark}{s}{\r{g}}{\textopeno}{r}{a}{ð}{\textscy}{\textsubring{r}}} \dicPos{adj}[3] \dicFlx{(f ‑uð)}[3] \dicSynonym{algjör} \dicDirectTranslationCS{neomezený, neohraničený} \dicExampleIS{full og óskoruð yfirráð} \dicExampleCS{plná a~ničím neomezená dominance}
\dicEntry[óskráður] \dicTerm{ó··skráður} \dicIPA{{ou}{\textlengthmark}{s}{\r{g}}{r}{au}{ð}{\textscy}{\textsubring{r}}} \dicPos{adj}[2]\dicFlx{}[-13] \textbf{1.} \dicDirectTranslationCS{nenapsaný, nepsaný} \dicExampleIS{óskráð lög} \dicExampleCS{nepsaný zákon}  \textbf{2.} \dicDirectTranslationCS{nezapsaný, nezaregistrovaný}
\dicEntry[óskrifandi] \dicTerm{ó··skrif·andi} \dicIPA{{ou}{\textlengthmark}{s}{\r{g}}{r}{\textsci}{v}{a}{n}{\textsubring{d}}{\textsci}} \dicPos{adj}[13] \dicFlx{indecl}[1] \textbf{1.} \dicDirectTranslationCS{neumějící psát (negramotný člověk ap.)} \dicExampleIS{ólæs og óskrifandi} \dicExampleCS{neumějící číst a~psát}  \textbf{2.} \dicDirectTranslationCS{neumějící psát (novinář ap.)}
\dicEntry[óskundi] \dicTerm{ó··skund|i} \dicIPA{{ou}{\textlengthmark}{s}{\r{g}}{\textscy}{n}{\textsubring{d}}{\textsci}} \dicPos{m}[1] \dicFlx{(‑a)}[3] \dicSynonym{skemmd} \dicDirectTranslationCS{spoušť, paseka};  \dicPhraseIS{gera óskunda} \dicDirectTranslationCS{nadělat spoušť}
\dicEntry[óskyldur] \dicTerm{ó··skyldur} \dicIPA{{ou}{\textlengthmark}{s}{\r{\textObardotlessj}}{\textsci}{l}{\textsubring{d}}{\textscy}{\textsubring{r}}} \dicPos{adj}[2]\dicFlx{}[-14] \textbf{1.} \dicSynonym{vandalaus} \dicDirectTranslationCS{nepříbuzný (osoba, jazyk ap.)} \dicExampleIS{Hann er óskyldur mér.} \dicExampleCS{Není se mnou příbuzný.} \dicAntonym{skyldur}  \textbf{2.} \dicDirectTranslationCS{nesouvisející, netýkající se} \dicExampleIS{Mér er það alveg óskylt mál.} \dicExampleCS{To se mě vůbec netýká.}
\dicEntry[óskynsamlegur] \dicTerm{ó··skyn·sam·legur} \dicIPA{{ou}{\textlengthmark}{s}{\r{\textObardotlessj}}{\textsci}{n}{s}{a}{m}{l}{\textepsilon}{\textbabygamma}{\textscy}{\textsubring{r}}} \dicPos{adj}[1]\dicFlx{}[-8] \dicDirectTranslationCS{nerozumný, neuvážený, pošetilý} \dicAntonym{skynsamlegur}
\dicEntry[óskýr] \dicTerm{ó··skýr} \dicIPA{{ou}{\textlengthmark}{s}{\r{\textObardotlessj}}{i}{\textsubring{r}}} \dicPos{adj}[5] \dicFlx{(f ‑)}[8] \dicSynonym{ógreinilegur} \dicDirectTranslationCS{nejasný, nezřetelný} \dicExampleIS{Myndin er óskýr.} \dicExampleCS{Obrázek není zřetelný.} \dicAntonym{skýr};  \dicPhraseIS{vera óskýr í hugsun} \dicDirectTranslationCS{nemít jasné myšlení} \dicIndirectTranslationCS{(z~důvodu šoku, nehody ap.)}
\dicEntry[óskýrleiki] \dicTerm{ó··skýr·leik|i} \dicIPA{{ou}{\textlengthmark}{s}{\r{\textObardotlessj}}{i}{r}{l}{ei}{\r{\textObardotlessj}}{\textsci}} \dicPos{m}[1] \dicFlx{(‑a)}[3] \dicDirectTranslationCS{nejasnost, nezřetelnost}
\dicEntry[óskýrt] \dicTerm{ó··skýrt} \dicIPA{{ou}{\textlengthmark}{s}{\r{\textObardotlessj}}{i}{\textsubring{r}}{\textsubring{d}}} \dicPos{adv} \dicDirectTranslationCS{nezřetelně, nesrozumitelně}
\dicEntry[ósköp] \dicTerm{ó··sköp\smash{\textsuperscript{1}}} \dicsymFrequent\  \dicIPA{{ou}{\textlengthmark}{s}{\r{g}}{\oe}{\textsubring{b}}} \dicPos{n}[2] \dicFlx{pl}[9] \textbf{1.} \dicSynonym{óhamingja} \dicDirectTranslationCS{neštěstí, rána osudu}  \textbf{2.} \dicSynonym{býsn} \dicDirectTranslationCS{hrůza, děs} \dicIndirectTranslationCS{(o~něčem strašném nebo nudném)} \dicExampleIS{ósköp eru að heyra} \dicExampleCS{hrůza to slyšet}  \textbf{3.} \dicSynonym{kynstur} \dicDirectTranslationCS{spousta, moc} \dicExampleIS{ósköp af e‑u} \dicExampleCS{spousta (čeho)}  \textbf{4.} \dicSynonym{læti} \dicDirectTranslationCS{rámus, povyk};  \dicPhraseIS{gera ósköp úr e‑u} \dicDirectTranslationCS{dělat kvůli (čemu) povyk};  \dicPhraseIS{hvar\,/\addthin hvernig í ósköpunum} \dicDirectTranslationCS{sakra, safra} \dicIndirectTranslationCS{(k~zdůraznění údivu v~tázacích větách)} \dicExampleIS{Hvernig í ósköpunum gat þetta gerst?} \dicExampleCS{Jak se to sakra mohlo stát?}
\dicEntry[ósköp] \dicTerm{ó··sköp\smash{\textsuperscript{2}}} \dicIPA{{ou}{\textlengthmark}{s}{\r{g}}{\oe}{\textsubring{b}}} \dicPos{adv} \dicDirectTranslationCS{strašně, hrozně} \dicIndirectTranslationCS{(používané k~zdůraznění)}
\dicEntry[ósléttur] \dicTerm{ó··sléttur} \dicIPA{{ou}{\textlengthmark}{s}{\textsubring{d}}{l}{j}{\textepsilon}{h}{\textsubring{d}}{\textscy}{\textsubring{r}}} \dicPos{adj}[1]\dicFlx{}[-10] \dicSynonym{ójafn} \dicDirectTranslationCS{nerovný, hrbolatý (povrch ap.)} \dicAntonym{sléttur}
\dicEntry[óslitinn] \dicTerm{ó··slitinn} \dicIPA{{ou}{\textlengthmark}{s}{\textsubring{d}}{l}{\textsci}{\textsubring{d}}{\textsci}{\textsubring{n}}} \dicPos{adj}[6]\dicFlx{}[-6] \textbf{1.} \dicSynonym*{ósjúskaður} \dicDirectTranslationCS{neobnošený, neotrhaný} \dicExampleIS{Jakkinn er óslitinn.} \dicExampleCS{Bunda není obnošená.}  \textbf{2.} \dicSynonym{samfelldur} \dicDirectTranslationCS{nepřetržitý, souvislý} \dicExampleIS{óslitin rigning} \dicExampleCS{nepřetržitý déšť}
\dicEntry[Ósló] \dicTerm{Ósló} \dicIPA{{ou}{s}{l}{ou}} \dicPos{f}[4] \dicFlx{(‑ar)}[4] \dicFieldCat{geog.} \dicDirectTranslationCS{Oslo} \dicIndirectTranslationCS{(hlavní město Norska)}
\dicEntry[óslökkvandi] \dicTerm{ó··slökkv·andi} \dicIPA{{ou}{\textlengthmark}{s}{\textsubring{d}}{l}{\oe}{h}{\r{g}}{v}{a}{n}{\textsubring{d}}{\textsci}} \dicPos{adj}[13] \dicFlx{indecl}[1] \dicDirectTranslationCS{neuhasitelný, nehasnoucí}
\dicEntry[ósmekklegur] \dicTerm{ó··smekk·legur} \dicIPA{{ou}{\textlengthmark}{s}{m}{\textepsilon}{h}{\r{g}}{l}{\textepsilon}{\textbabygamma}{\textscy}{\textsubring{r}}} \dicPos{adj}[1]\dicFlx{}[-8] \dicSynonym{smekklaus} \dicDirectTranslationCS{nevkusný, nevhodný} \dicExampleIS{ósmekkleg athugasemd} \dicExampleCS{nevhodná poznámka}
\dicEntry[ósnortinn] \dicTerm{ó··snortinn} \dicIPA{{ou}{\textlengthmark}{s}{\textsubring{d}}{n}{\textopeno}{\textsubring{r}}{\textsubring{d}}{\textsci}{\textsubring{n}}} \dicPos{adj}[6]\dicFlx{}[-6] \textbf{1.} \dicSynonym*{meylegur} \dicDirectTranslationCS{nedotčený, netknutý, panenský} \dicExampleIS{ósnortin náttúra} \dicExampleCS{panenská příroda}  \textbf{2.} \dicSynonym*{óhreyfður} \dicDirectTranslationCS{nepohnutý, nedotčený} \dicIndirectTranslationCS{(citově)};  \dicPhraseIS{e‑að lætur engan ósnortinn} \dicLangCat{přen.} \dicDirectTranslationCS{(co) nenechává nikoho chladným}
\dicEntry[ósnotur] \dicTerm{ó··snotur} \dicIPA{{ou}{\textlengthmark}{s}{\textsubring{d}}{n}{\textopeno}{\textsubring{d}}{\textscy}{\textsubring{r}}} \dicPos{adj}[9]\dicFlx{}[-1] \dicDirectTranslationCS{nevzhledný, nepohledný} \dicAntonym{snotur}
\dicEntry[ósoðinn] \dicTerm{ó··soðinn} \dicIPA{{ou}{\textlengthmark}{s}{\textopeno}{ð}{\textsci}{\textsubring{n}}} \dicPos{adj}[6]\dicFlx{}[-6] \dicSynonym{hrár} \dicDirectTranslationCS{nevařený, syrový} \dicAntonym{soðinn}
\dicEntry[óson] \dicTerm{óson} \dicIPA{{ou}{\textlengthmark}{s}{\textopeno}{\textsubring{n}}} \dicPos{n}[2] \dicFlx{(‑s)}[2] \dicFieldCat{chem.} \dicDirectTranslationCS{ozón}
\dicEntry[ósonlag] \dicTerm{óson··lag} \dicIPA{{ou}{\textlengthmark}{s}{\textopeno}{n}{l}{a}{x}} \dicPos{n}[2] \dicFlx{(‑s)}[2] \dicFieldCat{chem.} \dicDirectTranslationCS{ozónová vrstva}
\dicEntry[ósómi] \dicTerm{ó··sóm|i} \dicIPA{{ou}{\textlengthmark}{s}{ou}{m}{\textsci}} \dicPos{m}[1] \dicFlx{(‑a)}[3] \dicSynonym*{skömm} \dicDirectTranslationCS{nezvyk, neslušnost, neuctivost} \dicAntonym{sómi}
\dicEntry[óspar] \dicTerm{ó··|spar} \dicIPA{{ou}{\textlengthmark}{s}{\textsubring{b}}{a}{\textsubring{r}}} \dicPos{adj}[5] \dicFlx{(f ‑spör)}[9] \dicDirectTranslationCS{štědrý, nešetřící, přející} \dicExampleIS{óspar á e‑ð} \dicExampleCS{nešetřící (čím)}
\dicEntry[óspilltur] \dicTerm{ó··spilltur} \dicIPA{{ou}{\textlengthmark}{s}{\textsubring{b}}{\textsci}{\textsubring{l}}{\textsubring{d}}{\textscy}{\textsubring{r}}} \dicPos{adj}[1]\dicFlx{}[-10] \textbf{1.} \dicSynonym{ómengaður} \dicDirectTranslationCS{nezkažený, nedotčený} \dicAntonym{spilltur}  \textbf{2.} \dicDirectTranslationCS{nezkorumpovaný} \dicExampleIS{óspilltir embættismenn} \dicExampleCS{nezkorumpovaní úředníci}
\dicEntry[óspurður] \dicTerm{ó··spurður} \dicIPA{{ou}{\textlengthmark}{s}{\textsubring{b}}{\textscy}{r}{ð}{\textscy}{\textsubring{r}}} \dicPos{adj}[2]\dicFlx{}[-4] \dicPhraseIS{í óspurðum fréttum} \dicFlx{adv} \dicDirectTranslationCS{bez tázání, spontánně}
\dicEntry[óstaðfestur] \dicTerm{ó··stað·festur} \dicIPA{{ou}{\textlengthmark}{s}{\textsubring{d}}{a}{ð}{f}{\textepsilon}{s}{\textsubring{d}}{\textscy}{\textsubring{r}}} \dicPos{adj}[1]\dicFlx{}[-13] \dicSynonym{ístöðulaus} \dicDirectTranslationCS{nepotvrzený, nestvrzený} \dicExampleIS{óstaðfestur samningur} \dicExampleCS{nepotvrzená smlouva} \dicAntonym{staðfastur}
\dicEntry[óstarfhæfur] \dicTerm{ó··starf·hæfur} \dicIPA{{ou}{\textlengthmark}{s}{\textsubring{d}}{a}{r}{f}{h}{a}{i}{v}{\textscy}{\textsubring{r}}} \dicPos{adj}[1]\dicFlx{}[-1] \textbf{1.} \dicDirectTranslationCS{nezpůsobilý (k~práci ap.)}  \textbf{2.} \dicDirectTranslationCS{nepracující, (jsoucí) mimo provoz (o~strojích ap.)}
\dicEntry[óstilltur] \dicTerm{ó··stilltur} \dicIPA{{ou}{\textlengthmark}{s}{\textsubring{d}}{\textsci}{\textsubring{l}}{\textsubring{d}}{\textscy}{\textsubring{r}}} \dicPos{adj}[1]\dicFlx{}[-10] \textbf{1.} \dicSynonym{órólegur} \dicDirectTranslationCS{neklidný, nepokojný} \dicAntonym{stilltur}  \textbf{2.} \dicDirectTranslationCS{nenaladěný} \dicExampleIS{óstillt hljóðfæri} \dicExampleCS{nenaladěný hudební nástroj}
\dicEntry[óstjórn] \dicTerm{ó··stjórn} \dicIPA{{ou}{\textlengthmark}{s}{\textsubring{d}}{j}{ou}{r}{\textsubring{d}}{\textsubring{n}}} \dicPos{f}[7] \dicFlx{(‑ar)}[3] \dicSynonym{stjórnleysi} \dicDirectTranslationCS{bezvládí, anarchie}
\dicEntry[óstjórnlegur] \dicTerm{ó··stjórn·legur} \dicIPA{{ou}{\textlengthmark}{s}{\textsubring{d}}{j}{ou}{r}{\textsubring{d}}{n}{l}{\textepsilon}{\textbabygamma}{\textscy}{\textsubring{r}}} \dicPos{adj}[1]\dicFlx{}[-8] \dicSynonym{ofsafenginn} \dicDirectTranslationCS{neovladatelný, nekontrolovatelný} \dicExampleIS{óstjórnleg löngun} \dicExampleCS{neovladatelná touha}
\dicEntry[óstuddur] \dicTerm{ó··studdur} \dicIPA{{ou}{\textlengthmark}{s}{\textsubring{d}}{\textscy}{\textsubring{d}}{\textscy}{\textsubring{r}}} \dicPos{adj}[2]\dicFlx{}[-18] \dicDirectTranslationCS{nepodpořený, nepodporovaný, (jsoucí) bez podpory}
\dicEntry[óstyrkur] \dicTerm{ó··styrkur} \dicIPA{{ou}{\textlengthmark}{s}{\textsubring{d}}{\textsci}{\textsubring{r}}{\r{g}}{\textscy}{\textsubring{r}}} \dicPos{adj}[1]\dicFlx{}[-1] \textbf{1.} \dicSynonym{óhraustur} \dicDirectTranslationCS{nejistý, vratký} \dicExampleIS{óstyrkur í göngulagi} \dicExampleCS{vratký při chůzi} \dicAntonym{styrkur\smash{\textsuperscript{1}}}  \textbf{2.} \dicSynonym{óöruggur} \dicDirectTranslationCS{nejistý, rozpačitý, nervózní}
\dicEntry[óstýrilátur] \dicTerm{ó··stýri·látur} \dicIPA{{ou}{\textlengthmark}{s}{\textsubring{d}}{i}{r}{\textsci}{l}{au}{\textsubring{d}}{\textscy}{\textsubring{r}}} \dicPos{adj}[1]\dicFlx{}[-1] \dicSynonym{óþekkur} \dicDirectTranslationCS{nezvladatelný, neovladatelný} \dicExampleIS{óstýrilátur nemi} \dicExampleCS{nezvladatelný žák}
\dicEntry[óstýrilæti] \dicTerm{ó··stýri·læti} \dicIPA{{ou}{\textlengthmark}{s}{\textsubring{d}}{i}{r}{\textsci}{l}{a}{i}{\textsubring{d}}{\textsci}} \dicPos{n}[2] \dicFlx{(‑s)}[20] \dicDirectTranslationCS{nezvladatelnost, neovladatelnost}
\dicEntry[óstöðugleiki] \dicTerm{ó··stöðug·leik|i} \dicIPA{{ou}{\textlengthmark}{s}{\textsubring{d}}{\oe}{ð}{\textscy}{\textbabygamma}{l}{ei}{\r{\textObardotlessj}}{\textsci}} \dicPos{m}[1] \dicFlx{(‑a)}[3] \dicDirectTranslationCS{nestálost, nestabilita}
\dicEntry[óstöðugur] \dicTerm{ó··stöð·ugur} \dicIPA{{ou}{\textlengthmark}{s}{\textsubring{d}}{\oe}{ð}{\textscy}{\textbabygamma}{\textscy}{\textsubring{r}}} \dicPos{adj}[1]\dicFlx{}[-8] \dicSynonym{valtur} \dicDirectTranslationCS{nestálý, nestabilní, vratký} \dicAntonym{stöðugur}
\dicEntry[óstöðvandi] \dicTerm{ó··stöðv·andi} \dicIPA{{ou}{\textlengthmark}{s}{\textsubring{d}}{\oe}{ð}{v}{a}{n}{\textsubring{d}}{\textsci}} \dicPos{adj}[13] \dicFlx{indecl}[1] \dicDirectTranslationCS{nezastavitelný, nepotlačitelný} \dicExampleIS{óstöðvandi hlátur} \dicExampleCS{nezastavitelný smích}
\dicEntry[ósveigjanlegur] \dicTerm{ó··sveigjan·legur} \dicIPA{{ou}{\textlengthmark}{s}{v}{ei}{j}{a}{n}{l}{\textepsilon}{\textbabygamma}{\textscy}{\textsubring{r}}} \dicPos{adj}[1]\dicFlx{}[-8] \textbf{1.} \dicDirectTranslationCS{neohebný, nepoddajný (materiál ap.)} \dicAntonym{sveigjanlegur}  \textbf{2.} \dicDirectTranslationCS{neúprosný, nekompromisní}
\dicEntry[ósvikinn] \dicTerm{ó··svikinn} \dicIPA{{ou}{\textlengthmark}{s}{v}{\textsci}{\r{\textObardotlessj}}{\textsci}{\textsubring{n}}} \dicPos{adj}[6]\dicFlx{}[-6] \dicSynonym{sannur\smash{\textsuperscript{1}}} \dicDirectTranslationCS{pravý, skutečný, nefalšovaný, autentický} \dicAntonym{svikinn}
\dicEntry[ósvinna] \dicTerm{ó··svinn|a} \dicIPA{{ou}{\textlengthmark}{s}{v}{\textsci}{n}{a}} \dicPos{f}[1] \dicFlx{(‑u)}[5] \textbf{1.} \dicSynonym{fásinna} \dicDirectTranslationCS{nerozumnost, hloupost}  \textbf{2.} \dicSynonym{ruddaskapur} \dicDirectTranslationCS{neomalenost, hrubost}  \textbf{3.} \dicSynonym{óhæfa} \dicDirectTranslationCS{nepatřičnost, netaktnost}
\dicEntry[ósvífinn] \dicTerm{ó··svífinn} \dicIPA{{ou}{\textlengthmark}{s}{v}{i}{v}{\textsci}{\textsubring{n}}} \dicPos{adj}[6]\dicFlx{}[-2] \dicSynonym{óskammfeilinn} \dicDirectTranslationCS{drzý, nestydatý}
\dicEntry[ósvífni] \dicTerm{ó··svífn|i} \dicIPA{{ou}{\textlengthmark}{s}{v}{i}{\textsubring{b}}{n}{\textsci}} \dicPos{f}[3] \dicFlx{(‑i)}[3] \dicSynonym*{ófyrirleitni} \dicDirectTranslationCS{drzost, nestydatost}
\dicEntry[ósyndur] \dicTerm{ó··syndur} \dicIPA{{ou}{\textlengthmark}{s}{\textsci}{n}{\textsubring{d}}{\textscy}{\textsubring{r}}} \dicPos{adj}[2]\dicFlx{}[-14] \dicDirectTranslationCS{neumějící plavat} \dicExampleIS{Hann er ósyndur.} \dicExampleCS{Je neplavec.}
\dicEntry[ósýnilegur] \dicTerm{ó··sýni·legur} \dicsymFrequent\  \dicIPA{{ou}{\textlengthmark}{s}{i}{n}{\textsci}{l}{\textepsilon}{\textbabygamma}{\textscy}{\textsubring{r}}} \dicPos{adj}[1]\dicFlx{}[-8] \dicSynonym{hulinn} \dicDirectTranslationCS{neviditelný} \dicExampleIS{ósýnileg bönd} \dicExampleCS{neviditelná pouta} \dicAntonym{sýnilegur}
\dicEntry[ósæð] \dicTerm{ós··æð} \dicIPA{{ou}{\textlengthmark}{s}{a}{i}{\texttheta}} \dicPos{f}[4] \dicFlx{(‑ar, ‑ar)}[1] \dicFieldCat{anat.} \dicDirectTranslationCS{aorta, srdečnice}
\dicEntry[ósæmilegur] \dicTerm{ó··sæmi·legur} \dicIPA{{ou}{\textlengthmark}{s}{a}{i}{m}{\textsci}{l}{\textepsilon}{\textbabygamma}{\textscy}{\textsubring{r}}} \dicPos{adj}[1]\dicFlx{}[-8] \dicSynonym{óviðeigandi} \dicDirectTranslationCS{nevhodný, nepatřičný, nemístný}
\dicEntry[ósættanlegur] \dicTerm{ó··sættan·legur} \dicIPA{{ou}{\textlengthmark}{s}{a}{i}{h}{\textsubring{d}}{a}{n}{l}{\textepsilon}{\textbabygamma}{\textscy}{\textsubring{r}}} \dicPos{adj}[1]\dicFlx{}[-8] \dicDirectTranslationCS{nesmiřitelný, zarytý, zatvrzelý}
\dicEntry[ótakmarkaður] \dicTerm{ó··tak·|mark·aður} \dicIPA{{ou}{\textlengthmark}{t\smash{\textsuperscript{h}}}{a}{\r{g}}{m}{a}{\textsubring{r}}{\r{g}}{a}{ð}{\textscy}{\textsubring{r}}} \dicPos{adj}[3] \dicFlx{(f ‑mörkuð)}[1] \dicSynonym{takmarkalaus} \dicDirectTranslationCS{neomezený, neohraničený, nelimitovaný} \dicAntonym{takmarkaður}
\dicEntry[ótal] \dicTerm{ó··tal\smash{\textsuperscript{1}}} \dicIPA{{ou}{\textlengthmark}{t\smash{\textsuperscript{h}}}{a}{\textsubring{l}}} \dicPos{n}[2] \dicFlx{(‑s)}[2] \dicSynonym{fjöldi} \dicDirectTranslationCS{nespočet, spousta, masa}
\dicEntry[ótal] \dicTerm{ó··tal\smash{\textsuperscript{2}}} \dicsymFrequent\  \dicIPA{{ou}{\textlengthmark}{t\smash{\textsuperscript{h}}}{a}{\textsubring{l}}} \dicPos{adj}[13] \dicFlx{indecl}[1] \dicDirectTranslationCS{nesčetný, nespočetný} \dicExampleIS{ótal tegundir ávaxta} \dicExampleCS{nesčetné druhy ovoce}
\dicEntry[ótal] \dicTerm{ó··tal\smash{\textsuperscript{3}}} \dicIPA{{ou}{\textlengthmark}{t\smash{\textsuperscript{h}}}{a}{\textsubring{l}}} \dicPos{adv} \dicDirectTranslationCS{nesčetně, nespočetně} \dicExampleIS{ótal sinnum} \dicExampleCS{nesčetněkrát}
\dicEntry[ótalinn] \dicTerm{ó··talinn} \dicIPA{{ou}{\textlengthmark}{t\smash{\textsuperscript{h}}}{a}{l}{\textsci}{\textsubring{n}}} \dicPos{adj}[6]\dicFlx{}[-9] \dicDirectTranslationCS{nezapočtený, nevyčíslený} \dicExampleIS{ótalinn kostnaður} \dicExampleCS{nezapočtené náklady}
\dicEntry[ótaminn] \dicTerm{ó··taminn} \dicIPA{{ou}{\textlengthmark}{t\smash{\textsuperscript{h}}}{a}{m}{\textsci}{\textsubring{n}}} \dicPos{adj}[6]\dicFlx{}[-9] \dicDirectTranslationCS{nezkrocený, neochočený}
\dicEntry[óteljandi] \dicTerm{ó··telj·andi} \dicIPA{{ou}{\textlengthmark}{t\smash{\textsuperscript{h}}}{\textepsilon}{l}{j}{a}{n}{\textsubring{d}}{\textsci}} \dicPos{adj}[13] \dicFlx{indecl}[1] \dicSynonym*{ótalmargur} \dicDirectTranslationCS{nesčetný, nespočetný}
\dicEntry[ótilgreindur] \dicTerm{ó··til·greindur} \dicIPA{{ou}{\textlengthmark}{t\smash{\textsuperscript{h}}}{\textsci}{l}{\r{g}}{r}{ei}{n}{\textsubring{d}}{\textscy}{\textsubring{r}}} \dicPos{adj}[2]\dicFlx{}[-14] \dicSynonym{ónefndur} \dicDirectTranslationCS{neuvedený, neupřesněný, nespecifikovaný}
\dicEntry[ótilhlýðilegur] \dicTerm{ó··til·hlýði·legur} \dicIPA{{ou}{\textlengthmark}{t\smash{\textsuperscript{h}}}{\textsci}{\textsubring{l}}{i}{ð}{\textsci}{l}{\textepsilon}{\textbabygamma}{\textscy}{\textsubring{r}}} \dicPos{adj}[1]\dicFlx{}[-8] \dicSynonym{óhæfilegur} \dicDirectTranslationCS{nevhodný, nepatřičný} \dicAntonym{tilhlýðilegur}
\dicEntry[ótilkvaddur] \dicTerm{ó··til·|kvaddur} \dicIPA{{ou}{\textlengthmark}{t\smash{\textsuperscript{h}}}{\textsci}{l}{k\smash{\textsuperscript{h}}}{v}{a}{\textsubring{d}}{\textscy}{\textsubring{r}}} \dicPos{adj}[2] \dicFlx{(f ‑kvödd)}[20] \dicSynonym{óboðinn} \dicDirectTranslationCS{nevyžádaný, nevyzvaný}
\dicEntry[ótilneyddur] \dicTerm{ó··til·neyddur} \dicIPA{{ou}{\textlengthmark}{t\smash{\textsuperscript{h}}}{\textsci}{l}{n}{ei}{\textsubring{d}}{\textscy}{\textsubring{r}}} \dicPos{adj}[2]\dicFlx{}[-21] \dicSynonym{sjálfviljugur} \dicDirectTranslationCS{nenucený, nepřinucený, dobrovolný} \dicAntonym{tilneyddur}
\dicEntry[ótiltekinn] \dicTerm{ó··til·tekinn} \dicIPA{{ou}{\textlengthmark}{t\smash{\textsuperscript{h}}}{\textsci}{l}{t\smash{\textsuperscript{h}}}{\textepsilon}{\r{\textObardotlessj}}{\textsci}{\textsubring{n}}} \dicPos{adj}[6]\dicFlx{}[-6] \dicSynonym{óákveðinn} \dicDirectTranslationCS{neurčený, nestanovený, nespecifikovaný} \dicAntonym{tiltekinn}
\dicEntry[ótíð] \dicTerm{ó··tíð} \dicIPA{{ou}{\textlengthmark}{t\smash{\textsuperscript{h}}}{i}{\texttheta}} \dicPos{f}[7] \dicFlx{(‑ar)}[3] \dicDirectTranslationCS{špatné počasí}
\dicEntry[ótíðindi] \dicTerm{ó··tíð·indi} \dicIPA{{ou}{\textlengthmark}{t\smash{\textsuperscript{h}}}{i}{ð}{\textsci}{n}{\textsubring{d}}{\textsci}} \dicPos{n}[2] \dicFlx{pl}[19] \dicSynonym*{ófagnaður} \dicDirectTranslationCS{špatné zprávy\,/\addthin noviny}
\dicEntry[ótímabær] \dicTerm{ó··tíma·bær} \dicIPA{{ou}{\textlengthmark}{t\smash{\textsuperscript{h}}}{i}{m}{a}{\textsubring{b}}{a}{i}{\textsubring{r}}} \dicPos{adj}[5] \dicFlx{(f ‑)}[8] \dicSynonym{óviðeigandi} \dicDirectTranslationCS{předčasný} \dicExampleIS{ótímabær dauði} \dicExampleCS{předčasná smrt} \dicAntonym{tímabær}
\dicEntry[ótíndur] \dicTerm{ó··tíndur} \dicIPA{{ou}{\textlengthmark}{t\smash{\textsuperscript{h}}}{i}{n}{\textsubring{d}}{\textscy}{\textsubring{r}}} \dicPos{adj}[2]\dicFlx{}[-17] \dicSynonym*{óvalinn} \dicDirectTranslationCS{pravý, ryzí} \dicExampleIS{ótíndur glæpamaður} \dicExampleCS{ryzí zločinec}
\dicEntry[ótrauður] \dicTerm{ó··trauður} \dicIPA{{ou}{\textlengthmark}{t\smash{\textsuperscript{h}}}{r}{\oe i}{ð}{\textscy}{\textsubring{r}}} \dicPos{adj}[2]\dicFlx{}[-6] \textbf{1.} \dicSynonym{djarfur} \dicDirectTranslationCS{neohrožený, smělý} \dicAntonym{trauður}  \textbf{2.} \dicSynonym{fús} \dicDirectTranslationCS{odhodlaný, ochotný}
\dicEntry[ótraustur] \dicTerm{ó··traustur} \dicIPA{{ou}{\textlengthmark}{t\smash{\textsuperscript{h}}}{r}{\oe i}{s}{\textsubring{d}}{\textscy}{\textsubring{r}}} \dicPos{adj}[1]\dicFlx{}[-10] \dicSynonym{ótryggur} \dicDirectTranslationCS{nespolehlivý, nejistý} \dicAntonym{traustur}
\dicEntry[ótrú] \dicTerm{ó··trú} \dicIPA{{ou}{\textlengthmark}{t\smash{\textsuperscript{h}}}{r}{u}} \dicPos{f}[4] \dicFlx{(‑ar)}[27] \dicSynonym{vantraust} \dicDirectTranslationCS{nedůvěra, skepse, nevíra} \dicExampleIS{hafa ótrú á e‑u} \dicExampleCS{mít v~(co) nedůvěru}
\dicEntry[ótrúlega] \dicTerm{ó··trú·lega} \dicsymFrequent\  \dicIPA{{ou}{\textlengthmark}{t\smash{\textsuperscript{h}}}{r}{u}{l}{\textepsilon}{\textbabygamma}{a}} \dicPos{adv} \dicDirectTranslationCS{neuvěřitelně, nevěrohodně} \dicExampleIS{ótrúlega sterkur} \dicExampleCS{neuvěřitelně silný}
\dicEntry[ótrúlegur] \dicTerm{ó··trú·legur} \dicsymFrequent\  \dicIPA{{ou}{\textlengthmark}{t\smash{\textsuperscript{h}}}{r}{u}{l}{\textepsilon}{\textbabygamma}{\textscy}{\textsubring{r}}} \dicPos{adj}[1]\dicFlx{}[-8] \dicSynonym{furðulegur} \dicDirectTranslationCS{neuvěřitelný, nevěrohodný} \dicExampleIS{ótrúlegur sigur} \dicExampleCS{neuvěřitelné vítězství} \dicAntonym{trúlegur}
\dicEntry[ótrúr] \dicTerm{ó··trúr} \dicIPA{{ou}{\textlengthmark}{t\smash{\textsuperscript{h}}}{r}{u}{\textsubring{r}}} \dicPos{adj}[4]\dicFlx{}[-1] \dicSynonym{ótryggur} \dicDirectTranslationCS{nevěrný, neloajální} \dicExampleIS{vera húsbónda sínum ótrúr} \dicExampleCS{být neloajální ke svému hospodáři}
\dicEntry[ótryggð] \dicTerm{ó··tryggð} \dicIPA{{ou}{\textlengthmark}{t\smash{\textsuperscript{h}}}{r}{\textsci}{\textbabygamma}{\texttheta}} \dicPos{f}[7] \dicFlx{(‑ar)}[3] \dicDirectTranslationCS{nevěrnost, nevěra} \dicAntonym{tryggð}
\dicEntry[ótryggður] \dicTerm{ó··tryggður} \dicIPA{{ou}{\textlengthmark}{t\smash{\textsuperscript{h}}}{r}{\textsci}{\textbabygamma}{ð}{\textscy}{\textsubring{r}}} \dicPos{adj}[2]\dicFlx{}[-4] \dicDirectTranslationCS{nepojištěný}
\dicEntry[ótryggur] \dicTerm{ó··tryggur} \dicIPA{{ou}{\textlengthmark}{t\smash{\textsuperscript{h}}}{r}{\textsci}{\r{g}}{\textscy}{\textsubring{r}}} \dicPos{adj}[1]\dicFlx{}[-1] \dicSynonym{ótrúr} \dicDirectTranslationCS{nespolehlivý, nejistý} \dicExampleIS{ótryggur ís} \dicExampleCS{nespolehlivý led} \dicAntonym{tryggur}
\dicEntry[ótt] \dicTerm{ótt} \dicsymFrequent\  \dicIPA{{ou}{h}{\textsubring{d}}} \dicPos{adv} \dicFlx{(comp óðar\,/\addthin óðara, sup óðast)} \dicSynonym{hratt\smash{\textsuperscript{2}}} \dicDirectTranslationCS{rychle, chvatně} \dicExampleIS{Hjartað berst ótt í mér.} \dicExampleCS{Srdce mi rychle bije. };  \dicPhraseIS{ótt og títt} \dicFlx{adv} \dicDirectTranslationCS{ostošest, rychle a~často}
\dicEntry[óttalaus] \dicTerm{ótta··laus} \dicIPA{{ou}{h}{\textsubring{d}}{a}{l}{\oe i}{s}} \dicPos{adj}[5]\dicFlx{}[-1] \dicSynonym{djarfur} \dicDirectTranslationCS{nebojácný, neohrožený}
\dicEntry[óttalegur] \dicTerm{ótta··legur} \dicIPA{{ou}{h}{\textsubring{d}}{a}{l}{\textepsilon}{\textbabygamma}{\textscy}{\textsubring{r}}} \dicPos{adj}[1]\dicFlx{}[-8] \textbf{1.} \dicSynonym{hræðilegur} \dicDirectTranslationCS{strašlivý, děsivý}  \textbf{2.} \dicSynonym*{afarmikill} \dicDirectTranslationCS{příšerný, děsný (břídil ap.)} \dicExampleIS{Hann er óttalegur bjáni.} \dicExampleCS{Je to děsný hlupák.}
\dicEntry[óttaleysi] \dicTerm{ótta··leysi} \dicIPA{{ou}{h}{\textsubring{d}}{a}{l}{ei}{s}{\textsci}} \dicPos{n}[2] \dicFlx{(‑s)}[20] \dicDirectTranslationCS{nebojácnost, neohroženost}
\dicEntry[óttasleginn] \dicTerm{ótta··sleginn} \dicsymFrequent\  \dicIPA{{ou}{h}{\textsubring{d}}{a}{s}{\textsubring{d}}{l}{ei}{\textsci}{\textsubring{n}}} \dicPos{adj}[6]\dicFlx{}[-2] \dicSynonym{hræddur} \dicDirectTranslationCS{vystrašený, vyděšený} \dicExampleIS{óttaslegin augu} \dicExampleCS{vystrašené oči}
\dicEntry[óttast] \dicTerm{ótt|ast} \dicsymFrequent\  \dicIPA{{ou}{h}{\textsubring{d}}{a}{s}{\textsubring{d}}} \dicPos{v}[1] \dicFlx{(‑aðist)}[94] \dicFlx{refl} \dicSynonym*{kvíða fyrir} \dicDirectTranslationCS{mít strach, bát se, obávat se, strachovat se};  \dicPhraseIS{óttast e‑ð} \dicDirectTranslationCS{bát se (čeho), obávat se (čeho)} \dicExampleIS{óttast árás} \dicExampleCS{bát se napadení};  \dicPhraseIS{óttast um e‑n} \dicDirectTranslationCS{dělat si starosti o~(koho), obávat se o~(koho)}
\dicEntry[ótti] \dicTerm{ótt|i} \dicsymFrequent\  \dicIPA{{ou}{h}{\textsubring{d}}{\textsci}} \dicPos{m}[1] \dicFlx{(‑a)}[3] \textbf{1.} \dicSynonym{hræðsla} \dicDirectTranslationCS{strach, bázeň} \dicExampleIS{ótti við dauðann} \dicExampleCS{strach ze smrti};  \dicPhraseIS{e‑m stendur ótti af e‑u} \dicFlx{impers} \dicDirectTranslationCS{(kdo) se bojí (čeho), (kdo) má strach z~(čeho)}  \textbf{2.} \dicSynonym{hætta\smash{\textsuperscript{1}}} \dicDirectTranslationCS{ohrožení, nebezpečí}
\dicEntry[ótukt] \dicTerm{ó··tukt} \dicIPA{{ou}{\textlengthmark}{t\smash{\textsuperscript{h}}}{\textscy}{x}{\textsubring{d}}} \dicPos{f}[7] \dicFlx{(‑ar, ‑ir)}[1] \dicSynonym{þorpari} \dicDirectTranslationCS{zlosyn(ka), darebák, darebačka, padouch, padouška}
\dicEntry[ótuktarlegur] \dicTerm{ó··tuktar·legur} \dicIPA{{ou}{\textlengthmark}{t\smash{\textsuperscript{h}}}{\textscy}{x}{\textsubring{d}}{a}{r}{l}{\textepsilon}{\textbabygamma}{\textscy}{\textsubring{r}}} \dicPos{adj}[1]\dicFlx{}[-8] \dicSynonym{rætinn} \dicDirectTranslationCS{zlotřilý, ničemný, podlý}
\dicEntry[ótvíræður] \dicTerm{ó··tví·ræður} \dicIPA{{ou}{\textlengthmark}{t\smash{\textsuperscript{h}}}{v}{i}{r}{a}{i}{ð}{\textscy}{\textsubring{r}}} \dicPos{adj}[2]\dicFlx{}[-6] \dicSynonym{tvímælalaus} \dicDirectTranslationCS{jednoznačný, jasný} \dicAntonym{tvíræður}
\dicEntry[ótækur] \dicTerm{ó··tækur} \dicIPA{{ou}{\textlengthmark}{t\smash{\textsuperscript{h}}}{a}{i}{\r{g}}{\textscy}{\textsubring{r}}} \dicPos{adj}[1]\dicFlx{}[-1] \dicSynonym{óhæfur} \dicDirectTranslationCS{nepřijatelný, neakceptovatelný} \dicAntonym{tækur}
\dicEntry[ótæmandi] \dicTerm{ó··tæm·andi} \dicIPA{{ou}{\textlengthmark}{t\smash{\textsuperscript{h}}}{a}{i}{m}{a}{n}{\textsubring{d}}{\textsci}} \dicPos{adj}[13] \dicFlx{indecl}[1] \dicDirectTranslationCS{nevyčerpatelný, bezedný, nepřeberný} \dicExampleIS{ótæmandi sjóður} \dicExampleCS{nevyčerpatelné bohatství} \dicAntonym{tæmandi}
\dicEntry[óumbreytanlegur] \dicTerm{ó··um·breytan·legur} \dicIPA{{ou}{\textlengthmark}{\textscy}{m}{\textsubring{b}}{r}{ei}{\textsubring{d}}{a}{n}{l}{\textepsilon}{\textbabygamma}{\textscy}{\textsubring{r}}} \dicPos{adj}[1]\dicFlx{}[-8] \dicSynonym{ó\-breyt\-an\-legur} \dicDirectTranslationCS{neměnný, stálý}
\dicEntry[óumdeilanlegur] \dicTerm{ó··um·deilan·legur} \dicIPA{{ou}{\textlengthmark}{\textscy}{m}{\textsubring{d}}{ei}{l}{a}{n}{l}{\textepsilon}{\textbabygamma}{\textscy}{\textsubring{r}}} \dicPos{adj}[1]\dicFlx{}[-8] \dicSynonym{vafalaus} \dicDirectTranslationCS{neoddiskutovatelný, nesporný} \dicAntonym{umdeilanlegur}
\dicEntry[óumdeildur] \dicTerm{ó··um·deildur} \dicIPA{{ou}{\textlengthmark}{\textscy}{m}{\textsubring{d}}{ei}{l}{\textsubring{d}}{\textscy}{\textsubring{r}}} \dicPos{adj}[2]\dicFlx{}[-14] \dicDirectTranslationCS{nesporný, neoddiskutovatelný}
\dicEntry[óumflýjanlegur] \dicTerm{ó··um·flýjan·legur} \dicIPA{{ou}{\textlengthmark}{\textscy}{m}{f}{l}{i}{j}{a}{n}{l}{\textepsilon}{\textbabygamma}{\textscy}{\textsubring{r}}} \dicPos{adj}[1]\dicFlx{}[-6] \dicSynonym{óhjákvæmilegur} \dicDirectTranslationCS{neodvratný, nezbytný, nevyhnutelný} \dicExampleIS{óumflýjanleg ráðstöfun} \dicExampleCS{nevyhnutelné opatření}
\dicEntry[óumræðilegur] \dicTerm{ó··um·ræði·legur} \dicIPA{{ou}{\textlengthmark}{\textscy}{m}{r}{a}{i}{ð}{\textsci}{l}{\textepsilon}{\textbabygamma}{\textscy}{\textsubring{r}}} \dicPos{adj}[1]\dicFlx{}[-6] \dicSynonym{ólýsanlegur} \dicDirectTranslationCS{nepopsatelný, nevylíčitelný} \dicExampleIS{óumræðileg gleði} \dicExampleCS{nepopsatelná radost}
\dicEntry[óundirbúinn] \dicTerm{ó··undir·búinn} \dicIPA{{ou}{\textlengthmark}{\textscy}{n}{\textsubring{d}}{\textsci}{r}{\textsubring{b}}{u}{\textsci}{\textsubring{n}}} \dicPos{adj}[6]\dicFlx{}[-6] \dicSynonym*{vanbúinn} \dicDirectTranslationCS{nepřipravený, nepřichystaný} \dicAntonym{undirbúinn}
\dicEntry[óunninn] \dicTerm{ó··unninn} \dicIPA{{ou}{\textlengthmark}{\textscy}{n}{\textsci}{\textsubring{n}}} \dicPos{adj}[6]\dicFlx{}[-16] \textbf{1.} \dicDirectTranslationCS{nezpracovaný, syrový, hrubý (materiál ap.)} \dicExampleIS{óunnið efni} \dicExampleCS{hrubý materiál}  \textbf{2.} \dicDirectTranslationCS{neudělaný, nevykonaný}
\dicEntry[óupplýstur] \dicTerm{ó··upp·lýstur} \dicIPA{{ou}{\textlengthmark}{\textscy}{h}{\textsubring{b}}{l}{i}{s}{\textsubring{d}}{\textscy}{\textsubring{r}}} \dicPos{adj}[1]\dicFlx{}[-10] \textbf{1.} \dicSynonym{ómenntaður} \dicDirectTranslationCS{neosvícený, nevzdělaný}  \textbf{2.} \dicDirectTranslationCS{neosvětlený} \dicExampleIS{óupplýstur vegur} \dicExampleCS{neosvětlená cesta}  \textbf{3.} \dicDirectTranslationCS{neobjasněný, nevyjasněný (kriminální případ ap.)} \dicExampleIS{óupplýst sakamál} \dicExampleCS{neobjasněný případ}
\dicEntry[óuppsegjanlegur] \dicTerm{ó··upp·segjan·legur} \dicIPA{{ou}{\textlengthmark}{\textscy}{h}{\textsubring{b}}{s}{ei}{j}{a}{n}{l}{\textepsilon}{\textbabygamma}{\textscy}{\textsubring{r}}} \dicPos{adj}[1]\dicFlx{}[-6] \dicDirectTranslationCS{nevypověditelný (smlouva ap.)}
\dicEntry[óútkljáður] \dicTerm{ó··út·kljáður} \dicIPA{{ou}{\textlengthmark}{u}{\textsubring{d}}{k\smash{\textsuperscript{h}}}{l}{j}{au}{ð}{\textscy}{\textsubring{r}}} \dicPos{adj}[2]\dicFlx{}[-12] \dicDirectTranslationCS{nezakončený, nevyřešený} \dicExampleIS{Málið er enn óútkljáð.} \dicExampleCS{Záležitost je ještě nevyřešená.}
\dicEntry[óútreiknanlegur] \dicTerm{ó··út·reiknan·legur} \dicIPA{{ou}{\textlengthmark}{u}{\textsubring{d}}{r}{ei}{h}{\r{g}}{n}{a}{n}{l}{\textepsilon}{\textbabygamma}{\textscy}{\textsubring{r}}} \dicPos{adj}[1]\dicFlx{}[-8] \textbf{1.} \dicDirectTranslationCS{nevypočitatelný (příklad ap.)}  \textbf{2.} \dicDirectTranslationCS{nevypočitatelný, nepředvídatelný, nevyzpytatelný (člověk ap.)}
\dicEntry[óvanalegur] \dicTerm{ó··vana·legur} \dicIPA{{ou}{\textlengthmark}{v}{a}{n}{a}{l}{\textepsilon}{\textbabygamma}{\textscy}{\textsubring{r}}} \dicPos{adj}[1]\dicFlx{}[-8] \dicSynonym{óvenjulegur} \dicDirectTranslationCS{neobyčejný, netradiční} \dicAntonym{vanalegur}
\dicEntry[óvandaður] \dicTerm{ó··|vand·aður} \dicIPA{{ou}{\textlengthmark}{v}{a}{n}{\textsubring{d}}{a}{ð}{\textscy}{\textsubring{r}}} \dicPos{adj}[3] \dicFlx{(f ‑vönduð)}[1] \textbf{1.} \dicSynonym{lélegur} \dicDirectTranslationCS{odbytý, odfláknutý} \dicAntonym{vandaður}  \textbf{2.} \dicSynonym{óheiðarlegur} \dicDirectTranslationCS{nepoctivý, podvodný}
\dicEntry[óvandvirkni] \dicTerm{ó··vand·virkn|i} \dicIPA{{ou}{\textlengthmark}{v}{a}{n}{\textsubring{d}}{v}{\textsci}{\textsubring{r}}{\r{g}}{n}{\textsci}} \dicPos{f}[3] \dicFlx{(‑i)}[3] \dicDirectTranslationCS{nedbalost, nepečlivost}
\dicEntry[óvani] \dicTerm{ó··van|i} \dicIPA{{ou}{\textlengthmark}{v}{a}{n}{\textsci}} \dicPos{m}[1] \dicFlx{(‑a)}[3] \textbf{1.} \dicSynonym{ósiður} \dicDirectTranslationCS{zlozvyk, nešvar} \dicAntonym{vani}  \textbf{2.} \dicSynonym*{skortur á vana} \dicDirectTranslationCS{nezvyk}
\dicEntry[óvanur] \dicTerm{ó··|vanur} \dicIPA{{ou}{\textlengthmark}{v}{a}{n}{\textscy}{\textsubring{r}}} \dicPos{adj}[1] \dicFlx{(f ‑vön)}[2] \textbf{1.} \dicSynonym*{óþjálfaður} \dicDirectTranslationCS{nezvyklý, nenavyklý, nepřivyklý} \dicExampleIS{vera óvanur e‑u} \dicExampleCS{nebýt zvyklý na (co)} \dicAntonym{vanur\smash{\textsuperscript{2}}}  \textbf{2.} \dicSynonym*{afvanur} \dicDirectTranslationCS{nezvyklý, neobvyklý}
\dicEntry[óvar] \dicTerm{ó··|var} \dicsymFrequent\  \dicIPA{{ou}{\textlengthmark}{v}{a}{\textsubring{r}}} \dicPos{adj}[5] \dicFlx{(f ‑vör)}[9] \dicDirectTranslationCS{neostražitý, neopatrný, nepozorný} \dicAntonym{var\smash{\textsuperscript{1}}};  \dicPhraseIS{koma e‑m að óvörum} \dicDirectTranslationCS{zaskočit (koho), překvapit (koho)} \dicExampleIS{Það kemur mér ekki að óvörum.} \dicExampleCS{To mě nepřekvapuje.}
\dicEntry[óvarfærinn] \dicTerm{ó··var·færinn} \dicIPA{{ou}{\textlengthmark}{v}{a}{\textsubring{r}}{f}{a}{i}{r}{\textsci}{\textsubring{n}}} \dicPos{adj}[6]\dicFlx{}[-2] \dicSynonym{gálaus} \dicDirectTranslationCS{neuvážený, neobezřetný} \dicAntonym{varfærinn}
\dicEntry[óvarkár] \dicTerm{ó··var·kár} \dicIPA{{ou}{\textlengthmark}{v}{a}{\textsubring{r}}{k\smash{\textsuperscript{h}}}{au}{\textsubring{r}}} \dicPos{adj}[5] \dicFlx{(f ‑)}[8] \dicSynonym{fyrirhyggjulaus} \dicDirectTranslationCS{neopatrný, neostražitý, neobezřetný} \dicExampleIS{óvarkár forseti} \dicExampleCS{neostražitý prezident} \dicAntonym{varkár}
\dicEntry[óvart] \dicTerm{ó··vart} \dicIPA{{ou}{\textlengthmark}{v}{a}{\textsubring{r}}{\textsubring{d}}} \dicPos{adv} \dicDirectTranslationCS{neúmyslně, nechtěně, nevědomky};  \dicPhraseIS{koma e‑m á óvart} \dicDirectTranslationCS{překvapit (koho), zaskočit (koho)};  \dicPhraseIS{gera e‑ð óvart} \dicDirectTranslationCS{udělat (co) nevědomky\,/\addthin nechtěně}
\dicEntry[óveður] \dicTerm{ó··veður} \dicIPA{{ou}{\textlengthmark}{v}{\textepsilon}{ð}{\textscy}{\textsubring{r}}} \dicPos{n}[2] \dicFlx{(‑s, ‑)}[25] \dicSynonym{rok} \dicDirectTranslationCS{bouře, bouřlivé počasí}
\dicEntry[óvenja] \dicTerm{ó··venj|a} \dicIPA{{ou}{\textlengthmark}{v}{\textepsilon}{n}{j}{a}} \dicPos{f}[1] \dicFlx{(‑u, ‑ur)}[7] \textbf{1.} \dicSynonym{ósiður} \dicDirectTranslationCS{zlozvyk, nešvar, nezvyk}  \textbf{2.} \dicSynonym*{e‑að óvenjulegt} \dicDirectTranslationCS{nezvyk, nezvyklost}
\dicEntry[óvenju] \dicTerm{ó··venju} \dicsymFrequent\  \dicIPA{{ou}{\textlengthmark}{v}{\textepsilon}{n}{j}{\textscy}} \dicPos{adv} \dicDirectTranslationCS{nezvykle, neobvykle, nevídaně} \dicExampleIS{Kossinn var óvenju langur.} \dicExampleCS{Polibek byl nezvykle dlouhý.}
\dicEntry[óvenjulegur] \dicTerm{ó··venju·legur} \dicsymFrequent\  \dicIPA{{ou}{\textlengthmark}{v}{\textepsilon}{n}{j}{\textscy}{l}{\textepsilon}{\textbabygamma}{\textscy}{\textsubring{r}}} \dicPos{adj}[1]\dicFlx{}[-8] \dicSynonym{sérstakur} \dicDirectTranslationCS{neobyčejný, neobvyklý, nezvyklý} \dicExampleIS{óvenjulegur dagur} \dicExampleCS{neobyčejný den} \dicAntonym{venjulegur}
\dicEntry[óvera] \dicTerm{ó··ver|a} \dicIPA{{ou}{\textlengthmark}{v}{\textepsilon}{r}{a}} \dicPos{f}[1] \dicFlx{(‑u, ‑ur)}[7] \textbf{1.} \dicSynonym{ögn\smash{\textsuperscript{1}}} \dicDirectTranslationCS{maličkost, drobnost}  \textbf{2.} \dicFieldCat{filos.} \dicSynonym{neind} \dicDirectTranslationCS{nebytí}
\dicEntry[óverðskuldaður] \dicTerm{ó··verð·skuld·|aður} \dicIPA{{ou}{\textlengthmark}{v}{\textepsilon}{r}{ð}{s}{\r{g}}{\textscy}{l}{\textsubring{d}}{a}{ð}{\textscy}{\textsubring{r}}} \dicPos{adj}[3] \dicFlx{(f ‑uð)}[3] \dicSynonym{ómaklegur} \dicDirectTranslationCS{nezasloužený, nespravedlivý} \dicAntonym{verðskuldaður}
\dicEntry[óverðugur] \dicTerm{ó··verð·ugur} \dicIPA{{ou}{\textlengthmark}{v}{\textepsilon}{r}{ð}{\textscy}{\textbabygamma}{\textscy}{\textsubring{r}}} \dicPos{adj}[1]\dicFlx{}[-8] \dicDirectTranslationCS{nehodný, nezasluhující si (důvěru ap.)}
\dicEntry[óverjandi] \dicTerm{ó··verj·andi} \dicIPA{{ou}{\textlengthmark}{v}{\textepsilon}{r}{j}{a}{n}{\textsubring{d}}{\textsci}} \dicPos{adj}[13] \dicFlx{indecl}[1] \dicSynonym{óafsakanlegur} \dicDirectTranslationCS{neomluvitelný, neospravedlnitelný} \dicAntonym{verjandi\smash{\textsuperscript{2}}}
\dicEntry[óverulegur] \dicTerm{ó··veru·legur} \dicIPA{{ou}{\textlengthmark}{v}{\textepsilon}{r}{\textscy}{l}{\textepsilon}{\textbabygamma}{\textscy}{\textsubring{r}}} \dicPos{adj}[1]\dicFlx{}[-8] \dicSynonym*{lítilsverður} \dicDirectTranslationCS{nepatrný, zanedbatelný} \dicAntonym{verulegur}
\dicEntry[óvéfengjanlegur] \dicTerm{ó··vé·fengjan·legur} \dicIPA{{ou}{\textlengthmark}{v}{j}{\textepsilon}{v}{ei}{\textltailn}{\r{\textObardotlessj}}{a}{n}{l}{\textepsilon}{\textbabygamma}{\textscy}{\textsubring{r}}} \dicPos{adj}[1]\dicFlx{}[-8] \dicSynonym{vafalaus} \dicDirectTranslationCS{neoddiskutovatelný, nesporný}
\dicEntry[óviðbúinn] \dicTerm{ó··við·búinn} \dicIPA{{ou}{\textlengthmark}{v}{\textsci}{ð}{\textsubring{b}}{u}{\textsci}{\textsubring{n}}} \dicPos{adj}[6]\dicFlx{}[-2] \dicSynonym{grandalaus} \dicDirectTranslationCS{nepřipravený, nenachystaný} \dicAntonym{viðbúinn}
\dicEntry[óviðeigandi] \dicTerm{ó··við·eig·andi} \dicIPA{{ou}{\textlengthmark}{v}{\textsci}{ð}{ei}{\textbabygamma}{a}{n}{\textsubring{d}}{\textsci}} \dicPos{adj}[13] \dicFlx{indecl}[1] \dicSynonym{ósæmilegur} \dicDirectTranslationCS{nevhodný, nemístný, nepatřičný} \dicAntonym{viðeigandi}
\dicEntry[óviðfelldinn] \dicTerm{ó··við·felldinn} \dicIPA{{ou}{\textlengthmark}{v}{\textsci}{ð}{f}{\textepsilon}{l}{\textsubring{d}}{\textsci}{\textsubring{n}}} \dicPos{adj}[6]\dicFlx{}[-2] \dicSynonym{ógeðfelldur} \dicDirectTranslationCS{nepříjemný, nepřívětivý, nevlídný} \dicAntonym{viðfelldinn}
\dicEntry[óviðjafnanlegur] \dicTerm{ó··við·jafnan·legur} \dicIPA{{ou}{\textlengthmark}{v}{\textsci}{ð}{j}{a}{\textsubring{b}}{n}{a}{n}{l}{\textepsilon}{\textbabygamma}{\textscy}{\textsubring{r}}} \dicPos{adj}[1]\dicFlx{}[-8] \dicSynonym*{dæmafár} \dicDirectTranslationCS{nesrovnatelný, neporovnatelný}
\dicEntry[óviðkomandi] \dicTerm{ó··við·kom·andi} \dicIPA{{ou}{\textlengthmark}{v}{\textsci}{ð}{k\smash{\textsuperscript{h}}}{\textopeno}{m}{a}{n}{\textsubring{d}}{\textsci}} \dicPos{adj}[13] \dicFlx{indecl}[1] \textbf{1.} \dicSynonym*{utan við efnið} \dicDirectTranslationCS{irelevantní, netýkající se} \dicExampleIS{Þetta er mér óviðkomandi.} \dicExampleCS{To se mě netýká.} \dicAntonym{viðkomandi\smash{\textsuperscript{2}}}  \textbf{2.} \dicSynonym*{sem á ekki erindi} \dicDirectTranslationCS{nepovolaný, neoprávněný};  \dicPhraseIS{óviðkomandi bannaður aðgangur} \dicDirectTranslationCS{nepovolaným vstup zakázán}
\dicEntry[óviðkunnanlegur] \dicTerm{ó··við·kunnan·legur} \dicIPA{{ou}{\textlengthmark}{v}{\textsci}{ð}{k\smash{\textsuperscript{h}}}{\textscy}{n}{a}{n}{l}{\textepsilon}{\textbabygamma}{\textscy}{\textsubring{r}}} \dicPos{adj}[1]\dicFlx{}[-8] \dicDirectTranslationCS{nepříjemný, nesympatický}
\dicEntry[óviðráðanlegur] \dicTerm{ó··við·ráðan·legur} \dicIPA{{ou}{\textlengthmark}{v}{\textsci}{ð}{r}{au}{ð}{a}{n}{l}{\textepsilon}{\textbabygamma}{\textscy}{\textsubring{r}}} \dicPos{adj}[1]\dicFlx{}[-8] \textbf{1.} \dicSynonym*{óhemjandi} \dicDirectTranslationCS{nezvladatelný, nekontrolovatelný} \dicExampleIS{óviðráðanlegur hestur} \dicExampleCS{nezvladatelný kůň} \dicAntonym{viðráðanlegur}  \textbf{2.} \dicDirectTranslationCS{nepředvídatelný} \dicExampleIS{af óviðráðanlegum orsökum} \dicExampleCS{vzhledem k~nepředvídatelným okolnostem}
\dicEntry[óviðunandi] \dicTerm{ó··við·un·andi} \dicIPA{{ou}{\textlengthmark}{v}{\textsci}{ð}{\textscy}{n}{a}{n}{\textsubring{d}}{\textsci}} \dicPos{adj}[13] \dicFlx{indecl}[1] \dicDirectTranslationCS{neuspokojivý, nedostatečný} \dicAntonym{viðunandi}
\dicEntry[óvild] \dicTerm{ó··vild} \dicIPA{{ou}{\textlengthmark}{v}{\textsci}{l}{\textsubring{d}}} \dicPos{f}[4] \dicFlx{(‑ar)}[3] \dicSynonym{kali} \dicDirectTranslationCS{nevraživost, nepřátelství, zášť} \dicExampleIS{óvild í garð e‑rs} \dicExampleCS{nevraživost vůči (komu)}
\dicEntry[óvilhallur] \dicTerm{ó··vil·|hallur} \dicIPA{{ou}{\textlengthmark}{v}{\textsci}{l}{h}{a}{\textsubring{d}}{l}{\textscy}{\textsubring{r}}} \dicPos{adj}[1] \dicFlx{(f ‑höll)}[2] \dicSynonym{óhlutdrægur} \dicDirectTranslationCS{nestranný, nezaujatý, neutrální} \dicAntonym{vilhallur}
\dicEntry[óviljandi] \dicTerm{ó··vilj·andi} \dicIPA{{ou}{\textlengthmark}{v}{\textsci}{l}{j}{a}{n}{\textsubring{d}}{\textsci}} \dicPos{adj}[13] \dicFlx{indecl}[1] \dicSynonym{óvart} \dicDirectTranslationCS{neúmyslný, nechtěný, bezděčný} \dicAntonym{viljandi\smash{\textsuperscript{2}}}
\dicEntry[óvinátta] \dicTerm{ó··vin·átt|a} \dicIPA{{ou}{\textlengthmark}{v}{\textsci}{n}{au}{h}{\textsubring{d}}{a}} \dicPos{f}[1] \dicFlx{(‑u)}[5] \dicSynonym{fjandskapur} \dicDirectTranslationCS{nepřátelství} \dicAntonym{vinátta}
\dicEntry[óvingjarnlegur] \dicTerm{ó··vin·gjarn·legur} \dicIPA{{ou}{\textlengthmark}{v}{\textsci}{n}{\r{\textObardotlessj}}{a}{r}{\textsubring{d}}{n}{l}{\textepsilon}{\textbabygamma}{\textscy}{\textsubring{r}}} \dicPos{adj}[1]\dicFlx{}[-8] \dicSynonym{óblíður} \dicDirectTranslationCS{nepřátelský, nepřívětivý} \dicAntonym{vingjarnlegur}
\dicEntry[óvinnandi] \dicTerm{ó··vinn·andi} \dicIPA{{ou}{\textlengthmark}{v}{\textsci}{n}{a}{n}{\textsubring{d}}{\textsci}} \dicPos{adj}[13] \dicFlx{indecl}[1] \textbf{1.} \dicSynonym*{ósigranlegur} \dicDirectTranslationCS{nedobytný, neporazitelný} \dicExampleIS{óvinnandi vígi} \dicExampleCS{nedobytná pevnost}  \textbf{2.} \dicSynonym{ókleifur} \dicDirectTranslationCS{neproveditelný, nevykonatelný} \dicExampleIS{óvinnandi verk} \dicExampleCS{neproveditelná práce}  \textbf{3.} \dicDirectTranslationCS{nepracující, neschopný práce}
\dicEntry[óvinnufær] \dicTerm{ó··vinnu·fær} \dicIPA{{ou}{\textlengthmark}{v}{\textsci}{n}{\textscy}{f}{a}{i}{\textsubring{r}}} \dicPos{adj}[5] \dicFlx{(f ‑)}[8] \dicDirectTranslationCS{neschopný práce, invalidní}
\dicEntry[óvinsamlegur] \dicTerm{ó··vin·sam·legur} \dicIPA{{ou}{\textlengthmark}{v}{\textsci}{n}{s}{a}{m}{l}{\textepsilon}{\textbabygamma}{\textscy}{\textsubring{r}}} \dicPos{adj}[1]\dicFlx{}[-8] \dicSynonym{kaldranalegur} \dicDirectTranslationCS{nespolečenský, nedružný} \dicAntonym{vinsamlegur}
\dicEntry[óvinsæll] \dicTerm{ó··vin·sæll} \dicIPA{{ou}{\textlengthmark}{v}{\textsci}{n}{s}{a}{i}{\textsubring{d}}{\textsubring{l}}} \dicPos{adj}[8]\dicFlx{}[-1] \dicSynonym{hvimleiður} \dicDirectTranslationCS{nepopulární, neoblíbený} \dicExampleIS{óvinsæl ráðstöfun} \dicExampleCS{nepopulární krok} \dicAntonym{vinsæll}
\dicEntry[óvinur] \dicTerm{ó··vin|ur} \dicsymFrequent\  \dicIPA{{ou}{\textlengthmark}{v}{\textsci}{n}{\textscy}{\textsubring{r}}} \dicPos{m}[10] \dicFlx{(‑ar, ‑ir)}[17] \dicSynonym{fjandmaður} \dicDirectTranslationCS{nepřítel(kyně), protivník, protivnice} \dicExampleIS{Óttinn er helsti óvinur mannsins.} \dicExampleCS{Strach je největším nepřítelem člověka.} \dicAntonym{vinur}
\dicEntry[óvinveittur] \dicTerm{ó··vin·veittur} \dicIPA{{ou}{\textlengthmark}{v}{\textsci}{n}{v}{ei}{h}{\textsubring{d}}{\textscy}{\textsubring{r}}} \dicPos{adj}[1]\dicFlx{}[-10] \dicSynonym{fjandsamlegur} \dicDirectTranslationCS{nepřátelský} \dicAntonym{vinveittur};  \dicPhraseIS{vera e‑m óvinveittur} \dicDirectTranslationCS{chovat se nepřátelsky ke (komu)}
\dicEntry[óvirða] \dicTerm{ó··vir|ða} \dicIPA{{ou}{\textlengthmark}{v}{\textsci}{r}{ð}{a}} \dicPos{v}[2] \dicFlx{(‑ti, ‑t)}[47] \dicFlx{acc} \textbf{1.} \dicSynonym{smána} \dicDirectTranslationCS{pohrdat, přehlížet}  \textbf{2.} \dicSynonym*{sýna óvirðingu} \dicDirectTranslationCS{nevážit si, nemít úctu}
\dicEntry[óvirðing] \dicTerm{ó··virð·ing} \dicIPA{{ou}{\textlengthmark}{v}{\textsci}{r}{ð}{i}{\ng}{\r{g}}} \dicPos{f}[4] \dicFlx{(‑ar)}[7] \dicSynonym{fyrirlitning} \dicDirectTranslationCS{neúcta, pohrdání, despekt} \dicAntonym{virðing}
\dicEntry[óvirkur] \dicTerm{ó··virkur} \dicIPA{{ou}{\textlengthmark}{v}{\textsci}{\textsubring{r}}{\r{g}}{\textscy}{\textsubring{r}}} \dicPos{adj}[1]\dicFlx{}[-1] \textbf{1.} \dicSynonym{helgur} \dicDirectTranslationCS{volný (den ap.)} \dicIndirectTranslationCS{(nepracovní)} \dicExampleIS{óvirkur dagur} \dicExampleCS{volný den} \dicAntonym{virkur}  \textbf{2.} \dicSynonym*{aðgerðarlaus} \dicDirectTranslationCS{nečinný, neaktivní, pasivní}  \textbf{3.} \dicDirectTranslationCS{nefungující, nepracující}  \textbf{4.} \dicFieldCat{jaz.} \dicDirectTranslationCS{neřídící}
\dicEntry[óviss] \dicTerm{ó··viss} \dicIPA{{ou}{\textlengthmark}{v}{\textsci}{s}} \dicPos{adj}[5]\dicFlx{}[-1] \dicSynonym{hikandi} \dicDirectTranslationCS{nejistý, neurčitý} \dicExampleIS{vera óviss um e‑ð} \dicExampleCS{nebýt si jistý (čím)} \dicAntonym{viss}
\dicEntry[óvissa] \dicTerm{ó··viss|a} \dicsymFrequent\  \dicIPA{{ou}{\textlengthmark}{v}{\textsci}{s}{a}} \dicPos{f}[1] \dicFlx{(‑u)}[5] \dicSynonym*{vafasemi} \dicDirectTranslationCS{nejistota, neurčitost} \dicExampleIS{óvissa um stjórn landsins} \dicExampleCS{nejistota o~vládě v~zemi} \dicAntonym{vissa}
\dicEntry[óvistlegur] \dicTerm{ó··vist·legur} \dicIPA{{ou}{\textlengthmark}{v}{\textsci}{s}{\textsubring{d}}{l}{\textepsilon}{\textbabygamma}{\textscy}{\textsubring{r}}} \dicPos{adj}[1]\dicFlx{}[-8] \dicDirectTranslationCS{nepohodlný, neútulný}
\dicEntry[óvit] \dicTerm{ó··vit} \dicIPA{{ou}{\textlengthmark}{v}{\textsci}{\textsubring{d}}} \dicPos{n}[2] \dicFlx{(‑s)}[2] \textbf{1.} \dicSynonym{fásinna} \dicDirectTranslationCS{nerozum, pošetilost}  \textbf{2.} \dicSynonym{vitfirring} \dicDirectTranslationCS{šílenství} \dicAntonym{vit}  \textbf{3.} \dicSynonym{öngvit} \dicDirectTranslationCS{bezvědomí, mdloba} \dicExampleIS{falla í óvit} \dicExampleCS{upadnout do bezvědomí}
\dicEntry[óvitandi] \dicTerm{ó··vit·andi} \dicIPA{{ou}{\textlengthmark}{v}{\textsci}{\textsubring{d}}{a}{n}{\textsubring{d}}{\textsci}} \dicPos{adj}[13] \dicFlx{indecl}[1] \dicSynonym{óvar} \dicDirectTranslationCS{neznalý, nevědoucí} \dicAntonym{vitandi}
\dicEntry[óviti] \dicTerm{ó··vit|i} \dicIPA{{ou}{\textlengthmark}{v}{\textsci}{\textsubring{d}}{\textsci}} \dicPos{m}[1] \dicFlx{(‑a, ‑ar)}[1] \dicDirectTranslationCS{nemluvně, neviňátko} \dicExampleIS{Barnið er óviti.} \dicExampleCS{Dítě je neviňátko.}
\dicEntry[óvitlaus] \dicTerm{ó··vit·laus} \dicIPA{{ou}{\textlengthmark}{v}{\textsci}{h}{\textsubring{d}}{l}{\oe i}{s}} \dicPos{adj}[5]\dicFlx{}[-1] \dicDirectTranslationCS{inteligentní, rozumný}
\dicEntry[óviturlegur] \dicTerm{ó··vitur·legur} \dicIPA{{ou}{\textlengthmark}{v}{\textsci}{\textsubring{d}}{\textscy}{r}{l}{\textepsilon}{\textbabygamma}{\textscy}{\textsubring{r}}} \dicPos{adj}[1]\dicFlx{}[-8] \dicSynonym{heimskulegur} \dicDirectTranslationCS{neuvážený, nerozumný} \dicAntonym{viturlegur}
\dicEntry[óvís] \dicTerm{ó··vís} \dicIPA{{ou}{\textlengthmark}{v}{i}{s}} \dicPos{adj}[5]\dicFlx{}[-1] \textbf{1.} \dicSynonym{óviss} \dicDirectTranslationCS{nejistý, neurčitý} \dicAntonym{vís}  \textbf{2.} \dicSynonym{fáfróður} \dicDirectTranslationCS{neznalý, nemoudrý}
\dicEntry[óvon] \dicTerm{ó··von} \dicIPA{{ou}{\textlengthmark}{v}{\textopeno}{\textsubring{n}}} \dicPos{f}[7] \dicFlx{(‑ar)}[4] \dicPhraseIS{upp á von og óvon} \dicFlx{adv} \dicDirectTranslationCS{pro případ, kdyby náhodou}
\dicEntry[óvopnaður] \dicTerm{ó··vopn·|aður} \dicIPA{{ou}{\textlengthmark}{v}{\textopeno}{h}{\textsubring{b}}{n}{a}{ð}{\textscy}{\textsubring{r}}} \dicPos{adj}[3] \dicFlx{(f ‑uð)}[4] \dicDirectTranslationCS{neozbrojený, nevyzbrojený}
\dicEntry[óvæginn] \dicTerm{ó··væginn} \dicIPA{{ou}{\textlengthmark}{v}{a}{i}{j}{\textsci}{\textsubring{n}}} \dicPos{adj}[6]\dicFlx{}[-2] \dicSynonym*{harðlyndur} \dicDirectTranslationCS{nevybíravý, bezohledný} \dicAntonym{væginn}
\dicEntry[óvænt] \dicTerm{ó··vænt} \dicsymFrequent\  \dicIPA{{ou}{\textlengthmark}{v}{a}{i}{\textsubring{n}}{\textsubring{d}}} \dicPos{adv} \dicSynonym{snögglega} \dicDirectTranslationCS{neočekávaně, nečekaně, nenadále} \dicExampleIS{Þeir komu óvænt í heimsókn til okkar.} \dicExampleCS{Přišli k~nám nečekaně na návštěvu.}
\dicEntry[óvæntur] \dicTerm{ó··væntur} \dicsymFrequent\  \dicIPA{{ou}{\textlengthmark}{v}{a}{i}{\textsubring{n}}{\textsubring{d}}{\textscy}{\textsubring{r}}} \dicPos{adj}[1]\dicFlx{}[-10] \dicDirectTranslationCS{nečekaný, nenadálý, neočekávaný} \dicExampleIS{óvæntur sigur} \dicExampleCS{nečekané vítězství}
\dicEntry[óvær] \dicTerm{ó··vær} \dicIPA{{ou}{\textlengthmark}{v}{a}{i}{\textsubring{r}}} \dicPos{adj}[5] \dicFlx{(f ‑)}[8] \dicSynonym{órór} \dicDirectTranslationCS{neklidný, neposedný} \dicAntonym{vær};  \dicPhraseIS{e‑m er óvært e‑s staðar} \dicDirectTranslationCS{(kdo kde) nemůže vydržet, (kdo kde) nemá klid}
\dicEntry[óvættur] \dicTerm{ó··vætt|ur\smash{\textsuperscript{1}}}\dicTerm{, óvættur\smash{\textsuperscript{2}}} \dicIPA{{ou}\-{\textlengthmark}\-{v}\-{a}\-{i}\-{h}\-{\textsubring{d}}\-{\textscy}\-{\textsubring{r}}\-} \dicPos{m}[10] \dicFlx{(‑ar, ‑ir)}[4] \textbf{1.} \dicSynonym*{illur andi} \dicDirectTranslationCS{zlý duch}  \textbf{2.} \dicSynonym{grýla\smash{\textsuperscript{1}}} \dicDirectTranslationCS{bestie, nestvůra, monstrum}
\dicEntry[óvættur] \dicTerm{ó··vætt|ur\smash{\textsuperscript{2}}} \dicIPA{{ou}{\textlengthmark}{v}{a}{i}{h}{\textsubring{d}}{\textscy}{\textsubring{r}}} \dicPos{f}[7] \dicFlx{(‑ar, ‑ir)}[1] \dicLink{óvættur\smash{\textsuperscript{1}}}
\dicEntry[óx] \dicTerm{óx} \dicIPA{{ou}{x}{s}} \dicPos{v} \dicFlx{ind pf sg 1 pers} \dicLink{vaxa}
\dicEntry[óyfirstíganlegur] \dicTerm{ó··yfir·stígan·legur} \dicIPA{{ou}{\textlengthmark}{\textsci}{v}{\textsci}{\textsubring{r}}{s}{\textsubring{d}}{i}{\textbabygamma}{a}{n}{l}{\textepsilon}{\textbabygamma}{\textscy}{\textsubring{r}}} \dicPos{adj}[1]\dicFlx{}[-8] \dicDirectTranslationCS{nepřekonatelný (problém ap.)}
\dicEntry[óyggjandi] \dicTerm{ó··yggj·andi} \dicIPA{{ou}{\textlengthmark}{\textsci}{\r{\textObardotlessj}}{a}{n}{\textsubring{d}}{\textsci}} \dicPos{adj}[13] \dicFlx{indecl}[1] \dicSynonym{áreiðanlegur} \dicDirectTranslationCS{nepochybný, nezvratný, jistý} \dicExampleIS{óyggjandi sannanir} \dicExampleCS{nezvratné důkazy}
\dicEntry[óyndi] \dicTerm{ó··yndi} \dicIPA{{ou}{\textlengthmark}{\textsci}{n}{\textsubring{d}}{\textsci}} \dicPos{n}[2] \dicFlx{(‑s, ‑)}[14] \textbf{1.} \dicSynonym{leiðindi} \dicDirectTranslationCS{dlouhá chvíle, nuda} \dicExampleIS{Það sótti á hana óyndi.} \dicExampleCS{Přepadla ji nuda.} \dicAntonym{yndi\smash{\textsuperscript{1}}}  \textbf{2.} \dicSynonym{heimþrá} \dicDirectTranslationCS{stesk, zasmušilost}
\dicEntry[óyndisúrræði] \dicTerm{ó··yndis·úr·ræði} \dicIPA{{ou}{\textlengthmark}{\textsci}{n}{\textsubring{d}}{\textsci}{s}{u}{r}{a}{i}{ð}{\textsci}} \dicPos{n}[2] \dicFlx{(‑s, ‑)}[14] \dicDirectTranslationCS{nouzové řešení};  \dicPhraseIS{grípa til óyndisúrræða} \dicDirectTranslationCS{sáhnout k~nouzovému řešení}
\dicEntry[óþarfi] \dicTerm{ó··þarf|i} \dicIPA{{ou}{\textlengthmark}{\texttheta}{a}{r}{v}{\textsci}} \dicPos{m}[1] \dicFlx{(‑a)}[3] \dicSynonym*{gagnsleysi} \dicDirectTranslationCS{nepotřebnost, (co) nepotřebného} \dicExampleIS{Það er alger óþarfi.} \dicExampleCS{To není potřeba.}
\dicEntry[óþarfur] \dicTerm{ó··|þarfur} \dicsymFrequent\  \dicIPA{{ou}{\textlengthmark}{\texttheta}{a}{r}{v}{\textscy}{\textsubring{r}}} \dicPos{adj}[1] \dicFlx{(f ‑þörf)}[2] \dicSynonym{gagnslaus} \dicDirectTranslationCS{nepotřebný, zbytečný, nadbytečný, přebytečný} \dicExampleIS{óþarfur kostnaður} \dicExampleCS{zbytečné náklady} \dicAntonym{þarfur}
\dicEntry[óþefur] \dicTerm{ó··þef|ur} \dicIPA{{ou}{\textlengthmark}{\texttheta}{\textepsilon}{v}{\textscy}{\textsubring{r}}} \dicPos{m}[9] \dicFlx{(‑s\,/\addthin ‑jar)}[27] \dicSynonym{daunn} \dicDirectTranslationCS{(odporný) puch, smrad} \dicExampleIS{Óþefur leggur um allt hús.} \dicExampleCS{Smrdí to po celém domě.}
\dicEntry[óþekkjanlegur] \dicTerm{ó··þekkjan·legur} \dicIPA{{ou}{\textlengthmark}{\texttheta}{\textepsilon}{h}{\r{\textObardotlessj}}{a}{n}{l}{\textepsilon}{\textbabygamma}{\textscy}{\textsubring{r}}} \dicPos{adj}[1]\dicFlx{}[-8] \dicSynonym{torkennilegur} \dicDirectTranslationCS{nepoznatelný, (jsoucí) k~nepoznání} \dicAntonym{þekkjanlegur}
\dicEntry[óþekkt] \dicTerm{ó··þekkt} \dicIPA{{ou}{\textlengthmark}{\texttheta}{\textepsilon}{x}{\textsubring{d}}} \dicPos{f}[4] \dicFlx{(‑ar)}[3] \textbf{1.} \dicSynonym{óþægð} \dicDirectTranslationCS{neposlušnost, zlobivost}  \textbf{2.} \dicSynonym{viðbjóður} \dicDirectTranslationCS{odpor, hnus}
\dicEntry[óþekktur] \dicTerm{ó··þekktur} \dicsymFrequent\  \dicIPA{{ou}{\textlengthmark}{\texttheta}{\textepsilon}{x}{\textsubring{d}}{\textscy}{\textsubring{r}}} \dicPos{adj}[1]\dicFlx{}[-10] \dicSynonym*{ókenndur} \dicDirectTranslationCS{neznámý, cizí} \dicExampleIS{óþekktur listamaður} \dicExampleCS{neznámý umělec} \dicAntonym{þekktur}
\dicEntry[óþekkur] \dicTerm{ó··þekkur} \dicIPA{{ou}{\textlengthmark}{\texttheta}{\textepsilon}{h}{\r{g}}{\textscy}{\textsubring{r}}} \dicPos{adj}[1]\dicFlx{}[-1] \textbf{1.} \dicSynonym{óþægur} \dicDirectTranslationCS{zlobivý, neposlušný} \dicExampleIS{óþekkir krakkar} \dicExampleCS{neposlušné děti}  \textbf{2.} \dicSynonym*{ógeðþekkur} \dicDirectTranslationCS{nevzhledný, nepěkný}
\dicEntry[óþéttur] \dicTerm{ó··þéttur} \dicIPA{{ou}{\textlengthmark}{\texttheta}{j}{\textepsilon}{h}{\textsubring{d}}{\textscy}{\textsubring{r}}} \dicPos{adj}[1]\dicFlx{}[-10] \dicDirectTranslationCS{netěsný, prosakující} \dicAntonym{þéttur}
\dicEntry[óþjáll] \dicTerm{ó··þjáll} \dicIPA{{ou}{\textlengthmark}{\texttheta}{j}{au}{\textsubring{d}}{\textsubring{l}}} \dicPos{adj}[8]\dicFlx{}[-1] \textbf{1.} \dicSynonym{harður} \dicDirectTranslationCS{neohebný, nepoddajný}  \textbf{2.} \dicSynonym{ómeðfærilegur} \dicDirectTranslationCS{neflexibilní, nepoddajný}
\dicEntry[óþjóðalýður] \dicTerm{ó·þjóða··lýð|ur} \dicIPA{{ou}{\textlengthmark}{\texttheta}{j}{ou}{ð}{a}{l}{i}{ð}{\textscy}{\textsubring{r}}} \dicPos{m}[6] \dicFlx{(‑s)}[17] \dicSynonym{hyski} \dicDirectTranslationCS{chátra, holota}
\dicEntry[óþjóðlegur] \dicTerm{ó··þjóð·legur} \dicIPA{{ou}{\textlengthmark}{\texttheta}{j}{ou}{ð}{l}{\textepsilon}{\textbabygamma}{\textscy}{\textsubring{r}}} \dicPos{adj}[1]\dicFlx{}[-8] \dicDirectTranslationCS{nevlastenecký}
\dicEntry[óþokkalegur] \dicTerm{ó··þokka·legur} \dicIPA{{ou}{\textlengthmark}{\texttheta}{\textopeno}{h}{\r{g}}{a}{l}{\textepsilon}{\textbabygamma}{\textscy}{\textsubring{r}}} \dicPos{adj}[1]\dicFlx{}[-8] \textbf{1.} \dicSynonym{sóðalegur} \dicDirectTranslationCS{zašpiněný, zamazaný}  \textbf{2.} \dicSynonym{óhrjálegur} \dicDirectTranslationCS{zanedbaný, zchátralý}  \textbf{3.} \dicSynonym{níðingslegur} \dicDirectTranslationCS{podlý, nízký}
\dicEntry[óþokki] \dicTerm{ó··þokk|i} \dicIPA{{ou}{\textlengthmark}{\texttheta}{\textopeno}{h}{\r{\textObardotlessj}}{\textsci}} \dicPos{m}[1] \dicFlx{(‑a, ‑ar)}[1] \textbf{1.} \dicSynonym{illmenni} \dicDirectTranslationCS{darebák, darebačka, ničema}  \textbf{2.} \dicSynonym{óvinátta} \dicDirectTranslationCS{nepřátelství, zášť};  \dicPhraseIS{leggja óþokka á e‑n} \dicDirectTranslationCS{nepřátelit se s~(kým)}
\dicEntry[óþolandi] \dicTerm{ó··þol·andi} \dicIPA{{ou}{\textlengthmark}{\texttheta}{\textopeno}{l}{a}{n}{\textsubring{d}}{\textsci}} \dicPos{adj}[13] \dicFlx{indecl}[1] \dicSynonym{óbærilegur} \dicDirectTranslationCS{nesnesitelný, (jsoucí) k~nevydržení} \dicAntonym{þolandi\smash{\textsuperscript{2}}}
\dicEntry[óþolinmóður] \dicTerm{ó··þolin·móður} \dicsymFrequent\  \dicIPA{{ou}{\textlengthmark}{\texttheta}{\textopeno}{l}{\textsci}{n}{m}{ou}{\textlengthmark}{ð}{\textscy}{\textsubring{r}}} \dicPos{adj}[2]\dicFlx{}[-6] \dicSynonym{óþreyjufullur} \dicDirectTranslationCS{netrpělivý, nedočkavý} \dicExampleIS{Hún var orðin óþolinmóð.} \dicExampleCS{Už byla netrpělivá.} \dicAntonym{þolinmóður}
\dicEntry[óþolinmæði] \dicTerm{ó··þolin·mæð|i} \dicIPA{{ou}{\textlengthmark}{\texttheta}{\textopeno}{l}{\textsci}{n}{m}{a}{i}{ð}{\textsci}} \dicPos{f}[3] \dicFlx{(‑i)}[3] \dicSynonym{óþreyja} \dicDirectTranslationCS{netrpělivost, nedočkavost} \dicExampleIS{vera með óþolinmæði} \dicExampleCS{být netrpělivý}
\dicEntry[óþreyja] \dicTerm{ó··þreyj|a} \dicIPA{{ou}{\textlengthmark}{\texttheta}{r}{ei}{j}{a}} \dicPos{f}[1] \dicFlx{(‑u)}[5] \textbf{1.} \dicSynonym{óþolinmæði} \dicDirectTranslationCS{netrpělivost, nedočkavost} \dicAntonym{þreyja};  \dicPhraseIS{bíða eftir e‑u með óþreyju} \dicDirectTranslationCS{čekat na (co) s~netrpělivostí}  \textbf{2.} \dicSynonym{eftirvænting} \dicDirectTranslationCS{očekávání}
\dicEntry[óþreyjufullur] \dicTerm{ó··þreyju·|fullur} \dicIPA{{ou}{\textlengthmark}{\texttheta}{r}{ei}{j}{\textscy}{f}{\textscy}{\textsubring{d}}{l}{\textscy}{\textsubring{r}}} \dicPos{adj}[10] \dicFlx{(comp ‑fyllri, sup ‑fyllstur)}[7] \dicSynonym{óþolinmóður} \dicDirectTranslationCS{netrpělivý, nedočkavý}
\dicEntry[óþreytandi] \dicTerm{ó··þreyt·andi} \dicIPA{{ou}{\textlengthmark}{\texttheta}{r}{ei}{\textsubring{d}}{a}{n}{\textsubring{d}}{\textsci}} \dicPos{adj}[13] \dicFlx{indecl}[1] \dicSynonym{ótrauður} \dicDirectTranslationCS{neúnavný, neunavující} \dicExampleIS{óþreytandi baráttumaður} \dicExampleCS{neúnavný bojovník}
\dicEntry[óþreyttur] \dicTerm{ó··þreyttur} \dicIPA{{ou}{\textlengthmark}{\texttheta}{r}{ei}{h}{\textsubring{d}}{\textscy}{\textsubring{r}}} \dicPos{adj}[1]\dicFlx{}[-10] \dicSynonym{hress} \dicDirectTranslationCS{nevyčerpaný, neunavený, svěží} \dicAntonym{þreyttur}
\dicEntry[óþrifalegur] \dicTerm{ó··þrifa·legur} \dicIPA{{ou}{\textlengthmark}{\texttheta}{r}{\textsci}{v}{a}{l}{\textepsilon}{\textbabygamma}{\textscy}{\textsubring{r}}} \dicPos{adj}[1]\dicFlx{}[-8] \dicSynonym{óhreinn} \dicDirectTranslationCS{znečištěný, ušpiněný, špinavý} \dicExampleIS{óþrifalegt veitingahús} \dicExampleCS{špinavá restaurace} \dicAntonym{þrifalegur}
\dicEntry[óþrifnaður] \dicTerm{ó··þrif·nað|ur} \dicIPA{{ou}{\textlengthmark}{\texttheta}{r}{\textsci}{\textsubring{b}}{n}{a}{ð}{\textscy}{\textsubring{r}}} \dicPos{m}[10] \dicFlx{(‑ar)}[9] \dicSynonym{sóðaskapur} \dicDirectTranslationCS{ušpiněnost, špinavost, zamazanost} \dicAntonym{þrifnaður}
\dicEntry[óþrjótandi] \dicTerm{ó··þrjót·andi} \dicIPA{{ou}{\textlengthmark}{\texttheta}{r}{j}{ou}{\textsubring{d}}{a}{n}{\textsubring{d}}{\textsci}} \dicPos{adj}[13] \dicFlx{indecl}[1] \dicSynonym{sífelldur} \dicDirectTranslationCS{nevyčerpatelný, nekonečný, bezedný}
\dicEntry[óþroskaður] \dicTerm{ó··þrosk·|aður} \dicIPA{{ou}{\textlengthmark}{\texttheta}{r}{\textopeno}{s}{\r{g}}{a}{ð}{\textscy}{\textsubring{r}}} \dicPos{adj}[3] \dicFlx{(f ‑uð)}[3] \textbf{1.} \dicSynonym{vanþroska} \dicDirectTranslationCS{nezralý, nedospělý} \dicAntonym{þroskaður}  \textbf{2.} \dicSynonym{grænn} \dicDirectTranslationCS{nedozrálý, neuzrálý (ovoce ap.)}
\dicEntry[óþrotlegur] \dicTerm{ó··þrot·legur} \dicIPA{{ou}{\textlengthmark}{\texttheta}{r}{\textopeno}{\textsubring{d}}{l}{\textepsilon}{\textbabygamma}{\textscy}{\textsubring{r}}} \dicPos{adj}[1]\dicFlx{}[-8] \dicSynonym{óendanlegur} \dicDirectTranslationCS{nevyčerpatelný, nekonečný}
\dicEntry[óþurft] \dicTerm{ó··þurft} \dicIPA{{ou}{\textlengthmark}{\texttheta}{\textscy}{\textsubring{r}}{\textsubring{d}}} \dicPos{f}[7] \dicFlx{(‑ar, ‑ir)}[1] \dicSynonym{tjón} \dicDirectTranslationCS{škoda, újma, ztráta};  \dicPhraseIS{gera e‑ð e‑m til óþurftar} \dicDirectTranslationCS{způsobit (komu co) ke škodě}
\dicEntry[óþurrkur] \dicTerm{ó··þurrk|ur} \dicIPA{{ou}{\textlengthmark}{\texttheta}{\textscy}{\textsubring{r}}{\r{g}}{\textscy}{\textsubring{r}}} \dicPos{m}[6] \dicFlx{(‑s, ‑ar)}[4] \dicDirectTranslationCS{sychravo, sychravé počasí}
\dicEntry[óþveginn] \dicTerm{ó··þveginn} \dicIPA{{ou}{\textlengthmark}{\texttheta}{v}{ei}{\textsci}{\textsubring{n}}} \dicPos{adj}[6]\dicFlx{}[-6] \dicSynonym{óhreinn} \dicDirectTranslationCS{nemytý} \dicAntonym{þveginn};  \dicPhraseIS{óþvegið orð} \dicLangCat{přen.} \dicDirectTranslationCS{neučesané slovo, slovo bez obalu};  \dicPhraseIS{láta e‑n hafa það óþvegið} \dicLangCat{přen.} \dicSynonym*{skamma e‑n} \dicDirectTranslationCS{nebrat si před (kým) servítky}
\dicEntry[óþverri] \dicTerm{ó··þverr|i} \dicIPA{{ou}{\textlengthmark}{\texttheta}{v}{\textepsilon}{r}{\textsci}} \dicPos{m}[1] \dicFlx{(‑a, ‑ar)}[1] \textbf{1.} \dicSynonym{óhreinindi} \dicDirectTranslationCS{špína, nečistota}  \textbf{2.} \dicSynonym*{óþverraorð} \dicDirectTranslationCS{sprosté slovo, špína}  \textbf{3.} \dicSynonym{sóði} \dicDirectTranslationCS{špinavec, špína, mizera}
\dicEntry[óþvingaður] \dicTerm{ó··þving·|aður} \dicIPA{{ou}{\textlengthmark}{\texttheta}{v}{i}{\ng}{\r{g}}{a}{ð}{\textscy}{\textsubring{r}}} \dicPos{adj}[3] \dicFlx{(f ‑uð)}[3] \dicDirectTranslationCS{nenucený, uvolněný} \dicExampleIS{óþvingaðar samræður} \dicExampleCS{nenucené rozhovory}
\dicEntry[óþyrmilegur] \dicTerm{ó··þyrmi·legur} \dicIPA{{ou}{\textlengthmark}{\texttheta}{\textsci}{r}{m}{\textsci}{l}{\textepsilon}{\textbabygamma}{\textscy}{\textsubring{r}}} \dicPos{adj}[1]\dicFlx{}[-8] \dicSynonym{harður} \dicDirectTranslationCS{tvrdý, surový, nemilosrdný}
\dicEntry[óþýður] \dicTerm{ó··þýður} \dicIPA{{ou}{\textlengthmark}{\texttheta}{i}{ð}{\textscy}{\textsubring{r}}} \dicPos{adj}[2]\dicFlx{}[-6] \dicSynonym{kuldalegur} \dicDirectTranslationCS{nevlídný, nevrlý} \dicAntonym{þýður}
\dicEntry[óþægð] \dicTerm{ó··þægð} \dicIPA{{ou}{\textlengthmark}{\texttheta}{a}{i}{\textbabygamma}{\texttheta}} \dicPos{f}[4] \dicFlx{(‑ar)}[3] \textbf{1.} \dicSynonym{gremja} \dicDirectTranslationCS{zlost, vztek} \dicAntonym{þægð};  \dicPhraseIS{gera e‑m e‑ð til óþægðar} \dicDirectTranslationCS{dělat (co komu) na zlost}  \textbf{2.} \dicSynonym{óþekkt} \dicDirectTranslationCS{neposlušnost, neposlouchání}
\dicEntry[óþægilegur] \dicTerm{ó··þægi·legur} \dicsymFrequent\  \dicIPA{{ou}{\textlengthmark}{\texttheta}{a}{i}{j}{\textsci}{l}{\textepsilon}{\textbabygamma}{\textscy}{\textsubring{r}}} \dicPos{adj}[1]\dicFlx{}[-8] \textbf{1.} \dicSynonym{neyðarlegur} \dicDirectTranslationCS{nepohodlný, nepříjemný} \dicExampleIS{Stóllinn er óþægilegur.} \dicExampleCS{Židle je nepohodlná.}  \textbf{2.} \dicSynonym{ógeðfelldur} \dicDirectTranslationCS{nepříjemný, protivný, otravný}
\dicEntry[óþægindi] \dicTerm{ó··þæg·indi} \dicsymFrequent\  \dicIPA{{ou}{\textlengthmark}{\texttheta}{a}{i}{j}{\textsci}{n}{\textsubring{d}}{\textsci}} \dicPos{n}[2] \dicFlx{pl}[19] \textbf{1.} \dicSynonym{vanlíðan} \dicDirectTranslationCS{nepříjemný pocit, (fyzická) slabost} \dicExampleIS{finna til óþæginda í fætinum} \dicExampleCS{cítit slabost v~nohou}  \textbf{2.} \dicSynonym{trafali} \dicDirectTranslationCS{nepříjemnost, nepohodlí, nepohodlnost} \dicExampleIS{óþægindi í flugvélinni} \dicExampleCS{nepohodlí v~letadle}
\dicEntry[óþægur] \dicTerm{ó··þægur} \dicIPA{{ou}{\textlengthmark}{\texttheta}{a}{i}{\textbabygamma}{\textscy}{\textsubring{r}}} \dicPos{adj}[1]\dicFlx{}[-1] \dicSynonym{óþekkur} \dicDirectTranslationCS{zlobivý, neposlušný} \dicAntonym{þægur}
\dicEntry[óþökk] \dicTerm{ó··|þökk} \dicIPA{{ou}{\textlengthmark}{\texttheta}{\oe}{h}{\r{g}}} \dicPos{f}[7] \dicFlx{(‑þakkar)}[19] \dicDirectTranslationCS{nevole, nelibost};  \dicPhraseIS{í óþökk e‑rs} \dicDirectTranslationCS{k~(čí) nelibosti};  \dicPhraseIS{kunna e‑m óþökk fyrir e‑ð} \dicDirectTranslationCS{nést nelibě (co) od (koho)}
\dicEntry[óæfður] \dicTerm{ó··æfður} \dicIPA{{ou}{\textlengthmark}{a}{i}{v}{ð}{\textscy}{\textsubring{r}}} \dicPos{adj}[2]\dicFlx{}[-4] \dicSynonym*{óþjálfaður} \dicDirectTranslationCS{necvičený, netrénovaný} \dicAntonym{æfður}
\dicEntry[óæskilegur] \dicTerm{ó··æski·legur} \dicIPA{{ou}{\textlengthmark}{a}{i}{s}{\r{\textObardotlessj}}{\textsci}{l}{\textepsilon}{\textbabygamma}{\textscy}{\textsubring{r}}} \dicPos{adj}[1]\dicFlx{}[-8] \dicDirectTranslationCS{nežádoucí, nevítaný, negativní} \dicExampleIS{óæskilegt áhrif} \dicExampleCS{nežádoucí vliv}
\dicEntry[óæti] \dicTerm{ó··æti} \dicIPA{{ou}{\textlengthmark}{a}{i}{\textsubring{d}}{\textsci}} \dicPos{n}[2] \dicFlx{(‑s)}[20] \dicDirectTranslationCS{nepoživatelné\,/\addthin nestravitelné jídlo}
\dicEntry[óætur] \dicTerm{ó··ætur} \dicIPA{{ou}{\textlengthmark}{a}{i}{\textsubring{d}}{\textscy}{\textsubring{r}}} \dicPos{adj}[1]\dicFlx{}[-6] \dicSynonym*{óneysluhæfur} \dicDirectTranslationCS{nejedlý, nepoživatelný} \dicAntonym{ætur}
\dicEntry[óöruggur] \dicTerm{ó··ör·uggur} \dicIPA{{ou}{\textlengthmark}{\oe}{r}{\textscy}{\r{g}}{\textscy}{\textsubring{r}}} \dicPos{adj}[1]\dicFlx{}[-1] \textbf{1.} \dicDirectTranslationCS{nejistý, nepevný (led ap.)}  \textbf{2.} \dicSynonym{óstyrkur} \dicDirectTranslationCS{nejistý (člověk ap.)}

