\dicLetter{n}{letter17}
\dicEntry[n.] \dicTerm{n.} \dicPos{zkr} \dicPhraseIS{nafnorð} \dicFieldCat{jaz.} \dicDirectTranslationCS{podstatné jméno}
\dicEntry[n.á.] \dicTerm{n.á.} \dicPos{zkr} \dicPhraseIS{næsta ár} \dicFlx{adv} \dicDirectTranslationCS{příští rok}
\dicEntry[n.m.] \dicTerm{n.m.} \dicPos{zkr} \dicPhraseIS{næsta mánaðar} \dicFlx{adv} \dicDirectTranslationCS{příštího měsíce}
\dicEntry[naðra] \dicTerm{naðra} \dicIPA{{n}{a}{ð}{r}{a}} \dicPos{f}[1] \dicFlx{(nöðru, nöðrur)}[8] \textbf{1.} \dicSynonym{slanga} \dicDirectTranslationCS{had, zmije}  \textbf{2.} \dicLangCat{přen.} \dicDirectTranslationCS{had, zmije} \dicIndirectTranslationCS{(zrádný člověk)}
\dicEntry[naðurtunga] \dicTerm{naður··tung|a} \dicIPA{{n}{a}{\textlengthmark}{ð}{\textscy}{\textsubring{r}}{t\smash{\textsuperscript{h}}}{u}{\ng}{\r{g}}{a}} \dicPos{f}[1] \dicFlx{(‑u, ‑ur)}[13] \dicFieldCat{bot.} \dicDirectTranslationCS{hadí jazyk azorský} \textit{(l.~{\textLA{Ophioglossum azoricum}})}
\dicEntry[nafar] \dicTerm{naf|ar\smash{\textsuperscript{1}}} \dicIPA{{n}{a}{\textlengthmark}{v}{a}{\textsubring{r}}} \dicPos{m}[5] \dicFlx{(‑ars, ‑rar)}[8] \dicDirectTranslationCS{vrták, nebozez}
\dicEntry[nafar] \dicTerm{nafar\smash{\textsuperscript{2}}} \dicIPA{{n}{a}{\textlengthmark}{v}{a}{\textsubring{r}}} \dicPos{f} \dicFlx{sg gen} \dicLink{nöf}
\dicEntry[nafir] \dicTerm{nafir} \dicIPA{{n}{a}{\textlengthmark}{v}{\textsci}{\textsubring{r}}} \dicPos{f} \dicFlx{pl nom} \dicLink{nöf}
\dicEntry[naflagras] \dicTerm{nafla··|gras} \dicIPA{{n}{a}{\textsubring{b}}{l}{a}{\r{g}}{r}{a}{s}} \dicPos{n}[2] \dicFlx{(‑grass, ‑grös)}[8] \dicFieldCat{bot.} \dicIndirectTranslationCS{rostlina z~čeledi rdesnovitých vyskytující se na Islandu} \textit{(l.~{\textLA{Koenigia islandica}})}
\dicEntry[naflastrengur] \dicTerm{nafla··streng|ur} \dicIPA{{n}{a}{\textsubring{b}}{l}{a}{s}{\textsubring{d}}{r}{ei}{\ng}{\r{g}}{\textscy}{\textsubring{r}}} \dicPos{m}[9] \dicFlx{(‑s\,/\addthin ‑jar, ‑ir)}[26] \dicFieldCat{anat.} \dicDirectTranslationCS{pupeční šňůra}
\dicEntry[nafli] \dicTerm{nafl|i} \dicIPA{{n}{a}{\textsubring{b}}{l}{\textsci}} \dicPos{m}[1] \dicFlx{(‑a, ‑ar)}[8] \dicFieldCat{anat.} \dicDirectTranslationCS{pupek};  \dicPhraseIS{vera nafli alheimsins} \dicLangCat{přen.} \dicDirectTranslationCS{být pupkem světa}
\dicEntry[nafn] \dicTerm{nafn} \dicsymFrequent\  \dicIPA{{n}{a}{\textsubring{b}}{\textsubring{n}}} \dicPos{n}[2] \dicFlx{(‑s, nöfn)}[8] \textbf{1.} \dicSynonym{heiti\smash{\textsuperscript{1}}} \dicDirectTranslationCS{jméno, pojmenování};  \dicPhraseIS{bera nafn með rentu} \dicLangCat{přen.} \dicDirectTranslationCS{dělat čest svému jménu};  \dicPhraseIS{gefa e‑m nafn} \dicDirectTranslationCS{pojmenovat (koho), dát (komu) jméno};  \dicPhraseIS{í nafni e‑rs} \dicDirectTranslationCS{ve jménu (koho), jménem (koho)}  \textbf{2.} \dicSynonym{nafnbót} \dicDirectTranslationCS{titul, hodnost}
\dicEntry[nafnakall] \dicTerm{nafna··|kall} \dicIPA{{n}{a}{\textsubring{b}}{n}{a}{k\smash{\textsuperscript{h}}}{a}{\textsubring{d}}{\textsubring{l}}} \dicPos{n}[2] \dicFlx{(‑kalls, ‑köll)}[8] \dicDirectTranslationCS{(zjišťování) prezence, kontrola účasti, čtení listiny přítomných} \dicExampleIS{Kennari gerir nafnakall.} \dicExampleCS{Učitel zjišťuje prezenci.}
\dicEntry[nafnalisti] \dicTerm{nafna··list|i} \dicIPA{{n}{a}{\textsubring{b}}{n}{a}{l}{\textsci}{s}{\textsubring{d}}{\textsci}} \dicPos{m}[1] \dicFlx{(‑a, ‑ar)}[1] \dicDirectTranslationCS{seznam jmen, jmenný seznam}
\dicEntry[nafnaskrá] \dicTerm{nafna··skrá} \dicIPA{{n}{a}{\textsubring{b}}{n}{a}{s}{\r{g}}{r}{au}} \dicPos{f}[4] \dicFlx{(‑r\,/\addthin ‑ar, ‑r)}[21] \dicDirectTranslationCS{seznam jmen, jmenný seznam}
\dicEntry[nafnbót] \dicTerm{nafn··|bót} \dicIPA{{n}{a}{\textsubring{b}}{\textsubring{n}}{\textsubring{b}}{ou}{\textsubring{d}}} \dicPos{f}[8] \dicFlx{(‑bótar, ‑bætur)}[5] \dicSynonym{titill} \dicDirectTranslationCS{titul, hodnost} \dicExampleIS{vera prófessor að nafnbót} \dicExampleCS{užívat titul profesora}
\dicEntry[nafndagur] \dicTerm{nafn··dag|ur} \dicIPA{{n}{a}{\textsubring{b}}{\textsubring{n}}{\textsubring{d}}{a}{\textbabygamma}{\textscy}{\textsubring{r}}} \dicPos{m}[6] \dicFlx{(‑s, ‑ar)}[62] \dicDirectTranslationCS{svátek, jmeniny}
\dicEntry[nafnfræði] \dicTerm{nafn··fræð|i} \dicIPA{{n}{a}{\textsubring{b}}{\textsubring{n}}{f}{r}{a}{i}{ð}{\textsci}} \dicPos{f}[3] \dicFlx{(‑i)}[3] \dicDirectTranslationCS{onomastika, onomatologie}
\dicEntry[nafnfrægur] \dicTerm{nafn··frægur} \dicIPA{{n}{a}{\textsubring{b}}{\textsubring{n}}{f}{r}{a}{i}{\textbabygamma}{\textscy}{\textsubring{r}}} \dicPos{adj}[1]\dicFlx{}[-1] \dicSynonym{víðkunnur} \dicDirectTranslationCS{proslulý\,/\addthin slavný (jménem)}
\dicEntry[nafngift] \dicTerm{nafn··gift} \dicIPA{{n}{a}{\textsubring{b}}{\textsubring{n}}{\r{\textObardotlessj}}{\textsci}{f}{\textsubring{d}}} \dicPos{f}[7] \dicFlx{(‑ar, ‑ir)}[1] \textbf{1.} \dicSynonym*{það að gefa nafn} \dicDirectTranslationCS{(po)jmenování}  \textbf{2.} \dicSynonym{nafnbót} \dicDirectTranslationCS{titul, hodnost}
\dicEntry[nafngreina] \dicTerm{nafn··grein|a} \dicIPA{{n}{a}{\textsubring{b}}{\textsubring{n}}{\r{g}}{r}{ei}{n}{a}} \dicPos{v}[2] \dicFlx{(‑di, ‑t)}[145] \dicFlx{acc} \textbf{1.} \dicSynonym*{nefna með nafni} \dicDirectTranslationCS{(po)jmenovat, nazvat (jménem), uvést\,/\addthin uvádět jménem} \dicExampleIS{Hún vildi ekki nafngreina manninn.} \dicExampleCS{Nechtěla jmenovat toho člověka.}  \textbf{2.} \dicSynonym*{ákvarða til tegundar} \dicDirectTranslationCS{určit, určovat (název rostliny ap.)}
\dicEntry[nafnháttarmerki] \dicTerm{nafn·háttar··merki} \dicIPA{{n}{a}{\textsubring{b}}{\textsubring{n}}{h}{au}{h}{\textsubring{d}}{a}{r}{m}{\textepsilon}{\textsubring{r}}{\r{\textObardotlessj}}{\textsci}} \dicPos{n}[2] \dicFlx{(‑s, ‑)}[16] \dicFieldCat{jaz.} \dicDirectTranslationCS{infinitivní částice} \dicIndirectTranslationCS{(v~islandštině \clqq að\crqq )}
\dicEntry[nafnháttur] \dicTerm{nafn··|háttur} \dicIPA{{n}{a}{\textsubring{b}}{\textsubring{n}}{h}{au}{h}{\textsubring{d}}{\textscy}{\textsubring{r}}} \dicPos{m}[12] \dicFlx{(‑háttar, ‑hættir)}[7] \dicFieldCat{jaz.} \dicDirectTranslationCS{infinitiv}
\dicEntry[nafni] \dicTerm{nafn|i} \dicIPA{{n}{a}{\textsubring{b}}{n}{\textsci}} \dicPos{m}[1] \dicFlx{(‑a, ‑ar)}[8] \dicDirectTranslationCS{jmenovec, jmenovkyně}
\dicEntry[nafnkunnur] \dicTerm{nafn··kunnur} \dicIPA{{n}{a}{\textsubring{b}}{\textsubring{n}}{k\smash{\textsuperscript{h}}}{\textscy}{n}{\textscy}{\textsubring{r}}} \dicPos{adj}[1]\dicFlx{}[-1] \dicSynonym{kunnur} \dicDirectTranslationCS{známý (podle jména)}
\dicEntry[nafnlaus] \dicTerm{nafn··laus} \dicIPA{{n}{a}{\textsubring{b}}{n}{l}{\oe i}{s}} \dicPos{adj}[5]\dicFlx{}[-1] \dicDirectTranslationCS{anonymní, bezejmenný}
\dicEntry[nafnleynd] \dicTerm{nafn··leynd} \dicIPA{{n}{a}{\textsubring{b}}{n}{l}{ei}{n}{\textsubring{d}}} \dicPos{f}[7] \dicFlx{(‑ar)}[3] \dicDirectTranslationCS{anonymita, bezejmennost}
\dicEntry[nafnliður] \dicTerm{nafn··lið|ur} \dicIPA{{n}{a}{\textsubring{b}}{n}{l}{\textsci}{ð}{\textscy}{\textsubring{r}}} \dicPos{m}[10] \dicFlx{(‑ar\,/\addthin ‑s, ‑ir)}[33] \dicFieldCat{jaz.} \dicDirectTranslationCS{jmenná fráze}
\dicEntry[nafnnúmer] \dicTerm{nafn··númer} \dicIPA{{n}{a}{\textsubring{b}}{\textsubring{n}}{n}{u}{m}{\textepsilon}{\textsubring{r}}} \dicPos{n}[2] \dicFlx{(‑s, ‑)}[5] \dicFieldCat{hist.} \dicIndirectTranslationCS{(rodné) identifikační číslo (platné na Islandu v~letech 1964--1987)}
\dicEntry[nafnorð] \dicTerm{nafn··orð} \dicIPA{{n}{a}{\textsubring{b}}{n}{\textopeno}{r}{\texttheta}} \dicPos{n}[2] \dicFlx{(‑s, ‑)}[5] \dicFieldCat{jaz.} \dicDirectTranslationCS{podstatné jméno, substantivum} \textit{(l.~{\textLA{nomen substantivum}})}
\dicEntry[nafnskilti] \dicTerm{nafn··skilti} \dicIPA{{n}{a}{\textsubring{b}}{\textsubring{n}}{s}{\r{\textObardotlessj}}{\textsci}{\textsubring{l}}{\textsubring{d}}{\textsci}} \dicPos{n}[2] \dicFlx{(‑s, ‑)}[14] \dicDirectTranslationCS{jmenovka, štítek se jménem}
\dicEntry[nafnskipti] \dicTerm{nafn··skipti} \dicIPA{{n}{a}{\textsubring{b}}{\textsubring{n}}{s}{\r{\textObardotlessj}}{\textsci}{f}{\textsubring{d}}{\textsci}} \dicPos{n}[2] \dicFlx{pl}[19] \dicFieldCat{lit.\,/\addthin jaz.} \dicDirectTranslationCS{metonymie}
\dicEntry[nafnskírteini] \dicTerm{nafn··skír·teini} \dicIPA{{n}{a}{\textsubring{b}}{\textsubring{n}}{s}{\r{\textObardotlessj}}{i}{\textsubring{r}}{t\smash{\textsuperscript{h}}}{ei}{n}{\textsci}} \dicPos{n}[2] \dicFlx{(‑s, ‑)}[14] \dicFieldCat{práv.} \dicDirectTranslationCS{rodný list}
\dicEntry[nafnspjald] \dicTerm{nafn··|spjald} \dicIPA{{n}{a}{\textsubring{b}}{\textsubring{n}}{s}{\textsubring{b}}{j}{a}{l}{\textsubring{d}}} \dicPos{n}[2] \dicFlx{(‑spjalds, ‑spjöld)}[8] \dicDirectTranslationCS{navštívenka, vizitka}
\dicEntry[nafntogaður] \dicTerm{nafn··tog·|aður} \dicIPA{{n}{a}{\textsubring{b}}{\textsubring{n}}{t\smash{\textsuperscript{h}}}{\textopeno}{\textbabygamma}{a}{ð}{\textscy}{\textsubring{r}}} \dicPos{adj}[3] \dicFlx{(f ‑uð)}[3] \dicDirectTranslationCS{uznávaný, renomovaný, vyhlášený} \dicExampleIS{nafntogaður læknir} \dicExampleCS{renomovaný doktor}
\dicEntry[nag] \dicTerm{nag} \dicIPA{{n}{a}{\textlengthmark}{x}} \dicPos{n}[2] \dicFlx{(‑s)}[2] \dicDirectTranslationCS{hlodání, hryzání, kousání}
\dicEntry[naga] \dicTerm{nag|a} \dicIPA{{n}{a}{\textlengthmark}{\textbabygamma}{a}} \dicPos{v}[1] \dicFlx{(‑aði)}[13] \dicFlx{acc} \dicDirectTranslationCS{hlodat, hryzat, kousat, okusovat} \dicExampleIS{naga gat á dúkinn} \dicExampleCS{vykousat díru do ubrusu};  \dicPhraseIS{naga á sér neglurnar} \dicDirectTranslationCS{kousat si nehty}
\dicEntry[nagdýr] \dicTerm{nag··dýr} \dicIPA{{n}{a}{\textbabygamma}{\textsubring{d}}{i}{\textsubring{r}}} \dicPos{n}[2] \dicFlx{(‑s, ‑)}[5] \dicFieldCat{zool.} \dicDirectTranslationCS{hlodavec} \textit{(l.~{\textLA{Rodentia}})}
\dicEntry[nagg] \dicTerm{nagg} \dicIPA{{n}{a}{\r{g}}{\textlengthmark}} \dicPos{n}[2] \dicFlx{(‑s)}[2] \dicSynonym{rifrildi} \dicDirectTranslationCS{hašteření, hádka}
\dicEntry[naggrís] \dicTerm{nag··grís} \dicIPA{{n}{a}{\r{g}}{r}{i}{s}} \dicPos{m}[9] \dicFlx{(‑s, ‑ir)}[22] \dicFieldCat{zool.} \dicDirectTranslationCS{morče domácí} \textit{(l.~{\textLA{Cavia aperea f. porcelus}})}
\dicEntry[naglalakk] \dicTerm{nagla··|lakk} \dicIPA{{n}{a}{\r{g}}{l}{a}{l}{a}{h}{\r{g}}} \dicPos{n}[2] \dicFlx{(‑lakks, ‑lökk)}[8] \dicDirectTranslationCS{lak na nehty}
\dicEntry[naglar] \dicTerm{naglar} \dicIPA{{n}{a}{\r{g}}{l}{a}{\textsubring{r}}} \dicPos{f} \dicFlx{sg gen} \dicLink{nögl}
\dicEntry[naglaskapur] \dicTerm{nagla··skap|ur} \dicIPA{{n}{a}{\r{g}}{l}{a}{s}{\r{g}}{a}{\textsubring{b}}{\textscy}{\textsubring{r}}} \dicPos{m}[10] \dicFlx{(‑ar)}[15] \textbf{1.} \dicSynonym{níska} \dicDirectTranslationCS{lakota, lakomství}  \textbf{2.} \dicSynonym*{kvikindisháttur} \dicDirectTranslationCS{zvířeckost, bestialita}
\dicEntry[naglaþjöl] \dicTerm{nagla··|þjöl} \dicIPA{{n}{a}{\r{g}}{l}{a}{\texttheta}{j}{\oe}{\textsubring{l}}} \dicPos{f}[7] \dicFlx{(‑þjalar, ‑þjalir)}[16] \dicDirectTranslationCS{pilník\,/\addthin pilníček na nehty}
\dicEntry[naglbítur] \dicTerm{nagl··bít|ur} \dicIPA{{n}{a}{\r{g}}{\textsubring{l}}{\textsubring{b}}{i}{\textsubring{d}}{\textscy}{\textsubring{r}}} \dicPos{m}[6] \dicFlx{(‑s, ‑ir\,/\addthin ‑ar)}[70] \dicDirectTranslationCS{štípací kleště, štípačky}
\dicEntry[naglfastur] \dicTerm{nagl··|fastur} \dicIPA{{n}{a}{\r{g}}{\textsubring{l}}{f}{a}{s}{\textsubring{d}}{\textscy}{\textsubring{r}}} \dicPos{adj}[1] \dicFlx{(f ‑föst)}[12] \dicDirectTranslationCS{přibitý\,/\addthin připevněný hřebíky} \dicExampleIS{Rúmið er naglfast.} \dicExampleCS{Postel je připevněná hřebíky.}
\dicEntry[nagli] \dicTerm{nagl|i} \dicIPA{{n}{a}{\r{g}}{l}{\textsci}} \dicPos{m}[1] \dicFlx{(‑a, ‑ar)}[8] \dicDirectTranslationCS{hřebík} \dicExampleIS{reka naglann í spýtuna} \dicExampleCS{zatlouct hřebík do prkna};  \dicPhraseIS{hitta naglann á höfuðið} \dicLangCat{přen.} \dicDirectTranslationCS{uhodit hřebíček na hlavičku}
\dicEntry[Naíróbí] \dicTerm{Naíróbí} \dicIPA{{n}{a}{i}{r}{ou}{\textsubring{b}}{i}} \dicPos{subs} \dicFlx{indecl} \dicFieldCat{geog.} \dicDirectTranslationCS{Nairobi} \dicIndirectTranslationCS{(hlavní město Keni)}
\dicEntry[nakinn] \dicTerm{nakinn} \dicsymFrequent\  \dicIPA{{n}{a}{\textlengthmark}{\r{\textObardotlessj}}{\textsci}{\textsubring{n}}} \dicPos{adj}[6]\dicFlx{}[-12] \textbf{1.} \dicSynonym{ber\smash{\textsuperscript{3}}} \dicDirectTranslationCS{nahý, obnažený, nahatý} \dicExampleIS{nakinn maður} \dicExampleCS{nahý člověk}  \textbf{2.} \dicSynonym*{blaðlaus} \dicDirectTranslationCS{holý, bezlistý, (jsoucí) bez listí (strom ap.)}  \textbf{3.} \dicSynonym{gróðurlaus} \dicDirectTranslationCS{holý, lysý, neporostlý (skála ap.)}  \textbf{4.} \dicSynonym{einber} \dicDirectTranslationCS{holý, čirý}
\dicEntry[nam] \dicTerm{nam} \dicIPA{{n}{a}{\textlengthmark}{\textsubring{m}}} \dicPos{v} \dicFlx{ind pf sg 1 pers} \dicLink{nema\smash{\textsuperscript{1}}}
\dicEntry[Namibi] \dicTerm{Namib|i} \dicIPA{{n}{a}{\textlengthmark}{m}{\textsci}{\textsubring{b}}{\textsci}} \dicPos{m}[1] \dicFlx{(‑a, ‑ar)}[1] \dicLink{Namibíumaður}
\dicEntry[Namibía] \dicTerm{Namibí|a} \dicIPA{{n}{a}{\textlengthmark}{m}{\textsci}{\textsubring{b}}{i}{j}{a}} \dicPos{f}[1] \dicFlx{(‑u)}[6] \dicFieldCat{geog.} \dicDirectTranslationCS{Namibie}
\dicEntry[namibískur] \dicTerm{namibískur} \dicIPA{{n}{a}{\textlengthmark}{m}{\textsci}{\textsubring{b}}{i}{s}{\r{g}}{\textscy}{\textsubring{r}}} \dicPos{adj}[1]\dicFlx{}[-1] \dicDirectTranslationCS{namibijský}
\dicEntry[Namibíumaður] \dicTerm{Namibíu··|maður}\dicTerm{, Namibi} \dicIPA{{n}\-{a}\-{\textlengthmark}\-{m}\-{\textsci}\-{\textsubring{b}}\-{i}\-{j}\-{\textscy}\-{m}\-{a}\-{ð}\-{\textscy}\-{\textsubring{r}}\-} \dicPos{m}[13] \dicFlx{(‑manns, ‑menn)}[2] \dicDirectTranslationCS{Namibijec, Namibijka}
\dicEntry[namm] \dicTerm{namm} \dicIPA{{n}{a}{m}{\textlengthmark}} \dicPos{inter} \dicDirectTranslationCS{mňam} \dicIndirectTranslationCS{(citoslovce vyjadřující radost z~dobrého jídla ap.)}
\dicEntry[nammi] \dicTerm{nammi} \dicIPA{{n}{a}{m}{\textlengthmark}{\textsci}} \dicPos{n}[2] \dicFlx{(‑s)}[20] \dicLangCat{dět.} \dicSynonym{sælgæti} \dicDirectTranslationCS{sladkosti, bonbóny, dobroty}
\dicEntry[nappa] \dicTerm{napp|a} \dicIPA{{n}{a}{h}{\textsubring{b}}{a}} \dicPos{v}[1] \dicFlx{(‑aði)}[13] \dicFlx{dat\,/\addthin acc} \textbf{1.} \dicFlx{dat} \dicLangCat{hovor.} \dicSynonym{hnupla} \dicDirectTranslationCS{štípnout, otočit, ukrást} \dicExampleIS{nappa e‑u með sér} \dicExampleCS{štípnout (co)}  \textbf{2.} \dicFlx{acc} \dicLangCat{hovor.} \dicSynonym{handtaka\smash{\textsuperscript{2}}} \dicDirectTranslationCS{lapnout, dopadnout (zloděje ap.)} \dicExampleIS{Lögreglan nappaði þjófana í fyrradag.} \dicExampleCS{Policie dopadla zloděje předevčírem.}
\dicEntry[napur] \dicTerm{napur} \dicIPA{{n}{a}{\textlengthmark}{\textsubring{b}}{\textscy}{\textsubring{r}}} \dicPos{adj}[9] \dicFlx{(f nöpur)}[2] \textbf{1.} \dicSynonym*{sárkaldur} \dicDirectTranslationCS{ledový, mrazivý, kousavý (zima ap.)} \dicExampleIS{napur vindur} \dicExampleCS{ledový vítr}  \textbf{2.} \dicSynonym*{bituryrtur} \dicDirectTranslationCS{sarkastický, jízlivý} \dicExampleIS{napurt svar} \dicExampleCS{jízlivá odpověď}
\dicEntry[narra] \dicTerm{narr|a} \dicIPA{{n}{a}{r}{\textlengthmark}{a}} \dicPos{v}[1] \dicFlx{(‑aði)}[13] \dicFlx{acc} \dicSynonym{gabba} \dicDirectTranslationCS{(o)balamutit, (vy)mámit, (o)šálit} \dicExampleIS{narra út úr e‑m peninga} \dicExampleCS{vymámit z\,/\addthin od (koho) peníze}
\dicEntry[nart] \dicTerm{nart} \dicIPA{{n}{a}{\textsubring{r}}{\textsubring{d}}} \dicPos{n}[2] \dicFlx{(‑s)}[2] \textbf{1.} \dicSynonym{nag} \dicDirectTranslationCS{kousnutí, kousání}  \textbf{2.} \dicSynonym*{baktal} \dicDirectTranslationCS{pomluva, klep}
\dicEntry[narta] \dicTerm{nart|a} \dicIPA{{n}{a}{\textsubring{r}}{\textsubring{d}}{a}} \dicPos{v}[1] \dicFlx{(‑aði)}[13] \dicFlx{acc} \dicSynonym{naga} \dicDirectTranslationCS{hryzat, hlodat, okusovat} \dicExampleIS{narta grasið} \dicExampleCS{okusovat trávu};  \dicIdiom{narta}[í]{ \dicPhraseIS{narta í e‑ð}} \dicDirectTranslationCS{zakousnout se do (čeho) (chleba ap.)}; { \dicPhraseIS{narta í e‑n}} \dicSynonym*{tala illa um e‑n} \dicDirectTranslationCS{pomluvit\,/\addthin pomlouvat (koho)}
\dicEntry[nasa] \dicTerm{nas|a} \dicIPA{{n}{a}{\textlengthmark}{s}{a}} \dicPos{v}[1] \dicFlx{(‑aði)}[13] \textbf{1.} \dicSynonym{þefa} \dicDirectTranslationCS{přičichnout si, čichat} \dicExampleIS{Hundurinn nasar af sporum.} \dicExampleCS{Pes čichá stopu.};  \dicPhraseIS{nasa upp í loftið} \dicDirectTranslationCS{větřit}  \textbf{2.} \dicDirectTranslationCS{přičichnout si, okusit (kouzla matematiky ap.)} \dicIndirectTranslationCS{(obeznámit se zběžně)} \dicExampleIS{nasa af bók} \dicExampleCS{přičichnout si ke knížce}
\dicEntry[nasar] \dicTerm{nasar} \dicIPA{{n}{a}{\textlengthmark}{s}{a}{\textsubring{r}}} \dicPos{f} \dicFlx{sg gen} \dicLink{nös}
\dicEntry[nasasjón] \dicTerm{nasa··sjón} \dicIPA{{n}{a}{\textlengthmark}{s}{a}{s}{j}{ou}{\textsubring{n}}} \dicPos{f}[7] \dicFlx{(‑ar)}[3] \dicDirectTranslationCS{ponětí};  \dicPhraseIS{fá nasasjón af e‑u} \dicDirectTranslationCS{získat ponětí o~(čem), přičichnout si k~(čemu)}
\dicEntry[nasavængur] \dicTerm{nasa··væng|ur} \dicIPA{{n}{a}{\textlengthmark}{s}{a}{v}{a}{i}{\ng}{\r{g}}{\textscy}{\textsubring{r}}} \dicPos{m}[9] \dicFlx{(‑s\,/\addthin ‑jar, ‑ir)}[26] \dicFieldCat{anat.} \dicDirectTranslationCS{nosní dírky, nozdry, chřípí}
\dicEntry[nashyrningur] \dicTerm{nas··hyrn·ing|ur} \dicIPA{{n}{a}{\textlengthmark}{s}{h}{\textsci}{r}{\textsubring{d}}{n}{i}{\ng}{\r{g}}{\textscy}{\textsubring{r}}} \dicPos{m}[6] \dicFlx{(‑s, ‑ar)}[8] \dicFieldCat{zool.} \dicDirectTranslationCS{nosorožec} \textit{(l.~{\textLA{Rhinoceros}})}  \dicsymPhoto\ 
\dicFigure{ds_image_nashyrningur_0_1.jpg}{Nashyrningur}{Nashyrningur - Ikiwaner, GFDL v1.2}
\dicEntry[nasir] \dicTerm{nasir} \dicIPA{{n}{a}{\textlengthmark}{s}{\textsci}{\textsubring{r}}} \dicPos{f} \dicFlx{pl nom} \dicLink{nös}
\dicEntry[nasismi] \dicTerm{nas··ism|i} \dicIPA{{n}{a}{\textlengthmark}{s}{\textsci}{s}{m}{\textsci}} \dicPos{m}[1] \dicFlx{(‑a)}[3] \dicDirectTranslationCS{nacismus}
\dicEntry[nasisti] \dicTerm{nas··ist|i} \dicIPA{{n}{a}{\textlengthmark}{s}{\textsci}{s}{\textsubring{d}}{\textsci}} \dicPos{m}[1] \dicFlx{(‑a, ‑ar)}[1] \dicDirectTranslationCS{nacista, nacistka}
\dicEntry[naskur] \dicTerm{naskur} \dicIPA{{n}{a}{s}{\r{g}}{\textscy}{\textsubring{r}}} \dicPos{adj}[1] \dicFlx{(f nösk)}[2] \textbf{1.} \dicSynonym{laginn} \dicDirectTranslationCS{zručný, šikovný}  \textbf{2.} \dicSynonym{sniðugur} \dicDirectTranslationCS{prohnaný, mazaný} \dicExampleIS{Hún er býsna nösk að finna réttu svörin.} \dicExampleCS{Je pěkně mazaná, když má nalézt správnou odpověď.}
\dicEntry[nasl] \dicTerm{nasl} \dicIPA{{n}{a}{s}{\textsubring{d}}{\textsubring{l}}} \dicPos{n}[2] \dicFlx{(‑s)}[2] \dicSynonym{snarl} \dicDirectTranslationCS{svačina, přesnídávka}
\dicEntry[nasla] \dicTerm{nasl|a} \dicIPA{{n}{a}{s}{\textsubring{d}}{l}{a}} \dicPos{v}[1] \dicFlx{(‑aði)}[13] \dicSynonym{bíta} \dicDirectTranslationCS{kousat, ukusovat, zakousnout} \dicExampleIS{e‑að til að nasla} \dicExampleCS{(co) k~zakousnutí}
\dicEntry[natinn] \dicTerm{natinn} \dicIPA{{n}{a}{\textlengthmark}{\textsubring{d}}{\textsci}{\textsubring{n}}} \dicPos{adj}[6]\dicFlx{}[-3] \textbf{1.} \dicSynonym{umhyggjusamur} \dicDirectTranslationCS{pozorný, pečlivý, obezřetný} \dicExampleIS{vera natinn við skepnur} \dicExampleCS{být pozorný ke zvířatům}  \textbf{2.} \dicSynonym{iðjusamur} \dicDirectTranslationCS{přičinlivý, pracovitý}
\dicEntry[natni] \dicTerm{natn|i} \dicIPA{{n}{a}{h}{\textsubring{d}}{n}{\textsci}} \dicPos{f}[3] \dicFlx{(‑i)}[3] \textbf{1.} \dicSynonym{umhyggjusemi} \dicDirectTranslationCS{pozornost, pečlivost, obezřetnost}  \textbf{2.} \dicSynonym{iðjusemi} \dicDirectTranslationCS{přičinlivost, pracovitost, píle} \dicExampleIS{vinna heimilisstörfin af mikilli natni} \dicExampleCS{vykonávat domácí práce s~velkou pílí}
\dicEntry[natrín] \dicTerm{natrín} \dicIPA{{n}{a}{\textsubring{d}}{r}{i}{\textsubring{n}}} \dicPos{n}[2] \dicFlx{(‑s)}[2] \dicLink{natríum}
\dicEntry[natríum] \dicTerm{natríum}\dicTerm{, natrín} \dicIPA{{n}\-{a}\-{\textsubring{d}}\-{r}\-{i}\-{j}\-{\textscy}\-{\textsubring{m}}\-} \dicPos{n}[2] \dicFlx{(‑s)}[31] \dicFieldCat{chem.} \dicDirectTranslationCS{sodík} \textit{(l.~{\textLA{Na, Natrium}})}
\dicEntry[natúralismi] \dicTerm{natúral··ism|i} \dicIPA{{n}{a}{\textlengthmark}{\textsubring{d}}{u}{r}{a}{l}{\textsci}{s}{m}{\textsci}} \dicPos{m}[1] \dicFlx{(‑a)}[3] \dicDirectTranslationCS{naturalismus}
\dicEntry[natúralisti] \dicTerm{natúral··ist|i} \dicIPA{{n}{a}{\textlengthmark}{\textsubring{d}}{u}{r}{a}{l}{\textsci}{s}{\textsubring{d}}{\textsci}} \dicPos{m}[1] \dicFlx{(‑a, ‑ar)}[1] \dicDirectTranslationCS{naturalista, naturalistka, stoupenec\,/\addthin stoupenkyně naturalismu}
\dicEntry[nauð] \dicTerm{nauð\smash{\textsuperscript{1}}} \dicIPA{{n}{\oe i}{\textlengthmark}{\texttheta}} \dicPos{f}[7] \dicFlx{(‑ar, ‑ir)}[1] \textbf{1.} \dicSynonym{neyð} \dicDirectTranslationCS{dilema, tíseň, nesnáz};  \dicPhraseIS{ef í nauðirnar rekur} \dicDirectTranslationCS{jestliže dojde na nejhorší};  \dicPhraseIS{vera í nauðum staddur} \dicDirectTranslationCS{být v~nesnázích\,/\addthin tísni}  \textbf{2.} \dicSynonym{fjöldi} \dicDirectTranslationCS{spousta, nával (lidí ap.)} \dicExampleIS{gestanauð} \dicExampleCS{spousta hostů}
\dicEntry[nauð] \dicTerm{nauð\smash{\textsuperscript{2}}} \dicIPA{{n}{\oe i}{\textlengthmark}{\texttheta}} \dicPos{n}[2] \dicFlx{(‑s)}[2] \textbf{1.} \dicSynonym{gnýr\smash{\textsuperscript{1}}} \dicDirectTranslationCS{lomoz, hřmot}  \textbf{2.} \dicSynonym{þrábeiðni} \dicDirectTranslationCS{přemlouvání, ukecávání, hučení} \dicExampleIS{hlusta á nauðið í honum} \dicExampleCS{poslouchat jeho přemlouvání}  \textbf{3.} \dicSynonym{núningur} \dicDirectTranslationCS{(soustavné) tření} \dicExampleIS{Nauð vinda og vatns sverfur björg.} \dicExampleCS{Tření větru a~vody drolí skály.}
\dicEntry[nauða] \dicTerm{nauð|a} \dicIPA{{n}{\oe i}{\textlengthmark}{ð}{a}} \dicPos{v}[1] \dicFlx{(‑aði)}[44] \textbf{1.} \dicSynonym*{gera hávaða} \dicDirectTranslationCS{působit hluk\,/\addthin lomoz, hlučet}  \textbf{2.} \dicSynonym{þrábiðja} \dicDirectTranslationCS{přemlouvat, namlouvat, hučet} \dicExampleIS{nauða í\,/\addthin á e‑m} \dicExampleCS{hučet do (koho)}
\dicEntry[nauðalíkur] \dicTerm{nauða··líkur} \dicIPA{{n}{\oe i}{\textlengthmark}{ð}{a}{l}{i}{\r{g}}{\textscy}{\textsubring{r}}} \dicPos{adj}[1]\dicFlx{}[-6] \dicDirectTranslationCS{velmi podobný} \dicExampleIS{Þær eru nauðalíkar í útliti.} \dicExampleCS{Vzhledově jsou si velmi podobné.}
\dicEntry[nauðbeygður] \dicTerm{nauð··beygður} \dicIPA{{n}{\oe i}{ð}{\textsubring{b}}{ei}{\textbabygamma}{ð}{\textscy}{\textsubring{r}}} \dicPos{adj}[2]\dicFlx{}[-1] \dicDirectTranslationCS{nucený, přinucený} \dicExampleIS{Ég er nauðbeygður til að gera það.} \dicExampleCS{Jsem nucen to udělat.}
\dicEntry[nauðga] \dicTerm{nauðg|a} \dicIPA{{n}{\oe i}{ð}{\r{g}}{a}} \dicPos{v}[1] \dicFlx{(‑aði)}[44] \dicFlx{dat} \textbf{1.} \dicSynonym{neyða} \dicDirectTranslationCS{(při)nutit, donutit} \dicExampleIS{nauðga e‑m til að gera e‑ð} \dicExampleCS{donutit (koho) k~dělání (čeho)}  \textbf{2.} \dicDirectTranslationCS{znásilnit (ženu ap.)} \dicExampleIS{nauðga konu} \dicExampleCS{znásilnit ženu}
\dicEntry[nauðgari] \dicTerm{nauðg··ar|i} \dicIPA{{n}{\oe i}{ð}{\r{g}}{a}{r}{\textsci}} \dicPos{m}[1] \dicFlx{(‑a, ‑ar)}[13] \dicDirectTranslationCS{pachatel(ka) znásilnění}
\dicEntry[nauðgun] \dicTerm{nauðg|un} \dicIPA{{n}{\oe i}{ð}{\r{g}}{\textscy}{\textsubring{n}}} \dicPos{f}[7] \dicFlx{(‑unar, ‑anir)}[8] \dicDirectTranslationCS{znásilnění}
\dicEntry[nauðhyggja] \dicTerm{nauð··hyggj|a} \dicIPA{{n}{\oe i}{\textlengthmark}{\texttheta}{h}{\textsci}{\r{\textObardotlessj}}{a}} \dicPos{f}[1] \dicFlx{(‑u)}[5] \dicFieldCat{filos.} \dicDirectTranslationCS{determinismus}
\dicEntry[nauðlenda] \dicTerm{nauð··len|da} \dicIPA{{n}{\oe i}{ð}{l}{\textepsilon}{n}{\textsubring{d}}{a}} \dicPos{v}[2] \dicFlx{(‑ti, ‑t)}[43] \dicFlx{dat} \dicDirectTranslationCS{nouzově přistát\,/\addthin přistávat} \dicExampleIS{nauðlenda flugvél} \dicExampleCS{nouzově přistát s~letadlem}
\dicEntry[nauðlending] \dicTerm{nauð··lend·ing} \dicIPA{{n}{\oe i}{ð}{l}{\textepsilon}{n}{\textsubring{d}}{i}{\ng}{\r{g}}} \dicPos{f}[4] \dicFlx{(‑ar, ‑ar)}[5] \dicDirectTranslationCS{nouzové přistání}
\dicEntry[nauðrakaður] \dicTerm{nauð··|rak·aður} \dicIPA{{n}{\oe i}{ð}{r}{a}{\r{g}}{a}{ð}{\textscy}{\textsubring{r}}} \dicPos{adj}[3] \dicFlx{(f ‑rökuð)}[2] \dicDirectTranslationCS{hladce oholený}
\dicEntry[nauðsyn] \dicTerm{nauð··syn} \dicsymFrequent\  \dicIPA{{n}{\oe i}{ð}{s}{\textsci}{\textsubring{n}}} \dicPos{f}[4] \dicFlx{(‑jar, ‑jar)}[9] \textbf{1.} \dicSynonym{neyð} \dicDirectTranslationCS{nutnost, nezbytnost};  \dicPhraseIS{af nauðsyn} \dicFlx{adv} \dicDirectTranslationCS{z~nutnosti};  \dicPhraseIS{brýn nauðsyn} \dicDirectTranslationCS{naprostá nutnost} \dicExampleIS{brýn nauðsyn til að setja bráðabirgðalög} \dicExampleCS{naprostá nutnost ustavit prozatímní zákon};  \dicPhraseIS{ef nauðsyn krefur} \dicDirectTranslationCS{jestliže to nutnost vyžaduje}  \textbf{2.} \dicPhraseIS{nauðsynjar} \dicFlx{pl} \dicDirectTranslationCS{základní potřeby}
\dicEntry[nauðsynlegur] \dicTerm{nauð·syn··legur} \dicsymFrequent\  \dicIPA{{n}{\oe i}{ð}{s}{\textsci}{n}{l}{\textepsilon}{\textbabygamma}{\textscy}{\textsubring{r}}} \dicPos{adj}[1]\dicFlx{}[-8] \dicDirectTranslationCS{nutný, nezbytný} \dicExampleIS{Svefn er nauðsynlegur fyrir minnið.} \dicExampleCS{Spánek je nezbytný pro paměť.}
\dicEntry[nauðugur] \dicTerm{nauð··ugur} \dicIPA{{n}{\oe i}{\textlengthmark}{ð}{\textscy}{\textbabygamma}{\textscy}{\textsubring{r}}} \dicPos{adj}[1]\dicFlx{}[-8] \dicSynonym{tilneyddur} \dicDirectTranslationCS{(při)nucený, donucený} \dicExampleIS{gera e‑ð nauðugur} \dicExampleCS{dělat (co) z~donucení};  \dicPhraseIS{e‑m er nauðugur einn kostur} \dicFlx{impers} \dicDirectTranslationCS{(kdo) nemá na výběr};  \dicPhraseIS{nauðugur viljugur} \dicFlx{adv} \dicDirectTranslationCS{chtě nechtě, volky nevolky}
\dicEntry[nauðung] \dicTerm{nauð··ung} \dicIPA{{n}{\oe i}{\textlengthmark}{ð}{u}{\ng}{\r{g}}} \dicPos{f}[4] \dicFlx{(‑ar, ‑ar)}[1] \dicDirectTranslationCS{(ná)tlak, (do)nucení, přinucení} \dicExampleIS{beita e‑n nauðung} \dicExampleCS{vystavit (koho) nátlaku}
\dicEntry[nauðungarsala] \dicTerm{nauðungar··|sala} \dicIPA{{n}{\oe i}{\textlengthmark}{ð}{u}{\ng}{\r{g}}{a}{\textsubring{r}}{s}{a}{l}{a}} \dicPos{f}[1] \dicFlx{(‑sölu, ‑sölur)}[20] \dicFieldCat{práv.} \dicDirectTranslationCS{nucený prodej}
\dicEntry[nauðungaruppboð] \dicTerm{nauðungar··upp·boð} \dicIPA{{n}{\oe i}{\textlengthmark}{ð}{u}{\ng}{\r{g}}{a}{r}{\textscy}{h}{\textsubring{b}}{\textopeno}{\texttheta}} \dicPos{n}[2] \dicFlx{(‑s, ‑)}[5] \dicFieldCat{práv.} \dicDirectTranslationCS{nucená dražba}
\dicEntry[nauður] \dicTerm{nauð|ur} \dicIPA{{n}{\oe i}{\textlengthmark}{ð}{\textscy}{\textsubring{r}}} \dicPos{f}[7] \dicFlx{(‑ar, ‑ir)}[1] \dicSynonym{nauðsyn} \dicDirectTranslationCS{nutnost, nezbytnost};  \dicPhraseIS{ef mig rekur nauður til} \dicDirectTranslationCS{pokud budu nucen}
\dicEntry[nauðvörn] \dicTerm{nauð··|vörn} \dicIPA{{n}{\oe i}{ð}{v}{\oe}{r}{\textsubring{d}}{\textsubring{n}}} \dicPos{f}[7] \dicFlx{(‑varnar)}[19] \dicSynonym*{neyðarvörn} \dicDirectTranslationCS{sebeobrana}
\dicEntry[nauðþekkja] \dicTerm{nauð··þekk|ja} \dicIPA{{n}{\oe i}{\texttheta}{\textlengthmark}{\textepsilon}{h}{\r{\textObardotlessj}}{a}} \dicPos{v}[2] \dicFlx{(‑ti, ‑t)}[28] \dicFlx{acc} \dicDirectTranslationCS{znát velmi dobře, znát jako vlastní boty} \dicExampleIS{nauðþekkja rit} \dicExampleCS{znát dílo jako vlastní boty}
\dicEntry[naumast] \dicTerm{naum|ast} \dicIPA{{n}{\oe i}{\textlengthmark}{m}{a}{s}{\textsubring{d}}} \dicPos{adv} \dicFlx{sup (pos ‑t, comp ‑ar)} \dicSynonym{varla} \dicDirectTranslationCS{sotva, stěží};  \dicPhraseIS{það er naumast!} \dicDirectTranslationCS{já snad padnu!, to není možné!}
\dicEntry[naumindi] \dicTerm{naum··indi} \dicIPA{{n}{\oe i}{\textlengthmark}{m}{\textsci}{n}{\textsubring{d}}{\textsci}} \dicPos{n}[2] \dicFlx{pl}[19] \dicPhraseIS{með naumindum} \dicFlx{adv} \dicDirectTranslationCS{sotva, (jen) tak tak}
\dicEntry[naumraddaður] \dicTerm{naum··|radd·aður} \dicIPA{{n}{\oe i}{m}{r}{a}{\textsubring{d}}{a}{ð}{\textscy}{\textsubring{r}}} \dicPos{adj}[3] \dicFlx{(f ‑rödduð)}[1] \dicFieldCat{jaz.} \dicDirectTranslationCS{částečně znělý}
\dicEntry[naumröddun] \dicTerm{naum··rödd|un} \dicIPA{{n}{\oe i}{m}{r}{\oe}{\textsubring{d}}{\textscy}{\textsubring{n}}} \dicPos{f}[7] \dicFlx{(‑unar)}[12] \dicFieldCat{jaz.} \dicDirectTranslationCS{částečná znělost}
\dicEntry[naumt] \dicTerm{naum|t} \dicsymFrequent\  \dicIPA{{n}{\oe i}{\textsubring{m}}{\textsubring{d}}} \dicPos{adv} \dicFlx{(comp ‑ar, sup ‑ast)} \dicSynonym{lítt} \dicDirectTranslationCS{stěží, sotva, (jen) tak tak} \dicExampleIS{það er naumt um e‑ð hjá e‑m} \dicExampleCS{(čeho) u~(koho) není nazbyt}
\dicEntry[naumur] \dicTerm{naumur} \dicIPA{{n}{\oe i}{\textlengthmark}{m}{\textscy}{\textsubring{r}}} \dicPos{adj}[1]\dicFlx{}[-1] \textbf{1.} \dicSynonym{tæpur} \dicDirectTranslationCS{nedostatečný, skrovný};  \dicPhraseIS{hafa nauman tíma} \dicDirectTranslationCS{mít málo času}  \textbf{2.} \dicSynonym{nískur} \dicDirectTranslationCS{skoupý, skrblivý} \dicExampleIS{vera naumur í útlátum} \dicExampleCS{být skoupý ve výdajích}
\dicEntry[naust] \dicTerm{naust} \dicIPA{{n}{\oe i}{s}{\textsubring{d}}} \dicPos{n}[2] \dicFlx{(‑s, ‑)}[5] \dicDirectTranslationCS{kůlna na čluny} \dicIndirectTranslationCS{(často bez střechy)}
\dicEntry[naut] \dicTerm{naut\smash{\textsuperscript{1}}} \dicIPA{{n}{\oe i}{\textlengthmark}{\textsubring{d}}} \dicPos{n}[2] \dicFlx{(‑s, ‑)}[5] \textbf{1.} \dicDirectTranslationCS{býk}  \textbf{2.} \dicSynonym{auli} \dicDirectTranslationCS{blbec, vůl}  \textbf{3.} \dicDirectTranslationCS{Býk} \dicIndirectTranslationCS{(znamení zvěrokruhu)} \textit{(l.~{\textLA{Taurus}})}
\dicEntry[naut] \dicTerm{naut\smash{\textsuperscript{2}}} \dicIPA{{n}{\oe i}{\textlengthmark}{\textsubring{d}}} \dicPos{v} \dicFlx{ind pf sg 1 pers} \dicLink{njóta}
\dicEntry[nautaat] \dicTerm{nauta··|at} \dicIPA{{n}{\oe i}{\textlengthmark}{\textsubring{d}}{a}{a}{\textsubring{d}}} \dicPos{n}[2] \dicFlx{(‑ats, ‑öt)}[8] \dicDirectTranslationCS{býčí zápasy, korida}
\dicEntry[nautabani] \dicTerm{nauta··ban|i} \dicIPA{{n}{\oe i}{\textlengthmark}{\textsubring{d}}{a}{\textsubring{b}}{a}{n}{\textsci}} \dicPos{m}[1] \dicFlx{(‑a, ‑ar)}[8] \dicDirectTranslationCS{toreador(ka)}
\dicEntry[nautahakk] \dicTerm{nauta··hakk} \dicIPA{{n}{\oe i}{\textlengthmark}{\textsubring{d}}{a}{h}{a}{h}{\r{g}}} \dicPos{n}[2] \dicFlx{(‑s)}[2] \dicDirectTranslationCS{sekané hovězí maso}
\dicEntry[nautakjöt] \dicTerm{nauta··kjöt} \dicIPA{{n}{\oe i}{\textlengthmark}{\textsubring{d}}{a}{c\smash{\textsuperscript{h}}}{\oe}{\textsubring{d}}} \dicPos{n}[2] \dicFlx{(‑s)}[2] \dicDirectTranslationCS{hovězí (maso)}
\dicEntry[nautgripur] \dicTerm{naut··grip|ur} \dicIPA{{n}{\oe i}{\textlengthmark}{\textsubring{d}}{\r{g}}{r}{\textsci}{\textsubring{b}}{\textscy}{\textsubring{r}}} \dicPos{m}[9] \dicFlx{(‑s, ‑ir)}[8] \dicDirectTranslationCS{(hovězí) dobytek, skot}
\dicEntry[nautheimskur] \dicTerm{naut··heimskur} \dicIPA{{n}{\oe i}{\textlengthmark}{\textsubring{d}}{h}{ei}{m}{s}{\r{g}}{\textscy}{\textsubring{r}}} \dicPos{adj}[1]\dicFlx{}[-6] \dicDirectTranslationCS{pitomý, blbý}
\dicEntry[nautn] \dicTerm{nautn} \dicIPA{{n}{\oe i}{h}{\textsubring{d}}{\textsubring{n}}} \dicPos{f}[7] \dicFlx{(‑ar, ‑ir)}[1] \textbf{1.} \dicSynonym{notkun} \dicDirectTranslationCS{užitek}  \textbf{2.} \dicDirectTranslationCS{požitek, potěšení, radost} \dicExampleIS{hafa nautn af e‑u} \dicExampleCS{mít z~(čeho) potěšení}  \textbf{3.} \dicDirectTranslationCS{rozkoš, slast} \dicExampleIS{kynferðisleg nautn} \dicExampleCS{sexuální rozkoš}
\dicEntry[nautnalegur] \dicTerm{nautna··legur} \dicIPA{{n}{\oe i}{h}{\textsubring{d}}{n}{a}{l}{\textepsilon}{\textbabygamma}{\textscy}{\textsubring{r}}} \dicPos{adj}[1]\dicFlx{}[-8] \dicDirectTranslationCS{smyslný, erotický, vyzývavý}
\dicEntry[nautnamaður] \dicTerm{nautna··|maður} \dicIPA{{n}{\oe i}{h}{\textsubring{d}}{n}{a}{m}{a}{ð}{\textscy}{\textsubring{r}}} \dicPos{m}[13] \dicFlx{(‑manns, ‑menn)}[2] \dicDirectTranslationCS{požitkář(ka), rozkošník, rozkošnice}
\dicEntry[nautnaseggur] \dicTerm{nautna··segg|ur} \dicIPA{{n}{\oe i}{h}{\textsubring{d}}{n}{a}{s}{\textepsilon}{\r{g}}{\textscy}{\textsubring{r}}} \dicPos{m}[9] \dicFlx{(‑s, ‑ir)}[15] \dicDirectTranslationCS{požitkář(ka), rozkošník, rozkošnice}
\dicEntry[nautshaus] \dicTerm{nauts··haus} \dicIPA{{n}{\oe i}{\textsubring{d}}{s}{h}{\oe i}{s}} \dicPos{m}[4] \dicFlx{(‑s, ‑ar)}[14] \textbf{1.} \dicSynonym*{höfuð af nauti} \dicDirectTranslationCS{býčí hlava}  \textbf{2.} \dicSynonym{heimskingi} \dicDirectTranslationCS{hlupák, hlupačka, blbec}
\dicEntry[ná] \dicTerm{ná} \dicsymFrequent\  \dicIPA{{n}{au}{\textlengthmark}} \dicPos{v}[5] \dicFlx{(næ, náði, náðum, næði, náð)}[16] \dicFlx{dat} \textbf{1.} \dicSynonym*{komast} \dicDirectTranslationCS{dosáhnout, dohmátnout, dostat} \dicExampleIS{ná landi} \dicExampleCS{dosáhnout země}  \textbf{2.} \dicSynonym*{draga uppi} \dicDirectTranslationCS{dostihnout, chytit} \dicExampleIS{ná strætó} \dicExampleCS{stihnout autobus}  \textbf{3.} \dicDirectTranslationCS{dosáhnout, domoci se, dojít (cíle ap.)} \dicExampleIS{ná árangri} \dicExampleCS{dosáhnout výsledku}  \textbf{4.} \dicDirectTranslationCS{udělat, složit (zkoušku ap.)} \dicExampleIS{ná prófi} \dicExampleCS{složit zkoušku}  \textbf{5.} \dicDirectTranslationCS{dosáhnout, dosahovat} \dicIndirectTranslationCS{(nabývat rozměru\,/\addthin rozlohy)} \dicExampleIS{Vegurinn nær að fjallinu.} \dicExampleCS{Cesta dosahuje k~hoře.}  \textbf{6.} \dicDirectTranslationCS{porozumět, chápat, pochopit} \dicExampleIS{Ég næ ekki þessu.} \dicExampleCS{Nechápu to.};  \dicIdiom{ná}[af]{ \dicPhraseIS{ná e‑u af e‑m}} \dicSynonym{fá} \dicDirectTranslationCS{vzít (co) od (koho), dát (co) stranou od (koho)} \dicExampleIS{ná hnífnum af barninu} \dicExampleCS{vzít nůž z~dosahu dítěte};  \dicIdiom{ná}[í]{ \dicPhraseIS{ná í e‑n}} \dicDirectTranslationCS{kontaktovat (koho), spojit se s~(kým) (telefonem ap.)} \dicExampleIS{Nemandinn náði í kennarann.} \dicExampleCS{Žák kontaktoval učitele.}; { \dicPhraseIS{ná í e‑ð}} \dicDirectTranslationCS{podat (co), donést (co)} \dicExampleIS{ná í bókina fyrir mig} \dicExampleCS{donést mi knížku}; { \dicPhraseIS{ná sér í e‑ð}} \dicDirectTranslationCS{pořídit si (co)};  \dicIdiom{ná}[niðri]{ \dicPhraseIS{ná sér niðri á e‑m}} \dicDirectTranslationCS{(po)mstít se (komu)};  \dicIdiom{ná}[saman]{ \dicPhraseIS{ná saman}} \dicDirectTranslationCS{najít společnou cestu} \dicExampleIS{Þau náðu saman að lokum.} \dicExampleCS{Nakonec našli společnou cestu.};  \dicIdiom{ná}[sér]{ \dicPhraseIS{ná sér}} \dicDirectTranslationCS{dát se dohromady, sebrat se} \dicExampleIS{Hann náði sér fljótt eftir veikindin.} \dicExampleCS{Po nemoci se dal rychle dohromady.};  \dicIdiom{ná}[til]{ \dicPhraseIS{ná til e‑rs}} \dicDirectTranslationCS{dostat se ke (komu), zastihnout (koho)};  \dicIdiom{ná}[yfir]{ \dicPhraseIS{e‑að nær yfir e‑ð}} \dicDirectTranslationCS{(co) pokrývá\,/\addthin zahrnuje (co)} \dicExampleIS{Þorpið nær ekki yfir stórt svæði.} \dicExampleCS{Vesnice nezabírá velké území.};  \dicIdiom{nást}{ \dicPhraseIS{nást}} \dicFlx{refl} {\textbf{a.}} \dicDirectTranslationCS{být chycen\,/\addthin dopaden, chytit se} \dicExampleIS{Þjófurinn náðist að lokum.} \dicExampleCS{Zloděj byl nakonec chycen.};  {\textbf{b.}} \dicDirectTranslationCS{dosáhnout, dojít (cíle ap.)};  \dicIdiom{nást}[í]{ \dicPhraseIS{það næst í e‑n}} \dicFlx{impers refl} \dicDirectTranslationCS{(kdo) je k~zastižení};  \dicIdiom{nást}[til]{ \dicPhraseIS{það næst til e‑rs}} \dicFlx{impers refl} \dicDirectTranslationCS{(kdo) je k~zastižení, (koho) lze zastihnout}
\dicEntry[nábítur] \dicTerm{ná··bít|ur} \dicIPA{{n}{au}{\textlengthmark}{\textsubring{b}}{i}{\textsubring{d}}{\textscy}{\textsubring{r}}} \dicPos{m}[6] \dicFlx{(‑s, ‑ir\,/\addthin ‑ar)}[70] \dicFieldCat{med.} \dicDirectTranslationCS{pálení žáhy}
\dicEntry[nábjargir] \dicTerm{ná··bjargir} \dicIPA{{n}{au}{\textlengthmark}{\textsubring{b}}{j}{a}{r}{\r{\textObardotlessj}}{\textsci}{\textsubring{r}}} \dicPos{f}[7] \dicFlx{pl}[18] \dicFieldCat{náb.} \dicDirectTranslationCS{poslední pomazání} \dicExampleIS{veita e‑m nábjargir} \dicExampleCS{dát (komu) poslední pomazání}
\dicEntry[nábúi] \dicTerm{ná··bú|i} \dicIPA{{n}{au}{\textlengthmark}{\textsubring{b}}{u}{\textsci}} \dicPos{m}[1] \dicFlx{(‑a, ‑ar)}[1] \dicSynonym{granni} \dicDirectTranslationCS{soused(ka)}
\dicEntry[nábýli] \dicTerm{ná··býli} \dicIPA{{n}{au}{\textlengthmark}{\textsubring{b}}{i}{l}{\textsci}} \dicPos{n}[2] \dicFlx{(‑s)}[20] \dicSynonym{nágrenni} \dicDirectTranslationCS{sousedství}
\dicEntry[náð] \dicTerm{náð\smash{\textsuperscript{1}}} \dicIPA{{n}{au}{\textlengthmark}{\texttheta}} \dicPos{f}[7] \dicFlx{(‑ar, ‑ir)}[1] \textbf{1.} \dicSynonym{ró\smash{\textsuperscript{2}}} \dicDirectTranslationCS{(po)klid, odpočinek};  \dicPhraseIS{taka á sig náðir} \dicDirectTranslationCS{jít si odpočinout, dopřát si odpočinku}  \textbf{2.} \dicSynonym{vernd} \dicDirectTranslationCS{pomoc, ochrana};  \dicPhraseIS{flýja\,/\addthin leita á náðir e‑rs} \dicDirectTranslationCS{jít (komu) na pomoc}  \textbf{3.} \dicSynonym{miskunn} \dicDirectTranslationCS{milost, slitování, milosrdenství};  \dicPhraseIS{syndga upp á náðina} \dicLangCat{přen.} \dicDirectTranslationCS{hřešit na dobrotu};  \dicPhraseIS{vera í náðinni hjá e‑m} \dicDirectTranslationCS{být u~(koho) v~milosti}
\dicEntry[náð] \dicTerm{náð\smash{\textsuperscript{2}}} \dicIPA{{n}{au}{\textlengthmark}{\texttheta}} \dicPos{v} \dicFlx{supin} \dicLink{ná}
\dicEntry[náða] \dicTerm{náð|a} \dicIPA{{n}{au}{\textlengthmark}{ð}{a}} \dicPos{v}[1] \dicFlx{(‑aði)}[1] \dicFlx{acc} \dicDirectTranslationCS{omilostnit, omilostňovat, dát\,/\addthin udělit milost} \dicExampleIS{Hann var dæmdur í tíu ára fangelsi en náðaður eftir nokkur ár.} \dicExampleCS{Byl odsouzen k~deseti letům vězení, ale po několika letech byl omilostněn.}
\dicEntry[náðarsamlegur] \dicTerm{náðar·sam··legur} \dicIPA{{n}{au}{\textlengthmark}{ð}{a}{\textsubring{r}}{s}{a}{m}{l}{\textepsilon}{\textbabygamma}{\textscy}{\textsubring{r}}} \dicPos{adj}[1]\dicFlx{}[-8] \dicSynonym{náðugur} \dicDirectTranslationCS{milostivý, milosrdný}
\dicEntry[náðhús] \dicTerm{náð··hús} \dicIPA{{n}{au}{\textlengthmark}{\texttheta}{h}{u}{s}} \dicPos{n}[2] \dicFlx{(‑s, ‑)}[5] \dicLangCat{zast.} \dicSynonym{salerni} \dicDirectTranslationCS{latrína}
\dicEntry[náði] \dicTerm{náði} \dicIPA{{n}{au}{\textlengthmark}{ð}{\textsci}} \dicPos{v} \dicFlx{ind pf sg 1 pers} \dicLink{ná}
\dicEntry[náðugur] \dicTerm{náð··ugur} \dicIPA{{n}{au}{\textlengthmark}{ð}{\textscy}{\textbabygamma}{\textscy}{\textsubring{r}}} \dicPos{adj}[1]\dicFlx{}[-8] \textbf{1.} \dicSynonym{kyrr} \dicDirectTranslationCS{klidný, pokojný};  \dicPhraseIS{eiga náðuga daga} \dicDirectTranslationCS{užívat si klidu, žít poklidným životem};  \dicPhraseIS{hafa það náðugt} \dicDirectTranslationCS{být\,/\addthin zůstat v~klidu}  \textbf{2.} \dicSynonym{miskunnsamur} \dicDirectTranslationCS{milosrdný, milostivý} \dicExampleIS{Guð veri oss öllum náðugur.} \dicExampleCS{Buď k~nám, Bože, milosrdný.}
\dicEntry[náðum] \dicTerm{náðum} \dicIPA{{n}{au}{\textlengthmark}{ð}{\textscy}{\textsubring{m}}} \dicPos{v} \dicFlx{ind pf pl 1 pers} \dicLink{ná}
\dicEntry[náðun] \dicTerm{náð|un} \dicIPA{{n}{au}{\textlengthmark}{ð}{\textscy}{\textsubring{n}}} \dicPos{f}[7] \dicFlx{(‑unar, ‑anir)}[8] \dicDirectTranslationCS{omilostnění, amnestie}
\dicEntry[náfrændi] \dicTerm{ná··frænd|i} \dicIPA{{n}{au}{\textlengthmark}{f}{r}{a}{i}{n}{\textsubring{d}}{\textsci}} \dicPos{m}[2] \dicFlx{(‑a, ‑ur)}[3] \dicDirectTranslationCS{blízký příbuzný} \dicIndirectTranslationCS{(muž)}
\dicEntry[náfrænka] \dicTerm{ná··frænk|a} \dicIPA{{n}{au}{\textlengthmark}{f}{r}{a}{i}{\r{\ng}}{\r{g}}{a}} \dicPos{f}[1] \dicFlx{(‑u, ‑ur)}[19] \dicDirectTranslationCS{blízká příbuzná}
\dicEntry[náfölur] \dicTerm{ná··fölur} \dicIPA{{n}{au}{\textlengthmark}{f}{\oe}{l}{\textscy}{\textsubring{r}}} \dicPos{adj}[1]\dicFlx{}[-6] \dicDirectTranslationCS{smrtelně bledý}
\dicEntry[nágrannaland] \dicTerm{ná·granna··|land} \dicIPA{{n}{au}{\textlengthmark}{\r{g}}{r}{a}{n}{a}{l}{a}{n}{\textsubring{d}}} \dicPos{n}[2] \dicFlx{(‑lands, ‑lönd)}[8] \dicDirectTranslationCS{sousední země}
\dicEntry[nágranni] \dicTerm{ná··grann|i} \dicsymFrequent\  \dicIPA{{n}{au}{\textlengthmark}{\r{g}}{r}{a}{n}{\textsci}} \dicPos{m}[1] \dicFlx{(‑a, ‑ar)}[8] \dicDirectTranslationCS{soused(ka)} \dicExampleIS{Þarna er nágranni minn.} \dicExampleCS{Tamhle je můj soused.}
\dicEntry[nágrenni] \dicTerm{ná··grenni} \dicsymFrequent\  \dicIPA{{n}{au}{\textlengthmark}{\r{g}}{r}{\textepsilon}{n}{\textsci}} \dicPos{n}[2] \dicFlx{(‑s)}[20] \dicSynonym{grennd} \dicDirectTranslationCS{sousedství, (nejbližší) okolí, (bezprostřední) blízkost} \dicExampleIS{eiga heima í nágrenni borgarinnar} \dicExampleCS{bydlet v~sousedství města}
\dicEntry[náhvalur] \dicTerm{ná··hval|ur}\dicTerm{, náhveli} \dicIPA{{n}\-{au}\-{\textlengthmark}\-{k\smash{\textsuperscript{h}}}\-{v}\-{a}\-{l}\-{\textscy}\-{\textsubring{r}}\-} \dicPos{m}[9] \dicFlx{(‑s, ‑ir)}[9] \dicFieldCat{zool.} \dicDirectTranslationCS{narval, narval jednorohý} \textit{(l.~{\textLA{Monodon monoceros}})}  \dicsymPhoto\ 
\dicFigure{ds_image_nahvalur_0_2.jpg}{Náhvalur}{Náhvalur - Glenn Williams, PD}
\dicEntry[náhveli] \dicTerm{ná··hveli} \dicIPA{{n}{au}{\textlengthmark}{k\smash{\textsuperscript{h}}}{v}{\textepsilon}{l}{\textsci}} \dicPos{n}[2] \dicFlx{(‑s, ‑)}[14] \dicLink{náhvalur}
\dicEntry[náið] \dicTerm{náið} \dicsymFrequent\  \dicIPA{{n}{au}{\textlengthmark}{\textsci}{\texttheta}} \dicPos{adv} \dicFlx{(comp nánar, sup nánast)} \dicDirectTranslationCS{blízko, důvěrně} \dicExampleIS{Ég þekki hann náið.} \dicExampleCS{Znám ho důvěrně.}
\dicEntry[náinn] \dicTerm{náinn} \dicsymFrequent\  \dicIPA{{n}{au}{\textlengthmark}{\textsci}{\textsubring{n}}} \dicPos{adj}[6]\dicFlx{}[-2] \textbf{1.} \dicSynonym{skyldur} \dicDirectTranslationCS{blízký, spřízněný, příbuzný}  \textbf{2.} \dicSynonym{nálægur} \dicDirectTranslationCS{blízký, úzký, bezprostřední} \dicExampleIS{náinn vinur hennar} \dicExampleCS{její blízký přítel};  \dicPhraseIS{nánari} \dicFlx{comp} \dicDirectTranslationCS{bližší} \dicExampleIS{nánari upplýsingar} \dicExampleCS{bližší informace};  \dicProverb\  \dicPhraseIS{Náið er nef augum.} \dicLangCat{přís.} \dicDirectTranslationCS{Bližší košile nežli kabát.}
\dicEntry[nákominn] \dicTerm{ná··kominn} \dicIPA{{n}{au}{\textlengthmark}{k\smash{\textsuperscript{h}}}{\textopeno}{m}{\textsci}{\textsubring{n}}} \dicPos{adj}[6]\dicFlx{}[-2] \textbf{1.} \dicSynonym{skyldur} \dicDirectTranslationCS{blízký, příbuzný} \dicExampleIS{frændur og aðrir nákomnir} \dicExampleCS{strýčkové a~další příbuzní}  \textbf{2.} \dicDirectTranslationCS{vítaný, přicházející vhod} \dicExampleIS{Það er mér mjög nákomið.} \dicExampleCS{Tak to rád uvítám.}
\dicEntry[nákunnugur] \dicTerm{ná··kunn·ugur} \dicIPA{{n}{au}{\textlengthmark}{k\smash{\textsuperscript{h}}}{\textscy}{n}{\textscy}{\textbabygamma}{\textscy}{\textsubring{r}}} \dicPos{adj}[1]\dicFlx{}[-8] \dicDirectTranslationCS{důvěrně známý} \dicExampleIS{Kennarinn var nákunnugur foreldrum barnsins.} \dicExampleCS{Učitel se znal důvěrně s~rodiči dítěte.}
\dicEntry[nákvæmlega] \dicTerm{ná·kvæm··lega} \dicsymFrequent\  \dicIPA{{n}{au}{\textlengthmark}{k\smash{\textsuperscript{h}}}{v}{a}{i}{m}{l}{\textepsilon}{\textbabygamma}{a}} \dicPos{adv} \textbf{1.} \dicSynonym*{nákvæmt mælt} \dicDirectTranslationCS{přesně, precizně, důkladně} \dicExampleIS{Þetta þarf að skoða mjög nákvæmlega.} \dicExampleCS{Musí se to velmi důkladně prozkoumat.}  \textbf{2.} \dicSynonym{einmitt} \dicDirectTranslationCS{přesně, doslovně}
\dicEntry[nákvæmni] \dicTerm{ná··kvæmn|i} \dicIPA{{n}{au}{\textlengthmark}{k\smash{\textsuperscript{h}}}{v}{a}{i}{m}{n}{\textsci}} \dicPos{f}[3] \dicFlx{(‑i)}[3] \textbf{1.} \dicSynonym*{gjörhygli} \dicDirectTranslationCS{přesnost, preciznost} \dicExampleIS{reikna út fjarlægðina af nákvæmni} \dicExampleCS{přesně vypočítat vzdálenost}  \textbf{2.} \dicSynonym{umhyggja} \dicDirectTranslationCS{pozornost, všímavost} \dicExampleIS{sýna e‑m nákvæmni í veikindum} \dicExampleCS{věnovat (komu) pozornost v~průběhu nemoci}
\dicEntry[nákvæmur] \dicTerm{ná··kvæmur} \dicsymFrequent\  \dicIPA{{n}{au}{\textlengthmark}{k\smash{\textsuperscript{h}}}{v}{a}{i}{m}{\textscy}{\textsubring{r}}} \dicPos{adj}[1]\dicFlx{}[-1] \textbf{1.} \dicDirectTranslationCS{přesný, precizní, exaktní} \dicExampleIS{nákvæmur reikningur} \dicExampleCS{přesný výpočet}  \textbf{2.} \dicSynonym{nærgætinn} \dicDirectTranslationCS{pozorný, všímavý}
\dicEntry[nál] \dicTerm{nál} \dicsymFrequent\  \dicIPA{{n}{au}{\textlengthmark}{\textsubring{l}}} \dicPos{f}[4] \dicFlx{(‑ar, ‑ar)}[1] \textbf{1.} \dicSynonym{saumnál} \dicDirectTranslationCS{jehla} \dicExampleIS{saumnál} \dicExampleCS{šicí jehla}  \textbf{2.} \dicSynonym*{sprautunál} \dicDirectTranslationCS{(injekční) jehla}  \textbf{3.} \dicFieldCat{bot.} \dicSynonym*{grasnál} \dicDirectTranslationCS{klíček, výhonek};  \dicIdiom{nál}{ \dicPhraseIS{nýr af nálinni}} \dicFlx{adj} \dicDirectTranslationCS{(jsoucí) novinkou (v~technologii ap.)};  \dicPhraseIS{vera á nálum} \dicLangCat{přen.} \dicDirectTranslationCS{být jako na jehlách}; { \dicPhraseIS{vera ekki búinn að bíta úr nálinni}} \dicLangCat{přen.} \dicDirectTranslationCS{nemít ještě vyhráno}
\dicEntry[náladofi] \dicTerm{nála··dof|i} \dicIPA{{n}{au}{\textlengthmark}{l}{a}{\textsubring{d}}{\textopeno}{v}{\textsci}} \dicPos{m}[1] \dicFlx{(‑a)}[3] \dicDirectTranslationCS{(pocit) mravenčení} \dicExampleIS{náladofi í fæti} \dicExampleCS{mravenčení v~noze}
\dicEntry[nálaprentari] \dicTerm{nála··prent·ar|i} \dicIPA{{n}{au}{\textlengthmark}{l}{a}{p\smash{\textsuperscript{h}}}{r}{\textepsilon}{\textsubring{n}}{\textsubring{d}}{a}{r}{\textsci}} \dicPos{m}[1] \dicFlx{(‑a, ‑ar)}[13] \dicFieldCat{poč.} \dicDirectTranslationCS{jehličková tiskárna}
\dicEntry[nálapúði] \dicTerm{nála··púð|i} \dicIPA{{n}{au}{\textlengthmark}{l}{a}{p\smash{\textsuperscript{h}}}{u}{ð}{\textsci}} \dicPos{m}[1] \dicFlx{(‑a, ‑ar)}[1] \textbf{1.} \dicDirectTranslationCS{jehelníček}  \textbf{2.} \dicFieldCat{bot.} \dicDirectTranslationCS{azorela trojklaná} \textit{(l.~{\textLA{Azorella trifurcata}})}  \dicsymPhoto\ 
\dicFigure{ds_image_nalapudi_0_2.jpg}{Nálapúði}{Nálapúði - Sten at da.wikipedia, GFDL}
\dicEntry[nálarstunga] \dicTerm{nálar··stung|a}\dicTerm{, nálastunga} \dicIPA{{n}\-{au}\-{\textlengthmark}\-{l}\-{a}\-{\textsubring{r}}\-{s}\-{\textsubring{d}}\-{u}\-{\ng}\-{\r{g}}\-{a}\-} \dicPos{f}[1] \dicFlx{(‑u, ‑ur)}[13] \dicDirectTranslationCS{akupunktura}
\dicEntry[nálastunga] \dicTerm{nála··stung|a} \dicIPA{{n}{au}{\textlengthmark}{l}{a}{s}{\textsubring{d}}{u}{\ng}{\r{g}}{a}} \dicPos{f}[1] \dicFlx{(‑u, ‑ur)}[13] \dicLink{nálarstunga}
\dicEntry[nálegur] \dicTerm{ná··legur} \dicIPA{{n}{au}{\textlengthmark}{l}{\textepsilon}{\textbabygamma}{\textscy}{\textsubring{r}}} \dicPos{adj}[1]\dicFlx{}[-8] \dicSynonym{slæmur} \dicDirectTranslationCS{špatný, chatrný, škaredý}
\dicEntry[nálgast] \dicTerm{nálg|ast} \dicsymFrequent\  \dicIPA{{n}{au}{l}{\r{g}}{a}{s}{\textsubring{d}}} \dicPos{v}[1] \dicFlx{(‑aðist)}[91] \dicFlx{refl} \textbf{1.} \dicSynonym*{koma nær} \dicDirectTranslationCS{(při)blížit se, přibližovat se} \dicExampleIS{Flugvélin nálgast hægt og hægt.} \dicExampleCS{Letadlo se pomalu blíží.}  \textbf{2.} \dicSynonym{sækja} \dicDirectTranslationCS{dostat se, získat přístup (k~informacím ap.)} \dicExampleIS{Hvar er hægt að nálgast upplýsingar?} \dicExampleCS{Kde je možné získat přístup k~informacím?}
\dicEntry[nálgun] \dicTerm{nálg|un} \dicIPA{{n}{au}{l}{\r{g}}{\textscy}{\textsubring{n}}} \dicPos{f}[7] \dicFlx{(‑unar)}[9] \textbf{1.} \dicSynonym{aðferð} \dicDirectTranslationCS{přístup, postoj}  \textbf{2.} \dicSynonym{nálægð} \dicDirectTranslationCS{přiblížení, blízkost} \dicExampleIS{nálgun við dauðann} \dicExampleCS{přiblížení se smrti}  \textbf{3.} \dicFieldCat{mat.} \dicDirectTranslationCS{přiblížení, aproximace}
\dicEntry[nálgunarhljóð] \dicTerm{nálgunar··hljóð} \dicIPA{{n}{au}{l}{\r{g}}{\textscy}{n}{a}{\textsubring{r}}{\textsubring{l}}{j}{ou}{\texttheta}} \dicPos{n}[2] \dicFlx{(‑s, ‑)}[5] \dicFieldCat{jaz.} \dicDirectTranslationCS{aproximanta}
\dicEntry[nálægð] \dicTerm{ná··lægð} \dicIPA{{n}{au}{\textlengthmark}{l}{a}{i}{\textbabygamma}{\texttheta}} \dicPos{f}[7] \dicFlx{(‑ar)}[3] \textbf{1.} \dicDirectTranslationCS{blízkost, dohlednost}  \textbf{2.} \dicSynonym{viðurvist} \dicDirectTranslationCS{přítomnost (krále ap.)} \dicExampleIS{í nálægð konungs} \dicExampleCS{v~přítomnosti krále}  \textbf{3.} \dicDirectTranslationCS{(nejbližší) okolí, sousedství} \dicExampleIS{e‑s staðar hér í nálægðinni} \dicExampleCS{někde tady v~nejbližším okolí}
\dicEntry[nálægt] \dicTerm{ná··lægt} \dicsymFrequent\  \dicIPA{{n}{au}{\textlengthmark}{l}{a}{i}{x}{\textsubring{d}}} \dicPos{adv} \textbf{1.} \dicSynonym*{um það bil} \dicDirectTranslationCS{téměř, přibližně}  \textbf{2.} \dicSynonym*{nærri} \dicDirectTranslationCS{blízko, poblíž, nedaleko} \dicExampleIS{nálægt e‑u} \dicExampleCS{blízko (čeho)}
\dicEntry[nálægur] \dicTerm{ná··lægur} \dicsymFrequent\  \dicIPA{{n}{au}{\textlengthmark}{l}{a}{i}{\textbabygamma}{\textscy}{\textsubring{r}}} \dicPos{adj}[1]\dicFlx{}[-1] \textbf{1.} \dicDirectTranslationCS{blízký, nedaleký, (jsoucí) poblíž, (jsoucí) nedaleko} \dicExampleIS{nálæg kjördæmi} \dicExampleCS{nedaleký volební obvod}  \textbf{2.} \dicFieldCat{jaz.} \dicDirectTranslationCS{zavřený (samohláska ap.)}
\dicEntry[nám] \dicTerm{nám} \dicsymFrequent\  \dicIPA{{n}{au}{\textlengthmark}{\textsubring{m}}} \dicPos{n}[2] \dicFlx{(‑s)}[2] \dicDirectTranslationCS{studium, učení se} \dicExampleIS{nám í guðfræði} \dicExampleCS{studium teologie};  \dicPhraseIS{bóklegt nám} \dicDirectTranslationCS{akademické studium};  \dicPhraseIS{vera í námi} \dicDirectTranslationCS{studovat, učit se} \dicIndirectTranslationCS{(vzdělávat se na vysoké nebo střední škole)};  \dicPhraseIS{verklegt nám} \dicDirectTranslationCS{odborné studium}
\dicEntry[náma] \dicTerm{nám|a} \dicIPA{{n}{au}{\textlengthmark}{m}{a}} \dicPos{f}[1] \dicFlx{(‑u, ‑ur)}[7] \dicDirectTranslationCS{důl, šachta}
\dicEntry[námfús] \dicTerm{nám··fús} \dicIPA{{n}{au}{m}{f}{u}{s}} \dicPos{adj}[5]\dicFlx{}[-1] \dicDirectTranslationCS{učící se s~pílí, pilně studující}
\dicEntry[námsáfangi] \dicTerm{náms··á·fang|i} \dicIPA{{n}{au}{m}{s}{au}{f}{au}{\textltailn}{\r{\textObardotlessj}}{\textsci}} \dicPos{m}[1] \dicFlx{(‑a, ‑ar)}[8] \dicFieldCat{škol.} \dicDirectTranslationCS{studijní modul}
\dicEntry[námsár] \dicTerm{náms··ár} \dicIPA{{n}{au}{m}{s}{au}{\textsubring{r}}} \dicPos{n}[2] \dicFlx{(‑s, ‑)}[5] \dicDirectTranslationCS{školní rok}
\dicEntry[námsárangur] \dicTerm{náms··ár·ang|ur} \dicIPA{{n}{au}{m}{s}{au}{r}{au}{\ng}{\r{g}}{\textscy}{\textsubring{r}}} \dicPos{m}[5] \dicFlx{(‑urs)}[2] \dicDirectTranslationCS{studijní výsledky}
\dicEntry[námsbók] \dicTerm{náms··|bók} \dicIPA{{n}{au}{m}{s}{\textsubring{b}}{ou}{\r{g}}} \dicPos{f}[8] \dicFlx{(‑bókar, ‑bækur)}[5] \dicDirectTranslationCS{učebnice}
\dicEntry[námsbraut] \dicTerm{náms··braut} \dicIPA{{n}{au}{m}{s}{\textsubring{b}}{r}{\oe i}{\textsubring{d}}} \dicPos{f}[7] \dicFlx{(‑ar, ‑ir)}[1] \dicDirectTranslationCS{studijní program}
\dicEntry[námsdvöl] \dicTerm{náms··|dvöl} \dicIPA{{n}{au}{m}{s}{\textsubring{d}}{v}{\oe}{\textsubring{l}}} \dicPos{f}[7] \dicFlx{(‑dvalar, ‑dvalir)}[16] \dicDirectTranslationCS{studijní pobyt}
\dicEntry[námsefni] \dicTerm{náms··efni} \dicIPA{{n}{au}{m}{s}{\textepsilon}{\textsubring{b}}{n}{\textsci}} \dicPos{n}[2] \dicFlx{(‑s, ‑)}[14] \dicFieldCat{škol.} \dicDirectTranslationCS{učivo, učební\,/\addthin studijní program, učební plán, školní osnovy}
\dicEntry[námsflokkur] \dicTerm{náms··flokk|ur} \dicIPA{{n}{au}{m}{s}{f}{l}{\textopeno}{h}{\r{g}}{\textscy}{\textsubring{r}}} \dicPos{m}[6] \dicFlx{(‑s, ‑ar)}[8] \dicDirectTranslationCS{studijní skupina}
\dicEntry[námsgrein] \dicTerm{náms··grein} \dicIPA{{n}{au}{m}{s}{\r{g}}{r}{ei}{\textsubring{n}}} \dicPos{f}[4] \dicFlx{(‑ar, ‑ar)}[1] \dicSynonym{fag} \dicDirectTranslationCS{školní předmět}
\dicEntry[námsgögn] \dicTerm{náms··gögn} \dicIPA{{n}{au}{m}{s}{\r{g}}{\oe}{\r{g}}{\textsubring{n}}} \dicPos{n}[2] \dicFlx{pl}[9] \dicDirectTranslationCS{vyučovací pomůcky}
\dicEntry[námshestur] \dicTerm{náms··hest|ur} \dicIPA{{n}{au}{m}{s}{h}{\textepsilon}{s}{\textsubring{d}}{\textscy}{\textsubring{r}}} \dicPos{m}[6] \dicFlx{(‑s, ‑ar)}[4] \dicDirectTranslationCS{dříč(ka), bifloun, šprt(ka)}
\dicEntry[námskeið] \dicTerm{nám··skeið} \dicIPA{{n}{au}{m}{s}{\r{\textObardotlessj}}{ei}{\texttheta}} \dicPos{n}[2] \dicFlx{(‑s, ‑)}[5] \dicDirectTranslationCS{(výukový) kurz} \dicExampleIS{námskeið fyrir byrjendur} \dicExampleCS{kurz pro začátečníky}
\dicEntry[námskeiðsgjald] \dicTerm{nám·skeiðs··|gjald} \dicIPA{{n}{au}{m}{s}{\r{\textObardotlessj}}{ei}{ð}{s}{\r{\textObardotlessj}}{a}{l}{\textsubring{d}}} \dicPos{n}[2] \dicFlx{(‑gjalds, ‑gjöld)}[8] \dicDirectTranslationCS{poplatek za kurz}
\dicEntry[námskostnaður] \dicTerm{náms··kost·nað|ur} \dicIPA{{n}{au}{m}{s}{k\smash{\textsuperscript{h}}}{\textopeno}{s}{\textsubring{d}}{n}{a}{ð}{\textscy}{\textsubring{r}}} \dicPos{m}[10] \dicFlx{(‑ar)}[9] \dicDirectTranslationCS{náklady na studium}
\dicEntry[námskrá] \dicTerm{nám··skrá} \dicIPA{{n}{au}{m}{s}{\r{g}}{r}{au}} \dicPos{f}[4] \dicFlx{(‑r\,/\addthin ‑ar, ‑r)}[21] \dicLink{námsskrá}
\dicEntry[námslán] \dicTerm{náms··lán} \dicIPA{{n}{au}{m}{s}{l}{au}{\textsubring{n}}} \dicPos{n}[2] \dicFlx{(‑s, ‑)}[5] \dicDirectTranslationCS{půjčka na studium, studijní půjčka}
\dicEntry[námsleyfi] \dicTerm{náms··leyfi} \dicIPA{{n}{au}{m}{s}{l}{ei}{v}{\textsci}} \dicPos{n}[2] \dicFlx{(‑s, ‑)}[14] \dicDirectTranslationCS{studijní volno}
\dicEntry[námsmaður] \dicTerm{náms··|maður} \dicIPA{{n}{au}{m}{s}{m}{a}{ð}{\textscy}{\textsubring{r}}} \dicPos{m}[13] \dicFlx{(‑manns, ‑menn)}[2] \dicDirectTranslationCS{student(ka)}
\dicEntry[námsmat] \dicTerm{náms··mat} \dicIPA{{n}{au}{m}{s}{m}{a}{\textsubring{d}}} \dicPos{n}[2] \dicFlx{(‑s)}[2] \dicFieldCat{škol.} \dicDirectTranslationCS{známka, ohodnocení} \dicIndirectTranslationCS{(hodnocení výsledků studenta dosažených v~určitém předmětu)}
\dicEntry[námsskrá] \dicTerm{náms··skrá}\dicTerm{, námskrá} \dicIPA{{n}\-{au}\-{m}\-{s}\-{\r{g}}\-{r}\-{au}\-} \dicPos{f}[4] \dicFlx{(‑r\,/\addthin ‑ar, ‑r)}[21] \dicFieldCat{škol.} \dicDirectTranslationCS{učební plán, učební\,/\addthin studijní program, školní osnovy}
\dicEntry[námsstyrkur] \dicTerm{náms··styrk|ur} \dicIPA{{n}{au}{m}{s}{\textsubring{d}}{\textsci}{\textsubring{r}}{\r{g}}{\textscy}{\textsubring{r}}} \dicPos{m}[9] \dicFlx{(‑s, ‑ir)}[15] \dicDirectTranslationCS{(studijní) stipendium}
\dicEntry[námugröftur] \dicTerm{námu··|gröftur} \dicIPA{{n}{au}{\textlengthmark}{m}{\textscy}{\r{g}}{r}{\oe}{f}{\textsubring{d}}{\textscy}{\textsubring{r}}} \dicPos{m}[11] \dicFlx{(‑graftar\,/\addthin ‑graftrar)}[12] \dicDirectTranslationCS{těžení, dolování}
\dicEntry[námum] \dicTerm{námum} \dicIPA{{n}{au}{\textlengthmark}{m}{\textscy}{\textsubring{m}}} \dicPos{v} \dicFlx{ind pf pl 1 pers} \dicLink{nema\smash{\textsuperscript{1}}}
\dicEntry[námumaður] \dicTerm{námu··|maður} \dicIPA{{n}{au}{\textlengthmark}{m}{\textscy}{m}{a}{ð}{\textscy}{\textsubring{r}}} \dicPos{m}[13] \dicFlx{(‑manns, ‑menn)}[2] \dicDirectTranslationCS{horník, hornice}
\dicEntry[námundi] \dicTerm{ná··mund|i} \dicIPA{{n}{au}{\textlengthmark}{m}{\textscy}{n}{\textsubring{d}}{\textsci}} \dicPos{m}[1] \dicFlx{(‑a)}[3] \dicSynonym{nánd} \dicDirectTranslationCS{blízkost, blízké okolí};  \dicPhraseIS{í námunda við e‑ð} \dicFlx{prep} \dicDirectTranslationCS{v~blízkosti (čeho)}
\dicEntry[nánar] \dicTerm{nánar} \dicIPA{{n}{au}{\textlengthmark}{n}{a}{\textsubring{r}}} \dicPos{adv} \dicFlx{comp (pos náið, sup nánast)} \dicDirectTranslationCS{více (informací ap.), podrobněji} \dicExampleIS{nánar um höfundinn} \dicExampleCS{více o~autorovi}
\dicEntry[nánasarlegur] \dicTerm{nánasar··legur} \dicIPA{{n}{au}{\textlengthmark}{n}{a}{s}{a}{r}{l}{\textepsilon}{\textbabygamma}{\textscy}{\textsubring{r}}} \dicPos{adj}[1]\dicFlx{}[-8] \dicSynonym*{nirfilslegur} \dicDirectTranslationCS{skrblický, skoupý} \dicExampleIS{nánasarleg laun} \dicExampleCS{skoupý plat}
\dicEntry[nánasarskapur] \dicTerm{nánasar··skap|ur} \dicIPA{{n}{au}{\textlengthmark}{n}{a}{s}{a}{\textsubring{r}}{s}{\r{g}}{a}{\textsubring{b}}{\textscy}{\textsubring{r}}} \dicPos{m}[10] \dicFlx{(‑ar)}[15] \dicSynonym{níska} \dicDirectTranslationCS{skoupost, lakotnost}
\dicEntry[nánast] \dicTerm{nánast} \dicIPA{{n}{au}{\textlengthmark}{n}{a}{s}{\textsubring{d}}} \dicPos{adv} \dicFlx{sup (pos náið, comp nánar)} \dicLink{náið}
\dicEntry[nánd] \dicTerm{nánd} \dicIPA{{n}{au}{n}{\textsubring{d}}} \dicPos{f}[7] \dicFlx{(‑ar, ‑ir)}[1] \dicSynonym{nálægð} \dicDirectTranslationCS{blízkost, bezprostřednost};  \dicPhraseIS{vera í nánd} \dicDirectTranslationCS{blížit se} \dicExampleIS{Jólin eru í nánd.} \dicExampleCS{Vánoce se blíží.}
\dicEntry[nár] \dicTerm{ná|r} \dicIPA{{n}{au}{\textlengthmark}{\textsubring{r}}} \dicPos{m}[9] \dicFlx{(‑s, ‑ir)}[12] \dicDirectTranslationCS{mrtvola, mrtvé tělo};  \dicPhraseIS{bleikur\,/\addthin fölur sem nár} \dicFlx{adj} \dicLangCat{přen.} \dicDirectTranslationCS{bledý jako smrt}
\dicEntry[nári] \dicTerm{nár|i} \dicIPA{{n}{au}{\textlengthmark}{r}{\textsci}} \dicPos{m}[1] \dicFlx{(‑a, ‑ar)}[1] \dicFieldCat{anat.} \dicDirectTranslationCS{slabina, tříslo}
\dicEntry[náskyldur] \dicTerm{ná··skyldur} \dicIPA{{n}{au}{\textlengthmark}{s}{\r{\textObardotlessj}}{\textsci}{l}{\textsubring{d}}{\textscy}{\textsubring{r}}} \dicPos{adj}[2]\dicFlx{}[-14] \dicDirectTranslationCS{(blízce) příbuzný} \dicExampleIS{Þeir eru náskyldir.} \dicExampleCS{Jsou to blízcí příbuzní.}
\dicEntry[náttblinda] \dicTerm{nátt··blind|a} \dicIPA{{n}{au}{h}{\textsubring{d}}{\textsubring{b}}{l}{\textsci}{n}{\textsubring{d}}{a}} \dicPos{f}[1] \dicFlx{(‑u)}[5] \dicFieldCat{med.} \dicDirectTranslationCS{šeroslepost}
\dicEntry[náttblindur] \dicTerm{nátt··blindur} \dicIPA{{n}{au}{h}{\textsubring{d}}{\textsubring{b}}{l}{\textsci}{n}{\textsubring{d}}{\textscy}{\textsubring{r}}} \dicPos{adj}[2]\dicFlx{}[-14] \dicFieldCat{med.} \dicDirectTranslationCS{šeroslepý}
\dicEntry[náttborð] \dicTerm{nátt··borð} \dicIPA{{n}{au}{h}{\textsubring{d}}{\textsubring{b}}{\textopeno}{r}{\texttheta}} \dicPos{n}[2] \dicFlx{(‑s, ‑)}[5] \dicDirectTranslationCS{noční stolek}
\dicEntry[náttbuxur] \dicTerm{nátt··buxur} \dicIPA{{n}{au}{h}{\textsubring{d}}{\textsubring{b}}{\textscy}{x}{s}{\textscy}{\textsubring{r}}} \dicPos{f}[12] \dicFlx{pl}[7] \dicDirectTranslationCS{pyžamo (kalhoty), kalhoty od pyžama}
\dicEntry[náttfari] \dicTerm{nátt··far|i} \dicIPA{{n}{au}{h}{\textsubring{d}}{f}{a}{r}{\textsci}} \dicPos{m}[1] \dicFlx{(‑a, ‑ar)}[8] \dicFieldCat{zool.} \dicDirectTranslationCS{lelek lesní} \textit{(l.~{\textLA{Caprimulgus europaeus}})}  \dicsymPhoto\ 
\dicFigure{ds_image_nattfari_0_1.jpg}{Náttfari}{Náttfari - Durzan Cirano, CC BY-SA 3.0}
\dicEntry[náttföt] \dicTerm{nátt··föt} \dicIPA{{n}{au}{h}{\textsubring{d}}{f}{\oe}{\textsubring{d}}} \dicPos{n}[2] \dicFlx{pl}[9] \dicDirectTranslationCS{pyžamo}
\dicEntry[náttgali] \dicTerm{nátt··gal|i} \dicIPA{{n}{au}{h}{\textsubring{d}}{\r{g}}{a}{l}{\textsci}} \dicPos{m}[1] \dicFlx{(‑a, ‑ar)}[8] \dicLink{næturgali}
\dicEntry[nátthrafn] \dicTerm{nátt··hrafn} \dicIPA{{n}{au}{h}{\textsubring{d}}{\textsubring{r}}{a}{\textsubring{b}}{\textsubring{n}}} \dicPos{m}[4] \dicFlx{(‑s, ‑ar)}[4] \dicDirectTranslationCS{noční sova} \dicIndirectTranslationCS{(člověk, který chodí pozdě spát)}
\dicEntry[náttkjóll] \dicTerm{nátt··kjól|l} \dicIPA{{n}{au}{h}{\textsubring{d}}{c\smash{\textsuperscript{h}}}{ou}{\textsubring{d}}{\textsubring{l}}} \dicPos{m}[6] \dicFlx{(‑s, ‑ar)}[48] \dicDirectTranslationCS{noční košile}
\dicEntry[náttsloppur] \dicTerm{nátt··slopp|ur} \dicIPA{{n}{au}{h}{\textsubring{d}}{s}{\textsubring{d}}{l}{\textopeno}{h}{\textsubring{b}}{\textscy}{\textsubring{r}}} \dicPos{m}[6] \dicFlx{(‑s, ‑ar)}[24] \dicDirectTranslationCS{župan, nedbalky, negližé}
\dicEntry[nátttreyja] \dicTerm{nátt··treyj|a} \dicIPA{{n}{au}{h}{t\smash{\textsuperscript{h}}}{r}{ei}{j}{a}} \dicPos{f}[1] \dicFlx{(‑u, ‑ur)}[7] \dicDirectTranslationCS{noční triko}
\dicEntry[náttugla] \dicTerm{nátt··ugl|a} \dicIPA{{n}{au}{h}{\textsubring{d}}{\textscy}{\r{g}}{l}{a}} \dicPos{f}[1] \dicFlx{(‑u, ‑ur)}[19] \dicFieldCat{zool.} \dicDirectTranslationCS{puštík, puštík obecný} \textit{(l.~{\textLA{Strix aluco}})}  \dicsymPhoto\ 
\dicFigure{ds_image_nattugla_0_1.jpg}{Náttugla}{Náttugla - K.-M. Hansche, CC BY-SA 2.5}
\dicEntry[náttúra] \dicTerm{náttúr|a} \dicsymFrequent\  \dicIPA{{n}{au}{h}{\textsubring{d}}{\textlengthmark}{u}{r}{a}} \dicPos{f}[1] \dicFlx{(‑u, ‑ur)}[7] \textbf{1.} \dicDirectTranslationCS{příroda} \dicExampleIS{lifandi og dauð náttúra} \dicExampleCS{živá a~neživá příroda}  \textbf{2.} \dicSynonym{eðli} \dicDirectTranslationCS{přirozenost, podstata} \dicExampleIS{Það er náttúra hans.} \dicExampleCS{To je jeho přirozenost.}
\dicEntry[náttúraður] \dicTerm{náttúr··|aður} \dicIPA{{n}{au}{h}{\textsubring{d}}{\textlengthmark}{u}{r}{a}{ð}{\textscy}{\textsubring{r}}} \dicPos{adj}[3] \dicFlx{(f ‑uð)}[3] \dicDirectTranslationCS{(jsoucí) s~přirozeným talentem} \dicExampleIS{vera náttúraður fyrir e‑ð} \dicExampleCS{mít přirozený talent na (co)}
\dicEntry[náttúrlega] \dicTerm{náttúr··lega}\dicTerm{, náttúrulega} \dicIPA{{n}\-{au}\-{h}\-{\textsubring{d}}\-{\textlengthmark}\-{u}\-{r}\-{l}\-{\textepsilon}\-{\textbabygamma}\-{a}\-} \dicPos{adv} \dicSynonym{vitaskuld} \dicDirectTranslationCS{přirozeně, samosebou, ovšem} \dicExampleIS{Þetta var náttúrlega ekkert svar.} \dicExampleCS{To samosebou nebyla žádná odpověď.}
\dicEntry[náttúrlegur] \dicTerm{náttúr··legur}\dicTerm{, náttúrulegur} \dicIPA{{n}\-{au}\-{h}\-{\textsubring{d}}\-{\textlengthmark}\-{u}\-{r}\-{l}\-{\textepsilon}\-{\textbabygamma}\-{\textscy}\-{\textsubring{r}}\-} \dicPos{adj}[1]\dicFlx{}[-8] \textbf{1.} \dicDirectTranslationCS{přírodní, naturální, (jsoucí) v~souladu s~přírodou}  \textbf{2.} \dicSynonym{eðlilegur} \dicDirectTranslationCS{přirozený, normální, obvyklý}
\dicEntry[náttúruauðlind] \dicTerm{náttúru··auð·lind} \dicIPA{{n}{au}{h}{\textsubring{d}}{\textlengthmark}{u}{r}{\textscy}{\oe i}{ð}{l}{\textsci}{n}{\textsubring{d}}} \dicPos{f}[7] \dicFlx{(‑ar, ‑ir)}[1] \dicDirectTranslationCS{přírodní bohatství}
\dicEntry[náttúrufegurð] \dicTerm{náttúru··fegurð} \dicIPA{{n}{au}{h}{\textsubring{d}}{\textlengthmark}{u}{r}{\textscy}{f}{\textepsilon}{\textbabygamma}{\textscy}{r}{\texttheta}} \dicPos{f}[4] \dicFlx{(‑ar)}[3] \dicDirectTranslationCS{krása přírody}
\dicEntry[náttúrufræði] \dicTerm{náttúru··fræð|i} \dicIPA{{n}{au}{h}{\textsubring{d}}{\textlengthmark}{u}{r}{\textscy}{f}{r}{a}{i}{ð}{\textsci}} \dicPos{f}[3] \dicFlx{(‑i)}[3] \textbf{1.} \dicSynonym{vísindagrein} \dicDirectTranslationCS{přírodověda} \dicIndirectTranslationCS{(věda)}  \textbf{2.} \dicSynonym{námsgrein} \dicDirectTranslationCS{přírodověda} \dicIndirectTranslationCS{(vyučovací předmět)}
\dicEntry[náttúrufræðingur] \dicTerm{náttúru·fræð··ing|ur} \dicIPA{{n}{au}{h}{\textsubring{d}}{\textlengthmark}{u}{r}{\textscy}{f}{r}{a}{i}{ð}{i}{\ng}{\r{g}}{\textscy}{\textsubring{r}}} \dicPos{m}[6] \dicFlx{(‑s, ‑ar)}[8] \dicDirectTranslationCS{přírodovědec, přírodovědkyně}
\dicEntry[náttúrufyrirbæri] \dicTerm{náttúru··fyrir·bæri} \dicIPA{{n}{au}{h}{\textsubring{d}}{\textlengthmark}{u}{r}{\textscy}{f}{\textsci}{r}{\textsci}{r}{\textsubring{b}}{a}{i}{r}{\textsci}} \dicPos{n}[2] \dicFlx{(‑s, ‑)}[14] \dicDirectTranslationCS{přírodní úkaz}
\dicEntry[náttúrugripasafn] \dicTerm{náttúru·gripa··|safn} \dicIPA{{n}{au}{h}{\textsubring{d}}{\textlengthmark}{u}{r}{\textscy}{\r{g}}{r}{\textsci}{\textsubring{b}}{a}{s}{a}{\textsubring{b}}{\textsubring{n}}} \dicPos{n}[2] \dicFlx{(‑safns, ‑söfn)}[8] \dicDirectTranslationCS{přírodovědecké muzeum}
\dicEntry[náttúruhamfarir] \dicTerm{náttúru··ham·farir} \dicIPA{{n}{au}{h}{\textsubring{d}}{\textlengthmark}{u}{r}{\textscy}{h}{a}{m}{f}{a}{r}{\textsci}{\textsubring{r}}} \dicPos{f}[7] \dicFlx{pl}[18] \dicDirectTranslationCS{přírodní katastrofa}
\dicEntry[náttúrulaus] \dicTerm{náttúru··laus} \dicIPA{{n}{au}{h}{\textsubring{d}}{\textlengthmark}{u}{r}{\textscy}{l}{\oe i}{s}} \dicPos{adj}[5]\dicFlx{}[-1] \dicDirectTranslationCS{impotentní}
\dicEntry[náttúrulega] \dicTerm{náttúru··lega} \dicIPA{{n}{au}{h}{\textsubring{d}}{\textlengthmark}{u}{r}{\textscy}{l}{\textepsilon}{\textbabygamma}{a}} \dicPos{adv} \dicLink{náttúrlega}
\dicEntry[náttúrulegur] \dicTerm{náttúru··legur} \dicIPA{{n}{au}{h}{\textsubring{d}}{\textlengthmark}{u}{r}{\textscy}{l}{\textepsilon}{\textbabygamma}{\textscy}{\textsubring{r}}} \dicPos{adj}[1]\dicFlx{}[-8] \dicLink{náttúrlegur}
\dicEntry[náttúruleysi] \dicTerm{náttúru··leysi} \dicIPA{{n}{au}{h}{\textsubring{d}}{\textlengthmark}{u}{r}{\textscy}{l}{ei}{s}{\textsci}} \dicPos{n}[2] \dicFlx{(‑s)}[20] \dicDirectTranslationCS{impotence}
\dicEntry[náttúrulækningar] \dicTerm{náttúru··lækn·ingar} \dicIPA{{n}{au}{h}{\textsubring{d}}{\textlengthmark}{u}{r}{\textscy}{l}{a}{i}{h}{\r{g}}{n}{i}{\ng}{\r{g}}{a}{\textsubring{r}}} \dicPos{f}[4] \dicFlx{pl}[6] \dicDirectTranslationCS{přírodní medicína, léčitelství}
\dicEntry[náttúrulögmál] \dicTerm{náttúru··lög·mál} \dicIPA{{n}{au}{h}{\textsubring{d}}{\textlengthmark}{u}{r}{\textscy}{l}{\oe}{x}{m}{au}{\textsubring{l}}} \dicPos{n}[2] \dicFlx{(‑s, ‑)}[5] \dicDirectTranslationCS{přírodní zákon, zákon přírody}
\dicEntry[náttúruparadís] \dicTerm{náttúru··paradís} \dicIPA{{n}{au}{h}{\textsubring{d}}{\textlengthmark}{u}{r}{\textscy}{p\smash{\textsuperscript{h}}}{a}{r}{a}{\textsubring{d}}{i}{s}} \dicPos{f}[7] \dicFlx{(‑ar, ‑ir)}[1] \dicDirectTranslationCS{přírodní ráj}
\dicEntry[náttúruunnandi] \dicTerm{náttúru··unn·|andi} \dicIPA{{n}{au}{h}{\textsubring{d}}{\textlengthmark}{u}{r}{\textscy}{\textscy}{n}{a}{n}{\textsubring{d}}{\textsci}} \dicPos{m}[2] \dicFlx{(‑anda, ‑endur)}[1] \dicDirectTranslationCS{milovník\,/\addthin milovnice přírody}
\dicEntry[náttúruval] \dicTerm{náttúru··val} \dicIPA{{n}{au}{h}{\textsubring{d}}{\textlengthmark}{u}{r}{\textscy}{v}{a}{\textsubring{l}}} \dicPos{n}[2] \dicFlx{(‑s)}[2] \dicFieldCat{biol.} \dicDirectTranslationCS{přirozený výběr, přirozená selekce}
\dicEntry[náttúruvernd] \dicTerm{náttúru··vernd} \dicIPA{{n}{au}{h}{\textsubring{d}}{\textlengthmark}{u}{r}{\textscy}{v}{\textepsilon}{r}{n}{\textsubring{d}}} \dicPos{f}[4] \dicFlx{(‑ar)}[3] \dicDirectTranslationCS{ochrana přírody}
\dicEntry[náttúruverndarráð] \dicTerm{náttúru·verndar··ráð} \dicIPA{{n}{au}{h}{\textsubring{d}}{\textlengthmark}{u}{r}{\textscy}{v}{\textepsilon}{r}{n}{\textsubring{d}}{a}{r}{au}{\texttheta}} \dicPos{n}[2] \dicFlx{(‑s, ‑)}[5] \dicDirectTranslationCS{rada\,/\addthin komise pro ochranu přírody}
\dicEntry[náttúruvísindi] \dicTerm{náttúru··vís·indi} \dicIPA{{n}{au}{h}{\textsubring{d}}{\textlengthmark}{u}{r}{\textscy}{v}{i}{s}{\textsci}{n}{\textsubring{d}}{\textsci}} \dicPos{n}[2] \dicFlx{pl}[19] \dicDirectTranslationCS{přírodní vědy}
\dicEntry[náttúruöfl] \dicTerm{náttúru··öfl} \dicIPA{{n}{au}{h}{\textsubring{d}}{\textlengthmark}{u}{r}{\textscy}{\oe}{\textsubring{b}}{\textsubring{l}}} \dicPos{n}[2] \dicFlx{pl}[9] \dicDirectTranslationCS{přírodní síly}
\dicEntry[náungi] \dicTerm{ná··ung|i} \dicsymFrequent\  \dicIPA{{n}{au}{\textlengthmark}{u}{\textltailn}{\r{\textObardotlessj}}{\textsci}} \dicPos{m}[1] \dicFlx{(‑a, ‑ar)}[1] \textbf{1.} \dicSynonym{kumpáni} \dicDirectTranslationCS{chlapík, chlápek}  \textbf{2.} \dicSynonym{meðbróðir} \dicDirectTranslationCS{bližní} \dicExampleIS{Þú skalt elska náunga þinn einsog sjálfan þig.} \dicExampleCS{Miluj bližního svého jako sebe sama.}
\dicEntry[návist] \dicTerm{ná··vist} \dicsymFrequent\  \dicIPA{{n}{au}{\textlengthmark}{v}{\textsci}{s}{\textsubring{d}}} \dicPos{f}[7] \dicFlx{(‑ar, ‑ir)}[1] \dicDirectTranslationCS{přítomnost, prezence} \dicExampleIS{vera feiminn í návist hennar} \dicExampleCS{být nesmělý v~její přítomnosti}
\dicEntry[návígi] \dicTerm{ná··vígi} \dicIPA{{n}{au}{\textlengthmark}{v}{i}{j}{\textsci}} \dicPos{n}[2] \dicFlx{(‑s)}[20] \dicDirectTranslationCS{boj zblízka} \dicExampleIS{berjast í návígi} \dicExampleCS{bojovat zblízka}
\dicEntry[neðan] \dicTerm{neðan} \dicsymFrequent\  \dicIPA{{n}{\textepsilon}{\textlengthmark}{ð}{a}{\textsubring{n}}} \dicPos{prep} \dicFlx{gen} \dicDirectTranslationCS{(dole) pod} \dicExampleIS{neðan bæjarins} \dicExampleCS{(dole) pod městem} \dicAntonym{ofan};  \dicIdiom{að}[neðan]{ \dicPhraseIS{að neðan}} \dicFlx{adv} {\textbf{a.}} \dicDirectTranslationCS{zespod(u), (ze)zdola, odspodu};  {\textbf{b.}} \dicDirectTranslationCS{dole (v~textu ap.), níže};  \dicIdiom{neðan}[af]{ \dicPhraseIS{neðan af e‑u}} \dicFlx{prep} \dicDirectTranslationCS{zdola z~(čeho), zespod z~(čeho)};  \dicIdiom{neðan}[á]{ \dicPhraseIS{neðan á e‑ð}} \dicFlx{prep} \dicDirectTranslationCS{naspod (čeho), dospodu (čeho), vespod (čeho)} \dicIndirectTranslationCS{(o~pohybu)}; { \dicPhraseIS{neðan á e‑u}} \dicFlx{prep} \dicDirectTranslationCS{naspodu (čeho), vespodu (čeho)} \dicIndirectTranslationCS{(o~pozici)};  \dicIdiom{neðan}[frá]{ \dicPhraseIS{neðan frá e‑u}} \dicFlx{prep} \dicDirectTranslationCS{zezdola z~(čeho), zespod z~(čeho)}; { \dicPhraseIS{neðan frá}} \dicFlx{adv} \dicDirectTranslationCS{zdola, zezdola, odzdola, zespodu};  \dicIdiom{neðan}[í]{ \dicPhraseIS{neðan í e‑u}} \dicFlx{prep} \dicDirectTranslationCS{odspodu (čeho), zespodu (čeho)}; { \dicPhraseIS{fá sér neðan í því}} \dicLangCat{přen.} \dicDirectTranslationCS{dát si do nosu} \dicIndirectTranslationCS{(napít se alkoholu)};  \dicIdiom{neðan}[til]{ \dicPhraseIS{neðan til}} \dicFlx{adv} \dicDirectTranslationCS{v~dolní části, naspodu};  \dicIdiom{neðan}[undir]{ \dicPhraseIS{neðan undir e‑u}} \dicFlx{prep} \dicDirectTranslationCS{těsně pod (čím)} \dicIndirectTranslationCS{(o~pozici)}; { \dicPhraseIS{neðan undir e‑ð}} \dicFlx{prep} \dicDirectTranslationCS{těsně pod (co)} \dicIndirectTranslationCS{(o~pohybu)};  \dicIdiom{neðan}[við]{ \dicPhraseIS{neðan við e‑ð}} \dicFlx{prep} \dicDirectTranslationCS{dole pod (čím)}; { \dicPhraseIS{neðan við}} \dicFlx{adv} \dicDirectTranslationCS{zdola, dolů}
\dicEntry[neðanjarðargöng] \dicTerm{neðan·jarðar··göng} \dicIPA{{n}{\textepsilon}{\textlengthmark}{ð}{a}{n}{j}{a}{r}{ð}{a}{r}{\r{g}}{\oe i}{\ng}{\r{g}}} \dicPos{n}[2] \dicFlx{pl}[9] \dicDirectTranslationCS{podzemní tunel}
\dicEntry[neðanjarðarlest] \dicTerm{neðan·jarðar··lest} \dicIPA{{n}{\textepsilon}{\textlengthmark}{ð}{a}{n}{j}{a}{r}{ð}{a}{r}{l}{\textepsilon}{s}{\textsubring{d}}} \dicPos{f}[7] \dicFlx{(‑ar, ‑ir)}[1] \dicDirectTranslationCS{metro, podzemní dráha}
\dicEntry[neðanmáls] \dicTerm{neðan··máls} \dicIPA{{n}{\textepsilon}{\textlengthmark}{ð}{a}{n}{m}{au}{l}{s}} \dicPos{adv} \dicDirectTranslationCS{pod čarou (poznámky v~knize ap.)}
\dicEntry[neðanmálsgrein] \dicTerm{neðan·máls··grein} \dicIPA{{n}{\textepsilon}{\textlengthmark}{ð}{a}{n}{m}{au}{l}{s}{\r{g}}{r}{ei}{\textsubring{n}}} \dicPos{f}[4] \dicFlx{(‑ar, ‑ar)}[1] \dicDirectTranslationCS{poznámka pod čarou, anotace}
\dicEntry[neðanverður] \dicTerm{neðan··verður} \dicIPA{{n}{\textepsilon}{\textlengthmark}{ð}{a}{n}{v}{\textepsilon}{r}{ð}{\textscy}{\textsubring{r}}} \dicPos{adj}[2]\dicFlx{}[-4] \dicDirectTranslationCS{spodní, dolní};  \dicPhraseIS{að neðanverðu} \dicFlx{adv} \dicDirectTranslationCS{zezdola, odspodu, vespod, ze spodní strany}
\dicEntry[neðar] \dicTerm{neðar} \dicsymFrequent\  \dicIPA{{n}{\textepsilon}{\textlengthmark}{ð}{a}{\textsubring{r}}} \dicPos{adv} \dicFlx{comp (pos niðri, sup neðst)} \dicDirectTranslationCS{níž(e)} \dicExampleIS{á smærri mynd neðar í fréttinni} \dicExampleCS{na menším obrázku níže ve zprávě}
\dicEntry[neðri] \dicTerm{neðri} \dicsymFrequent\  \dicIPA{{n}{\textepsilon}{ð}{r}{\textsci}} \dicPos{adj}[12] \dicFlx{comp (sup neðstur)}[6] \dicDirectTranslationCS{nižší, dolní, spodní} \dicExampleIS{mælingarnar í neðri bekkjunum} \dicExampleCS{měření v~nižších třídách}
\dicEntry[neðst] \dicTerm{neðst} \dicIPA{{n}{\textepsilon}{ð}{s}{\textsubring{d}}} \dicPos{adv} \dicFlx{sup} \dicLink{niðri}
\dicEntry[neðstur] \dicTerm{neðstur} \dicIPA{{n}{\textepsilon}{ð}{s}{\textsubring{d}}{\textscy}{\textsubring{r}}} \dicPos{adj}[12] \dicFlx{sup (comp neðri)}[6] \dicDirectTranslationCS{nejnižší}
\dicEntry[nef] \dicTerm{nef} \dicsymFrequent\  \dicIPA{{n}{\textepsilon}{\textlengthmark}{f}} \dicPos{n}[2] \dicFlx{(‑s, ‑)}[13] \textbf{1.} \dicFieldCat{anat.} \dicDirectTranslationCS{nos} \dicExampleIS{stórt nef} \dicExampleCS{velký nos};  \dicPhraseIS{fitja upp á nefið} \dicLangCat{přen.} \dicDirectTranslationCS{ohrnout\,/\addthin ohrnovat nos};  \dicPhraseIS{taka í nefið} \dicDirectTranslationCS{šňupnout si}  \textbf{2.} \dicDirectTranslationCS{zobák};  \dicIdiom{nef}{ \dicPhraseIS{hafa nef fyrir e‑u}} \dicLangCat{přen.} \dicDirectTranslationCS{mít na (co) nos}; { \dicPhraseIS{stinga saman nefjum}} \dicLangCat{přen.} \dicDirectTranslationCS{dát hlavy dohromady}; { \dicPhraseIS{stökkva upp á nef sér}} \dicDirectTranslationCS{rozčílit se}; { \dicPhraseIS{vera með nefið niðri í öllu}} \dicLangCat{přen.} \dicDirectTranslationCS{do všeho strkat nos}
\dicEntry[nefbroddur] \dicTerm{nef··brodd|ur} \dicIPA{{n}{\textepsilon}{v}{\textsubring{b}}{r}{\textopeno}{\textsubring{d}}{\textscy}{\textsubring{r}}} \dicPos{m}[6] \dicFlx{(‑s, ‑ar)}[4] \dicDirectTranslationCS{špička nosu}
\dicEntry[nefdýr] \dicTerm{nef··dýr} \dicIPA{{n}{\textepsilon}{v}{\textsubring{d}}{i}{\textsubring{r}}} \dicPos{n}[2] \dicFlx{(‑s, ‑)}[5] \dicPhraseIS{nefdýr} \dicFlx{pl} \dicFieldCat{zool.} \dicDirectTranslationCS{ptakořitní (ptakopysk, ježura)} \textit{(l.~{\textLA{Monotremata}})}
\dicEntry[nefhljóð] \dicTerm{nef··hljóð} \dicIPA{{n}{\textepsilon}{f}{\textsubring{l}}{j}{ou}{\texttheta}} \dicPos{n}[2] \dicFlx{(‑s, ‑)}[5] \dicFieldCat{jaz.} \dicDirectTranslationCS{nosová hláska, nosovka, nazála}
\dicEntry[nefjun] \dicTerm{nefj|un} \dicIPA{{n}{\textepsilon}{v}{j}{\textscy}{\textsubring{n}}} \dicPos{f}[7] \dicFlx{(‑unar)}[9] \dicFieldCat{jaz.} \dicDirectTranslationCS{nazalizace}
\dicEntry[nefkveðinn] \dicTerm{nef··kveðinn} \dicIPA{{n}{\textepsilon}{f}{k\smash{\textsuperscript{h}}}{v}{\textepsilon}{ð}{\textsci}{\textsubring{n}}} \dicPos{adj}[6]\dicFlx{}[-6] \dicFieldCat{jaz.} \dicDirectTranslationCS{nosový, nazální}
\dicEntry[nefmæltur] \dicTerm{nef··mæltur} \dicIPA{{n}{\textepsilon}{v}{m}{a}{i}{\textsubring{l}}{\textsubring{d}}{\textscy}{\textsubring{r}}} \dicPos{adj}[1]\dicFlx{}[-10] \dicFieldCat{jaz.} \dicSynonym{nefkveðinn} \dicDirectTranslationCS{nosový, nazální}
\dicEntry[nefna] \dicTerm{nefn|a} \dicsymFrequent\  \dicIPA{{n}{\textepsilon}{\textsubring{b}}{n}{a}} \dicPos{v}[2] \dicFlx{(‑di, ‑t)}[141] \dicFlx{acc} \textbf{1.} \dicSynonym*{gefa nafn} \dicDirectTranslationCS{(po)jmenovat, nazvat, nazývat} \dicExampleIS{nefna hann Sveinn} \dicExampleCS{pojmenovat ho Sveinn}  \textbf{2.} \dicSynonym*{minnast á} \dicDirectTranslationCS{zmínit (se), zmiňovat (se), uvést, uvádět, (vy)jmenovat (příklad ap.)};  \dicPhraseIS{nefna e‑ð við e‑n} \dicDirectTranslationCS{zmínit se (komu) o~(čem)}  \textbf{3.} \dicSynonym{tilnefna} \dicDirectTranslationCS{jmenovat, nominovat};  \dicIdiom{nefna}[til]{ \dicPhraseIS{nefna e‑n til e‑s}} \dicDirectTranslationCS{zmínit (koho) ve spojení s~(čím)};  \dicIdiom{nefnast}{ \dicPhraseIS{nefnast}} \dicFlx{refl} \dicSynonym{heita} \dicDirectTranslationCS{jmenovat se, nazývat se}
\dicEntry[nefnari] \dicTerm{nefn··ar|i} \dicIPA{{n}{\textepsilon}{\textsubring{b}}{n}{a}{r}{\textsci}} \dicPos{m}[1] \dicFlx{(‑a, ‑ar)}[13] \dicFieldCat{mat.} \dicDirectTranslationCS{jmenovatel}
\dicEntry[nefnd] \dicTerm{nefnd} \dicsymFrequent\  \dicIPA{{n}{\textepsilon}{m}{\textsubring{d}}} \dicPos{f}[7] \dicFlx{(‑ar, ‑ir)}[1] \dicDirectTranslationCS{výbor, komise, rada} \dicExampleIS{kalla nefndina saman} \dicExampleCS{svolat výbor}
\dicEntry[nefndarmaður] \dicTerm{nefndar··|maður} \dicIPA{{n}{\textepsilon}{m}{\textsubring{d}}{a}{r}{m}{a}{ð}{\textscy}{\textsubring{r}}} \dicPos{m}[13] \dicFlx{(‑manns, ‑menn)}[2] \dicDirectTranslationCS{člen(ka) výboru\,/\addthin komise}
\dicEntry[nefnifall] \dicTerm{nefni··|fall} \dicIPA{{n}{\textepsilon}{\textsubring{b}}{n}{\textsci}{f}{a}{\textsubring{d}}{\textsubring{l}}} \dicPos{n}[2] \dicFlx{(‑falls, ‑föll)}[8] \dicFieldCat{jaz.} \dicDirectTranslationCS{1. pád, nominativ}
\dicEntry[nefnilega] \dicTerm{nefni··lega} \dicsymFrequent\  \dicIPA{{n}{\textepsilon}{\textsubring{b}}{n}{\textsci}{l}{\textepsilon}{\textbabygamma}{a}} \dicPos{adv} \dicSynonym*{sem sagt} \dicDirectTranslationCS{a~to, jmenovitě, totiž} \dicExampleIS{það er nefnilega þannig að} \dicExampleCS{je to totiž tak, že}
\dicEntry[nefskattur] \dicTerm{nef··skatt|ur} \dicIPA{{n}{\textepsilon}{f}{s}{\r{g}}{a}{h}{\textsubring{d}}{\textscy}{\textsubring{r}}} \dicPos{m}[6] \dicFlx{(‑s, ‑ar)}[5] \dicFieldCat{ekon.} \dicDirectTranslationCS{daň z~hlavy}
\dicEntry[neftóbak] \dicTerm{nef··tóbak} \dicIPA{{n}{\textepsilon}{f}{t\smash{\textsuperscript{h}}}{ou}{\textsubring{b}}{a}{\r{g}}} \dicPos{n}[2] \dicFlx{(‑s)}[2] \dicDirectTranslationCS{šňupací tabák}
\dicEntry[negatífur] \dicTerm{negatífur} \dicIPA{{n}{\textepsilon}{\textlengthmark}{\r{g}}{a}{\textsubring{d}}{i}{v}{\textscy}{\textsubring{r}}} \dicPos{adj}[1]\dicFlx{}[-1] \dicSynonym{neikvæður} \dicDirectTranslationCS{negativní, záporný} \dicExampleIS{Niðurstaða er negatíf.} \dicExampleCS{Výsledek je negativní.}
\dicEntry[negla] \dicTerm{negl|a\smash{\textsuperscript{1}}} \dicIPA{{n}{\textepsilon}{\r{g}}{l}{a}} \dicPos{f}[1] \dicFlx{(‑u, ‑ur)}[19] \dicDirectTranslationCS{zátka (lodi)}
\dicEntry[negla] \dicTerm{negl|a\smash{\textsuperscript{2}}} \dicIPA{{n}{\textepsilon}{\r{g}}{l}{a}} \dicPos{v}[2] \dicFlx{(‑di, ‑t)}[141] \dicFlx{acc} \dicDirectTranslationCS{přibít, přibíjet, přitlouct (hřebík ap.)} \dicExampleIS{negla myndina á vegginn} \dicExampleCS{přibít obraz na zeď}
\dicEntry[neglur] \dicTerm{neglur} \dicIPA{{n}{\textepsilon}{\r{g}}{l}{\textscy}{\textsubring{r}}} \dicPos{f} \dicFlx{pl nom} \dicLink{nögl}
\dicEntry[negrakoss] \dicTerm{negra··koss} \dicIPA{{n}{\textepsilon}{\textbabygamma}{r}{a}{k\smash{\textsuperscript{h}}}{\textopeno}{s}} \dicPos{m}[4] \dicFlx{(‑, ‑ar)}[28] \dicFieldCat{kulin.} \dicDirectTranslationCS{indiánek}
\dicEntry[negrasálmur] \dicTerm{negra··sálm|ur} \dicIPA{{n}{\textepsilon}{\textbabygamma}{r}{a}{s}{au}{l}{m}{\textscy}{\textsubring{r}}} \dicPos{m}[6] \dicFlx{(‑s, ‑ar)}[4] \dicDirectTranslationCS{černošský spirituál}
\dicEntry[negri] \dicTerm{negr|i} \dicIPA{{n}{\textepsilon}{\textbabygamma}{r}{\textsci}} \dicPos{m}[1] \dicFlx{(‑a, ‑ar)}[1] \dicLangCat{han.} \dicSynonym{blökkumaður} \dicDirectTranslationCS{negr, negerka}
\dicEntry[nei] \dicTerm{nei} \dicsymFrequent\  \dicIPA{{n}{ei}{\textlengthmark}} \dicPos{adv} \textbf{1.} \dicDirectTranslationCS{ne} \dicIndirectTranslationCS{(při záporné odpovědi)} \dicExampleIS{Nei, takk.} \dicExampleCS{Ne, díky.}  \textbf{2.} \dicDirectTranslationCS{no ne} \dicIndirectTranslationCS{(vyjadřuje údiv, překvapení)}
\dicEntry[neikvæðni] \dicTerm{nei··kvæðn|i} \dicIPA{{n}{ei}{\textlengthmark}{k\smash{\textsuperscript{h}}}{v}{a}{i}{ð}{n}{\textsci}} \dicPos{f}[3] \dicFlx{(‑i)}[3] \dicDirectTranslationCS{negativní přístup\,/\addthin nastavení, negativismus}
\dicEntry[neikvæður] \dicTerm{nei··kvæður} \dicIPA{{n}{ei}{\textlengthmark}{k\smash{\textsuperscript{h}}}{v}{a}{i}{ð}{\textscy}{\textsubring{r}}} \dicPos{adj}[2]\dicFlx{}[-6] \textbf{1.} \dicSynonym{neitandi} \dicDirectTranslationCS{negativní, záporný};  \dicPhraseIS{neikvæð mynd} \dicDirectTranslationCS{negativ}  \textbf{2.} \dicFieldCat{fyz.} \dicDirectTranslationCS{záporný (náboj ap.)} \dicAntonym{jákvæður}
\dicEntry[neind] \dicTerm{neind} \dicIPA{{n}{ei}{n}{\textsubring{d}}} \dicPos{f}[7] \dicFlx{(‑ar, ‑ir)}[1] \textbf{1.} \dicFieldCat{filos.} \dicDirectTranslationCS{nicota}  \textbf{2.} \dicFieldCat{fyz.} \dicSynonym{nifteind} \dicDirectTranslationCS{neutron}
\dicEntry[neinn] \dicTerm{neinn} \dicsymFrequent\  \dicIPA{{n}{ei}{\textsubring{d}}{\textsubring{n}}} \dicPos{pron}[10] \dicFlx{indef} \dicDirectTranslationCS{nikdo} \dicIndirectTranslationCS{(po záporu)};  \dicPhraseIS{ekki neinn} \dicDirectTranslationCS{nikdo};  \dicPhraseIS{ekki neins staðar} \dicFlx{adv} \dicDirectTranslationCS{nikde};  \dicPhraseIS{ekki neitt} \dicDirectTranslationCS{nic};  \dicPhraseIS{það er ekki til neins} \dicDirectTranslationCS{to je k~ničemu}
\dicEntry[neista] \dicTerm{neist|a} \dicIPA{{n}{ei}{s}{\textsubring{d}}{a}} \dicPos{v}[1] \dicFlx{(‑aði)}[44] \dicDirectTranslationCS{(za)jiskřit};  \dicPhraseIS{það neistar af e‑u} \dicFlx{impers} \dicDirectTranslationCS{(co) jiskří (oheň ap.)}
\dicEntry[neisti] \dicTerm{neist|i} \dicIPA{{n}{ei}{s}{\textsubring{d}}{\textsci}} \dicPos{m}[1] \dicFlx{(‑a, ‑ar)}[1] \dicSynonym*{gneisti} \dicDirectTranslationCS{jiskra, jiskřička}
\dicEntry[neita] \dicTerm{neit|a} \dicIPA{{n}{ei}{\textlengthmark}{\textsubring{d}}{a}} \dicPos{v}[1] \dicFlx{(‑aði)}[44] \dicFlx{dat} \textbf{1.} \dicSynonym{aftaka\smash{\textsuperscript{2}}} \dicDirectTranslationCS{odmítnout, odmítat, zamítnout, zamítat, popřít, popírat} \dicExampleIS{neita öllum sakargiftum} \dicExampleCS{odmítnout všechna obvinění}  \textbf{2.} \dicSynonym{hafna\smash{\textsuperscript{1}}} \dicDirectTranslationCS{upřít, upírat, odepřít, odepírat};  \dicPhraseIS{neita e‑m um e‑ð} \dicDirectTranslationCS{odepřít (komu co)} \dicExampleIS{neita honum um hjálp} \dicExampleCS{odepřít mu pomoc}
\dicEntry[neitandi] \dicTerm{neit··andi} \dicIPA{{n}{ei}{\textlengthmark}{\textsubring{d}}{a}{n}{\textsubring{d}}{\textsci}} \dicPos{adj}[13] \dicFlx{indecl}[1] \dicDirectTranslationCS{odmítavý, zamítavý, záporný};  \dicPhraseIS{svara e‑u neitandi} \dicDirectTranslationCS{odpovědět na (co) záporně};  \dicPhraseIS{neitandi forskeyti} \dicFieldCat{jaz.} \dicDirectTranslationCS{záporná předpona}
\dicEntry[neitt] \dicTerm{neitt} \dicIPA{{n}{ei}{h}{\textsubring{d}}} \dicPos{pron} \dicFlx{n nom} \dicLink{neinn}
\dicEntry[neitun] \dicTerm{neit|un} \dicIPA{{n}{ei}{\textlengthmark}{\textsubring{d}}{\textscy}{\textsubring{n}}} \dicPos{f}[7] \dicFlx{(‑unar, ‑anir)}[8] \textbf{1.} \dicSynonym{synjun} \dicDirectTranslationCS{odmítnutí, zamítnutí}  \textbf{2.} \dicSynonym{afsvar} \dicDirectTranslationCS{odepření, odřeknutí, upření}  \textbf{3.} \dicSynonym{afneitun} \dicDirectTranslationCS{odmítnutí, zřeknutí se (víry ap.)}  \textbf{4.} \dicFieldCat{jaz.} \dicDirectTranslationCS{negace, zápor}
\dicEntry[neitunarvald] \dicTerm{neitunar··|vald} \dicIPA{{n}{ei}{\textlengthmark}{\textsubring{d}}{\textscy}{n}{a}{r}{v}{a}{l}{\textsubring{d}}} \dicPos{n}[2] \dicFlx{(‑valds, ‑völd)}[8] \dicFieldCat{práv.} \dicDirectTranslationCS{veto, právo veta}
\dicEntry[nekt] \dicTerm{nekt} \dicIPA{{n}{\textepsilon}{x}{\textsubring{d}}} \dicPos{f}[4] \dicFlx{(‑ar)}[3] \dicDirectTranslationCS{nahota, nahost} \dicExampleIS{hylja nekt sína} \dicExampleCS{zakrýt svou nahotu}
\dicEntry[nektardans] \dicTerm{nektar··dans} \dicIPA{{n}{\textepsilon}{x}{\textsubring{d}}{a}{r}{\textsubring{d}}{a}{n}{s}} \dicPos{m}[4] \dicFlx{(‑, ‑ar)}[29] \dicDirectTranslationCS{striptýz}
\dicEntry[nektardansari] \dicTerm{nektar··dans·ar|i} \dicIPA{{n}{\textepsilon}{x}{\textsubring{d}}{a}{r}{\textsubring{d}}{a}{n}{s}{a}{r}{\textsci}} \dicPos{m}[1] \dicFlx{(‑a, ‑ar)}[10] \dicDirectTranslationCS{striptér(ka)}
\dicEntry[nektarína] \dicTerm{nektarín|a} \dicIPA{{n}{\textepsilon}{x}{\textsubring{d}}{a}{r}{i}{n}{a}} \dicPos{f}[1] \dicFlx{(‑u, ‑ur)}[7] \dicDirectTranslationCS{nektarinka}
\dicEntry[nellika] \dicTerm{nellik|a}\dicTerm{, drottningarblóm} \dicIPA{{n}\-{\textepsilon}\-{l}\-{\textlengthmark}\-{\textsci}\-{\r{g}}\-{a}\-} \dicPos{f}[1] \dicFlx{(‑u, ‑ur)}[7] \dicFieldCat{bot.} \dicDirectTranslationCS{karafiát, hvozdík} \textit{(l.~{\textLA{Dianthus}})}  \dicsymPhoto\ 
\dicFigure{1241.jpg}{Nellika}{Nellika - Zicha Ondřej, Biolib, Copyright/CC-BY-NC}
\dicEntry[nem] \dicTerm{nem} \dicIPA{{n}{\textepsilon}{\textlengthmark}{\textsubring{m}}} \dicPos{v} \dicFlx{ind praes sg 1 pers} \dicLink{nema\smash{\textsuperscript{1}}}
\dicEntry[nema] \dicTerm{nema\smash{\textsuperscript{1}}} \dicsymFrequent\  \dicIPA{{n}{\textepsilon}{\textlengthmark}{m}{a}} \dicPos{v}[6] \dicFlx{(nem, nam, námum, næmi, numið)}[19] \dicFlx{acc\,/\addthin dat} \textbf{1.} \dicFlx{acc} \dicSynonym{skynja} \dicDirectTranslationCS{vnímat} \dicExampleIS{nema hreyfingu} \dicExampleCS{vnímat pohyb}  \textbf{2.} \dicFlx{acc} \dicSynonym{læra} \dicDirectTranslationCS{(na)učit se, osvojit si, studovat} \dicExampleIS{nema sögur og ævintýri af vörum ömmu sinnar} \dicExampleCS{naučit se příběhy a~pohádky z~vyprávění babičky}  \textbf{3.} \dicFlx{dat} \dicSynonym{samsvara} \dicDirectTranslationCS{činit, dělat dohromady (o~ceně)} \dicExampleIS{Kostnaðurinn nemur 900 krónum.} \dicExampleCS{Náklady činí 900 korun.}  \textbf{4.} \dicPhraseIS{nema e‑u (á) brott} \dicDirectTranslationCS{odstranit (co), odklidit (co)}  \textbf{5.} \dicPhraseIS{nema staðar} \dicSynonym{stöðva} \dicDirectTranslationCS{zastavit (se)}  \textbf{6.} \dicDirectTranslationCS{zabrat, přisvojit si} \dicExampleIS{nema land} \dicExampleCS{zabrat zemi}
\dicEntry[nema] \dicTerm{nema\smash{\textsuperscript{2}}} \dicsymFrequent\  \dicIPA{{n}{\textepsilon}{\textlengthmark}{m}{a}} \dicPos{conj} \textbf{1.} \dicDirectTranslationCS{pokud  ne, jestli  ne} \dicExampleIS{Þú mátt ekki fara út nema foreldrar þínir gefi þér leyfi.} \dicExampleCS{Nemůžeš jít ven, pokud ti to rodiče nedovolí.}  \textbf{2.} \dicDirectTranslationCS{kromě, vyjma, s~výjimkou} \dicExampleIS{nema á sunnudögum} \dicExampleCS{kromě nedělí};  \dicIdiom{nema}{ \dicPhraseIS{nema hvað}} \dicFlx{adv} {\textbf{a.}} \dicSynonym{nema\smash{\textsuperscript{2}}} \dicDirectTranslationCS{až na, kromě};  {\textbf{b.}} \dicDirectTranslationCS{no nicméně} \dicIndirectTranslationCS{(používané po delší odbočce ve vyprávění pro návrat k~hlavnímu tématu)};  {\textbf{c.}} \dicSynonym{auðvitað} \dicDirectTranslationCS{samozřejmě, určitě};  \dicPhraseIS{nema síður sé} \dicFlx{adv} \dicDirectTranslationCS{naopak};  \dicPhraseIS{nema því aðeins} \dicFlx{conj} \dicDirectTranslationCS{leda že, ledaže}
\dicEntry[nemandi] \dicTerm{nem··|andi} \dicsymFrequent\  \dicIPA{{n}{\textepsilon}{\textlengthmark}{m}{a}{n}{\textsubring{d}}{\textsci}} \dicPos{m}[2] \dicFlx{(‑anda, ‑endur)}[1] \dicDirectTranslationCS{žák(yně), student(ka)} \dicExampleIS{lesblindir nemendur} \dicExampleCS{dyslektičtí žáci}
\dicEntry[nemendaráð] \dicTerm{nemenda··ráð} \dicIPA{{n}{\textepsilon}{\textlengthmark}{m}{\textepsilon}{n}{\textsubring{d}}{a}{r}{au}{\texttheta}} \dicPos{n}[2] \dicFlx{(‑s, ‑)}[5] \dicDirectTranslationCS{studentská rada}
\dicEntry[nemi] \dicTerm{nem|i} \dicIPA{{n}{\textepsilon}{\textlengthmark}{m}{\textsci}} \dicPos{m}[1] \dicFlx{(‑a, ‑ar)}[1] \textbf{1.} \dicSynonym{nemandi} \dicDirectTranslationCS{žák(yně), student(ka)}  \textbf{2.} \dicDirectTranslationCS{čidlo, snímač, senzor} \dicIndirectTranslationCS{(zvláště ve složeninách)} \dicExampleIS{hljóðnemi} \dicExampleCS{mikrofon}
\dicEntry[nenna] \dicTerm{nenn|a} \dicsymFrequent\  \dicIPA{{n}{\textepsilon}{n}{\textlengthmark}{a}} \dicPos{v}[2] \dicFlx{(‑ti, ‑t)}[78] \dicFlx{dat} \textbf{1.} \dicDirectTranslationCS{chtít, mít chuť\,/\addthin náladu} \dicIndirectTranslationCS{(užívané většinou v~záporu)} \dicExampleIS{nenna ekki að læra} \dicExampleCS{nemít chuť se učit}  \textbf{2.} \dicDirectTranslationCS{být tak laskav, nevadit} \dicIndirectTranslationCS{(zdvořilá žádost)} \dicExampleIS{Nennirðu að ná í diskinn fyrir mig?} \dicExampleCS{Byl bys tak hodný a~došel mi pro talíř?}
\dicEntry[neon] \dicTerm{neon} \dicIPA{{n}{\textepsilon}{\textopeno}{\textsubring{n}}} \dicPos{n}[2] \dicFlx{(‑s)}[2] \dicFieldCat{chem.} \dicDirectTranslationCS{neon} \textit{(l.~{\textLA{Ne, Neon}})}
\dicEntry[neonljós] \dicTerm{neon··ljós} \dicIPA{{n}{\textepsilon}{\textopeno}{n}{l}{j}{ou}{s}} \dicPos{n}[2] \dicFlx{(‑s, ‑)}[5] \dicDirectTranslationCS{neonové světlo}
\dicEntry[Nepal] \dicTerm{Nepal} \dicIPA{{n}{\textepsilon}{\textlengthmark}{\textsubring{b}}{a}{\textsubring{l}}} \dicPos{n}[4] \dicFlx{indecl}[2] \dicFieldCat{geog.} \dicDirectTranslationCS{Nepál}
\dicEntry[Nepali] \dicTerm{Nepal|i} \dicIPA{{n}{\textepsilon}{\textlengthmark}{\textsubring{b}}{a}{l}{\textsci}} \dicPos{m}[1] \dicFlx{(‑a, ‑ar)}[8] \dicDirectTranslationCS{Nepálec, Nepálka}
\dicEntry[nepalska] \dicTerm{nepalska} \dicIPA{{n}{\textepsilon}{\textlengthmark}{\textsubring{b}}{a}{l}{s}{\r{g}}{a}} \dicPos{f}[1] \dicFlx{(nepölsku)}[2] \dicDirectTranslationCS{nepálština}
\dicEntry[nepalskur] \dicTerm{nepalskur} \dicIPA{{n}{\textepsilon}{\textlengthmark}{\textsubring{b}}{a}{l}{s}{\r{g}}{\textscy}{\textsubring{r}}} \dicPos{adj}[1] \dicFlx{(f nepölsk)}[3] \dicDirectTranslationCS{nepálský}
\dicEntry[nepja] \dicTerm{nepj|a} \dicIPA{{n}{\textepsilon}{\textlengthmark}{\textsubring{b}}{j}{a}} \dicPos{f}[1] \dicFlx{(‑u, ‑ur)}[7] \dicSynonym*{bitur kuldi} \dicDirectTranslationCS{třeskutý mráz}
\dicEntry[neptúnín] \dicTerm{neptúnín} \dicIPA{{n}{\textepsilon}{\textsubring{b}}{\textsubring{d}}{u}{n}{i}{\textsubring{n}}} \dicPos{n}[2] \dicFlx{(‑s)}[2] \dicFieldCat{chem.} \dicDirectTranslationCS{neptunium} \textit{(l.~{\textLA{Np, Neptunium}})}
\dicEntry[Neptúnus] \dicTerm{Neptúnus} \dicIPA{{n}{\textepsilon}{\textsubring{b}}{\textsubring{d}}{u}{n}{\textscy}{s}} \dicPos{m}[10] \dicFlx{(‑ar)}[3] \dicFieldCat{astro.} \dicDirectTranslationCS{Neptun}
\dicEntry[neri] \dicTerm{neri} \dicIPA{{n}{\textepsilon}{\textlengthmark}{r}{\textsci}} \dicPos{v} \dicFlx{ind\,/\addthin con pf sg 1 pers} \dicLink{núa}
\dicEntry[nería] \dicTerm{nerí|a} \dicIPA{{n}{\textepsilon}{\textlengthmark}{r}{i}{j}{a}} \dicPos{f}[1] \dicFlx{(‑u, ‑ur)}[7] \dicFieldCat{bot.} \dicSynonym{lárviðarrós} \dicDirectTranslationCS{oleandr, oleandr obecný} \textit{(l.~{\textLA{Nerium oleander}})}
\dicEntry[nerum] \dicTerm{nerum} \dicIPA{{n}{\textepsilon}{\textlengthmark}{r}{\textscy}{\textsubring{m}}} \dicPos{v} \dicFlx{ind pf pl 1 pers} \dicLink{núa}
\dicEntry[nervus] \dicTerm{nervus} \dicIPA{{n}{\textepsilon}{r}{v}{\textscy}{s}} \dicPos{adj}[5]\dicFlx{}[-1] \dicLangCat{hovor.} \dicSynonym{áhyggjufullur} \dicDirectTranslationCS{nervózní, rozrušený}
\dicEntry[nes] \dicTerm{nes} \dicIPA{{n}{\textepsilon}{\textlengthmark}{s}} \dicPos{n}[2] \dicFlx{(‑s, ‑)}[13] \dicDirectTranslationCS{mys}
\dicEntry[nesti] \dicTerm{nesti} \dicIPA{{n}{\textepsilon}{s}{\textsubring{d}}{\textsci}} \dicPos{n}[2] \dicFlx{(‑s)}[20] \dicDirectTranslationCS{svačina} \dicExampleIS{taka til nesti handa henni} \dicExampleCS{připravit pro ni svačinu}
\dicEntry[net] \dicTerm{net} \dicsymFrequent\  \dicIPA{{n}{\textepsilon}{\textlengthmark}{\textsubring{d}}} \dicPos{n}[2] \dicFlx{(‑s, ‑)}[5] \textbf{1.} \dicDirectTranslationCS{síť (rybářská ap.)}  \textbf{2.} \dicFieldCat{poč.} \dicDirectTranslationCS{síť, net} \dicExampleIS{á netinu} \dicExampleCS{na síti};  \dicPhraseIS{þráðlaust net} \dicFieldCat{poč.} \dicDirectTranslationCS{bezdrátová síť}
\dicEntry[netbanki] \dicTerm{net··bank|i} \dicIPA{{n}{\textepsilon}{\textlengthmark}{\textsubring{d}}{\textsubring{b}}{au}{\r{\textltailn}}{\r{\textObardotlessj}}{\textsci}} \dicPos{m}[1] \dicFlx{(‑a, ‑ar)}[8] \dicDirectTranslationCS{internetové bankovnictví}
\dicEntry[netfang] \dicTerm{net··|fang} \dicIPA{{n}{\textepsilon}{\textlengthmark}{\textsubring{d}}{f}{au}{\ng}{\r{g}}} \dicPos{n}[2] \dicFlx{(‑fangs, ‑föng)}[8] \dicFieldCat{poč.} \dicSynonym{tölvupóstfang} \dicDirectTranslationCS{e‑mailová adresa, e‑mail}
\dicEntry[nethimna] \dicTerm{net··himn|a} \dicIPA{{n}{\textepsilon}{\textlengthmark}{\textsubring{d}}{h}{\textsci}{m}{n}{a}} \dicPos{f}[1] \dicFlx{(‑u, ‑ur)}[7] \dicFieldCat{anat.} \dicSynonym{sjóna} \dicDirectTranslationCS{sítnice, retina}
\dicEntry[netla] \dicTerm{netl|a} \dicIPA{{n}{\textepsilon}{h}{\textsubring{d}}{l}{a}} \dicPos{f}[1] \dicFlx{(‑u, ‑ur)}[13] \dicFieldCat{bot.} \dicDirectTranslationCS{kopřiva} \textit{(l.~{\textLA{Urtica}})}  \dicsymPhoto\ 
\dicFigure{ds_image_netla_0_1.jpg}{Netla}{Netla - Frank Vincentz, CC BY-SA 3.0}
\dicEntry[netspjall] \dicTerm{net··|spjall} \dicIPA{{n}{\textepsilon}{\textlengthmark}{\textsubring{d}}{s}{\textsubring{b}}{j}{a}{\textsubring{d}}{\textsubring{l}}} \dicPos{n}[2] \dicFlx{(‑spjalls, ‑spjöll)}[8] \dicFieldCat{poč.} \dicDirectTranslationCS{(internetové) chatování, chat, internetová diskuze}
\dicEntry[nettálmi] \dicTerm{net··tálm|i} \dicIPA{{n}{\textepsilon}{\textlengthmark}{t\smash{\textsuperscript{h}}}{au}{l}{m}{\textsci}} \dicPos{m}[1] \dicFlx{(‑a, ‑ar)}[1] \dicFieldCat{poč.} \dicDirectTranslationCS{firewall}
\dicEntry[nettengdur] \dicTerm{net··tengdur} \dicIPA{{n}{\textepsilon}{\textlengthmark}{t\smash{\textsuperscript{h}}}{\textepsilon}{\ng}{\textsubring{d}}{\textscy}{\textsubring{r}}} \dicPos{adj}[2]\dicFlx{}[-17] \dicFieldCat{poč.} \dicDirectTranslationCS{připojený k~síti}
\dicEntry[nettenging] \dicTerm{net··teng·ing} \dicIPA{{n}{\textepsilon}{\textlengthmark}{t\smash{\textsuperscript{h}}}{ei}{\textltailn}{\r{\textObardotlessj}}{i}{\ng}{\r{g}}} \dicPos{f}[4] \dicFlx{(‑ar)}[7] \dicFieldCat{poč.} \dicDirectTranslationCS{internetové připojení, připojení k~síti}
\dicEntry[nettó] \dicTerm{nettó} \dicIPA{{n}{\textepsilon}{h}{\textsubring{d}}{ou}} \dicPos{adv} \dicDirectTranslationCS{netto} \dicAntonym{brúttó}
\dicEntry[nettur] \dicTerm{nettur} \dicIPA{{n}{\textepsilon}{h}{\textsubring{d}}{\textscy}{\textsubring{r}}} \dicPos{adj}[1]\dicFlx{}[-10] \dicSynonym{fríður} \dicDirectTranslationCS{elegantní, ladný}
\dicEntry[netútgáfa] \dicTerm{net··út·gáf|a} \dicIPA{{n}{\textepsilon}{\textlengthmark}{\textsubring{d}}{u}{\textsubring{d}}{\r{g}}{au}{v}{a}} \dicPos{f}[1] \dicFlx{(‑u, ‑ur)}[13] \dicDirectTranslationCS{elektronické vydání}
\dicEntry[netvarp] \dicTerm{net··|varp} \dicIPA{{n}{\textepsilon}{\textlengthmark}{\textsubring{d}}{v}{a}{\textsubring{r}}{\textsubring{b}}} \dicPos{n}[2] \dicFlx{(‑varps, ‑vörp)}[8] \dicFieldCat{poč.} \dicDirectTranslationCS{podcast}
\dicEntry[netverji] \dicTerm{net··verj|i} \dicIPA{{n}{\textepsilon}{\textlengthmark}{\textsubring{d}}{v}{\textepsilon}{r}{j}{\textsci}} \dicPos{m}[1] \dicFlx{(‑a, ‑ar)}[1] \dicFieldCat{poč.} \dicDirectTranslationCS{internaut(ka)} \dicIndirectTranslationCS{(pokročilý uživatel internetu)}
\dicEntry[netverslun] \dicTerm{net··versl|un} \dicIPA{{n}{\textepsilon}{\textlengthmark}{\textsubring{d}}{v}{\textepsilon}{\textsubring{r}}{s}{\textsubring{d}}{l}{\textscy}{\textsubring{n}}} \dicPos{f}[7] \dicFlx{(‑unar, ‑anir)}[8] \dicSynonym{vefverslun} \dicDirectTranslationCS{internetový\,/\addthin elektronický obchod, e‑shop}
\dicEntry[netþjónn] \dicTerm{net··þjón|n} \dicIPA{{n}{\textepsilon}{\textlengthmark}{\textsubring{d}}{\texttheta}{j}{ou}{\textsubring{d}}{\textsubring{n}}} \dicPos{m}[6] \dicFlx{(‑s, ‑ar)}[42] \dicFieldCat{poč.} \dicDirectTranslationCS{(webový) server}
\dicEntry[neyð] \dicTerm{neyð} \dicIPA{{n}{ei}{\textlengthmark}{\texttheta}} \dicPos{f}[7] \dicFlx{(‑ar, ‑ir)}[1] \textbf{1.} \dicSynonym{nauðsyn} \dicDirectTranslationCS{donucení, přinucení} \dicExampleIS{gera e‑ð af neyð} \dicExampleCS{dělat (co) z~donucení}  \textbf{2.} \dicSynonym{nauð\smash{\textsuperscript{1}}} \dicDirectTranslationCS{tíseň, nesnáz, nouze} \dicExampleIS{vera staddur í neyð} \dicExampleCS{být v~nesnázích}  \textbf{3.} \dicSynonym{skortur} \dicDirectTranslationCS{nedostatek, bída, nouze} \dicExampleIS{Það er mikil neyð í landinu eftir styrjöldina.} \dicExampleCS{Po válce je v~zemi velká nouze.};  \dicProverb\  \dicPhraseIS{Neyðin kennir naktri konu að spinna.} \dicLangCat{přís.} \dicDirectTranslationCS{Nouze naučila Dalibora housti.}
\dicEntry[neyða] \dicTerm{ney|ða} \dicsymFrequent\  \dicIPA{{n}{ei}{\textlengthmark}{ð}{a}} \dicPos{v}[2] \dicFlx{(‑ddi, ‑tt)}[167] \dicFlx{acc\,/\addthin dat} \dicDirectTranslationCS{(při)nutit, donutit};  \dicPhraseIS{neyða e‑n til e‑s} \dicDirectTranslationCS{(při)nutit (koho) k~(čemu)} \dicExampleIS{neyða hana til viðbragða} \dicExampleCS{přinutit ji k~reakci};  \dicIdiom{neyðast}[til]{ \dicPhraseIS{neyðast til e‑s}} \dicFlx{refl} \dicDirectTranslationCS{být nucen k~(čemu), muset (co)}
\dicEntry[neyðarhöfn] \dicTerm{neyðar··|höfn} \dicIPA{{n}{ei}{\textlengthmark}{ð}{a}{\textsubring{r}}{h}{\oe}{\textsubring{b}}{\textsubring{n}}} \dicPos{f}[7] \dicFlx{(‑hafnar, ‑hafnir)}[16] \dicFieldCat{nám.} \dicSynonym*{lífhöfn} \dicDirectTranslationCS{nouzový přístav}
\dicEntry[neyðarkall] \dicTerm{neyðar··|kall} \dicIPA{{n}{ei}{\textlengthmark}{ð}{a}{\textsubring{r}}{k\smash{\textsuperscript{h}}}{a}{\textsubring{d}}{\textsubring{l}}} \dicPos{n}[2] \dicFlx{(‑kalls, ‑köll)}[8] \dicDirectTranslationCS{nouzové\,/\addthin tísňové volání}
\dicEntry[neyðarlegur] \dicTerm{neyðar··legur} \dicIPA{{n}{ei}{\textlengthmark}{ð}{a}{r}{l}{\textepsilon}{\textbabygamma}{\textscy}{\textsubring{r}}} \dicPos{adj}[1]\dicFlx{}[-8] \textbf{1.} \dicSynonym{óþægilegur} \dicDirectTranslationCS{trapný, rozpačitý, nepříjemný} \dicExampleIS{neyðarlegt tilsvar} \dicExampleCS{rozpačitá odpověď}  \textbf{2.} \dicSynonym{lúalegur} \dicDirectTranslationCS{sprostý, hanebný}
\dicEntry[neyðarlína] \dicTerm{neyðar··lín|a} \dicIPA{{n}{ei}{\textlengthmark}{ð}{a}{r}{l}{i}{n}{a}} \dicPos{f}[1] \dicFlx{(‑u, ‑ur)}[7] \dicDirectTranslationCS{nouzová\,/\addthin horká linka, hotline}
\dicEntry[neyðarnúmer] \dicTerm{neyðar··númer} \dicIPA{{n}{ei}{\textlengthmark}{ð}{a}{r}{n}{u}{m}{\textepsilon}{\textsubring{r}}} \dicPos{n}[2] \dicFlx{(‑s, ‑)}[5] \dicDirectTranslationCS{nouzové\,/\addthin tísňové číslo}
\dicEntry[neyðarráðstöfun] \dicTerm{neyðar··ráð·|stöfun} \dicIPA{{n}{ei}{\textlengthmark}{ð}{a}{r}{au}{ð}{s}{\textsubring{d}}{\oe}{v}{\textscy}{\textsubring{n}}} \dicPos{f}[7] \dicFlx{(‑stöfunar, ‑stafanir)}[11] \dicDirectTranslationCS{nouzové opatření}
\dicEntry[neyðarskeyti] \dicTerm{neyðar··skeyti} \dicIPA{{n}{ei}{\textlengthmark}{ð}{a}{\textsubring{r}}{s}{\r{\textObardotlessj}}{ei}{\textsubring{d}}{\textsci}} \dicPos{n}[2] \dicFlx{(‑s, ‑)}[14] \dicDirectTranslationCS{tísňový signál, SOS}
\dicEntry[neyðartilfelli] \dicTerm{neyðar··til·felli} \dicIPA{{n}{ei}{\textlengthmark}{ð}{a}{\textsubring{r}}{t\smash{\textsuperscript{h}}}{\textsci}{\textsubring{l}}{f}{\textepsilon}{\textsubring{d}}{l}{\textsci}} \dicPos{n}[2] \dicFlx{(‑s, ‑)}[14] \dicDirectTranslationCS{naléhavý\,/\addthin nouzový případ, případ nouze};  \dicPhraseIS{í neyðartilfelli} \dicFlx{adv} \dicDirectTranslationCS{v~případě nouze}
\dicEntry[neyðarúrræði] \dicTerm{neyðar··úr·ræði} \dicIPA{{n}{ei}{\textlengthmark}{ð}{a}{r}{u}{r}{a}{i}{ð}{\textsci}} \dicPos{n}[2] \dicFlx{(‑s, ‑)}[14] \dicDirectTranslationCS{nouzové\,/\addthin krajní východisko}
\dicEntry[neyðarútgangur] \dicTerm{neyðar··út·gang|ur} \dicIPA{{n}{ei}{\textlengthmark}{ð}{a}{r}{u}{\textsubring{d}}{\r{g}}{au}{\ng}{\r{g}}{\textscy}{\textsubring{r}}} \dicPos{m}[6] \dicFlx{(‑s, ‑ar)}[5] \dicDirectTranslationCS{nouzový východ}
\dicEntry[neysla] \dicTerm{neysl|a} \dicIPA{{n}{ei}{s}{\textsubring{d}}{l}{a}} \dicPos{f}[1] \dicFlx{(‑u, ‑ur)}[13] \dicDirectTranslationCS{spotřeba, konzumace} \dicExampleIS{neysla á matvælum} \dicExampleCS{spotřeba potravin}
\dicEntry[neyslulán] \dicTerm{neyslu··lán} \dicIPA{{n}{ei}{s}{\textsubring{d}}{l}{\textscy}{l}{au}{\textsubring{n}}} \dicPos{n}[2] \dicFlx{(‑s, ‑)}[5] \dicFieldCat{ekon.} \dicDirectTranslationCS{spotřebitelský úvěr}
\dicEntry[neyslusamfélag] \dicTerm{neyslu··sam·fé·|lag} \dicIPA{{n}{ei}{s}{\textsubring{d}}{l}{\textscy}{s}{a}{m}{f}{j}{\textepsilon}{l}{a}{x}} \dicPos{n}[2] \dicFlx{(‑lags, ‑lög)}[8] \dicDirectTranslationCS{spotřební\,/\addthin konzumní společnost}
\dicEntry[neysluvarningur] \dicTerm{neyslu··varn·ing|ur} \dicIPA{{n}{ei}{s}{\textsubring{d}}{l}{\textscy}{v}{a}{r}{\textsubring{d}}{n}{i}{\ng}{\r{g}}{\textscy}{\textsubring{r}}} \dicPos{m}[6] \dicFlx{(‑s)}[9] \dicDirectTranslationCS{spotřební zboží}
\dicEntry[neysluvatn] \dicTerm{neyslu··vatn} \dicIPA{{n}{ei}{s}{\textsubring{d}}{l}{\textscy}{v}{a}{h}{\textsubring{d}}{\textsubring{n}}} \dicPos{n}[2] \dicFlx{(‑s)}[2] \dicDirectTranslationCS{pitná voda}
\dicEntry[neyta] \dicTerm{neyt|a} \dicIPA{{n}{ei}{\textlengthmark}{\textsubring{d}}{a}} \dicPos{v}[2] \dicFlx{(‑ti, ‑t)}[54] \dicFlx{gen} \textbf{1.} \dicSynonym{njóta} \dicDirectTranslationCS{spotřebovat, (z)konzumovat} \dicExampleIS{neyta áfengis í hófi} \dicExampleCS{konzumovat alkohol s~mírou}  \textbf{2.} \dicSynonym{nota} \dicDirectTranslationCS{použít, (vy)užít} \dicExampleIS{neyta afls síns} \dicExampleCS{použít svou sílu}
\dicEntry[neytandi] \dicTerm{neyt··|andi} \dicIPA{{n}{ei}{\textlengthmark}{\textsubring{d}}{a}{n}{\textsubring{d}}{\textsci}} \dicPos{m}[2] \dicFlx{(‑anda, ‑endur)}[1] \dicDirectTranslationCS{spotřebitel(ka), konzument(ka)}
\dicEntry[neytendalán] \dicTerm{neytenda··lán} \dicIPA{{n}{ei}{\textlengthmark}{\textsubring{d}}{\textepsilon}{n}{\textsubring{d}}{a}{l}{au}{\textsubring{n}}} \dicPos{n}[2] \dicFlx{(‑s, ‑)}[5] \dicFieldCat{ekon.} \dicDirectTranslationCS{spotřebitelský úvěr}
\dicEntry[né] \dicTerm{né} \dicsymFrequent\  \dicIPA{{n}{j}{\textepsilon}{\textlengthmark}} \dicPos{conj} \dicPhraseIS{hvorki -- né} \dicDirectTranslationCS{ani -- ani} \dicExampleIS{hvorki meira né minna} \dicExampleCS{ani víc, ani míň};  \dicPhraseIS{né heldur} \dicDirectTranslationCS{ani}
\dicEntry[néri] \dicTerm{néri} \dicIPA{{n}{j}{\textepsilon}{\textlengthmark}{r}{\textsci}} \dicPos{v} \dicFlx{ind\,/\addthin con pf sg 1 pers} \dicLink{núa}
\dicEntry[nérum] \dicTerm{nérum} \dicIPA{{n}{j}{\textepsilon}{\textlengthmark}{r}{\textscy}{\textsubring{m}}} \dicPos{v} \dicFlx{ind pf pl 1 pers} \dicLink{núa}
\dicEntry[nf.] \dicTerm{nf.} \dicPos{zkr} \dicPhraseIS{nefnifall} \dicFieldCat{jaz.} \dicDirectTranslationCS{1. pád, nominativ}
\dicEntry[nfl.] \dicTerm{nfl.} \dicPos{zkr} \dicPhraseIS{nefnilega} \dicFlx{adv} \dicDirectTranslationCS{a~to, jmenovitě, totiž}
\dicEntry[nh.] \dicTerm{nh.} \dicPos{zkr} \dicPhraseIS{nafnháttur} \dicFieldCat{jaz.} \dicDirectTranslationCS{infinitiv}
\dicEntry[nið] \dicTerm{nið\smash{\textsuperscript{1}}} \dicIPA{{n}{\textsci}{\textlengthmark}{\texttheta}} \dicPos{n}[2] \dicFlx{(‑s, ‑)}[5] \dicDirectTranslationCS{novoluní}
\dicEntry[nið] \dicTerm{nið\smash{\textsuperscript{2}}} \dicIPA{{n}{\textsci}{\textlengthmark}{\texttheta}} \dicPos{n}[2] \dicFlx{(‑s)}[2] \dicSynonym{niður\smash{\textsuperscript{1}}} \dicDirectTranslationCS{zurčení, bublání, klokotání}
\dicEntry[niða] \dicTerm{nið|a} \dicIPA{{n}{\textsci}{\textlengthmark}{ð}{a}} \dicPos{v}[1] \dicFlx{(‑aði)}[44] \dicDirectTranslationCS{zurčet, bublat, klokotat (řeka ap.)} \dicExampleIS{Lækurinn niðar.} \dicExampleCS{Potok zurčí.}
\dicEntry[niðamyrkur] \dicTerm{niða··myrkur} \dicIPA{{n}{\textsci}{\textlengthmark}{ð}{a}{m}{\textsci}{\textsubring{r}}{\r{g}}{\textscy}{\textsubring{r}}} \dicPos{n}[2] \dicFlx{(‑s)}[28] \dicDirectTranslationCS{černočerná tma}
\dicEntry[Niðarós] \dicTerm{Niðar··ós} \dicIPA{{n}{\textsci}{\textlengthmark}{ð}{a}{r}{ou}{s}} \dicPos{m}[4] \dicFlx{(‑s)}[2] \dicFieldCat{geog., hist.} \dicDirectTranslationCS{Nidaros (Trondheim)} \dicIndirectTranslationCS{(město v~Norsku)}
\dicEntry[niðji] \dicTerm{niðj|i} \dicIPA{{n}{\textsci}{ð}{j}{\textsci}} \dicPos{m}[1] \dicFlx{(‑a, ‑ar)}[1] \dicSynonym{afkomandi} \dicDirectTranslationCS{potomek, potomkyně}
\dicEntry[niðra] \dicTerm{niðr|a} \dicIPA{{n}{\textsci}{ð}{r}{a}} \dicPos{v}[1] \dicFlx{(‑aði)}[1] \dicFlx{dat} \textbf{1.} \dicSynonym{lasta} \dicDirectTranslationCS{očerňovat, pomlouvat} \dicExampleIS{niðra e‑m\,/\addthin e‑u} \dicExampleCS{pomlouvat (koho\,/\addthin co)}  \textbf{2.} \dicSynonym{lítillækka} \dicDirectTranslationCS{ponížit, ponižovat, znevážit, znevažovat} \dicExampleIS{Skaðsamlegt er að niðra börnum á ýmsa vegu.} \dicExampleCS{Je škodlivé znevažovat děti jakýmkoliv způsobem.}
\dicEntry[niðrandi] \dicTerm{niðr··andi} \dicIPA{{n}{\textsci}{ð}{r}{a}{n}{\textsubring{d}}{\textsci}} \dicPos{adj}[13] \dicFlx{indecl}[2] \dicDirectTranslationCS{ponižující, znevažující};  \dicPhraseIS{á niðrandi hátt} \dicFlx{adv} \dicDirectTranslationCS{ponižujícím způsobem}
\dicEntry[niðri] \dicTerm{niðri} \dicsymFrequent\  \dicIPA{{n}{\textsci}{ð}{r}{\textsci}} \dicPos{adv} \dicFlx{(comp neðar, sup neðst)} \dicDirectTranslationCS{dole} \dicIndirectTranslationCS{(o~pobytu dole)} \dicExampleIS{niðri við ána} \dicExampleCS{dole u~řeky} \dicAntonym{uppi};  \dicPhraseIS{vera langt niðri} \dicLangCat{přen.} \dicDirectTranslationCS{být na dně, mít deprese}
\dicEntry[niðrun] \dicTerm{niðr|un} \dicIPA{{n}{\textsci}{ð}{r}{\textscy}{\textsubring{n}}} \dicPos{f}[7] \dicFlx{(‑unar)}[9] \dicSynonym{lítilsvirðing} \dicDirectTranslationCS{ponížení, ponižování, znevážení, znevažování}
\dicEntry[niður] \dicTerm{nið|ur\smash{\textsuperscript{1}}} \dicIPA{{n}{\textsci}{\textlengthmark}{ð}{\textscy}{\textsubring{r}}} \dicPos{m}[10] \dicFlx{(‑ar)}[20] \dicDirectTranslationCS{bublání, zurčení (vody ap.)} \dicExampleIS{eintóna niður árinnar} \dicExampleCS{monotónní bublání řeky}
\dicEntry[niður] \dicTerm{niður\smash{\textsuperscript{2}}} \dicsymFrequent\  \dicIPA{{n}{\textsci}{\textlengthmark}{ð}{\textscy}{\textsubring{r}}} \dicPos{adv} \dicDirectTranslationCS{dolů} \dicIndirectTranslationCS{(o~pohybu směrem dolů)} \dicExampleIS{ganga niður fjallið} \dicExampleCS{jít dolů z~hory} \dicAntonym{upp};  \dicIdiom{niður}[af]{ \dicPhraseIS{niður af e‑u}} \dicFlx{prep} \dicDirectTranslationCS{pod (čím)};  \dicIdiom{niður}[á við]{ \dicPhraseIS{niður á við}} \dicFlx{adv} \dicDirectTranslationCS{dolů};  \dicIdiom{niður}[eftir]{ \dicPhraseIS{niður eftir e‑u}} \dicFlx{prep} \dicDirectTranslationCS{dolů po (čem)}; { \dicPhraseIS{niður eftir}} \dicFlx{adv} \dicDirectTranslationCS{dolů};  \dicIdiom{niður}[frá]{ \dicPhraseIS{niður frá}} \dicFlx{adv} \dicDirectTranslationCS{dole, (ze)zdola};  \dicIdiom{niður}[fyrir]{ \dicPhraseIS{niður fyrir e‑ð}} \dicFlx{prep} \dicDirectTranslationCS{pod (co)} \dicExampleIS{niður fyrir hné} \dicExampleCS{pod kolena};  \dicIdiom{niður}[með]{ \dicPhraseIS{niður með e‑u}} \dicFlx{prep} \dicDirectTranslationCS{dolů podél (čeho)};  \dicIdiom{niður}[um]{ \dicPhraseIS{niður um e‑ð}} \dicFlx{prep} \dicDirectTranslationCS{dolů skrz (co)};  \dicIdiom{niður}[undan]{ \dicPhraseIS{niður undan e‑u}} \dicFlx{prep} \dicDirectTranslationCS{dole pod (čím)};  \dicIdiom{niður}[undir]{ \dicPhraseIS{niður undir e‑ð}} \dicFlx{prep} \dicDirectTranslationCS{(těsně) pod (co)} \dicIndirectTranslationCS{(o~pohybu)}; { \dicPhraseIS{niður undir e‑u}} \dicFlx{prep} \dicDirectTranslationCS{(těsně) pod (čím)} \dicIndirectTranslationCS{(o~pozici)};  \dicIdiom{niður}[úr]{ \dicPhraseIS{niður úr e‑u}} \dicFlx{prep} \dicDirectTranslationCS{dolů z~(čeho)}; { \dicPhraseIS{niður úr}} \dicFlx{adv} \dicDirectTranslationCS{dolů}
\dicEntry[niðurbrotinn] \dicTerm{niður··brotinn} \dicIPA{{n}{\textsci}{\textlengthmark}{ð}{\textscy}{r}{\textsubring{b}}{r}{\textopeno}{\textsubring{d}}{\textsci}{\textsubring{n}}} \dicPos{adj}[6]\dicFlx{}[-2] \dicDirectTranslationCS{zdrcený, zničený (bankrotem ap.)}
\dicEntry[niðurdreginn] \dicTerm{niður··dreginn} \dicIPA{{n}{\textsci}{\textlengthmark}{ð}{\textscy}{r}{\textsubring{d}}{r}{ei}{\textsci}{\textsubring{n}}} \dicPos{adj}[6]\dicFlx{}[-2] \dicSynonym{dapur} \dicDirectTranslationCS{skleslý, sklíčený, deprimovaný} \dicExampleIS{niðurdreginn yfir e‑u} \dicExampleCS{sklíčený (čím)}
\dicEntry[niðurfall] \dicTerm{niður··|fall} \dicIPA{{n}{\textsci}{\textlengthmark}{ð}{\textscy}{\textsubring{r}}{f}{a}{\textsubring{d}}{\textsubring{l}}} \dicPos{n}[2] \dicFlx{(‑falls, ‑föll)}[8] \dicDirectTranslationCS{odtok, odpad} \dicIndirectTranslationCS{(zařízení, kudy tekutina odtéká)}
\dicEntry[niðurgangur] \dicTerm{niður··gang|ur} \dicIPA{{n}{\textsci}{\textlengthmark}{ð}{\textscy}{r}{\r{g}}{au}{\ng}{\r{g}}{\textscy}{\textsubring{r}}} \dicPos{m}[6] \dicFlx{(‑s)}[7] \textbf{1.} \dicSynonym*{niðurganga} \dicDirectTranslationCS{sestup, sestupování, klesání} \dicExampleIS{niðurgangur sólarinnar} \dicExampleCS{klesání slunce}  \textbf{2.} \dicFieldCat{med.} \dicSynonym{þunnlífi} \dicDirectTranslationCS{průjem}
\dicEntry[niðurgreiða] \dicTerm{niður··grei|ða} \dicIPA{{n}{\textsci}{\textlengthmark}{ð}{\textscy}{r}{\r{g}}{r}{ei}{ð}{a}} \dicPos{v}[2] \dicFlx{(‑ddi, ‑tt)}[167] \dicFlx{acc} \dicDirectTranslationCS{subvencovat, dotovat}
\dicEntry[niðurgreiðsla] \dicTerm{niður··greiðsl|a} \dicIPA{{n}{\textsci}{\textlengthmark}{ð}{\textscy}{r}{\r{g}}{r}{ei}{ð}{s}{\textsubring{d}}{l}{a}} \dicPos{f}[1] \dicFlx{(‑u, ‑ur)}[13] \dicDirectTranslationCS{subvence, dotace}
\dicEntry[niðurkominn] \dicTerm{niður··kominn} \dicIPA{{n}{\textsci}{\textlengthmark}{ð}{\textscy}{\textsubring{r}}{k\smash{\textsuperscript{h}}}{\textopeno}{m}{\textsci}{\textsubring{n}}} \dicPos{adj}[6]\dicFlx{}[-6] \dicSynonym{staddur} \dicDirectTranslationCS{nacházející se, umístěný} \dicExampleIS{Hvar er hann niðurkominn?} \dicExampleCS{Kde se nachází?}
\dicEntry[niðurlag] \dicTerm{niður··|lag} \dicIPA{{n}{\textsci}{\textlengthmark}{ð}{\textscy}{r}{l}{a}{x}} \dicPos{n}[2] \dicFlx{(‑lags, ‑lög)}[8] \dicSynonym{lok} \dicDirectTranslationCS{zakončení, konec, závěr, finále} \dicExampleIS{niðurlagið á kvæðinu} \dicExampleCS{závěr básně};  \dicPhraseIS{ráða niðurlögum e‑s } \dicDirectTranslationCS{učinit (čemu) přítrž}
\dicEntry[niðurlagður] \dicTerm{niður··|lagður} \dicIPA{{n}{\textsci}{\textlengthmark}{ð}{\textscy}{r}{l}{a}{\textbabygamma}{ð}{\textscy}{\textsubring{r}}} \dicPos{adj}[2] \dicFlx{(f ‑lögð)}[3] \dicDirectTranslationCS{nakládaný, naložený} \dicExampleIS{niðurlagður fiskur} \dicExampleCS{naložená ryba}
\dicEntry[niðurlagning] \dicTerm{niður··lag·ning} \dicIPA{{n}{\textsci}{\textlengthmark}{ð}{\textscy}{r}{l}{a}{\r{g}}{n}{i}{\ng}{\r{g}}} \dicPos{f}[4] \dicFlx{(‑ar)}[7] \textbf{1.} \dicSynonym*{það að hætta e‑u} \dicDirectTranslationCS{ukončení, zakončení, likvidace}  \textbf{2.} \dicDirectTranslationCS{naložení, nakládání} \dicExampleIS{niðurlagning síldar} \dicExampleCS{nakládání sledě}
\dicEntry[niðurleið] \dicTerm{niður··leið} \dicIPA{{n}{\textsci}{\textlengthmark}{ð}{\textscy}{r}{l}{ei}{\texttheta}} \dicPos{f}[7] \dicFlx{(‑ar)}[3] \textbf{1.} \dicDirectTranslationCS{cesta dolů}  \textbf{2.} \dicSynonym{afturför} \dicDirectTranslationCS{úpadek, zhoršení}
\dicEntry[Niðurlendingur] \dicTerm{Niður·lend··ing|ur} \dicIPA{{n}{\textsci}{\textlengthmark}{ð}{\textscy}{r}{l}{\textepsilon}{n}{\textsubring{d}}{i}{\ng}{\r{g}}{\textscy}{\textsubring{r}}} \dicPos{m}[6] \dicFlx{(‑s, ‑ar)}[8] \dicSynonym{Hollendingur} \dicDirectTranslationCS{Nizozemec, Nizozemka}
\dicEntry[niðurlenska] \dicTerm{niður··lensk|a} \dicIPA{{n}{\textsci}{\textlengthmark}{ð}{\textscy}{r}{l}{\textepsilon}{n}{s}{\r{g}}{a}} \dicPos{f}[1] \dicFlx{(‑u)}[5] \dicSynonym{hollenska} \dicDirectTranslationCS{nizozemština}
\dicEntry[niðurlenskur] \dicTerm{niður··lenskur} \dicIPA{{n}{\textsci}{\textlengthmark}{ð}{\textscy}{r}{l}{\textepsilon}{n}{s}{\r{g}}{\textscy}{\textsubring{r}}} \dicPos{adj}[1]\dicFlx{}[-6] \dicSynonym{hollenskur} \dicDirectTranslationCS{nizozemský}
\dicEntry[niðurlot] \dicTerm{niður··lot} \dicIPA{{n}{\textsci}{\textlengthmark}{ð}{\textscy}{r}{l}{\textopeno}{\textsubring{d}}} \dicPos{n}[2] \dicFlx{pl}[1] \dicPhraseIS{vera að niðurlotum kominn} \dicDirectTranslationCS{být na hranici vyčerpání (fyzického ap.)}
\dicEntry[niðurlútur] \dicTerm{niður··lútur} \dicIPA{{n}{\textsci}{\textlengthmark}{ð}{\textscy}{r}{l}{u}{\textsubring{d}}{\textscy}{\textsubring{r}}} \dicPos{adj}[1]\dicFlx{}[-1] \dicSynonym{skömmustulegur} \dicDirectTranslationCS{skleslý, sklíčený}
\dicEntry[niðurlæging] \dicTerm{niður··læg·ing} \dicIPA{{n}{\textsci}{\textlengthmark}{ð}{\textscy}{r}{l}{a}{i}{j}{i}{\ng}{\r{g}}} \dicPos{f}[4] \dicFlx{(‑ar, ‑ar)}[5] \dicSynonym{smán} \dicDirectTranslationCS{ponížení, ponižování} \dicExampleIS{verða fyrir niðurlægingu} \dicExampleCS{být ponižován}
\dicEntry[niðurlægja] \dicTerm{niður··læg|ja} \dicIPA{{n}{\textsci}{\textlengthmark}{ð}{\textscy}{r}{l}{a}{i}{j}{a}} \dicPos{v}[2] \dicFlx{(‑ði, ‑t)}[89] \dicFlx{acc} \dicSynonym{óvirða} \dicDirectTranslationCS{ponížit, ponižovat, pokořit, pokořovat} \dicExampleIS{niðurlægja e‑n með e‑u} \dicExampleCS{ponížit (koho čím)}
\dicEntry[Niðurlönd] \dicTerm{Niður··lönd} \dicIPA{{n}{\textsci}{\textlengthmark}{ð}{\textscy}{r}{l}{\oe}{n}{\textsubring{d}}} \dicPos{n}[2] \dicFlx{pl}[9] \dicFieldCat{geog.} \dicSynonym{Holland} \dicDirectTranslationCS{Nizozemsko}
\dicEntry[niðurníðsla] \dicTerm{niður··níðsl|a} \dicIPA{{n}{\textsci}{\textlengthmark}{ð}{\textscy}{r}{n}{i}{ð}{s}{\textsubring{d}}{l}{a}} \dicPos{f}[1] \dicFlx{(‑u)}[5] \dicDirectTranslationCS{zanedbanost, zchátralost} \dicExampleIS{Húsið er í mikilli niðurníðslu.} \dicExampleCS{Dům je velmi zanedbaný.}
\dicEntry[niðurrif] \dicTerm{niður··rif} \dicIPA{{n}{\textsci}{\textlengthmark}{ð}{\textscy}{r}{\textsci}{f}} \dicPos{n}[2] \dicFlx{(‑s)}[2] \dicDirectTranslationCS{zbourání, stržení, demolice} \dicExampleIS{niðurrif húsa} \dicExampleCS{demolice domů}
\dicEntry[niðurröðun] \dicTerm{niður··röð|un} \dicIPA{{n}{\textsci}{\textlengthmark}{ð}{\textscy}{r}{\oe}{ð}{\textscy}{\textsubring{n}}} \dicPos{f}[7] \dicFlx{(‑unar)}[12] \dicDirectTranslationCS{uspořádání, seřazení} \dicExampleIS{niðurröðun efnisins í ritgerð} \dicExampleCS{uspořádání látky v~písemné práci}
\dicEntry[niðursauð] \dicTerm{niður··sauð} \dicIPA{{n}{\textsci}{\textlengthmark}{ð}{\textscy}{\textsubring{r}}{s}{\oe i}{\texttheta}} \dicPos{v} \dicFlx{ind pf sg 1 pers} \dicLink{niðursjóða}
\dicEntry[niðursjóða] \dicTerm{niður··|sjóða} \dicIPA{{n}{\textsci}{\textlengthmark}{ð}{\textscy}{\textsubring{r}}{s}{j}{ou}{ð}{a}} \dicPos{v}[6] \dicFlx{(‑sýð, ‑sauð, ‑suðum, ‑syði, ‑soðið)}[104] \dicFlx{acc} \dicDirectTranslationCS{zavařit, zavařovat, (za)konzervovat}
\dicEntry[niðurskurður] \dicTerm{niður··skurð|ur} \dicIPA{{n}{\textsci}{\textlengthmark}{ð}{\textscy}{\textsubring{r}}{s}{\r{g}}{\textscy}{r}{ð}{\textscy}{\textsubring{r}}} \dicPos{m}[10] \dicFlx{(‑ar)}[7] \dicDirectTranslationCS{snížení, omezení (nákladů ap.)} \dicExampleIS{niðurskurður á framlagi til sjóðsins} \dicExampleCS{snížení příspěvku do nadace}
\dicEntry[niðursoðið] \dicTerm{niður··soðið} \dicIPA{{n}{\textsci}{\textlengthmark}{ð}{\textscy}{\textsubring{r}}{s}{\textopeno}{ð}{\textsci}{\texttheta}} \dicPos{v} \dicFlx{supin} \dicLink{niðursjóða}
\dicEntry[niðursoðinn] \dicTerm{niður··soðinn} \dicIPA{{n}{\textsci}{\textlengthmark}{ð}{\textscy}{\textsubring{r}}{s}{\textopeno}{ð}{\textsci}{\textsubring{n}}} \dicPos{adj}[6]\dicFlx{}[-6] \dicDirectTranslationCS{zavařený, zavařovaný, konzervovaný} \dicExampleIS{niðursoðnir ávextir} \dicExampleCS{zavařované ovoce}
\dicEntry[niðursokkinn] \dicTerm{niður··sokkinn} \dicIPA{{n}{\textsci}{\textlengthmark}{ð}{\textscy}{\textsubring{r}}{s}{\textopeno}{h}{\r{\textObardotlessj}}{\textsci}{\textsubring{n}}} \dicPos{adj}[6]\dicFlx{}[-2] \dicDirectTranslationCS{pohroužený, ponořený, zabraný (do práce ap.)} \dicExampleIS{niðursokkinn í e‑ð} \dicExampleCS{pohroužený do (čeho)}
\dicEntry[niðurstaða] \dicTerm{niður··|staða} \dicsymFrequent\  \dicIPA{{n}{\textsci}{\textlengthmark}{ð}{\textscy}{\textsubring{r}}{s}{\textsubring{d}}{a}{ð}{a}} \dicPos{f}[1] \dicFlx{(‑stöðu, ‑stöður)}[20] \dicDirectTranslationCS{závěr, rozhodnutí, verdikt, výrok} \dicExampleIS{niðurstaða Hæstaréttar} \dicExampleCS{rozhodnutí Nejvyššího soudu}
\dicEntry[niðursuða] \dicTerm{niður··suð|a} \dicIPA{{n}{\textsci}{\textlengthmark}{ð}{\textscy}{\textsubring{r}}{s}{\textscy}{ð}{a}} \dicPos{f}[1] \dicFlx{(‑u)}[5] \dicDirectTranslationCS{zavařování, konzervování, konzervace}
\dicEntry[niðursuðudós] \dicTerm{niður·suðu··dós} \dicIPA{{n}{\textsci}{\textlengthmark}{ð}{\textscy}{\textsubring{r}}{s}{\textscy}{ð}{\textscy}{\textsubring{d}}{ou}{s}} \dicPos{f}[7] \dicFlx{(‑ar, ‑ir)}[1] \dicDirectTranslationCS{zavařovací sklenice, zavařovačka}
\dicEntry[niðursuðuiðnaður] \dicTerm{niður·suðu··iðn·að|ur} \dicIPA{{n}\-{\textsci}\-{\textlengthmark}\-{ð}\-{\textscy}\-{\textsubring{r}}\-{s}\-{\textscy}\-{ð}\-{\textscy}\-{\textsci}\-{ð}\-{n}\-{a}\-{ð}\-{\textscy}\-{\textsubring{r}}\-} \dicPos{m}[10] \dicFlx{(‑ar)}[9] \dicDirectTranslationCS{konzervárenský průmysl}
\dicEntry[niðursuðum] \dicTerm{niður··suðum} \dicIPA{{n}{\textsci}{\textlengthmark}{ð}{\textscy}{\textsubring{r}}{s}{\textscy}{ð}{\textscy}{\textsubring{m}}} \dicPos{v} \dicFlx{ind pf pl 1 pers} \dicLink{niðursjóða}
\dicEntry[niðursyði] \dicTerm{niður··syði} \dicIPA{{n}{\textsci}{\textlengthmark}{ð}{\textscy}{\textsubring{r}}{s}{\textsci}{ð}{\textsci}} \dicPos{v} \dicFlx{con pf sg 1 pers} \dicLink{niðursjóða}
\dicEntry[niðursýð] \dicTerm{niður··sýð} \dicIPA{{n}{\textsci}{\textlengthmark}{ð}{\textscy}{\textsubring{r}}{s}{i}{\texttheta}} \dicPos{v} \dicFlx{ind praes sg 1 pers} \dicLink{niðursjóða}
\dicEntry[Niflheimur] \dicTerm{Nifl··heim|ur} \dicIPA{{n}{\textsci}{\textsubring{b}}{\textsubring{l}}{h}{ei}{m}{\textscy}{\textsubring{r}}} \dicPos{m}[6] \dicFlx{(‑s)}[3] \dicFieldCat{myt.} \dicDirectTranslationCS{Niflheim} \dicIndirectTranslationCS{(říše chladu, ledu, mlhy a~temnoty)}
\dicEntry[nifteind] \dicTerm{nift··eind} \dicIPA{{n}{\textsci}{f}{\textsubring{d}}{ei}{n}{\textsubring{d}}} \dicPos{f}[7] \dicFlx{(‑ar, ‑ir)}[1] \dicFieldCat{fyz.} \dicDirectTranslationCS{neutron}
\dicEntry[nikka] \dicTerm{nikk|a\smash{\textsuperscript{1}}} \dicIPA{{n}{\textsci}{h}{\r{g}}{a}} \dicPos{f}[1] \dicFlx{(‑u, ‑ur)}[19] \dicSynonym{harmónika} \dicDirectTranslationCS{akordeon, (tahací) harmonika}
\dicEntry[nikka] \dicTerm{nikk|a\smash{\textsuperscript{2}}} \dicIPA{{n}{\textsci}{h}{\r{g}}{a}} \dicPos{v}[1] \dicFlx{(‑aði)}[1] \dicLangCat{hovor.} \dicDirectTranslationCS{(při)kývnout (hlavou ap.)}
\dicEntry[nikkel] \dicTerm{nikkel} \dicIPA{{n}{\textsci}{h}{\r{\textObardotlessj}}{\textepsilon}{\textsubring{l}}} \dicPos{n}[2] \dicFlx{(‑s)}[2] \dicFieldCat{chem.} \dicDirectTranslationCS{nikl} \textit{(l.~{\textLA{Ni, Niccolum}})}
\dicEntry[nirfill] \dicTerm{nirf|ill} \dicIPA{{n}{\textsci}{r}{v}{\textsci}{\textsubring{d}}{\textsubring{l}}} \dicPos{m}[6] \dicFlx{(‑ils, ‑lar)}[35] \dicDirectTranslationCS{lakomec, skrblík}
\dicEntry[nitur] \dicTerm{nitur} \dicIPA{{n}{\textsci}{\textlengthmark}{\textsubring{d}}{\textscy}{\textsubring{r}}} \dicPos{n}[2] \dicFlx{(‑s)}[28] \dicFieldCat{chem.} \dicDirectTranslationCS{dusík} \textit{(l.~{\textLA{N, Nitrogenium}})}
\dicEntry[nía] \dicTerm{ní|a} \dicIPA{{n}{i}{\textlengthmark}{j}{a}} \dicPos{f}[1] \dicFlx{(‑u, ‑ur)}[7] \textbf{1.} \dicDirectTranslationCS{devítka} \dicIndirectTranslationCS{(v~kartách)} \dicExampleIS{tígulnía} \dicExampleCS{kárová devítka}  \textbf{2.} \dicDirectTranslationCS{devítka (autobus číslo 9 ap.)}
\dicEntry[níð] \dicTerm{níð} \dicIPA{{n}{i}{\textlengthmark}{\texttheta}} \dicPos{n}[2] \dicFlx{(‑s, ‑)}[5] \textbf{1.} \dicSynonym{óhróður} \dicDirectTranslationCS{hanění, pomluva};  \dicPhraseIS{kveða\,/\addthin yrkja\,/\addthin skrifa níð um e‑n} \dicDirectTranslationCS{pomluvit (koho)}  \textbf{2.} \dicSynonym{þorpari} \dicDirectTranslationCS{ničema, lotr(yně)}  \textbf{3.} \dicSynonym{strit} \dicDirectTranslationCS{dřina, lopota}
\dicEntry[níða] \dicTerm{ní|ða} \dicIPA{{n}{i}{\textlengthmark}{ð}{a}} \dicPos{v}[2] \dicFlx{(‑ddi, ‑tt)}[167] \dicFlx{acc} \textbf{1.} \dicDirectTranslationCS{hanobit, pomluvit, pomlouvat, očernit, očerňovat} \dicExampleIS{níða e‑n niður} \dicExampleCS{očernit (koho)}  \textbf{2.} \dicSynonym{slíta} \dicDirectTranslationCS{poškodit, uškodit} \dicExampleIS{Vond meðferð níðir netin.} \dicExampleCS{Špatné zacházení poškozuje síť.}  \textbf{3.} \dicDirectTranslationCS{(nadměrně) využívat, zneužívat} \dicExampleIS{níða húsdýr} \dicExampleCS{zneužívat domácí zvířata};  \dicIdiom{níðast}[á]{ \dicPhraseIS{níðast á e‑m}} \dicFlx{refl} \dicDirectTranslationCS{využívat (koho), zneužívat (koho) (čí pohostinnosti ap.)} \dicExampleIS{níðast á gestrisni hennar} \dicExampleCS{zneužívat její pohostinnosti}
\dicEntry[níðingslegur] \dicTerm{níðings··legur} \dicIPA{{n}{i}{\textlengthmark}{ð}{i}{\ng}{s}{l}{\textepsilon}{\textbabygamma}{\textscy}{\textsubring{r}}} \dicPos{adj}[1]\dicFlx{}[-8] \dicSynonym{svívirðilegur} \dicDirectTranslationCS{hanebný, podlý} \dicExampleIS{níðingsleg athöfn} \dicExampleCS{podlý čin}
\dicEntry[níðingsverk] \dicTerm{níðings··verk} \dicIPA{{n}{i}{\textlengthmark}{ð}{i}{\ng}{s}{v}{\textepsilon}{\textsubring{r}}{\r{g}}} \dicPos{n}[2] \dicFlx{(‑s, ‑)}[5] \dicDirectTranslationCS{podlý čin, podlost, hanebnost}
\dicEntry[níðingur] \dicTerm{níð··ing|ur} \dicIPA{{n}{i}{\textlengthmark}{ð}{i}{\ng}{\r{g}}{\textscy}{\textsubring{r}}} \dicPos{m}[6] \dicFlx{(‑s, ‑ar)}[8] \textbf{1.} \dicSynonym*{ódrengur} \dicDirectTranslationCS{lump, darebák, darebačka, ničema}  \textbf{2.} \dicSynonym{þrælmenni} \dicDirectTranslationCS{surovec, surovkyně, tyran(ka)} \dicExampleIS{hestaníðingur} \dicExampleCS{člověk, který je surový ke koním}  \textbf{3.} \dicSynonym{nirfill} \dicDirectTranslationCS{lakomec, skrblík}
\dicEntry[níðvísa] \dicTerm{níð··vís|a} \dicIPA{{n}{i}{ð}{v}{i}{s}{a}} \dicPos{f}[1] \dicFlx{(‑u, ‑ur)}[13] \dicDirectTranslationCS{hanlivá báseň, hanlivý verš}
\dicEntry[nífaldur] \dicTerm{ní··|faldur} \dicIPA{{n}{i}{\textlengthmark}{f}{a}{l}{\textsubring{d}}{\textscy}{\textsubring{r}}} \dicPos{adj}[2] \dicFlx{(f ‑föld)}[16] \textbf{1.} \dicDirectTranslationCS{devítinásobný}  \textbf{2.} \dicDirectTranslationCS{devítidílný}
\dicEntry[Níger] \dicTerm{Níger} \dicIPA{{n}{i}{\textlengthmark}{\textbabygamma}{\textepsilon}{\textsubring{r}}} \dicPos{n}[4] \dicFlx{indecl}[2] \textbf{1.} \dicFieldCat{geog.} \dicDirectTranslationCS{Niger} \dicIndirectTranslationCS{(stát v~Africe)}  \textbf{2.} \dicFieldCat{geog.} \dicSynonym{Nígerfljót} \dicDirectTranslationCS{Niger} \dicIndirectTranslationCS{(řeka v~západní Africe)}
\dicEntry[Nígerfljót] \dicTerm{Níger··fljót} \dicIPA{{n}{i}{\textlengthmark}{\textbabygamma}{\textepsilon}{\textsubring{r}}{f}{l}{j}{ou}{\textsubring{d}}} \dicPos{n}[2] \dicFlx{(‑s)}[4] \dicFieldCat{geog.} \dicDirectTranslationCS{Niger} \dicIndirectTranslationCS{(řeka v~západní Africe)}
\dicEntry[Nígería] \dicTerm{Nígerí|a} \dicIPA{{n}{i}{\textlengthmark}{\textbabygamma}{\textepsilon}{r}{i}{j}{a}} \dicPos{f}[1] \dicFlx{(‑u)}[6] \dicFieldCat{geog.} \dicDirectTranslationCS{Nigérie}
\dicEntry[nígerískur] \dicTerm{nígerískur} \dicIPA{{n}{i}{\textlengthmark}{\textbabygamma}{\textepsilon}{r}{i}{s}{\r{g}}{\textscy}{\textsubring{r}}} \dicPos{adj}[1]\dicFlx{}[-6] \dicDirectTranslationCS{nigerijský}
\dicEntry[Nígeríumaður] \dicTerm{Nígeríu··|maður} \dicIPA{{n}{i}{\textlengthmark}{\textbabygamma}{\textepsilon}{r}{i}{j}{\textscy}{m}{a}{ð}{\textscy}{\textsubring{r}}} \dicPos{m}[13] \dicFlx{(‑manns, ‑menn)}[2] \dicDirectTranslationCS{Nigerijec, Nigerijka}
\dicEntry[Nígermaður] \dicTerm{Níger··|maður} \dicIPA{{n}{i}{\textlengthmark}{\textbabygamma}{\textepsilon}{r}{m}{a}{ð}{\textscy}{\textsubring{r}}} \dicPos{m}[13] \dicFlx{(‑manns, ‑menn)}[2] \dicDirectTranslationCS{Nigeřan(ka)}
\dicEntry[nígerskur] \dicTerm{nígerskur} \dicIPA{{n}{i}{\textlengthmark}{\textbabygamma}{\textepsilon}{\textsubring{r}}{s}{\r{g}}{\textscy}{\textsubring{r}}} \dicPos{adj}[1]\dicFlx{}[-6] \dicDirectTranslationCS{nigerský}
\dicEntry[níhílismi] \dicTerm{níhíl··ism|i} \dicIPA{{n}{i}{\textlengthmark}{h}{i}{l}{\textsci}{s}{m}{\textsci}} \dicPos{m}[1] \dicFlx{(‑a)}[3] \dicDirectTranslationCS{nihilismus}
\dicEntry[níhyrningur] \dicTerm{ní··hyrn·ing|ur} \dicIPA{{n}{i}{\textlengthmark}{h}{\textsci}{r}{\textsubring{d}}{n}{i}{\ng}{\r{g}}{\textscy}{\textsubring{r}}} \dicPos{m}[6] \dicFlx{(‑s, ‑ar)}[8] \dicFieldCat{mat.} \dicDirectTranslationCS{devítiúhelník}
\dicEntry[níkaragskur] \dicTerm{níkaragskur} \dicIPA{{n}{i}{\textlengthmark}{\r{g}}{a}{r}{a}{x}{s}{\r{g}}{\textscy}{\textsubring{r}}} \dicPos{adj}[1] \dicFlx{(f níkarögsk)}[3] \dicDirectTranslationCS{nikaragujský}
\dicEntry[Níkaragva] \dicTerm{Níkaragva} \dicIPA{{n}{i}{\textlengthmark}{\r{g}}{a}{r}{a}{\textbabygamma}{v}{a}} \dicPos{n}[4] \dicFlx{indecl}[2] \dicFieldCat{geog.} \dicDirectTranslationCS{Nikaragua}
\dicEntry[Níkaragvamaður] \dicTerm{Níkaragva··|maður} \dicIPA{{n}{i}{\textlengthmark}{\r{g}}{a}{r}{a}{\textbabygamma}{v}{a}{m}{a}{ð}{\textscy}{\textsubring{r}}} \dicPos{m}[13] \dicFlx{(‑manns, ‑menn)}[2] \dicDirectTranslationCS{Nikaragujec, Nikaragujka}
\dicEntry[Níl] \dicTerm{Níl} \dicIPA{{n}{i}{\textlengthmark}{\textsubring{l}}} \dicPos{f}[4] \dicFlx{(‑ar)}[4] \dicFieldCat{geog.} \dicDirectTranslationCS{Nil} \dicIndirectTranslationCS{(řeka v~Africe)}
\dicEntry[nílhestur] \dicTerm{níl··hest|ur} \dicIPA{{n}{i}{\textlengthmark}{\textsubring{l}}{h}{\textepsilon}{s}{\textsubring{d}}{\textscy}{\textsubring{r}}} \dicPos{m}[6] \dicFlx{(‑s, ‑ar)}[4] \dicFieldCat{zool.} \dicDirectTranslationCS{hroch, hroch obojživelný} \textit{(l.~{\textLA{Hippopotamus amphibius}})}  \dicsymPhoto\ 
\dicFigure{ds_image_nilhestur_0_1.jpg}{Nílhestur}{Nílhestur - cloudzilla, CC BY 2.0}
\dicEntry[níóbín] \dicTerm{níóbín} \dicIPA{{n}{i}{j}{ou}{\textsubring{b}}{i}{\textsubring{n}}} \dicPos{n}[2] \dicFlx{(‑s)}[2] \dicFieldCat{chem.} \dicDirectTranslationCS{niob} \textit{(l.~{\textLA{Nb, Niobium}})}
\dicEntry[níræðisaldur] \dicTerm{ní·ræðis··ald|ur} \dicIPA{{n}{i}{\textlengthmark}{r}{a}{i}{ð}{\textsci}{s}{a}{l}{\textsubring{d}}{\textscy}{\textsubring{r}}} \dicPos{m}[5] \dicFlx{(‑urs, ‑rar)}[3] \dicDirectTranslationCS{věk po osmdesátce, věk mezi 80. a~89. rokem}
\dicEntry[níræður] \dicTerm{ní··ræður} \dicIPA{{n}{i}{\textlengthmark}{r}{a}{i}{ð}{\textscy}{\textsubring{r}}} \dicPos{adj}[2]\dicFlx{}[-10] \textbf{1.} \dicDirectTranslationCS{devadesátiletý} \dicExampleIS{níræður að aldri} \dicExampleCS{devadesátiletý}  \textbf{2.} \dicDirectTranslationCS{měřící devadesát sáhů}
\dicEntry[níska] \dicTerm{nísk|a} \dicIPA{{n}{i}{s}{\r{g}}{a}} \dicPos{f}[1] \dicFlx{(‑u)}[5] \dicDirectTranslationCS{lakota, lakomost}
\dicEntry[nískur] \dicTerm{nískur} \dicIPA{{n}{i}{s}{\r{g}}{\textscy}{\textsubring{r}}} \dicPos{adj}[1]\dicFlx{}[-1] \dicSynonym{sínkur} \dicDirectTranslationCS{lakomý, lakotný, chamtivý}
\dicEntry[nísta] \dicTerm{níst|a} \dicIPA{{n}{i}{s}{\textsubring{d}}{a}} \dicPos{v}[2] \dicFlx{(‑i, ‑)}[12] \textbf{1.} \dicPhraseIS{nísta tönnum} \dicDirectTranslationCS{skřípat zuby}  \textbf{2.} \dicSynonym*{fara í gegnum} \dicDirectTranslationCS{projít skrz (o~větru, chladu ap.)} \dicExampleIS{Kuldinn nísti mig í gegn.} \dicExampleCS{Promrznul jsem na kost.}
\dicEntry[nístandi] \dicTerm{níst··andi} \dicIPA{{n}{i}{s}{\textsubring{d}}{a}{n}{\textsubring{d}}{\textsci}} \dicPos{adj}[13] \dicFlx{indecl}[1] \dicDirectTranslationCS{bodavý, štiplavý} \dicExampleIS{nístandi kuldi} \dicExampleCS{štiplavý mráz}
\dicEntry[nístingskuldi] \dicTerm{nístings··kuld|i} \dicIPA{{n}{i}{s}{\textsubring{d}}{i}{\ng}{s}{k\smash{\textsuperscript{h}}}{\textscy}{l}{\textsubring{d}}{\textsci}} \dicPos{m}[1] \dicFlx{(‑a, ‑ar)}[1] \dicDirectTranslationCS{štiplavý\,/\addthin ostrý mráz}
\dicEntry[nítján] \dicTerm{ní··tján} \dicIPA{{n}{i}{\textlengthmark}{\textsubring{d}}{j}{au}{\textsubring{n}}} \dicPos{num} \dicFlx{indecl} \dicDirectTranslationCS{devatenáct}
\dicEntry[nítjándi] \dicTerm{ní··tjándi} \dicIPA{{n}{i}{\textlengthmark}{\textsubring{d}}{j}{au}{n}{\textsubring{d}}{\textsci}} \dicPos{num}[5] \dicFlx{ord} \dicDirectTranslationCS{devatenáctý}
\dicEntry[nítugasti] \dicTerm{ní·tug··asti} \dicIPA{{n}{i}{\textlengthmark}{\textsubring{d}}{\textscy}{\textbabygamma}{a}{s}{\textsubring{d}}{\textsci}} \dicPos{num}[6] \dicFlx{ord} \dicDirectTranslationCS{devadesátý}
\dicEntry[nítugfaldur] \dicTerm{ní·tug··|faldur} \dicIPA{{n}{i}{\textlengthmark}{t\smash{\textsuperscript{h}}}{\textscy}{x}{f}{a}{l}{\textsubring{d}}{\textscy}{\textsubring{r}}} \dicPos{adj}[2] \dicFlx{(f ‑föld)}[16] \textbf{1.} \dicDirectTranslationCS{devadesátinásobný}  \textbf{2.} \dicDirectTranslationCS{devadesátidílný}
\dicEntry[níu] \dicTerm{níu} \dicsymFrequent\  \dicIPA{{n}{i}{\textlengthmark}{j}{\textscy}} \dicPos{num} \dicFlx{indecl} \dicDirectTranslationCS{devět} \dicExampleIS{þýðingar á níu útlend mál} \dicExampleCS{překlady do devíti cizích jazyků}
\dicEntry[níundi] \dicTerm{níundi} \dicIPA{{n}{i}{\textlengthmark}{j}{\textscy}{n}{\textsubring{d}}{\textsci}} \dicPos{num}[5] \dicFlx{ord} \dicDirectTranslationCS{devátý}
\dicEntry[níutíu] \dicTerm{níu··tíu} \dicIPA{{n}{i}{\textlengthmark}{j}{\textscy}{\textsubring{d}}{i}{j}{\textscy}} \dicPos{num} \dicFlx{indecl} \dicDirectTranslationCS{devadesát}
\dicEntry[níutíukall] \dicTerm{níu·tíu··kall} \dicIPA{{n}{i}{\textlengthmark}{j}{\textscy}{t\smash{\textsuperscript{h}}}{i}{j}{\textscy}{k\smash{\textsuperscript{h}}}{a}{\textsubring{d}}{\textsubring{l}}} \dicPos{m}[4] \dicFlx{(‑s, ‑ar)}[4] \dicLangCat{hovor.} \dicDirectTranslationCS{devadesát (korun)} \dicIndirectTranslationCS{(částka)}
\dicEntry[njóli] \dicTerm{njól|i} \dicIPA{{n}{j}{ou}{\textlengthmark}{l}{\textsci}} \dicPos{m}[1] \dicFlx{(‑a, ‑ar)}[1] \dicFieldCat{bot.} \dicDirectTranslationCS{šťovík, šťovík dlouholistý} \textit{(l.~{\textLA{Rumex longifolius}})}  \dicsymPhoto\ 
\dicFigure{ds_image_njoli_0_1.jpg}{Njóli}{Njóli - Jutta234, CC BY-SA 3.0}
\dicEntry[njósn] \dicTerm{njósn} \dicIPA{{n}{j}{ou}{s}{\textsubring{d}}{\textsubring{n}}} \dicPos{f}[7] \dicFlx{(‑ar, ‑ir)}[1] \textbf{1.} \dicSynonym{frétt} \dicDirectTranslationCS{tajná\,/\addthin špionážní informace}  \textbf{2.} \dicPhraseIS{njósnir} \dicFlx{pl} \dicDirectTranslationCS{špionáž, vyzvědačství}
\dicEntry[njósna] \dicTerm{njósn|a} \dicIPA{{n}{j}{ou}{s}{\textsubring{d}}{n}{a}} \dicPos{v}[1] \dicFlx{(‑aði)}[44] \dicDirectTranslationCS{špehovat, vyzvídat, provádět špionáž} \dicExampleIS{njósna um e‑ð} \dicExampleCS{vyzvídat (co)}
\dicEntry[njósnari] \dicTerm{njósn··ar|i} \dicIPA{{n}{j}{ou}{s}{\textsubring{d}}{n}{a}{r}{\textsci}} \dicPos{m}[1] \dicFlx{(‑a, ‑ar)}[13] \dicDirectTranslationCS{špeh, špión(ka), zvěd, tajný agent, tajná agentka}
\dicEntry[njóta] \dicTerm{njóta} \dicsymFrequent\  \dicIPA{{n}{j}{ou}{\textlengthmark}{\textsubring{d}}{a}} \dicPos{v}[6] \dicFlx{(nýt, naut, nutum, nyti, notið)}[44] \dicFlx{gen} \textbf{1.} \dicDirectTranslationCS{užít si, užívat si, těšit se} \dicExampleIS{njóta ferðarinnar} \dicExampleCS{užívat si cestu}  \textbf{2.} \dicPhraseIS{njóta sín} \dicDirectTranslationCS{dařit se, prospívat, přijít si na své} \dicExampleIS{njóta sín vel í hópnum} \dicExampleCS{cítit se dobře ve skupině}  \textbf{3.} \dicPhraseIS{e‑s nýtur við} \dicFlx{impers} \dicDirectTranslationCS{(kdo) je zde, (kdo) zde pracuje}
\dicEntry[nk.] \dicTerm{nk.} \dicPos{zkr} \dicPhraseIS{næstkomandi} \dicDirectTranslationCS{následující, příští}
\dicEntry[nlt.] \dicTerm{nlt.} \dicPos{zkr} \dicPhraseIS{núliðin tíð} \dicFieldCat{jaz.} \dicDirectTranslationCS{předpřítomný čas}
\dicEntry[nm.] \dicTerm{nm.} \dicPos{zkr} \dicPhraseIS{neðanmáls} \dicFlx{adv} \dicDirectTranslationCS{v~poznámce pod čarou}
\dicEntry[no.] \dicTerm{no.} \dicPos{zkr} \dicPhraseIS{nafnorð} \dicFieldCat{jaz.} \dicDirectTranslationCS{podstatné jméno}
\dicEntry[nokkuð] \dicTerm{nokkuð} \dicIPA{{n}{\textopeno}{h}{\r{g}}{\textscy}{\texttheta}} \dicPos{adv} \dicSynonym{dálítið} \dicDirectTranslationCS{trochu, docela, poněkud}
\dicEntry[nokkur] \dicTerm{nokkur} \dicsymFrequent\  \dicIPA{{n}{\textopeno}{h}{\r{g}}{\textscy}{\textsubring{r}}} \dicPos{pron}[7] \dicFlx{indef} \textbf{1.} \dicDirectTranslationCS{několik, pár, trochu} \dicIndirectTranslationCS{(používané jako podst. jméno)} \dicExampleIS{nokkrir af okkur} \dicExampleCS{někteří z~nás}  \textbf{2.} \dicSynonym{einhver} \dicDirectTranslationCS{nějaký, některý} \dicIndirectTranslationCS{(používané jako příd. jméno v~záporných větách)} \dicExampleIS{Hvergi var nokkur maður.} \dicExampleCS{Nikde nikdo nebyl.}  \textbf{3.} \dicDirectTranslationCS{jeden} \dicIndirectTranslationCS{(v~označení něčeho, co nemůžeme blíže určit)} \dicExampleIS{maður nokkur} \dicExampleCS{jeden člověk}
\dicEntry[nokkurskonar] \dicTerm{nokkurs··konar}\dicTerm{, nokkurs konar} \dicIPA{{n}\-{\textopeno}\-{h}\-{\r{g}}\-{\textscy}\-{\textsubring{r}}\-{s}\-{k\smash{\textsuperscript{h}}}\-{\textopeno}\-{n}\-{a}\-{\textsubring{r}}\-} \dicPos{adj}[13] \dicFlx{indecl}[2] \dicDirectTranslationCS{jakýsi, určitý}
\dicEntry[norðan] \dicTerm{norðan} \dicsymFrequent\  \dicIPA{{n}{\textopeno}{r}{ð}{a}{\textsubring{n}}} \dicPos{prep\,/\addthin adv} \dicFlx{adv} \dicSynonym*{úr norðurátt} \dicDirectTranslationCS{ze severu};  \dicPhraseIS{norðan e‑s} \dicFlx{prep} \dicDirectTranslationCS{severně od (čeho), na severní straně (čeho)} \dicExampleIS{norðan fjallsins} \dicExampleCS{severně od hory};  \dicIdiom{að}[norðan]{ \dicPhraseIS{að norðan}} \dicFlx{adv} {\textbf{a.}} \dicDirectTranslationCS{ze\,/\addthin od severu};  {\textbf{b.}} \dicDirectTranslationCS{po\,/\addthin na severní straně};  \dicIdiom{norðan}[að]{ \dicPhraseIS{norðan að}} \dicFlx{adv} \dicDirectTranslationCS{ze severu};  \dicIdiom{norðan}[á]{ \dicPhraseIS{norðan á e‑u}} \dicFlx{prep} \dicDirectTranslationCS{na severní straně (čeho)};  \dicIdiom{norðan}[frá]{ \dicPhraseIS{norðan frá}} \dicFlx{adv} \dicDirectTranslationCS{ze severu, ze severní strany};  \dicIdiom{norðan}[í móti]{ \dicPhraseIS{norðan í móti}} \dicFlx{adv} \dicDirectTranslationCS{směrem na sever};  \dicIdiom{norðan}[megin]{ \dicPhraseIS{norðan megin}} \dicFlx{adv} \dicDirectTranslationCS{na severní straně, v~severní části};  \dicIdiom{norðan}[til]{ \dicPhraseIS{norðan til}} \dicFlx{adv} \dicDirectTranslationCS{v~severní části};  \dicIdiom{norðan}[undir]{ \dicPhraseIS{norðan undir e‑u}} \dicFlx{prep} \dicDirectTranslationCS{severně pod (čím), ze severní strany (čeho)};  \dicIdiom{norðan}[við]{ \dicPhraseIS{norðan við e‑ð}} \dicFlx{prep} \dicDirectTranslationCS{na severní straně (čeho), severně od (čeho)};  \dicIdiom{norðan}[yfir]{ \dicPhraseIS{norðan yfir e‑ð}} \dicFlx{prep} \dicDirectTranslationCS{ze severu přes (co)};  \dicIdiom{norðan}{ \dicPhraseIS{norðan úr\,/\addthin af landi}} \dicFlx{adv} \dicDirectTranslationCS{ze severní části země}
\dicEntry[norðanátt] \dicTerm{norðan··átt} \dicIPA{{n}{\textopeno}{r}{ð}{a}{n}{au}{h}{\textsubring{d}}} \dicPos{f}[7] \dicFlx{(‑ar, ‑ir)}[1] \dicDirectTranslationCS{severní vítr, severák}
\dicEntry[norðanverður] \dicTerm{norðan··verður} \dicIPA{{n}{\textopeno}{r}{ð}{a}{n}{v}{\textepsilon}{r}{ð}{\textscy}{\textsubring{r}}} \dicPos{adj}[2]\dicFlx{}[-4] \dicDirectTranslationCS{severně položený} \dicExampleIS{við norðanverðan fjörðinn} \dicExampleCS{na severní straně zálivu};  \dicPhraseIS{að norðanverðu} \dicFlx{adv} \dicDirectTranslationCS{na severní straně}
\dicEntry[norðanvindur] \dicTerm{norðan··vind|ur} \dicIPA{{n}{\textopeno}{r}{ð}{a}{n}{v}{\textsci}{n}{\textsubring{d}}{\textscy}{\textsubring{r}}} \dicPos{m}[6] \dicFlx{(‑s, ‑ar)}[4] \dicDirectTranslationCS{severní vítr, severák}
\dicEntry[norðar] \dicTerm{norðar} \dicIPA{{n}{\textopeno}{r}{ð}{a}{\textsubring{r}}} \dicPos{adv} \dicFlx{comp} \dicLink{norður\smash{\textsuperscript{2}}}
\dicEntry[norðaustur] \dicTerm{norð··austur\smash{\textsuperscript{1}}} \dicIPA{{n}{\textopeno}{r}{ð}{\oe i}{s}{\textsubring{d}}{\textscy}{\textsubring{r}}} \dicPos{n}[2] \dicFlx{(‑s)}[28] \dicDirectTranslationCS{severovýchod}
\dicEntry[norðaustur] \dicTerm{norð··aust|ur\smash{\textsuperscript{2}}} \dicIPA{{n}{\textopeno}{r}{ð}{\oe i}{s}{\textsubring{d}}{\textscy}{\textsubring{r}}} \dicPos{adv} \dicFlx{(comp ‑ar, sup ‑ast)} \dicDirectTranslationCS{severovýchodně, na severovýchod}
\dicEntry[norðlenskur] \dicTerm{norð··lenskur} \dicIPA{{n}{\textopeno}{r}{ð}{l}{\textepsilon}{n}{s}{\r{g}}{\textscy}{\textsubring{r}}} \dicPos{adj}[1]\dicFlx{}[-1] \dicDirectTranslationCS{(jsoucí) ze Severního Islandu}
\dicEntry[norðlægur] \dicTerm{norð··lægur} \dicIPA{{n}{\textopeno}{r}{ð}{l}{a}{i}{\textbabygamma}{\textscy}{\textsubring{r}}} \dicPos{adj}[1]\dicFlx{}[-1] \textbf{1.} \dicDirectTranslationCS{severní (ležící na severu)}  \textbf{2.} \dicDirectTranslationCS{severní (vítr ap.)}
\dicEntry[Norðmaður] \dicTerm{Norð··|maður} \dicsymFrequent\  \dicIPA{{n}{\textopeno}{r}{ð}{m}{a}{ð}{\textscy}{\textsubring{r}}} \dicPos{m}[13] \dicFlx{(‑manns, ‑menn)}[2] \dicDirectTranslationCS{Nor(ka)} \dicExampleIS{tala einsog innfæddur Norðmaður} \dicExampleCS{mluvit jako rodilý Nor}
\dicEntry[norður] \dicTerm{norður\smash{\textsuperscript{1}}} \dicsymFrequent\  \dicIPA{{n}{\textopeno}{r}{ð}{\textscy}{\textsubring{r}}} \dicPos{n}[2] \dicFlx{(‑s)}[28] \dicDirectTranslationCS{sever} \dicExampleIS{leyndardómsfull eyja í norðri} \dicExampleCS{tajuplný ostrov na severu}
\dicEntry[norður] \dicTerm{norður\smash{\textsuperscript{2}}} \dicsymFrequent\  \dicIPA{{n}{\textopeno}{r}{ð}{\textscy}{\textsubring{r}}} \dicPos{adv} \dicFlx{(comp norðar, sup nyrst)} \textbf{1.} \dicSynonym*{í norðurátt} \dicDirectTranslationCS{severně, na sever} \dicExampleIS{eyja langt norður í Atlantshafi} \dicExampleCS{ostrov daleko na severu v~Atlantiku}  \textbf{2.} \dicSynonym*{um Norðurland} \dicDirectTranslationCS{na sever (Islandu ap.), do severní části země};  \dicIdiom{norður}[af]{ \dicPhraseIS{norður af e‑u}} \dicFlx{prep} \dicDirectTranslationCS{na severní straně (čeho), severně od (čeho)}; { \dicPhraseIS{norður af}} \dicFlx{adv} \dicDirectTranslationCS{severně, (směrem) na sever};  \dicIdiom{norður}[eftir]{ \dicPhraseIS{norður eftir}} \dicFlx{adv} \dicDirectTranslationCS{(směrem) k~severu, (směrem) na sever};  \dicIdiom{norður}[frá]{ \dicPhraseIS{norður frá}} \dicFlx{adv} \dicDirectTranslationCS{na severu, na sever};  \dicIdiom{norður}[fyrir]{ \dicPhraseIS{norður fyrir e‑ð}} \dicFlx{prep} \dicDirectTranslationCS{na sever od (čeho)};  \dicIdiom{norður}[með]{ \dicPhraseIS{norður með e‑u}} \dicFlx{prep} \dicDirectTranslationCS{na sever podél (čeho)};  \dicIdiom{norður}[um]{ \dicPhraseIS{norður um e‑ð}} \dicFlx{prep} \dicDirectTranslationCS{na sever (čeho)}; { \dicPhraseIS{norður um}} \dicFlx{adv} \dicDirectTranslationCS{na sever, severně};  \dicIdiom{norður}[undir]{ \dicPhraseIS{norður undir e‑ð}} \dicFlx{prep} \dicDirectTranslationCS{severně pod (čím)};  \dicIdiom{norður}[úr]{ \dicPhraseIS{norður úr e‑u}} \dicFlx{prep} \dicDirectTranslationCS{severně z~(čeho)};  \dicIdiom{norður}{ \dicPhraseIS{fara norður og niður}} \dicLangCat{přen.} \dicDirectTranslationCS{jít k~čertu}
\dicEntry[Norður-Íri] \dicTerm{Norður-Ír|i} \dicIPA{{n}{\textopeno}{r}{ð}{\textscy}{r}{i}{r}{\textsci}} \dicPos{m}[1] \dicFlx{(‑a, ‑ar)}[1] \dicDirectTranslationCS{Severoir(ka)}
\dicEntry[Norður-Írland] \dicTerm{Norður-Ír··land} \dicIPA{{n}{\textopeno}{r}{ð}{\textscy}{r}{i}{r}{l}{a}{n}{\textsubring{d}}} \dicPos{n}[2] \dicFlx{(‑s)}[4] \dicFieldCat{geog.} \dicDirectTranslationCS{Severní Irsko}
\dicEntry[Norður-Íshaf] \dicTerm{Norður-Ís··haf} \dicIPA{{n}{\textopeno}{r}{ð}{\textscy}{r}{i}{s}{h}{a}{f}} \dicPos{n}[2] \dicFlx{(‑s)}[4] \dicFieldCat{geog.} \dicDirectTranslationCS{Severní ledový oceán}
\dicEntry[Norður-Kórea] \dicTerm{Norður-Kóre|a} \dicIPA{{n}{\textopeno}{r}{ð}{\textscy}{\textsubring{r}}{k\smash{\textsuperscript{h}}}{ou}{r}{\textepsilon}{a}} \dicPos{f}[1] \dicFlx{(‑u)}[6] \dicFieldCat{geog.} \dicDirectTranslationCS{Severní Korea}
\dicEntry[Norður-Kóreumaður] \dicTerm{Norður-Kóreu··|maður} \dicIPA{{n}{\textopeno}{r}{ð}{\textscy}{\textsubring{r}}{k\smash{\textsuperscript{h}}}{ou}{r}{\textepsilon}{\textscy}{m}{a}{ð}{\textscy}{\textsubring{r}}} \dicPos{m}[13] \dicFlx{(‑manns, ‑menn)}[2] \dicDirectTranslationCS{Severokorejec, Severokorejka}
\dicEntry[norðurheimskaut] \dicTerm{norður··heim·skaut} \dicIPA{{n}{\textopeno}{r}{ð}{\textscy}{\textsubring{r}}{h}{ei}{m}{s}{\r{g}}{\oe i}{\textsubring{d}}} \dicPos{n}[2] \dicFlx{(‑s, ‑)}[5] \dicFieldCat{geog.} \dicDirectTranslationCS{severní pól}
\dicEntry[norðurheimskautsbaugur] \dicTerm{norður·heim·skauts··baug|ur} \dicIPA{{n}\-{\textopeno}\-{r}\-{ð}\-{\textscy}\-{\textsubring{r}}\-{h}\-{ei}\-{m}\-{s}\-{\r{g}}\-{\oe i}\-{\textsubring{d}}\-{s}\-{\textsubring{b}}\-{\oe i}\-{\textbabygamma}\-{\textscy}\-{\textsubring{r}}\-} \dicPos{m}[6] \dicFlx{(‑s, ‑ar)}[24] \dicFieldCat{geog.} \dicDirectTranslationCS{severní polární kruh}
\dicEntry[norðurhvel] \dicTerm{norður··hvel} \dicIPA{{n}{\textopeno}{r}{ð}{\textscy}{\textsubring{r}}{k\smash{\textsuperscript{h}}}{v}{\textepsilon}{\textsubring{l}}} \dicPos{n}[2] \dicFlx{(‑s, ‑)}[5] \dicFieldCat{geog.} \dicDirectTranslationCS{severní polokoule}
\dicEntry[norðurírskur] \dicTerm{norður··írskur} \dicIPA{{n}{\textopeno}{r}{ð}{\textscy}{r}{i}{\textsubring{r}}{s}{\r{g}}{\textscy}{\textsubring{r}}} \dicPos{adj}[1]\dicFlx{}[-6] \dicDirectTranslationCS{severoirský}
\dicEntry[norðurkóreskur] \dicTerm{norður··kóreskur} \dicIPA{{n}{\textopeno}{r}{ð}{\textscy}{\textsubring{r}}{k\smash{\textsuperscript{h}}}{ou}{r}{\textepsilon}{s}{\r{g}}{\textscy}{\textsubring{r}}} \dicPos{adj}[1]\dicFlx{}[-6] \dicDirectTranslationCS{severokorejský}
\dicEntry[Norðurland] \dicTerm{Norður··land} \dicsymFrequent\  \dicIPA{{n}{\textopeno}{r}{ð}{\textscy}{r}{l}{a}{n}{\textsubring{d}}} \dicPos{n}[2] \dicFlx{(‑s)}[4] \dicFieldCat{geog.} \dicDirectTranslationCS{Severní Island} \dicExampleIS{tíu daga ferð um Norðurland} \dicExampleCS{desetidenní cesta po Severním Islandu}
\dicEntry[norðurlandamál] \dicTerm{norður·landa··mál} \dicIPA{{n}{\textopeno}{r}{ð}{\textscy}{r}{l}{a}{n}{\textsubring{d}}{a}{m}{au}{\textsubring{l}}} \dicPos{n}[2] \dicFlx{(‑s, ‑)}[5] \dicDirectTranslationCS{severský jazyk}
\dicEntry[norðurljós] \dicTerm{norður··ljós} \dicIPA{{n}{\textopeno}{r}{ð}{\textscy}{r}{l}{j}{ou}{s}} \dicPos{n}[2] \dicFlx{pl}[1] \dicDirectTranslationCS{severní polární záře} \textit{(l.~{\textLA{Aurora borealis}})}
\dicEntry[Norðurlönd] \dicTerm{Norður··lönd} \dicIPA{{n}{\textopeno}{r}{ð}{\textscy}{r}{l}{\oe}{n}{\textsubring{d}}} \dicPos{n}[2] \dicFlx{pl}[9] \dicDirectTranslationCS{Severské země} \dicIndirectTranslationCS{(Island, Faerské ostrovy, Švédsko, Finsko, Dánsko, Norsko, Grónsko)}
\dicEntry[norðurpóll] \dicTerm{norður··pól|l} \dicIPA{{n}{\textopeno}{r}{ð}{\textscy}{\textsubring{r}}{p\smash{\textsuperscript{h}}}{ou}{\textsubring{d}}{\textsubring{l}}} \dicPos{m}[6] \dicFlx{(‑s, ‑ar)}[48] \dicFieldCat{geog.} \dicDirectTranslationCS{severní pól}
\dicEntry[Norðursjór] \dicTerm{Norður··|sjór} \dicIPA{{n}{\textopeno}{r}{ð}{\textscy}{\textsubring{r}}{s}{j}{ou}{\textsubring{r}}} \dicPos{m}[8] \dicFlx{(‑sjávar)}[6] \dicFieldCat{geog.} \dicDirectTranslationCS{Severní moře}
\dicEntry[norðurslóðir] \dicTerm{norður··slóðir} \dicIPA{{n}{\textopeno}{r}{ð}{\textscy}{\textsubring{r}}{s}{\textsubring{d}}{l}{ou}{ð}{\textsci}{\textsubring{r}}} \dicPos{f}[7] \dicFlx{pl}[2] \dicFieldCat{geog.} \dicDirectTranslationCS{Arktida}
\dicEntry[norðvestur] \dicTerm{norð··vestur\smash{\textsuperscript{1}}} \dicIPA{{n}{\textopeno}{r}{ð}{v}{\textepsilon}{s}{\textsubring{d}}{\textscy}{\textsubring{r}}} \dicPos{n}[2] \dicFlx{(‑s)}[28] \dicDirectTranslationCS{severozápad}
\dicEntry[norðvestur] \dicTerm{norð··vest|ur\smash{\textsuperscript{2}}} \dicIPA{{n}{\textopeno}{r}{ð}{v}{\textepsilon}{s}{\textsubring{d}}{\textscy}{\textsubring{r}}} \dicPos{adv} \dicFlx{(comp ‑ar, sup ‑ast)} \dicDirectTranslationCS{severozápadně, na severozápad}
\dicEntry[Noregur] \dicTerm{Noreg|ur} \dicsymFrequent\  \dicIPA{{n}{\textopeno}{\textlengthmark}{r}{\textepsilon}{\textbabygamma}{\textscy}{\textsubring{r}}} \dicPos{m}[6] \dicFlx{(‑s)}[3] \dicFieldCat{geog.} \dicDirectTranslationCS{Norsko} \dicExampleIS{sigla frá Noregi til Íslands} \dicExampleCS{plout z~Norska na Island}
\dicEntry[normal] \dicTerm{normal}\dicTerm{, normall} \dicIPA{{n}\-{\textopeno}\-{r}\-{m}\-{a}\-{\textsubring{l}}\-} \dicPos{adj}[13] \dicFlx{indecl}[1] \dicLangCat{hovor.} \dicDirectTranslationCS{normální}
\dicEntry[normall] \dicTerm{nor|mall} \dicIPA{{n}{\textopeno}{r}{m}{a}{\textsubring{d}}{\textsubring{l}}} \dicPos{adj}[8] \dicFlx{(f ‑möl)}[6] \dicLink{normal}
\dicEntry[Normandí] \dicTerm{Nor··mandí} \dicIPA{{n}{\textopeno}{r}{m}{a}{n}{\textsubring{d}}{i}} \dicPos{n}[2] \dicFlx{(‑s)}[32] \dicFieldCat{geog.} \dicDirectTranslationCS{Normandie}
\dicEntry[norn] \dicTerm{norn} \dicsymFrequent\  \dicIPA{{n}{\textopeno}{r}{\textsubring{d}}{\textsubring{n}}} \dicPos{f}[7] \dicFlx{(‑ar, ‑ir)}[1] \textbf{1.} \dicSynonym*{galdrakvendi} \dicDirectTranslationCS{čarodějnice, čarodějka, ježibaba} \dicExampleIS{vondar nornir} \dicExampleCS{zlé čarodějnice}  \textbf{2.} \dicFieldCat{myt.} \dicSynonym*{örlagagyðja} \dicDirectTranslationCS{norna, sudička}
\dicEntry[norpa] \dicTerm{norp|a} \dicIPA{{n}{\textopeno}{\textsubring{r}}{\textsubring{b}}{a}} \dicPos{v}[1] \dicFlx{(‑aði)}[44] \dicDirectTranslationCS{mrznout (stát na mraze), klepat se zimou} \dicExampleIS{norpa úti í kuldanum} \dicExampleCS{stát venku na mraze}
\dicEntry[norræna] \dicTerm{nor··ræn|a} \dicIPA{{n}{\textopeno}{r}{\textlengthmark}{a}{i}{n}{a}} \dicPos{f}[1] \dicFlx{(‑u)}[5] \textbf{1.} \dicDirectTranslationCS{staroseverština}  \textbf{2.} \dicSynonym{norðanvindur} \dicDirectTranslationCS{severní vítr}
\dicEntry[norrænn] \dicTerm{nor··rænn} \dicsymFrequent\  \dicIPA{{n}{\textopeno}{r}{\textlengthmark}{a}{i}{\textsubring{d}}{\textsubring{n}}} \dicPos{adj}[7]\dicFlx{}[-1] \textbf{1.} \dicDirectTranslationCS{severní (vítr ap.)}  \textbf{2.} \dicDirectTranslationCS{skandinávský, severský, nordický} \dicExampleIS{norrænt mál} \dicExampleCS{severská řeč}  \textbf{3.} \dicDirectTranslationCS{staroseverský}
\dicEntry[norska] \dicTerm{norsk|a} \dicIPA{{n}{\textopeno}{\textsubring{r}}{s}{\r{g}}{a}} \dicPos{f}[1] \dicFlx{(‑u)}[5] \dicDirectTranslationCS{norština}
\dicEntry[norskur] \dicTerm{norskur} \dicsymFrequent\  \dicIPA{{n}{\textopeno}{\textsubring{r}}{s}{\r{g}}{\textscy}{\textsubring{r}}} \dicPos{adj}[1]\dicFlx{}[-1] \dicDirectTranslationCS{norský} \dicExampleIS{Hún var norsk að uppruna.} \dicExampleCS{Byla původem Norka.}
\dicEntry[nostra] \dicTerm{nostr|a} \dicIPA{{n}{\textopeno}{s}{\textsubring{d}}{r}{a}} \dicPos{v}[1] \dicFlx{(‑aði)}[44] \dicDirectTranslationCS{hrát si, piplat se, dělat se} \dicExampleIS{Hann hefur nostrað við þetta.} \dicExampleCS{Piplal se s~tím.}
\dicEntry[nostur] \dicTerm{nostur} \dicIPA{{n}{\textopeno}{s}{\textsubring{d}}{\textscy}{\textsubring{r}}} \dicPos{n}[2] \dicFlx{(‑s)}[28] \dicDirectTranslationCS{piplání se, dělání se}
\dicEntry[not] \dicTerm{not} \dicsymFrequent\  \dicIPA{{n}{\textopeno}{\textlengthmark}{\textsubring{d}}} \dicPos{n}[2] \dicFlx{pl}[1] \textbf{1.} \dicSynonym{notkun} \dicDirectTranslationCS{(po)užití}  \textbf{2.} \dicSynonym{nytsemi} \dicDirectTranslationCS{užitek};  \dicPhraseIS{koma að notum} \dicDirectTranslationCS{přijít k~užitku};  \dicPhraseIS{verða að notum} \dicDirectTranslationCS{být k~užitku}
\dicEntry[nota] \dicTerm{not|a} \dicsymFrequent\  \dicIPA{{n}{\textopeno}{\textlengthmark}{\textsubring{d}}{a}} \dicPos{v}[1] \dicFlx{(‑aði)}[1] \dicDirectTranslationCS{(po)užít, (po)užívat, upotřebit} \dicExampleIS{nota símann oft} \dicExampleCS{používat často mobil};  \dicPhraseIS{nota tækifærið} \dicDirectTranslationCS{využít příležitosti};  \dicPhraseIS{nota sér e‑ð} \dicSynonym{notfæra} \dicDirectTranslationCS{využívat (co)} \dicExampleIS{nota sér neyð fólksins} \dicExampleCS{využívat chudobu lidí};  \dicIdiom{notast}[við]{ \dicPhraseIS{notast við e‑ð}} \dicFlx{refl} \dicDirectTranslationCS{vystačit si s~(čím)}
\dicEntry[notaður] \dicTerm{not··|aður} \dicIPA{{n}{\textopeno}{\textlengthmark}{\textsubring{d}}{a}{ð}{\textscy}{\textsubring{r}}} \dicPos{adj}[3] \dicFlx{(f ‑uð)}[3] \dicDirectTranslationCS{(po)užitý} \dicExampleIS{notuð föt} \dicExampleCS{užité šaty}
\dicEntry[notagildi] \dicTerm{nota··gildi} \dicIPA{{n}{\textopeno}{\textlengthmark}{\textsubring{d}}{a}{\r{\textObardotlessj}}{\textsci}{l}{\textsubring{d}}{\textsci}} \dicPos{n}[2] \dicFlx{(‑s)}[20] \dicDirectTranslationCS{užitečnost, užitná hodnota}
\dicEntry[notalegur] \dicTerm{nota··legur} \dicsymFrequent\  \dicIPA{{n}{\textopeno}{\textlengthmark}{\textsubring{d}}{a}{l}{\textepsilon}{\textbabygamma}{\textscy}{\textsubring{r}}} \dicPos{adj}[1]\dicFlx{}[-8] \textbf{1.} \dicSynonym{þægilegur} \dicDirectTranslationCS{útulný, pohodlný, příjemný} \dicExampleIS{notaleg tilhugsun} \dicExampleCS{příjemné pomyšlení}  \textbf{2.} \dicSynonym{vingjarnlegur} \dicDirectTranslationCS{přátelský}
\dicEntry[notandanafn] \dicTerm{notanda··|nafn} \dicIPA{{n}{\textopeno}{\textlengthmark}{\textsubring{d}}{a}{n}{\textsubring{d}}{a}{n}{a}{\textsubring{b}}{\textsubring{n}}} \dicPos{n}[2] \dicFlx{(‑nafns, ‑nöfn)}[8] \dicFieldCat{poč.} \dicDirectTranslationCS{uživatelské jméno}
\dicEntry[notandi] \dicTerm{not··|andi} \dicIPA{{n}{\textopeno}{\textlengthmark}{\textsubring{d}}{a}{n}{\textsubring{d}}{\textsci}} \dicPos{m}[2] \dicFlx{(‑anda, ‑endur)}[1] \textbf{1.} \dicDirectTranslationCS{uživatel(ka), spotřebitel(ka)}  \textbf{2.} \dicFieldCat{poč.} \dicDirectTranslationCS{uživatel(ka)}
\dicEntry[notendahandbók] \dicTerm{notenda··hand·|bók} \dicIPA{{n}{\textopeno}{\textlengthmark}{\textsubring{d}}{\textepsilon}{n}{\textsubring{d}}{a}{h}{a}{n}{\textsubring{d}}{\textsubring{b}}{ou}{\r{g}}} \dicPos{f}[8] \dicFlx{(‑bókar, ‑bækur)}[5] \dicDirectTranslationCS{uživatelská příručka}
\dicEntry[notendavænn] \dicTerm{notenda··vænn} \dicIPA{{n}{\textopeno}{\textlengthmark}{\textsubring{d}}{\textepsilon}{n}{\textsubring{d}}{a}{v}{a}{i}{\textsubring{d}}{\textsubring{n}}} \dicPos{adj}[7]\dicFlx{}[-1] \dicDirectTranslationCS{uživatelsky přívětivý}
\dicEntry[notfæra] \dicTerm{not··fær|a} \dicIPA{{n}{\textopeno}{\textlengthmark}{\textsubring{d}}{f}{a}{i}{r}{a}} \dicPos{v}[2] \dicFlx{(‑ði, ‑t)}[105] \dicPhraseIS{notfæra sér e‑ð} \dicSynonym{hagnýta} \dicDirectTranslationCS{(po)užívat (co), využívat (co)} \dicExampleIS{notfæra sér nýjustu tækni} \dicExampleCS{využívat nejnovějších technologií}
\dicEntry[nothæfur] \dicTerm{not··hæfur} \dicIPA{{n}{\textopeno}{\textlengthmark}{\textsubring{d}}{h}{a}{i}{v}{\textscy}{\textsubring{r}}} \dicPos{adj}[1]\dicFlx{}[-1] \dicSynonym{gjaldgengur} \dicDirectTranslationCS{použitelný, (jsoucí) k~použití}
\dicEntry[notið] \dicTerm{notið} \dicIPA{{n}{\textopeno}{\textlengthmark}{\textsubring{d}}{\textsci}{\texttheta}} \dicPos{v} \dicFlx{supin} \dicLink{njóta}
\dicEntry[notkun] \dicTerm{notk|un} \dicsymFrequent\  \dicIPA{{n}{\textopeno}{\textsubring{d}}{\r{g}}{\textscy}{\textsubring{n}}} \dicPos{f}[7] \dicFlx{(‑unar)}[9] \dicDirectTranslationCS{(po)užití, (po)užívání, aplikace} \dicExampleIS{notkun á vélum} \dicExampleCS{použití strojů}
\dicEntry[Nóbelsverðlaun] \dicTerm{Nóbels··verð·laun} \dicIPA{{n}{ou}{\textlengthmark}{\textsubring{b}}{\textepsilon}{l}{s}{v}{\textepsilon}{r}{ð}{l}{\oe i}{\textsubring{n}}} \dicPos{n}[2] \dicFlx{pl}[1] \dicDirectTranslationCS{Nobelova cena}
\dicEntry[nóg] \dicTerm{nóg} \dicIPA{{n}{ou}{\textlengthmark}} \dicPos{adv} \dicDirectTranslationCS{dost, hodně} \dicExampleIS{nóg af peningum} \dicExampleCS{dost peněz}
\dicEntry[nógu] \dicTerm{nógu} \dicIPA{{n}{ou}{\textlengthmark}{\textscy}} \dicPos{adv} \dicDirectTranslationCS{dost, dostatečně} \dicExampleIS{nógu lengi} \dicExampleCS{dostatečně dlouho}
\dicEntry[nógur] \dicTerm{nógur} \dicsymFrequent\  \dicIPA{{n}{ou}{\textlengthmark}{\textscy}{\textsubring{r}}} \dicPos{adj}[1] \dicFlx{(n nóg)}[14] \dicSynonym{nægur} \dicDirectTranslationCS{dostačující, dostatečný} \dicExampleIS{hafa nógan tíma} \dicExampleCS{mít dostatek času}
\dicEntry[nón] \dicTerm{nón} \dicIPA{{n}{ou}{\textlengthmark}{\textsubring{n}}} \dicPos{n}[2] \dicFlx{(‑s, ‑)}[5] \dicLangCat{zast.} \dicDirectTranslationCS{třetí hodina odpoledne}
\dicEntry[nót] \dicTerm{nót} \dicIPA{{n}{ou}{\textlengthmark}{\textsubring{d}}} \dicPos{f}[7] \dicFlx{(‑ar, nætur\,/\addthin ‑ir)}[27] \dicFieldCat{nám.} \dicDirectTranslationCS{nevod} \dicIndirectTranslationCS{(tažná rybářská síť)};  \dicPhraseIS{draga langa nót að e‑u} \dicDirectTranslationCS{vynaložit velké úsilí na (co)}
\dicEntry[nóta] \dicTerm{nót|a} \dicIPA{{n}{ou}{\textlengthmark}{\textsubring{d}}{a}} \dicPos{f}[1] \dicFlx{(‑u, ‑ur)}[13] \textbf{1.} \dicFieldCat{hud.} \dicDirectTranslationCS{nota}  \textbf{2.} \dicSynonym{reikningur} \dicDirectTranslationCS{účet, účtenka, stvrzenka};  \dicIdiom{nóta}{ \dicPhraseIS{á sömu\,/\addthin svipuðum nótum}} \dicFlx{adv} \dicDirectTranslationCS{stejným tónem}; { \dicPhraseIS{dansa\,/\addthin syngja eftir nótum e‑rs}} \dicLangCat{přen.} \dicDirectTranslationCS{tancovat\,/\addthin skákat, jak (kdo) píská};  \dicPhraseIS{vera með á nótunum} \dicLangCat{přen.} \dicDirectTranslationCS{být v~obraze, rozumět (co se děje ap.)}
\dicEntry[nótaríus] \dicTerm{nótaríus} \dicIPA{{n}{ou}{\textlengthmark}{\textsubring{d}}{a}{r}{i}{j}{\textscy}{s}} \dicPos{m}[6] \dicFlx{(‑ar, ‑ar)}[57] \dicFieldCat{práv.} \dicDirectTranslationCS{notář(ka)}
\dicEntry[nótera] \dicTerm{nóter|a} \dicIPA{{n}{ou}{\textlengthmark}{\textsubring{d}}{\textepsilon}{r}{a}} \dicPos{v}[1] \dicFlx{(‑aði)}[1] \dicFlx{acc} \dicPhraseIS{nótera e‑ð hjá sér} \dicLangCat{hovor.} \dicDirectTranslationCS{poznamenat si (co) (do zápisníku ap.)}
\dicEntry[nóti] \dicTerm{nót|i} \dicIPA{{n}{ou}{\textlengthmark}{\textsubring{d}}{\textsci}} \dicPos{m}[1] \dicFlx{(‑a, ‑ar)}[1] \dicSynonym{félagi} \dicDirectTranslationCS{kamarád(ka), kámoš(ka)} \dicExampleIS{hann og hans nótar} \dicExampleCS{on a~jemu podobní}
\dicEntry[nótnaborð] \dicTerm{nótna··borð} \dicIPA{{n}{ou}{h}{\textsubring{d}}{n}{a}{\textsubring{b}}{\textopeno}{r}{\texttheta}} \dicPos{n}[2] \dicFlx{(‑s, ‑)}[5] \dicDirectTranslationCS{klaviatura}
\dicEntry[nótnahefti] \dicTerm{nótna··hefti} \dicIPA{{n}{ou}{h}{\textsubring{d}}{n}{a}{h}{\textepsilon}{f}{\textsubring{d}}{\textsci}} \dicPos{n}[2] \dicFlx{(‑s, ‑)}[14] \dicDirectTranslationCS{notový sešit}
\dicEntry[nótnalykill] \dicTerm{nótna··lyk|ill} \dicIPA{{n}{ou}{h}{\textsubring{d}}{n}{a}{l}{\textsci}{\r{\textObardotlessj}}{\textsci}{\textsubring{d}}{\textsubring{l}}} \dicPos{m}[6] \dicFlx{(‑ils, ‑lar)}[35] \dicFieldCat{hud.} \dicDirectTranslationCS{notový klíč}
\dicEntry[nótt] \dicTerm{nótt} \dicsymFrequent\  \dicIPA{{n}{ou}{h}{\textsubring{d}}} \dicPos{f}[11] \dicFlx{(nætur, nætur)}[5] \dicDirectTranslationCS{noc} \dicExampleIS{Við dvöldum þar í eina eða tvær nætur.} \dicExampleCS{Zdrželi jsme se tam jednu nebo dvě noci.};  \dicPhraseIS{á nóttunni} \dicFlx{adv} \dicDirectTranslationCS{v~noci};  \dicPhraseIS{á næturnar} \dicFlx{adv} \dicDirectTranslationCS{v~noci, po nocích};  \dicPhraseIS{fram á nótt} \dicFlx{adv} \dicDirectTranslationCS{do noci};  \dicPhraseIS{góða nótt!} \dicDirectTranslationCS{dobrou noc!};  \dicPhraseIS{í alla nótt} \dicFlx{adv} \dicDirectTranslationCS{celou noc};  \dicPhraseIS{í nótt} \dicFlx{adv} \dicDirectTranslationCS{(dnes) v~noci};  \dicPhraseIS{í nótt sem leið} \dicFlx{adv} \dicDirectTranslationCS{v~noci, minulou noc};  \dicPhraseIS{lengi nætur} \dicFlx{adv} \dicDirectTranslationCS{dlouho do noci};  \dicPhraseIS{um miðja nótt} \dicFlx{adv} \dicDirectTranslationCS{uprostřed noci};  \dicIdiom{nótt}{ \dicPhraseIS{á einni nóttu}} \dicFlx{adv} \dicDirectTranslationCS{během jedné noci (zchudnout ap.)};  \dicPhraseIS{í skjóli nætur} \dicFlx{adv} \dicDirectTranslationCS{ve stínu noci}; { \dicPhraseIS{leggja nótt við dag}} \dicDirectTranslationCS{pracovat od rána do večera};  \dicPhraseIS{nótt sem nýtan dag} \dicDirectTranslationCS{od rána do večera (pracovat ap.)};  \dicPhraseIS{það er ekki öll nótt úti} \dicLangCat{přen.} \dicDirectTranslationCS{ještě není všechno ztraceno}
\dicEntry[nóv.] \dicTerm{nóv.} \dicPos{zkr} \dicPhraseIS{nóvember} \dicDirectTranslationCS{listopad}
\dicEntry[nóvember] \dicTerm{nóvember} \dicsymFrequent\  \dicIPA{{n}{ou}{\textlengthmark}{v}{\textepsilon}{m}{\textsubring{b}}{\textepsilon}{\textsubring{r}}} \dicPos{m}[15] \dicFlx{indecl}[2] \dicDirectTranslationCS{listopad} \dicExampleIS{Nóvember er ellefti mánuður ársins.} \dicExampleCS{Listopad je jedenáctý měsíc roku.}
\dicEntry[nóvembermánuður] \dicTerm{nóvember··mán·|uður} \dicIPA{{n}{ou}{v}{\textepsilon}{m}{\textsubring{b}}{\textepsilon}{r}{m}{au}{n}{\textscy}{ð}{\textscy}{\textsubring{r}}} \dicPos{m}[10] \dicFlx{(‑aðar, ‑uðir)}[42] \dicDirectTranslationCS{(měsíc) listopad}
\dicEntry[nr.] \dicTerm{nr.} \dicPos{zkr} \dicPhraseIS{númer} \dicDirectTranslationCS{číslo}
\dicEntry[nt.] \dicTerm{nt.} \dicPos{zkr} \dicPhraseIS{nútíð} \dicFieldCat{jaz.} \dicDirectTranslationCS{přítomný čas}
\dicEntry[nudd] \dicTerm{nudd} \dicIPA{{n}{\textscy}{\textsubring{d}}{\textlengthmark}} \dicPos{n}[2] \dicFlx{(‑s)}[2] \textbf{1.} \dicSynonym{núningur} \dicDirectTranslationCS{tření}  \textbf{2.} \dicDirectTranslationCS{masáž, masírování}  \textbf{3.} \dicSynonym{nauð\smash{\textsuperscript{1}}} \dicDirectTranslationCS{dotírání, naléhání}
\dicEntry[nudda] \dicTerm{nudd|a} \dicsymFrequent\  \dicIPA{{n}{\textscy}{\textsubring{d}}{\textlengthmark}{a}} \dicPos{v}[1] \dicFlx{(‑aði)}[1] \dicFlx{acc} \textbf{1.} \dicSynonym{núa} \dicDirectTranslationCS{třít (si), (pro)mnout (si)} \dicExampleIS{nudda augun} \dicExampleCS{mnout si oči}  \textbf{2.} \dicSynonym*{mýkja vöðvana} \dicDirectTranslationCS{(na)masírovat, (pro)mnout} \dicExampleIS{nudda vöðvana} \dicExampleCS{masírovat svaly}  \textbf{3.} \dicSynonym{þrábiðja} \dicDirectTranslationCS{naléhat, dotírat} \dicExampleIS{nudda í e‑m} \dicExampleCS{naléhat na (koho)};  \dicIdiom{nuddast}{ \dicPhraseIS{nuddast}} \dicFlx{refl} \dicDirectTranslationCS{třít se, dřít se}
\dicEntry[nuddari] \dicTerm{nudd··ar|i} \dicIPA{{n}{\textscy}{\textsubring{d}}{\textlengthmark}{a}{r}{\textsci}} \dicPos{m}[1] \dicFlx{(‑a, ‑ar)}[13] \dicDirectTranslationCS{masér(ka)}
\dicEntry[nuddpottur] \dicTerm{nudd··pott|ur} \dicIPA{{n}{\textscy}{\textsubring{d}}{p\smash{\textsuperscript{h}}}{\textopeno}{h}{\textsubring{d}}{\textscy}{\textsubring{r}}} \dicPos{m}[6] \dicFlx{(‑s, ‑ar)}[4] \dicDirectTranslationCS{vířivka}
\dicEntry[nuddstofa] \dicTerm{nudd··stof|a} \dicIPA{{n}{\textscy}{\textsubring{d}}{s}{\textsubring{d}}{\textopeno}{v}{a}} \dicPos{f}[1] \dicFlx{(‑u, ‑ur)}[7] \dicDirectTranslationCS{masážní salón}
\dicEntry[nugga] \dicTerm{nugg|a} \dicIPA{{n}{\textscy}{\r{g}}{\textlengthmark}{a}} \dicPos{v}[1] \dicFlx{(‑aði)}[1] \dicFlx{acc} \dicSynonym{núa} \dicDirectTranslationCS{(pro)třít, (pro)mnout (oči ap.)} \dicExampleIS{nugga augun} \dicExampleCS{mnout si oči}
\dicEntry[numið] \dicTerm{numið} \dicIPA{{n}{\textscy}{\textlengthmark}{m}{\textsci}{\texttheta}} \dicPos{v} \dicFlx{supin} \dicLink{nema\smash{\textsuperscript{1}}}
\dicEntry[nunna] \dicTerm{nunn|a} \dicIPA{{n}{\textscy}{n}{\textlengthmark}{a}} \dicPos{f}[1] \dicFlx{(‑u, ‑ur)}[7] \dicDirectTranslationCS{jeptiška, řádová sestra}
\dicEntry[nunnuklaustur] \dicTerm{nunnu··klaustur} \dicIPA{{n}{\textscy}{n}{\textlengthmark}{\textscy}{k\smash{\textsuperscript{h}}}{l}{\oe i}{s}{\textsubring{d}}{\textscy}{\textsubring{r}}} \dicPos{n}[2] \dicFlx{(‑s, ‑)}[25] \dicDirectTranslationCS{ženský klášter}
\dicEntry[nurla] \dicTerm{nurl|a} \dicIPA{{n}{\textscy}{r}{\textsubring{d}}{l}{a}} \dicPos{v}[1] \dicFlx{(‑aði)}[1] \dicFlx{dat} \dicDirectTranslationCS{skrblit, lakotit} \dicExampleIS{nurla e‑u saman} \dicExampleCS{skrblit (co)}
\dicEntry[nurlari] \dicTerm{nurl··ar|i} \dicIPA{{n}{\textscy}{r}{\textsubring{d}}{l}{a}{r}{\textsci}} \dicPos{m}[1] \dicFlx{(‑a, ‑ar)}[13] \dicDirectTranslationCS{skrblík, škrt, škudlil(ka)}
\dicEntry[nutum] \dicTerm{nutum} \dicIPA{{n}{\textscy}{\textlengthmark}{\textsubring{d}}{\textscy}{\textsubring{m}}} \dicPos{v} \dicFlx{ind pf pl 1 pers} \dicLink{njóta}
\dicEntry[Nuuk] \dicTerm{Nuuk} \dicIPA{{n}{\textscy}{\textlengthmark}{\r{g}}} \dicPos{subs} \dicFlx{indecl} \dicFieldCat{geog.} \dicDirectTranslationCS{Nuuk} \dicIndirectTranslationCS{(hlavní město Grónska)}
\dicEntry[nú] \dicTerm{nú\smash{\textsuperscript{1}}} \dicIPA{{n}{u}{\textlengthmark}} \dicPos{n}[2] \dicFlx{(‑s)}[2] \dicSynonym{nútími} \dicDirectTranslationCS{přítomnost, přítomný okamžik} \dicExampleIS{lifa í núinu} \dicExampleCS{žít v~přítomnosti}
\dicEntry[nú] \dicTerm{nú\smash{\textsuperscript{2}}} \dicsymFrequent\  \dicIPA{{n}{u}{\textlengthmark}} \dicPos{adv} \dicDirectTranslationCS{teď, nyní} \dicExampleIS{Nú er að koma ný ferja.} \dicExampleCS{Teď přijede další trajekt.};  \dicPhraseIS{nú á tímum} \dicFlx{adv} \dicDirectTranslationCS{v~současné\,/\addthin dnešní době, dnes} \dicIndirectTranslationCS{(opak v~minulosti)};  \dicPhraseIS{nú sem stendur} \dicFlx{adv} \dicDirectTranslationCS{v~tomto okamžiku, v~současné době}
\dicEntry[nú] \dicTerm{nú\smash{\textsuperscript{3}}} \dicIPA{{n}{u}{\textlengthmark}} \dicPos{inter} \dicDirectTranslationCS{no} \dicIndirectTranslationCS{(vyjadřuje údiv, často v~záporu)}
\dicEntry[núa] \dicTerm{núa} \dicIPA{{n}{u}{\textlengthmark}{a}} \dicPos{v}[5] \dicFlx{(ný, neri\,/\addthin néri, nerum\,/\addthin nérum, neri\,/\addthin néri, núið)}[4] \dicFlx{dat} \dicSynonym{nudda} \dicDirectTranslationCS{(pro)třít, (pro)mnout, (pro)hníst} \dicExampleIS{núa saman höndum} \dicExampleCS{mnout si ruce};  \dicIdiom{núa}[um]{ \dicPhraseIS{núa e‑m e‑ð um nasir}} \dicSynonym{brigsla} \dicDirectTranslationCS{vyčítat (komu co), vytýkat (komu co)}
\dicEntry[núðla] \dicTerm{núðl|a} \dicIPA{{n}{u}{ð}{l}{a}} \dicPos{f}[1] \dicFlx{(‑u, ‑ur)}[7] \dicDirectTranslationCS{nudle}
\dicEntry[núggat] \dicTerm{núggat} \dicIPA{{n}{u}{\r{g}}{\textlengthmark}{a}{\textsubring{d}}} \dicPos{n}[2] \dicFlx{(‑s)}[2] \dicDirectTranslationCS{nugát}
\dicEntry[núgildandi] \dicTerm{nú··gild·andi} \dicIPA{{n}{u}{\textlengthmark}{\r{\textObardotlessj}}{\textsci}{l}{\textsubring{d}}{a}{n}{\textsubring{d}}{\textsci}} \dicPos{adj}[13] \dicFlx{indecl}[1] \dicDirectTranslationCS{současný, nynější, aktuální} \dicExampleIS{núgildandi lög} \dicExampleCS{současný zákon}
\dicEntry[núið] \dicTerm{núið} \dicIPA{{n}{u}{\textlengthmark}{\textsci}{\texttheta}} \dicPos{v} \dicFlx{supin} \dicLink{núa}
\dicEntry[núliðinn] \dicTerm{nú··liðinn} \dicIPA{{n}{u}{\textlengthmark}{l}{\textsci}{ð}{\textsci}{\textsubring{n}}} \dicPos{adj}[6]\dicFlx{}[-6] \dicPhraseIS{núliðin tíð} \dicFieldCat{jaz.} \dicDirectTranslationCS{předpřítomný čas}
\dicEntry[núlifandi] \dicTerm{nú··lif·andi} \dicIPA{{n}{u}{\textlengthmark}{l}{\textsci}{v}{a}{n}{\textsubring{d}}{\textsci}} \dicPos{adj}[13] \dicFlx{indecl}[1] \dicDirectTranslationCS{žijící}
\dicEntry[núll] \dicTerm{núll} \dicIPA{{n}{u}{l}{\textlengthmark}} \dicPos{n}[2] \dicFlx{(‑s, ‑)}[5] \textbf{1.} \dicFieldCat{mat.} \dicDirectTranslationCS{nula};  \dicPhraseIS{byrja á núlli} \dicLangCat{přen.} \dicDirectTranslationCS{začít od nuly}  \textbf{2.} \dicDirectTranslationCS{nula, nikdo} \dicIndirectTranslationCS{(bezvýznamný člověk)} \dicExampleIS{Hann er núll.} \dicExampleCS{Je to nula.}
\dicEntry[núllbaugur] \dicTerm{núll··baug|ur} \dicIPA{{n}{u}{l}{\textlengthmark}{\textsubring{b}}{\oe i}{\textbabygamma}{\textscy}{\textsubring{r}}} \dicPos{m}[6] \dicFlx{(‑s, ‑ar)}[24] \dicFieldCat{geog.} \dicDirectTranslationCS{základní\,/\addthin nultý poledník}
\dicEntry[númer] \dicTerm{númer} \dicsymFrequent\  \dicIPA{{n}{u}{\textlengthmark}{m}{\textepsilon}{\textsubring{r}}} \dicPos{n}[2] \dicFlx{(‑s, ‑)}[5] \textbf{1.} \dicSynonym*{tölvustafur} \dicDirectTranslationCS{číslo} \dicIndirectTranslationCS{(vyjádření počtu)} \dicExampleIS{í herbergi númer 304} \dicExampleCS{v~pokoji číslo 304}  \textbf{2.} \dicSynonym*{skemmtiatriði} \dicDirectTranslationCS{číslo} \dicIndirectTranslationCS{(část programu\,/\addthin vystoupení)}  \textbf{3.} \dicSynonym{hefti} \dicDirectTranslationCS{číslo} \dicIndirectTranslationCS{(výtisk publikace)}
\dicEntry[númera] \dicTerm{númer|a} \dicIPA{{n}{u}{\textlengthmark}{m}{\textepsilon}{r}{a}} \dicPos{v}[1] \dicFlx{(‑aði)}[1] \dicDirectTranslationCS{(o)číslovat} \dicExampleIS{Hann númeraði blaðsíðurnar.} \dicExampleCS{Očísloval stránky.}
\dicEntry[númeraröð] \dicTerm{númera··|röð} \dicIPA{{n}{u}{\textlengthmark}{m}{\textepsilon}{r}{a}{r}{\oe}{\texttheta}} \dicPos{f}[7] \dicFlx{(‑raðar, ‑raðir)}[16] \dicDirectTranslationCS{číselné pořadí}
\dicEntry[núna] \dicTerm{núna} \dicsymFrequent\  \dicIPA{{n}{u}{\textlengthmark}{n}{a}} \dicPos{adv} \dicSynonym{nú\smash{\textsuperscript{2}}} \dicDirectTranslationCS{nyní, teď} \dicExampleIS{Ég þarf að fara núna.} \dicExampleCS{Musím teď odejít.}
\dicEntry[núningur] \dicTerm{nún··ing|ur} \dicIPA{{n}{u}{\textlengthmark}{n}{i}{\ng}{\r{g}}{\textscy}{\textsubring{r}}} \dicPos{m}[6] \dicFlx{(‑s)}[9] \textbf{1.} \dicDirectTranslationCS{tření} \dicExampleIS{framleiða rafmagn við núning} \dicExampleCS{vyrábět elektřinu třením}  \textbf{2.} \dicSynonym*{ósamkomulag} \dicDirectTranslationCS{třenice, spor}
\dicEntry[nútíð] \dicTerm{nú··tíð} \dicIPA{{n}{u}{\textlengthmark}{t\smash{\textsuperscript{h}}}{i}{\texttheta}} \dicPos{f}[7] \dicFlx{(‑ar)}[3] \textbf{1.} \dicDirectTranslationCS{současnost, přítomnost}  \textbf{2.} \dicFieldCat{jaz.} \dicDirectTranslationCS{přítomný čas}
\dicEntry[nútíma] \dicTerm{nú··tíma\smash{\textsuperscript{1}}} \dicsymFrequent\  \dicIPA{{n}{u}{\textlengthmark}{t\smash{\textsuperscript{h}}}{i}{m}{a}} \dicPos{adj}[13] \dicFlx{indecl}[1] \dicDirectTranslationCS{současný, dnešní, soudobý, moderní} \dicExampleIS{nútíma samfélag} \dicExampleCS{dnešní společnost}
\dicEntry[nútíma] \dicTerm{nú··tíma-\smash{\textsuperscript{2}}} \dicIPA{{n}{u}{\textlengthmark}{t\smash{\textsuperscript{h}}}{i}{m}{a}} \dicPos{predp} \dicDirectTranslationCS{současný, dnešní, soudobý, moderní} \dicExampleIS{nútímalist} \dicExampleCS{soudobé umění}
\dicEntry[nútímalegur] \dicTerm{nú·tíma··legur} \dicIPA{{n}{u}{\textlengthmark}{t\smash{\textsuperscript{h}}}{i}{m}{a}{l}{\textepsilon}{\textbabygamma}{\textscy}{\textsubring{r}}} \dicPos{adj}[1]\dicFlx{}[-8] \dicDirectTranslationCS{moderní, novodobý}
\dicEntry[nútími] \dicTerm{nú··tím|i} \dicIPA{{n}{u}{\textlengthmark}{t\smash{\textsuperscript{h}}}{i}{m}{\textsci}} \dicPos{m}[1] \dicFlx{(‑a)}[3] \textbf{1.} \dicDirectTranslationCS{současnost, přítomnost}  \textbf{2.} \dicFieldCat{geol.} \dicDirectTranslationCS{holocén}
\dicEntry[núv.] \dicTerm{núv.} \dicPos{zkr} \dicPhraseIS{núverandi} \dicDirectTranslationCS{současný, nynější, aktuální}
\dicEntry[núverandi] \dicTerm{nú··ver·andi} \dicsymFrequent\  \dicIPA{{n}{u}{\textlengthmark}{v}{\textepsilon}{r}{a}{n}{\textsubring{d}}{\textsci}} \dicPos{adj}[13] \dicFlx{indecl}[1] \dicDirectTranslationCS{současný, nynější, aktuální} \dicExampleIS{núverandi ríkisstjórn} \dicExampleCS{současná vláda}
\dicEntry[núvirði] \dicTerm{nú··virði} \dicIPA{{n}{u}{\textlengthmark}{v}{\textsci}{r}{ð}{\textsci}} \dicPos{n}[2] \dicFlx{(‑s, ‑)}[14] \dicFieldCat{ekon.} \dicDirectTranslationCS{současná cena}
\dicEntry[núþálegur] \dicTerm{nú·þá··legur} \dicIPA{{n}{u}{\textlengthmark}{\texttheta}{au}{l}{\textepsilon}{\textbabygamma}{\textscy}{\textsubring{r}}} \dicPos{adj}[1]\dicFlx{}[-6] \dicPhraseIS{núþáleg beyging} \dicFieldCat{jaz.} \dicDirectTranslationCS{časování silných nepravidelných sloves};  \dicPhraseIS{núþáleg sögn, núþálegt sagnorð} \dicFieldCat{jaz.} \dicDirectTranslationCS{silné nepravidelné sloveso} \dicExampleIS{(eiga, mega, unna, kunna, muna, munu, skulu, þurfa, vita, vilja)}
\dicEntry[nykur] \dicTerm{nyk|ur} \dicIPA{{n}{\textsci}{\textlengthmark}{\r{g}}{\textscy}{\textsubring{r}}} \dicPos{m}[5] \dicFlx{(‑urs, ‑rar)}[1] \dicFieldCat{pov.} \dicIndirectTranslationCS{zvíře z~lidových vyprávění podobné koni s~obrácenými kopyty, které láká lidi do vody}
\dicEntry[nyrðri] \dicTerm{nyrðri} \dicIPA{{n}{\textsci}{r}{ð}{r}{\textsci}} \dicPos{adj}[12] \dicFlx{comp (sup nyrstur)}[7] \dicDirectTranslationCS{severnější}
\dicEntry[nyrðri-hvarfbaugur] \dicTerm{nyrðri-hvarf··baug|ur} \dicIPA{{n}\-{\textsci}\-{r}\-{ð}\-{r}\-{\textsci}\-{k\smash{\textsuperscript{h}}}\-{v}\-{a}\-{r}\-{f}\-{\textsubring{b}}\-{\oe i}\-{\textbabygamma}\-{\textscy}\-{\textsubring{r}}\-} \dicPos{m}[6] \dicFlx{(‑s)}[26] \dicFieldCat{geog.} \dicDirectTranslationCS{obratník Raka}
\dicEntry[nyrst] \dicTerm{nyrst} \dicIPA{{n}{\textsci}{\textsubring{r}}{s}{\textsubring{d}}} \dicPos{adv} \dicFlx{sup} \dicLink{norður\smash{\textsuperscript{2}}}
\dicEntry[nyrstur] \dicTerm{nyrstur} \dicIPA{{n}{\textsci}{\textsubring{r}}{s}{\textsubring{d}}{\textscy}{\textsubring{r}}} \dicPos{adj} \dicFlx{m sg nom sup} \dicLink{nyrðri}
\dicEntry[nyt] \dicTerm{nyt} \dicIPA{{n}{\textsci}{\textlengthmark}{\textsubring{d}}} \dicPos{f}[4] \dicFlx{(‑jar\,/\addthin ‑ar, ‑jar\,/\addthin ‑ar)}[23] \textbf{1.} \dicSynonym{gagn\smash{\textsuperscript{1}}} \dicDirectTranslationCS{užitek};  \dicPhraseIS{færa sér e‑ð í nyt} \dicDirectTranslationCS{těžit z~(čeho), mít užitek z~(čeho)}  \textbf{2.} \dicDirectTranslationCS{nádoj, podojené mléko}
\dicEntry[nyti] \dicTerm{nyti} \dicIPA{{n}{\textsci}{\textlengthmark}{\textsubring{d}}{\textsci}} \dicPos{v} \dicFlx{con pf sg 1 pers} \dicLink{njóta}
\dicEntry[nytja] \dicTerm{nytj|a} \dicIPA{{n}{\textsci}{\textlengthmark}{\textsubring{d}}{j}{a}} \dicPos{v}[1] \dicFlx{(‑aði)}[1] \dicFlx{acc} \dicSynonym{hagnýta} \dicDirectTranslationCS{(po)užívat, využívat}
\dicEntry[nytjafall] \dicTerm{nytja··|fall} \dicIPA{{n}{\textsci}{\textlengthmark}{\textsubring{d}}{j}{a}{f}{a}{\textsubring{d}}{\textsubring{l}}} \dicPos{n}[2] \dicFlx{(‑falls, ‑föll)}[8] \dicFieldCat{jaz.} \dicDirectTranslationCS{benefaktiv}
\dicEntry[nytjahyggja] \dicTerm{nytja··hyggj|a} \dicIPA{{n}{\textsci}{\textlengthmark}{\textsubring{d}}{j}{a}{h}{\textsci}{\r{\textObardotlessj}}{a}} \dicPos{f}[1] \dicFlx{(‑u)}[5] \dicFieldCat{filos.} \dicDirectTranslationCS{prospěchářství, utilitarismus}
\dicEntry[nytjar] \dicTerm{nytjar} \dicIPA{{n}{\textsci}{\textlengthmark}{\textsubring{d}}{j}{a}{\textsubring{r}}} \dicPos{f}[4] \dicFlx{pl}[24] \dicDirectTranslationCS{užitek, přínos, přínosnost} \dicExampleIS{hafa miklar nytjar af landinu} \dicExampleCS{mít ze země velký užitek}
\dicEntry[nytsamlegur] \dicTerm{nyt·sam··legur} \dicIPA{{n}{\textsci}{\textlengthmark}{\textsubring{d}}{s}{a}{m}{l}{\textepsilon}{\textbabygamma}{\textscy}{\textsubring{r}}} \dicPos{adj}[1]\dicFlx{}[-8] \dicSynonym{gagnlegur} \dicDirectTranslationCS{užitečný, prospěšný}
\dicEntry[nytsemi] \dicTerm{nyt··sem|i} \dicIPA{{n}{\textsci}{\textlengthmark}{\textsubring{d}}{s}{\textepsilon}{m}{\textsci}} \dicPos{f}[2] \dicFlx{(‑i)}[2] \dicDirectTranslationCS{užitečnost, prospěšnost}
\dicEntry[ný] \dicTerm{ný\smash{\textsuperscript{1}}} \dicIPA{{n}{i}{\textlengthmark}} \dicPos{adj} \dicFlx{f sg nom pos} \dicLink{nýr}
\dicEntry[ný] \dicTerm{ný\smash{\textsuperscript{2}}} \dicIPA{{n}{i}{\textlengthmark}} \dicPos{v} \dicFlx{ind praes sg 1 pers} \dicLink{núa}
\dicEntry[ný] \dicTerm{ný-\smash{\textsuperscript{3}}} \dicIPA{{n}{i}{\textlengthmark}} \dicPos{predp} \dicDirectTranslationCS{neo-, novo-, nově, právě}
\dicEntry[nýár] \dicTerm{ný··ár} \dicIPA{{n}{i}{\textlengthmark}{j}{au}{\textsubring{r}}} \dicPos{n}[2] \dicFlx{(‑s, ‑)}[5] \dicDirectTranslationCS{nový rok}
\dicEntry[nýársdagur] \dicTerm{ný·árs··dag|ur} \dicIPA{{n}{i}{\textlengthmark}{j}{au}{\textsubring{r}}{s}{\textsubring{d}}{a}{\textbabygamma}{\textscy}{\textsubring{r}}} \dicPos{m}[6] \dicFlx{(‑s, ‑ar)}[62] \dicDirectTranslationCS{Nový rok} \dicIndirectTranslationCS{(svátek)}
\dicEntry[nýbakaður] \dicTerm{ný··|bak·aður} \dicIPA{{n}{i}{\textlengthmark}{\textsubring{b}}{a}{\r{g}}{a}{ð}{\textscy}{\textsubring{r}}} \dicPos{adj}[3] \dicFlx{(f ‑bökuð)}[2] \textbf{1.} \dicDirectTranslationCS{čerstvý, právě upečený (chléb ap.)}  \textbf{2.} \dicLangCat{přen.} \dicDirectTranslationCS{novopečený (otec ap.)}
\dicEntry[nýbóla] \dicTerm{ný··ból|a} \dicIPA{{n}{i}{\textlengthmark}{\textsubring{b}}{ou}{l}{a}} \dicPos{f}[1] \dicFlx{(‑u, ‑ur)}[19] \dicSynonym{nýjung} \dicDirectTranslationCS{novinka, novota}
\dicEntry[nýbreytni] \dicTerm{ný··breytn|i} \dicIPA{{n}{i}{\textlengthmark}{\textsubring{b}}{r}{ei}{h}{\textsubring{d}}{n}{\textsci}} \dicPos{f}[3] \dicFlx{(‑i)}[3] \dicDirectTranslationCS{inovace, novinka}
\dicEntry[nýbúi] \dicTerm{ný··bú|i} \dicIPA{{n}{i}{\textlengthmark}{\textsubring{b}}{u}{\textsci}} \dicPos{m}[1] \dicFlx{(‑a, ‑ar)}[1] \dicDirectTranslationCS{přistěhovalec, přistěhovalkyně, imigrant(ka)}
\dicEntry[nýbúinn] \dicTerm{ný··búinn} \dicIPA{{n}{i}{\textlengthmark}{\textsubring{b}}{u}{\textsci}{\textsubring{n}}} \dicPos{adj}[6]\dicFlx{}[-6] \dicDirectTranslationCS{právě skončený\,/\addthin hotový}
\dicEntry[nýfarinn] \dicTerm{ný··farinn} \dicIPA{{n}{i}{\textlengthmark}{f}{a}{r}{\textsci}{\textsubring{n}}} \dicPos{adj}[6]\dicFlx{}[-4] \textbf{1.} \dicDirectTranslationCS{právě odešlý}  \textbf{2.} \dicDirectTranslationCS{nově započatý}
\dicEntry[nýfluttur] \dicTerm{ný··fluttur} \dicIPA{{n}{i}{\textlengthmark}{f}{l}{\textscy}{h}{\textsubring{d}}{\textscy}{\textsubring{r}}} \dicPos{adj}[1]\dicFlx{}[-13] \dicDirectTranslationCS{nově přistěhovaný}
\dicEntry[Nýfundnaland] \dicTerm{Ný·fundna··land} \dicIPA{{n}{i}{\textlengthmark}{f}{\textscy}{n}{\textsubring{d}}{n}{a}{l}{a}{n}{\textsubring{d}}} \dicPos{n}[2] \dicFlx{(‑s)}[4] \dicFieldCat{geog.} \dicDirectTranslationCS{Newfoundland}
\dicEntry[nýfæddur] \dicTerm{ný··fæddur} \dicsymFrequent\  \dicIPA{{n}{i}{\textlengthmark}{f}{a}{i}{\textsubring{d}}{\textscy}{\textsubring{r}}} \dicPos{adj}[2]\dicFlx{}[-21] \dicDirectTranslationCS{novorozený, nově narozený} \dicExampleIS{nýfædd börn} \dicExampleCS{novorozenci}
\dicEntry[nýgiftur] \dicTerm{ný··giftur} \dicIPA{{n}{i}{\textlengthmark}{\r{\textObardotlessj}}{\textsci}{f}{\textsubring{d}}{\textscy}{\textsubring{r}}} \dicPos{adj}[1]\dicFlx{}[-13] \dicDirectTranslationCS{čerstvě ženatý}
\dicEntry[Nýja-Gínea] \dicTerm{Nýja-Gíne|a} \dicIPA{{n}{i}{j}{\textlengthmark}{a}{\textbabygamma}{i}{n}{\textepsilon}{a}} \dicPos{f}[1] \dicFlx{(‑u)}[1] \dicFieldCat{geog.} \dicDirectTranslationCS{Nová Guinea} \dicIndirectTranslationCS{(ostrov v~Tichém oceánu)}
\dicEntry[Nýja-Kaledónía] \dicTerm{Nýja-Kaledóní|a} \dicIPA{{n}{i}{j}{\textlengthmark}{a}{k\smash{\textsuperscript{h}}}{a}{l}{\textepsilon}{\textsubring{d}}{ou}{n}{i}{j}{a}} \dicPos{f}[1] \dicFlx{(‑u)}[1] \dicFieldCat{geog.} \dicDirectTranslationCS{Nová Kaledonie}
\dicEntry[Nýja-Sjáland] \dicTerm{Nýja-Sjá··land} \dicIPA{{n}{i}{j}{\textlengthmark}{a}{s}{j}{au}{l}{a}{n}{\textsubring{d}}} \dicPos{n}[2] \dicFlx{(‑s)}[4] \dicFieldCat{geog.} \dicDirectTranslationCS{Nový Zéland}
\dicEntry[nýjung] \dicTerm{nýj··ung} \dicIPA{{n}{i}{j}{\textlengthmark}{u}{\ng}{\r{g}}} \dicPos{f}[4] \dicFlx{(‑ar, ‑ar)}[1] \dicDirectTranslationCS{novinka, novota} \dicExampleIS{Stofnunin kynnir margar nýjungar á þessu ári.} \dicExampleCS{Instituce představí letošní rok řadu novinek.}
\dicEntry[nýjungagirni] \dicTerm{nýjunga··girn|i} \dicIPA{{n}{i}{j}{\textlengthmark}{u}{\ng}{\r{g}}{a}{\r{\textObardotlessj}}{\textsci}{r}{\textsubring{d}}{n}{\textsci}} \dicPos{f}[3] \dicFlx{(‑i)}[3] \dicDirectTranslationCS{novátorství, inovativnost}
\dicEntry[nýjungagjarn] \dicTerm{nýjunga··|gjarn} \dicIPA{{n}{i}{j}{\textlengthmark}{u}{\ng}{\r{g}}{a}{\r{\textObardotlessj}}{a}{r}{\textsubring{d}}{\textsubring{n}}} \dicPos{adj}[5] \dicFlx{(f ‑gjörn)}[6] \dicDirectTranslationCS{novátorský, inovační, inovativní}
\dicEntry[nýkominn] \dicTerm{ný··kominn} \dicsymFrequent\  \dicIPA{{n}{i}{\textlengthmark}{k\smash{\textsuperscript{h}}}{\textopeno}{m}{\textsci}{\textsubring{n}}} \dicPos{adj}[6]\dicFlx{}[-6] \dicDirectTranslationCS{právě přišlý\,/\addthin došlý} \dicExampleIS{Hann er nýkominn.} \dicExampleCS{Právě přišel.}
\dicEntry[nýlega] \dicTerm{ný··lega} \dicsymFrequent\  \dicIPA{{n}{i}{\textlengthmark}{l}{\textepsilon}{\textbabygamma}{a}} \dicPos{adv} \dicSynonym*{fyrir skömmu} \dicDirectTranslationCS{nedávno, před časem, tuhle} \dicExampleIS{Það hefur komið fram í fréttum nú nýlega.} \dicExampleCS{Bylo to teď nedávno ve zprávách.}
\dicEntry[nýlegur] \dicTerm{ný··legur} \dicIPA{{n}{i}{\textlengthmark}{l}{\textepsilon}{\textbabygamma}{\textscy}{\textsubring{r}}} \dicPos{adj}[1]\dicFlx{}[-8] \dicSynonym*{næstum nýr} \dicDirectTranslationCS{zánovní, (téměř) nový}
\dicEntry[nýlenda] \dicTerm{ný··lend|a} \dicIPA{{n}{i}{\textlengthmark}{l}{\textepsilon}{n}{\textsubring{d}}{a}} \dicPos{f}[1] \dicFlx{(‑u, ‑ur)}[19] \dicDirectTranslationCS{kolonie} \dicExampleIS{bresk nýlenda} \dicExampleCS{britská kolonie}
\dicEntry[nýlendustefna] \dicTerm{ný·lendu··stefn|a} \dicIPA{{n}{i}{\textlengthmark}{l}{\textepsilon}{n}{\textsubring{d}}{\textscy}{s}{\textsubring{d}}{\textepsilon}{\textsubring{b}}{n}{a}} \dicPos{f}[1] \dicFlx{(‑u)}[5] \dicFieldCat{pol.} \dicDirectTranslationCS{kolonialismus}
\dicEntry[nýliði] \dicTerm{ný··lið|i} \dicIPA{{n}{i}{\textlengthmark}{l}{\textsci}{ð}{\textsci}} \dicPos{m}[1] \dicFlx{(‑a, ‑ar)}[1] \dicSynonym{byrjandi} \dicDirectTranslationCS{začátečník, začátečnice, nováček} \dicExampleIS{nýliði í her} \dicExampleCS{branec}
\dicEntry[nýlunda] \dicTerm{ný··lund|a} \dicIPA{{n}{i}{\textlengthmark}{l}{\textscy}{n}{\textsubring{d}}{a}} \dicPos{f}[1] \dicFlx{(‑u)}[5] \textbf{1.} \dicSynonym{nýjung} \dicDirectTranslationCS{novost, novota}  \textbf{2.} \dicSynonym*{e‑að merkilegt} \dicDirectTranslationCS{zvláštnost, nezvyklost (budova ap.)}
\dicEntry[nýmjólk] \dicTerm{ný··mjólk} \dicIPA{{n}{i}{\textlengthmark}{m}{j}{ou}{\textsubring{l}}{\r{g}}} \dicPos{f}[10] \dicFlx{(‑ur)}[2] \dicDirectTranslationCS{plnotučné mléko}
\dicEntry[nýmæli] \dicTerm{ný··mæli} \dicIPA{{n}{i}{\textlengthmark}{m}{a}{i}{l}{\textsci}} \dicPos{n}[2] \dicFlx{(‑s, ‑)}[14] \textbf{1.} \dicSynonym{nýlunda} \dicDirectTranslationCS{novost, novota}  \textbf{2.} \dicSynonym*{ný frétt} \dicDirectTranslationCS{novinka, čerstvá zpráva}  \textbf{3.} \dicFieldCat{práv.} \dicDirectTranslationCS{novela}
\dicEntry[nýnasisti] \dicTerm{ný··nas·ist|i} \dicIPA{{n}{i}{\textlengthmark}{n}{a}{s}{\textsci}{s}{\textsubring{d}}{\textsci}} \dicPos{m}[1] \dicFlx{(‑a, ‑ar)}[1] \dicDirectTranslationCS{neonacista, neonacistka}
\dicEntry[nýnorska] \dicTerm{ný··norsk|a} \dicIPA{{n}{i}{\textlengthmark}{n}{\textopeno}{\textsubring{r}}{s}{\r{g}}{a}} \dicPos{f}[1] \dicFlx{(‑u)}[5] \dicDirectTranslationCS{nová norština, nynorsk} \dicIndirectTranslationCS{(jedna ze dvou oficiálních spisovných variant norštiny)}
\dicEntry[nýnæmi] \dicTerm{ný··næmi} \dicIPA{{n}{i}{\textlengthmark}{n}{a}{i}{m}{\textsci}} \dicPos{n}[2] \dicFlx{(‑s)}[20] \dicSynonym{nýjung} \dicDirectTranslationCS{novinka, novota};  \dicPhraseIS{e‑m er nýnæmi í\,/\addthin  að e‑u} \dicFlx{impers} \dicDirectTranslationCS{(co) je pro (koho) novinka}
\dicEntry[nýorðinn] \dicTerm{ný··orðinn} \dicIPA{{n}{i}{\textlengthmark}{\textopeno}{r}{ð}{\textsci}{\textsubring{n}}} \dicPos{adj}[6]\dicFlx{}[-1] \dicDirectTranslationCS{nově\,/\addthin čerstvě\,/\addthin právě zvolený\,/\addthin určený\,/\addthin učiněný, novopečený} \dicExampleIS{Hann er nýorðinn forstjóri.} \dicExampleCS{Je nově zvolený ředitel.}
\dicEntry[nýr] \dicTerm{nýr} \dicsymFrequent\  \dicIPA{{n}{i}{\textlengthmark}{\textsubring{r}}} \dicPos{adj}[4]\dicFlx{}[-7] \textbf{1.} \dicSynonym{ónotaður} \dicDirectTranslationCS{nový} \dicExampleIS{nýtt hús} \dicExampleCS{nový dům}  \textbf{2.} \dicSynonym{ferskur} \dicDirectTranslationCS{čerstvý} \dicExampleIS{nýr fiskur} \dicExampleCS{čerstvá ryba}  \textbf{3.} \dicSynonym{síðastur} \dicDirectTranslationCS{nejnovější, poslední} \dicExampleIS{nýjar fréttir} \dicExampleCS{poslední zprávy};  \dicIdiom{nýr}{ \dicPhraseIS{á ný}} \dicFlx{adv} \dicDirectTranslationCS{nanovo, opětovně}
\dicEntry[nýra] \dicTerm{nýr|a} \dicIPA{{n}{i}{\textlengthmark}{r}{a}} \dicPos{n}[1] \dicFlx{(‑a, ‑u)}[2] \dicFieldCat{anat.} \dicDirectTranslationCS{ledvina}
\dicEntry[nýrnasteinn] \dicTerm{nýrna··stein|n} \dicIPA{{n}{i}{r}{\textsubring{d}}{n}{a}{s}{\textsubring{d}}{ei}{\textsubring{d}}{\textsubring{n}}} \dicPos{m}[6] \dicFlx{(‑s, ‑ar)}[42] \dicFieldCat{med.} \dicDirectTranslationCS{ledvinový kámen}
\dicEntry[nýrómantík] \dicTerm{ný··rómantík} \dicIPA{{n}{i}{\textlengthmark}{r}{ou}{m}{a}{\textsubring{n}}{\textsubring{d}}{i}{\r{g}}} \dicPos{f}[10] \dicFlx{(‑ur)}[2] \dicFieldCat{lit.} \dicDirectTranslationCS{novoromantismus}
\dicEntry[Nýsjálendingur] \dicTerm{Ný·sjá·lend··ing|ur} \dicIPA{{n}{i}{\textlengthmark}{s}{j}{au}{l}{\textepsilon}{n}{\textsubring{d}}{i}{\ng}{\r{g}}{\textscy}{\textsubring{r}}} \dicPos{m}[6] \dicFlx{(‑s, ‑ar)}[8] \dicDirectTranslationCS{Novozélanďan(ka)}
\dicEntry[nýsjálenskur] \dicTerm{ný··sjá·lenskur} \dicIPA{{n}{i}{\textlengthmark}{s}{j}{au}{l}{\textepsilon}{n}{s}{\r{g}}{\textscy}{\textsubring{r}}} \dicPos{adj}[1]\dicFlx{}[-6] \dicDirectTranslationCS{novozélandský}
\dicEntry[nýsköpun] \dicTerm{ný··sköp|un} \dicIPA{{n}{i}{\textlengthmark}{s}{\r{g}}{\oe}{\textsubring{b}}{\textscy}{\textsubring{n}}} \dicPos{f}[7] \dicFlx{(‑unar)}[12] \dicDirectTranslationCS{inovace, novinka}
\dicEntry[nýstárlegur] \dicTerm{ný·stár··legur} \dicIPA{{n}{i}{s}{\textsubring{d}}{au}{r}{l}{\textepsilon}{\textbabygamma}{\textscy}{\textsubring{r}}} \dicPos{adj}[1]\dicFlx{}[-8] \dicSynonym{óvenjulegur} \dicDirectTranslationCS{inovativní, pokrokový, nový} \dicExampleIS{nýstárleg aðferð} \dicExampleCS{inovativní postup}
\dicEntry[nýt] \dicTerm{nýt} \dicIPA{{n}{i}{\textlengthmark}{\textsubring{d}}} \dicPos{v} \dicFlx{ind praes sg 1 pers} \dicLink{njóta}
\dicEntry[nýta] \dicTerm{nýt|a} \dicsymFrequent\  \dicIPA{{n}{i}{\textlengthmark}{\textsubring{d}}{a}} \dicPos{v}[2] \dicFlx{(‑ti, ‑t)}[54] \dicFlx{acc} \dicDirectTranslationCS{využít, upotřebit, zužitkovat} \dicExampleIS{nýta tækifærið} \dicExampleCS{využít příležitost};  \dicIdiom{nýtast}{ \dicPhraseIS{nýtast}} \dicFlx{refl} \dicDirectTranslationCS{využít se, upotřebit se} \dicExampleIS{Tækið nýtist vel.} \dicExampleCS{Přístroj se dobře využívá.}
\dicEntry[nýtanlegur] \dicTerm{nýtan··legur} \dicIPA{{n}{i}{\textlengthmark}{\textsubring{d}}{a}{n}{l}{\textepsilon}{\textbabygamma}{\textscy}{\textsubring{r}}} \dicPos{adj}[1]\dicFlx{}[-8] \dicDirectTranslationCS{využitelný, zužitkovatelný}
\dicEntry[nýtilegur] \dicTerm{nýti··legur} \dicIPA{{n}{i}{\textlengthmark}{\textsubring{d}}{\textsci}{l}{\textepsilon}{\textbabygamma}{\textscy}{\textsubring{r}}} \dicPos{adj}[1]\dicFlx{}[-8] \dicSynonym{nothæfur} \dicDirectTranslationCS{užitečný, hodící  se}
\dicEntry[nýtilkominn] \dicTerm{ný··til·kominn} \dicIPA{{n}{i}{\textlengthmark}{t\smash{\textsuperscript{h}}}{\textsci}{\textsubring{l}}{k\smash{\textsuperscript{h}}}{\textopeno}{m}{\textsci}{\textsubring{n}}} \dicPos{adj}[6]\dicFlx{}[-6] \dicSynonym{nýr} \dicDirectTranslationCS{nový, nově se objevující}
\dicEntry[nýting] \dicTerm{nýt··ing} \dicIPA{{n}{i}{\textlengthmark}{\textsubring{d}}{i}{\ng}{\r{g}}} \dicPos{f}[4] \dicFlx{(‑ar, ‑ar)}[5] \dicDirectTranslationCS{využití, využívání} \dicExampleIS{nýting á orku} \dicExampleCS{využívání energie}
\dicEntry[nýtinn] \dicTerm{nýtinn} \dicIPA{{n}{i}{\textlengthmark}{\textsubring{d}}{\textsci}{\textsubring{n}}} \dicPos{adj}[6]\dicFlx{}[-2] \dicSynonym{sparsamur} \dicDirectTranslationCS{úsporný, hospodárný, ekonomický}
\dicEntry[nýtískulegur] \dicTerm{ný·tísku··legur} \dicIPA{{n}{i}{\textlengthmark}{t\smash{\textsuperscript{h}}}{i}{s}{\r{g}}{\textscy}{l}{\textepsilon}{\textbabygamma}{\textscy}{\textsubring{r}}} \dicPos{adj}[1]\dicFlx{}[-8] \dicDirectTranslationCS{módní, moderní, trendy}
\dicEntry[nýtni] \dicTerm{nýtn|i} \dicIPA{{n}{i}{h}{\textsubring{d}}{n}{\textsci}} \dicPos{f}[3] \dicFlx{(‑i)}[3] \textbf{1.} \dicDirectTranslationCS{úspornost, hospodárnost}  \textbf{2.} \dicDirectTranslationCS{účinnost, efektivnost}
\dicEntry[nýtur] \dicTerm{nýtur} \dicIPA{{n}{i}{\textlengthmark}{\textsubring{d}}{\textscy}{\textsubring{r}}} \dicPos{adj}[1]\dicFlx{}[-1] \textbf{1.} \dicSynonym{nothæfur} \dicDirectTranslationCS{upotřebitelný, použitelný}  \textbf{2.} \dicSynonym{gagnlegur} \dicDirectTranslationCS{užitečný, prospěšný} \dicExampleIS{vera nýtur starfsmaður} \dicExampleCS{být užitečným pracovníkem};  \dicPhraseIS{einskis nýtur} \dicFlx{adj} \dicDirectTranslationCS{neužitečný, neprospěšný}
\dicEntry[nýverið] \dicTerm{ný··verið} \dicIPA{{n}{i}{\textlengthmark}{v}{\textepsilon}{r}{\textsci}{\texttheta}} \dicPos{adv} \dicSynonym{nýlega} \dicDirectTranslationCS{nedávno, před časem, onehdy}
\dicEntry[nývirki] \dicTerm{ný··virki} \dicIPA{{n}{i}{\textlengthmark}{v}{\textsci}{\textsubring{r}}{\r{\textObardotlessj}}{\textsci}} \dicPos{n}[2] \dicFlx{(‑s, ‑)}[16] \dicDirectTranslationCS{novostavba}
\dicEntry[nýyrði] \dicTerm{ný··yrði} \dicIPA{{n}{i}{\textlengthmark}{\textsci}{r}{ð}{\textsci}} \dicPos{n}[2] \dicFlx{(‑s, ‑)}[14] \dicFieldCat{jaz.} \dicDirectTranslationCS{neologismus, novotvar}
\dicEntry[næ] \dicTerm{næ} \dicIPA{{n}{a}{i}{\textlengthmark}} \dicPos{v} \dicFlx{ind praes sg 1 pers} \dicLink{ná}
\dicEntry[næða] \dicTerm{næ|ða} \dicIPA{{n}{a}{i}{\textlengthmark}{ð}{a}} \dicPos{v}[2] \dicFlx{(‑ddi, ‑tt)}[177] \dicSynonym*{blása kalt} \dicDirectTranslationCS{(chladně) vanout, vát, foukat} \dicExampleIS{Þar næða kaldir vindar.} \dicExampleCS{Vanou chladné větry.}
\dicEntry[næði] \dicTerm{næði\smash{\textsuperscript{1}}} \dicsymFrequent\  \dicIPA{{n}{a}{i}{\textlengthmark}{ð}{\textsci}} \dicPos{n}[2] \dicFlx{(‑s)}[20] \dicSynonym{ró\smash{\textsuperscript{1}}} \dicDirectTranslationCS{(po)klid, pokoj, ticho, mír} \dicExampleIS{í ró og næði} \dicExampleCS{v~klidu a~míru}
\dicEntry[næði] \dicTerm{næði\smash{\textsuperscript{2}}} \dicIPA{{n}{a}{i}{\textlengthmark}{ð}{\textsci}} \dicPos{v} \dicFlx{con pf sg 1 pers} \dicLink{ná}
\dicEntry[næfurþunnur] \dicTerm{næfur··þunnur} \dicIPA{{n}{a}{i}{\textlengthmark}{v}{\textscy}{\textsubring{r}}{\texttheta}{\textscy}{n}{\textscy}{\textsubring{r}}} \dicPos{adj}[10]\dicFlx{}[-8] \dicDirectTranslationCS{tenounký, tenoučký}
\dicEntry[nægð] \dicTerm{nægð} \dicIPA{{n}{a}{i}{\textbabygamma}{\texttheta}} \dicPos{f}[7] \dicFlx{(‑ar, ‑ir)}[1] \dicSynonym{gnægð} \dicDirectTranslationCS{dostatek, spousta}
\dicEntry[nægilega] \dicTerm{nægi··lega} \dicsymFrequent\  \dicIPA{{n}{a}{i}{j}{\textlengthmark}{\textsci}{l}{\textepsilon}{\textbabygamma}{a}} \dicPos{adv} \dicSynonym{nógu} \dicDirectTranslationCS{dostatečně, uspokojivě} \dicExampleIS{heyra ekki nægilega vel} \dicExampleCS{neslyšet dostatečně dobře}
\dicEntry[nægilegur] \dicTerm{nægi··legur} \dicsymFrequent\  \dicIPA{{n}{a}{i}{j}{\textlengthmark}{\textsci}{l}{\textepsilon}{\textbabygamma}{\textscy}{\textsubring{r}}} \dicPos{adj}[1]\dicFlx{}[-8] \dicDirectTranslationCS{dostatečný, uspokojivý, postačující} \dicExampleIS{nægileg virðing} \dicExampleCS{dostatečná úcta}
\dicEntry[nægja] \dicTerm{næg|ja} \dicsymFrequent\  \dicIPA{{n}{a}{i}{j}{\textlengthmark}{a}} \dicPos{v}[2] \dicFlx{(‑ði, ‑t)}[98] \dicFlx{acc\,/\addthin dat} \dicDirectTranslationCS{(vy)stačit, postačit} \dicExampleIS{Þessi skammtur nægir mér ekki.} \dicExampleCS{Tato porce mi nestačí.};  \dicPhraseIS{e‑m nægir e‑að} \dicFlx{impers} \dicDirectTranslationCS{(komu co) stačí};  \dicPhraseIS{láta sér e‑ð nægja} \dicDirectTranslationCS{spokojit se s~(čím)}
\dicEntry[nægjusamur] \dicTerm{nægju··|samur} \dicIPA{{n}{a}{i}{j}{\textlengthmark}{\textscy}{s}{a}{m}{\textscy}{\textsubring{r}}} \dicPos{adj}[1] \dicFlx{(f ‑söm)}[2] \dicDirectTranslationCS{nenáročný, skromný}
\dicEntry[nægjusemi] \dicTerm{nægju··sem|i} \dicIPA{{n}{a}{i}{j}{\textlengthmark}{\textscy}{s}{\textepsilon}{m}{\textsci}} \dicPos{f}[2] \dicFlx{(‑i)}[2] \dicDirectTranslationCS{nenáročnost, skromnost}
\dicEntry[nægt] \dicTerm{nægt} \dicIPA{{n}{a}{i}{x}{\textsubring{d}}} \dicPos{f}[7] \dicFlx{(‑ar, ‑ir)}[1] \dicDirectTranslationCS{dostatek, hojnost} \dicExampleIS{hafa nægtir alls} \dicExampleCS{mít všeho dostatek}
\dicEntry[nægur] \dicTerm{nægur} \dicsymFrequent\  \dicIPA{{n}{a}{i}{\textlengthmark}{\textbabygamma}{\textscy}{\textsubring{r}}} \dicPos{adj}[1]\dicFlx{}[-1] \dicDirectTranslationCS{dostatečný, dostačující, uspokojivý} \dicExampleIS{fá næga hvíld} \dicExampleCS{mít dostatek odpočinku}
\dicEntry[næla] \dicTerm{næl|a\smash{\textsuperscript{1}}} \dicIPA{{n}{a}{i}{\textlengthmark}{l}{a}} \dicPos{f}[1] \dicFlx{(‑u, ‑ur)}[19] \textbf{1.} \dicSynonym{öryggisnæla} \dicDirectTranslationCS{spínací špendlík}  \textbf{2.} \dicSynonym{skartgripur} \dicDirectTranslationCS{brož}
\dicEntry[næla] \dicTerm{næl|a\smash{\textsuperscript{2}}} \dicIPA{{n}{a}{i}{\textlengthmark}{l}{a}} \dicPos{v}[2] \dicFlx{(‑di, ‑t)}[141] \dicFlx{acc} \dicDirectTranslationCS{přišpendlit};  \dicPhraseIS{næla e‑ð saman} \dicDirectTranslationCS{sešpendlit (co)};  \dicIdiom{næla}[í]{ \dicPhraseIS{næla sér í e‑n}} \dicLangCat{přen.} \dicDirectTranslationCS{nalepit se na (koho), upnout se na (koho)}
\dicEntry[nælon] \dicTerm{nælon} \dicIPA{{n}{a}{i}{\textlengthmark}{l}{\textopeno}{\textsubring{n}}} \dicPos{n}[2] \dicFlx{(‑s)}[2] \dicDirectTranslationCS{nylon}
\dicEntry[nælonsokkabuxur] \dicTerm{nælon··sokka·buxur} \dicIPA{{n}{a}{i}{\textlengthmark}{l}{\textopeno}{n}{s}{\textopeno}{h}{\r{g}}{a}{\textsubring{b}}{\textscy}{x}{s}{\textscy}{\textsubring{r}}} \dicPos{f}[12] \dicFlx{pl}[7] \dicDirectTranslationCS{nylonky, nylonové punčochy, silonky}
\dicEntry[nælonsokkur] \dicTerm{nælon··sokk|ur} \dicIPA{{n}{a}{i}{\textlengthmark}{l}{\textopeno}{n}{s}{\textopeno}{h}{\r{g}}{\textscy}{\textsubring{r}}} \dicPos{m}[6] \dicFlx{(‑s, ‑ar)}[15] \dicDirectTranslationCS{nylonka, nylonová punčocha, silonka}
\dicEntry[næmi] \dicTerm{næm|i\smash{\textsuperscript{1}}} \dicIPA{{n}{a}{i}{\textlengthmark}{m}{\textsci}} \dicPos{f}[3] \dicFlx{(‑i)}[3] \dicLink{næmi\smash{\textsuperscript{2}}}
\dicEntry[næmi] \dicTerm{næmi\smash{\textsuperscript{2}}}\dicTerm{, næmi\smash{\textsuperscript{1}}} \dicIPA{{n}\-{a}\-{i}\-{\textlengthmark}\-{m}\-{\textsci}\-} \dicPos{n}[2] \dicFlx{(‑s)}[20] \dicSynonym{viðkvæmni} \dicDirectTranslationCS{náchylnost, citlivost, (pre)dispozice}
\dicEntry[næmi] \dicTerm{næmi\smash{\textsuperscript{3}}} \dicIPA{{n}{a}{i}{\textlengthmark}{m}{\textsci}} \dicPos{v} \dicFlx{con pf sg 1 pers} \dicLink{nema\smash{\textsuperscript{1}}}
\dicEntry[næmleiki] \dicTerm{næm··leik|i} \dicIPA{{n}{a}{i}{m}{l}{ei}{\r{\textObardotlessj}}{\textsci}} \dicPos{m}[1] \dicFlx{(‑a)}[3] \textbf{1.} \dicDirectTranslationCS{citlivost, vnímavost, senzibilita (přístroje, člověka ap.)}  \textbf{2.} \dicDirectTranslationCS{náchylnost, (pre)dispozice} \dicExampleIS{næmleiki fyrir sjúkdómum} \dicExampleCS{náchylnost k~nemocem}
\dicEntry[næmur] \dicTerm{næmur} \dicsymFrequent\  \dicIPA{{n}{a}{i}{\textlengthmark}{m}{\textscy}{\textsubring{r}}} \dicPos{adj}[1]\dicFlx{}[-1] \textbf{1.} \dicDirectTranslationCS{učenlivý, vnímavý} \dicExampleIS{vera næmur á skáldskap} \dicExampleCS{mít básnický talent};  \dicPhraseIS{hafa næmt auga fyrir e‑ð} \dicLangCat{přen.} \dicDirectTranslationCS{mít citlivé oko pro (co)}  \textbf{2.} \dicSynonym{viðkvæmur} \dicDirectTranslationCS{citlivý, senzibilní} \dicExampleIS{Mælirinn er næmur.} \dicExampleCS{Měřidlo je citlivé.}  \textbf{3.} \dicSynonym{smitnæmur} \dicDirectTranslationCS{náchylný, predisponovaný (k~nemoci ap.)};  \dicPhraseIS{vera næmur fyrir e‑ð} \dicDirectTranslationCS{být choulostivý na (co) (chlad ap.)}
\dicEntry[næpa] \dicTerm{næp|a} \dicIPA{{n}{a}{i}{\textlengthmark}{\textsubring{b}}{a}} \dicPos{f}[1] \dicFlx{(‑u, ‑ur)}[19] \dicFieldCat{bot.} \dicDirectTranslationCS{brukev řepák} \textit{(l.~{\textLA{Brassica rapa}})}
\dicEntry[nær] \dicTerm{nær} \dicIPA{{n}{a}{i}{\textlengthmark}{\textsubring{r}}} \dicPos{prep\,/\addthin adv} \dicFlx{dat} \dicDirectTranslationCS{blízko} \dicExampleIS{nær mér} \dicExampleCS{blízko mě};  \dicPhraseIS{e‑m var nær að (gera e‑ð)} \dicFlx{impers} \dicDirectTranslationCS{(kdo) měl raději ((co) udělat)};  \dicPhraseIS{vera litlu\,/\addthin engu nær} \dicDirectTranslationCS{nebýt ani trochu blízko (k~pochopení čeho)};  \dicPhraseIS{öðru nær} \dicFlx{adv} \dicDirectTranslationCS{(právě) naopak}
\dicEntry[nær-fjarlægur] \dicTerm{nær-fjar··lægur} \dicIPA{{n}{a}{i}{\textsubring{r}}{f}{j}{a}{r}{l}{a}{i}{\textbabygamma}{\textscy}{\textsubring{r}}} \dicPos{adj}[1]\dicFlx{}[-1] \dicFieldCat{jaz.} \dicDirectTranslationCS{téměř otevřený (samohláska ap.)}
\dicEntry[nær-frammæltur] \dicTerm{nær-fram··mæltur} \dicIPA{{n}{a}{i}{\textsubring{r}}{f}{r}{a}{m}{a}{i}{\textsubring{l}}{\textsubring{d}}{\textscy}{\textsubring{r}}} \dicPos{adj}[1]\dicFlx{}[-13] \dicFieldCat{jaz.} \dicDirectTranslationCS{téměř přední (samohláska ap.)}
\dicEntry[nær-nálægur] \dicTerm{nær-ná··lægur} \dicIPA{{n}{a}{i}{r}{n}{au}{l}{a}{i}{\textbabygamma}{\textscy}{\textsubring{r}}} \dicPos{adj}[1]\dicFlx{}[-1] \dicFieldCat{jaz.} \dicDirectTranslationCS{téměř zavřený (samohláska ap.)}
\dicEntry[nær-uppmæltur] \dicTerm{nær-upp··mæltur} \dicIPA{{n}{a}{i}{r}{\textscy}{h}{\textsubring{b}}{m}{a}{i}{\textsubring{l}}{\textsubring{d}}{\textscy}{\textsubring{r}}} \dicPos{adj}[1]\dicFlx{}[-13] \dicFieldCat{jaz.} \dicDirectTranslationCS{téměř zadní (samohláska ap.)}
\dicEntry[næra] \dicTerm{nær|a} \dicIPA{{n}{a}{i}{\textlengthmark}{r}{a}} \dicPos{v}[2] \dicFlx{(‑ði, ‑t)}[99] \dicFlx{acc} \dicSynonym{fæða\smash{\textsuperscript{2}}} \dicDirectTranslationCS{živit, krmit, vyživovat};  \dicIdiom{nærast}{ \dicPhraseIS{nærast}} \dicFlx{refl} \dicSynonym{borða} \dicDirectTranslationCS{živit se (potravou ap.)}; { \dicPhraseIS{nærast á e‑u}} \dicFlx{refl} \dicDirectTranslationCS{živit se (čím)} \dicExampleIS{Gerlar nærast á sykri.} \dicExampleCS{Bakterie se živí cukrem.}
\dicEntry[nærbolur] \dicTerm{nær··bol|ur} \dicIPA{{n}{a}{i}{r}{\textsubring{b}}{\textopeno}{l}{\textscy}{\textsubring{r}}} \dicPos{m}[9] \dicFlx{(‑s, ‑ir)}[8] \dicDirectTranslationCS{tílko, nátělník}
\dicEntry[nærbuxur] \dicTerm{nær··buxur} \dicIPA{{n}{a}{i}{r}{\textsubring{b}}{\textscy}{x}{s}{\textscy}{\textsubring{r}}} \dicPos{f}[12] \dicFlx{pl}[7] \dicDirectTranslationCS{slipy, kalhotky}
\dicEntry[nærfærni] \dicTerm{nær··færn|i} \dicIPA{{n}{a}{i}{\textsubring{r}}{f}{a}{i}{r}{\textsubring{d}}{n}{\textsci}} \dicPos{f}[3] \dicFlx{(‑i)}[3] \dicDirectTranslationCS{taktnost, ohleduplnost} \dicExampleIS{umgangast hana af sérstakri nærfærni} \dicExampleCS{chovat se k~ní obzvlášť ohleduplně}
\dicEntry[nærföt] \dicTerm{nær··föt} \dicIPA{{n}{a}{i}{\textsubring{r}}{f}{\oe}{\textsubring{d}}} \dicPos{n}[2] \dicFlx{pl}[9] \dicSynonym{undirföt} \dicDirectTranslationCS{spodní prádlo}
\dicEntry[nærgætinn] \dicTerm{nær··gætinn} \dicIPA{{n}{a}{i}{r}{\r{\textObardotlessj}}{a}{i}{\textsubring{d}}{\textsci}{\textsubring{n}}} \dicPos{adj}[6]\dicFlx{}[-2] \dicDirectTranslationCS{ohleduplný, šetrný, empatický} \dicExampleIS{vera nærgætinn við e‑n} \dicExampleCS{být ke (komu) ohleduplný}
\dicEntry[nærgætni] \dicTerm{nær··gætn|i} \dicIPA{{n}{a}{i}{r}{\r{\textObardotlessj}}{a}{i}{h}{\textsubring{d}}{n}{\textsci}} \dicPos{f}[3] \dicFlx{(‑i)}[3] \dicDirectTranslationCS{ohleduplnost, šetrnost, empatie} \dicExampleIS{sýna e‑n nærgætni} \dicExampleCS{chovat se ke (komu) ohleduplně}
\dicEntry[nærgöngull] \dicTerm{nær··göngull} \dicIPA{{n}{a}{i}{r}{\r{g}}{\oe i}{\ng}{\r{g}}{\textscy}{\textsubring{d}}{\textsubring{l}}} \dicPos{adj}[8]\dicFlx{}[-4] \dicSynonym{áleitinn} \dicDirectTranslationCS{dotěrný, neodbytný} \dicExampleIS{nærgöngular spurningar} \dicExampleCS{dotěrné otázky}
\dicEntry[næring] \dicTerm{nær··ing} \dicIPA{{n}{a}{i}{\textlengthmark}{r}{i}{\ng}{\r{g}}} \dicPos{f}[4] \dicFlx{(‑ar)}[7] \dicDirectTranslationCS{výživa} \dicExampleIS{næring barna} \dicExampleCS{výživa dětí}
\dicEntry[næringarefni] \dicTerm{næringar··efni} \dicIPA{{n}{a}{i}{\textlengthmark}{r}{i}{\ng}{\r{g}}{a}{r}{\textepsilon}{\textsubring{b}}{n}{\textsci}} \dicPos{n}[2] \dicFlx{(‑s, ‑)}[14] \dicFieldCat{biol.} \dicDirectTranslationCS{živina}
\dicEntry[næringarfræði] \dicTerm{næringar··fræð|i} \dicIPA{{n}{a}{i}{\textlengthmark}{r}{i}{\ng}{\r{g}}{a}{\textsubring{r}}{f}{r}{a}{i}{ð}{\textsci}} \dicPos{f}[3] \dicFlx{(‑i)}[3] \dicDirectTranslationCS{dietetika, nauka o~výživě}
\dicEntry[næringarfræðingur] \dicTerm{næringar·fræð··ing|ur} \dicIPA{{n}\-{a}\-{i}\-{\textlengthmark}\-{r}\-{i}\-{\ng}\-{\r{g}}\-{a}\-{\textsubring{r}}\-{f}\-{r}\-{a}\-{i}\-{ð}\-{i}\-{\ng}\-{\r{g}}\-{\textscy}\-{\textsubring{r}}\-} \dicPos{m}[6] \dicFlx{(‑s, ‑ar)}[8] \dicDirectTranslationCS{dietetik, dietetička, odborník\,/\addthin odbornice na výživu}
\dicEntry[næringarríkur] \dicTerm{næringar··ríkur} \dicIPA{{n}{a}{i}{\textlengthmark}{r}{i}{\ng}{\r{g}}{a}{r}{i}{\r{g}}{\textscy}{\textsubring{r}}} \dicPos{adj}[1]\dicFlx{}[-1] \dicDirectTranslationCS{výživný}
\dicEntry[næringarskortur] \dicTerm{næringar··skort|ur} \dicIPA{{n}{a}{i}{\textlengthmark}{r}{i}{\ng}{\r{g}}{a}{\textsubring{r}}{s}{\r{g}}{\textopeno}{\textsubring{r}}{\textsubring{d}}{\textscy}{\textsubring{r}}} \dicPos{m}[6] \dicFlx{(‑s)}[7] \dicDirectTranslationCS{podvýživa}
\dicEntry[nærkominn] \dicTerm{nær··kominn} \dicIPA{{n}{a}{i}{\textsubring{r}}{k\smash{\textsuperscript{h}}}{\textopeno}{m}{\textsci}{\textsubring{n}}} \dicPos{adj}[6]\dicFlx{}[-2] \dicSynonym{nákominn} \dicDirectTranslationCS{blízký, spřízněný} \dicExampleIS{nærkominn frændi} \dicExampleCS{blízký strýc}
\dicEntry[nærpils] \dicTerm{nær··pils} \dicIPA{{n}{a}{i}{\textsubring{r}}{p\smash{\textsuperscript{h}}}{\textsci}{l}{s}} \dicPos{n}[2] \dicFlx{(‑, ‑)}[22] \dicSynonym{undirpils} \dicDirectTranslationCS{spodnička}
\dicEntry[nærri] \dicTerm{nærri\smash{\textsuperscript{1}}} \dicIPA{{n}{a}{i}{r}{\textlengthmark}{\textsci}} \dicPos{adj} \dicFlx{comp m} \dicLink{næstur}
\dicEntry[nærri] \dicTerm{nærri\smash{\textsuperscript{2}}} \dicsymFrequent\  \dicIPA{{n}{a}{i}{r}{\textlengthmark}{\textsci}} \dicPos{prep\,/\addthin adv} \dicFlx{comp (sup næst)} \dicFlx{dat} \textbf{1.} \dicFlx{prep} \dicSynonym{nálægt} \dicDirectTranslationCS{blízko} \dicIndirectTranslationCS{(o~pozici)} \dicExampleIS{Hún býr nærri miðborginni.} \dicExampleCS{Bydlí blízko centra.}  \textbf{2.} \dicFlx{prep} \dicDirectTranslationCS{blízko, kolem} \dicIndirectTranslationCS{(o~čase)} \dicExampleIS{nærri páskum} \dicExampleCS{kolem Velikonoc}  \textbf{3.} \dicFlx{adv} \dicDirectTranslationCS{poblíž, v~blízkosti}  \textbf{4.} \dicFlx{adv} \dicDirectTranslationCS{téměř, málem} \dicExampleIS{Ég var nærri sofnaður.} \dicExampleCS{Málem jsem usnul.};  \dicPhraseIS{nærri því} \dicFlx{adv} \dicDirectTranslationCS{skoro, téměř}
\dicEntry[nærskorinn] \dicTerm{nær··skorinn} \dicIPA{{n}{a}{i}{\textsubring{r}}{s}{\r{g}}{\textopeno}{r}{\textsci}{\textsubring{n}}} \dicPos{adj}[6]\dicFlx{}[-2] \dicDirectTranslationCS{(dokonale) padnoucí, přiléhavý} \dicExampleIS{nærskorin flík} \dicExampleCS{padnoucí šaty}
\dicEntry[nærskyrta] \dicTerm{nær··skyrt|a} \dicIPA{{n}{a}{i}{\textsubring{r}}{s}{\r{\textObardotlessj}}{\textsci}{\textsubring{r}}{\textsubring{d}}{a}} \dicPos{f}[1] \dicFlx{(‑u, ‑ur)}[19] \dicDirectTranslationCS{tílko, nátělník}
\dicEntry[nærstaddur] \dicTerm{nær··|staddur} \dicIPA{{n}{a}{i}{\textsubring{r}}{s}{\textsubring{d}}{a}{\textsubring{d}}{\textscy}{\textsubring{r}}} \dicPos{adj}[2] \dicFlx{(f ‑stödd)}[20] \dicDirectTranslationCS{přítomný, jsoucí poblíž}
\dicEntry[nærsýni] \dicTerm{nær··sýn|i} \dicIPA{{n}{a}{i}{\textsubring{r}}{s}{i}{n}{\textsci}} \dicPos{f}[3] \dicFlx{(‑i)}[3] \textbf{1.} \dicFieldCat{med.} \dicDirectTranslationCS{krátkozrakost} \dicAntonym{fjarsýni}  \textbf{2.} \dicSynonym*{skammsýni} \dicDirectTranslationCS{krátkozrakost, neuváženost, neprozíravost}
\dicEntry[nærsýnn] \dicTerm{nær··sýnn} \dicIPA{{n}{a}{i}{\textsubring{r}}{s}{i}{\textsubring{d}}{\textsubring{n}}} \dicPos{adj}[7]\dicFlx{}[-1] \textbf{1.} \dicFieldCat{med.} \dicDirectTranslationCS{krátkozraký}  \textbf{2.} \dicLangCat{přen.} \dicSynonym{skammsýnn} \dicDirectTranslationCS{krátkozraký, neuvážený, neprozíravý}
\dicEntry[nærtækur] \dicTerm{nær··tækur} \dicIPA{{n}{a}{i}{\textsubring{r}}{t\smash{\textsuperscript{h}}}{a}{i}{\r{g}}{\textscy}{\textsubring{r}}} \dicPos{adj}[1]\dicFlx{}[-1] \dicDirectTranslationCS{blízký, (jsoucí) po\,/\addthin při ruce (obchod ap.)}
\dicEntry[nærvera] \dicTerm{nær··ver|a} \dicIPA{{n}{a}{i}{r}{v}{\textepsilon}{r}{a}} \dicPos{f}[1] \dicFlx{(‑u)}[5] \textbf{1.} \dicSynonym{návist} \dicDirectTranslationCS{přítomnost, výskyt}  \textbf{2.} \dicSynonym{áhrif} \dicDirectTranslationCS{vliv, působení} \dicExampleIS{Hún hefur góða nærveru.} \dicExampleCS{Má dobrý vliv.}
\dicEntry[nærverufall] \dicTerm{nær·veru··|fall} \dicIPA{{n}{a}{i}{r}{v}{\textepsilon}{r}{\textscy}{f}{a}{\textsubring{d}}{\textsubring{l}}} \dicPos{n}[2] \dicFlx{(‑falls, ‑föll)}[8] \dicFieldCat{jaz.} \dicDirectTranslationCS{adessiv}
\dicEntry[næs] \dicTerm{næs} \dicIPA{{n}{a}{i}{\textlengthmark}{s}} \dicPos{adj}[13] \dicFlx{indecl}[1] \textbf{1.} \dicLangCat{hovor.} \dicSynonym{fallegur} \dicDirectTranslationCS{hezký, pěkný}  \textbf{2.} \dicLangCat{hovor.} \dicSynonym{þægilegur} \dicDirectTranslationCS{milý, příjemný}
\dicEntry[næst] \dicTerm{næst\smash{\textsuperscript{1}}} \dicIPA{{n}{a}{i}{s}{\textsubring{d}}} \dicPos{adv} \dicFlx{sup (comp nærri)} \dicSynonym*{næsta skipti} \dicDirectTranslationCS{(na)příště}
\dicEntry[næst] \dicTerm{næst\smash{\textsuperscript{2}}} \dicIPA{{n}{a}{i}{s}{\textsubring{d}}} \dicPos{prep\,/\addthin adv} \dicFlx{sup (comp nærri)} \dicFlx{dat} \dicFlx{adv} \dicDirectTranslationCS{nejblíže, blízko, vedle};  \dicPhraseIS{næst e‑u} \dicFlx{prep} \dicDirectTranslationCS{nejblíže (čeho), vedle (čeho)};  \dicPhraseIS{því næst} \dicFlx{adv} \dicDirectTranslationCS{pak, potom, poté};  \dicPhraseIS{(því) sem næst} \dicFlx{adv} \dicDirectTranslationCS{téměř}
\dicEntry[næsta] \dicTerm{næst|a\smash{\textsuperscript{1}}} \dicIPA{{n}{a}{i}{s}{\textsubring{d}}{a}} \dicPos{f}[1] \dicFlx{(‑u)}[5] \dicPhraseIS{á næstunni} \dicFlx{adv} \dicDirectTranslationCS{zanedlouho, brzy} \dicExampleIS{Hann kemur á næstunni.} \dicExampleCS{Zanedlouho přijde.}
\dicEntry[næsta] \dicTerm{næsta\smash{\textsuperscript{2}}} \dicsymFrequent\  \dicIPA{{n}{a}{i}{s}{\textsubring{d}}{a}} \dicPos{adv} \dicSynonym{mjög} \dicDirectTranslationCS{docela, dost};  \dicPhraseIS{næsta víst} \dicFlx{adv} \dicDirectTranslationCS{docela jistě}
\dicEntry[næstbestur] \dicTerm{næst··bestur} \dicIPA{{n}{a}{i}{s}{\textsubring{d}}{\textsubring{b}}{\textepsilon}{s}{\textsubring{d}}{\textscy}{\textsubring{r}}} \dicPos{adj}[11]\dicFlx{}[-2] \dicFlx{m sg nom sup} \dicDirectTranslationCS{druhý nejlepší}
\dicEntry[næstkomandi] \dicTerm{næst··kom·andi} \dicIPA{{n}{a}{i}{s}{\textsubring{d}}{k\smash{\textsuperscript{h}}}{\textopeno}{m}{a}{n}{\textsubring{d}}{\textsci}} \dicPos{adj}[13] \dicFlx{indecl}[1] \dicDirectTranslationCS{následující, příští} \dicExampleIS{næstkomandi laugardagur} \dicExampleCS{příští sobota}
\dicEntry[næstliðinn] \dicTerm{næst··liðinn} \dicIPA{{n}{a}{i}{s}{\textsubring{d}}{l}{\textsci}{ð}{\textsci}{\textsubring{n}}} \dicPos{adj}[6]\dicFlx{}[-6] \dicSynonym{síðastliðinn} \dicDirectTranslationCS{předchozí, minulý (uplynulý)} \dicExampleIS{næstliðið sumar} \dicExampleCS{minulé léto}
\dicEntry[næstsíðastur] \dicTerm{næst··síð|astur} \dicIPA{{n}{a}{i}{s}{\textsubring{d}}{s}{i}{ð}{a}{s}{\textsubring{d}}{\textscy}{\textsubring{r}}} \dicPos{adj}[2] \dicFlx{(f ‑ust)}[9] \dicDirectTranslationCS{předposlední} \dicExampleIS{næstsíðasti dagur ársins} \dicExampleCS{předposlední den v~roce}
\dicEntry[næstum] \dicTerm{næstum} \dicIPA{{n}{a}{i}{s}{\textsubring{d}}{\textscy}{\textsubring{m}}} \dicPos{adv} \dicSynonym*{hér um bil} \dicDirectTranslationCS{téměř, skoro, takřka, bezmála};  \dicPhraseIS{næstum því} \dicFlx{adv} \dicDirectTranslationCS{téměř, skoro, takřka, bezmála}
\dicEntry[næstur] \dicTerm{næstur} \dicIPA{{n}{a}{i}{s}{\textsubring{d}}{\textscy}{\textsubring{r}}} \dicPos{adj}[12] \dicFlx{sup (comp nærri)}[11] \dicDirectTranslationCS{nejbližší, příští, další} \dicExampleIS{í næstu viku} \dicExampleCS{příští týden}
\dicEntry[nætur] \dicTerm{nætur} \dicIPA{{n}{a}{i}{\textlengthmark}{\textsubring{d}}{\textscy}{\textsubring{r}}} \dicPos{f} \dicFlx{pl nom} \dicLink{nótt}
\dicEntry[næturgagn] \dicTerm{nætur··|gagn} \dicIPA{{n}{a}{i}{\textlengthmark}{\textsubring{d}}{\textscy}{r}{\r{g}}{a}{\r{g}}{\textsubring{n}}} \dicPos{n}[2] \dicFlx{(‑gagns, ‑gögn)}[8] \dicDirectTranslationCS{nočník}
\dicEntry[næturgali] \dicTerm{nætur··gal|i}\dicTerm{, náttgali} \dicIPA{{n}\-{a}\-{i}\-{\textlengthmark}\-{\textsubring{d}}\-{\textscy}\-{r}\-{\r{g}}\-{a}\-{l}\-{\textsci}\-} \dicPos{m}[1] \dicFlx{(‑a, ‑ar)}[8] \dicFieldCat{zool.} \dicDirectTranslationCS{slavík, slavík obecný} \textit{(l.~{\textLA{Luscinia megarhynchos}})}  \dicsymPhoto\ 
\dicFigure{ds_image_naeturgali_0_1.jpg}{Næturgali}{Næturgali - Augustin Povedano, CC BY-SA 2.5}
\dicEntry[næturgestur] \dicTerm{nætur··gest|ur} \dicIPA{{n}{a}{i}{\textlengthmark}{\textsubring{d}}{\textscy}{r}{\r{\textObardotlessj}}{\textepsilon}{s}{\textsubring{d}}{\textscy}{\textsubring{r}}} \dicPos{m}[9] \dicFlx{(‑s, ‑ir)}[4] \dicDirectTranslationCS{nocležník, nocležnice}
\dicEntry[næturgisting] \dicTerm{nætur··gist·ing} \dicIPA{{n}{a}{i}{\textlengthmark}{\textsubring{d}}{\textscy}{r}{\r{\textObardotlessj}}{\textsci}{s}{\textsubring{d}}{i}{\ng}{\r{g}}} \dicPos{f}[4] \dicFlx{(‑ar, ‑ar)}[5] \dicDirectTranslationCS{nocleh, přenocování} \dicExampleIS{biðja um næturgistingu} \dicExampleCS{požádat o~nocleh}
\dicEntry[næturlíf] \dicTerm{nætur··líf} \dicIPA{{n}{a}{i}{\textlengthmark}{\textsubring{d}}{\textscy}{r}{l}{i}{f}} \dicPos{n}[2] \dicFlx{(‑s, ‑)}[5] \dicDirectTranslationCS{noční život}
\dicEntry[næturljóð] \dicTerm{nætur··ljóð} \dicIPA{{n}{a}{i}{\textlengthmark}{\textsubring{d}}{\textscy}{r}{l}{j}{ou}{\texttheta}} \dicPos{n}[2] \dicFlx{(‑s, ‑)}[5] \dicFieldCat{hud.} \dicDirectTranslationCS{nokturno}
\dicEntry[næturlæknir] \dicTerm{nætur··lækn|ir} \dicIPA{{n}{a}{i}{\textlengthmark}{\textsubring{d}}{\textscy}{r}{l}{a}{i}{h}{\r{g}}{n}{\textsci}{\textsubring{r}}} \dicPos{m}[7] \dicFlx{(‑is, ‑ar)}[1] \dicDirectTranslationCS{noční lékař(ka)}
\dicEntry[næturstaður] \dicTerm{nætur··stað|ur} \dicIPA{{n}{a}{i}{\textlengthmark}{\textsubring{d}}{\textscy}{\textsubring{r}}{s}{\textsubring{d}}{a}{ð}{\textscy}{\textsubring{r}}} \dicPos{m}[10] \dicFlx{(‑ar, ‑ir)}[14] \dicDirectTranslationCS{noclehárna, místo na přenocování}
\dicEntry[næturvakt] \dicTerm{nætur··vakt} \dicIPA{{n}{a}{i}{\textlengthmark}{\textsubring{d}}{\textscy}{r}{v}{a}{x}{\textsubring{d}}} \dicPos{f}[7] \dicFlx{(‑ar, ‑ir)}[1] \textbf{1.} \dicDirectTranslationCS{noční směna} \dicIndirectTranslationCS{(střídání skupiny pracovníků)}  \textbf{2.} \dicDirectTranslationCS{noční směna} \dicIndirectTranslationCS{(střídající skupina osob pracující v~noci)}
\dicEntry[næturvarsla] \dicTerm{nætur··|varsla} \dicIPA{{n}{a}{i}{\textlengthmark}{\textsubring{d}}{\textscy}{r}{v}{a}{\textsubring{r}}{s}{\textsubring{d}}{l}{a}} \dicPos{f}[1] \dicFlx{(‑vörslu)}[2] \dicDirectTranslationCS{noční směna\,/\addthin hlídka}
\dicEntry[næturvörður] \dicTerm{nætur··|vörður} \dicIPA{{n}{a}{i}{\textlengthmark}{\textsubring{d}}{\textscy}{r}{v}{\oe}{r}{ð}{\textscy}{\textsubring{r}}} \dicPos{m}[11] \dicFlx{(‑varðar, ‑verðir)}[5] \dicDirectTranslationCS{noční hlídač(ka)}
\dicEntry[næturþel] \dicTerm{nætur··þel} \dicIPA{{n}{a}{i}{\textlengthmark}{\textsubring{d}}{\textscy}{\textsubring{r}}{\texttheta}{\textepsilon}{\textsubring{l}}} \dicPos{n}[2] \dicFlx{(‑s)}[2] \dicPhraseIS{á næturþeli} \dicFlx{adv} \dicDirectTranslationCS{v~noci, během noci}
\dicEntry[nöðru] \dicTerm{nöðru} \dicIPA{{n}{\oe}{ð}{r}{\textscy}} \dicPos{f} \dicFlx{sg gen} \dicLink{naðra}
\dicEntry[nöf] \dicTerm{nöf} \dicIPA{{n}{\oe}{\textlengthmark}{f}} \dicPos{f}[7] \dicFlx{(nafar, nafir)}[16] \textbf{1.} \dicDirectTranslationCS{náboj (kola ap.)}  \textbf{2.} \dicSynonym*{bjargbrún} \dicDirectTranslationCS{okraj\,/\addthin hrana útesu\,/\addthin skály};  \dicPhraseIS{vera kominn á fremstu\,/\addthin ystu nöf með e‑ð} {\textbf{a.}} \dicLangCat{přen.} \dicDirectTranslationCS{být krůček od (čeho)};  {\textbf{b.}} \dicLangCat{přen.} \dicDirectTranslationCS{dojít až na samý okraj s~(čím)}
\dicEntry[nöfn] \dicTerm{nöfn} \dicIPA{{n}{\oe}{\textsubring{b}}{\textsubring{n}}} \dicPos{n} \dicFlx{pl nom} \dicLink{nafn}
\dicEntry[nögl] \dicTerm{nögl} \dicsymFrequent\  \dicIPA{{n}{\oe}{\r{g}}{\textsubring{l}}} \dicPos{f}[8] \dicFlx{(naglar, neglur)}[2] \textbf{1.} \dicDirectTranslationCS{nehet};  \dicPhraseIS{naga á sér neglurnar} \dicDirectTranslationCS{kousat si nehty};  \dicPhraseIS{vera sem lús milli tveggja nagla} \dicLangCat{přen.} \dicDirectTranslationCS{být jako veš mezi dvěma nehty} \dicIndirectTranslationCS{(být ve velkých obtížích)}  \textbf{2.} \dicFieldCat{hud.} \dicDirectTranslationCS{trsátko}
\dicEntry[nöldra] \dicTerm{nöldr|a} \dicIPA{{n}{\oe}{l}{\textsubring{d}}{r}{a}} \dicPos{v}[1] \dicFlx{(‑aði)}[44] \dicDirectTranslationCS{reptat, remcat} \dicExampleIS{nöldra yfir e‑u, nöldra um e‑ð} \dicExampleCS{reptat na (co)}
\dicEntry[nöldrari] \dicTerm{nöldr··ar|i} \dicIPA{{n}{\oe}{l}{\textsubring{d}}{r}{a}{r}{\textsci}} \dicPos{m}[1] \dicFlx{(‑a, ‑ar)}[13] \dicDirectTranslationCS{bručoun(ka), reptal}
\dicEntry[nöldur] \dicTerm{nöldur} \dicIPA{{n}{\oe}{l}{\textsubring{d}}{\textscy}{\textsubring{r}}} \dicPos{n}[2] \dicFlx{(‑s)}[28] \dicDirectTranslationCS{reptání, remcání} \dicExampleIS{vera með eilíft nöldur yfir e‑u} \dicExampleCS{neustále na (co) remcat}
\dicEntry[nöldurseggur] \dicTerm{nöldur··segg|ur} \dicIPA{{n}{\oe}{l}{\textsubring{d}}{\textscy}{\textsubring{r}}{s}{\textepsilon}{\r{g}}{\textscy}{\textsubring{r}}} \dicPos{m}[9] \dicFlx{(‑s, ‑ir)}[15] \dicDirectTranslationCS{kverulant(ka), reptal}
\dicEntry[nöp] \dicTerm{nöp} \dicIPA{{n}{\oe}{\textlengthmark}{\textsubring{b}}} \dicPos{f}[7] \dicFlx{(napar)}[21] \dicPhraseIS{vera í nöp við e‑n} \dicDirectTranslationCS{být zaujatý vůči (komu)}
\dicEntry[nöpur] \dicTerm{nöpur} \dicIPA{{n}{\oe}{\textlengthmark}{\textsubring{b}}{\textscy}{\textsubring{r}}} \dicPos{adj} \dicFlx{f sg nom pos} \dicLink{napur}
\dicEntry[nös] \dicTerm{nös} \dicsymFrequent\  \dicIPA{{n}{\oe}{\textlengthmark}{s}} \dicPos{f}[7] \dicFlx{(nasar, nasir)}[16] \dicFieldCat{anat.} \dicDirectTranslationCS{nosní dírka, nozdra} \dicExampleIS{blása reyknum út um nasir} \dicExampleCS{vypustit dým nosními dírkami}
\dicEntry[nösk] \dicTerm{nösk} \dicIPA{{n}{\oe}{s}{\r{g}}} \dicPos{adj} \dicFlx{f sg nom pos} \dicLink{naskur}
\dicEntry[nötra] \dicTerm{nötr|a} \dicIPA{{n}{\oe}{\textlengthmark}{\textsubring{d}}{r}{a}} \dicPos{v}[1] \dicFlx{(‑aði)}[44] \dicSynonym{titra} \dicDirectTranslationCS{chvět se, třást se (strachem, zimou ap.)}
\dicEntry[nötur] \dicTerm{nötur} \dicIPA{{n}{\oe}{\textlengthmark}{\textsubring{d}}{\textscy}{\textsubring{r}}} \dicPos{n}[2] \dicFlx{(‑s)}[28] \dicSynonym{skjálfti} \dicDirectTranslationCS{chvění, třesení}
\dicEntry[nöturlegur] \dicTerm{nötur··legur} \dicIPA{{n}{\oe}{\textlengthmark}{\textsubring{d}}{\textscy}{r}{l}{\textepsilon}{\textbabygamma}{\textscy}{\textsubring{r}}} \dicPos{adj}[1]\dicFlx{}[-8] \textbf{1.} \dicSynonym{eymdarlegur} \dicDirectTranslationCS{ubohý, bezútěšný, bídný} \dicExampleIS{nöturlegt umhverfi} \dicExampleCS{bezútěšné prostředí}  \textbf{2.} \dicSynonym{kuldalegur} \dicDirectTranslationCS{nevlídný, chladný}

