\item[{abb}] {stýfður}
\item[{acc}] {þolfall}
\item[{adj}] {lýsingarorð}
\item[{adv}] {atviksorð}
\item[{akt}] {germynd}
\item[{anat.}] {líffærafræði}
\item[{angl.}] {enska}
\item[{ap.}] {e.þ.h.}
\item[{astro.}] {geimvísindi, stjörnufræði}
\item[{básn.}] {ljóðrænn}
\item[{biol.}] {líffræði}
\item[{bot.}] {grasafræði}

\item[{comp}] {miðstig}
\item[{con}] {viðtengingarháttur}
\item[{conj}] {samtenging}
\item[{dat}] {þágufall}
\item[{def}] {með greini}
\item[{dem}] {ábendingarfornafn}
\item[{dět.}] {barnamál}

\item[{e-að}] {(et. nf. ekki lifandi) (\textit{eitthvað})}
\item[{e-ð}] { (et. þf. ekki lifandi) (\textit{eitthvað})}
\item[{e-ir}] { (ft. nf. lifandi) (\textit{einhverjir})}
\item[{e-ja}] {(ft. þf. lifandi) (\textit{einhverja})}
\item[{e-jum}] {(ft. þgf. lifandi) (\textit{einhverjum})}
\item[{e-m}] {(et. þgf. lifandi) (\textit{einhverjum})}
\item[{e-n}] {(et. þf. lifandi) (\textit{einhvern})}
\item[{e-r}] {(et. nf. lifandi) (\textit{einhver})}
\item[{e-rra}] {(ft. ef. lifandi) (\textit{einhverra})}
\item[{e-rs}] {(et. ef. lifandi) (\textit{einhvers})}
\item[{e-s}] {(et. ef. ekki lifandi) (\textit{einhvers})}
\item[{e-u}] {(et. þgf. ekki lifandi) (\textit{einhverju})}

\item[{ekon.}] {hagfræði, viðskipti}
\item[{elek.}] {rafmagn}
\item[{f}] {kvenkyn}
\item[{filos.}] {heimspeki, rökfræði}
\item[{form.}] {formlegt mál}
\item[{fyz.}] {eðlisfræði}
\item[{gen}] {eignarfall}
\item[{geog.}] {landafræði}
\item[{geol.}] {jarðfræði}
\item[{han.}] {niðrandi}
\item[{hist.}] {sagnfræði}
\item[{hovor.}] {talmál}
\item[{hrub.}] {gróft}
\item[{hud.}] {tónlist}
\item[{chem.}] {efnafræði}
\item[{imper}] {boðháttur}
\item[{impers}] {ópersónuleg sögn}
\item[{ind}] {framsöguháttur}
\item[{indecl}] {óbeygjanlegur}
\item[{indef}] {óákveðinn, (pron +) óákveðið fornafn}
\item[{inf}] {nafnháttur}
\item[{int}] {spurnarfornafn}
\item[{inter}] {upphrópun}
\item[{jaz.}] {málfræði, málvísindi}
\item[{kulin.}] {matreiðsla}
\item[{l.}] {latína}
\item[{let.}] {flugmál}
\item[{lit.}] {skáldskapur}
\item[{m}] {karlkyn}
\item[{mat.}] {stærðfræði}
\item[{med. }] {læknisfræði}
\item[{meteo.}] {veðurfræði}
\item[{myt.}] {goðafræði}
\item[{n}] {hvorugkyn}
\item[{náb.}] {trúarbrögð}
\item[{nám.}] {sjómennska}
\item[{nom}] {nefnifall}
\item[{num}] {töluorð}
\item[{ord}] {raðtala}
\item[{p}] {persóna}
\item[{part}] {ögn}
\item[{pers}] {(v +) persóna, (pron +) persónufornafn}
\item[{pf}] {þátíð}
\item[{pl}] {fleirtala}
\item[{poč.}] {tölvufræði}
\item[{pol.}] {stjórnmál, stjórnmálafræði}
\item[{pos (s\,/\addthin w)}] {frumstig (sterk\,/\addthin veik beyging)}
\item[{poss}] {eignarfornafn}
\item[{pov.}] {þjóðtrú}
\item[{pp}] {lýsingarháttur þátíðar}
\item[{praes}] {nútíð}
\item[{práv.}] {lögfræði, dómsmál}
\item[{predp}] {forskeyti}
\item[{prep}] {forsetning}
\item[{presp}] {lýsingarháttur nútíðar}
\item[{pron}] {fornafn}
\item[{prop}] {sérnafn}
\item[{přen.}] {afleidd merking}
\item[{přís.}] {málsháttur}
\item[{psych.}] {sálfræði}
\item[{refl}] {(v +) miðmynd, (pron +) afturbeygt fornafn}
\item[{rel}] {tilvísunarfornafn}
\item[{sg}] {eintala}
\item[{slang.}] {slangur}
\item[{sport.}] {íþróttir}
\item[{stav.}] {húsasmíði, byggingarlist}
\item[{subs}] {nafnorð (án kyns)}
\item[{sup (s\,/\addthin w)}] {efstastig (sterk\,/\addthin veik beyging)}
\item[{supin}] {sagnbót}
\item[{škol.}] {skólamál}
\item[{techn.}] {tækni, verkfræði}
\item[{v}] {sagnorð}
\item[{voj.}] {hermál}
\item[{zast.}] {úrelt mál}
\item[{zkr}] {skammstöfun}
\item[{zool.}] {dýrafræði}

\item[{;}] {semíkomma í miðri skýringu aðgreinir setningar}

\item[{(...)}] {innan sviga eru viðbótarupplýsingar}
\item[{[...]}] {innan hornklofa er hljóðritaður framburður}
\item[{|}] {merkir stað, þar sem beygingarending bætist við stofn}
\item[{·}] {aukasamskeyti samsetts orðs}
\item[{··}] {aðalsamskeyti samsetts orðs}
\item[{+}] {plús}
\item[{/}] {skástrik táknar fleiri möguleika}
\item[{\dicsymSee}] {ör vísar til uppflettiorðs í orðabókinni}
\item[{\dicsymCompare}] {ör vísar til uppflettiorðs í orðabókinni til samanburðar}
\item[{\dicsymExampleIS}] {kemur á undan dæmi á íslensku}
\item[{\dicsymExampleCS}] {kemur á undan tékkneskri þýðingu á dæminu}
\item[{\dicsymIdiom}] {kemur á undan orðasambandi}
\item[{\dicsymFrequent}] {orð með háa tíðni}

\item[{\dicsymPhoto}] {ljósmynd}
\item[{\dicsymProverb}] {málsháttur}
