\item[{abb}] {zkrácený}
\item[{acc}] {4. pád, akuzativ}
\item[{adj}] {přídavné jméno}
\item[{adv}] {příslovce}
\item[{akt}] {rod činný, aktivum}
\item[{anat.}] {anatomie}
\item[{angl.}] {anglicky}
\item[{ap.}] {a podobně}
\item[{astro.}] {astronomie, výzkum vesmíru}
\item[{básn.}] {básnický, poetický výraz}
\item[{biol.}] {biologie}
\item[{bot.}] {botanika}

\item[{comp}] {2. stupeň stupňování, komparativ}
\item[{con}] {spojovací způsob, konjunktiv}
\item[{conj}] {spojka}
\item[{dat}] {3. pád, dativ}
\item[{def}] {určitý tvar}
\item[{dem}] {zájmeno ukazovací}
\item[{dět.}] {dětsky}

\item[{e-að}] {co (sg nom neživotný) (\textit{eitthvað})}
\item[{e-ð}] {co (sg acc neživotný) (\textit{eitthvað})}
\item[{e-ir}] {kdo (pl nom životný) (\textit{einhverjir})}
\item[{e-ja}] {koho (pl acc životný) (\textit{einhverja})}
\item[{e-jum}] {komu (pl dat životný) (\textit{einhverjum})}
\item[{e-m}] {komu (sg dat životný) (\textit{einhverjum})}
\item[{e-n}] {koho (sg acc životný) (\textit{einhvern})}
\item[{e-r}] {kdo (sg nom životný) (\textit{einhver})}
\item[{e-rra}] {koho (pl gen životný) (\textit{einhverra})}
\item[{e-rs}] {koho (sg gen životný) (\textit{einhvers})}
\item[{e-s}] {čeho (sg gen neživotný) (\textit{einhvers})}
\item[{e-u}] {čemu (sg dat neživotný) (\textit{einhverju})}

\item[{ekon.}] {ekonomika, obchod}
\item[{elek.}] {elektřina}
\item[{f}] {rod ženský}
\item[{filos.}] {filosofie, logika}
\item[{form.}] {formálně}
\item[{fyz.}] {fyzika}
\item[{gen}] {2. pád, genitiv}
\item[{geog.}] {zeměpis, geografie}
\item[{geol.}] {geologie}
\item[{han.}] {hanlivě, pejorativně}
\item[{hist.}] {historie}
\item[{hovor.}] {hovorově}
\item[{hrub.}] {hrubě}
\item[{hud.}] {hudba}
\item[{chem.}] {chemie}
\item[{imper}] {rozkazovací způsob, imperativ}
\item[{impers}] {sloveso neosobní}
\item[{ind}] {oznamovací způsob, indikativ}
\item[{indecl}] {nesklonný}
\item[{indef}] {neurčitý tvar, (pron +) zájmeno neurčité}
\item[{inf}] {infinitiv}
\item[{int}] {tázací zájmeno}
\item[{inter}] {citoslovce}
\item[{jaz.}] {jazykověda}
\item[{kulin.}] {kulinářství, vaření}
\item[{l.}] {latinsky}
\item[{let.}] {letectví}
\item[{lit.}] {literatura, vydavatelství}
\item[{m}] {rod mužský}
\item[{mat.}] {matematika}
\item[{med. }] {lékařství}
\item[{meteo.}] {meteorologie}
\item[{myt.}] {mytologie}
\item[{n}] {rod střední}
\item[{náb.}] {náboženství}
\item[{nám.}] {námořnictví, rybolov}
\item[{nom}] {1. pád, nominativ}
\item[{num}] {číslovka}
\item[{ord}] {řadová číslovka}
\item[{p}] {osoba}
\item[{part}] {částice}
\item[{pers}] {(v +) osoba, (pron +) zájmeno osobní}
\item[{pf}] {minulý čas}
\item[{pl}] {množné číslo, plurál}
\item[{poč.}] {informatika}
\item[{pol.}] {politika, politologie}
\item[{pos (s\,/\addthin w)}] {1. stupeň stupňování, pozitiv (silné\,/\addthin slabé skloňování)}
\item[{poss}] {zájmeno přivlastňovací}
\item[{pov.}] {lidové pověsti, folkloristika}
\item[{pp}] {příčestí minulé}
\item[{praes}] {přítomný čas}
\item[{práv.}] {právnictví, soudnictví}
\item[{predp}] {předpona}
\item[{prep}] {předložka}
\item[{presp}] {příčestí přítomné}
\item[{pron}] {zájmeno}
\item[{prop}] {vlastní jméno, proprium}
\item[{přen.}] {přeneseně}
\item[{přís.}] {přísloví}
\item[{psych.}] {psychologie}
\item[{refl}] {(v +) médium, (pron +) zvratné zájmeno}
\item[{rel}] {vztažné zájmeno}
\item[{sg}] {jednotné číslo, singulár}
\item[{slang.}] {slang}
\item[{sport.}] {sport}
\item[{stav.}] {stavebnictví, architektura}
\item[{subs}] {podstatné jméno (bez rodu)}
\item[{sup (s\,/\addthin w)}] {3. stupeň stupňování, superlativ (silné\,/\addthin slabé skloňování)}
\item[{supin}] {supinum}
\item[{škol.}] {školství}
\item[{techn.}] {technika, mechanika}
\item[{v}] {sloveso}
\item[{voj.}] {vojenství}
\item[{zast.}] {zastarale}
\item[{zkr}] {zkratka}
\item[{zool.}] {zoologie}

\item[{;}] {středník uprostřed definice odděluje věty}

\item[{(...)}] {v kulatých závorkách jsou uvedeny doplňkové informace}
\item[{[...]}] {v hranatých závorkách je uveden fonetický zápis výslovnosti}\footnote{Vysvětlení fonetických symbolů je uvedeno v kapitole Seznam islandských fonémů na straně \pageref{sec:phon_phonems}.}
\item[{|}] {naznačuje místo, kde se deklinační\,/\addthin konjugační koncovka připojuje ke slovnímu základu}
\item[{·}] {vedlejší dělení složeného slova}
\item[{··}] {hlavní dělení složeného slova}
\item[{+}] {plus}
\item[{/}] {lomítko označuje více možností}
\item[{\dicsymSee}] {šipka odkazuje na heslo ve slovníku}
\item[{\dicsymCompare}] {šipka odkazuje na heslo ve slovníku ke srovnání}
\item[{\dicsymExampleIS}] {uvozuje islandský příklad}
\item[{\dicsymExampleCS}] {uvozuje český překlad příkladu}
\item[{\dicsymIdiom}] {uvozuje slovní spojení}
\item[{\dicsymFrequent}] {heslo patří mezi frekventované výrazy}

\item[{\dicsymPhoto}] {fotografie, ilustrace}
\item[{\dicsymProverb}] {přísloví, rčení}
