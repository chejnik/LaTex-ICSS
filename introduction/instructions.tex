\def\tableA{\noindent\begin{tabularx}{\columnwidth}{>{\bfseries}lll}
  \toprule
  \textit{přípona} & \textit{přípona s~koncovkou} & \textit{příklad} \\
  \midrule
  -ing-   & -ingur   & rugl··ingur \\
          & -ing     & hreyf··ing \\
          & -ingi    & erf··ingi \\
          & -ingar   & lækn··ingar \\
  -ling-  & -lingur  & vett··lingur \\
  -ning-  & -ningur  & flut··ningur \\
          & -ning    & set··ning \\
          & -ningar  & kos··ningar \\
  -að-    & -aður    & mark··aður \\
  -nað-   & -naður   & skil··naður \\
  -ar-    & -ari     & dóm··ari \\
  -skap-  & -skapur  & kveð··skapur \\
  -dóm-   & -dómur   & lær··dómur \\
  -leik-  & -leikur  & sann··leikur \\
          & -leiki   & mögu··leiki \\
  -lát-   & -látur   & laus··látur \\
  -átt-   & -átta    & víð··átta \\
  -sam-   & -samur   & starf··samur \\
  -sem-   & -semi    & hamingju··semi \\
  -nesk-  & -neskur  & tékk··neskur \\
          & -neska   & tékk··neska \\
  -leg-   & -legur   & blíð··legur \\
          & -lega    & blíð··lega \\
  -ung-   & -ungur   & gler··ungur \\
          & -unga    & snopp··unga \\
          & -ungi    & sveit··ungi \\
          & -ung     & nýj··ung \\
  -ótt-   & -óttur   & blett··óttur \\
  -ræn-   & -rænn    & suð··rænn \\
          & -ræna    & nor··ræna \\
  -ern-   & -erni    & fað··erni \\
  -ind-   & -indi    & vís··indi \\
  -and-   & -andi    & gef··andi \\
  -ug-    & -ugur    & göf··ugur \\
          & -ugt     & stöð··ugt \\
  -ust-   & -usta    & þjón··usta \\
  -ald-   & -aldur   & far··aldur \\
  -uð-    & -uður    & söfn··uður \\
  -úð-    & -úð      & dul··úð \\
  -versk- & -verskur & ung··verskur \\
          & -verska  & ung··verska \\
  -ist-   & -isti    & hass··isti \\
  -ism-   & -ismi    & húman··ismi \\
  -ang-   & -angur   & far··angur \\
  \bottomrule
  \end{tabularx}
  \captionof{table}{Seznam přípon označených ve slovníku}
  \label{table1}}

\def\tableB{\noindent\begin{tabularx}{\columnwidth}{>{\bfseries}lX}
  \toprule
  \textbf{symbol} & \textbf{vysvětlivka} \\
  \midrule
  m                   & označuje slovní druh \\
  \textsubscript{1}   & označuje třídu \\
  (-a, -ar)           & označuje deklinační / konjugační koncovky hesla \\
  \textsubscript{8}   & označuje podtřídu\\
  \bottomrule
  \end{tabularx}
  \captionof{table}{Morfologická informace}
  \label{table2}}

\def\tableC{\noindent\begin{tabularx}{\columnwidth}
    {>{\bfseries}l>{\footnotesize\itshape}lllll}
  \toprule
  \multirow{3}{*}{{{\textbf{f{\textsubscript{1}}} \Large{\textbf{7}}}}}
    & nom & kisa & kisan    & kisur & kisurnar \\
    & acc & kisu & kisuna   & kisur & kisurnar \\
    & dat & kisu & kisunni  & kisum & kisunum \\
    & gen & kisu & kisunnar & kisa  & kisanna \\
  \bottomrule
  \end{tabularx}
  \captionof{table}{Skloňování hesla \textit{kisa}}
  \label{table3}}

\def\tableD{\noindent\begin{tabularx}{\columnwidth}{>{\bfseries}lllll}
  \toprule
  \textit{zkratka} & \textit{slovo} & \textit{pád} & \textit{číslo}
    & \specialcell{\textit{předmět}\\\textit{podmět}} \\
  \midrule
  e-að  & eitthvað   & 1. & sg & neživotný \\
  e-ð   & eitthvað   & 4. & sg & neživotný \\
  e-ir  & einhverjir & 1. & pl & životný \\
  e-ja  & einhverja  & 4. & pl & životný \\
  e-jum & einhverjum & 3. & pl & životný \\
  e-m   & einhverjum & 3. & sg & životný \\
  e-n   & einhvern   & 4. & sg & životný \\
  e-r   & einhver    & 1. & sg & životný \\
  e-rra & einhverra  & 2. & pl & životný \\
  e-rs  & einhvers   & 2. & sg & životný \\
  e-s   & einhvers  & 2. & sg & neživotný \\
  e-u   & einhverju  & 3. & sg & neživotný \\
  \bottomrule
  \end{tabularx}
  \captionof{table}{Seznam použitých zkratek neurčitých zájmen}
  \label{table4}}

%#############################################################################

\section{Heslo}

\subsection*{Řazení slov}

Heslová slova jsou zobrazena tučně a~jsou seřazena podle islandské abecedy:

\blspace
{\bfseries\centering
  a~á b c d ð e é f g h i~í j k~l m n\\
  o~ó p q r s~t u~ú v~w x y ý z~þ æ ö\par}
\blspace

K~snadnější orientaci v~heslové části je použit u~každého písmene tzv. palcový index, který se nachází na hřbetě slovníku. Každé písmeno je v~palcovém indexu barevně odlišeno.

\subsection*{Homonyma}

Homonyma (heslová slova stejného reprezentativního tvaru) jsou označena povýšenými arabskými číslicemi uváděnými za heslovým slovem.

\blspace
  \dicEntry \dicTerm{vor\textsuperscript{1}} \dicPos{n} \dicFlx{(-s, -)}
  \dicEntry \dicTerm{vor\textsuperscript{2}} \dicPos{pron} \dicFlx{pers gen pl}
\blspace

Homonyma jsou seřazena podle českého ustáleného řazení slovních druhů. Poté následují předpony a~zkratky.
V~případě dvou podstatných jmen je pořadí mužský rod, ženský rod, střední rod. V~případě dvou podstatných jmen stejného rodu má přednost slovo se slabým skloňováním před slovem se silným skloňováním. V~případě dvou sloves stejného reprezentativního tvaru má přednost sloveso se slabým časováním před slovesem se silným časováním.

\subsection*{Varianty a~dublety}

U~některých heslových slov jsou v~záhlaví hesla uvedeny a) varianty (obměny)

\blspace
  \dicEntry \dicTerm{ærsla··fenginn}, \dicTerm{ærslafullur} \dicPos{adj}
\blspace

nebo b) dublety (dvojtvary).

\blspace
  \dicEntry \dicTerm{{ævin··týri}, \dicTerm{ævintýr}} \dicPos{n}
\blspace

Varianty a~dublety jsou méně frekventované než samotné heslo. Lze je vyhledat ve slovníku jako hesla obsahující fonetické a~morfologické informace s~odkazem na původní heslové slovo.

\blspace
  \dicEntry \dicTerm{ærsla··|fullur} \dicIPA{{a}{i}{\textsubring{r}}{s}{\textsubring{d}}{l}{a}{f}{\textscy}{\textsubring{d}}{l}{\textscy}{\textsubring{r}}} \dicPos{adj} \dicFlx{(comp -fyllri, sup -fyllstur)} \dicLink{ærslafenginn}
  \dicEntry \dicTerm{ævin··týr} \dicIPA{{a}{i}{\textlengthmark}{v}{\textsci}{n}{t\textsuperscript{h}}{i}{\textsubring{r}}} \dicPos{n} \dicFlx{(-s, -)} \dicLink{ævintýri}
\blspace

\subsection*{Vybrané tvary jako hesla}

Ve slovníku jsou uvedeny vybrané tvary heslových slov, u~kterých by mohl mít uživatel potíže s~identifikací. Jedná se o~a) genitiv jednotného čísla ({\textbf{sg}}) a~nominativ množného čísla ({\textbf{pl}}) podstatných jmen, b) ženský a~střední rod přídavných jmen a~2. a~3. stupeň rodu mužského přídavných jmen, c) oznamovací způsob ({\textbf{ind}}) přítomného času ({\textbf{praes}}) a~minulého času ({\textbf{pf}}) 1. osoby ({\textbf{pers}}) sg a~pl a~spojovací způsob ({\textbf{con}}) minulého času 1. osoby sg a~jiné nepravidelné tvary.

\blspace
  \dicEntry \dicTerm{öttum} \dicPos{v} \dicFlx{ind pf pl 1 pers} \dicLink{etja}
\blspace

\subsection*{Dělení heslových slov}

K~dělení heslových slov je použita střední tečka \textbf{'·'}, která odděluje části složeného slova, přípony a~předpony. Platí následující pravidla:

\renewcommand{\labelenumi}{\alph{enumi})}
\begin{enumerate}
\item jediné dělení heslového slova je označeno dvěma středními tečkami \textbf{'··'}. Může se jednat o~dělení částí složeného slova
\blspace
\dicEntry \dicTerm{ráð··hús}
\blspace

nebo o~dělení mezi kořenem a~předponou / příponou.

\blspace
\dicEntry \dicTerm{þjón··usta}
\blspace
\item V~případě, že se v~hesle vyskytuje více dělení, jsou použity dvě střední tečky \textbf{'··'} pro významově poslední dělení a~jedna střední tečka \textbf{'·'} pro zbylé / zbylá dělení.
\blspace
\dicEntry \dicTerm{við·bótar··líf·eyris·sparn·aður}
\blspace
\item V~hesle s~předponou je označena jedna nebo i~více předpon (každá zvlášť).
\blspace
\dicEntry \dicTerm{ó··að·finnan·legur}
\blspace
\item V~hesle s~příponou je označena jedna nebo i~více přípon (každá zvlášť).
\blspace
\dicEntry \dicTerm{létt·úð··ugur}
\blspace
\item V~hesle s~příponou je označena pouze přípona / přípony poslední části složeného slova. Důvodem pro neoznačení přípon jiných částí složených slov je lepší čitelnost hesla. Příklad označení pouze přípony poslední části složeniny:
\blspace
\dicEntry \dicTerm{flutninga··miðl·ari}
\end{enumerate}

Následuje seznam přípon, které jsou ve slovníku označeny:

\tableA

\section{Frekvence slov}

Symbolem \dicsymFrequent\ je označeno 2~825 hesel patřících mezi velmi frekventovaná islandská slova. Tento seznam vychází z~korpusu \textit{Íslenskur Orðasjóður} (\cite{int7}) a~knihy \textit{Íslensk orðtíðnibók} (\cite{is2}).

\blspace
  \dicEntry \dicTerm{koma}\textsuperscript{2} \dicsymFrequent
\blspace

\section{Fonetický zápis}

V~hranatých závorkách v~záhlaví hesla je uveden fonetický zápis výslovnosti hesla v~mezinárodní fonetické abecedě (IPA, angl. International Phonetic Alphabet).

\blspace
  \dicEntry \dicTerm{öldu··rót} \dicIPA{{\oe}{l}{\textsubring{d}}{\textscy}{r}{ou}{\textsubring{d}}}
\blspace

Znak '{\textlengthmark}' označuje dlouhou samohlásku nebo souhlásku.

\blspace
  \dicEntry \dicTerm{ör}\textsuperscript{2} \dicIPA{{\oe}{\textlengthmark}{\textsubring{r}}}
\blspace

\section{Slovní druhy, předpony a~zkratky}

V~záhlaví hesla je uvedena morfologická informace popisující a) slovní druh hesla, b) předponu nebo c) zkratku. Pokud se heslo řadí do více slovních druhů, jsou jednotlivé zkratky pro slovní druhy odděleny lomítkem.

\blspace
  \dicEntry \dicTerm{af} \dicPos{prep / adv}
\blspace

Informace o~slovním druhu, předponě či zkratce je vysázena barevně a~tučně.

\subsection*{Podstatná jména}

Podstatná jména jsou uvedena v~1. pádě (\textbf{nom}) jednotného čísla (\textbf{sg}) nebo v~1. pádě množného čísla (\textbf{pl}), pokud se heslo v~jednotném čísle nevyskytuje. Podstatná jména jsou označena následovně: \textbf{m} (l. masculinum) - podstatné jméno rodu mužského,
\textbf{f} (l. femininum) - podstatné jméno rodu ženského, \textbf{n} (l. neutrum) - podstatné jméno rodu středního, \textbf{subs} (l. substantivum) - podstatné jméno bez jednoznačného mluvnického rodu.

\blspace
  \dicEntry \dicTerm{hestur} \dicPos{m}
  \dicEntry \dicTerm{Hanoí} \dicPos{subs}
\blspace

\subsection*{Přídavná jména}

Přídavná jména jsou uvedena v~1. pádě (\textbf{nom}) rodu mužského (\textbf{m}) jednotného čísla (\textbf{sg}) 1. stupně (\textbf{pos}) (l. positivus) a~jsou označena zkratkou \textbf{adj} (l. adjectivum).

\blspace
  \dicEntry \dicTerm{fallegur} \dicPos{adj}
\blspace

V~případě, že u~přídavného jména dochází v~rodě ženském (\textbf{f}) nebo středním (\textbf{n}) k~přehlásce anebo je tvar druhého stupně (\textbf{comp} (l. comparativus)) či třetího stupně (\textbf{sup} (l. superlativus)) nepravidelný, pak jsou tyto tvary uvedeny v~nominativu jako heslová slova s~odkazem na hlavní heslový tvar

\blspace
  \dicEntry \dicTerm{yngri} \dicPos{adj} \dicFlx{comp m} \dicLink{ungur}
\blspace

nebo jako svébytná hesla s~významy.

\blspace
  \dicEntry \dicTerm{fremri} \dicPos{adj} \dicFlx{comp (sup fremstur)}
\blspace

\subsection*{Zájmena}

Zájmena jsou označena zkratkou \textbf{pron} (l. pronomen) a~jsou tříděna na zájmena ukazovací \textbf{dem} (angl. demonstrative), přivlastňovací \textbf{poss} (angl. possessive), osobní \textbf{pers} (angl. personal), neurčitá \textbf{indef} (angl. indefinite), zvratná \textbf{refl} (angl. reflexive) a~vztažná \textbf{rel} (angl. relative).

\blspace
  \dicEntry \dicTerm{þessi} \dicPos{pron} \dicFlx{dem}
\blspace

U~zájmen přivlastňovacích je v~3. osobě označen rod.

\blspace
  \dicEntry \dicTerm{hans} \dicPos{pron} \dicFlx{poss m}
\blspace

U~zájmen osobních je označen pád a~číslo.

\blspace
  \dicEntry \dicTerm{mig} \dicPos{pron} \dicFlx{pers acc sg}
\blspace

\subsection*{Číslovky}

Číslovky jsou označeny zkratkou \textbf{num} (l. numerale). V~případě vybraných číslovek (1 až 4) jsou ve slovníku uvedeny jako heslová slova také tvary ženského i~středního rodu. V~takovém případě jsou označeny zkratkou \textbf{m} pro mužský rod, zkratkou \textbf{f} pro ženský rod a~zkratkou \textbf{n} pro střední rod. Zkratkou \textbf{ord} (angl. ordinal number) jsou označeny číslovky řadové.

\blspace
  \dicEntry \dicTerm{tut··tugu} \dicPos{num}
  \dicEntry \dicTerm{tveir} \dicPos{num} \dicFlx{pl m}
  \dicEntry \dicTerm{fyrsti} \dicPos{num} \dicFlx{ord}
\blspace

\subsection*{Slovesa}

Slovesa jsou uvedena v~infinitivu a~jsou označena zkratkou \textbf{v} (l. verbum).

\blspace
  \dicEntry \dicTerm{fara} \dicPos{v}
\blspace

Slovesa ve středním rodě jsou označena jako \textbf{refl} (angl. reflexive).

\blspace
  \dicEntry \dicTerm{nálgast} \dicPos{v} \dicFlx{refl}
\blspace

Slovesa, která jsou vždy neosobní, jsou označena \textbf{impers} (angl. impersonal).

\blspace
  \dicEntry \dicTerm{svima} \dicPos{v} \dicFlx{impers}
\blspace

U~sloves je dále uvedeno, s~jakým pádem se pojí. Zkratka \textbf{acc} označuje 4. pád, \textbf{dat} 3. pád, \textbf{gen} 2. pád a~\textbf{nom} 1. pád. V~případě, že se sloveso pojí s~různými pády, je použito lomítko.

\blspace
  \dicEntry \dicTerm{klóra} \dicPos{v} \dicFlx{acc / dat}
\blspace

V~případě, že se sloveso pojí s~více pády, je použito znaménko plus.

\blspace
  \dicEntry \dicTerm{gefa} \dicPos{v} \dicFlx{dat + acc}
\blspace

\subsection*{Příslovce}

Příslovce jsou označena zkratkou \textbf{adv} (l. adverbium). Pokud chybí informace o~stupni, jedná se o~1. stupeň stupňování (\textbf{pos}). 2. stupeň je vždy označen zkratkou \textbf{comp} a~3. stupeň zkratkou \textbf{sup}.

\blspace
  \dicEntry \dicTerm{þar} \dicPos{adv}
  \dicEntry \dicTerm{ofar} \dicPos{adv} \dicFlx{comp (pos uppi, sup efst)}
  \dicEntry \dicTerm{síðast} \dicPos{adv} \dicFlx{sup (comp síðar)}
\blspace

\subsection*{Předložky}

Předložky jsou označeny zkratkou \textbf{prep} (l. praepositio). Pokud se předložka pojí výhradně s~jedním pádem, jsou použity zkratky pro 4. pád (\textbf{acc}), 3. pád (\textbf{dat}), 2. pád (\textbf{gen}).

\blspace
  \dicEntry \dicTerm{af} \dicPos{prep / adv} \dicFlx{dat}
\blspace

V~případě, že se předložka pojí s~různými pády, je použito lomítko.

\blspace
  \dicEntry \dicTerm{fyrir} \dicPos{prep / adv} \dicFlx{acc / dat}
\blspace

\subsection*{Spojky}

Spojky jsou označeny zkratkou \textbf{conj} (l. conjunctio).

\subsection*{Částice}

Částice jsou označeny zkratkou \textbf{part} (l. particula).

\subsection*{Citoslovce}

Citoslovce jsou označeny zkratkou \textbf{inter} (l. interjectio).

\subsection*{Předpony}

Předpony jsou označeny zkratkou \textbf{predp}.

\subsection*{Zkratky}

Zkratky jsou označeny zkratkou \textbf{zkr}.

\section{Skloňování, časování a~stupňování}

Každé heslo obsahuje zápis morfologické informace, která přesně informuje, jak se heslo skloňuje / časuje nebo stupňuje.

V~slovníku jsou použity dva odlišné systémy zápisu morfologické informace, které se vzájemně doplňují:

\begin{enumerate}
\item tradiční systém - do závorky za slovním druhem se zapisují deklinační / konjugační koncovky. Počet koncovek závisí na slovním druhu (viz dále).
\item numerický systém (vychází ze systému Helgi Haraldssona (\cite {is6})) - číslice odkazující na morfologický klíč, kde jsou podrobně vypsány deklinační / konjugační třídy.
Další informace o~numerickém systému lze nalézt na straně \pageref{sec:dec}.
\end{enumerate}

Deklinační / konjugační tvary nebo koncovky podstatných jmen, přídavných jmen, zájmen, číslovek, sloves a~příslovcí jsou uvedeny v~závorkách za slovním druhem.

Hnízdovací šev \textbf{'|'} označuje místo ve slově, kde se deklinační koncovka připojuje ke slovu.

\blspace
  \dicEntry \dicTerm{heils|a{\textsuperscript{1}}} \dicPos{f} \dicFlx{(-u)}
\blspace

Pokud není hnízdovací šev \textbf{'|'} uveden, tak se koncovka připojuje k~samotnému heslu.

\blspace
  \dicEntry \dicTerm{mynd} \dicPos{f} \dicFlx{(-ar, -ir)}
\blspace

Pokud dochází v~ohýbání k~přehlásce nebo je ohýbání nepravidelné, pak jsou uvedeny celé tvary hesla.

\blspace
  \dicEntry \dicTerm{maður} \dicPos{m} \dicFlx{(manns, menn)}
\blspace

V~případě, že heslové slovo má více tvarů, jsou varianty koncovky odděleny lomítkem.

\blspace
  \dicEntry \dicTerm{beð|ur} \dicPos{m} \dicFlx{(-s / -jar, -ir)}
\blspace

\subsection*{Podstatná jména}

U~podstatných jmen je uveden tvar hesla pro 2.~pád jednotného čísla a~1.~pád množného čísla.

\blspace
  \dicEntry \dicTerm{hval|ur} \dicPos{m} \dicFlx{(-s, -ir)}
\blspace

kde '\textit{hvals}' je 2.~pád jednotného čísla a~'\textit{hvalir}' 1.~pád množného čísla.
Pokud je uveden pouze jeden tvar, jedná se o~2.~pád jednotného čísla a~znamená to rovněž, že heslové slovo se neohýbá v~množném čísle.

\blspace
  \dicEntry \dicTerm{heils|a{\textsuperscript{1}}} \dicPos{f} \dicFlx{(-u)}
\blspace

Pokud je použita zkratka pro množné číslo (\textbf{pl}), znamená to, že se heslo ohýbá pouze v~množném čísle.

\blspace
  \dicEntry \dicTerm{afar··kostir} \dicPos{m} \dicFlx{pl}
\blspace

Pokud je použita zkratka \textbf{indecl}, znamená to, že podstatné jméno je nesklonné.

\blspace
  \dicEntry \dicTerm{október} \dicPos{m} \dicFlx{indecl}
\blspace

\subsection*{Přídavná jména}

U~přídavných jmen jsou uvedeny v~závorkách tvary, které poukazují na nepravidelnosti ve skloňování.

Pokud přídavné jméno v~ženském rodě neztrácí koncové '\textit{-r}', je uveden tvar hesla v~ženském rodě.

\blspace
  \dicEntry \dicTerm{hýr} \dicPos{adj} \dicFlx{(f -)}
\blspace

Dále jsou uvedeny tvary ženského rodu, ve~kterých dochází k~přehlásce:

\blspace
  \dicEntry \dicTerm{reglu··|samur} \dicPos{adj} \dicFlx{(f -söm)}
\blspace

Pokud je přídavné jméno nesklonné, je uvedena zkratka \textbf{indecl}.

\blspace
  \dicEntry \dicTerm{tví··mála} \dicPos{adj} \dicFlx{indecl}
\blspace

Pokud má přídavné jméno nepravidelné tvary 2.~a~3.~stupně stupňování, jsou tyto tvary uvedeny v~závorce spolu se zkratkou pro 2.~stupeň (\textbf{comp}) a~3.~stupeň (\textbf{sup}).

\blspace
  \dicEntry \dicTerm{þunnur} \dicPos{adj} \dicFlx{(comp þynnri, sup þynnstur)}
\blspace

V~případě, že je uveden pouze jeden ze dvou zbývajících stupňů, znamená to, že heslo se stupňuje pouze v~uvedených stupních.

\blspace
  \dicEntry \dicTerm{syðri} \dicPos{adj} \dicFlx{comp (sup syðstur)}
\blspace

\subsection*{Slovesa}

Islandská slovesa se dělí na slabá, silná a~nepravidelná. Vzhledem k~tomuto dělení se liší počet slovních tvarů uvedených v~závorce za zkratkou \textbf{v}. Princip uvádění slovních tvarů spočívá v~tom, že uživatel
je schopen (více méně) z~těchto tvarů odvodit časování slovesa. V~tomto slovníku jsme
záměrně umístili některá slabá slovesa mezi silná (místo dvou slovesných tvarů je uvedeno pět tvarů) kvůli tomu, aby byl uživatel přesněji informován o~časování slovesa.

Slabá slovesa, která tvoří 1.~osobu jednotného čísla minulého času oznamovacího způsobu koncovkou '\textit{-aði}', mají v~závorce uvedenu jen jednu koncovku, jmenovitě '\textit{(-aði)}'.

\blspace
  \dicEntry \dicTerm{ætl|a} \dicPos{v} \dicFlx{(-aði)}
\blspace

U~zbývajících skupin slabých sloves se v~závorce nachází koncovka 1.~osoby jednotného čísla minulého času oznamovacího způsobu a~supinum.

\blspace
  \dicEntry \dicTerm{kenn|a} \dicPos{v} \dicFlx{(-di, -t)}
\blspace

Silná nebo nepravidelná slovesa mají v~závorce vždy pět tvarů, jmenovitě 1.~osobu jednotného čísla přítomného času oznamovacího způsobu, 1.~osobu jednotného čísla minulého času oznamovacího způsobu, 1.~osobu množného čísla minulého času oznamovacího způsobu, 1.~osobu jednotného čísla minulého času spojovacího způsobu a~supinum.

\blspace
  \dicEntry \dicTerm{grípa} \dicPos{v} \dicFlx{(gríp, greip, gripum, gripi, gripið)}
\blspace

Silná nebo nepravidelná slovesa vyskytující se pouze v~neosobním užití mají v~závorce 4 tvary, jmenovitě 3.~osobu jednotného čísla přítomného času oznamovacího způsobu, 3.~osobu jednotného čísla minulého času oznamovacího způsobu, 3.~osobu jednotného čísla minulého času spojovacího způsobu a~supinum.

\blspace
  \dicEntry \dicTerm{kross··|bregða} \dicPos{v} \dicFlx{(-bregður, -brá, -brygði, -brugðið)}
\blspace

U~některých významů v~heslovém slově se vyskytují gramatické informace, které popisují chování daného významu.

Zkratka \textbf{impers} označuje, že se daný význam slovesa vyskytuje v~neosobním užití. Pomocí slovního spojení je vyjádřeno užití slovesa.

\blspace
  \dicEntry \dicTerm{batn|a} \dicPos{v} \dicFlx{(-aði)} \dicTerm{e-m batnar} \dicFlx{impers} (kdo) se uzdravuje, (kdo) se zotavuje, (komu) je lépe
\blspace

Zkratka \textbf{refl} označuje mediopasivní tvary slovesa.

\blspace
  \dicEntry \dicTerm{að··lag|a} \dicPos{v} \dicFlx{(-aði)} \dicIdiom{aðlagast} {\dicPhraseIS{aðlagast e-u}} \dicFlx{refl} {přizpůsobit se (čemu)}
\blspace

\subsection*{Příslovce}

U~příslovcí se v~záhlaví nachází tvary stupňování. Tyto tvary jsou uvedeny v~závorce spolu se zkratkou pro 2.~stupeň (\textbf{comp}) a~3.~stupeň (\textbf{sup}).

\blspace
  \dicEntry \dicTerm{vel\textsuperscript{2}} \dicPos{adv} \dicFlx{(comp betur, sup best)}
\blspace

Nepravidelné tvary 2.~a~3.~stupně jsou uvedeny buď jako hesla odkazující na základní tvar v~1.~stupni (\textbf{pos}), anebo jako svébytná hesla. V~obou případech jsou uvedeny v~závorce zbývající stupně.

\blspace
  \dicEntry \dicTerm{síst} \dicPos{adv} \dicFlx{sup (pos varla, comp síður)}
\blspace

Pokud některý ze stupňů uveden není, znamená to, že pro dané slovo neexistuje.

\blspace
  \dicEntry \dicTerm{síðar} \dicPos{adv} \dicFlx{comp (sup síðast)}
\blspace

\section{Morfologická informace} \label{sec:dec}

V~záhlaví hesla se kromě deklinačních / konjugačních koncovek nachází také morfologické numerické informace, které odkazují na morfologický klíč (viz strana \pageref{sec:morpho}), kde jsou podrobně vypsány kompletní deklinační / konjugační tabulky.

\blspace
  \dicEntry \dicTerm{ap|i} \dicPos{m}[1] \dicFlx{(-a, -ar)}[8]
\blspace

Morfologická informace je zapsána podle následujícího schématu: \textbf{m{\textsubscript{1}} (-a, -ar){\textsubscript{8}}}

\blspace
\tableB
\blspace

Deklinační / konjugační třídy a~jejich počet nejsou v~islandštině ustálené a~slouží pouze pro rozlišení odlišných skloňování / časování.

Názvy podtříd slouží pro rozlišení odlišných skloňování / časování.

\subsection*{Práce s~morfologickým klíčem}

Pro vyhledání kompletních deklinačních / konjugačních tabulek slouží následující postup, který demonstrujeme na ukázce:

\blspace
  \dicEntry \dicTerm{kis|a} \dicPos{f}[1] \dicFlx{(-u, -ur)}[7]
\blspace

Nejdříve určíme slovní druh (v~případě podstatných jmen také rod), v~tomto případě podstatné jméno rodu ženského, a~navštívíme morfologický klíč (str. \pageref{sec:morpho}) pro podstatná jména rodu ženského (str. \pageref{sec:morpho_f}).
Následně dohledáme třídu (v~tomto případě číslo {\textsubscript{\textbf{1}}}) a~následně dohledáme dodatečné numerické informace za závorkou (v~tomto případě číslo {\textsubscript{\textbf{7}}}).

\tableC

\section{Definice}

\subsection*{Základní tvar}

Základním tvarem definice je, že jednomu islandskému slovu odpovídá jeden nebo více českých významů.

\blspace
  \dicEntry \dicTerm{landa··fræð|i} \dicPos{f} \dicFlx{(-i)} \textCS{zeměpis, geografie}
\blspace

V~některých případech je v~české části použita závorka. V~prvním případě je závorka užívaná ke~zkrácení zápisu dvou slov ('\textit{pomoc}', '\textit{výpomoc}').

\blspace
  \dicEntry \dicTerm{að··stoð} \dicPos{f} \dicFlx{(-ar)} \textCS{(vý)pomoc, asistence}
\blspace

V~druhém případě se v~závorce nachází upřesňující informace, která zabraňuje dvojznačnosti.

\blspace
  \dicEntry \dicTerm{fíkj|a} \dicPos{f} \dicFlx{(-u, -ur)} \textCS{fík} {\footnotesize {\textCS{(plod)}}}
\blspace

V~mnoha případech je závorka v~české části použita pro příklad užití českého slova a~tím vymezení jeho významu.

\blspace
  \dicEntry \dicTerm{að·gengi··legur} \dicPos{adj} {přístupný, dostupný (informace ap.)}
\blspace

Lomítko \textbf{'/'} slouží v~české části významu k~oddělení více významů pojících se s~jedním nebo více slovy a~je použito v~české části následujícím způsobem:

\blspace
  \dicEntry \dicTerm{aðal··|gata} \dicPos{f} \dicFlx{(-götu, -götur)} \textCS{hlavní třída / ulice}
\blspace

Význam přečteme jako '\textit{hlavní třída, hlavní ulice}'.

\subsection*{Opis heslového slova}

V~případech, kdy islandské slovo nemá ekvivalent v~českém jazyce, jsou významy islandských slov opsány delším opisem a~zobrazeny menším písmem. Příkladem jsou slova z~lidových pověstí, kulinářské speciality nebo botanické pojmy.

\blspace
  \dicEntry \dicTerm{til··ber|i} \dicPos{m} \dicFlx{(-a, -ar)} {\footnotesize{pov.}} {\footnotesize{stvoření, které dojí krávy a~ovce hospodářů}}
\blspace

\subsection*{Slovní spojení}

V~definici hesla jsou kromě samotných významů uváděna také slovní spojení, která jsou sázena tučně. Po slovním spojení následuje překlad slovního spojení.

\blspace
  \dicEntry \dicTerm{lær|a} \dicPos{v} \dicFlx{(-ði, -t)} \textbf{læra e-ð utanbókar} {učit se (co) zpaměti, memorovat (co)}
\blspace

Závorky jsou použity na tu část slovního spojení, kterou je možné vypustit.

\blspace
  \dicEntry \dicTerm{ár\textsuperscript{1}} \dicPos{f} \dicFlx{(-ar, -ar)} \textbf{taka (of) djúpt í árinni} {\footnotesize{přen.}} {přehnat (co v~tvrzení)}
\blspace

Příklad můžeme přečíst jako '\textit{taka djúpt í árinni}' nebo '\textit{taka of djúpt í árinni}'.

Mezi slovní spojení v~tomto slovníku řadíme a) tvary heslového slova, b) příkladová spojení, c) ustálená slovní spojení a~d) rčení nebo přísloví.

\begin{enumerate}
\item Tvary heslového slova\\
Do této skupiny řadíme tvar množného čísla hesla (zkratka \textbf{pl}).
\blspace
\dicEntry \dicTerm{bót} \dicPos{f} \dicFlx{(-ar, bætur)} \textbf{bætur} \dicFlx{pl} {dávky, přídavky, příspěvky}\\

Dále do této skupiny řadíme tvar slovesa v~středním slovesném rodě (zkratka \textbf{refl}).\\

\dicEntry \dicTerm{finna} \dicPos{v} \dicFlx{(finn, fann, fundum, fyndi, fundið)} \dicIdiom{finnast} {\dicPhraseIS{finnast}} \dicFlx{refl} {být nalezen, najít se, nalézt se, nacházet se}
\blspace
\item Příkladová spojení\\
Příkladová spojení ilustrují komunikační chování heslových slov v~příslušných významech a~v~různých syntagmatech (např. sloveso s~určitou předložkou ap.). Jsou označena tučně modrou barvou a~odsazená pro snadnou orientaci jako svébytné heslo.
\blspace
\dicEntry \dicTerm{koma\textsuperscript{2}} \dicPos{v} \dicFlx{(kem, kom, komum, kæmi, komið)} \dicIdiom{koma}[að] {\dicPhraseIS{koma að e-m}} {přiblížit se ke (komu), přijít ke (komu)}
\blspace
\item Ustálená slovní spojení\\
Tento druh slovních spojení je charakterizován většinou přeneseným významem. Tato spojení jsou uváděna za posledním významem nebo příkladovým spojením a~jsou označená tučně modrou barvou a~odsazená pro snadnou orientaci jako svébytné heslo.
\blspace
\dicEntry \dicTerm{kast} \dicPos{n} \dicFlx{(-s, köst)} \dicIdiom{kast} {\dicPhraseIS{þegar til kastanna kemur}} {\footnotesize{přen.}} {když jde do tuhého}
\blspace
\item Rčení a~přísloví\\
Rčení a~přísloví se nacházejí vždy na konci definice a~jsou odlišená symbolem \dicsymProverb\ a~odsazená pro snadnou orientaci.
\blspace
\dicEntry \dicTerm{æfing} \dicPos{f} \dicFlx{(-ar, -ar)} \dicProverb{} \dicPhraseIS{Æfing skapar meistara.} {\footnotesize{přís.}} \textCS{Cvičení dělá mistra.}
\end{enumerate}

\section{Vícevýznamovost}

\subsection*{Vícevýznamová hesla}

Značné množství hesel má více než jeden význam. Významy jsou odlišeny arabskými číslicemi a~v~některých případech i~islandskými synonymy.

\blspace
  \dicEntry \dicTerm{land} \dicPos{n} \dicFlx{(-s, lönd)} {\textbf{1.}} \dicSynonym{þurrlendi} {souš, pevnina, země} {\textbf{2.}} \dicSynonym{árbakki} {břeh} {\textbf{3.}} \dicSynonym{ríki} {země, stát, vlast} {\textbf{4.}} \dicSynonym{landareign} {země, půda, pozemek}
\blspace

\subsection*{Vícevýznamová slovní spojení}

V~případě, že slovní spojení uvnitř definice má více významů, jsou jednotlivé významy odlišeny písmeny {\textbf{a., b., c.}} atd.

\blspace
  \dicEntry \dicTerm{drag} {\dicPos{n} \dicFlx{(-s, drög)} {\textbf{1.}} {mokřina, podmoklý terén} {\textbf{2.}} \textbf{drög} {\footnotesize{pl}} {\textbf{a.}} \dicSynonym{uppsprettur} {prameny (řeky ap.)} {\textbf{b.}} \dicSynonym{undirbúningur} {náčrt, návrh.}}
\blspace

\section{Řazení významů a~slovních spojení v~definici}
\longsectionskip

V~definici jsou nejdříve seřazeny významy slova a~odlišeny arabskými číslicemi.

\subsection*{V rámci významu}

Slovní spojení jsou uváděna v~rámci jednotlivého významu, pokud se s~ním významově vážou.

\blspace
  \dicEntry \dicTerm{tím|i} \dicPos{m} \dicFlx{(-a, -ar)} {\textbf{1.}} \dicSynonym{tíð} {čas, doba}; \textbf{í þann tíma} \dicFlx{adv} \dicSynonym{þá\textsuperscript{2}} {pak} {\textbf{2.}} \dicSynonym{rétt stund} {(správný) čas / moment}
\blspace

\subsection*{Za posledním očíslovaným významem}

Pokud se slovní spojení neváže k~žádnému významu, je řazeno za posledním očíslovaným významem. Ustálená slovní spojení jsou uvedena symbolem \dicsymIdiom.

\blspace
  \dicEntry \dicTerm{finna} \dicPos{v} \dicFlx{(finn, fann, fundum, fyndi, fundið)} {\textbf{3.}} \dicSynonym{skynja} {cítit, vnímat} \dicIdiom{finna}[að] {\dicPhraseIS{finna að e-u}} {kritizovat (co), nacházet chyby na (čem)}
\blspace

\subsection*{Způsob řazení slovních spojení}

Slovní spojení, která se nachází za posledním očíslovaným významem, jsou řazena podle předložky / příslovce / zájmena, se kterým se pojí. Pro snadnou orientaci v~rozsáhlých definicích je uvedeno heslové slovo nebo tvar heslového slova plus předložka / příslovce / zájmeno a~barevně odlišeno modrou barvou.

\blspace
  \dicEntry \dicTerm{koma\textsuperscript{2}} \dicPos{v} \dicFlx{(kem, kom, komum, kæmi, komið)} \dicIdiom{koma}[út] {\dicPhraseIS{koma út}} {vyjít (kniha ap.), objevit se (na trhu ap.)}
\blspace

Poté následují slovní spojení v~slovesném středním rodě řazená opět podle předložky / příslovce / zájmena, se kterým se pojí:

\blspace
  \dicEntry \dicTerm{koma\textsuperscript{2}} \dicPos{v} \dicFlx{(kem, kom, komum, kæmi, komið)} \dicIdiom{komast} {\dicPhraseIS{komast}} \dicFlx{refl} {dostat se, dostávat se}
\blspace

Poté následují ustálená slovní spojení, která nebylo možné zařadit do předchozích položek:

\blspace
  \dicEntry \dicTerm{koma\textsuperscript{2}} \dicPos{v} \dicFlx{(kem, kom, komum, kæmi, komið)} \dicIdiom{koma} {\dicPhraseIS{(komdu) sæll (og blessaður)}} {ahoj, nazdar, dobrý den}
\blspace

Nakonec jsou uvedena přísloví a~rčení označená symbolem \dicsymProverb\:

\blspace
  \dicEntry \dicTerm{eik} \dicPos{n} \dicFlx{(-ar / -ur, -ur)} \dicProverb{} \dicPhraseIS{Ekki / eigi fellur eik við fyrsta högg.} {\footnotesize{přís.}} {Jednou ranou dub nespadne.}
\blspace

\section{Synonyma a~antonyma / hesla ke~srovnání}
\longsectionskip

\subsection*{Synonyma}

Synonyma nebo synonymní slovní spojení jsou uvedeny před opisem významu hesla. Jsou umístěna v~závorkách. Islandská synonyma slouží a) islandskému uživateli k~rozlišení českých významů b) českému uživateli jako dodatečná informace o~významu slova.

\blspace
  \dicEntry \dicTerm{staða} \dicPos{f} \dicFlx{(stöðu, stöður)} {\textbf{1.}} \dicSynonym{ástand} {situace, stav, poměry} {\textbf{2.}} \dicSynonym{embætti} {post, (pracovní) místo / pozice (ve firmě ap.)}
\blspace

V~případě, že synonymum lze nalézt jako heslo ve slovníku, je použita šipka '\dicsymSee\ '.

\blspace
  \dicEntry \dicTerm{bygging} {\dicPos{f} \dicFlx{(-ar, -ar)} {\textbf{1.}} \dicSynonym{hús} {budova, stavba, stavení} {\textbf{2.}} \textit{(það að byggja)} {(vý)stavba} }
\blspace

\subsection*{Antonyma / Hesla ke~srovnání}

Antonyma či hesla ke~srovnání jsou uvedena za opisem významu heslového slova a~jsou označena šipkou '$\shortuparrow$'.

\blspace
  \dicEntry \dicTerm{kaldur} \dicPos{adj} \dicFlx{(f köld)} {\textbf{1.}} {studený, chladný} \textit{($\shortuparrow$ heitur)}
\blspace

\section{Oborové a~stylové charakteristiky}

K~upřesnění významu heslového slova slouží oborové a~stylové charakteristiky.

\subsection*{Oborové charakteristiky}

Oborové charakteristiky zařazují islandské slovo do určitého oboru a~naznačují použití daného slova v~jazyce

\blspace
  \dicEntry \dicTerm{berg} \dicPos{n} \dicFlx{(-s, -)} {\textbf{1.}} {\footnotesize{geol.}} {hornina, kámen} {\textbf{2.}} \dicSynonym{klöpp\textsuperscript{1}} {skalní stěna, skalisko}
\blspace

a~vymezují význam českého slova.

\blspace
  \dicEntry \dicTerm{mús} \dicPos{f} \dicFlx{(-ar, mýs)} {\textbf{1.}} {\footnotesize{zool.}} {myš} (\textit{l. Mus}) {\textbf{2.}} {\footnotesize{poč.}} {myš}
\blspace

\subsection*{Stylové charakteristiky}

Stylové charakteristiky jsou uváděny v~případě, kdy význam heslového slova je možné použít pouze při určité příležitosti nebo je význam jazykově zabarven.

\blspace
  \dicEntry \dicTerm{gaur} \dicPos{m} \dicFlx{(-s, -ar)} {\textbf{1.}} {\footnotesize{hovor.}} {chlápek, maník, člověk}
\blspace

\section{Syntax}

Syntaktické informace popisují:

\begin{enumerate}
\item zdali je sloveso tranzitivní nebo intranzitivní a~s~jakým(i) pádem (pády) se pojí, pokud je tranzitivní,
\item zdali je podmět / předmět životný či neživotný,
\item zdali je sloveso osobní či neosobní a~v~jakém pádu je podmět, pokud je sloveso neosobní,
\item zdali je podmět gramatický,
\item zdali je sloveso použito ve středním rodě,
\item nejčastější příklady, které se pojí se slovesem,
\item se kterými příslovci nebo předložkami se sloveso pojí.
\end{enumerate}

Syntaktické informace se ve slovníku vyskytují ve třech formách: a) informace uvedené zkratkou, b) teoretické příklady a~c) praktické příklady.

\subsection*{Syntaktické informace uvedené zkratkou}

Zde máme na mysli informace o~pádě (\textbf{nom, acc, dat, gen}), slovesu v~středním slovesném rodě (\textbf{refl}), neosobním slovesu (\textbf{impers}), čísle (\textbf{sg, pl}), rozkazovacím způsobu (\textbf{imper}) a~spojovacím způsobu (\textbf{con}).

\subsection*{Teoretické příklady}

Teoretickým příkladem nazýváme příklad, kdy podmět nebo předmět je zastoupen zájmenem neurčitým v~pádě, se~kterým se slovo pojí a~sloveso je zpravidla v~infinitivu. V~případě, že podmět není v~1.~pádě (např. u~sloves neosobních), je sloveso uvedeno v~3.~osobě jednotného čísla přítomného času.
Ve slovních spojeních používáme zkratky pro tvary neurčitého zájmena \textbf{einhver} (někdo) a~\textbf{eitthvað} (něco). Tyto zkratky označují a) pád, b) číslo a~c) životnost nebo neživotnost podmětu / předmětu.\\
Následuje seznam zkratek neurčitých zájmen použitých ve slovníku: \\

\tableD
\blspace
Příklad použití zkratek s~neurčitými zájmeny:

\blspace
  \dicEntry \dicTerm{leit|a} \dicPos{v} \dicFlx{(-aði)} \textbf{leita að e-u / e-m} {hledat (co / koho)} \dicExampleIS{leita að lyklunum} \dicExampleCS{hledat klíče}
\blspace

Příklad tedy přečteme jako '\textit{leita að einhverju}' (\textit{hledat co}) a~'\textit{leita að einhverjum}' (\textit{hledat koho}). V~tomto případě se sloveso pojí s~3.~pádem a~předmět může být životný i~neživotný.

\subsection*{Praktické příklady}

Ve slovníku se vyskytuje velké množství praktických příkladů a~jejich překladů. Praktický příklad se liší od příkladu teoretického tím, že místo zájmen neurčitých jsou uvedena frekventovaná slova z~běžného projevu. Praktické příklady slouží k~ilustraci použití
heslového slova.\\
Praktickým příkladem může být a) (složené) slovo, b) slovní spojení nebo c) věta.

\begin{enumerate}
\item Slovo jako praktický příklad:
\blspace
\dicEntry \dicTerm{hett|a} \dicPos{f} \dicFlx{(-u, -ur)} \textbf{1.} {kapuce, kápě} \dicExampleIS{hettuúlpa} \dicExampleCS{bunda s~kapucí}
\blspace
\item Slovní spojení jako praktický příklad:
\blspace
\dicEntry \dicTerm{tím|i} \dicPos{m} \dicFlx{(-a, -ar)} \textbf{3.} \dicSynonym{klukkustund} {hodina} {\footnotesize{(šedesát minut)}} \dicExampleIS{tveggja tíma gangur} \dicExampleCS{dvouhodinový pochod}
\blspace
\item Věta jako praktický příklad:
\blspace
\dicEntry \dicTerm{mæt|a} \dicPos{v} \dicFlx{(-ti, -t)} \dicIdiom{mætast} \dicPhraseIS{mætast} \dicSynonym*{ná saman} \dicFlx{refl} {setkat se, potkat se} \dicExampleIS{Þau mættust á gangstéttinni.} \dicExampleCS{Potkali se na chodníku.}
\end{enumerate}

\section{Fotografie a~ilustrace}

Ve slovníku je publikováno 630 fotografií rostlin, živočichů a~fotografií, která dokumentují místopisně heslová slova (např. kulinářské speciality).
Fotografie jsou publikovány pod veřejnými licencemi. Jména autorů a~názvy licencí lze nalézt v~Seznamu autorů fotografií a~ilustrací na straně \pageref{sec:photo}. U~hesla a~konkrétně u~daného významu, ke kterému se fotografie vztahuje, je použit symbol fotoaparátu \dicsymPhoto. Fotografie je umístěna buď pod heslem, anebo v~horní části stránky, kde se heslo nachází, anebo na stránce (stránkách) následující(ch).

%\par\begin{center}\setlength\fboxsep{0pt}\setlength\fboxrule{0.5pt}
%\fbox{\includegraphics[width=6cm]{ds_image_posthus_0_2.jpg}}\end{center}
%\par\begin{center}\footnotesize {Pósthús}\end{center}
