Cílem vytvoření\textit{ Islandsko-českého studijního slovníku} je zpřístupnění islandského jazyka všem zájemcům o~tento jazyk. 

\textit{Islandsko-český studijní slovník} je určen především českým a~slovenským studentům islandštiny, začátečníkům i~pokročilým, kteří zde naleznou ucelené fonetické, morfologické a~syntaktické informace, jakož
i~slovní zásobu potřebnou k~aktivnímu i~pasivnímu užití s~rozsahem vhodným ke studiu islandštiny na vysoké škole.
Slovník je rovněž určen všem zájemcům o~islandštinu a~Island, jimž slovník nabízí encyklopedické informace, úvod do fonetiky a~gramatiky a~velké množství fotografií a~ilustrací. 
Třebaže se jedná o~slovník primárně zaměřený na potřeby českých a~slovenských uživatelů, obsahuje také řadu informací, které usnadní práci islandskému uživateli (např. použití více jak 10 000 synonym a~antonym, překlady příkladů ap.).

Slovní zásoba původně vychází z~\textit{Concise Icelandic-English Dictionary} (\cite {ic_en}), značně však byla rozšířena díky přístupu k~webovému projektu \textit{ISLEX} (\cite {int1}). Značné množství hesel bylo
přidáno na podnět uživatelů online verze \textit{Islandsko-českého studijního slovníku}.

Slovník obsahuje 30~575 hesel s~46~725 významy nebo slovními spojeními. Z~celkového počtu 30~575 je 28~097 svébytných hesel obsahujících alespoň jeden význam nebo slovní spojení. 
Zbylá hesla (v~počtu 2~478) jsou varianty, dublety nebo nepravidelné tvary.

Slovník je primárně online elektronický slovník, volně přístupný na internetových stránkách \url{www.hvalur.org}  (\cite {int14}). 
Tištěná verze slovníku je publikována na vlastní náklady v~počtu 30 kusů, přičemž většina exemplářů je určena knihovnám a~institucím v~České republice, Slovenské republice a~na Islandu
a~několik exemplářů je určeno autorům nebo zájemcům z~řad veřejnosti. Dvě PDF verze (vyhledávání / prohlížení offline a~vlastní tisk) jsou dostupné zdarma na internetových stránkách slovníku (viz výše).

\textit{Islandsko-český studijní slovník} je otevřený svobodný projekt, jehož se v~průběhu tvorby zúčastnilo velké množství spoluautorů (viz Historie tvorby IČSS), kterým tímto jménem autorů velmi děkuji.
Děkuji také všem uživatelům online verze slovníku za trpělivost, dobrou vůli a~veškeré připomínky a~návrhy, jmenovitě především Petrovi Mikešovi, Adamu Kožouškovi a~Samoriele za neutuchající víru v~náš projekt.
Poděkování patří také našim rodinám a~blízkým, díky kterým jsme se mohli soustavně věnovat tvorbě slovníku.
Srdečné poděkování patří mé ženě Dorotce za podporu a~nadšení potřebné k~tvorbě slovníku.

\blspace[5]

{\centering Aleš Chejn, Brwinów, 24. prosince 2015\par}
