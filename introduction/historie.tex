Projekt \textit{Islandsko-českého studijního slovníku} (IČSS) vznikl v~roce 2001 na nepřímý podnět doc. PhDr. Heleny Kadečkové, CSc. z~Univerzity Karlovy,
kdy byl vypracován seznam slov. 
Tento seznam obsahoval islandská slova a jejich překlad, později byly přidány základní morfologické informace o~slovním druhu a deklinační / konjugační koncovky podle vzoru z~\textit{Concise Icelandic-English Dictionary} (\cite {ic_en}). 
Seznam v~této době obsahoval okolo 1 200 slov. Byl zveřejněn pomocí statických webových stránek a později byl umístěn na webu Univerzity Karlovy. 

V~roce 2006 byl slovník rozšířen o~nových 1 500 heslových slov a stávající hesla byla rozšířena o~nové významy, zvláště významy frázových sloves. 
Ke konci roku 2006  byl slovník převeden z programu Excel do databáze MySQL a byly zahájeny práce na vytvoření aplikace \textit{Dictionary System} (DS) (\cite {int19}), která by podporovala tvorbu IČSS. 
Obsah slovníku byl zveřejněn pod svobodnou licencí na webových stránkách \textit{www.hvalur.org} (\cite {int14}), kde se nachází dodnes. 

V~roce 2007 byla vytvořena první verze aplikace DS. V~této době se k~projektu přidala Renata Pešková Emilsson, která se nyní věnuje podpoře výuky češtiny na Islandu. 
Ta slovník obohatila rozsáhlým seznamem slov z~praktického života na Islandu. Dále se slovník rozšiřoval o~heslová slova podle \textit{Concise Icelandic-English Dictionary} (\cite {ic_en}). 
V průběhu následujících třech let bylo doplněno přibližně 15 000 slov. V této době byla hesla rozšířena o~synonyma, příklady, frekvenci, oborovou a stylovou charakteristiku. K projektu se na Islandu připojila na krátkou dobu Kristýna Antonová, studentka islandštiny na Islandské univerzitě, která 
nás upozornila na nesrovnalosti v~uživatelském rozhraní aplikace DS.

Během léta 2009 byl vytvořen skript na generování skloňování a časování heslových slov. Výsledky generování byly porovnány s~\textit{Beygingarlýsing íslensks nútímamáls} (\cite {DalvikVM}) a během následujících let konzultovány
s~Jónem Gíslasonem. Kristín Bjarnadóttir, vedoucí projektu \textit{Beygingarlýsing íslensks nútímamáls}, zodpověděla řadu našich dotazů týkajících se deklinací a konjugací.

V~roce 2009 začala spolupráce s~\textit{Biolib.cz} (\cite {int5}), jmenovitě s~Ondřejem Zichou. Obdrželi jsme  databázi s~náhledy obrázků, autory a licencemi a latinskými názvy botanických a zoologických druhů.

V~zimním semestru 2009 jsme požádali s~Amirem Mulahumicem, bosenským studentem islandštiny na Islandské univerzitě, Studentský fond Islandské univerzity o~grant na tvorbu zvukové databáze islandských slov. 
Studentský fond návrh přijal. Peníze z~grantu jsme určili pro islandského rodilého mluvčího, Jóna Gíslasona, lektora islandského jazyka na Islandské univerzitě v~Reykjavíku, který nahrál více jak 22 000 islandských slov (obsah slovníku) do elektronické formy. 
S~Dorotou Nierychlewskou-Chejn jsme připravili dva seznamy heslových slov (jeden pro nahrávajícího a druhý pro zpracování nahrávek). Na podporu nahrávek jsme vytvořili webové stránky nazvané \textit{Databáze s~výslovností islandských slov} (\cite {int13}) a tyto stránky jsme přeložili do 5 jazyků (Jón Gíslason do islandštiny, Ján Zaťko do slovenštiny, francouzštiny a angličtiny, Amir Mulamuhic do srbochorvatštiny a Dorota Nierychlewska-Chejn do polštiny).

V~roce 2010 jsme přidali do databáze pravidla islandské výslovnosti, která jsou uvedena v~knize \textit{Handbók um íslenskan framburð} (\cite {is2}). Z~těchto pravidel jsme automaticky vygenerovali fonetický zápis výslovnosti všech hesel ve slovníku. 
Jón Gíslason ověřil správnost fonetického zápisu u~všech heslových slov.

Na konci roku 2011 jsme spojili webové stránky slovníku se stránkami aplikace a sjednotili jsme a vylepšili vzhled webových stránek. Cennými připomínkami se na vytvoření aplikace podílela
RNDr. Ingrid Nagyová, PhD., vysokoškolský pedagog Ostravské Univerzity. Ján Zaťko přeložil nové webové stránky, Průvodce po slovníku do slovenštiny, francouzštiny a angličtiny a celý manuál aplikace do angličtiny. 
Dorota Nierychlewska-Chejn přeložila stránky do polštiny, Jón Gíslason do islandštiny a Marcos Helena do portugalštiny.

Od prosince 2011 do března 2014 jsme realizovali první čtení, ve kterém jsme doplnili chybějící informace, přidali nová hesla a opravili pravopisné chyby. 
Obsah slovníku jsme porovnávali především s~\textit{Íslensk orðabók}  (\cite {is}) a \textit{Islandsko-českým slovníkem} (\cite {is7}). S~Vojtěchem Kupčou jsme konzultovali mnohé nesnáze při překladech.

V~letech 2013 až 2015 byl vytvořen morfologický klíč, který rozšiřuje klasický systém zobrazení morfologické informace v~islandských slovnících o~přesné zařazení daného hesla do určité deklinační / konjugační skupiny, jejíž
skloňování / časování je ukázáno na příkladu. V~roce 2014 jsme obdrželi roční grant od Ostravské univerzity v~Ostravě na tvorbu morfologického klíče. S~realizací grantu nám velmi pomohla doc. PhDr. Diana Svobodová, Ph.D. a Ing. Zuzana Havrlantová.

Od března 2014 do června 2015 jsme realizovali druhé finální čtení, ve kterém jsme porovnávali informace s~webovým slovníkem \textit{ISLEX} (\cite {int1}). 
Na požádání jsme obdrželi od Þórdís Úlfarsdóttir (vedoucí projektu \textit{ISLEX}) seznam víceslovných příslovečných frází. Takto poskytnutý seznam nám usnadnil opravy slovníku. 
S~Þórdís Úlfarsdóttir jsme konzultovali různé gramatické jevy a záležitosti týkající se deklinací / konjugací.
Martina Kašparová, studentka Univerzity Karlovy v~Praze a islandštiny na Islandské univerzitě, prošla celý slovník, přičemž se soustředila mimo jiné na stylistické a pravopisné korektury překladů. 
Lucie Brabcová udělala jazykové a typografické korektury celého slovníku.

Vojtěch Kupča vydal v~roce 2014 \textit{Islandskou gramatiku} (\cite {is77}) a souhlasil s~připojením této publikace k~IČSS.
Dorota Nierychlewska-Chejn, Ján Zaťko a Jiřina Chejnová navrhly design obálky IČSS.

V~prosinci 2015 byl převeden obsah databáze IČSS do typografického prostředí {\LaTeX}. V~lednu 2016 byl slovník vytištěn v~tiskárně Melmen v~Pardubicích.

\blspace[5]

{\centering Všem spoluautorům děkujeme za spolupráci.\par}
