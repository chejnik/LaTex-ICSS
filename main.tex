% ############################################### PREAMBLE AND CONTROL PANEL #

% ########################################################### DOCUMENT CLASS #
\documentclass[ %     draft
             % , demo  % black boxes instead of figures
              , 8pt, twoside, openright]{extbook}

% ########################################################## AND CONTROL PANEL #

% true - hmarginration 1:1, urlcolor - darkgreen, version - for searching
\newif\ifscreen 	   \screentrue
% true - no apendix, no author of grammer, no backmatter ISBN, 
\newif\ifPDF               \PDFtrue
% true - rotate pages of morph verb tables
\newif\ifPDFrotate         \PDFrotatetrue
% true - adds white searchable value !
\newif\ifPDFsearchvalue    \PDFsearchvaluetrue
\newif\ifmakecovers        \makecoverstrue
\newif\ifmakefrontmatter   \makefrontmattertrue
\newif\ifinputintroduction \inputintroductiontrue
\newif\ifinputletters      \inputletterstrue
\newif\ifinputphon         \inputphontrue
\newif\ifinputmorpho       \inputmorphotrue
\newif\ifmakebackmatter    \makebackmattertrue
\newif\ifshowcrop           \showcroptrue
% corrections
\newif\ifC                  \Ctrue

\newif\ifPRINTHOUSE 	   \PRINTHOUSEtrue
\newif\ifPDFSEARCH  	   \PDFSEARCHfalse
\newif\ifPDFPRINT 	   \PDFPRINTfalse
\newif\ifPDFtest	   \PDFtestfalse

%versions
%PRINTHOUSE
\ifPRINTHOUSE
\makecoversfalse
\screenfalse
\PDFfalse
\PDFrotatefalse
\PDFsearchvaluefalse
\showcroptrue
\Ctrue
\fi

%PDFSEARCH
\ifPDFSEARCH
\screentrue
\PDFtrue
\PDFrotatetrue
\PDFsearchvaluetrue
\showcropfalse
\Ctrue
\fi

%PDFPRINT
\ifPDFPRINT
\screenfalse
\PDFtrue
\PDFrotatefalse
\PDFsearchvaluefalse
\showcropfalse
\Ctrue
\fi

%PDFtest
\ifPDFtest
\makecoversfalse
\makefrontmattertrue
\inputintroductiontrue
\inputletterstrue
\inputphonfalse
\inputmorphofalse
\makebackmattertrue
\screentrue
\PDFfalse
\PDFrotatefalse
\PDFsearchvaluefalse
\showcropfalse
\Cfalse
\fi

% 1. PDF (for printing)  
% a) searchable value, b) no apendix, c) version= PDF d) no ISBN and no ISBN bar
% e) automatic rotation of landscape pages, f) no author of Icelandic Grammer
% 2. PDF (for viewing offline and searching) + screen (horizontal ratio 1:1) 
% 3. PRINTHOUSE
% a) no searchable value, b) apendix, c) version ISBN d) ISBN bar
% e) no rotation of landscape pages, f)urllinkcolor black
% f) A4 paper dimentions, C5 layout, shows cropping marks

% ########################################################## PAGE GEOMETRIES #

\usepackage%[showframe]
           {geometry}
           

\ifshowcrop
% Basic geometry of document
\geometry
  { headsep    =   \baselineskip
  , textwidth  = 42\baselineskip
  , textheight = 60\baselineskip
  , hmarginratio = \ifscreen 1:1\else 2:3\fi
  , vmarginratio = 2:3
  , bindingoffset = 0cm
  , onecolumn
  , a4paper 
  , layout=c5paper,
  , layouthoffset=\dimexpr(\paperwidth-\csname Gm@layoutwidth\endcsname)/2\relax
   , layoutvoffset=\dimexpr(\paperheight-\csname Gm@layoutheight\endcsname)/2\relax
  , showcrop
  }
\else
\geometry
  { headsep    =   \baselineskip
  , textwidth  = 42\baselineskip
  , textheight = 60\baselineskip
  , hmarginratio = \ifscreen 1:1\else 2:3\fi
  , vmarginratio = 2:3
  , bindingoffset = 0cm
  , onecolumn,
  , c5paper
  }
\fi

\ifshowcrop
% Geometry of the dictionary section
\newcommand\dictionarygeometry{\newgeometry
  { textwidth  = 42\baselineskip
  , textheight = 60\baselineskip
  , hmarginratio = \ifscreen 1:1\else 2:3\fi
  , vmarginratio = 3:2
  , bindingoffset = 0cm
  , twocolumn
  , a4paper 
  , layout=c5paper
  , layouthoffset=\dimexpr(\paperwidth-\csname Gm@layoutwidth\endcsname)/2\relax
   , layoutvoffset=\dimexpr(\paperheight-\csname Gm@layoutheight\endcsname)/2\relax
  , showcrop
  }}
   % Geometry of the cover pages
\newcommand\covergeometry{\newgeometry
  { textwidth  = 42\baselineskip
  , textheight = 60\baselineskip
  , hmarginratio = 1:1
  , vmarginratio = 2:3
  , bindingoffset = 0cm
  , onecolumn
  , layout=c5paper
 , layouthoffset=\dimexpr(\paperwidth-\csname Gm@layoutwidth\endcsname)/2\relax
   , layoutvoffset=\dimexpr(\paperheight-\csname Gm@layoutheight\endcsname)/2\relax
  , showcrop
  }} 
\else
% Geometry of the dictionary section
\newcommand\dictionarygeometry{\newgeometry
  { textwidth  = 42\baselineskip
  , textheight = 60\baselineskip
  , hmarginratio = \ifscreen 1:1\else 2:3\fi
  , vmarginratio = 3:2
  , bindingoffset = 0cm
  , twocolumn,
  , c5paper
  }}
   % Geometry of the cover pages
\newcommand\covergeometry{\newgeometry
  { textwidth  = 42\baselineskip
  , textheight = 60\baselineskip
  , hmarginratio = 1:1
  , vmarginratio = 2:3
  , bindingoffset = 0cm
  , onecolumn
  , c5paper
  }} 
\fi

\def\cropmarkgap{7}% mm

\makeatletter
\def\Gm@cropmark(#1,#2,#3,#4){% #1 = x direction, #2 = y direction, #3 & #4 no longet used
  \begin{picture}(0,0)
    \setlength\unitlength{1truemm}%
    \linethickness{0.25pt}%
    \put(\the\numexpr #1*\cropmarkgap\relax,0){\line(#1,0){\the\numexpr 20-\cropmarkgap}}%
    \put(0,\the\numexpr #2*\cropmarkgap\relax){\line(0,#2){\the\numexpr 20-\cropmarkgap}}%
  \end{picture}}%
\makeatother
% ################################################################# LANGUAGES #

\usepackage[icelandic, latin, czech]{babel}
\usepackage{csquotes}


% ################################################################## ENCODING #

\usepackage[utf8]{inputenc}

% Smashing uppercase letters with diacritics
% ------------------ THIS IS EVIL AND SHOULD NEVER BE DONE -------------------
% Don't smash anything if we are on the first run to produce the index
\IfFileExists{figures.idx}{
  % Czech uppercase letters with diacritics:
  \DeclareUnicodeCharacter{00C1}{\strut\smash{\' A}} % Á
  \DeclareUnicodeCharacter{010C}{\strut\smash{\v C}} % Č
  \DeclareUnicodeCharacter{010E}{\strut\smash{\v D}} % Ď
  \DeclareUnicodeCharacter{00C9}{\strut\smash{\' E}} % É
  \DeclareUnicodeCharacter{011A}{\strut\smash{\v E}} % Ě
  \DeclareUnicodeCharacter{00CD}{\strut\smash{\' I}} % Í
  \DeclareUnicodeCharacter{0147}{\strut\smash{\v N}} % Ň
  \DeclareUnicodeCharacter{00D3}{\strut\smash{\' O}} % Ó
  \DeclareUnicodeCharacter{0158}{\strut\smash{\v R}} % Ř
  \DeclareUnicodeCharacter{0160}{\strut\smash{\v S}} % Š
  \DeclareUnicodeCharacter{0164}{\strut\smash{\v T}} % Ť
  \DeclareUnicodeCharacter{00DA}{\strut\smash{\' U}} % Ú
  \DeclareUnicodeCharacter{016E}{\strut\smash{\r U}} % Ů
  \DeclareUnicodeCharacter{00DD}{\strut\smash{\' Y}} % Ý
  \DeclareUnicodeCharacter{017D}{\strut\smash{\v Z}} % Ž
  % Icelandic uppercase letters with diacritics: ÁÉÍÓÚÝ (ÐÞÆ) and
  \DeclareUnicodeCharacter{00D6}{\strut\smash{\" O}} % Ö
}{}
% ------------------ THAT WAS EVIL AND SHOULD NEVER BE DONE ------------------

% Non-breakable space
\DeclareUnicodeCharacter{00A0}{~}

% Non-breakable hyphen
\newcommand*{\nobreakhyphen}{\mbox{-}}
\DeclareUnicodeCharacter{2011}{\nobreakhyphen}


% ############################################################ FONTS, SYMBOLS #

\usepackage[T1]{fontenc}

\usepackage{tgpagella}
\usepackage[scaled=0.88]{helvet}      % relative scale of the two fonts

\def\phvfamily{\fontfamily{phv}\selectfont}

%\usepackage{stmaryrd}
% Let's keep this minimal instead of using the package:
\DeclareSymbolFont{stmry}{U}{stmry}{m}{n}
%\SetSymbolFont{stmry}{bold}{U}{stmry}{b}{n}
\DeclareMathSymbol\shortrightarrow\mathrel{stmry}{"01}
\DeclareMathSymbol\shortuparrow\mathrel{stmry}{"02}

% Methinks these are not necessary
%\usepackage{dingbat}
%\usepackage{manfnt}
%\usepackage{latexsym}

\usepackage{pifont}




% #################################################################### LAYOUT #

% \normalsize should be {8pt}{9.6pt}
\def\HUGE {\fontsize{23.887872pt}{3\baselineskip}\selectfont}
\def\Huge {\fontsize{19.906560pt}{3\baselineskip}\selectfont}
\def\huge {\fontsize{16.588800pt}{3\baselineskip}\selectfont}
\def\LARGE{\fontsize{13.824000pt}{2\baselineskip}\selectfont}
\def\Large{\fontsize{11.520000pt}{2\baselineskip}\selectfont}
\def\large{\fontsize{ 9.600000pt}{2\baselineskip}\selectfont}

% Two columns layout ruler
\setlength\columnsep    {2\baselineskip}
\setlength\columnseprule{0.4pt}

% Necessary for baseline alignment
\topskip=\baselineskip
\raggedbottom
\setlength\parskip{0pt} % it's better to avoid glue

% Temporarily suppress warnings
\hbadness=2000
\vbadness=5000

\setlength\emergencystretch{17pt}

% Allow smaller emergencystretch in several cases
\newenvironment{xtolerant}[2]{%
  \par
  \ifx\empty#1\empty\else\tolerance=#1\relax\fi
  \ifx\empty#2\empty\else\emergencystretch=#2\relax\fi
}{%
  \par
}

% ######################################################### ASSORTED PACKAGES #

\usepackage{tikz}
\usetikzlibrary{calc}

\usepackage{tipa}                      % IPA phonetics

\usepackage{fourier-orns}              % ornaments
\usepackage{amsmath}                   % non-breakable dash
\usepackage{hyphenat}                  % no hyphen in abbreviations

\usepackage{hanging}
\usepackage{fix2col}
\usepackage{fixltx2e}                  % subscript
\usepackage{pagecolor}
\usepackage{afterpage}                 % Used in coloring trick for covers
\usepackage{pdfpages}

\usepackage{multirow}
\usepackage{multicol}
\usepackage{rotating}

\usepackage{etoolbox} % http://ctan.org/pkg/etoolbox
%\usepackage{newunicodechar} 		% new definition of middle dot
\usepackage[ISBN=978-80-260-2385-2,SC0]{ean13isbn}

\usepackage{pgffor}

\usepackage{newunicodechar}

\newunicodechar{·}{\chejnikcdot}
\makeatletter
\newcommand{\chejnikcdot}{%
  \ifnum\lastpenalty=10042
    \kern-0.06667em\relax % dots should always be kerned
  \fi
  \textperiodcentered
  \@ifnextchar|{\kern\chejnikkern\relax}{\penalty10042 }%
}

\newcommand{\closeupdotbar}{%
  \renewcommand\chejnikkern{-0.06667em}%
}
\newcommand\chejnikkern{0pt} % default

% Middle dot + | little bit closer, nicer
%\newunicodechar{·}{\advancedcdot}
%\makeatletter
%\newcommand*\advancedcdot{\textperiodcentered
%  \@ifnextchar|{\hspace{-0.03667em}}{\checkifcdotnext}}
%\newcommand*\checkifcdotnext[2]
%  {\def\tmpa{·}\def\tmpb{#1#2}\ifx\tmpa\tmpb\hspace{-0.06667em}\fi#1#2}
%\makeatother

% ############################################################ BASELINE GRID #

\usepackage{eso-pic}
\usepackage{tikzpagenodes}

% Command to draw a baseline grid
\newcommand\drawbaselinegrid{%
  \begin{tikzpicture}[overlay,remember picture]
    \draw [red!30!white, ultra thin, dashed]
      (0,0) grid [ ystep  = \baselineskip, xstep = \textwidth
                 , shift  = (current page text area.north west)
                 , yshift = -\dp\strutbox
                 ] ++(\textwidth,-\textheight);
    \draw [red!30!white, thin]
      (0,0) grid [ step = \baselineskip, xstep = \textwidth
                 , shift=(current page text area.north)
                 ] ++(0.5\textwidth,-\textheight)
      (0,0) grid [ step = \baselineskip, xstep = \textwidth
                 , shift=(current page text area.north)
                 ] ++(-0.5\textwidth,-\textheight);
    \draw[black!10!white]
      (current page text area.north west)
        rectangle (current page text area.south east);
  \end{tikzpicture}}

% Draw a baseline grid on every page
% \AddToShipoutPicture{\drawbaselinegrid}


% ################################################################## METADATA #

\title{%
  \textbf{Islandsko-český studijní slovník}%
  \thanks{Tato kniha byla vytvořena v \LaTeX{}u pod Ubuntu 14.04.%
    Poděkování patří všem autorům, kteří publikují pod svobodnými licencemi.}}
\author{Aleš Chejn, Ján Zaťko, Jón Gíslason}
\date{říjen 2011}


% ############################################################## BIBLIOGRAPHY #

\usepackage[style=authoryear]{biblatex}
\addbibresource{slovnik.bib}


% ##################################################################### INDEX #

\usepackage{imakeidx}

% Index of authors of photographs
\makeindex
  [ name = figures
  , title = {Seznam autorů fotografií a ilustrací}
  , intoc
  , columnseprule
  , columnsep = \columnsep
  , noautomatic                       % We handle this in make.sh
  ]

% icelandic rules start
\usepackage{filecontents}
\def\mygroup#1{{\bfseries#1}}
\begin{filecontents*}{style.xdy}
;; style.xdy
(markup-letter-group :open-head "~n\mygroup{" :close-head "}")
\end{filecontents*}
% icelandic rules end


% ############################################################### PAGE STYLES #

\usepackage{fancyhdr}

% -------------------------------------- BASIC PAGE STYLE
\fancypagestyle{plain}{
  \fancyhf\relax                       % Clear header/footer
  \renewcommand \headrulewidth {0.0pt} % No header rule
  \renewcommand \footrulewidth {0.0pt} % No footer rule
  \fancyfoot[C]{\thepage}              % Page number in footer, centred
}

% -------------------------------------- INDEX PAGE STYLE  
%\fancypagestyle{indexstyle}{
%  \fancyhf\relax                       % Clear header/footer
%  \renewcommand \headrulewidth {0.4pt} % Thin header rule
%  \renewcommand \footrulewidth {0.0pt} % No footer rule
%  \fancyhead[C]{\thepage}              % Page number in footer, centred
%}

% -------------------------------------- DICTIONARY PAGE STYLE
\fancypagestyle{myheadings}{
  \fancyhf\relax                       % Clear header/footer
  \renewcommand \headrulewidth {0.4pt} % Thin header rule
  \fancyhead[CO,CE]{\thepage}          % Page number in header, centred
  % NOTE: the page numbers will be printed when the dictionary is ready
  \fancyhead[LE,LO]{\phvfamily\bfseries\rightmark}
  \fancyhead[RE,RO]{\phvfamily\bfseries\leftmark}
}

\pagestyle{plain}  

% ##################################################################### LISTS #

\usepackage{expdlist}
\usepackage{enumitem}

\setlist{nosep, topsep=\baselineskip}
\setlist[itemize]{leftmargin=*,label=\scalebox{.6}{\textbullet}}


% #################################################################### FLOATS #

\usepackage{caption}
\captionsetup{labelformat=empty}

% Alter some LaTeX defaults for better treatment of figures:
% See p.105 of "TeX Unbound" for suggested values. 
% See pp. 199-200 of Lamport's "LaTeX" book for details.
% ------------------------------------- Parameters for ALL pages:
\renewcommand \topfraction    {0.9}   %   max fraction of floats at top
\renewcommand \bottomfraction {0.8}   %   max fraction of floats at bottom
% ------------------------------------- Parameters for TEXT (non float) pages:
\setcounter{topnumber}    {1}
\setcounter{bottomnumber} {1}
\setcounter{totalnumber}  {1}         %   2 may work better
\setcounter{dbltopnumber} {2}         %   for 2-column pages
\renewcommand \dbltopfraction   {0.9} %   fit big float above 2-col. text
\renewcommand \textfraction     {0.07}%   allow minimal text w. figs
% ------------------------------------- Parameters for FLOAT (non text) pages:
\renewcommand \floatpagefraction{0.7} %   require fuller float pages
% N.B.: floatpagefraction MUST be less than topfraction !!
\renewcommand{\dblfloatpagefraction}{0.7} % require fuller float pages
% remember to use [htp] or [htpb] for placement

\makeatletter
\setlength{\@fptop}{0pt}
\setlength{\@fpbot}{0pt plus 1fil}
\makeatother

% This kills spacing around floats
\setlength \intextsep        {0pt}
\setlength \floatsep         {0pt}
\setlength \textfloatsep     {0pt}
\setlength \dbltextfloatsep  {0pt}
\setlength \dblfloatsep      {0pt}

% This kills spacing around captions
\setlength \abovecaptionskip {0pt}
\setlength \belowcaptionskip {0pt}


% #################################################################### TABLES #

\usepackage{booktabs}
\usepackage{tabularx}
\usepackage{longtable}
\usepackage{ltxtable}
\usepackage{lscape}

\ifPDFrotate
\else
\makeatletter			       
\renewcommand\PLS@Rotate[1]{}          % this command prevents pdflscape 
\makeatother                           % to rotate pages
\fi

%\LetLtxMacro\oldtabularx\tabularx
%\LetLtxMacro\oldendtabularx\endtabularx
%\renewenvironment{tabularx}[1]
%  {\oldtabularx{\columnwidth}{#2}}
%  {\oldendtabularx}

\newcommand\LTXfw[1]{\LTXtable{\columnwidth}{#1}}

%%% \LetLtxMacro\oldtabularx\tabularx
%%% \LetLtxMacro\oldendtabularx\endtabularx
%%% \newenvironment{longtabularx}[2]
%%%   {\oldtabularx{#1}{#2}}
%%%   {\oldendtabularx}

% ######################################################### czech babel vs cmidrule
% ###################################### http://tex.stackexchange.com/a/112001/8380
\begingroup
  \makeatletter
  \catcode`\-=\active
  \AtBeginDocument{
  \def\@@@cmidrule[#1-#2]#3#4{\global\@cmidla#1\relax
    \global\advance\@cmidla\m@ne
    \ifnum\@cmidla>0\global\let\@gtempa\@cmidrulea\else
    \global\let\@gtempa\@cmidruleb\fi
    \global\@cmidlb#2\relax
    \global\advance\@cmidlb-\@cmidla
    \global\@thisrulewidth=#3
    \@setrulekerning{#4}
    \ifnum\@lastruleclass=\z@\vskip \aboverulesep\fi
    \ifnum0=`{\fi}\@gtempa
    \noalign{\ifnum0=`}\fi\futurenonspacelet\@tempa\@xcmidrule}
  }
\endgroup

% macro that supress cmidrule at page brake
% http://tex.stackexchange.com/questions/110548/using-longtable-with-cmidrule-page-breaking-issues
\makeatletter
\def\LT@output{%
  \ifnum\outputpenalty <-\@Mi
    \ifnum\outputpenalty > -\LT@end@pen
      \LT@err{floats and marginpars not allowed in a longtable}\@ehc
    \else
      \setbox\z@\vbox{\unvbox\@cclv}%
      \ifdim \ht\LT@lastfoot>\ht\LT@foot
        \dimen@\pagegoal
        \advance\dimen@-\ht\LT@lastfoot
        \ifdim\dimen@<\ht\z@
          \setbox\@cclv\vbox{\unvbox\z@
\copy\LT@foot\vss}%
          \@makecol
          \@outputpage
          \setbox\z@\vbox{\box\LT@head}%
        \fi
      \fi
      \global\@colroom\@colht
      \global\vsize\@colht
      \vbox
        {\unvbox\z@
\box\ifvoid\LT@lastfoot\LT@foot\else\LT@lastfoot\fi}%
    \fi
  \else
    \setbox\@cclv\vbox{\unvbox\@cclv

\setbox0\lastbox
\ifdim\ht0=0.29999pt \unskip % hope that was a cline we threw away
\else
\nointerlineskip
\box0     % put it back, whatever it was
\fi

\copy\LT@foot\vss}%
    \@makecol
    \@outputpage
      \global\vsize\@colroom
    \copy\LT@head\nobreak
  \fi}

\makeatother

% These values ensure that for top, mid and bottom rules
% space above + rule width + space below = baselineskip
\setlength \abovetopsep    {0.43\baselineskip}
\setlength \heavyrulewidth {0.10\baselineskip}
\setlength \belowbottomsep {0.43\baselineskip}
\setlength \aboverulesep   {0.47\baselineskip}
\setlength \lightrulewidth {0.06\baselineskip}
\setlength \belowrulesep   {0.47\baselineskip}

%\usepackage{letltxmacro}

%%% \LetLtxMacro\oldtabular\tabular
%%% \LetLtxMacro\oldendtabular\endtabular
%%% \renewenvironment{tabular}[2][t]
%%%   {\oldtabular[#1]{#2}}
%%%   {\oldendtabular}

%%% \LetLtxMacro\oldtable\table
%%% \LetLtxMacro\oldendtable\endtable
%%% \renewenvironment{table}[1][h]
%%%   {\leavevmode\oldtable[#1]\vspace{\dp\strutbox}}
%%%   {\vspace{-\dp\strutbox}\oldendtable}
%%%   % This aligns a [t] aligned tab-envo to baseline grid


%%% % declention and conjugation tables
%%% \usepackage{floatrow}
%%% \DeclareFloatFont{footnotesize}{\footnotesize}
%%% % "scriptsize" is defined by floatrow, "tiny" not
%%% \floatsetup[table]{font=footnotesize}

% ################################################################### FIGURES #

\usepackage{adjustbox} % loads also graphicx

\graphicspath{%
  {/home/chejnik/Dokumenty/hvalur.org/images/biolib/full/}%
  {/home/chejnik/Dokumenty/hvalur.org/images/uploaded_files/}}

\newlength\dicfigheight
\newlength\dicfigskip

\newcommand\dicFigure[3]{ % {filename}{caption}{index entry}
  \setlength\fboxsep{0pt}\setlength\fboxrule{0.5pt}
  \def\dicfigbox{\fbox{%
    \adjincludegraphics [ max height = 0.8\columnwidth
                        , max width  = 0.8\columnwidth ] {#1}}}
  \setlength\abovecaptionskip {\dp\strutbox}
  \setlength\belowcaptionskip {\ht\strutbox} 
  \setlength\dicfigheight {\heightof\dicfigbox+\fboxrule}
  \setlength\dicfigskip
    {\baselineskip*\numexpr1+\dicfigheight/\baselineskip\relax-\dicfigheight}
  \begin{figure}[ht]
    \centering\vspace*{\dicfigskip}\dicfigbox\caption{#2}\index[figures]{#3}
  \end{figure}
}


% #################################################################### COLORS #

\usepackage{color}

% http://www.colorschemer.com/online.html
% main color - grammar color
% #66023C
\definecolor {darkgreen}          {rgb} {0.40, 0.01, 0.24}

\definecolor {royalazure}         {rgb} {0.00, 0.22, 0.66}
\definecolor {brown}              {rgb} {0.40, 0.01, 0.24}

% COLORS FOR THUMB INDEXES
\definecolor {babyblueeyes}       {rgb} {0.63, 0.79, 0.95}
\definecolor {unitednationsblue}  {rgb} {0.36, 0.57, 0.90}
\definecolor {blue(ryb)}          {rgb} {0.01, 0.28, 1.00}
\definecolor {darkblue}           {rgb} {0.00, 0.00, 0.55}
\definecolor {screamingreen}      {rgb} {0.46, 1.00, 0.44}
\definecolor {limegreen}          {rgb} {0.20, 0.80, 0.20}
\definecolor {islamicgreen}       {rgb} {0.00, 0.56, 0.00}
\definecolor {upforestgreen}      {rgb} {0.00, 0.27, 0.13}
\definecolor {icterine}           {rgb} {0.99, 0.97, 0.37}
\definecolor {orange(colorwheel)} {rgb} {1.00, 0.50, 0.00}
\definecolor {orange-red}         {rgb} {1.00, 0.27, 0.00}
\definecolor {oucrimsonred}       {rgb} {0.60, 0.00, 0.00}
\definecolor {cottoncandy}        {rgb} {1.00, 0.74, 0.85}
\definecolor {orchid}             {rgb} {0.85, 0.44, 0.84}
\definecolor {vividcerise}        {rgb} {0.85, 0.11, 0.51}
\definecolor {patriarch}          {rgb} {0.50, 0.00, 0.50}

\definecolor {title}              {RGB} {  16,   13,   32}
\definecolor {golden}             {RGB} { 241,  184,   45}
\definecolor {freqstar}           {RGB} {  91,    3,   99}
\definecolor {newdev}             {RGB} {   3,   11,   99}

\definecolor {color1}             {RGB} { 182,   86,    0}
\definecolor {color2}             {RGB} { 143,    9,    6}
\definecolor {color3}             {RGB} {   3,   23,  118}
\definecolor {color4}             {RGB} {   0,   82,  168}
\definecolor {color5}             {RGB} {   0,   85,  142}
\definecolor {color6}             {RGB} {   0,  115,  162}
\definecolor {color7}             {RGB} {  34,  146,  186}
\definecolor {color8}             {RGB} {  40,  159,  153}
\definecolor {color9}             {RGB} {   0,  125,  111}
\definecolor {color10}            {RGB} {   4,  107,   60}
\definecolor {color11}            {RGB} {  71,  134,   81}
\definecolor {color12}            {RGB} { 109,  134,   42}
\definecolor {color13}            {RGB} { 205,  194,   18}
\definecolor {color14}            {RGB} { 204,  162,   24}

%\definecolor {color1}             {RGB} { 239,  222,  205}
%\definecolor {color2}             {RGB} { 251,  206,  177}
%\definecolor {color3}             {RGB} { 251,  185,  142}
%\definecolor {color4}             {RGB} { 248,  166,  112}
%\definecolor {color5}             {RGB} { 216,  231,  246}
%\definecolor {color6}             {RGB} { 186,  215,  244}
%\definecolor {color7}             {RGB} { 152,  197,  243}
%\definecolor {color8}             {RGB} { 118,  181,  245}
%\definecolor {color9}             {RGB} { 181,  249,  185}
%\definecolor {color10}            {RGB} { 160,  248,  167}
%\definecolor {color11}            {RGB} { 128,  247,  136}
%\definecolor {color12}            {RGB} {  87,  248,   98}
%\definecolor {color13}            {RGB} { 250,  241,  138}
%\definecolor {color14}            {RGB} { 250,  240,  111}
%\definecolor {color15}            {RGB} { 244,  239,   85}
%\definecolor {color16}            {RGB} { 245,  239,   57}

\definecolor {lightgrey}          {RGB} { 105,  105,  105}
\definecolor {darkgreen_real}     {rgb} {0.00, 0.50, 0.00}


% ############################################################# THUMB INDEXES #

% Thumb indexes' colors
%\newcommand\BoxColor{%
%  \ifcase\theletternum darkgreen!30\or babyblueeyes\or unitednationsblue\or blue(ryb)\or screamingreen\or limegreen\or islamicgreen\or upforestgreen\or icterine\or orange(colorwheel)\or orange-red%
%  \or oucrimsonred\or cottoncandy\or orchid\or vividcerise\or patriarch\or babyblueeyes\or unitednationsblue\or blue(ryb)\or screamingreen\or limegreen\or islamicgreen\or upforestgreen\or icterine\or orange(colorwheel)\or orange-red%
%  \or oucrimsonred\or cottoncandy\or orchid\or vividcerise\or patriarch\else darkgreen!30\fi}
  
% Thumb indexes' colors
%\newcommand\BoxColor{%
%  \ifcase\theletternum darkgreen!30\or color1\or color2\or color3\or color4\or color5\or color6\or color7\or color8\or color9\or color10%
%  \or color11\or color12\or color13\or color14\or color1\or color2\or color3\or color4\or color5\or color6\or color7\or color8\or color9%
%  \or color10\or color11\or color12\or color13\or color14\or color1\or color2\or color3\or color4\or color5\or color6\else color7\or color8\or color9\or color10\or color11\or color12\fi}

%\def\BoxColor#1{darkgreen!\the\numexpr103-#1-#1-#1\relax!blue}

\newcommand\BoxColor[1]{%
\ifcase#1 darkgreen!30\or color13\or color14\or color1\or color2\or color3\or color4\or color5\or color6\or color7\or color8\or color9\or color10%
\or color11\or color12\or color13\or color14\or color1\or color2\or color3\or color4\or color5\or color6\or color7\or color8\or color9%
\or color10\or color11\or color12\or color13\or color14\or color1\or color2\or color3\or color4\or color5\or color6\else color7\or color8\or color9\or color10\or color11\or color12\fi}

\ifshowcrop
%%%%%%%%%%%%%%%%%%%%%%%%%%%%%%%%%%%%%%%%%%%%%%%%%%%%%%%%%%%%%%%%%%%%%%%%%%%%%%%%

\newcounter{letternum}
\newcounter{lettersum}                 \setcounter{lettersum}{34}
% these margins separate top thumb from top of page and last thumb from
% bottom of page. BEWARE, it's from top of the (cut out) page, not the folio!
\newlength{\thumbtopmargin}            \setlength{\thumbtopmargin}{1cm}
\newlength{\thumbbottommargin}         \setlength{\thumbbottommargin}{1cm}
\newlength{\thumbwidth}                \setlength{\thumbwidth}{2.20cm}
\newlength{\thumbheight}
\pgfmathsetlength{\thumbheight}%
  {(\csname Gm@layoutheight\endcsname-\thumbtopmargin-\thumbbottommargin)/\value{lettersum}}
% inner xsep of letter from box margin
\newlength{\thumbsep}                  \setlength{\thumbsep}{.30cm}
% overlap over cutting line
\newlength{\thumboverlap}              \setlength{\thumboverlap}{3cm}

\tikzset{
  thumb/.style={
    text=white,
    outer sep=0pt,
    inner xsep=\thumbsep,
    minimum height=\thumbheight,
    text width=\thumbwidth-2\thumbsep,
    font=\sffamily\bfseries,},
  thumb east/.style={thumb,
    xshift=\dimexpr(+\thumboverlap-\paperwidth+\csname Gm@layoutwidth\endcsname)/2\relax,
    align=left,anchor=north east,},
  thumb west/.style={thumb,
    xshift=\dimexpr(-\thumboverlap+\paperwidth-\csname Gm@layoutwidth\endcsname)/2\relax,
    align=right,anchor=north west,},
}

\def\thumbnew{}
\def\thumbold{}
\usepackage{everypage}
\AddEverypageHook{\if\relax\thumbnew\relax\xdef\thumbnew{\thumbold}\fi}

\def\ethumbs#1,#2\relax%
  {\if\relax#1\relax\else\eventhumb{#1}\fi%
   \if\relax#2\relax\else\ethumbs#2\relax\fi%
   \gdef\thumbnew{}%
   \gdef\thumbold{#1,}}

\def\othumbs#1,#2\relax%
  {\if\relax#1\relax\else\oddthumb{#1}\fi%
   \if\relax#2\relax\else\othumbs#2\relax\fi%
   \gdef\thumbold{#1,}%
   \gdef\thumbnew{}}

\newcommand{\drawthumb}[2]{%
  \begin{tikzpicture}[remember picture, overlay]
    \node [thumb #2, fill=\BoxColor{#1}]
       at ($(current page.north #2)-%
            (0,\csname Gm@layoutvoffset\endcsname+\thumbtopmargin-\thumbheight+%
               #1*\thumbheight)$) {\csname Let#1\endcsname};
   \end{tikzpicture}}



%%%%%%%%%%%%%%%%%%%%%%%%%%%%%%%%%%%%%%%%%%%%%%%%%%%%%%%%%%%%%%%%%%%%%%%%%%%%%%%%
\else
% THUMB INDEXES
% new counter to hold the current number of the letter to determine the vertical position
\newcounter{letternum}
% newcounter for the sum of all letters to get the right height of a box
\newcounter{lettersum}
\setcounter{lettersum}{34}
% some margin settings
\newlength{\thumbtopmargin}
\setlength{\thumbtopmargin}{\ifshowcrop 3cm\else 1.5cm\fi} %1cm
\newlength{\thumbbottommargin}
\setlength{\thumbbottommargin}{\ifshowcrop 6cm\else 2cm\fi} %2.5cm
% calculate the box height by dividing the page height
\newlength{\thumbheight}
\pgfmathsetlength{\thumbheight}{%
(\paperheight-\thumbtopmargin-\thumbbottommargin)%
/%
\value{lettersum}
}
% box width
\newlength{\thumbwidth}
\setlength{\thumbwidth}{\ifshowcrop 2cm\else 0.5cm\fi} %0.5cm
% style the boxes
\tikzset{
thumb/.style={
   text=white,
   minimum height=\thumbheight,
   text width=\thumbwidth,
   outer sep=0pt,
   font=\sffamily\bfseries,
 }
 }

\def\thumbnew{}
\def\thumbold{}
\usepackage{everypage}
\AddEverypageHook{\if\relax\thumbnew\relax\xdef\thumbnew{\thumbold}\fi}

\def\ethumbs#1,#2\relax{\if\relax#1\relax\else\eventhumb{#1}\fi%
                        \if\relax#2\relax\else\ethumbs#2\relax\fi%
                        \gdef\thumbnew{}%
                        \gdef\thumbold{#1,}%
}

\def\othumbs#1,#2\relax{\if\relax#1\relax\else\oddthumb{#1}\fi%
                        \if\relax#2\relax\else\othumbs#2\relax\fi%
                        \gdef\thumbold{#1,}%
                        \gdef\thumbnew{}%
}

\newcommand{\drawthumb}[2]{%
  % see pgfmanual.pdf for more information about this part
  \begin{tikzpicture}[remember picture, overlay]
    \node [thumb, fill=\BoxColor{#1}, text centered, anchor=north #2]
       at ($(current page.north #2)-%
%            (0,\thumbtopmargin+\value{letternum}*\thumbheight)%
            (0,\thumbtopmargin+#1*\thumbheight)$) {\csname Let#1\endcsname};
   \end{tikzpicture}}
\fi

\newcommand{\oddthumb} [1]{\drawthumb{#1}{west}}
\newcommand{\eventhumb}[1]{\drawthumb{#1}{east}}

\newcommand{\lettergroup}[1]%
  {\refstepcounter{letternum}%
   \expandafter\gdef\csname Let\theletternum\endcsname{#1}%
   \xdef\thumbnew{\theletternum,\thumbnew}%
   \fancyhead[LO]{\phvfamily\bfseries\rightmark%
     \expandafter\ethumbs\thumbnew\relax\relax\relax}%
   \fancyhead[RE]{\phvfamily\bfseries\leftmark%
     \expandafter\othumbs\thumbnew\relax\relax\relax}}
     
% ############################################################## LOCALIZATION #

% renames the index name
%\addto\captionsczech{%
%  \renewcommand\indexname{{Seznam autorů fotografií a ilustrací}}}

% renames the contents name
\addto\captionsczech{%
  \renewcommand\contentsname{Obsah}}

\renewcommand \contentsname {Obsah}
\renewcommand \chaptername  {Kapitola}


% ################################################################ SECTIONING #
\usepackage[explicit]{titlesec}

\setlength{\beforetitleunit}{\baselineskip}
\setlength{\aftertitleunit} {\baselineskip}

\titlespacing*{\chapter}   {0em}{*0}{*3} % Remembere there's a topskip too
\titlespacing*{\section}   {0em}{*1}{*2}
\titlespacing*{\subsection}{0em}{*1}{*2}

\newcommand\ghostdrop[2]{%
  \raisebox{-#1\baselineskip}[0pt][0pt]{#2}}

\titleformat {\chapter} {} {} {0pt} % {#1}
  {\ghostdrop{1.0}{\parbox[t]{\columnwidth}
     {\LARGE\bfseries\centering\MakeUppercase{#1}}}}

\def\longchapterskip{\vspace{2\baselineskip}}
\def\longsectionskip{\vspace{2\baselineskip}}

\titleformat {\section} {\bfseries} {} {0pt}
  {\ghostdrop{1.0}{\parbox[t]{\columnwidth}
     {\Large\bfseries\centering#1}}}

\titleformat {\subsection} {\bfseries} {} {0pt}
  {\ghostdrop{1.0}{\hspace{1em}\parbox[t]{\columnwidth-1em}
     {\large\bfseries#1}}}


\setlength\multicolsep{0pt}
%\BeforeBeginEnvironment{multicols}{\vspace{-\dp\strutbox}}


% ######################################################## TABLE OF CONTENTS #
\usepackage{titletoc}

\titlecontents {chapter} [0pt] {\addvspace\baselineskip\bfseries}
  {} {} {\titlerule*[2\baselineskip]{}\contentspage}

\titlecontents {section} [\baselineskip] {} {} {}
  {\titlerule*[1\baselineskip]{.}\contentspage}


% #################################################### ASSORTED CUSTOM MACROS #

\def\textIS#1{\foreignlanguage{icelandic}{#1}}
\def\textCS#1{\foreignlanguage{czech}{#1}}
\def\textLA#1{\foreignlanguage{latin}{#1}}

\usepackage{accsupp}

\makeatletter                                                                                                          
  \def\ACCSUPP@bdc{\special{pdf:code \ACCSUPP@span BDC}}                                                               
  \def\ACCSUPP@emc{\special{pdf:code EMC}}                                                                             
\makeatother

\usepackage{pgffor}

\newcommand\blspace[1][1]{\vspace{#1\baselineskip}}

\usepackage{xparse}


\newcommand \dicsymFrequent  {\raisebox{-.2ex}{\color{darkgreen}\ding{167}}}
\newcommand \dicsymSee       {$\shortrightarrow$}
\newcommand \dicsymCompare   {$\shortuparrow$}
\newcommand \dicsymIdiom     {$\diamondsuit$}
\newcommand \dicsymExampleIS {$\triangleright$}
\newcommand \dicsymExampleCS {\guilsinglright}
\newcommand \dicsymProverb   {{\color{newdev}\ding{96}}}
\newcommand \dicsymPhoto
  {\includegraphics[keepaspectratio,height=0.65em]{photo_icon}}

%\NewDocumentCommand\dicTerm{mo}
 % {{\phvfamily\bfseries\textIS{#1}}\IfValueT{#2}{, \phvfamily\bfseries\textIS{#2}}}

\NewDocumentCommand\dicEntry{o}
  {\par\hangpara{0.25\baselineskip}{1}%
   \IfValueT{#1}{\markboth{#1}{#1}}%
    \ifPDFsearchvalue
    \settowidth{\eyja}{#1}%
  \makebox[\eyja]{\color{white}#1}\hspace{-\eyja}%
  \fi
   \ignorespaces}
   
   \NewDocumentCommand\dicEntryGuide{o}
  {\par\hangpara{0.29\baselineskip}{1}%
   \IfValueT{#1}{\markboth{#1}{#1}}%
    \ifPDFsearchvalue
    \settowidth{\eyja}{#1}%
  \makebox[\eyja]{\color{white}#1}\hspace{-\eyja}%
  \fi
   \ignorespaces}
   
\newcommand\dicSubEntry
  {\\\hspace*{-\hangindent}}

\NewDocumentCommand\textAlt{mm}
  {\BeginAccSupp{method=pdfstringdef,ActualText={#2}}#1\EndAccSupp{}}

\newcounter{terms}

\NewDocumentCommand\dicStyleIntTerm{m}{\textIS{{\phvfamily\bfseries#1}}}
\NewDocumentCommand\dicStyleExtTerm{m}{\textIS{\textit{#1}}}

   \NewDocumentCommand\dicTerm{mo}
  {{\closeupdotbar \phvfamily\bfseries\textIS{#1}}\IfValueT{#2}{, \phvfamily\bfseries\textIS{#2}}}

  \newcommand\dicTermList[1]
  {\foreach \t [count=\n] in {#1}{\ifnum\n=1\else, \fi\dicTerm{\t}}}
  
\def\dicIPA#1{\textipa{[#1]}}

\NewDocumentCommand\dicPos{smo}
  {{\color{darkgreen}\textbf{\small#2}%
   \IfValueT{#3}{\nobreak\textsubscript{#3}}}}

%\NewDocumentCommand\dicFlx{smo}
 % {{\color{darkgreen}\footnotesize#1\IfValueT{#2}{\textsubscript{#2}}}}

  \NewDocumentCommand\dicFlx{smo}
  {{\color{darkgreen}\footnotesize#2%
   \IfValueT{#3}{\nobreak\textsubscript{#3}}}}
  
\newcommand\dicExampleCS[1]{\dicsymExampleCS~\textit{#1}}
\newcommand\dicExampleIS[1]{\dicsymExampleIS~\textIS{\textit{#1}}}

\NewDocumentCommand\dicLink{sm}
  {\IfBooleanF#1{\dicsymSee\nobreak\hspace{.08667em plus .03333em}}\nobreak\hspace{.03667em}\dicTerm{#2}}
\NewDocumentCommand\dicSynonym{sm}
  {(\IfBooleanF#1{\dicsymSee\nobreak\hspace{.08667em plus .03333em}}\nobreak\hspace{.03667em}\textIS{\textit{#2}})}
\NewDocumentCommand\dicAntonym{sm}
  {(\IfBooleanF#1{\dicsymCompare\nobreak\hspace{.08667em plus .03333em}}\nobreak\hspace{.03667em}\textIS{\textit{#2}})}
\NewDocumentCommand\dicCompare{sm}
  {(\IfBooleanF#1{\dicsymCompare\nobreak\hspace{.08667em plus .03333em}}\nobreak\hspace{.03667em}\textIS{\textit{#2}})}


\newcommand\dicPhraseIS[1]{\textIS{\textbf{#1}}}

\NewDocumentCommand\dicIdiom{mo}
  {\dicSubEntry{\color{newdev}\dicStyleIntTerm{#1} %
     \IfValueTF{#2}{+ \dicStyleIntTerm{#2} \markboth{#1 + #2}{#1 + #2}}{\markboth{#1}{#1}}}\dicsymIdiom}

\NewDocumentCommand\dicIdiomIntro{mo}
  {{\color{newdev}\dicStyleIntTerm{#1} %
     \IfValueTF{#2}{+ \dicStyleIntTerm{#2} \markboth{#1 + #2}{#1 + #2}}{\markboth{#1}{#1}}}\dicsymIdiom}
     
\newcommand\dicProverb
  {\dicSubEntry\dicsymProverb}

\newcommand\dicProverbIntro
  {\dicsymProverb}

\newcommand\dicLangCat[1]{{\footnotesize#1}}
\newcommand\dicFieldCat[1]{{\footnotesize#1}}


\newcommand\dicDirectTranslationCS[1]{#1}
\newcommand\dicIndirectTranslationCS[1]{{\footnotesize#1}}

\usepackage{placeins}

\newcommand*{\dicLetter}[2]{%
  \FloatBarrier\newpage\phantomsection\pdfbookmark[1]{#1}{#2}%
  \noindent\parbox[b][9\baselineskip][c]{\columnwidth}
    {\centering\HUGE\strut\smash{\MakeUppercase{#1}~\MakeLowercase{#1}}}%
      %\par\xdef\tpd{\the\prevdepth}
      %\par\prevdepth\tpd%
  \expandafter\lettergroup{\MakeLowercase{#1}}}

  
%tabulky
\newcommand{\specialcell}[2][l]{%
\begin{tabular}[#1]{@{}l@{}}#2\end{tabular}}


%\newcommand{\devisionguide}[1]{\hspace*{0.3cm}{{{{\foreignlanguage{icelandic}{\color{newdev}{\phvfamily{\textbf{#1}}}}}}}}}
%\newcommand{\devisionguideindent}[1]{\hspace*{-0.1cm}{{{{\foreignlanguage{icelandic}{\color{newdev}{\phvfamily{\textbf{#1}}}}}}}}}

%\newcommand\n{$n$\nobreakdash-\hspace{0pt}} %nonbreakable dash in grammar endings
%\def \nobreakseq {\nobreak \hskip 0pt \hbox} %nolinebreak in case like (-s, -)

% rule line
\newcommand{\HRule}{\rule{\linewidth}{0.1mm}}

% width of pdf found box
\newlength\eyja

%\newcommand\strutsmash[1]{\strut\smash{#1}}


% Dictionary symbols


% Command synonyms
%\let\freqstar\dicsymFrequent
%\let\photoicon\dicsymPhoto

\newcommand{\vstretch}[1]{\vspace*{\stretch{#1}}}

\newcommand*{\addthin}{\hskip0.16667em\relax}
\newcommand*{\addthinS}{\hskip0.06667em\relax}
\newcommand*{\addthinM}{\hskip0.10007em\relax}
\newcommand*{\addthinSS}{\hskip0.00007em\relax}
% \textls[15]{
% #################################################################### COVERS #

\def\dictnameCZ{islandsko-český studijní slovník}
\def\dictnameIS{íslensk-tékknesk stúdentaorðabók}

\newcommand*{\frontcoverimages}{%
  ds_image_blaberjalyng_0_2.jpg,
  ds_image_lundi_0_1.jpg,
  ds_image_hestur_0_1.jpg,
  ds_image_hrafnafifa_0_2.jpg,
  ds_image_baldursbra_0_1.jpg,
  20948.jpg,
  ds_image_hvalur_0_1.jpg,
  ds_image_blodberg_0_1.jpg}

\newcommand*{\backcoverimages}{%
  ds_image_spoi_0_1.jpg,
  ds_image_mosalyng_0_1.jpg,
  ds_image_blaklukkulyng_0_1.jpg,
  20787.jpg,
  ds_image_ufsi_0_2.jpg,
  %ds_image_gullbra_0_1.jpg,
  ds_image_vatnsberi_0_1.jpg,
  ds_image_islandsfifill_0_2.jpg,
  21137.jpg}

\newcommand\makecoverwith[1]{
 \pagecolor{title}\afterpage\nopagecolor
 \begin{center}
 \vspace*{0.3cm}%{\fill}
 {\Huge\scshape\color{white}%
   \dictnameCZ\\\dictnameIS}
 \vspace*{1em}                        % I think this is necessary
 \begin{figure}[ht]\centering%
   \setlength\fboxsep{0pt}\setlength\fboxrule{0.5pt}%
   \foreach \file in #1{
     \fbox{\includegraphics[width=5.5cm]{\file}}}
 \end{figure}
 \vspace*{\fill}
 \end{center}
}


% ###################################################### HYPERREF & PDF INFO #

% unicode neccessary so that national characters in hypersetup appear ok
\usepackage[pdftex, unicode, hyperfootnotes=false]{hyperref}

% hyperlinks in black
\makeatletter
\let\Hy@linktoc\Hy@linktoc@none
\makeatother

\hypersetup
  { pdftitle={Islandsko-český studijní slovník}
  , pdfauthor={Aleš Chejn, Jón Gíslason, Ján Zaťko}
  , pdfsubject={Islandsko-český studijní slovník}
  , pdfkeywords={Islandsko-český studijní slovník%
               , islandština, čeština, slovník}
  , bookmarks=true
  , colorlinks=true
  , citecolor=black
  , urlcolor=\ifscreen darkgreen\else black\fi
  , linkcolor=black}


% ################################################################# MICROTYPE #

\usepackage[final]{microtype}

% ############################################################### HYPHENATION #

% this allows to include brackets inside hyphenation rules f.e (pro)-dis-ku-to-vat
\lccode`\(`\(
\lccode`\)`\)
\lccode`\[`\[
\lccode`\]`\]

\begin{hyphenrules}{icelandic}
\hyphenation{
  frum-semja
  burm-neskur
  ó-breytan-le-gur
  pla-tó-nskt
  skatta-fram-tal
  skrúf-gang-ur
  strand-rauð-viður
  prent-sver-ta
  grun-nur
  uppi-skrop-pa
  ger-vi-mennska
  sö-lu-stað-ur
  sý-na-gó-ga
  mark-mið
  trau-st-ur
  ís-kyggi-leg-ur
  brjó-sta-höld
  kné-krjú-pa
  nel-li-ka
  kaffi-kvörn
  margs-konar
}
\end{hyphenrules}
\begin{hyphenrules}{czech}
\hyphenation{
  teo-rie
  mi-liardy
  (z)-od-po-věd-nost
  tan-tié-ma
  alian-ce
  pe-dia-tr(ička)
  pa-cienty
  zá-chra-nář(ka)
  dia-man-to-vá
  poe-zie
  ma-te-riál-ní
  egoi-s-ta
  egoist-ka
  nai-vi-ta
  ne-upřím-nost
  pě-ti-úhel-ník
  za-šmo-dr-chat
  ar-cheo-lo-gic-ký
  ne-omlu-vi-tel-ný
  (vy)-pro-du-ko-vat
  dia-gno-s-ti-ko-vat
  teo-lož-ka
  ateist-ka
  oceá-nu
  im-pe-ria-li-s-ta
  (vy)-zdvih-nout
  ne-úměr-ný
  kon-ti-nuál-ní
  neu-t-rum
  ne-opráv-ně-ně
  vig-vam
  pro-zaic-ký
  geo-lož-ka
  seiz-mic-ká
  teo-rém
  olean-dr
  poe-ta
  (vy)-fo-to-gra-fo-vat
  (vy)-ven-ti-lo-vat
  pneu-mo-nie
  Lu-cem-bur-čan-(ka)
  ma-te-riály
  am-bi-cióz-ní
  Myan-ma(r)
  Myan-ma-řan-(ka)
  mol-dav-šti-na
  si-tua-cí
  ne-ukáz-ně-ný
  he-te-ro-trof-ní
  ne-omlu-vi-tel-ná
  ne-upřím-ný
  dvou-bliz-ný
  ne-úmy-sl-ný
  ne-ustá-lý-mi
  ne-ohle-du-pl-ný
  ne-osob-ní
  ne-oso-le-ný
  ne-uvě-ři-tel-ný
  ne-ohro-že-ný
  ne-oby-čej-ný
  ne-útul-ný
  ne-únav-ný
  ne-ože-nit
  po-ukáz-ka
  za-úto-čit
  aero-dy-na-mic-ký
  hru-bián-ství
  (pro)-di-s-ku-to-vat
  (vy)-ča-lou-nit
  (na)-oč-ko-vá-ní
  Bhú-tán-ka
  (za)-šně-ro-vat
  soud-ní
  re-vo-lu-cio-nář(ka)
  á-zer-báj-d-žán
  čtyř-hran-ný
  ze-m-dle-lý
  klou-zač-ka
  Vest-mann-ské
  tří-stran-ný
  (za)-bu-rá-ce-ní
  os-tud-ný
  pro-du-cent-(ka)
  chvá-s-tal-(ka)
  apar-theid
  asis-ten-ce
  asis-tent-(ka)
  (u)-či-nit
  (u)-za-mknout
  past-vi-na
  (za)-zá-lo-ho-vat
  Af-ri-čan-(ka)
  (za)-an-ga-žo-vá-ní
  se-rióz-ní
  opo-nent-(ka)
  pro-tiv-ník
  }
\end{hyphenrules}


% ############################################################# CZECH TYPOGRAPHY #
% https://math.feld.cvut.cz/tkadlec/ftp/text/textypo.pdf



% ################################################################# DOCUMENT #

\begin{document}
\pagenumbering{gobble}

% -------------------------------------------------------------- Front cover -

\ifmakecovers
\covergeometry
  \makecoverwith\frontcoverimages
\restoregeometry  
\fi


% ============================================================= FRONT MATTER =

\ifmakefrontmatter

% ------------------------------------------------------------------ Patitul -

\cleardoublepage

\begin{center}
  %\vspace*{2cm}{\fontsize{20}{20}\selectfont\scshape\dictnameCZ\\\dictnameIS}
  \vspace*{\stretch 1}
  {\Huge\scshape\dictnameCZ\\\dictnameIS}
  \vspace{\stretch 4}
\end{center}

% -------------------------------------------------------------------- Titul -

\cleardoublepage

\begin{center}
  %\vspace*{2cm}{\fontsize{24}{24}\selectfont\scshape\dictnameCZ\\\dictnameIS}
  \vspace*{\stretch 1}
  {\HUGE\scshape\dictnameCZ\\\dictnameIS}
  \vspace{\stretch 2}
\end{center}

% ----------------------------------------------------------------- Impresum -

\clearpage

\noindent\begin{minipage}[t][\textheight][t]{\textwidth}
Autoři \textit{Islandsko-českého studijního slovníku}:\\
  \textbf{Aleš Chejn, Ján Zaťko, Jón Gíslason}\\[\baselineskip]
\ifPDF
\else
Autor \textit{Islandské gramatiky}:\\
  \textbf{Vojtěch Kupča}\\[\baselineskip]
\fi
Poděkování:\\
  \textbf{Dorotě Nierychlewské-Chejn, Jiřině Hanzlové, Jiřině Chejnové,
          Simoně Petrásové,\\ Petru Mikešovi, Vojtěchu Kupčovi, 
          Renatě Peškové Emilsson, Amiru Mulahumici, \\ Kristýně Antonové, 
          Martině Kašparové, Lucii Brabcové}\\[\baselineskip]
Finanční příspěvek na výtisk exemplářů určených knihovnám\\ v České republice,
Slovenské republice a na Islandu poskytli:\\
  \textbf{Petr Mikeš, Dorota Nierychlewska-Chejn, Jiřina Chejnová, Otta Chejn}
  \ifPDF
  \\[39\baselineskip]
  \else
  \\[36\baselineskip]
  \fi
\textbf{Hvalur.org - 2016}\\
Licence: \textbf{GNU Free Documentation License}\\
Sazba: \textbf{Paolo Brasolin} v prostředí  \textbf\LaTeX\\
  Zvláštní poděkování patří uživatelům  \textbf{\TeX\ - \LaTeX\ Stack Exchange}\\[\baselineskip]
\ifPDF
Verze: \textbf{PDF verze} \textit{Islandsko-českého studijního slovníku} vhodná pro \textbf{\ifscreen offline prohlížení a vyhledávání\else vlastní tisk\fi}\\ volně dostupná na \url{www.hvalur.org}\\[\baselineskip]
\else
\textbf\ISBN\\[\baselineskip]
\fi
\ifPDF
\else
Tisk: \textbf{Tiskárna Melmen, Pardubice}\\
\fi
Heslová část vysázena patkovým fontem \TeX\ Gyre Pagella,
hesla bezpatkovým fontem {\phvfamily Helvetica}.
\end{minipage}

% -------------------------------------------------------- Table of contents -

\cleardoublepage 

\pdfbookmark\contentsname{toc}    

\tableofcontents

\fi

\pagenumbering{arabic}

% ============================================================= INTRODUCTION =

\ifinputintroduction

% ------------------------------------- The study Icelandic-Czech dictionary -

\chapter{O Islandsko-českém studijním slovníku}
Cílem vytvoření\textit{ Islandsko-českého studijního slovníku} je zpřístupnění islandského jazyka všem zájemcům o~tento jazyk. 

\textit{Islandsko-český studijní slovník} je určen především českým a~slovenským studentům islandštiny, začátečníkům i~pokročilým, kteří zde naleznou ucelené fonetické, morfologické a~syntaktické informace, jakož
i~slovní zásobu potřebnou k~aktivnímu i~pasivnímu užití s~rozsahem vhodným ke studiu islandštiny na vysoké škole.
Slovník je rovněž určen všem zájemcům o~islandštinu a~Island, jimž slovník nabízí encyklopedické informace, úvod do fonetiky a~gramatiky a~velké množství fotografií a~ilustrací. 
Třebaže se jedná o~slovník primárně zaměřený na potřeby českých a~slovenských uživatelů, obsahuje také řadu informací, které usnadní práci islandskému uživateli (např. použití více jak 11 000 synonym a~antonym, překlady příkladů ap.).

Slovní zásoba původně vychází z~\textit{Concise Icelandic-English Dictionary} (\cite {ic_en}), značně však byla rozšířena díky přístupu k~webovému projektu \textit{ISLEX} (\cite {int1}). Velké množství hesel bylo
přidáno na podnět uživatelů online verze \textit{Islandsko-českého studijního slovníku}.

Slovník obsahuje 30~575 hesel s~46~725 významy nebo slovními spojeními. Z~celkového počtu 30~575 je 28~097 svébytných hesel obsahujících alespoň jeden význam nebo slovní spojení. 
Zbylá hesla (v~počtu 2~478) jsou varianty, dublety nebo nepravidelné tvary.

Slovník je primárně online elektronický slovník, volně přístupný na internetových stránkách \url{www.hvalur.org}  (\cite {int14}). 
Tištěná verze slovníku je publikována na vlastní náklady v~počtu 40 kusů, přičemž většina exemplářů je určena knihovnám a~institucím v~České republice, Slovenské republice a~na Islandu
a~několik exemplářů je určeno autorům nebo zájemcům z~řad veřejnosti. Dvě PDF verze (vyhledávání\,/\addthin prohlížení offline a~vlastní tisk) jsou dostupné zdarma na internetových stránkách slovníku (viz výše).

\textit{Islandsko-český studijní slovník} je otevřený svobodný projekt, jehož se v~průběhu tvorby zúčastnilo velké množství spoluautorů (viz Historie tvorby Islandsko-českého studijního slovníku), kterým tímto jménem autorů velmi děkuji.
Děkuji také všem uživatelům online verze slovníku za trpělivost, dobrou vůli a~veškeré připomínky a~návrhy, jmenovitě především Petru Mikešovi, Adamu Kožouškovi a~Samoriele za neutuchající víru v~náš projekt.
Poděkování patří také našim rodinám a~blízkým, díky kterým jsme se mohli soustavně věnovat tvorbě slovníku.
Srdečné poděkování patří mé ženě Dorotce za podporu a~nadšení potřebné k~tvorbě slovníku.

\blspace[5]

{\centering Aleš Chejn, Brwinów, 25. ledna 2016\par}


% ------------ The history of the formation Icelandic-Czech dictionary study -

\chapter{Historie tvorby Islandsko-českého studijního slovníku}
\longchapterskip
\input{introduction/historie}

% -------------------------------------------------  Guide to the dictionary -

\chapter{Průvodce po slovníku}
\begin{multicols}{2}
\input{introduction/instructions}
\end{multicols}

% ---------------------------------------- List of abbreviations and symbols -

\chapter{Seznam použitých zkratek a symbolů}
\begin{multicols}{2}
\begin{description}
%[\breaklabel\setleftmargin{40pt}\setlabelstyle{\bfseries}
%\addtolength{\itemsep}{-0.5\baselineskip}]
\item[{abb}] {zkrácený}
\item[{acc}] {4. pád, akuzativ}
\item[{adj}] {přídavné jméno}
\item[{adv}] {příslovce}
\item[{akt}] {rod činný, aktivum}
\item[{anat.}] {anatomie}
\item[{angl.}] {anglicky}
\item[{ap.}] {a podobně}
\item[{astro.}] {astronomie, výzkum vesmíru}
\item[{básn.}] {básnický, poetický výraz}
\item[{biol.}] {biologie}
\item[{bot.}] {botanika}

\item[{comp}] {2. stupeň stupňování, komparativ}
\item[{con}] {spojovací způsob, konjunktiv}
\item[{conj}] {spojka}
\item[{dat}] {3. pád, dativ}
\item[{def}] {určitý tvar}
\item[{dem}] {zájmeno ukazovací}
\item[{dět.}] {dětsky}

\item[{e-að}] {co (sg nom neživotný) (\textit{eitthvað})}
\item[{e-ð}] {co (sg acc neživotný) (\textit{eitthvað})}
\item[{e-ir}] {kdo (pl nom životný) (\textit{einhverjir})}
\item[{e-ja}] {koho (pl acc životný) (\textit{einhverja})}
\item[{e-jum}] {komu (pl dat životný) (\textit{einhverjum})}
\item[{e-m}] {komu (sg dat životný) (\textit{einhverjum})}
\item[{e-n}] {koho (sg acc životný) (\textit{einhvern})}
\item[{e-r}] {kdo (sg nom životný) (\textit{einhver})}
\item[{e-rra}] {koho (pl gen životný) (\textit{einhverra})}
\item[{e-rs}] {koho (sg gen životný) (\textit{einhvers})}
\item[{e-s}] {čeho (sg gen neživotný) (\textit{einhvers})}
\item[{e-u}] {čemu (sg dat neživotný) (\textit{einhverju})}

\item[{ekon.}] {ekonomika, obchod}
\item[{elek.}] {elektřina}
\item[{f}] {rod ženský}
\item[{filos.}] {filosofie, logika}
\item[{form.}] {formálně}
\item[{fyz.}] {fyzika}
\item[{gen}] {2. pád, genitiv}
\item[{geog.}] {zeměpis, geografie}
\item[{geol.}] {geologie}
\item[{han.}] {hanlivě, pejorativně}
\item[{hist.}] {historie}
\item[{hovor.}] {hovorově}
\item[{hrub.}] {hrubě}
\item[{hud.}] {hudba}
\item[{chem.}] {chemie}
\item[{imper}] {rozkazovací způsob, imperativ}
\item[{impers}] {sloveso neosobní}
\item[{ind}] {oznamovací způsob, indikativ}
\item[{indecl}] {nesklonný}
\item[{indef}] {neurčitý tvar, (pron +) zájmeno neurčité}
\item[{inf}] {infinitiv}
\item[{int}] {tázací zájmeno}
\item[{inter}] {citoslovce}
\item[{jaz.}] {jazykověda}
\item[{kulin.}] {kulinářství, vaření}
\item[{l.}] {latinsky}
\item[{let.}] {letectví}
\item[{lit.}] {literatura, vydavatelství}
\item[{m}] {rod mužský}
\item[{mat.}] {matematika}
\item[{med. }] {lékařství}
\item[{meteo.}] {meteorologie}
\item[{myt.}] {mytologie}
\item[{n}] {rod střední}
\item[{náb.}] {náboženství}
\item[{nám.}] {námořnictví, rybolov}
\item[{nom}] {1. pád, nominativ}
\item[{num}] {číslovka}
\item[{ord}] {řadová číslovka}
\item[{p}] {osoba}
\item[{part}] {částice}
\item[{pers}] {(v +) osoba, (pron +) zájmeno osobní}
\item[{pf}] {minulý čas}
\item[{pl}] {množné číslo, plurál}
\item[{poč.}] {informatika}
\item[{pol.}] {politika, politologie}
\item[{pos (s\,/\addthin w)}] {1. stupeň stupňování, pozitiv (silné\,/\addthin slabé skloňování)}
\item[{poss}] {zájmeno přivlastňovací}
\item[{pov.}] {lidové pověsti, folkloristika}
\item[{pp}] {příčestí minulé}
\item[{praes}] {přítomný čas}
\item[{práv.}] {právnictví, soudnictví}
\item[{predp}] {předpona}
\item[{prep}] {předložka}
\item[{presp}] {příčestí přítomné}
\item[{pron}] {zájmeno}
\item[{prop}] {vlastní jméno, proprium}
\item[{přen.}] {přeneseně}
\item[{přís.}] {přísloví}
\item[{psych.}] {psychologie}
\item[{refl}] {(v +) médium, (pron +) zvratné zájmeno}
\item[{rel}] {vztažné zájmeno}
\item[{sg}] {jednotné číslo, singulár}
\item[{slang.}] {slang}
\item[{sport.}] {sport}
\item[{stav.}] {stavebnictví, architektura}
\item[{subs}] {podstatné jméno (bez rodu)}
\item[{sup (s\,/\addthin w)}] {3. stupeň stupňování, superlativ (silné\,/\addthin slabé skloňování)}
\item[{supin}] {supinum}
\item[{škol.}] {školství}
\item[{techn.}] {technika, mechanika}
\item[{v}] {sloveso}
\item[{voj.}] {vojenství}
\item[{zast.}] {zastarale}
\item[{zkr}] {zkratka}
\item[{zool.}] {zoologie}

\item[{;}] {středník uprostřed definice odděluje věty}

\item[{(...)}] {v kulatých závorkách jsou uvedeny doplňkové informace}
\item[{[...]}] {v hranatých závorkách je uveden fonetický zápis výslovnosti}\footnote{Vysvětlení fonetických symbolů je uvedeno v kapitole Seznam islandských fonémů na straně \pageref{sec:phon_phonems}.}
\item[{|}] {naznačuje místo, kde se deklinační\,/\addthin konjugační koncovka připojuje ke slovnímu základu}
\item[{·}] {vedlejší dělení složeného slova}
\item[{··}] {hlavní dělení složeného slova}
\item[{+}] {plus}
\item[{/}] {lomítko označuje více možností}
\item[{\dicsymSee}] {šipka odkazuje na heslo ve slovníku}
\item[{\dicsymCompare}] {šipka odkazuje na heslo ve slovníku ke srovnání}
\item[{\dicsymExampleIS}] {uvozuje islandský příklad}
\item[{\dicsymExampleCS}] {uvozuje český překlad příkladu}
\item[{\dicsymIdiom}] {uvozuje slovní spojení}
\item[{\dicsymFrequent}] {heslo patří mezi frekventované výrazy}

\item[{\dicsymPhoto}] {fotografie, ilustrace}
\item[{\dicsymProverb}] {přísloví, rčení}

\end{description}
\end{multicols}

\chapter{Skrá yfir skammstafanir og tákn}
\begin{multicols}{2}
\begin{description}
%[\breaklabel\setleftmargin{40pt}\setlabelstyle{\bfseries}
%\addtolength{\itemsep}{-0.5\baselineskip}]
\item[{abb}] {stýfður}
\item[{acc}] {þolfall}
\item[{adj}] {lýsingarorð}
\item[{adv}] {atviksorð}
\item[{akt}] {germynd}
\item[{anat.}] {líffærafræði}
\item[{angl.}] {enska}
\item[{ap.}] {e.þ.h.}
\item[{astro.}] {geimvísindi, stjörnufræði}
\item[{básn.}] {ljóðrænn}
\item[{biol.}] {líffræði}
\item[{bot.}] {grasafræði}

\item[{comp}] {miðstig}
\item[{con}] {viðtengingarháttur}
\item[{conj}] {samtenging}
\item[{dat}] {þágufall}
\item[{def}] {með greini}
\item[{dem}] {ábendingarfornafn}
\item[{dět.}] {barnamál}

\item[{e-að}] {(et. nf. ekki lifandi) (\textit{eitthvað})}
\item[{e-ð}] { (et. þf. ekki lifandi) (\textit{eitthvað})}
\item[{e-ir}] { (ft. nf. lifandi) (\textit{einhverjir})}
\item[{e-ja}] {(ft. þf. lifandi) (\textit{einhverja})}
\item[{e-jum}] {(ft. þgf. lifandi) (\textit{einhverjum})}
\item[{e-m}] {(et. þgf. lifandi) (\textit{einhverjum})}
\item[{e-n}] {(et. þf. lifandi) (\textit{einhvern})}
\item[{e-r}] {(et. nf. lifandi) (\textit{einhver})}
\item[{e-rra}] {(ft. ef. lifandi) (\textit{einhverra})}
\item[{e-rs}] {(et. ef. lifandi) (\textit{einhvers})}
\item[{e-s}] {(et. ef. ekki lifandi) (\textit{einhvers})}
\item[{e-u}] {(et. þgf. ekki lifandi) (\textit{einhverju})}

\item[{ekon.}] {hagfræði, viðskipti}
\item[{elek.}] {rafmagn}
\item[{f}] {kvenkyn}
\item[{filos.}] {heimspeki, rökfræði}
\item[{form.}] {formlegt mál}
\item[{fyz.}] {eðlisfræði}
\item[{gen}] {eignarfall}
\item[{geog.}] {landafræði}
\item[{geol.}] {jarðfræði}
\item[{han.}] {niðrandi}
\item[{hist.}] {sagnfræði}
\item[{hovor.}] {talmál}
\item[{hrub.}] {gróft}
\item[{hud.}] {tónlist}
\item[{chem.}] {efnafræði}
\item[{imper}] {boðháttur}
\item[{impers}] {ópersónuleg sögn}
\item[{ind}] {framsöguháttur}
\item[{indecl}] {óbeygjanlegur}
\item[{indef}] {óákveðinn, (pron +) óákveðið fornafn}
\item[{inf}] {nafnháttur}
\item[{int}] {spurnarfornafn}
\item[{inter}] {upphrópun}
\item[{jaz.}] {málfræði, málvísindi}
\item[{kulin.}] {matreiðsla}
\item[{l.}] {latína}
\item[{let.}] {flugmál}
\item[{lit.}] {skáldskapur}
\item[{m}] {karlkyn}
\item[{mat.}] {stærðfræði}
\item[{med. }] {læknisfræði}
\item[{meteo.}] {veðurfræði}
\item[{myt.}] {goðafræði}
\item[{n}] {hvorugkyn}
\item[{náb.}] {trúarbrögð}
\item[{nám.}] {sjómennska}
\item[{nom}] {nefnifall}
\item[{num}] {töluorð}
\item[{ord}] {raðtala}
\item[{p}] {persóna}
\item[{part}] {ögn}
\item[{pers}] {(v +) persóna, (pron +) persónufornafn}
\item[{pf}] {þátíð}
\item[{pl}] {fleirtala}
\item[{poč.}] {tölvufræði}
\item[{pol.}] {stjórnmál, stjórnmálafræði}
\item[{pos (s\,/\addthin w)}] {frumstig (sterk\,/\addthin veik beyging)}
\item[{poss}] {eignarfornafn}
\item[{pov.}] {þjóðtrú}
\item[{pp}] {lýsingarháttur þátíðar}
\item[{praes}] {nútíð}
\item[{práv.}] {lögfræði, dómsmál}
\item[{predp}] {forskeyti}
\item[{prep}] {forsetning}
\item[{presp}] {lýsingarháttur nútíðar}
\item[{pron}] {fornafn}
\item[{prop}] {sérnafn}
\item[{přen.}] {afleidd merking}
\item[{přís.}] {málsháttur}
\item[{psych.}] {sálfræði}
\item[{refl}] {(v +) miðmynd, (pron +) afturbeygt fornafn}
\item[{rel}] {tilvísunarfornafn}
\item[{sg}] {eintala}
\item[{slang.}] {slangur}
\item[{sport.}] {íþróttir}
\item[{stav.}] {húsasmíði, byggingarlist}
\item[{subs}] {nafnorð (án kyns)}
\item[{sup (s\,/\addthin w)}] {efstastig (sterk\,/\addthin veik beyging)}
\item[{supin}] {sagnbót}
\item[{škol.}] {skólamál}
\item[{techn.}] {tækni, verkfræði}
\item[{v}] {sagnorð}
\item[{voj.}] {hermál}
\item[{zast.}] {úrelt mál}
\item[{zkr}] {skammstöfun}
\item[{zool.}] {dýrafræði}

\item[{;}] {semíkomma í miðri skýringu aðgreinir setningar}

\item[{(...)}] {innan sviga eru viðbótarupplýsingar}
\item[{[...]}] {innan hornklofa er hljóðritaður framburður}
\item[{|}] {merkir stað, þar sem beygingarending bætist við stofn}
\item[{·}] {aukasamskeyti samsetts orðs}
\item[{··}] {aðalsamskeyti samsetts orðs}
\item[{+}] {plús}
\item[{/}] {skástrik táknar fleiri möguleika}
\item[{\dicsymSee}] {ör vísar til uppflettiorðs í orðabókinni}
\item[{\dicsymCompare}] {ör vísar til uppflettiorðs í orðabókinni til samanburðar}
\item[{\dicsymExampleIS}] {kemur á undan dæmi á íslensku}
\item[{\dicsymExampleCS}] {kemur á undan tékkneskri þýðingu á dæminu}
\item[{\dicsymIdiom}] {kemur á undan orðasambandi}
\item[{\dicsymFrequent}] {orð með háa tíðni}

\item[{\dicsymPhoto}] {ljósmynd}
\item[{\dicsymProverb}] {málsháttur}

\end{description}
\end{multicols}

% ------------------------------------------------- Headword part and alphabet

\chapter{Heslová část}
{\renewcommand{\arraystretch}{2}%
%\topskip0pt
\vstretch{1}
{\setlength{\tabcolsep}{1.5em}
\begin{table}[htpb]
\centering

\begin{tabular}{ccccccc} \hline
A a &  & G g &  & O o &  & V v \\
Á á  &  & H h &  & Ó ó &  & W w \\
B b &  & I i &  & P p &  & X x \\
C c &  & Í í &  & Q q &  & Y y \\
D d &  & J j &  & R r &  & Ý ý \\
Ð ð &  & K k &  & S s &  & Z z \\
E e &  & L l &  & T t &  & Þ þ \\
É é  &  & M m &  & U u &  & Æ æ \\
F f &  & N n &  & Ú ú &  & Ö ö \\
\hline
\end{tabular}

%\caption{\textbf{Islandská abeceda}}
\label{my-islandska-abeceda}
\end{table}
}
\vstretch{1}
{\renewcommand{\arraystretch}{1}%

\clearpage

\fi

% =========================================================== THE DICTIONARY =

\ifinputletters

\cleardoublepage

\dictionarygeometry
\pagestyle{myheadings}

%\newcommand*{\alphabet}{%
%  a,aa,b,c,d,dd,e,ee,f,g,h,i,ii,j,k,l,m,
%  %n,o,oo,p,r,s,t,u,uu,v,w,y,yy,th,ae,oe
%  }

%\foreach \l in \alphabet{
%  \dicLetter{l}{letter15}
\dicEntry[l] \dicTerm{l} \dicPos{zkr} \dicPhraseIS{lítri} \dicDirectTranslationCS{litr}
\dicEntry[l.] \dicTerm{l.} \dicPos{zkr} \dicPhraseIS{lýsingarorð} \dicFieldCat{jaz.} \dicDirectTranslationCS{přídavné jméno}
\dicEntry[labb] \dicTerm{labb} \dicIPA{{l}{a}{\textsubring{b}}{\textlengthmark}} \dicPos{n}[2] \dicFlx{(‑s)}[2] \dicDirectTranslationCS{procházka, špacír}
\dicEntry[labba] \dicTerm{labb|a} \dicsymFrequent\  \dicIPA{{l}{a}{\textsubring{b}}{\textlengthmark}{a}} \dicPos{v}[1] \dicFlx{(‑aði)}[13] \dicDirectTranslationCS{projít se, procházet se, jít na procházku, špacírovat} \dicExampleIS{Ég labbaði í bæinn.} \dicExampleCS{Prošel jsem se do města.}
\dicEntry[labbakútur] \dicTerm{labba··kút|ur} \dicIPA{{l}{a}{\textsubring{b}}{\textlengthmark}{a}{k\smash{\textsuperscript{h}}}{u}{\textsubring{d}}{\textscy}{\textsubring{r}}} \dicPos{m}[6] \dicFlx{(‑s, ‑ar)}[22] \textbf{1.} \dicSynonym{barn} \dicDirectTranslationCS{klouček, prcek}  \textbf{2.} \dicSynonym*{mannleysa} \dicDirectTranslationCS{trdlo, janek}
\dicEntry[laða] \dicTerm{lað|a} \dicIPA{{l}{a}{\textlengthmark}{ð}{a}} \dicPos{v}[1] \dicFlx{(‑aði)}[13] \dicFlx{acc} \dicDirectTranslationCS{přitahovat, (při)lákat, (při)vábit} \dicExampleIS{laða e‑n að sér} \dicExampleCS{přilákat (koho) k~sobě};  \dicIdiom{laða}[fram]{ \dicPhraseIS{laða e‑u fram}} \dicDirectTranslationCS{vyvolat (co), přivolat (co), přivodit (co)};  \dicIdiom{laðast}{ \dicPhraseIS{laðast}} \dicFlx{refl} \dicDirectTranslationCS{být přitahován} \dicExampleIS{laðast að e‑u} \dicExampleCS{být přitahován (čím)}
\dicEntry[lafa] \dicTerm{laf|a} \dicIPA{{l}{a}{\textlengthmark}{v}{a}} \dicPos{v}[2] \dicFlx{(‑ði, ‑að)}[125] \textbf{1.} \dicSynonym{hanga} \dicDirectTranslationCS{viset} \dicExampleIS{láta fæturna lafa} \dicExampleCS{nechat nohy viset}  \textbf{2.} \dicDirectTranslationCS{prodlévat, zdržovat se} \dicExampleIS{lafa í þorpinu} \dicExampleCS{zdržovat se na vsi}
\dicEntry[lag] \dicTerm{lag} \dicsymFrequent\  \dicIPA{{l}{a}{\textlengthmark}{x}} \dicPos{n}[2] \dicFlx{(‑s, lög)}[8] \textbf{1.} \dicSynonym{samfella} \dicDirectTranslationCS{vrstva} \dicExampleIS{lag af kolum} \dicExampleCS{vrstva uhlí}  \textbf{2.} \dicDirectTranslationCS{(pro)bodnutí (zbraní ap.)} \dicExampleIS{Lagið nam í hjartastað.} \dicExampleCS{Bodnutí zasáhlo srdce.}  \textbf{3.} \dicSynonym{aðferð} \dicDirectTranslationCS{způsob, metoda};  \dicPhraseIS{finna lagið á e‑u} \dicDirectTranslationCS{přijít na to, jak se (co) dělá};  \dicPhraseIS{hafa lag á e‑u} \dicDirectTranslationCS{mít talent na (co)};  \dicPhraseIS{koma e‑m (upp) á lagið með e‑ð} \dicDirectTranslationCS{naučit (koho), jak se (co) dělá};  \dicPhraseIS{með þessu lagi} \dicFlx{adv} \dicDirectTranslationCS{takto, tímto způsobem}  \textbf{4.} \dicSynonym{magn} \dicDirectTranslationCS{množství};  \dicPhraseIS{í einu lagi} \dicFlx{adv} \dicSynonym*{í heild} \dicDirectTranslationCS{najednou, naráz};  \dicPhraseIS{í fyrsta lagi} \dicFlx{adv} {\textbf{a.}} \dicDirectTranslationCS{ne dříve než} \dicExampleIS{Ég kem í fyrsta lagi kl. þrjú.} \dicExampleCS{Nepřijdu dřív než ve tři.};  {\textbf{b.}} \dicDirectTranslationCS{za prvé, v~první řadě};  \dicPhraseIS{í hæsta lagi} \dicFlx{adv} \dicDirectTranslationCS{nanejvýš, maximálně, nejvýše} \dicExampleIS{Þetta tekur í hæsta lagi fimm mínútur.} \dicExampleCS{Zabere to nejvýše pět minut.};  \dicPhraseIS{í meira lagi} \dicFlx{adv} \dicDirectTranslationCS{dosti, značně};  \dicPhraseIS{í mesta lagi} \dicFlx{adv} \dicDirectTranslationCS{ne více než};  \dicPhraseIS{í öðru lagi} \dicFlx{adv} \dicDirectTranslationCS{za druhé}  \textbf{5.} \dicSynonym{regla} \dicDirectTranslationCS{řád, pořádek};  \dicPhraseIS{allt í lagi!} \dicDirectTranslationCS{ok!, všechno v~pořádku!};  \dicPhraseIS{e‑að er fjarri lagi} \dicDirectTranslationCS{(co) není zdaleka v~pořádku};  \dicPhraseIS{koma e‑u í lag} \dicSynonym{lagfæra} \dicDirectTranslationCS{dát (co) do pořádku};  \dicPhraseIS{nærri lagi} \dicFlx{adv} \dicDirectTranslationCS{téměř v~pořádku}  \textbf{6.} \dicSynonym{tónverk} \dicDirectTranslationCS{píseň, písnička, melodie} \dicExampleIS{lag eftir e‑n} \dicExampleCS{(čí) písnička};  \dicPhraseIS{fara út af laginu} \dicDirectTranslationCS{vypadnout z~rytmu (v~písni), zazpívat falešný tón};  \dicPhraseIS{taka lagið} \dicDirectTranslationCS{(začít) zpívat}  \textbf{7.} \dicSynonym{verðlag} \dicDirectTranslationCS{cena, hodnota};  \dicPhraseIS{leggja lag á e‑ð} \dicDirectTranslationCS{ocenit (co), ohodnotit (co), určit cenu (čeho)}  \textbf{8.} \dicSynonym{samneyti} \dicDirectTranslationCS{společnost, přítomnost} \dicIndirectTranslationCS{(společenský styk)};  \dicPhraseIS{leggja lag sitt við e‑n} \dicDirectTranslationCS{dělat (komu) společnost}  \textbf{9.} \dicPhraseIS{lög} \dicFlx{pl} \dicLink{lög};  \dicIdiom{lag}{ \dicPhraseIS{slá e‑n út af laginu}} \dicDirectTranslationCS{vyvést (koho) z~míry, vytočit (koho)}; { \dicPhraseIS{sæta lagi}} \dicDirectTranslationCS{počkat si (na příležitost)}
\dicEntry[laga] \dicTerm{lag|a} \dicsymFrequent\  \dicIPA{{l}{a}{\textlengthmark}{\textbabygamma}{a}} \dicPos{v}[1] \dicFlx{(‑aði)}[13] \dicFlx{acc} \textbf{1.} \dicSynonym{prýða} \dicDirectTranslationCS{upravit, upravovat, urovnat, urovnávat} \dicExampleIS{laga á sér hárið} \dicExampleCS{upravit si vlasy}  \textbf{2.} \dicSynonym*{koma e‑u í lag} \dicDirectTranslationCS{opravit, opravovat, spravit, spravovat, dát\,/\addthin dávat do pořádku} \dicExampleIS{laga vélina} \dicExampleCS{opravit stroj}  \textbf{3.} \dicDirectTranslationCS{připravit, připravovat (kávu ap.)} \dicExampleIS{laga kaffi og te} \dicExampleCS{připravit kávu a~čaj}  \textbf{4.} \dicSynonym{streyma} \dicDirectTranslationCS{proudit, prýštit (krev z~rány ap.)};  \dicIdiom{laga}[að]{ \dicPhraseIS{laga sig að e‑u}} \dicSynonym{aðhæfa} \dicDirectTranslationCS{přizpůsobit se (čemu)};  \dicIdiom{laga}[til]{ \dicPhraseIS{laga til}} \dicDirectTranslationCS{uklízet, poklízet, uklidit} \dicExampleIS{laga til í íbúðinni} \dicExampleCS{uklidit v~bytě}; { \dicPhraseIS{laga sig til}} \dicDirectTranslationCS{poupravit se};  \dicIdiom{lagast}{ \dicPhraseIS{lagast}} \dicFlx{refl} {\textbf{a.}} \dicDirectTranslationCS{(z)lepšit se, vylepšit se} \dicExampleIS{Ástandið lagast af sjálfu sér.} \dicExampleCS{Situace se zlepšuje sama od sebe.};  {\textbf{b.}} \dicDirectTranslationCS{ustoupit, povolit (bolest ap.)}
\dicEntry[lagaákvæði] \dicTerm{laga··á·kvæði} \dicIPA{{l}{a}{\textlengthmark}{\textbabygamma}{a}{au}{k\smash{\textsuperscript{h}}}{v}{a}{i}{ð}{\textsci}} \dicPos{n}[2] \dicFlx{(‑s, ‑)}[14] \dicFieldCat{práv.} \dicDirectTranslationCS{právní předpis\,/\addthin ustanovení}
\dicEntry[lagabálkur] \dicTerm{laga··bálk|ur} \dicIPA{{l}{a}{\textlengthmark}{\textbabygamma}{a}{\textsubring{b}}{au}{\textsubring{l}}{\r{g}}{\textscy}{\textsubring{r}}} \dicPos{m}[6] \dicFlx{(‑s, ‑ar)}[24] \dicFieldCat{práv.} \dicDirectTranslationCS{kodex, zákoník}
\dicEntry[lagabreyting] \dicTerm{laga··breyt·ing} \dicIPA{{l}{a}{\textlengthmark}{\textbabygamma}{a}{\textsubring{b}}{r}{ei}{\textsubring{d}}{i}{\ng}{\r{g}}} \dicPos{f}[4] \dicFlx{(‑ar, ‑ar)}[5] \dicFieldCat{práv.} \dicDirectTranslationCS{novela zákona}
\dicEntry[lagabrot] \dicTerm{laga··brot} \dicIPA{{l}{a}{\textlengthmark}{\textbabygamma}{a}{\textsubring{b}}{r}{\textopeno}{\textsubring{d}}} \dicPos{n}[2] \dicFlx{(‑s, ‑)}[5] \dicDirectTranslationCS{porušení zákona}
\dicEntry[lagadeild] \dicTerm{laga··deild} \dicIPA{{l}{a}{\textlengthmark}{\textbabygamma}{a}{\textsubring{d}}{ei}{l}{\textsubring{d}}} \dicPos{f}[7] \dicFlx{(‑ar, ‑ir)}[1] \dicDirectTranslationCS{právnická fakulta}
\dicEntry[lagafrumvarp] \dicTerm{laga··frum·|varp} \dicIPA{{l}{a}{\textlengthmark}{\textbabygamma}{a}{f}{r}{\textscy}{m}{v}{a}{\textsubring{r}}{\textsubring{b}}} \dicPos{n}[2] \dicFlx{(‑varps, ‑vörp)}[8] \dicFieldCat{práv.} \dicDirectTranslationCS{návrh zákona}
\dicEntry[lagagrein] \dicTerm{laga··grein} \dicIPA{{l}{a}{\textlengthmark}{\textbabygamma}{a}{\r{g}}{r}{ei}{\textsubring{n}}} \dicPos{f}[4] \dicFlx{(‑ar, ‑ar)}[1] \dicFieldCat{práv.} \dicSynonym{grein} \dicDirectTranslationCS{odstavec zákona, paragraf}
\dicEntry[lagaheimild] \dicTerm{laga··heimild} \dicIPA{{l}{a}{\textlengthmark}{\textbabygamma}{a}{h}{ei}{m}{\textsci}{l}{\textsubring{d}}} \dicPos{f}[7] \dicFlx{(‑ar, ‑ir)}[1] \dicDirectTranslationCS{moc zákona, zákonnost}
\dicEntry[lagahöfundur] \dicTerm{laga··höf·und|ur} \dicIPA{{l}{a}{\textlengthmark}{\textbabygamma}{a}{h}{\oe}{v}{\textscy}{n}{\textsubring{d}}{\textscy}{\textsubring{r}}} \dicPos{m}[6] \dicFlx{(‑ar, ‑ar)}[56] \dicDirectTranslationCS{písničkář(ka), skladatel(ka)}
\dicEntry[lagalegur] \dicTerm{laga··legur} \dicIPA{{l}{a}{\textlengthmark}{\textbabygamma}{a}{l}{\textepsilon}{\textbabygamma}{\textscy}{\textsubring{r}}} \dicPos{adj}[1]\dicFlx{}[-8] \dicDirectTranslationCS{zákonný, právní}
\dicEntry[laganir] \dicTerm{laganir} \dicIPA{{l}{a}{\textlengthmark}{\textbabygamma}{a}{n}{\textsci}{\textsubring{r}}} \dicPos{f} \dicFlx{pl nom} \dicLink{lögun}
\dicEntry[lagar] \dicTerm{lagar} \dicIPA{{l}{a}{\textlengthmark}{\textbabygamma}{a}{\textsubring{r}}} \dicPos{m} \dicFlx{sg gen} \dicLink{lögur}
\dicEntry[lagaregla] \dicTerm{laga··regl|a} \dicIPA{{l}{a}{\textlengthmark}{\textbabygamma}{a}{r}{\textepsilon}{\r{g}}{l}{a}} \dicPos{f}[1] \dicFlx{(‑u, ‑ur)}[19] \dicFieldCat{práv.} \dicDirectTranslationCS{právní předpis}
\dicEntry[lagasafn] \dicTerm{laga··|safn} \dicIPA{{l}{a}{\textlengthmark}{\textbabygamma}{a}{s}{a}{\textsubring{b}}{\textsubring{n}}} \dicPos{n}[2] \dicFlx{(‑safns, ‑söfn)}[8] \dicFieldCat{práv.} \dicDirectTranslationCS{sbírka zákonů}
\dicEntry[lager] \dicTerm{lager} \dicIPA{{l}{a}{\textlengthmark}{\textbabygamma}{\textepsilon}{\textsubring{r}}} \dicPos{m}[4] \dicFlx{(‑s, ‑ar)}[14] \textbf{1.} \dicSynonym{vörubirgðir} \dicDirectTranslationCS{sklad, zásoba} \dicExampleIS{Vörur eru sendar samdægurs séu þær til á lager.} \dicExampleCS{Zboží je odesíláno ve stejný den, pokud je na skladě.};  \dicPhraseIS{hafa e‑ð á lager} \dicDirectTranslationCS{mít (co) na skladě}  \textbf{2.} \dicSynonym{vörugeymsla} \dicDirectTranslationCS{sklad, skladiště} \dicExampleIS{vinna á lager} \dicExampleCS{pracovat ve skladu}
\dicEntry[lagermaður] \dicTerm{lager··|maður} \dicIPA{{l}{a}{\textlengthmark}{\textbabygamma}{\textepsilon}{r}{m}{a}{ð}{\textscy}{\textsubring{r}}} \dicPos{m}[13] \dicFlx{(‑manns, ‑menn)}[2] \dicDirectTranslationCS{skladník, skladnice}
\dicEntry[lagfæra] \dicTerm{lag··fær|a} \dicIPA{{l}{a}{x}{f}{a}{i}{r}{a}} \dicPos{v}[2] \dicFlx{(‑ði, ‑t)}[105] \dicFlx{acc} \textbf{1.} \dicDirectTranslationCS{dát\,/\addthin uvést do pořádku, opravit} \dicExampleIS{lagfæra skemmdir á húsi} \dicExampleCS{dát do pořádku poškození na domě}  \textbf{2.} \dicSynonym{endurbæta} \dicDirectTranslationCS{zlepšit, vylepšit, uzpůsobit}
\dicEntry[lagfæring] \dicTerm{lag··fær·ing} \dicIPA{{l}{a}{x}{f}{a}{i}{r}{i}{\ng}{\r{g}}} \dicPos{f}[4] \dicFlx{(‑ar, ‑ar)}[5] \dicSynonym{endurbót} \dicDirectTranslationCS{oprava, náprava, korekce}
\dicEntry[laggar] \dicTerm{laggar} \dicIPA{{l}{a}{\r{g}}{\textlengthmark}{a}{\textsubring{r}}} \dicPos{f} \dicFlx{sg gen} \dicLink{lögg}
\dicEntry[laggir] \dicTerm{laggir} \dicIPA{{l}{a}{\r{\textObardotlessj}}{\textlengthmark}{\textsci}{\textsubring{r}}} \dicPos{f} \dicFlx{pl nom} \dicLink{lögg}
\dicEntry[laggóður] \textls[15]{\dicTerm{lag··|góður} \dicIPA{{l}{a}{\textbabygamma}{\r{g}}{ou}{ð}{\textscy}{\textsubring{r}}} \dicPos{adj}[11] \dicFlx{(comp ‑betri, sup ‑bestur)}[1] \dicSynonym{hnyttinn} \dicDirectTranslationCS{výstižný, trefný} \dicExampleIS{stutt og laggott svar} \dicExampleCS{krátká a~výstižná odpověď}}
\dicEntry[laghendur] \dicTerm{lag··hendur}\dicTerm{, laghentur} \dicIPA{{l}\-{a}\-{\textlengthmark}\-{x}\-{h}\-{\textepsilon}\-{n}\-{\textsubring{d}}\-{\textscy}\-{\textsubring{r}}\-} \dicPos{adj}[2]\dicFlx{}[-14] \dicSynonym{handlaginn} \dicDirectTranslationCS{obratný, zručný}
\dicEntry[laghentur] \dicTerm{lag··hentur} \dicIPA{{l}{a}{\textlengthmark}{x}{h}{\textepsilon}{\textsubring{n}}{\textsubring{d}}{\textscy}{\textsubring{r}}} \dicPos{adj}[1]\dicFlx{}[-10] \dicLink{laghendur}
\dicEntry[laginn] \dicTerm{laginn} \dicIPA{{l}{a}{i}{j}{\textlengthmark}{\textsci}{\textsubring{n}}} \dicPos{adj}[6]\dicFlx{}[-3] \dicDirectTranslationCS{zručný, šikovný};  \dicPhraseIS{e‑r er laginn í höndunum} \dicDirectTranslationCS{(kdo) má šikovné ruce};  \dicPhraseIS{e‑r er laginn við e‑n} \dicDirectTranslationCS{(kdo) má dobrý přístup ke (komu) (zvířatům ap.)}
\dicEntry[lagkaka] \dicTerm{lag··|kaka} \dicIPA{{l}{a}{\textlengthmark}{\textbabygamma}{k\smash{\textsuperscript{h}}}{a}{\r{g}}{a}} \dicPos{f}[1] \dicFlx{(‑köku, ‑kökur)}[20] \dicFieldCat{kulin.} \dicDirectTranslationCS{řezy (čokoládové ap.), vrstvený koláč, dortík, zákusek}
\dicEntry[laglaus] \dicTerm{lag··laus} \dicIPA{{l}{a}{\textbabygamma}{l}{\oe i}{s}} \dicPos{adj}[5]\dicFlx{}[-1] \dicDirectTranslationCS{(jsoucí) bez hudebního sluchu, hluchý (k~hudbě ap.)}
\dicEntry[laglegur] \dicTerm{lag··legur} \dicsymFrequent\  \dicIPA{{l}{a}{\textbabygamma}{l}{\textepsilon}{\textbabygamma}{\textscy}{\textsubring{r}}} \dicPos{adj}[1]\dicFlx{}[-8] \textbf{1.} \dicSynonym{fríður} \dicDirectTranslationCS{pohledný, půvabný} \dicExampleIS{lagleg stúlka} \dicExampleCS{pohledná dívka}  \textbf{2.} \dicSynonym{snotur} \dicDirectTranslationCS{krásný, hezký}  \textbf{3.} \dicDirectTranslationCS{pěkný, pořádný} \dicIndirectTranslationCS{(k~zdůraznění)}
\dicEntry[laglína] \dicTerm{lag··lín|a} \dicIPA{{l}{a}{\textbabygamma}{l}{i}{n}{a}} \dicPos{f}[1] \dicFlx{(‑u, ‑ur)}[7] \dicDirectTranslationCS{melodie}
\dicEntry[lagnar] \dicTerm{lagnar} \dicIPA{{l}{a}{\r{g}}{n}{a}{\textsubring{r}}} \dicPos{f} \dicFlx{sg gen} \dicLink{lögn}
\dicEntry[lagni] \dicTerm{lagn|i} \dicIPA{{l}{a}{\r{g}}{n}{\textsci}} \dicPos{f}[3] \dicFlx{(‑i)}[3] \dicSynonym{leikni} \dicDirectTranslationCS{šikovnost, zručnost, obratnost}
\dicEntry[lagning] \dicTerm{lag··ning} \dicIPA{{l}{a}{\r{g}}{n}{i}{\ng}{\r{g}}} \dicPos{f}[4] \dicFlx{(‑ar, ‑ar)}[5] \textbf{1.} \dicDirectTranslationCS{kladení, pokládání, instalace} \dicExampleIS{lagning vegar} \dicExampleCS{kladení cesty}  \textbf{2.} \dicSynonym*{hárlagning} \dicDirectTranslationCS{úprava\,/\addthin upravení vlasů}
\dicEntry[lagnir] \dicTerm{lagnir} \dicIPA{{l}{a}{\r{g}}{n}{\textsci}{\textsubring{r}}} \dicPos{f} \dicFlx{pl nom} \dicLink{lögn}
\dicEntry[lagskipting] \dicTerm{lag··skipt·ing} \dicIPA{{l}{a}{x}{s}{\r{\textObardotlessj}}{\textsci}{f}{\textsubring{d}}{i}{\ng}{\r{g}}} \dicPos{f}[4] \dicFlx{(‑ar)}[7] \dicDirectTranslationCS{(roz)vrstvení}
\dicEntry[lagsmaður] \dicTerm{lags··|maður} \dicIPA{{l}{a}{x}{s}{m}{a}{ð}{\textscy}{\textsubring{r}}} \dicPos{m}[13] \dicFlx{(‑manns, ‑menn)}[2] \dicSynonym{félagi} \dicDirectTranslationCS{kamarád(ka), přítel(kyně), společník, společnice}
\dicEntry[lagviss] \dicTerm{lag··viss} \dicIPA{{l}{a}{\textbabygamma}{v}{\textsci}{s}} \dicPos{adj}[5]\dicFlx{}[-1] \dicDirectTranslationCS{zpívající čistě, muzikální}
\dicEntry[lak] \dicTerm{lak\smash{\textsuperscript{1}}} \dicIPA{{l}{a}{\textlengthmark}{\r{g}}} \dicPos{n}[2] \dicFlx{(‑s, lök)}[8] \dicDirectTranslationCS{prostěradlo}
\dicEntry[lak] \dicTerm{lak\smash{\textsuperscript{2}}} \dicIPA{{l}{a}{\textlengthmark}{\r{g}}} \dicPos{v} \dicFlx{ind pf sg 1 pers} \dicLink{leka}
\dicEntry[lakk] \dicTerm{lakk} \dicIPA{{l}{a}{h}{\r{g}}} \dicPos{n}[2] \dicFlx{(‑s, lökk)}[8] \textbf{1.} \dicSynonym*{lakkmálning} \dicDirectTranslationCS{lak, nátěr}  \textbf{2.} \dicSynonym{innsigli} \dicDirectTranslationCS{pečeť}  \textbf{3.} \dicSynonym*{innsiglislakk} \dicDirectTranslationCS{pečetní vosk}
\dicEntry[lakka] \dicTerm{lakk|a} \dicIPA{{l}{a}{h}{\r{g}}{a}} \dicPos{v}[1] \dicFlx{(‑aði)}[13] \dicFlx{acc} \textbf{1.} \dicDirectTranslationCS{(na)lakovat} \dicExampleIS{lakka gluggana} \dicExampleCS{nalakovat okna}  \textbf{2.} \dicDirectTranslationCS{zapečetit} \dicExampleIS{lakka bréf} \dicExampleCS{zapečetit dopis}
\dicEntry[lakkrís] \dicTerm{lakk··rís} \dicIPA{{l}{a}{h}{\r{g}}{r}{i}{s}} \dicPos{m}[4] \dicFlx{(‑s)}[18] \dicDirectTranslationCS{lékořice} \dicExampleIS{fá einn poka af lakkrís} \dicExampleCS{dostat jeden pytlík lékořice}
\dicEntry[lakkrísrót] \dicTerm{lakk·rís··|rót} \dicIPA{{l}{a}{h}{\r{g}}{r}{i}{s}{r}{ou}{\textsubring{d}}} \dicPos{f}[8] \dicFlx{(‑rótar, ‑rætur)}[5] \dicFieldCat{bot.} \dicDirectTranslationCS{kořen lékořice lysé} \textit{(l.~{\textLA{Glycyrrhiza glabra}})}
\dicEntry[laktósi] \dicTerm{laktós|i} \dicIPA{{l}{a}{x}{\textsubring{d}}{ou}{s}{\textsci}} \dicPos{m}[1] \dicFlx{(‑a)}[3] \dicFieldCat{chem.} \dicSynonym*{mjólkursykur} \dicDirectTranslationCS{laktóza, mléčný cukr}
\dicEntry[lakur] \dicTerm{lakur} \dicIPA{{l}{a}{\textlengthmark}{\r{g}}{\textscy}{\textsubring{r}}} \dicPos{adj}[1] \dicFlx{(f lök)}[2] \textbf{1.} \dicSynonym{lélegur} \dicDirectTranslationCS{chabý, mizerný (výsledek ap.)} \dicExampleIS{lakur árangur} \dicExampleCS{mizerný výsledek}  \textbf{2.} \dicSynonym{lasinn} \dicDirectTranslationCS{chorý, neduživý}  \textbf{3.} \dicSynonym{tæpur} \dicDirectTranslationCS{necelý, pouhý}
\dicEntry[lama] \dicTerm{lam|a} \dicIPA{{l}{a}{\textlengthmark}{m}{a}} \dicPos{v}[1] \dicFlx{(‑aði)}[13] \dicDirectTranslationCS{ochromit, paralyzovat} \dicExampleIS{Sjúkdómur lamar líkamann.} \dicExampleCS{Nemoc ochromuje tělo.};  \dicIdiom{lamast}{ \dicPhraseIS{lamast}} \dicFlx{refl} {\textbf{a.}} \dicDirectTranslationCS{ochrnout (o~člověku ap.)} \dicExampleIS{lamast við slysið} \dicExampleCS{ochrnout při nehodě};  {\textbf{b.}} \dicDirectTranslationCS{být paralyzován\,/\addthin ochromen (dopravní ruch ap.)}
\dicEntry[lamadýr] \dicTerm{lama··dýr} \dicIPA{{l}{a}{\textlengthmark}{m}{a}{\textsubring{d}}{i}{\textsubring{r}}} \dicPos{n}[2] \dicFlx{(‑s, ‑)}[5] \dicFieldCat{zool.} \dicDirectTranslationCS{lama} \textit{(l.~{\textLA{Lama}})}  \dicsymPhoto\ 
\dicFigure{ds_image_lamadyr_0_1.jpg}{Lamadýr}{Lamadýr - Luca Galuzzi, CC BY-SA 2.5}
\dicEntry[lamaður] \dicTerm{lam··aður} \dicIPA{{l}{a}{\textlengthmark}{m}{a}{ð}{\textscy}{\textsubring{r}}} \dicPos{adj}[3] \dicFlx{(f lömuð)}[1] \dicDirectTranslationCS{chromý, ochrnutý}
\dicEntry[lamasess] \dicTerm{lama··sess} \dicIPA{{l}{a}{\textlengthmark}{m}{a}{s}{\textepsilon}{s}} \dicPos{m}[4] \dicFlx{(‑)}[34] \dicPhraseIS{liggja\,/\addthin vera í lamasessi} \dicDirectTranslationCS{být mimo provoz, nefungovat (o~věcech)}
\dicEntry[lamb] \dicTerm{lamb} \dicsymFrequent\  \dicIPA{{l}{a}{m}{\textsubring{b}}} \dicPos{n}[2] \dicFlx{(‑s, lömb)}[8] \textbf{1.} \dicDirectTranslationCS{jehně} \dicExampleIS{nýfædd lömb} \dicExampleCS{novorozená jehňata}  \textbf{2.} \dicDirectTranslationCS{drahoušek, miláček};  \dicPhraseIS{launa e‑m lambið gráa} \dicLangCat{přen.} \dicDirectTranslationCS{oplatit (komu) stejnou mincí}
\dicEntry[lambagras] \dicTerm{lamba··|gras} \dicIPA{{l}{a}{m}{\textsubring{b}}{a}{\r{g}}{r}{a}{s}} \dicPos{n}[2] \dicFlx{(‑grass, ‑grös)}[8] \dicFieldCat{bot.} \dicDirectTranslationCS{silenka bezlodyžná} \textit{(l.~{\textLA{Silene acaulis}})}  \dicsymPhoto\ 
\dicFigure{9260.jpg}{Lambagras}{Lambagras - Daněk Pavel, Biolib, Copyright/CC-BY-SA}
\dicEntry[lambakjöt] \dicTerm{lamba··kjöt} \dicIPA{{l}{a}{m}{\textsubring{b}}{a}{c\smash{\textsuperscript{h}}}{\oe}{\textsubring{d}}} \dicPos{n}[2] \dicFlx{(‑s)}[2] \dicDirectTranslationCS{jehněčí (maso)}
\dicEntry[lambalæri] \dicTerm{lamba··læri} \dicIPA{{l}{a}{m}{\textsubring{b}}{a}{l}{a}{i}{r}{\textsci}} \dicPos{n}[2] \dicFlx{(‑s, ‑)}[14] \dicDirectTranslationCS{jehněčí kýta}
\dicEntry[lamdi] \dicTerm{lamdi} \dicIPA{{l}{a}{m}{\textsubring{d}}{\textsci}} \dicPos{v} \dicFlx{ind pf sg 1 pers} \dicLink{lemja}
\dicEntry[lamið] \dicTerm{lamið} \dicIPA{{l}{a}{\textlengthmark}{m}{\textsci}{\texttheta}} \dicPos{v} \dicFlx{supin} \dicLink{lemja}
\dicEntry[lampi] \dicTerm{lamp|i} \dicsymFrequent\  \dicIPA{{l}{a}{\textsubring{m}}{\textsubring{b}}{\textsci}} \dicPos{m}[1] \dicFlx{(‑a, ‑ar)}[8] \textbf{1.} \dicDirectTranslationCS{lampa, svítilna} \dicExampleIS{kveikja á lampa} \dicExampleCS{zapálit lampu}  \textbf{2.} \dicFieldCat{elek.} \dicDirectTranslationCS{výbojka}
\dicEntry[land] \dicTerm{land} \dicsymFrequent\  \dicIPA{{l}{a}{n}{\textsubring{d}}} \dicPos{n}[2] \dicFlx{(‑s, lönd)}[8] \textbf{1.} \dicSynonym{þurrlendi} \dicDirectTranslationCS{souš, pevnina, země} \dicExampleIS{á sjó og landi} \dicExampleCS{na moři a~na pevnině}  \textbf{2.} \dicSynonym{árbakki} \dicDirectTranslationCS{břeh} \dicExampleIS{á öðru landinu} \dicExampleCS{na druhém břehu}  \textbf{3.} \dicSynonym{ríki} \dicDirectTranslationCS{země, stát, vlast} \dicExampleIS{höfuðborg landsins} \dicExampleCS{hlavní město země}  \textbf{4.} \dicSynonym{landareign} \dicDirectTranslationCS{země, půda, pozemek};  \dicIdiom{land}{ \dicPhraseIS{e‑að á langt í land}} \dicLangCat{přen.} \dicDirectTranslationCS{(co) je ještě daleko (cíl ap.)}; { \dicPhraseIS{eiga langt í land}} \dicLangCat{přen.} \dicDirectTranslationCS{mít ještě hodně před sebou (větší část práce ap.)}; { \dicPhraseIS{draga í land}} \dicLangCat{přen.} \dicDirectTranslationCS{ustoupit (ze zásad ap.), rezignovat, upustit (od plánů ap.)}; { \dicPhraseIS{draga e‑n að landi}} \dicDirectTranslationCS{dojíst po (kom)}; { \dicPhraseIS{fara með löndum}} \dicLangCat{přen.} \dicDirectTranslationCS{postupovat\,/\addthin jít opatrně}; { \dicPhraseIS{láta e‑ð lönd og leið}} \dicLangCat{přen.} \dicDirectTranslationCS{nestarat se o~(co), nechat (co) být}; { \dicPhraseIS{leggja land undir fót}} \dicDirectTranslationCS{jít pěšky, pochodovat, kráčet}
\dicEntry[landa] \dicTerm{land|a} \dicIPA{{l}{a}{n}{\textsubring{d}}{a}} \dicPos{v}[1] \dicFlx{(‑aði)}[13] \dicFlx{dat} \dicDirectTranslationCS{vyložit, vylodit} \dicExampleIS{landa aflanum} \dicExampleCS{vyložit úlovek}
\dicEntry[landabréf] \dicTerm{landa··bréf} \dicIPA{{l}{a}{n}{\textsubring{d}}{a}{\textsubring{b}}{r}{j}{\textepsilon}{f}} \dicPos{n}[2] \dicFlx{(‑s, ‑)}[5] \dicSynonym{landakort} \dicDirectTranslationCS{mapa (zeměpisná ap.)}
\dicEntry[landafræði] \dicTerm{landa··fræð|i} \dicIPA{{l}{a}{n}{\textsubring{d}}{a}{f}{r}{a}{i}{ð}{\textsci}} \dicPos{f}[3] \dicFlx{(‑i)}[3] \dicDirectTranslationCS{zeměpis, geografie}
\dicEntry[landakort] \dicTerm{landa··kort} \dicIPA{{l}{a}{n}{\textsubring{d}}{a}{k\smash{\textsuperscript{h}}}{\textopeno}{\textsubring{r}}{\textsubring{d}}} \dicPos{n}[2] \dicFlx{(‑s, ‑)}[5] \dicDirectTranslationCS{mapa}
\dicEntry[landakortabók] \dicTerm{landa·korta··|bók} \dicIPA{{l}{a}{n}{\textsubring{d}}{a}{k\smash{\textsuperscript{h}}}{\textopeno}{\textsubring{r}}{\textsubring{d}}{a}{\textsubring{b}}{ou}{\r{g}}} \dicPos{f}[8] \dicFlx{(‑bókar, ‑bækur)}[5] \dicSynonym{atlas} \dicDirectTranslationCS{atlas}
\dicEntry[landalda] \dicTerm{land··|alda} \dicIPA{{l}{a}{n}{\textsubring{d}}{a}{l}{\textsubring{d}}{a}} \dicPos{f}[1] \dicFlx{(‑öldu, ‑öldur)}[20] \dicDirectTranslationCS{pobřežní vlna}
\dicEntry[landamerki] \dicTerm{landa··merki} \dicIPA{{l}{a}{n}{\textsubring{d}}{a}{m}{\textepsilon}{\textsubring{r}}{\r{\textObardotlessj}}{\textsci}} \dicPos{n}[2] \dicFlx{(‑s, ‑)}[16] \dicSynonym{landamæri} \dicDirectTranslationCS{pomezí, hranice (státní ap.)}
\dicEntry[landamæri] \dicTerm{landa··mæri} \dicIPA{{l}{a}{n}{\textsubring{d}}{a}{m}{a}{i}{r}{\textsci}} \dicPos{n}[2] \dicFlx{pl}[19] \dicDirectTranslationCS{(státní) hranice} \dicExampleIS{vera kominn yfir landamærin} \dicExampleCS{překročit státní hranici}
\dicEntry[landareign] \dicTerm{landar··eign} \dicIPA{{l}{a}{n}{\textsubring{d}}{a}{r}{ei}{\r{g}}{\textsubring{n}}} \dicPos{f}[7] \dicFlx{(‑ar, ‑ir)}[1] \dicDirectTranslationCS{pozemek, půda, pozemkové vlastnictví}
\dicEntry[landbúnaðarframleiðsla] \dicTerm{land·búnaðar··fram·leiðsl|a} \dicIPA{{l}\-{a}\-{n}\-{\textsubring{d}}\-{\textsubring{b}}\-{u}\-{n}\-{a}\-{ð}\-{a}\-{\textsubring{r}}\-{f}\-{r}\-{a}\-{m}\-{l}\-{ei}\-{ð}\-{s}\-{\textsubring{d}}\-{l}\-{a}\-} \dicPos{f}[1] \dicFlx{(‑u)}[5] \dicDirectTranslationCS{zemědělská produkce\,/\addthin výroba}
\dicEntry[landbúnaðarráðherra] \dicTerm{land·búnaðar··ráð·herr|a} \dicIPA{{l}\-{a}\-{n}\-{\textsubring{d}}\-{\textsubring{b}}\-{u}\-{n}\-{a}\-{ð}\-{a}\-{r}\-{au}\-{\texttheta}\-{h}\-{\textepsilon}\-{r}\-{a}\-} \dicPos{m}[1] \dicFlx{(‑a, ‑ar)}[17] \dicDirectTranslationCS{ministr(yně) zemědělství}
\dicEntry[landbúnaðarráðuneyti] \textls[15]{\dicTerm{land·búnaðar··ráðu·neyti} \dicIPA{{l}\-{a}\-{n}\-{\textsubring{d}}\-{\textsubring{b}}\-{u}\-{n}\-{a}\-{ð}\-{a}\-{r}\-{au}\-{ð}\-{\textscy}\-{n}\-{ei}\-{\textsubring{d}}\-{\textsci}}} \dicPos{n}[2] \dicFlx{(‑s, ‑)}[14] \dicDirectTranslationCS{ministerstvo zemědělství}
\dicEntry[landbúnaður] \dicTerm{land··bú·nað|ur} \dicIPA{{l}{a}{n}{\textsubring{d}}{\textsubring{b}}{u}{n}{a}{ð}{\textscy}{\textsubring{r}}} \dicPos{m}[10] \dicFlx{(‑ar)}[9] \dicDirectTranslationCS{zemědělství}
\dicEntry[landeigandi] \dicTerm{land··eig·|andi} \dicIPA{{l}{a}{n}{\textsubring{d}}{ei}{\textbabygamma}{a}{n}{\textsubring{d}}{\textsci}} \dicPos{m}[2] \dicFlx{(‑anda, ‑endur)}[1] \dicDirectTranslationCS{majitel(ka) půdy\,/\addthin pozemku}
\dicEntry[landfastur] \dicTerm{land··|fastur} \dicIPA{{l}{a}{n}{\textsubring{d}}{f}{a}{s}{\textsubring{d}}{\textscy}{\textsubring{r}}} \dicPos{adj}[1] \dicFlx{(f ‑föst)}[11] \dicDirectTranslationCS{(jsoucí) při pobřeží (led ap.)}
\dicEntry[landflótta] \dicTerm{land··flótta} \dicIPA{{l}{a}{n}{\textsubring{d}}{f}{l}{ou}{h}{\textsubring{d}}{a}} \dicPos{adj}[13] \dicFlx{indecl}[1] \dicDirectTranslationCS{(jsoucí) v~exilu, (jsoucí) ve vyhnanství}
\dicEntry[landfræðilegur] \dicTerm{land·fræði··legur} \dicIPA{{l}{a}{n}{\textsubring{d}}{f}{r}{a}{i}{ð}{\textsci}{l}{\textepsilon}{\textbabygamma}{\textscy}{\textsubring{r}}} \dicPos{adj}[1]\dicFlx{}[-8] \dicDirectTranslationCS{geografický}
\dicEntry[landfræðingur] \dicTerm{land·fræð··ing|ur} \dicIPA{{l}{a}{n}{\textsubring{d}}{f}{r}{a}{i}{ð}{i}{\ng}{\r{g}}{\textscy}{\textsubring{r}}} \dicPos{m}[6] \dicFlx{(‑s, ‑ar)}[8] \dicDirectTranslationCS{geograf(ka)}
\dicEntry[landgrunn] \dicTerm{land··grunn} \dicIPA{{l}{a}{n}{\textsubring{d}}{\r{g}}{r}{\textscy}{\textsubring{n}}} \dicPos{n}[2] \dicFlx{(‑s, ‑)}[5] \dicFieldCat{geol.} \dicDirectTranslationCS{kontinentální\,/\addthin pevninský šelf, pevninský práh}
\dicEntry[landgræðsla] \dicTerm{land··græðsl|a} \dicIPA{{l}{a}{n}{\textsubring{d}}{\r{g}}{r}{a}{i}{ð}{s}{\textsubring{d}}{l}{a}} \dicPos{f}[1] \dicFlx{(‑u)}[5] \dicDirectTranslationCS{meliorace}
\dicEntry[landgöngubrú] \dicTerm{land··göngu·|brú} \dicIPA{{l}{a}{n}{\textsubring{d}}{\r{g}}{\oe i}{\ng}{\r{g}}{\textscy}{\textsubring{b}}{r}{u}} \dicPos{f}[9] \dicFlx{(‑brúar, ‑brýr)}[4] \dicDirectTranslationCS{lodní lávka}
\dicEntry[landhelgi] \dicTerm{land··helg|i} \dicsymFrequent\  \dicIPA{{l}{a}{n}{\textsubring{d}}{h}{\textepsilon}{l}{\r{\textObardotlessj}}{\textsci}} \dicPos{f}[3] \dicFlx{(‑i)}[3] \dicDirectTranslationCS{výsostné\,/\addthin teritoriální vody} \dicExampleIS{í landhelgi Íslands} \dicExampleCS{v~islandských výsostných vodách}
\dicEntry[landhelgisgæsla] \dicTerm{land·helgis··gæsl|a} \dicIPA{{l}{a}{n}{\textsubring{d}}{h}{\textepsilon}{l}{\r{\textObardotlessj}}{\textsci}{s}{\r{\textObardotlessj}}{a}{i}{s}{\textsubring{d}}{l}{a}} \dicPos{f}[1] \dicFlx{(‑u)}[5] \dicDirectTranslationCS{pobřežní stráž\,/\addthin hlídka}
\dicEntry[landher] \dicTerm{land··her} \dicIPA{{l}{a}{n}{\textsubring{d}}{h}{\textepsilon}{\textsubring{r}}} \dicPos{m}[9] \dicFlx{(‑s, ‑ir)}[21] \dicDirectTranslationCS{pozemní vojsko}
\dicEntry[landhreinsun] \dicTerm{land··hreins|un} \dicIPA{{l}{a}{n}{\textsubring{d}}{\textsubring{r}}{ei}{n}{s}{\textscy}{\textsubring{n}}} \dicPos{f}[7] \dicFlx{(‑unar)}[9] \dicPhraseIS{það er landhreinsun að e‑m} \dicLangCat{přen.} \dicDirectTranslationCS{no sláva, že už je (kdo) pryč}
\dicEntry[landi] \dicTerm{land|i} \dicIPA{{l}{a}{n}{\textsubring{d}}{\textsci}} \dicPos{m}[1] \dicFlx{(‑a, ‑ar)}[8] \textbf{1.} \dicSynonym{samlandi} \dicDirectTranslationCS{krajan(ka)}  \textbf{2.} \dicSynonym{brugg} \dicDirectTranslationCS{(domácí) pálenka}
\dicEntry[landkrabbi] \dicTerm{land··krabb|i} \dicIPA{{l}{a}{n}{\textsubring{d}}{k\smash{\textsuperscript{h}}}{r}{a}{\textsubring{b}}{\textsci}} \dicPos{m}[1] \dicFlx{(‑a, ‑ar)}[8] \textbf{1.} \dicFieldCat{zool.} \dicDirectTranslationCS{krab} \textit{(l.~{\textLA{Gecarcinus}})}  \dicsymPhoto\   \textbf{2.} \dicDirectTranslationCS{suchozemská krysa} \dicIndirectTranslationCS{(o~člověku, který ví málo o~práci na moři)}
\dicFigure{ds_image_landkrabbi_0_1.jpg}{Landkrabbi}{Landkrabbi - Dozenist, CC BY-SA 3.0}
\dicEntry[landkynning] \dicTerm{land··kynn·ing} \dicIPA{{l}{a}{n}{\textsubring{d}}{c\smash{\textsuperscript{h}}}{\textsci}{n}{i}{\ng}{\r{g}}} \dicPos{f}[4] \dicFlx{(‑ar, ‑ar)}[5] \dicDirectTranslationCS{propagace země}
\dicEntry[landkönnuður] \dicTerm{land··könn·uð|ur} \dicIPA{{l}{a}{n}{\textsubring{d}}{k\smash{\textsuperscript{h}}}{\oe}{n}{\textscy}{ð}{\textscy}{\textsubring{r}}} \dicPos{m}[10] \dicFlx{(‑ar, ‑ir)}[4] \dicDirectTranslationCS{cestovatel(ka), průzkumník, průzkumnice, objevitel(ka) (neznámých končin ap.)}
\dicEntry[landluktur] \dicTerm{land··luktur} \dicIPA{{l}{a}{n}{\textsubring{d}}{l}{\textscy}{x}{\textsubring{d}}{\textscy}{\textsubring{r}}} \dicPos{adj}[1]\dicFlx{}[-10] \dicDirectTranslationCS{vnitrozemský}
\dicEntry[landlægur] \dicTerm{land··lægur} \dicIPA{{l}{a}{n}{\textsubring{d}}{l}{a}{i}{\textbabygamma}{\textscy}{\textsubring{r}}} \dicPos{adj}[1]\dicFlx{}[-1] \textbf{1.} \dicFieldCat{med.} \dicDirectTranslationCS{endemický} \dicExampleIS{landlægur sjúkdómur} \dicExampleCS{endemické onemocnění}  \textbf{2.} \dicSynonym{fastur} \dicDirectTranslationCS{zakořeněný} \dicExampleIS{landlægur siður} \dicExampleCS{zakořeněný zvyk}
\dicEntry[landlæknir] \dicTerm{land··lækn|ir} \dicIPA{{l}{a}{n}{\textsubring{d}}{l}{a}{i}{h}{\r{g}}{n}{\textsci}{\textsubring{r}}} \dicPos{m}[7] \dicFlx{(‑is, ‑ar)}[1] \dicDirectTranslationCS{ředitel(ka) zdravotnictví}
\dicEntry[Landmannalaugar] \dicTerm{Land·manna··laugar} \dicIPA{{l}{a}{n}{\textsubring{d}}{m}{a}{n}{a}{l}{\oe i}{\textbabygamma}{a}{\textsubring{r}}} \dicPos{f}[4] \dicFlx{pl}[2] \dicFieldCat{geog.} \dicDirectTranslationCS{Landmannalaugar} \dicIndirectTranslationCS{(oblast s~horkými prameny a~barevnými skalami na jihu Islandu, oblíbený turistický cíl)}
\dicEntry[landmótunarfræði] \dicTerm{land·mótunar··fræð|i} \dicIPA{{l}{a}{n}{\textsubring{d}}{m}{ou}{\textsubring{d}}{\textscy}{n}{a}{\textsubring{r}}{f}{r}{a}{i}{ð}{\textsci}} \dicPos{f}[3] \dicFlx{(‑i)}[3] \dicDirectTranslationCS{geomorfologie}
\dicEntry[landmælingafræði] \dicTerm{land·mælinga··fræð|i} \dicIPA{{l}{a}{n}{\textsubring{d}}{m}{a}{i}{l}{i}{\ng}{\r{g}}{a}{f}{r}{a}{i}{ð}{\textsci}} \dicPos{f}[3] \dicFlx{(‑i)}[3] \dicDirectTranslationCS{geodézie, zeměměřičství}
\dicEntry[landmælingar] \dicTerm{land··mæl·ingar} \dicIPA{{l}{a}{n}{\textsubring{d}}{m}{a}{i}{l}{i}{\ng}{\r{g}}{a}{\textsubring{r}}} \dicPos{f}[4] \dicFlx{pl}[6] \dicDirectTranslationCS{geodetické měření}
\dicEntry[landnám] \dicTerm{land··nám} \dicIPA{{l}{a}{n}{\textsubring{d}}{n}{au}{\textsubring{m}}} \dicPos{n}[2] \dicFlx{(‑s)}[2] \dicDirectTranslationCS{osídlení, osídlování, kolonizace, kolonizování}

\dicEntry[landnámsmaður] \textls[19]{\dicTerm{land·náms··|maður} \dicIPA{{l}{a}{n}{\textsubring{d}}{n}{au}{m}{s}{m}{a}{ð}{\textscy}{\textsubring{r}}} \dicPos{m}[13] \dicFlx{(‑manns, ‑menn)}[2]} \dicDirectTranslationCS{osadník, osadnice, kolonista, kolonistka}

\dicEntry[landnámsöld] \dicTerm{land·náms··|öld} \dicIPA{{l}{a}{n}{\textsubring{d}}{n}{au}{m}{s}{\oe}{l}{\textsubring{d}}} \dicPos{f}[7] \dicFlx{(‑aldar)}[19] \dicFieldCat{hist.} \dicDirectTranslationCS{doba osídlování}
\dicEntry[landnemi] \dicTerm{land··nem|i} \dicIPA{{l}{a}{n}{\textsubring{d}}{n}{\textepsilon}{m}{\textsci}} \dicPos{m}[1] \dicFlx{(‑a, ‑ar)}[1] \dicDirectTranslationCS{osadník, osadnice, kolonista, kolonistka} \dicExampleIS{íslensku landnemarnir í Vesturheimi} \dicExampleCS{islandští osadníci v~Severní Americe}
\dicEntry[landráð] \dicTerm{land··ráð} \dicIPA{{l}{a}{n}{\textsubring{d}}{r}{au}{\texttheta}} \dicPos{n}[2] \dicFlx{pl}[1] \dicFieldCat{práv.} \dicDirectTranslationCS{velezrada, vlastizrada}
\dicEntry[landrek] \dicTerm{land··rek} \dicIPA{{l}{a}{n}{\textsubring{d}}{r}{\textepsilon}{\r{g}}} \dicPos{n}[2] \dicFlx{(‑s)}[2] \dicFieldCat{geol.} \dicDirectTranslationCS{kontinentální drift}
\dicEntry[landsbanki] \dicTerm{lands··bank|i} \dicIPA{{l}{a}{n}{\textsubring{d}}{s}{\textsubring{b}}{au}{\r{\textltailn}}{\r{\textObardotlessj}}{\textsci}} \dicPos{m}[1] \dicFlx{(‑a, ‑ar)}[8] \dicDirectTranslationCS{národní banka}
\dicEntry[landsbyggð] \dicTerm{lands··byggð} \dicIPA{{l}{a}{n}{\textsubring{d}}{s}{\textsubring{b}}{\textsci}{\textbabygamma}{\texttheta}} \dicPos{f}[7] \dicFlx{(‑ar)}[3] \dicDirectTranslationCS{venkov, venkovský kraj} \dicExampleIS{úti á landsbyggðinni} \dicExampleCS{na venkově}
\dicEntry[landsfrægur] \dicTerm{lands··frægur} \dicIPA{{l}{a}{n}{\textsubring{d}}{s}{f}{r}{a}{i}{\textbabygamma}{\textscy}{\textsubring{r}}} \dicPos{adj}[1]\dicFlx{}[-1] \dicDirectTranslationCS{známý\,/\addthin proslavený po celé zemi}
\dicEntry[landshluti] \dicTerm{lands··hlut|i} \dicIPA{{l}{a}{n}{\textsubring{d}}{s}{\textsubring{l}}{\textscy}{\textsubring{d}}{\textsci}} \dicPos{m}[1] \dicFlx{(‑a, ‑ar)}[1] \dicDirectTranslationCS{část země, region}
\dicEntry[landsími] \dicTerm{land··sím|i} \dicIPA{{l}{a}{n}{\textsubring{d}}{s}{i}{m}{\textsci}} \dicPos{m}[1] \dicFlx{(‑a, ‑ar)}[1] \dicFieldCat{hist.} \dicDirectTranslationCS{národní telefonní společnost}
\dicEntry[landskjálfti] \dicTerm{land··skjálft|i} \dicIPA{{l}{a}{n}{\textsubring{d}}{s}{\r{\textObardotlessj}}{au}{\textsubring{l}}{\textsubring{d}}{\textsci}} \dicPos{m}[1] \dicFlx{(‑a, ‑ar)}[1] \dicFieldCat{geol.} \dicLangCat{zast.} \dicSynonym{jarðskjálfti} \dicDirectTranslationCS{zemětřesení}
\dicEntry[landslag] \dicTerm{lands··lag} \dicsymFrequent\  \dicIPA{{l}{a}{n}{\textsubring{d}}{s}{l}{a}{x}} \dicPos{n}[2] \dicFlx{(‑s)}[2] \dicDirectTranslationCS{krajina, terén} \dicExampleIS{skoða landslag í firðinum} \dicExampleCS{zkoumat krajinu ve fjordu}
\dicEntry[landslagsarkitekt] \dicTerm{lands·lags··arkitekt} \dicIPA{{l}{a}{n}{\textsubring{d}}{s}{l}{a}{x}{s}{a}{\textsubring{r}}{\r{\textObardotlessj}}{\textsci}{\textsubring{d}}{\textepsilon}{x}{\textsubring{d}}} \dicPos{m}[4] \dicFlx{(‑s, ‑ar)}[10] \dicDirectTranslationCS{krajinný architekt, krajinná architektka}
\dicEntry[landslagsmynd] \dicTerm{lands·lags··mynd} \dicIPA{{l}{a}{n}{\textsubring{d}}{s}{l}{a}{x}{s}{m}{\textsci}{n}{\textsubring{d}}} \dicPos{f}[7] \dicFlx{(‑ar, ‑ir)}[1] \dicDirectTranslationCS{krajina (obraz), krajinomalba}
\dicEntry[landsleikur] \dicTerm{lands··leik|ur} \dicIPA{{l}{a}{n}{\textsubring{d}}{s}{l}{ei}{\r{g}}{\textscy}{\textsubring{r}}} \dicPos{m}[9] \dicFlx{(‑s, ‑ir)}[15] \dicDirectTranslationCS{mezistátní\,/\addthin mezinárodní zápas\,/\addthin utkání (ve fotbale ap.)}
\dicEntry[landslið] \dicTerm{lands··lið} \dicIPA{{l}{a}{n}{\textsubring{d}}{s}{l}{\textsci}{\texttheta}} \dicPos{n}[2] \dicFlx{(‑s, ‑)}[5] \dicDirectTranslationCS{národní tým\,/\addthin mužstvo}
\dicEntry[landsliðsþjálfari] \dicTerm{lands·liðs··þjálf·ar|i} \dicIPA{{l}\-{a}\-{n}\-{\textsubring{d}}\-{s}\-{l}\-{\textsci}\-{ð}\-{s}\-{\texttheta}\-{j}\-{au}\-{l}\-{v}\-{a}\-{r}\-{\textsci}\-} \dicPos{m}[1] \dicFlx{(‑a, ‑ar)}[13] \dicDirectTranslationCS{trenér(ka) národního týmu}
\dicEntry[landslýður] \dicTerm{lands··lýð|ur} \dicIPA{{l}{a}{n}{\textsubring{d}}{s}{l}{i}{ð}{\textscy}{\textsubring{r}}} \dicPos{m}[9] \dicFlx{(‑s, ‑ir)}[8] \dicDirectTranslationCS{obyvatelstvo}
\dicEntry[landsmaður] \dicTerm{lands··|maður} \dicsymFrequent\  \dicIPA{{l}{a}{n}{\textsubring{d}}{s}{m}{a}{ð}{\textscy}{\textsubring{r}}} \dicPos{m}[13] \dicFlx{(‑manns, ‑menn)}[2] \dicDirectTranslationCS{obyvatel(ka)} \dicExampleIS{allir landsmenn undir fimmtugu} \dicExampleCS{všichni obyvatelé pod padesát let}
\dicEntry[landsmót] \dicTerm{lands··mót} \dicIPA{{l}{a}{n}{\textsubring{d}}{s}{m}{ou}{\textsubring{d}}} \dicPos{n}[2] \dicFlx{(‑s, ‑)}[5] \textbf{1.} \dicDirectTranslationCS{celorepublikový\,/\addthin národní sjezd\,/\addthin setkání}  \textbf{2.} \dicDirectTranslationCS{celorepublikový\,/\addthin národní turnaj (v~házené ap.)}
\dicEntry[landspítali] \dicTerm{land··spítal|i} \dicIPA{{l}{a}{n}{\textsubring{d}}{s}{\textsubring{b}}{i}{t\smash{\textsuperscript{h}}}{a}{l}{\textsci}} \dicPos{m}[1] \dicFlx{(‑a, ‑ar)}[8] \dicDirectTranslationCS{národní univerzitní nemocnice}
\begin{xtolerant}{}{1pt}
\dicEntry[landssamband] \dicTerm{lands··sam·|band} \dicIPA{{l}{a}{n}{\textsubring{d}}{s}{a}{m}{\textsubring{b}}{a}{n}{\textsubring{d}}} \dicPos{n}[2] \dicFlx{(‑bands, ‑bönd)}[8] \dicDirectTranslationCS{národní\,/\addthin celorepublikový svaz}
\end{xtolerant}
\dicEntry[landstjóri] \dicTerm{land··stjór|i} \dicIPA{{l}{a}{n}{\textsubring{d}}{s}{\textsubring{d}}{j}{ou}{r}{\textsci}} \dicPos{m}[1] \dicFlx{(‑a, ‑ar)}[1] \dicDirectTranslationCS{místodržitel(ka), guvernér(ka), zemský správce, zemská správkyně}
\dicEntry[landsvala] \dicTerm{land··|svala} \dicIPA{{l}{a}{n}{\textsubring{d}}{s}{v}{a}{l}{a}} \dicPos{f}[1] \dicFlx{(‑svölu, ‑svölur)}[20] \dicFieldCat{zool.} \dicDirectTranslationCS{vlaštovka, vlaštovka obecná} \textit{(l.~{\textLA{Hirundo rustica}})}  \dicsymPhoto\ 
\dicFigure{20731.jpg}{Landsvala}{Landsvala - Menke, Dave, Biolib, PD}
\dicEntry[Landsvirkjun] \dicTerm{Lands··virkj|un} \dicIPA{{l}{a}{n}{\textsubring{d}}{s}{v}{\textsci}{\textsubring{r}}{\r{\textObardotlessj}}{\textscy}{\textsubring{n}}} \dicPos{f}[7] \dicFlx{(‑unar)}[10] \dicIndirectTranslationCS{islandská společnost provozující elektrárny}
\dicEntry[landsvæði] \dicTerm{land··svæði} \dicIPA{{l}{a}{n}{\textsubring{d}}{s}{v}{a}{i}{ð}{\textsci}} \dicPos{n}[2] \dicFlx{(‑s, ‑)}[14] \dicDirectTranslationCS{území, oblast, teritorium}
\dicEntry[langa] \dicTerm{langa\smash{\textsuperscript{1}}} \dicIPA{{l}{au}{\ng}{\r{g}}{a}} \dicPos{f}[1] \dicFlx{(löngu, löngur)}[20] \dicFieldCat{zool.} \dicDirectTranslationCS{mník mořský} \textit{(l.~{\textLA{Molva molva}})}
\dicEntry[langa] \dicTerm{lang|a\smash{\textsuperscript{2}}} \dicsymFrequent\  \dicIPA{{l}{au}{\ng}{\r{g}}{a}} \dicPos{v}[1] \dicFlx{(‑aði)}[79] \dicFlx{impers} \dicPhraseIS{e‑n langar} \dicDirectTranslationCS{(kdo) touží, (kdo) chce, (kdo) má chuť} \dicExampleIS{Mig langar að fara.} \dicExampleCS{Chce se mi jít.};  \dicPhraseIS{e‑n langar í e‑ð} \dicDirectTranslationCS{(kdo) touží po (čem), (kdo) chce (co), (kdo) má chuť na (co)} \dicExampleIS{Mig langar í sælgæti.} \dicExampleCS{Mám chuť na něco sladkého.};  \dicPhraseIS{e‑n langar til e‑s} \dicDirectTranslationCS{(kdo) touží po (čem), (kdo) chce (co)} \dicExampleIS{Mig langar til að lesa bókina.} \dicExampleCS{Mám chuť si číst knížku.};  \dicPhraseIS{e‑n langar eftir e‑u} \dicDirectTranslationCS{(kdo) touží po (čem), (kdo) prahne po (čem)} \dicExampleIS{Hana langaði eftir friði.} \dicExampleCS{Toužila po klidu.}
\dicEntry[langafasta] \dicTerm{langa··fasta} \dicIPA{{l}{au}{\ng}{\r{g}}{a}{f}{a}{s}{\textsubring{d}}{a}} \dicPos{f}[1] \dicFlx{(lönguföstu, lönguföstur)}[21] \dicFieldCat{náb.} \dicDirectTranslationCS{půst} \dicIndirectTranslationCS{(trvající 7 týdnů od Popeleční středy do Velikonoc)}
\dicEntry[langafi] \dicTerm{lang··af|i} \dicIPA{{l}{au}{\ng}{\r{g}}{a}{v}{\textsci}} \dicPos{m}[1] \dicFlx{(‑a, ‑ar)}[8] \dicDirectTranslationCS{pradědeček} \dicIndirectTranslationCS{(otec dědečka nebo babičky)}
\dicEntry[langamma] \dicTerm{lang··|amma} \dicIPA{{l}{au}{\ng}{\r{g}}{a}{m}{a}} \dicPos{f}[1] \dicFlx{(‑ömmu, ‑ömmur)}[8] \dicDirectTranslationCS{prababička} \dicIndirectTranslationCS{(matka dědečka nebo babičky)}
\dicEntry[langanir] \dicTerm{langanir} \dicIPA{{l}{au}{\ng}{\r{g}}{a}{n}{\textsci}{\textsubring{r}}} \dicPos{f} \dicFlx{pl nom} \dicLink{löngun}
\dicEntry[langatöng] \dicTerm{langa··töng}\dicTerm{, löngutöng} \dicIPA{{l}\-{au}\-{\ng}\-{\r{g}}\-{a}\-{t\smash{\textsuperscript{h}}}\-{\oe i}\-{\ng}\-{\r{g}}\-} \dicPos{f}[8] \dicFlx{(löngutangar, löngutengur\,/\addthin löngutangir)}[10] \dicDirectTranslationCS{prostředník (prst), prostředníček}
\dicEntry[langbestur] \dicTerm{lang··bestur} \dicIPA{{l}{au}{\ng}{\r{g}}{\textsubring{b}}{\textepsilon}{s}{\textsubring{d}}{\textscy}{\textsubring{r}}} \dicPos{adj}[11]\dicFlx{}[-2] \dicFlx{m sg nom sup} \dicDirectTranslationCS{úplně\,/\addthin zdaleka nejlepší}
\dicEntry[langbylgja] \dicTerm{lang··bylgj|a} \dicIPA{{l}{au}{\ng}{\r{g}}{\textsubring{b}}{\textsci}{l}{\r{\textObardotlessj}}{a}} \dicPos{f}[1] \dicFlx{(‑u, ‑ur)}[25] \dicFieldCat{fyz.} \dicDirectTranslationCS{dlouhá vlna} \dicAntonym{stuttbylgja}
\dicEntry[langdreginn] \dicTerm{lang··dreginn} \dicIPA{{l}{au}{\ng}{\r{g}}{\textsubring{d}}{r}{ei}{\textsci}{\textsubring{n}}} \dicPos{adj}[6]\dicFlx{}[-2] \textbf{1.} \dicDirectTranslationCS{zdlouhavý, rozvláč\-ný} \dicExampleIS{langdreginn ræða} \dicExampleCS{zdlouhavý proslov}  \textbf{2.} \dicDirectTranslationCS{dlouhotrvající} \dicExampleIS{langdreginn sjúkdómur} \dicExampleCS{dlouhotrvající onemocnění}
\dicEntry[langdvöl] \dicTerm{lang··|dvöl} \dicIPA{{l}{au}{\ng}{\r{g}}{\textsubring{d}}{v}{\oe}{\textsubring{l}}} \dicPos{f}[7] \dicFlx{(‑dvalar, ‑dvalir)}[16] \dicDirectTranslationCS{dlouhotrvající pobyt};  \dicPhraseIS{dvelja\,/\addthin vera e‑s staðar langdvölum} \dicDirectTranslationCS{pobývat (kde) dlouho}
\dicEntry[langeygur] \dicTerm{lang··eygur} \dicIPA{{l}{au}{\ng}{\r{g}}{ei}{\textbabygamma}{\textscy}{\textsubring{r}}} \dicPos{adj}[1]\dicFlx{}[-1] \dicPhraseIS{vera langeygur eftir e‑u} \dicDirectTranslationCS{být nedočkavý (čeho), nemoct se dočkat (čeho)}
\dicEntry[langframi] \dicTerm{lang··fram|i} \dicIPA{{l}{au}{\ng}{\r{g}}{f}{r}{a}{m}{\textsci}} \dicPos{m}[1] \dicFlx{(‑a)}[3] \dicPhraseIS{til langframa} \dicFlx{adv} \dicDirectTranslationCS{dlouhodobě}
\dicEntry[langhlaup] \dicTerm{lang··hlaup} \dicIPA{{l}{au}{\ng}{\r{g}}{\textsubring{l}}{\oe i}{\textsubring{b}}} \dicPos{n}[2] \dicFlx{(‑s, ‑)}[5] \dicFieldCat{sport.} \dicDirectTranslationCS{dálkový běh}
\dicEntry[langlífi] \dicTerm{lang··lífi} \dicIPA{{l}{au}{\ng}{\r{g}}{l}{i}{v}{\textsci}} \dicPos{n}[2] \dicFlx{(‑s)}[20] \dicDirectTranslationCS{dlouhověkost}
\dicEntry[langlífur] \dicTerm{lang··lífur} \dicIPA{{l}{au}{\ng}{\r{g}}{l}{i}{v}{\textscy}{\textsubring{r}}} \dicPos{adj}[1]\dicFlx{}[-1] \dicDirectTranslationCS{dlouhověký}
\dicEntry[langloka] \dicTerm{lang··lok|a} \dicIPA{{l}{au}{\ng}{\r{g}}{l}{\textopeno}{\r{g}}{a}} \dicPos{f}[1] \dicFlx{(‑u, ‑ur)}[19] \textbf{1.} \dicSynonym{romsa\smash{\textsuperscript{1}}} \dicDirectTranslationCS{litanie, nudná a~zdlouhavá řeč, nudný a~zdlouhavý text}  \textbf{2.} \dicSynonym{samloka} \dicDirectTranslationCS{bageta}
\dicEntry[langorður] \dicTerm{lang··orður} \dicIPA{{l}{au}{\ng}{\r{g}}{\textopeno}{r}{ð}{\textscy}{\textsubring{r}}} \dicPos{adj}[2]\dicFlx{}[-1] \dicDirectTranslationCS{rozvláčný, upovídaný}
\dicEntry[langreyður] \dicTerm{lang··reyð|ur} \dicIPA{{l}{au}{\ng}{\r{g}}{r}{ei}{ð}{\textscy}{\textsubring{r}}} \dicPos{f}[5] \dicFlx{(‑ar, ‑ar)}[1] \dicFieldCat{zool.} \dicDirectTranslationCS{plejtvák myšok} \textit{(l.~{\textLA{Balaenoptera physalus}})}
\dicEntry[langrækinn] \dicTerm{lang··rækinn} \dicIPA{{l}{au}{\ng}{\r{g}}{r}{a}{i}{\r{\textObardotlessj}}{\textsci}{\textsubring{n}}} \dicPos{adj}[6]\dicFlx{}[-2] \dicDirectTranslationCS{neodpouštějící, nesmiřitelný}
\dicEntry[langskip] \dicTerm{lang··skip} \dicIPA{{l}{au}{\ng}{\r{g}}{s}{\r{\textObardotlessj}}{\textsci}{\textsubring{b}}} \dicPos{n}[2] \dicFlx{(‑s, ‑)}[5] \dicFieldCat{nám.} \dicDirectTranslationCS{vikinská válečná loď}
\dicEntry[langsnið] \dicTerm{lang··snið} \dicIPA{{l}{au}{\ng}{\r{g}}{s}{\textsubring{d}}{n}{\textsci}{\texttheta}} \dicPos{n}[2] \dicFlx{(‑s, ‑)}[5] \dicFieldCat{poč.} \dicDirectTranslationCS{formát na délku}
\dicEntry[langsóttur] \dicTerm{lang··sóttur} \dicIPA{{l}{au}{\ng}{\r{g}}{s}{ou}{h}{\textsubring{d}}{\textscy}{\textsubring{r}}} \dicPos{adj}[1]\dicFlx{}[-10] \dicDirectTranslationCS{nevěrohodný, přitažený za vlasy} \dicExampleIS{langsótt sönnun} \dicExampleCS{nevěrohodný důkaz}
\dicEntry[langspil] \dicTerm{lang··spil} \dicIPA{{l}{au}{\ng}{\r{g}}{s}{\textsubring{b}}{\textsci}{\textsubring{l}}} \dicPos{n}[2] \dicFlx{(‑s, ‑)}[5] \dicFieldCat{hud.} \dicIndirectTranslationCS{islandský strunný hudební nástroj}
\dicEntry[langstökk] \dicTerm{lang··stökk} \dicIPA{{l}{au}{\ng}{\r{g}}{s}{\textsubring{d}}{\oe}{h}{\r{g}}} \dicPos{n}[2] \dicFlx{(‑s, ‑)}[5] \dicFieldCat{sport.} \dicDirectTranslationCS{skok do dálky, skok daleký}
\dicEntry[langt] \dicTerm{langt} \dicsymFrequent\  \dicIPA{{l}{au}{\r{\ng}}{\textsubring{d}}} \dicPos{adv} \dicFlx{(comp lengra, sup lengst)} \dicDirectTranslationCS{daleko} \dicExampleIS{Þetta liggur langt inni í fortíðinni.} \dicExampleCS{Má to původ v~daleké minulosti.};  \dicPhraseIS{langt í burtu} \dicFlx{adv} \dicDirectTranslationCS{daleko};  \dicPhraseIS{langt síðan} \dicFlx{adv} \dicDirectTranslationCS{dávno}
\dicEntry[langtíma] \dicTerm{lang··tíma-} \dicIPA{{l}{au}{\ng}{\r{g}}{t\smash{\textsuperscript{h}}}{i}{m}{a}} \dicPos{predp} \dicDirectTranslationCS{dlouhodobý}
\dicEntry[langtímalán] \dicTerm{lang·tíma··lán} \dicIPA{{l}{au}{\ng}{\r{g}}{t\smash{\textsuperscript{h}}}{i}{m}{a}{l}{au}{\textsubring{n}}} \dicPos{n}[2] \dicFlx{(‑s, ‑)}[5] \dicFieldCat{ekon.} \dicDirectTranslationCS{dlouhodobá půjčka}
\dicEntry[langtímasamningur] \dicTerm{lang·tíma··samn·ing|ur} \dicIPA{{l}\-{au}\-{\ng}\-{\r{g}}\-{t\smash{\textsuperscript{h}}}\-{i}\-{m}\-{a}\-{s}\-{a}\-{m}\-{n}\-{i}\-{\ng}\-{\r{g}}\-{\textscy}\-{\textsubring{r}}\-} \dicPos{m}[6] \dicFlx{(‑s, ‑ar)}[8] \dicDirectTranslationCS{dlouhodobá smlouva}
\dicEntry[langur] \dicTerm{lang|ur\smash{\textsuperscript{1}}} \dicIPA{{l}{au}{\ng}{\r{g}}{\textscy}{\textsubring{r}}} \dicPos{m}[6] \dicFlx{(‑s, ‑ar)}[16] \dicPhraseIS{draga e‑ð á langinn} \dicDirectTranslationCS{odložit (co), odkládat (co)}
\dicEntry[langur] \dicTerm{langur\smash{\textsuperscript{2}}} \dicsymFrequent\  \dicIPA{{l}{au}{\ng}{\r{g}}{\textscy}{\textsubring{r}}} \dicPos{adj}[10] \dicFlx{(f löng, comp lengri, sup lengstur)}[2] \textbf{1.} \dicDirectTranslationCS{dlouhý} \dicIndirectTranslationCS{(o~vzdálenosti)} \dicExampleIS{löng leið} \dicExampleCS{dlouhá cesta} \dicAntonym{stuttur}  \textbf{2.} \dicDirectTranslationCS{vysoký} \dicExampleIS{langur maður} \dicExampleCS{vysoký člověk}  \textbf{3.} \dicDirectTranslationCS{dlouhý} \dicIndirectTranslationCS{(o~čase)} \dicExampleIS{langur tími} \dicExampleCS{dlouhý čas};  \dicPhraseIS{fyrir löngu} \dicFlx{adv} \dicDirectTranslationCS{dávno, před dlouhou dobou};  \dicPhraseIS{löngum stundum} \dicFlx{adv} \dicDirectTranslationCS{dlouho, dlouhé hodiny};  \dicPhraseIS{það er langt til e‑s} \dicDirectTranslationCS{je daleko do (čeho) (Vánoc ap.)}
\dicEntry[langvarandi] \dicTerm{lang··var·andi} \dicsymFrequent\  \dicIPA{{l}{au}{\ng}{\r{g}}{v}{a}{r}{a}{n}{\textsubring{d}}{\textsci}} \dicPos{adj}[13] \dicFlx{indecl}[1] \dicDirectTranslationCS{dlouhotrvající, prodlužovaný} \dicExampleIS{langvarandi erfiðleikar} \dicExampleCS{dlouhotrvající potíže}
\dicEntry[langvinnur] \textls[15]{\dicTerm{lang··vinnur} \dicIPA{{l}{au}{\ng}{\r{g}}{v}{\textsci}{n}{\textscy}{\textsubring{r}}} \dicPos{adj}[1]\dicFlx{}[-1] \dicDirectTranslationCS{přetrvávající, táhlý, dlouhotrvající} \dicExampleIS{langvinnur sjúkdómur} \dicExampleCS{přetrvávající onemocnění}}
\dicEntry[langvía] \dicTerm{lang··ví|a} \dicIPA{{l}{au}{\ng}{\r{g}}{v}{i}{j}{a}} \dicPos{f}[1] \dicFlx{(‑u, ‑ur)}[7] \dicFieldCat{zool.} \dicDirectTranslationCS{alkoun úzkozobý} \textit{(l.~{\textLA{Uria aalge}})}  \dicsymPhoto\ 
\dicFigure{20877.jpg}{Langvía}{Langvía - Sowls, Art, Biolib, PD}
\dicEntry[langþráður] \dicTerm{lang··þráður} \dicIPA{{l}{au}{\ng}{\r{g}}{\texttheta}{r}{au}{ð}{\textscy}{\textsubring{r}}} \dicPos{adj}[2]\dicFlx{}[-12] \dicDirectTranslationCS{vytoužený, (dlouho) očekávaný}
\dicEntry[langþreyttur] \dicTerm{lang··þreyttur} \dicIPA{{l}{au}{\ng}{\r{g}}{\texttheta}{r}{ei}{h}{\textsubring{d}}{\textscy}{\textsubring{r}}} \dicPos{adj}[1]\dicFlx{}[-13] \dicDirectTranslationCS{znavený, (velmi) unavený}
\dicEntry[lantan] \dicTerm{lantan} \dicIPA{{l}{a}{\textsubring{n}}{\textsubring{d}}{a}{\textsubring{n}}} \dicPos{n}[2] \dicFlx{(‑s)}[2] \dicFieldCat{chem.} \dicDirectTranslationCS{lanthan} \textit{(l.~{\textLA{La, Lanthanum}})}
\dicEntry[Laos] \dicTerm{Laos} \dicIPA{{l}{a}{\textlengthmark}{\textopeno}{s}} \dicPos{n}[4] \dicFlx{indecl}[2] \dicFieldCat{geog.} \dicDirectTranslationCS{Laos}
\dicEntry[Laosi] \dicTerm{Laos|i} \dicIPA{{l}{a}{\textlengthmark}{\textopeno}{s}{\textsci}} \dicPos{m}[1] \dicFlx{(‑a, ‑ar)}[1] \dicDirectTranslationCS{Laosan(ka)}
\dicEntry[laoska] \dicTerm{laosk|a} \dicIPA{{l}{a}{\textlengthmark}{\textopeno}{s}{\r{g}}{a}} \dicPos{f}[1] \dicFlx{(‑u)}[5] \dicDirectTranslationCS{laoština}
\dicEntry[laoskur] \dicTerm{laoskur} \dicIPA{{l}{a}{\textlengthmark}{\textopeno}{s}{\r{g}}{\textscy}{\textsubring{r}}} \dicPos{adj}[1]\dicFlx{}[-6] \dicDirectTranslationCS{laoský}
\dicEntry[lapið] \dicTerm{lapið} \dicIPA{{l}{a}{\textlengthmark}{\textsubring{b}}{\textsci}{\texttheta}} \dicPos{v} \dicFlx{supin} \dicLink{lepja}
\dicEntry[lappa] \dicTerm{lapp|a} \dicIPA{{l}{a}{h}{\textsubring{b}}{a}} \dicPos{v}[1] \dicFlx{(‑aði)}[13] \dicDirectTranslationCS{ochodit, prochodit (boty ap.)} \dicExampleIS{lappa skóna} \dicExampleCS{prochodit boty};  \dicPhraseIS{e‑að lappar e‑n} \dicDirectTranslationCS{(co) je (komu) volné (boty ap.)} \dicExampleIS{Skórinn lappar mig.} \dicExampleCS{Boty mi jsou volné.};  \dicIdiom{lappa}[upp á]{ \dicPhraseIS{lappa upp á e‑ð}} \dicDirectTranslationCS{osvěžit (co), vyspravit (co)}
\dicEntry[lappajaðrakan] \dicTerm{lappa··jað·rakan} \dicIPA{{l}{a}{h}{\textsubring{b}}{a}{j}{a}{ð}{r}{a}{\r{g}}{a}{\textsubring{n}}} \dicPos{m}[4] \dicFlx{(‑s, ‑ar)}[12] \dicFieldCat{zool.} \dicDirectTranslationCS{břehouš rudý} \textit{(l.~{\textLA{Limosa lapponica}})}  \dicsymPhoto\ 
\dicFigure{ds_image_lappajadrakan_0_1.jpg}{Lappajaðrakan}{Lappajaðrakan - Andreas Trepte, CC BY-SA 2.5}
\dicEntry[lappalaus] \dicTerm{lappa··laus} \dicIPA{{l}{a}{h}{\textsubring{b}}{a}{l}{\oe i}{s}} \dicPos{adj}[5]\dicFlx{}[-1] \dicPhraseIS{vera orðinn lappalaus} \dicLangCat{přen.} \dicDirectTranslationCS{být k~smrti unavený}
\dicEntry[lappar] \dicTerm{lappar} \dicIPA{{l}{a}{h}{\textsubring{b}}{a}{\textsubring{r}}} \dicPos{f} \dicFlx{sg gen} \dicLink{löpp}
\dicEntry[Lappi] \dicTerm{Lapp|i} \dicIPA{{l}{a}{h}{\textsubring{b}}{\textsci}} \dicPos{m}[1] \dicFlx{(‑a, ‑ar)}[8] \dicDirectTranslationCS{Laponec, Laponka}
\dicEntry[lappir] \dicTerm{lappir} \dicIPA{{l}{a}{h}{\textsubring{b}}{\textsci}{\textsubring{r}}} \dicPos{f} \dicFlx{pl nom} \dicLink{löpp}
\dicEntry[Lappland] \dicTerm{Lapp··land} \dicIPA{{l}{a}{h}{\textsubring{b}}{l}{a}{n}{\textsubring{d}}} \dicPos{n}[2] \dicFlx{(‑s)}[4] \dicFieldCat{geog.} \dicSynonym{Samaland} \dicDirectTranslationCS{Laponsko}
\dicEntry[lappneska] \dicTerm{lapp··nesk|a} \dicIPA{{l}{a}{h}{\textsubring{b}}{n}{\textepsilon}{s}{\r{g}}{a}} \dicPos{f}[1] \dicFlx{(‑u)}[5] \dicDirectTranslationCS{laponština}
\dicEntry[lappneskur] \dicTerm{lapp··neskur} \dicIPA{{l}{a}{h}{\textsubring{b}}{n}{\textepsilon}{s}{\r{g}}{\textscy}{\textsubring{r}}} \dicPos{adj}[1]\dicFlx{}[-1] \dicDirectTranslationCS{laponský}
\dicEntry[lapti] \dicTerm{lapti} \dicIPA{{l}{a}{f}{\textsubring{d}}{\textsci}} \dicPos{v} \dicFlx{ind pf sg 1 pers} \dicLink{lepja}
\dicEntry[las] \dicTerm{las} \dicIPA{{l}{a}{\textlengthmark}{s}} \dicPos{v} \dicFlx{ind pf sg 1 pers} \dicLink{lesa}
\dicEntry[lasanja] \dicTerm{lasanja} \dicIPA{{l}{a}{\textlengthmark}{s}{a}{n}{j}{a}} \dicPos{n}[4] \dicFlx{indecl}[1] \dicFieldCat{kulin.} \dicDirectTranslationCS{lasagne}
\dicEntry[lasburða] \dicTerm{las··burða} \dicIPA{{l}{a}{\textlengthmark}{s}{\textsubring{b}}{\textscy}{r}{ð}{a}} \dicPos{adj}[13] \dicFlx{indecl}[1] \dicDirectTranslationCS{churavý, neduživý}
\dicEntry[lasinn] \dicTerm{lasinn} \dicIPA{{l}{a}{\textlengthmark}{s}{\textsci}{\textsubring{n}}} \dicPos{adj}[6]\dicFlx{}[-3] \textbf{1.} \dicSynonym{veikur} \dicDirectTranslationCS{nemocný, indisponovaný}  \textbf{2.} \dicSynonym{lélegur} \dicDirectTranslationCS{opotřebovaný}
\dicEntry[laska] \dicTerm{lask|a} \dicIPA{{l}{a}{s}{\r{g}}{a}} \dicPos{v}[1] \dicFlx{(‑aði)}[13] \dicFlx{acc} \dicSynonym{skadda} \dicDirectTranslationCS{poškodit, způsobit škodu}
\dicEntry[lasleiki] \dicTerm{las··leik|i} \dicIPA{{l}{a}{\textlengthmark}{s}{l}{ei}{\r{\textObardotlessj}}{\textsci}} \dicPos{m}[1] \dicFlx{(‑a)}[3] \dicDirectTranslationCS{indispozice, churavost}
\dicEntry[last] \dicTerm{last} \dicIPA{{l}{a}{s}{\textsubring{d}}} \dicPos{n}[2] \dicFlx{(‑s)}[2] \dicSynonym*{baktal} \dicDirectTranslationCS{pomluva, osočení};  \dicPhraseIS{leggja e‑m e‑ð til lasts} \dicDirectTranslationCS{dávat (co komu) za vinu}
\dicEntry[lasta] \dicTerm{last|a} \dicIPA{{l}{a}{s}{\textsubring{d}}{a}} \dicPos{v}[1] \dicFlx{(‑aði)}[13] \dicFlx{acc} \dicSynonym{gagnrýna} \dicDirectTranslationCS{očernit, očerňovat, (po)hanit, pohanět} \dicExampleIS{lasta þýðingu á bók} \dicExampleCS{očerňovat překlad knížky}
\dicEntry[lastar] \dicTerm{lastar} \dicIPA{{l}{a}{s}{\textsubring{d}}{a}{\textsubring{r}}} \dicPos{m} \dicFlx{sg gen} \dicLink{löstur}
\dicEntry[lastyrði] \dicTerm{last··yrði} \dicIPA{{l}{a}{s}{\textsubring{d}}{\textsci}{r}{ð}{\textsci}} \dicPos{n}[2] \dicFlx{(‑s, ‑)}[14] \dicSynonym{last} \dicDirectTranslationCS{hana, hanobení}
\dicEntry[latína] \dicTerm{latín|a} \dicIPA{{l}{a}{\textlengthmark}{t\smash{\textsuperscript{h}}}{i}{n}{a}} \dicPos{f}[1] \dicFlx{(‑u)}[5] \dicDirectTranslationCS{latina}
\dicEntry[latínusegl] \dicTerm{latínu··segl} \dicIPA{{l}{a}{\textlengthmark}{t\smash{\textsuperscript{h}}}{i}{n}{\textscy}{s}{\textepsilon}{\r{g}}{\textsubring{l}}} \dicPos{n}[2] \dicFlx{(‑s, ‑)}[5] \dicFieldCat{nám.} \dicDirectTranslationCS{latinská plachta}
\dicEntry[latneskur] \dicTerm{latn··eskur} \dicIPA{{l}{a}{h}{\textsubring{d}}{n}{\textepsilon}{s}{\r{g}}{\textscy}{\textsubring{r}}} \dicPos{adj}[1]\dicFlx{}[-1] \dicDirectTranslationCS{latinský}
\dicEntry[latt] \dicTerm{latt} \dicIPA{{l}{a}{h}{\textsubring{d}}} \dicPos{v} \dicFlx{supin} \dicLink{letja}
\dicEntry[latti] \dicTerm{latti} \dicIPA{{l}{a}{h}{\textsubring{d}}{\textsci}} \dicPos{v} \dicFlx{ind pf sg 1 pers} \dicLink{letja}
\dicEntry[latur] \dicTerm{latur} \dicIPA{{l}{a}{\textlengthmark}{\textsubring{d}}{\textscy}{\textsubring{r}}} \dicPos{adj}[1] \dicFlx{(f löt)}[2] \dicDirectTranslationCS{líný, lenivý}
\dicEntry[lauf] \dicTerm{lauf} \dicsymFrequent\  \dicIPA{{l}{\oe i}{\textlengthmark}{f}} \dicPos{n}[2] \dicFlx{(‑s, ‑)}[5] \textbf{1.} \dicSynonym{laufblað} \dicDirectTranslationCS{list (stromu ap.)} \dicExampleIS{gulnuð lauf að hausti} \dicExampleCS{zežloutlé listy na podzim}  \textbf{2.} \dicSynonym*{litur í spilum} \dicDirectTranslationCS{trefy, kříže} \dicIndirectTranslationCS{(barva v~kartách)}
\dicEntry[laufabrauð] \dicTerm{laufa··brauð} \dicIPA{{l}{\oe i}{\textlengthmark}{v}{a}{\textsubring{b}}{r}{\oe i}{\texttheta}} \dicPos{n}[2] \dicFlx{(‑s, ‑)}[5] \dicFieldCat{kulin.} \dicIndirectTranslationCS{islandský vánoční chléb ručně zdobený různými motivy} \dicsymPhoto\ 
\dicFigure{ds_image_laufabraud_0_2.jpg}{Laufabrauð}{Laufabrauð - Fany Larota Catunta, PD}
\dicEntry[laufblað] \dicTerm{lauf··|blað} \dicIPA{{l}{\oe i}{v}{\textsubring{b}}{l}{a}{\texttheta}} \dicPos{n}[2] \dicFlx{(‑blaðs, ‑blöð)}[8] \dicSynonym{lauf} \dicDirectTranslationCS{list (stromu ap.)}
\dicEntry[laufgast] \dicTerm{laufg|ast} \dicIPA{{l}{\oe i}{v}{\r{g}}{a}{s}{\textsubring{d}}} \dicPos{v}[1] \dicFlx{(‑aðist)}[95] \dicFlx{refl} \dicDirectTranslationCS{nasadit listy, zazelenat se}
\dicEntry[laufglói] \dicTerm{lauf··gló|i} \dicIPA{{l}{\oe i}{v}{\r{g}}{l}{ou}{\textsci}} \dicPos{m}[1] \dicFlx{(‑a, ‑ar)}[1] \dicFieldCat{zool.} \dicSynonym{gullþröstur} \dicDirectTranslationCS{žluva, žluva hajní} \textit{(l.~{\textLA{Oriolus oriolus}})}  \dicsymPhoto\ 
\dicFigure{ds_image_laufgloi_0_1.jpg}{Laufglói}{Laufglói - Dixi, GFDL}
\dicEntry[laufgræna] \dicTerm{lauf··græn|a} \dicIPA{{l}{\oe i}{v}{\r{g}}{r}{a}{i}{n}{a}} \dicPos{f}[1] \dicFlx{(‑u)}[5] \dicFieldCat{bot.} \dicSynonym{blaðgræna} \dicDirectTranslationCS{chlorofyl}
\dicEntry[laufskógur] \dicTerm{lauf··skóg|ur} \dicIPA{{l}{\oe i}{f}{s}{\r{g}}{ou}{\textscy}{\textsubring{r}}} \dicPos{m}[6] \dicFlx{(‑ar, ‑ar)}[56] \dicDirectTranslationCS{listnatý les}
\dicEntry[laufsöngvari] \dicTerm{lauf··söngv·ar|i} \dicIPA{{l}{\oe i}{f}{s}{\oe i}{\ng}{\r{g}}{v}{a}{r}{\textsci}} \dicPos{m}[1] \dicFlx{(‑a, ‑ar)}[13] \dicFieldCat{zool.} \dicDirectTranslationCS{budníček větší} \textit{(l.~{\textLA{Phylloscopus trochilus}})}  \dicsymPhoto\ 
\dicFigure{ds_image_laufsongvari_0_1.jpg}{Laufsöngvari}{Laufsöngvari - Vogelartinfo, CC BY-SA 3.0}
\dicEntry[lauftré] \dicTerm{lauf··tré} \dicIPA{{l}{\oe i}{v}{t\smash{\textsuperscript{h}}}{r}{j}{\textepsilon}} \dicPos{n}[2] \dicFlx{(‑s, ‑)}[36] \dicDirectTranslationCS{listnatý strom, listnáč}
\dicEntry[laug] \dicTerm{laug\smash{\textsuperscript{1}}} \dicIPA{{l}{\oe i}{\textlengthmark}{x}} \dicPos{f}[4] \dicFlx{(‑ar, ‑ar)}[1] \textbf{1.} \dicSynonym{sundlaug} \dicDirectTranslationCS{(plavecký) bazén, koupaliště, plovárna}  \textbf{2.} \dicSynonym*{heit lind} \dicDirectTranslationCS{vřídlo, horký pramen}
\dicEntry[laug] \dicTerm{laug\smash{\textsuperscript{2}}} \dicIPA{{l}{\oe i}{\textlengthmark}{x}} \dicPos{v} \dicFlx{ind pf sg 1 pers} \dicLink{ljúga}
\dicEntry[lauga] \dicTerm{laug|a} \dicIPA{{l}{\oe i}{\textlengthmark}{\textbabygamma}{a}} \dicPos{v}[1] \dicFlx{(‑aði)}[1] \dicFlx{acc} \dicLangCat{zast.} \dicSynonym{þvo} \dicDirectTranslationCS{(vy)koupat} \dicExampleIS{lauga barn} \dicExampleCS{umýt dítě}
\dicEntry[laugardagskvöld] \dicTerm{laugar·dags··kvöld} \dicIPA{{l}{\oe i}{\textlengthmark}{\textbabygamma}{a}{r}{\textsubring{d}}{a}{x}{s}{k\smash{\textsuperscript{h}}}{v}{\oe}{l}{\textsubring{d}}} \dicPos{n}[2] \dicFlx{(‑s, ‑)}[5] \dicDirectTranslationCS{sobotní večer, sobota večer}
\dicEntry[laugardagur] \dicTerm{laugar··dag|ur} \dicIPA{{l}{\oe i}{\textlengthmark}{\textbabygamma}{a}{r}{\textsubring{d}}{a}{\textbabygamma}{\textscy}{\textsubring{r}}} \dicPos{m}[6] \dicFlx{(‑s, ‑ar)}[62] \dicDirectTranslationCS{sobota}
\dicEntry[Laugavegur] \dicTerm{Lauga··veg|ur} \dicIPA{{l}{\oe i}{\textlengthmark}{\textbabygamma}{a}{v}{\textepsilon}{\textbabygamma}{\textscy}{\textsubring{r}}} \dicPos{m}[10] \dicFlx{(‑ar\,/\addthin ‑s)}[28] \dicFlx{prop} \dicDirectTranslationCS{Laugavegur} \dicIndirectTranslationCS{(hlavní nákupní třída v~Reykjavíku)}
\dicEntry[lauk] \dicTerm{lauk} \dicIPA{{l}{\oe i}{\textlengthmark}{\r{g}}} \dicPos{v} \dicFlx{ind pf sg 1 pers} \dicLink{ljúka}
\dicEntry[laukur] \dicTerm{lauk|ur} \dicIPA{{l}{\oe i}{\textlengthmark}{\r{g}}{\textscy}{\textsubring{r}}} \dicPos{m}[6] \dicFlx{(‑s, ‑ar)}[22] \dicFieldCat{bot.} \dicDirectTranslationCS{cibule, cibule kuchyňská} \textit{(l.~{\textLA{Allium cepa}})};  \dicPhraseIS{laukur ættar} \dicLangCat{přen.} \dicDirectTranslationCS{pýcha rodiny}
\dicEntry[laum] \dicTerm{laum} \dicIPA{{l}{\oe i}{\textlengthmark}{\textsubring{m}}} \dicPos{n}[2] \dicFlx{(‑s)}[2] \dicPhraseIS{í laumi} \dicFlx{adv} \dicDirectTranslationCS{potají, v~tajnosti, tajně}
\dicEntry[lauma] \dicTerm{laum|a} \dicsymFrequent\  \dicIPA{{l}{\oe i}{\textlengthmark}{m}{a}} \dicPos{v}[1] \dicFlx{(‑aði)}[1] \dicFlx{dat} \dicDirectTranslationCS{(nenápadně) dát, vsunout, podstrčit, propašovat} \dicExampleIS{lauma bréfi að e‑m} \dicExampleCS{nenápadně (komu) podstrčit dopis};  \dicIdiom{laumast}{ \dicPhraseIS{laumast}} \dicFlx{refl} \dicSynonym*{læðast} \dicDirectTranslationCS{(v)plížit se, (v)krást se} \dicExampleIS{laumast úr bænum} \dicExampleCS{vykrást se z~města}
\dicEntry[laumufarþegi] \dicTerm{laumu··far·þeg|i} \dicIPA{{l}{\oe i}{\textlengthmark}{m}{\textscy}{f}{a}{\textsubring{r}}{\texttheta}{ei}{\textsci}} \dicPos{m}[1] \dicFlx{(‑a, ‑ar)}[1] \dicDirectTranslationCS{černý pasažér, černá pasažérka}
\dicEntry[laumulegur] \dicTerm{laumu··legur} \dicIPA{{l}{\oe i}{\textlengthmark}{m}{\textscy}{l}{\textepsilon}{\textbabygamma}{\textscy}{\textsubring{r}}} \dicPos{adj}[1]\dicFlx{}[-8] \dicDirectTranslationCS{tajný, skrytý, kradmý}
\dicEntry[laumuspil] \dicTerm{laumu··spil} \dicIPA{{l}{\oe i}{\textlengthmark}{m}{\textscy}{s}{\textsubring{b}}{\textsci}{\textsubring{l}}} \dicPos{n}[2] \dicFlx{(‑s, ‑)}[5] \dicDirectTranslationCS{tajnosti, tajnůstkaření}
\dicEntry[laun] \dicTerm{laun\smash{\textsuperscript{1}}} \dicIPA{{l}{\oe i}{\textlengthmark}{\textsubring{n}}} \dicPos{f}[4] \dicFlx{(‑ar)}[3] \dicSynonym{leynd} \dicDirectTranslationCS{tajemství};  \dicPhraseIS{á laun} \dicFlx{adv} \dicDirectTranslationCS{potají, v~tajnosti}
\dicEntry[laun] \dicTerm{laun\smash{\textsuperscript{2}}} \dicsymFrequent\  \dicIPA{{l}{\oe i}{\textlengthmark}{\textsubring{n}}} \dicPos{n}[2] \dicFlx{pl}[1] \textbf{1.} \dicSynonym{kaup} \dicDirectTranslationCS{plat, výplata, mzda} \dicExampleIS{há laun} \dicExampleCS{vysoký plat}  \textbf{2.} \dicSynonym{endurgjald} \dicDirectTranslationCS{vděk, odměna}  \textbf{3.} \dicSynonym{borgun} \dicDirectTranslationCS{(peněžitá) odměna, honorář}
\dicEntry[launa] \dicTerm{laun|a} \dicIPA{{l}{\oe i}{\textlengthmark}{n}{a}} \dicPos{v}[1] \dicFlx{(‑aði)}[1] \dicFlx{dat + acc} \textbf{1.} \dicSynonym{endurgjalda} \dicDirectTranslationCS{odplatit, oplatit, odvděčit se};  \dicPhraseIS{launa e‑m e‑ð} \dicDirectTranslationCS{odvděčit se (komu čím), oplatit (komu čím)}  \textbf{2.} \dicDirectTranslationCS{ohodnotit, honorovat, platit} \dicExampleIS{illa launað starf} \dicExampleCS{špatně honorovaná práce}
\dicEntry[launaauki] \dicTerm{launa··auk|i} \dicIPA{{l}{\oe i}{\textlengthmark}{n}{a}{\oe i}{\r{\textObardotlessj}}{\textsci}} \dicPos{m}[1] \dicFlx{(‑a)}[3] \dicDirectTranslationCS{příplatek\,/\addthin přídavek k~platu, prémie, bonus}
\dicEntry[launadeila] \dicTerm{launa··deil|a} \dicIPA{{l}{\oe i}{\textlengthmark}{n}{a}{\textsubring{d}}{ei}{l}{a}} \dicPos{f}[1] \dicFlx{(‑u, ‑ur)}[13] \dicDirectTranslationCS{mzdový spor}
\dicEntry[launaflokkur] \dicTerm{launa··flokk|ur} \dicIPA{{l}{\oe i}{\textlengthmark}{n}{a}{f}{l}{\textopeno}{h}{\r{g}}{\textscy}{\textsubring{r}}} \dicPos{m}[6] \dicFlx{(‑s, ‑ar)}[8] \dicDirectTranslationCS{platové zařazení, platová skupina\,/\addthin třída}
\dicEntry[launagreiðsla] \dicTerm{launa··greiðsl|a} \dicIPA{{l}{\oe i}{\textlengthmark}{n}{a}{\r{g}}{r}{ei}{ð}{s}{\textsubring{d}}{l}{a}} \dicPos{f}[1] \dicFlx{(‑u, ‑ur)}[13] \dicDirectTranslationCS{výplata mzdy}
\dicEntry[launahækkun] \dicTerm{launa··hækk|un} \dicIPA{{l}{\oe i}{\textlengthmark}{n}{a}{h}{a}{i}{h}{\r{g}}{\textscy}{\textsubring{n}}} \dicPos{f}[7] \dicFlx{(‑unar, ‑anir)}[8] \dicDirectTranslationCS{zvýšení platu, mzdový růst}
\dicEntry[launakjör] \dicTerm{launa··kjör} \dicIPA{{l}{\oe i}{\textlengthmark}{n}{a}{c\smash{\textsuperscript{h}}}{\oe}{\textsubring{r}}} \dicPos{n}[2] \dicFlx{pl}[9] \dicDirectTranslationCS{platové\,/\addthin mzdové podmínky}
\begin{xtolerant}{}{1pt}
\dicEntry[launakostnaður] \dicTerm{launa··kost·nað|ur}\addthinS\dicIPA{{l}{\oe i}{\textlengthmark}{n}{a}{k\smash{\textsuperscript{h}}}{\textopeno}{s}{\textsubring{d}}{n}{a}{ð}{\textscy}{\textsubring{r}}}\addthinS\dicPos{m}[10]\addthinS\dicFlx{(‑ar)}[9] \dicDirectTranslationCS{mzdové náklady}
\end{xtolerant}
\dicEntry[launalækkun] \dicTerm{launa··lækk|un} \dicIPA{{l}{\oe i}{\textlengthmark}{n}{a}{l}{a}{i}{h}{\r{g}}{\textscy}{\textsubring{n}}} \dicPos{f}[7] \dicFlx{(‑unar)}[9] \dicDirectTranslationCS{snížení platu, mzdový pokles}
\dicEntry[launamaður] \dicTerm{launa··|maður} \dicIPA{{l}{\oe i}{\textlengthmark}{n}{a}{m}{a}{ð}{\textscy}{\textsubring{r}}} \dicPos{m}[13] \dicFlx{(‑manns, ‑menn)}[2] \dicDirectTranslationCS{zaměstnanec, zaměstnankyně}
\dicEntry[launaskrá] \dicTerm{launa··skrá} \dicIPA{{l}{\oe i}{\textlengthmark}{n}{a}{s}{\r{g}}{r}{au}} \dicPos{f}[4] \dicFlx{(‑r\,/\addthin ‑ar, ‑r)}[21] \dicDirectTranslationCS{výplatní listina}
\dicEntry[launaskrið] \dicTerm{launa··skrið} \dicIPA{{l}{\oe i}{\textlengthmark}{n}{a}{s}{\r{g}}{r}{\textsci}{\texttheta}} \dicPos{n}[2] \dicFlx{(‑s)}[2] \dicFieldCat{ekon.} \dicDirectTranslationCS{stoupání mezd}
\dicEntry[launastigi] \dicTerm{launa··stig|i} \dicIPA{{l}{\oe i}{\textlengthmark}{n}{a}{s}{\textsubring{d}}{i}{j}{\textsci}} \dicPos{m}[1] \dicFlx{(‑a)}[3] \dicDirectTranslationCS{mzdový tarif}
\dicEntry[launatengdur] \dicTerm{launa··tengdur} \dicIPA{{l}{\oe i}{\textlengthmark}{n}{a}{t\smash{\textsuperscript{h}}}{\textepsilon}{\ng}{\textsubring{d}}{\textscy}{\textsubring{r}}} \dicPos{adj}[2]\dicFlx{}[-17] \dicDirectTranslationCS{mzdový, spojený se mzdou (náklady ap.)};  \dicPhraseIS{launatengd gjöld} \dicDirectTranslationCS{příspěvky zaměstnavatele}
\dicEntry[launauppbót] \dicTerm{launa··upp·|bót} \dicIPA{{l}{\oe i}{\textlengthmark}{n}{a}{\textscy}{h}{\textsubring{b}}{ou}{\textsubring{d}}} \dicPos{f}[8] \dicFlx{(‑bótar, ‑bætur)}[5] \dicDirectTranslationCS{platový bonus}
\dicEntry[launráð] \dicTerm{laun··ráð} \dicIPA{{l}{\oe i}{n}{r}{au}{\texttheta}} \dicPos{n}[2] \dicFlx{pl}[1] \dicSynonym{ráðabrugg} \dicDirectTranslationCS{pikle, intriky, komplot}
\dicEntry[launsátur] \dicTerm{laun··sátur} \dicIPA{{l}{\oe i}{n}{s}{au}{\textsubring{d}}{\textscy}{\textsubring{r}}} \dicPos{n}[2] \dicFlx{(‑s, ‑)}[25] \dicSynonym{fyrirsát} \dicDirectTranslationCS{léčka, nástraha}
\dicEntry[launung] \dicTerm{laun··ung} \dicIPA{{l}{\oe i}{\textlengthmark}{n}{u}{\ng}{\r{g}}} \dicPos{f}[4] \dicFlx{(‑ar)}[3] \dicSynonym{leynd} \dicDirectTranslationCS{tajemství, tajnost};  \dicPhraseIS{með launung} \dicFlx{adv} \dicSynonym*{leynilega} \dicDirectTranslationCS{tajně};  \dicPhraseIS{e‑m er engin launung á e‑u} \dicFlx{impers} \dicDirectTranslationCS{(kdo) se netají (čím)}
\dicEntry[launþegi] \dicTerm{laun··þeg|i} \dicIPA{{l}{\oe i}{n}{\texttheta}{ei}{\textsci}} \dicPos{m}[1] \dicFlx{(‑a, ‑ar)}[1] \dicDirectTranslationCS{zaměstnanec, zaměstnankyně, pracovník, pracovnice}
\dicEntry[laus] \dicTerm{laus} \dicsymFrequent\  \dicIPA{{l}{\oe i}{\textlengthmark}{s}} \dicPos{adj}[5] \dicFlx{(f ‑)}[1] \textbf{1.} \dicSynonym{óbundinn} \dicDirectTranslationCS{volný, uvolněný, povolený} \dicExampleIS{Steinninn er orðinn laus frá veggnum.} \dicExampleCS{Kámen se už uvolnil ze zdi.}  \textbf{2.} \dicSynonym{frjáls} \dicDirectTranslationCS{volný, svobodný} \dicExampleIS{Ertu laus í kvöld?} \dicExampleCS{Máš večer čas?};  \dicPhraseIS{ganga laus} \dicDirectTranslationCS{jít volně (pes ap.)};  \dicPhraseIS{láta e‑n lausan} \dicDirectTranslationCS{pustit (koho) na svobodu};  \dicPhraseIS{vera laus við e‑ð} \dicDirectTranslationCS{zbavit se (čeho), oprostit se od (čeho)}  \textbf{3.} \dicSynonym*{auður} \dicDirectTranslationCS{volný, prázdný, neobsazený} \dicExampleIS{Er þetta sæti laust?} \dicExampleCS{Je toto místo volné?} \dicAntonym{upptekinn}  \textbf{4.} \dicSynonym{léttur} \dicDirectTranslationCS{lehký} \dicExampleIS{laust högg} \dicExampleCS{lehká rána}  \textbf{5.} \dicSynonym{óáreiðanlegur} \dicDirectTranslationCS{nespolehlivý, nestabilní};  \dicPhraseIS{vera á lausu} \dicDirectTranslationCS{být nezadaný};  \dicPhraseIS{vera laus á kostunum} \dicSynonym{lauslátur} \dicDirectTranslationCS{mít volné mravy} \dicIndirectTranslationCS{(zvláště o~ženě)}  \textbf{6.} \dicSynonym{örlátur} \dicDirectTranslationCS{štědrý};  \dicPhraseIS{vera laus á fé} \dicDirectTranslationCS{být štědrý}
\dicEntry[lausafé] \dicTerm{lausa··|fé} \dicIPA{{l}{\oe i}{\textlengthmark}{s}{a}{f}{j}{\textepsilon}} \dicPos{n}[3] \dicFlx{(‑fjár)}[5] \dicDirectTranslationCS{hotovost}
\dicEntry[lausafjárstaða] \dicTerm{lausa··fjár·|staða} \dicIPA{{l}{\oe i}{\textlengthmark}{s}{a}{f}{j}{au}{\textsubring{r}}{s}{\textsubring{d}}{a}{ð}{a}} \dicPos{f}[1] \dicFlx{(‑stöðu)}[2] \dicFieldCat{ekon.} \dicDirectTranslationCS{likvidita}
\dicEntry[lausaleikur] \dicTerm{lausa··leik|ur} \dicIPA{{l}{\oe i}{\textlengthmark}{s}{a}{l}{ei}{\r{g}}{\textscy}{\textsubring{r}}} \dicPos{m}[6] \dicFlx{(‑s)}[17] \dicPhraseIS{eiga\,/\addthin eignast barn í lausaleik} \dicDirectTranslationCS{mít nemanželské dítě}
\dicEntry[lausamennska] \dicTerm{lausa··mennsk|a} \dicIPA{{l}{\oe i}{\textlengthmark}{s}{a}{m}{\textepsilon}{n}{s}{\r{g}}{a}} \dicPos{f}[1] \dicFlx{(‑u)}[5] \dicDirectTranslationCS{práce na volné noze}
\dicEntry[lausamjöll] \dicTerm{lausa··|mjöll} \dicIPA{{l}{\oe i}{\textlengthmark}{s}{a}{m}{j}{\oe}{\textsubring{d}}{\textsubring{l}}} \dicPos{f}[7] \dicFlx{(‑mjallar)}[19] \dicDirectTranslationCS{čerstvě napadlý sníh, prašan}
\dicEntry[lausasala] \dicTerm{lausa··|sala} \dicIPA{{l}{\oe i}{\textlengthmark}{s}{a}{s}{a}{l}{a}} \dicPos{f}[1] \dicFlx{(‑sölu, ‑sölur)}[20] \dicDirectTranslationCS{volný prodej}
\dicEntry[lausavísa] \dicTerm{lausa··vís|a} \dicIPA{{l}{\oe i}{\textlengthmark}{s}{a}{v}{i}{s}{a}} \dicPos{f}[1] \dicFlx{(‑u, ‑ur)}[13] \dicFieldCat{lit.} \dicDirectTranslationCS{epigram} \dicIndirectTranslationCS{(často od neznámého autora, velmi populární na Islandu)}
\dicEntry[lausblaðabók] \dicTerm{laus·blaða··|bók} \dicIPA{{l}{\oe i}{s}{\textsubring{b}}{l}{a}{ð}{a}{\textsubring{b}}{ou}{\r{g}}} \dicPos{f}[8] \dicFlx{(‑bókar, ‑bækur)}[5] \dicDirectTranslationCS{zápisník\,/\addthin sešit s~volnými listy}

\dicEntry[lausblaðamappa] \textls[15]{\dicTerm{laus·blaða··|mappa} \dicIPA{{l}{\oe i}{s}{\textsubring{b}}{l}{a}{ð}{a}{m}{a}{h}{\textsubring{b}}{a}} \dicPos{f}[1] \dicFlx{(‑möppu, ‑möpp\-ur)}[8]} \dicDirectTranslationCS{pořadač, šanon}

\dicEntry[lauslátur] \dicTerm{laus··látur} \dicIPA{{l}{\oe i}{s}{l}{au}{\textsubring{d}}{\textscy}{\textsubring{r}}} \dicPos{adj}[1]\dicFlx{}[-1] \dicDirectTranslationCS{nezávazný (flirt ap.), promiskuitní}
\dicEntry[lauslegur] \dicTerm{laus··legur} \dicIPA{{l}{\oe i}{s}{l}{\textepsilon}{\textbabygamma}{\textscy}{\textsubring{r}}} \dicPos{adj}[1]\dicFlx{}[-8] \textbf{1.} \dicSynonym{yfirborðslegur} \dicDirectTranslationCS{zběžný, letmý, hrubý (odhad ap.)} \dicExampleIS{lauslegur uppdráttur} \dicExampleCS{zběžný nákres}  \textbf{2.} \dicSynonym{óstöðugur} \dicDirectTranslationCS{volný, nepřipevněný}
\dicEntry[lauslæti] \dicTerm{laus··læti} \dicIPA{{l}{\oe i}{s}{l}{a}{i}{\textsubring{d}}{\textsci}} \dicPos{n}[2] \dicFlx{(‑s)}[20] \dicDirectTranslationCS{nezávaznost, promiskuita}
\dicEntry[lausmáll] \dicTerm{laus··máll} \dicIPA{{l}{\oe i}{s}{m}{au}{\textsubring{d}}{\textsubring{l}}} \dicPos{adj}[8]\dicFlx{}[-1] \textbf{1.} \dicSynonym{málgefinn} \dicDirectTranslationCS{upovídaný, užvaněný}  \textbf{2.} \dicSynonym*{óorðheldinn} \dicDirectTranslationCS{indiskrétní, netaktní}
\dicEntry[lausn] \dicTerm{lausn} \dicsymFrequent\  \dicIPA{{l}{\oe i}{s}{\textsubring{d}}{\textsubring{n}}} \dicPos{f}[7] \dicFlx{(‑ar, ‑ir)}[1] \textbf{1.} \dicSynonym{frelsun} \dicDirectTranslationCS{osvobození, vysvobození} \dicExampleIS{lausn úr þrældómi} \dicExampleCS{vysvobození z~otroctví}  \textbf{2.} \dicDirectTranslationCS{uvolnění, propuštění (z~práce ap.)} \dicExampleIS{lausn frá embætti} \dicExampleCS{uvolnění z~úřadu};  \dicPhraseIS{biðjast lausnar} \dicFlx{refl} \dicDirectTranslationCS{žádat o~uvolnění (z~pozice ap.)}  \textbf{3.} \dicSynonym{úrslit} \dicDirectTranslationCS{(vy)řešení, rozřešení, rozuzlení} \dicExampleIS{lausn á vandanum} \dicExampleCS{vyřešení problému}  \textbf{4.} \dicSynonym{endurlausn} \dicDirectTranslationCS{vykoupení, spasení}  \textbf{5.} \dicFieldCat{chem.} \dicDirectTranslationCS{roztok};  \dicPhraseIS{mettuð lausn} \dicFieldCat{chem.} \dicDirectTranslationCS{nasycený roztok}
\dicEntry[lausnarbeiðni] \dicTerm{lausnar··beiðn|i} \dicIPA{{l}{\oe i}{s}{\textsubring{d}}{n}{a}{r}{\textsubring{b}}{ei}{ð}{n}{\textsci}} \dicPos{f}[3] \dicFlx{(‑i, ‑ir)}[1] \dicDirectTranslationCS{výpověď, žádost o~uvolnění (z~práce ap.)};  \dicPhraseIS{senda inn lausnarbeiðni} \dicDirectTranslationCS{podat výpověď (z~práce ap.)}
\dicEntry[lausnargjald] \dicTerm{lausnar··|gjald} \dicIPA{{l}{\oe i}{s}{\textsubring{d}}{n}{a}{r}{\r{\textObardotlessj}}{a}{l}{\textsubring{d}}} \dicPos{n}[2] \dicFlx{(‑gjalds, ‑gjöld)}[8] \dicDirectTranslationCS{výkupné}
\dicEntry[lausnarhnappur] \dicTerm{lausnar··hnapp|ur} \dicIPA{{l}{\oe i}{s}{\textsubring{d}}{n}{a}{\textsubring{r}}{\textsubring{n}}{a}{h}{\textsubring{b}}{\textscy}{\textsubring{r}}} \dicPos{m}[6] \dicFlx{(‑s, ‑ar)}[25] \dicFieldCat{poč.} \dicDirectTranslationCS{klávesa Escape}
\dicEntry[lausnari] \dicTerm{lausn··ar|i} \dicIPA{{l}{\oe i}{s}{\textsubring{d}}{n}{a}{r}{\textsci}} \dicPos{m}[1] \dicFlx{(‑a, ‑ar)}[13] \dicFieldCat{náb.} \dicSynonym{frelsari} \dicDirectTranslationCS{spasitel, vykupitel}
\dicEntry[laust] \dicTerm{laust\smash{\textsuperscript{1}}} \dicIPA{{l}{\oe i}{s}{\textsubring{d}}} \dicPos{v} \dicFlx{ind pf sg 1 pers} \dicLink{ljósta}
\dicEntry[laust] \dicTerm{laus|t\smash{\textsuperscript{2}}} \dicIPA{{l}{\oe i}{s}{\textsubring{d}}} \dicPos{adv} \dicFlx{(comp ‑ar, sup ‑ast)} \textbf{1.} \dicDirectTranslationCS{lehce, slabě, mírně} \dicExampleIS{berja laust að dyrum} \dicExampleCS{zaklepat lehce na dveře}  \textbf{2.} \dicDirectTranslationCS{krátce, těsně} \dicExampleIS{laust fyrir klukkan fimm} \dicExampleCS{krátce před pátou hodinou}
\dicEntry[lausung] \dicTerm{laus··ung} \dicIPA{{l}{\oe i}{\textlengthmark}{s}{u}{\ng}{\r{g}}} \dicPos{f}[4] \dicFlx{(‑ar)}[3] \textbf{1.} \dicSynonym*{staðfestuleysi} \dicDirectTranslationCS{nestabilita, nestálost}  \textbf{2.} \dicSynonym{lauslæti} \dicDirectTranslationCS{promiskuita};  \dicIdiom{lausung}{ \dicPhraseIS{gjalda lausung við lygi}} \dicLangCat{přen.} \dicDirectTranslationCS{oplatit faleš lží}
\dicEntry[laut] \dicTerm{laut\smash{\textsuperscript{1}}} \dicIPA{{l}{\oe i}{\textlengthmark}{\textsubring{d}}} \dicPos{f}[7] \dicFlx{(‑ar, ‑ir)}[6] \dicSynonym{lægð} \dicDirectTranslationCS{prohlubeň, dolík}
\dicEntry[laut] \dicTerm{laut\smash{\textsuperscript{2}}} \dicIPA{{l}{\oe i}{\textlengthmark}{\textsubring{d}}} \dicPos{v} \dicFlx{ind pf sg 1 pers} \dicLink{lúta}
\dicEntry[lautinant] \dicTerm{lautinant} \dicIPA{{l}{\oe i}{\textlengthmark}{\textsubring{d}}{\textsci}{n}{a}{\textsubring{n}}{\textsubring{d}}} \dicPos{m}[4] \dicFlx{(‑s, ‑ar)}[11] \dicFieldCat{voj.} \dicDirectTranslationCS{poručík, poručice}
\dicEntry[lax] \dicTerm{lax}\dicTerm{, laxfiskur} \dicIPA{{l}\-{a}\-{x}\-{s}\-} \dicPos{m}[4] \dicFlx{(‑, ‑ar)}[29] \dicFieldCat{zool.} \dicDirectTranslationCS{losos, losos obecný} \textit{(l.~{\textLA{Salmo salar}})}
\dicEntry[laxá] \dicTerm{lax··á} \dicIPA{{l}{a}{x}{s}{au}} \dicPos{f}[4] \dicFlx{(‑r, ‑r)}[18] \dicDirectTranslationCS{lososí řeka}
\dicEntry[laxerolía] \dicTerm{laxer··olí|a} \dicIPA{{l}{a}{x}{s}{\textepsilon}{r}{\textopeno}{l}{i}{j}{a}} \dicPos{f}[1] \dicFlx{(‑u)}[5] \dicDirectTranslationCS{ricinový olej}
\dicEntry[laxfiskur] \dicTerm{lax··fisk|ur} \dicIPA{{l}{a}{x}{s}{f}{\textsci}{s}{\r{g}}{\textscy}{\textsubring{r}}} \dicPos{m}[6] \dicFlx{(‑s, ‑ar)}[10] \dicLink{lax}
\dicEntry[laxveiði] \dicTerm{lax··veið|i} \dicIPA{{l}{a}{x}{s}{v}{ei}{ð}{\textsci}} \dicPos{f}[2] \dicFlx{(‑i\,/\addthin ‑ar, ‑ar)}[3] \textbf{1.} \dicDirectTranslationCS{lososí úlovek}  \textbf{2.} \dicDirectTranslationCS{chytání\,/\addthin lov lososů}
\dicEntry[lá] \dicTerm{lá\smash{\textsuperscript{1}}} \dicIPA{{l}{au}{\textlengthmark}} \dicPos{v}[2] \dicFlx{(‑ði, ‑ð)}[117] \dicFlx{dat} \dicSynonym{ásaka} \dicDirectTranslationCS{vyčíst, vyčítat, dávat za vinu} \dicIndirectTranslationCS{(používané v~otázce nebo záporu)} \dicExampleIS{Ég lái þér það ekki.} \dicExampleCS{Já ti to nevyčítám.};  \dicIdiom{lást}{ \dicPhraseIS{e‑m láist að (gera e‑ð)}} \dicFlx{refl impers} \dicDirectTranslationCS{(kdo) zapomíná (udělat (co)), (komu) vypadá z~paměti (udělat (co))} \dicExampleIS{Mér láðist að senda bréfið.} \dicExampleCS{Vypadlo mi (z~paměti), že mám poslat dopis.}
\dicEntry[lá] \dicTerm{lá\smash{\textsuperscript{2}}} \dicIPA{{l}{au}{\textlengthmark}} \dicPos{v} \dicFlx{ind pf sg 1 pers} \dicLink{liggja}
\dicEntry[láð] \dicTerm{láð} \dicIPA{{l}{au}{\textlengthmark}{\texttheta}} \dicPos{n}[2] \dicFlx{(‑s, ‑)}[5] \dicSynonym{land} \dicDirectTranslationCS{souš, země} \dicExampleIS{um láð og lög} \dicExampleCS{po souši a~po vodě}
\dicEntry[lág] \dicTerm{lág} \dicIPA{{l}{au}{\textlengthmark}} \dicPos{f}[4] \dicFlx{(‑ar, ‑ar)}[1] \dicSynonym{dæld} \dicDirectTranslationCS{dolina, kotlina};  \dicPhraseIS{láta e‑ð liggja í láginni} \dicLangCat{přen.} \dicDirectTranslationCS{nechat (co) ležet u~ledu}
\dicEntry[lágfiðla] \dicTerm{lág··fiðl|a} \dicIPA{{l}{au}{\textlengthmark}{f}{\textsci}{ð}{l}{a}} \dicPos{f}[1] \dicFlx{(‑u, ‑ur)}[7] \dicFieldCat{hud.} \dicDirectTranslationCS{viola}
\dicEntry[lágkúra] \dicTerm{lág··kúr|a} \dicIPA{{l}{au}{\textlengthmark}{k\smash{\textsuperscript{h}}}{u}{r}{a}} \dicPos{f}[1] \dicFlx{(‑u)}[5] \dicSynonym*{svipleysa} \dicDirectTranslationCS{banalita, přízemnost}
\dicEntry[lágkúrulegur] \textls[15]{\dicTerm{lág·kúru··legur} \dicIPA{{l}{au}{\textlengthmark}{k\smash{\textsuperscript{h}}}{u}{r}{\textscy}{l}{\textepsilon}{\textbabygamma}{\textscy}{\textsubring{r}}} \dicPos{adj}[1]\dicFlx{}[-8] \textbf{1.} \dicSynonym*{flatneskjulegur} \dicDirectTranslationCS{banální, všední, nezajímavý}} \dicExampleIS{lágkúrulegt hús} \dicExampleCS{všední dům}  \textbf{2.} \dicSynonym{lítilmótlegur} \dicDirectTranslationCS{přízemní, nízký, mrzký} \dicExampleIS{lágkúrulegur maður} \dicExampleCS{přízemní člověk}
\dicEntry[lágkveða] \dicTerm{lág··kveð|a} \dicIPA{{l}{au}{\textlengthmark}{k\smash{\textsuperscript{h}}}{v}{\textepsilon}{ð}{a}} \dicPos{f}[1] \dicFlx{(‑u, ‑ur)}[19] \dicFieldCat{lit.} \dicDirectTranslationCS{sudá stopa se slabším přízvukem}
\dicEntry[láglaunafólk] \dicTerm{lág·launa··fólk} \dicIPA{{l}{au}{\textlengthmark}{l}{\oe i}{n}{a}{f}{ou}{\textsubring{l}}{\r{g}}} \dicPos{n}[2] \dicFlx{(‑s)}[2] \dicDirectTranslationCS{osoby s~nízkým příjmem}
\dicEntry[láglendi] \dicTerm{lág··lendi} \dicIPA{{l}{au}{\textlengthmark}{l}{\textepsilon}{n}{\textsubring{d}}{\textsci}} \dicPos{n}[2] \dicFlx{(‑s)}[20] \dicDirectTranslationCS{nížina} \dicAntonym{hálendi}
\dicEntry[lágmark] \dicTerm{lág··|mark} \dicIPA{{l}{au}{\textlengthmark}{m}{a}{\textsubring{r}}{\r{g}}} \dicPos{n}[2] \dicFlx{(‑marks, ‑mörk)}[8] \dicDirectTranslationCS{minimum} \dicExampleIS{Kuldinn er í algjöru lágmarki.} \dicExampleCS{Zima je na úplném minimu.} \dicAntonym{hámark}
\dicEntry[lágmarksframfærsla] \dicTerm{lág·marks··fram·færsl|a} \dicIPA{{l}\-{au}\-{\textlengthmark}\-{m}\-{a}\-{\textsubring{r}}\-{x}\-{s}\-{f}\-{r}\-{a}\-{m}\-{f}\-{a}\-{i}\-{\textsubring{r}}\-{s}\-{\textsubring{d}}\-{l}\-{a}\-} \dicPos{f}[1] \dicFlx{(‑u)}[5] \dicFieldCat{ekon.} \dicDirectTranslationCS{životní minimum}
\dicEntry[lágmarksgjald] \dicTerm{lág·marks··|gjald} \dicIPA{{l}{au}{\textlengthmark}{m}{a}{\textsubring{r}}{x}{s}{\r{\textObardotlessj}}{a}{l}{\textsubring{d}}} \dicPos{n}[2] \dicFlx{(‑gjalds, ‑gjöld)}[8] \dicDirectTranslationCS{minimální sazba}
\begin{xtolerant}{}{1pt}
\dicEntry[lágmarkslaun] \dicTerm{lág·marks··laun}\addthinS\dicIPA{{l}{au}{\textlengthmark}{m}{a}{\textsubring{r}}{x}{s}{l}{\oe i}{\textsubring{n}}}\addthinS\dicPos{n}[2]\addthinS\dicFlx{pl}[1]\addthinS\dicDirectTranslationCS{minimální mzda\hspace{-0.06667em}\,/\addthin plat}
\end{xtolerant}
\dicEntry[lágmarkstímasókn] \dicTerm{lág·marks··tíma·sókn} \dicIPA{{l}{au}{\textlengthmark}{m}{a}{\textsubring{r}}{x}{s}{t\smash{\textsuperscript{h}}}{i}{m}{a}{s}{ou}{h}{\r{g}}{\textsubring{n}}} \dicPos{f}[7] \dicFlx{(‑ar)}[3] \dicDirectTranslationCS{minimální počet hodin školní docházky} \dicIndirectTranslationCS{(pro ukončení kurzu ap.)}
\dicEntry[lágmarksverð] \dicTerm{lág·marks··verð} \dicIPA{{l}{au}{\textlengthmark}{m}{a}{\textsubring{r}}{x}{s}{v}{\textepsilon}{r}{\texttheta}} \dicPos{n}[2] \dicFlx{(‑s)}[2] \dicDirectTranslationCS{minimální cena}
\dicEntry[lágmynd] \dicTerm{lág··mynd} \dicIPA{{l}{au}{\textlengthmark}{m}{\textsci}{n}{\textsubring{d}}} \dicPos{f}[7] \dicFlx{(‑ar, ‑ir)}[1] \dicDirectTranslationCS{reliéf}
\dicEntry[lágnætti] \dicTerm{lág··nætti} \dicIPA{{l}{au}{\textlengthmark}{n}{a}{i}{h}{\textsubring{d}}{\textsci}} \dicPos{n}[2] \dicFlx{(‑s, ‑)}[14] \dicDirectTranslationCS{půlnoc} \dicIndirectTranslationCS{(zvláště v~létě)}
\dicEntry[lágspenna] \dicTerm{lág··spenn|a} \dicIPA{{l}{au}{\textlengthmark}{s}{\textsubring{b}}{\textepsilon}{n}{a}} \dicPos{f}[1] \dicFlx{(‑u)}[5] \dicFieldCat{elek.} \dicDirectTranslationCS{nízké napětí} \dicAntonym{háspenna}
\dicEntry[lágstétt] \dicTerm{lág··stétt} \dicIPA{{l}{au}{\textlengthmark}{s}{\textsubring{d}}{j}{\textepsilon}{h}{\textsubring{d}}} \dicPos{f}[7] \dicFlx{(‑ar, ‑ir)}[1] \dicDirectTranslationCS{nižší (společenská) třída, proletariát} \dicAntonym{hástétt}
\dicEntry[lágt] \dicTerm{lágt} \dicsymFrequent\  \dicIPA{{l}{au}{x}{\textsubring{d}}} \dicPos{adv} \dicFlx{(comp lægra, sup lægst)} \textbf{1.} \dicSynonym*{lítill hljóðstyrkur} \dicDirectTranslationCS{tiše, potichu, tlumeně} \dicExampleIS{tala lágt} \dicExampleCS{mluvit potichu}  \textbf{2.} \dicSynonym{niðri} \dicDirectTranslationCS{nízko, dole}
\dicEntry[lágum] \dicTerm{lágum} \dicIPA{{l}{au}{\textlengthmark}{\textbabygamma}{\textscy}{\textsubring{m}}} \dicPos{v} \dicFlx{ind pf pl 1 pers} \dicLink{liggja}
\dicEntry[lágur] \dicTerm{lágur} \dicsymFrequent\  \dicIPA{{l}{au}{\textlengthmark}{\textscy}{\textsubring{r}}} \dicPos{adj}[10] \dicFlx{(comp lægri, sup lægstur)}[16] \textbf{1.} \dicSynonym{stuttur} \dicDirectTranslationCS{nízký, malý} \dicIndirectTranslationCS{(o~výšce)};  \dicPhraseIS{vera lágur vexti} \dicDirectTranslationCS{být malého vzrůstu}  \textbf{2.} \dicDirectTranslationCS{nízký} \dicIndirectTranslationCS{(o~čísle, ceně ap.)}  \textbf{3.} \dicDirectTranslationCS{tichý, tlumený} \dicExampleIS{lágur söngur} \dicExampleCS{tichý zpěv};  \dicPhraseIS{í lágum hljóðum} \dicFlx{adv} \dicDirectTranslationCS{tlumeně, tiše}  \textbf{4.} \dicDirectTranslationCS{nízký (tón ap.)} \dicExampleIS{lágur tónn} \dicExampleCS{nízký tón};  \dicIdiom{lágur}{ \dicPhraseIS{bera\,/\addthin bíða lægri hlut}} \dicLangCat{přen.} \dicDirectTranslationCS{tahat za kratší konec provazu}
\dicEntry[lágvaxinn] \dicTerm{lág··vaxinn} \dicIPA{{l}{au}{\textlengthmark}{v}{a}{x}{s}{\textsci}{\textsubring{n}}} \dicPos{adj}[6]\dicFlx{}[-3] \dicDirectTranslationCS{malý (vzrůstem), (jsoucí) malého vzrůstu} \dicAntonym{hávaxinn}
\dicEntry[lágþrýstingur] \dicTerm{lág··þrýst·ing|ur} \dicIPA{{l}{au}{\textlengthmark}{\texttheta}{r}{i}{s}{\textsubring{d}}{i}{\ng}{\r{g}}{\textscy}{\textsubring{r}}} \dicPos{m}[6] \dicFlx{(‑s)}[9] \textbf{1.} \dicFieldCat{med.} \dicDirectTranslationCS{nízký tlak}  \textbf{2.} \dicFieldCat{meteo.} \dicDirectTranslationCS{nízký tlak}
\dicEntry[lágþrýstisvæði] \dicTerm{lág·þrýsti··svæði} \dicIPA{{l}{au}{\textlengthmark}{\texttheta}{r}{i}{s}{\textsubring{d}}{\textsci}{s}{v}{a}{i}{ð}{\textsci}} \dicPos{n}[2] \dicFlx{(‑s, ‑)}[14] \dicFieldCat{meteo.} \dicDirectTranslationCS{oblast nízkého tlaku} \dicAntonym{háþrýstisvæði}
\dicEntry[lágþýska] \dicTerm{lág··þýsk|a} \dicIPA{{l}{au}{\textlengthmark}{\texttheta}{i}{s}{\r{g}}{a}} \dicPos{f}[1] \dicFlx{(‑u)}[5] \dicDirectTranslationCS{dolní němčina} \dicIndirectTranslationCS{(dialekt němčiny používaný v~severních částech Německa)} \dicAntonym{háþýska}
\dicEntry[lákum] \dicTerm{lákum} \dicIPA{{l}{au}{\textlengthmark}{\r{g}}{\textscy}{\textsubring{m}}} \dicPos{v} \dicFlx{ind pf pl 1 pers} \dicLink{leka}
\dicEntry[lán] \dicTerm{lán} \dicsymFrequent\  \dicIPA{{l}{au}{\textlengthmark}{\textsubring{n}}} \dicPos{n}[2] \dicFlx{(‑s, ‑)}[5] \textbf{1.} \dicSynonym{lánsfé} \dicDirectTranslationCS{(vý)půjčka, úvěr} \dicExampleIS{borga lán} \dicExampleCS{splatit půjčku};  \dicPhraseIS{taka lán} \dicDirectTranslationCS{vzít si půjčku\,/\addthin úvěr}  \textbf{2.} \dicSynonym*{það að lána} \dicDirectTranslationCS{(vy)půjčení};  \dicPhraseIS{fá e‑ð að láni} \dicDirectTranslationCS{(vy)půjčit si (co)};  \dicPhraseIS{hafa e‑ð að láni, vera með e‑ð í láni} \dicDirectTranslationCS{(vy)půjčit si (co), mít (co) půjčené}  \textbf{3.} \dicSynonym{gæfa} \dicDirectTranslationCS{štěstí, štěstěna};  \dicPhraseIS{lánið leikur við e‑n} \dicLangCat{přen.} \dicDirectTranslationCS{(kdo) má štěstí, na (koho) se usmívá štěstí};  \dicPhraseIS{það er lán í óláni að} \dicLangCat{přen.} \dicDirectTranslationCS{je to štěstí v~neštěstí, že}
\dicEntry[lána] \dicTerm{lán|a} \dicsymFrequent\  \dicIPA{{l}{au}{\textlengthmark}{n}{a}} \dicPos{v}[1] \dicFlx{(‑aði)}[1] \dicFlx{dat + acc} \dicDirectTranslationCS{(vy)půjčit, zapůjčit};  \dicPhraseIS{lána e‑m e‑ð} \dicDirectTranslationCS{půjčit (komu co)} \dicExampleIS{lána e‑m bókina} \dicExampleCS{půjčit (komu) knížku};  \dicPhraseIS{fá e‑ð lánað} \dicDirectTranslationCS{(vy)půjčit si (co)};  \dicIdiom{lána}[út]{ \dicPhraseIS{lána út e‑ð}} \dicDirectTranslationCS{(za)půjčovat (co)}
\dicEntry[lánardrottinn] \dicTerm{lánar··drott|inn} \dicIPA{{l}{au}{\textlengthmark}{n}{a}{r}{\textsubring{d}}{r}{\textopeno}{h}{\textsubring{d}}{\textsci}{\textsubring{n}}} \dicPos{m}[6] \dicFlx{(‑ins, ‑nar)}[39] \dicDirectTranslationCS{věřitel(ka)}
\dicEntry[lánasjóður] \dicTerm{lána··sjóð|ur} \dicIPA{{l}{au}{\textlengthmark}{n}{a}{s}{j}{ou}{ð}{\textscy}{\textsubring{r}}} \dicPos{m}[9] \dicFlx{(‑s, ‑ir)}[6] \dicFieldCat{ekon.} \dicDirectTranslationCS{úvěrový fond}
\dicEntry[lánastofnun] \dicTerm{lána··stofn|un} \dicIPA{{l}{au}{\textlengthmark}{n}{a}{s}{\textsubring{d}}{\textopeno}{\textsubring{b}}{n}{\textscy}{\textsubring{n}}} \dicPos{f}[7] \dicFlx{(‑unar, ‑anir)}[8] \dicFieldCat{ekon.} \dicDirectTranslationCS{úvěrová společnost}
\dicEntry[lánaveitandi] \dicTerm{lána··veit·|andi} \dicIPA{{l}{au}{\textlengthmark}{n}{a}{v}{ei}{\textsubring{d}}{a}{n}{\textsubring{d}}{\textsci}} \dicPos{m}[2] \dicFlx{(‑anda, ‑endur)}[1] \dicDirectTranslationCS{věřitel(ka)}
\dicEntry[lánlaus] \dicTerm{lán··laus} \dicIPA{{l}{au}{n}{l}{\oe i}{s}} \dicPos{adj}[5]\dicFlx{}[-1] \dicDirectTranslationCS{smolařský, nemající štěstí}
\dicEntry[lánsamur] \dicTerm{lán··|samur} \dicIPA{{l}{au}{n}{s}{a}{m}{\textscy}{\textsubring{r}}} \dicPos{adj}[1] \dicFlx{(f ‑söm)}[2] \dicSynonym{gæfusamur} \dicDirectTranslationCS{šťastný, mající štěstí}
\dicEntry[lánsfé] \dicTerm{láns··|fé} \dicIPA{{l}{au}{n}{s}{f}{j}{\textepsilon}} \dicPos{n}[3] \dicFlx{(‑fjár)}[5] \dicDirectTranslationCS{půjčka, půjčené peníze}
\dicEntry[lánsfjármögnun] \dicTerm{láns·fjár··mögn|un} \dicIPA{{l}{au}{n}{s}{f}{j}{au}{r}{m}{\oe}{\r{g}}{n}{\textscy}{\textsubring{n}}} \dicPos{f}[7] \dicFlx{(‑unar)}[12] \dicFieldCat{ekon.} \dicDirectTranslationCS{financování dluhu}
\begin{xtolerant}{}{1pt}
\dicEntry[lánskjaravísitala] \dicTerm{láns·kjara··vísi·|tala}\addthinS\dicIPA{{l}\-{au}\-{n}\-{s}\-{c\smash{\textsuperscript{h}}}\-{a}\-{r}\-{a}\-{v}\-{i}\-{s}\-{\textsci}\-{t\smash{\textsuperscript{h}}}\-{a}\-{l}\-{a}\-}\addthinS\dicPos{f}[1]\addthinS\dicFlx{(‑tölu, ‑tölur)}[14] \dicFieldCat{ekon.} \dicDirectTranslationCS{index úvěrových podmínek}
\end{xtolerant}
\dicEntry[lánskjör] \dicTerm{láns··kjör} \dicIPA{{l}{au}{n}{s}{c\smash{\textsuperscript{h}}}{\oe}{\textsubring{r}}} \dicPos{n}[2] \dicFlx{pl}[9] \dicFieldCat{ekon.} \dicDirectTranslationCS{úvěrové podmínky}
\dicEntry[lánstími] \dicTerm{láns··tím|i} \dicIPA{{l}{au}{n}{s}{t\smash{\textsuperscript{h}}}{i}{m}{\textsci}} \dicPos{m}[1] \dicFlx{(‑a)}[3] \textbf{1.} \dicFieldCat{ekon.} \dicDirectTranslationCS{doba splatnosti úvěru}  \textbf{2.} \dicDirectTranslationCS{výpůjční doba (v~knihovně ap.)}
\dicEntry[lánstraust] \dicTerm{láns··traust} \dicIPA{{l}{au}{n}{s}{t\smash{\textsuperscript{h}}}{r}{\oe i}{s}{\textsubring{d}}} \dicPos{n}[2] \dicFlx{(‑s)}[2] \dicFieldCat{ekon.} \dicDirectTranslationCS{úvěruschopnost}
\dicEntry[lántakandi] \dicTerm{lán··tak·|andi} \dicIPA{{l}{au}{n}{t\smash{\textsuperscript{h}}}{a}{\r{g}}{a}{n}{\textsubring{d}}{\textsci}} \dicPos{m}[2] \dicFlx{(‑anda, ‑endur)}[1] \dicSynonym{lánþegi} \dicDirectTranslationCS{vypůjčovatel(ka), příjemce\,/\addthin příjemkyně půjčky}
\dicEntry[lántaki] \dicTerm{lán··tak|i} \dicIPA{{l}{au}{n}{t\smash{\textsuperscript{h}}}{a}{\r{\textObardotlessj}}{\textsci}} \dicPos{m}[1] \dicFlx{(‑a, ‑ar)}[8] \dicSynonym{lánþegi} \dicDirectTranslationCS{vypůjčovatel(ka), příjemce\,/\addthin příjemkyně půjčky}
\dicEntry[lánveiting] \dicTerm{lán··veit·ing} \dicIPA{{l}{au}{n}{v}{ei}{\textsubring{d}}{i}{\ng}{\r{g}}} \dicPos{f}[4] \dicFlx{(‑ar, ‑ar)}[5] \dicFieldCat{ekon.} \dicDirectTranslationCS{poskytnutí\,/\addthin poskytování půjčky\,/\addthin úvěru}
\dicEntry[lánþegi] \dicTerm{lán··þeg|i} \dicIPA{{l}{au}{n}{\texttheta}{ei}{\textsci}} \dicPos{m}[1] \dicFlx{(‑a, ‑ar)}[1] \dicDirectTranslationCS{vypůjčovatel(ka), příjemce\,/\addthin příjemkyně půjčky}
\dicEntry[láréttur] \dicTerm{lá··réttur} \dicIPA{{l}{au}{\textlengthmark}{r}{j}{\textepsilon}{h}{\textsubring{d}}{\textscy}{\textsubring{r}}} \dicPos{adj}[1]\dicFlx{}[-10] \dicDirectTranslationCS{vodorovný, horizontální}
\dicEntry[lárpera] \dicTerm{lár··per|a} \dicIPA{{l}{au}{r}{p\smash{\textsuperscript{h}}}{\textepsilon}{r}{a}} \dicPos{f}[1] \dicFlx{(‑u, ‑ur)}[7] \dicDirectTranslationCS{avokádo} \dicIndirectTranslationCS{(plod)}
\dicEntry[lárperutré] \dicTerm{lár·peru··tré} \dicIPA{{l}{au}{r}{p\smash{\textsuperscript{h}}}{\textepsilon}{r}{\textscy}{t\smash{\textsuperscript{h}}}{r}{j}{\textepsilon}} \dicPos{n}[2] \dicFlx{(‑s, ‑)}[36] \dicFieldCat{bot.} \dicDirectTranslationCS{hruškovec přelahodný} \textit{(l.~{\textLA{Persea americana}})}  \dicsymPhoto\ 
\dicFigure{54128.jpg}{Lárperutré}{Lárperutré - Forest \& Kim [[p:2684;Starr]], Biolib, CC-BY}
\dicEntry[lárviðarlauf] \dicTerm{lár·viðar··lauf} \dicIPA{{l}{au}{r}{v}{\textsci}{ð}{a}{r}{l}{\oe i}{f}} \dicPos{n}[2] \dicFlx{(‑s, ‑)}[5] \dicDirectTranslationCS{bobkový list}
\dicEntry[lárviðarrós] \dicTerm{lár·viðar··rós} \dicIPA{{l}{au}{r}{v}{\textsci}{ð}{a}{r}{ou}{s}} \dicPos{f}[7] \dicFlx{(‑ar, ‑ir)}[1] \dicFieldCat{bot.} \dicSynonym{nería} \dicDirectTranslationCS{oleandr, oleandr obecný} \textit{(l.~{\textLA{Nerium oleander}})}  \dicsymPhoto\ 
\dicFigure{ds_image_larvidarros_0_1.jpg}{Lárviðarrós}{Lárviðarrós - Alvesgaspar, CC BY-SA 3.0}
\dicEntry[lárviður] \dicTerm{lár··við|ur} \dicIPA{{l}{au}{r}{v}{\textsci}{ð}{\textscy}{\textsubring{r}}} \dicPos{m}[10] \dicFlx{(‑ar, ‑ir)}[19] \dicFieldCat{bot.} \dicDirectTranslationCS{vavřín, vavřín vznešený} \textit{(l.~{\textLA{Laurus nobilis}})}  \dicsymPhoto\ 
\dicFigure{55905.jpg}{Lárviður}{Lárviður - Forest \& Kim [[p:2684;Starr]], Biolib, CC-BY}
\dicEntry[lás] \dicTerm{lás} \dicIPA{{l}{au}{\textlengthmark}{s}} \dicPos{m}[4] \dicFlx{(‑s, ‑ar)}[25] \textbf{1.} \dicDirectTranslationCS{zámek (u~dveří ap.)} \dicExampleIS{lásinn á hjólinu} \dicExampleCS{zámek na kolo};  \dicPhraseIS{skella e‑u í lás} \dicDirectTranslationCS{zabouchnout (si) (co) (dveře ap.)}  \textbf{2.} \dicDirectTranslationCS{pojistka} \dicIndirectTranslationCS{(součást střelné zbraně)} \dicExampleIS{lás á byssu} \dicExampleCS{pojistka pistole}
\dicEntry[lásasmiður] \dicTerm{lása··smið|ur} \dicIPA{{l}{au}{\textlengthmark}{s}{a}{s}{m}{\textsci}{ð}{\textscy}{\textsubring{r}}} \dicPos{m}[9] \dicFlx{(‑s, ‑ir)}[8] \dicDirectTranslationCS{zámečník, zámečnice}
\dicEntry[lásbogi] \dicTerm{lás··bog|i} \dicIPA{{l}{au}{\textlengthmark}{s}{\textsubring{b}}{\textopeno i}{j}{\textsci}} \dicPos{m}[1] \dicFlx{(‑a, ‑ar)}[1] \dicDirectTranslationCS{kuše, samostříl}
\dicEntry[lásum] \dicTerm{lásum} \dicIPA{{l}{au}{\textlengthmark}{s}{\textscy}{\textsubring{m}}} \dicPos{v} \dicFlx{ind pf pl 1 pers} \dicLink{lesa}
\dicEntry[lát] \dicTerm{lát} \dicIPA{{l}{au}{\textlengthmark}{\textsubring{d}}} \dicPos{n}[2] \dicFlx{(‑s, ‑)}[5] \textbf{1.} \dicSynonym{andlát} \dicDirectTranslationCS{skon, úmrtí} \dicExampleIS{auglýsa lát e‑rs} \dicExampleCS{oznámit (čí) úmrtí}  \textbf{2.} \dicSynonym{hlé} \dicDirectTranslationCS{přestávka, chvíle oddechu} \dicExampleIS{Það var ekkert lát á storminum alla nóttina.} \dicExampleCS{Bouře zuřila bez přestávky celou noc.}  \textbf{3.} \dicSynonym*{undanlát} \dicDirectTranslationCS{poddajnost} \dicExampleIS{Það er ekkert lát á honum.} \dicExampleCS{Není v~něm kousek poddajnosti.}
\dicEntry[láta] \dicTerm{láta} \dicsymFrequent\  \dicIPA{{l}{au}{\textlengthmark}{\textsubring{d}}{a}} \dicPos{v}[6] \dicFlx{(læt, lét, létum, léti, látið)}[84] \dicFlx{acc} \textbf{1.} \dicSynonym*{leyfa að e‑að gerist} \dicDirectTranslationCS{(po)nechat, (po)nechávat} \dicExampleIS{láta e‑n lesa} \dicExampleCS{nechat (koho) číst};  \dicPhraseIS{láta e‑n í friði} \dicDirectTranslationCS{nechat (koho) na pokoji};  \dicPhraseIS{láta e‑n vera} \dicDirectTranslationCS{nechat (koho) být}  \textbf{2.} \dicDirectTranslationCS{(po)nechat, (po)nechávat, zanechat, zanechávat} \dicExampleIS{láta bókina þar} \dicExampleCS{nechat tam knížku};  \dicPhraseIS{láta e‑ð af hendi} \dicDirectTranslationCS{dát (co) z~ruky};  \dicPhraseIS{láta e‑m e‑ð í té} \dicDirectTranslationCS{poskytnout (komu co), předat (komu co)}  \textbf{3.} \dicSynonym*{hegða sér} \dicDirectTranslationCS{chovat se, jednat} \dicExampleIS{láta eins og fífl} \dicExampleCS{chovat se jako hlupák};  \dicPhraseIS{láta sem ekkert sé} \dicDirectTranslationCS{dělat jakoby nic; dělat, jako by se nic nestalo};  \dicPhraseIS{láta öllum illum látum} \dicDirectTranslationCS{dělat problémy\,/\addthin potíže};  \dicPhraseIS{láttu ekki svona} \dicFlx{imper} \dicDirectTranslationCS{nech toho}  \textbf{4.} \dicPhraseIS{láta lífið} \dicDirectTranslationCS{přijít o~život}  \textbf{5.} \dicPhraseIS{láta í haf} \dicDirectTranslationCS{vyplout na moře};  \dicIdiom{láta}[að]{ \dicPhraseIS{láta vel að e‑n}} \dicDirectTranslationCS{být milý\,/\addthin laskavý ke (komu)};  \dicIdiom{láta}[af]{ \dicPhraseIS{láta af e‑u}} \dicDirectTranslationCS{(za)nechat (čeho) (práce ap.)};  \dicIdiom{láta}[aftur]{ \dicPhraseIS{láta aftur e‑ð}} \dicDirectTranslationCS{zavřít\,/\addthin zavírat (co)} \dicExampleIS{láta aftur hurðina\,/\addthin dyrnar} \dicExampleCS{zavřít dveře};  \dicIdiom{láta}[eftir]{ \dicPhraseIS{láta allt eftir sér}} \dicDirectTranslationCS{dopřávat si, nic si neodepřít}; { \dicPhraseIS{láta e‑ð eftir e‑m}} \dicDirectTranslationCS{nechat (co) na (kom), neodepřít (komu co)}; { \dicPhraseIS{láta eftir}} \dicDirectTranslationCS{podřídit se, podvolit se} \dicExampleIS{Hann varð að láta eftir.} \dicExampleCS{Musel se podřídit.};  \dicIdiom{láta}[nærri]{ \dicPhraseIS{það lætur nærri}} \dicFlx{impers} \dicDirectTranslationCS{moc tomu nechybí, téměř};  \dicIdiom{láta}[sér]{ \dicPhraseIS{láta sér e‑ð vel líka}} \dicDirectTranslationCS{být s~(čím) spokojený}; { \dicPhraseIS{láta sér ekki bregða}} \dicDirectTranslationCS{nenechat se překvapit\,/\addthin zaskočit}; { \dicPhraseIS{láta sér fátt um finnast}} \dicDirectTranslationCS{nedat najevo příliš velký zájem};  \dicIdiom{láta}[sig]{ \dicPhraseIS{láta sig ekki}} \dicDirectTranslationCS{nevzdat se, nevzdávat se};  \dicIdiom{láta}[undan]{ \dicPhraseIS{láta undan e‑m}} \dicSynonym{beygja\smash{\textsuperscript{2}}} \dicDirectTranslationCS{podvolit se (komu), podrobit se (komu)}; { \dicPhraseIS{láta undan}} \dicDirectTranslationCS{povolit, spadnout (střecha ap.)};  \dicIdiom{láta}[uppi]{ \dicPhraseIS{láta e‑ð uppi}} \dicDirectTranslationCS{vyjevit (co), dát (co) najevo};  \dicIdiom{láta}[út úr]{ \dicPhraseIS{láta e‑ð út úr sér}} \dicDirectTranslationCS{(dovolit si) říct (co)};  \dicIdiom{láta}[yfir]{ \dicPhraseIS{láta illa yfir e‑u}} \dicDirectTranslationCS{neschvalovat (co), odsuzovat (co)} \dicExampleIS{láta illa yfir ástandinu} \dicExampleCS{neschvalovat takovýto stav};  \dicIdiom{látast}{ \dicPhraseIS{látast}} \dicFlx{refl} {\textbf{a.}} \dicDirectTranslationCS{předstírat} \dicExampleIS{látast vera reiður} \dicExampleCS{předstírat rozčilení};  {\textbf{b.}} \dicDirectTranslationCS{skonat, zemřít} \dicExampleIS{látast úr elli} \dicExampleCS{zemřít v~důsledku vysokého věku};  \dicIdiom{láta}{ \dicPhraseIS{láta e‑ð eiga sig}} \dicDirectTranslationCS{odpustit si (co)}; { \dicPhraseIS{láta e‑ð ekki á sig fá}} \dicDirectTranslationCS{nepoddat se (čemu)}; { \dicPhraseIS{láta e‑ð í ljós}} \dicDirectTranslationCS{dát (co) najevo, vyjádřit (co)} \dicExampleIS{láta skoðun sína í ljós} \dicExampleCS{dát najevo své názory}; { \dicPhraseIS{láta fara vel um sig}} \dicDirectTranslationCS{udělat si pohodlí} \dicExampleIS{Afi lætur fara vel um sig í hægindastólnum.} \dicExampleCS{Děda si dělá pohodlí v~křesle.}; { \dicPhraseIS{láta ekki hugfallast}} \dicDirectTranslationCS{neztratit kuráž}; { \dicPhraseIS{láta ekki e‑ð hjá líða}} \dicDirectTranslationCS{nenechat si ujít (co)}; { \dicPhraseIS{láta til skarar skríða}} \dicLangCat{přen.} \dicDirectTranslationCS{jít na věc}
\dicEntry[látalæti] \dicTerm{láta··læti} \dicIPA{{l}{au}{\textlengthmark}{\textsubring{d}}{a}{l}{a}{i}{\textsubring{d}}{\textsci}} \dicPos{n}[2] \dicFlx{pl}[19] \dicSynonym{uppgerð} \dicDirectTranslationCS{předstírání, přetvářka, pokrytectví}
\dicEntry[látbragð] \dicTerm{lát··bragð} \dicIPA{{l}{au}{\textlengthmark}{\textsubring{d}}{\textsubring{b}}{r}{a}{\textbabygamma}{\texttheta}} \dicPos{n}[2] \dicFlx{(‑s)}[2] \textbf{1.} \dicDirectTranslationCS{posunek, gesto}  \textbf{2.} \dicDirectTranslationCS{pantomima}
\dicEntry[látbragðsleikur] \dicTerm{lát·bragðs··leik|ur} \dicIPA{{l}{au}{\textlengthmark}{\textsubring{d}}{\textsubring{b}}{r}{a}{\textbabygamma}{ð}{s}{l}{ei}{\r{g}}{\textscy}{\textsubring{r}}} \dicPos{m}[6] \dicFlx{(‑s)}[17] \dicDirectTranslationCS{němohra, pantomima}
\dicEntry[látið] \dicTerm{látið} \dicIPA{{l}{au}{\textlengthmark}{\textsubring{d}}{\textsci}{\texttheta}} \dicPos{v} \dicFlx{supin} \dicLink{láta}
\dicEntry[látinn] \dicTerm{látinn} \dicsymFrequent\  \dicIPA{{l}{au}{\textlengthmark}{\textsubring{d}}{\textsci}{\textsubring{n}}} \dicPos{adj}[6]\dicFlx{}[-6] \dicSynonym{dáinn} \dicDirectTranslationCS{zesnulý, zemřelý} \dicExampleIS{níu látnir og fjölmargir slasaðir} \dicExampleCS{devět mrtvých a~mnoho zraněných};  \dicPhraseIS{vera vant við látinn} \dicDirectTranslationCS{mít napilno, být zaneprázdněný};  \dicPhraseIS{vera vel látinn} \dicDirectTranslationCS{být populární\,/\addthin uznávaný}
\dicEntry[látlaus] \dicTerm{lát··laus} \dicIPA{{l}{au}{\textlengthmark}{\textsubring{d}}{l}{\oe i}{s}} \dicPos{adj}[5]\dicFlx{}[-1] \textbf{1.} \dicSynonym{tilgerðarlaus} \dicDirectTranslationCS{prostý, skromný, nenáročný}  \textbf{2.} \dicSynonym{linnulaus} \dicDirectTranslationCS{vytrvalý, soustavný, nepřetržitý} \dicExampleIS{látlaus rigning} \dicExampleCS{soustavný déšť}
\dicEntry[látleysi] \dicTerm{lát··leysi} \dicIPA{{l}{au}{\textlengthmark}{\textsubring{d}}{l}{ei}{s}{\textsci}} \dicPos{n}[2] \dicFlx{(‑s)}[20] \dicSynonym*{tilgerðarleysi} \dicDirectTranslationCS{prostota, přirozenost, jednoduchost}
\dicEntry[látún] \dicTerm{látún} \dicIPA{{l}{au}{\textlengthmark}{t\smash{\textsuperscript{h}}}{u}{\textsubring{n}}} \dicPos{n}[2] \dicFlx{(‑s)}[2] \dicDirectTranslationCS{mosaz}
\dicEntry[látæði] \dicTerm{lát··æði} \dicIPA{{l}{au}{\textlengthmark}{\textsubring{d}}{a}{i}{ð}{\textsci}} \dicPos{n}[2] \dicFlx{(‑s)}[20] \textbf{1.} \dicSynonym{hegðun} \dicDirectTranslationCS{chování, jednání}  \textbf{2.} \dicSynonym{látbragð} \dicDirectTranslationCS{posunek, gesto}
\dicEntry[lávarður] \dicTerm{lávarð|ur} \dicIPA{{l}{au}{\textlengthmark}{v}{a}{r}{ð}{\textscy}{\textsubring{r}}} \dicPos{m}[6] \dicFlx{(‑s\,/\addthin ‑ar, ‑ar)}[64] \dicDirectTranslationCS{lord}
\dicEntry[Lbs.] \dicTerm{Lbs.} \dicPos{zkr} \dicPhraseIS{Landsbókasafn} \dicDirectTranslationCS{Národní knihovna}
\dicEntry[ld.] \dicTerm{ld.} \dicPos{zkr} \dicPhraseIS{laugardagur} \dicDirectTranslationCS{sobota}
\dicEntry[leðja] \dicTerm{leðj|a} \dicIPA{{l}{\textepsilon}{ð}{j}{a}} \dicPos{f}[1] \dicFlx{(‑u)}[5] \dicSynonym{for\smash{\textsuperscript{1}}} \dicDirectTranslationCS{bláto, bahno} \dicExampleIS{sökkva ofan í leðjuna} \dicExampleCS{zapadnout do bláta}
\dicEntry[leður] \dicTerm{leður} \dicsymFrequent\  \dicIPA{{l}{\textepsilon}{\textlengthmark}{ð}{\textscy}{\textsubring{r}}} \dicPos{n}[2] \dicFlx{(‑s, ‑)}[25] \dicDirectTranslationCS{kůže} \dicIndirectTranslationCS{(materiál)} \dicExampleIS{vörur úr leðri} \dicExampleCS{výrobky z~kůže}
\dicEntry[leðurblaka] \dicTerm{leður··|blaka} \dicIPA{{l}{\textepsilon}{\textlengthmark}{ð}{\textscy}{r}{\textsubring{b}}{l}{a}{\r{g}}{a}} \dicPos{f}[1] \dicFlx{(‑blöku, ‑blökur)}[20] \textbf{1.} \dicPhraseIS{leðurblökur} \dicFlx{pl} \dicFieldCat{zool.} \dicDirectTranslationCS{letouni} \textit{(l.~{\textLA{Chiroptera}})}  \dicsymPhoto\   \textbf{2.} \dicFieldCat{zool.} \dicDirectTranslationCS{netopýr} \textit{(l.~{\textLA{Microchiroptera}})}
\dicFigure{62444.jpg}{Leðurblaka}{Leðurblaka - Krejčík Stanislav, Biolib, Copyright/CC-BY}
\dicEntry[leðurjakki] \dicTerm{leður··jakk|i} \dicIPA{{l}{\textepsilon}{\textlengthmark}{ð}{\textscy}{r}{j}{a}{h}{\r{\textObardotlessj}}{\textsci}} \dicPos{m}[1] \dicFlx{(‑a, ‑ar)}[8] \dicDirectTranslationCS{kožená bunda}
\dicEntry[leðurlíki] \dicTerm{leður··líki} \dicIPA{{l}{\textepsilon}{\textlengthmark}{ð}{\textscy}{r}{l}{i}{\r{\textObardotlessj}}{\textsci}} \dicPos{n}[2] \dicFlx{(‑s, ‑)}[16] \dicDirectTranslationCS{koženka}
\dicEntry[leg] \dicTerm{leg} \dicIPA{{l}{\textepsilon}{\textlengthmark}{x}} \dicPos{n}[2] \dicFlx{(‑s, ‑)}[5] \textbf{1.} \dicDirectTranslationCS{umístění, položení}  \textbf{2.} \dicFieldCat{anat.} \dicDirectTranslationCS{děloha}
\dicEntry[lega] \dicTerm{leg|a} \dicsymFrequent\  \dicIPA{{l}{\textepsilon}{\textlengthmark}{\textbabygamma}{a}} \dicPos{f}[1] \dicFlx{(‑u, ‑ur)}[19] \textbf{1.} \dicSynonym{staðsetning} \dicDirectTranslationCS{uložení, pozice} \dicExampleIS{landfræðileg lega Íslands} \dicExampleCS{geografická pozice Islandu}  \textbf{2.} \dicSynonym{rúmlega\smash{\textsuperscript{1}}} \dicDirectTranslationCS{ležení (na lůžku)}  \textbf{3.} \dicFieldCat{techn.} \dicDirectTranslationCS{ložisko}
\dicEntry[leggja] \dicTerm{leggja} \dicsymFrequent\  \dicIPA{{l}{\textepsilon}{\r{\textObardotlessj}}{\textlengthmark}{a}} \dicPos{v}[4] \dicFlx{(legg, lagði, lögðum, legði, lagt)}[20] \dicFlx{acc\,/\addthin dat} \textbf{1.} \dicFlx{acc} \dicDirectTranslationCS{položit, pokládat, umístit, umísťovat, uložit, ukládat} \dicExampleIS{leggja hlutinn á borðið} \dicExampleCS{položit věc na stůl};  \dicPhraseIS{leggja sig} \dicDirectTranslationCS{položit se, lehnout si, natáhnout se}  \textbf{2.} \dicFlx{acc} \dicDirectTranslationCS{(na)instalovat (potrubí ap.), (na)táhnout (elek. vedení ap.), položit, pokládat (dlaždice ap.)} \dicExampleIS{leggja nýjan veg} \dicExampleCS{položit novou silnici}  \textbf{3.} \dicDirectTranslationCS{vyjít, vyjet, vyrazit};  \dicPhraseIS{leggja af\,/\addthin á stað} \dicDirectTranslationCS{vyrazit (na cestu ap.), vydat se} \dicExampleIS{Hún leggur af stað til Akureyrar.} \dicExampleCS{Vyráží do Akureyri.}  \textbf{4.} \dicFlx{dat} \dicDirectTranslationCS{(za)parkovat} \dicExampleIS{leggja bílnum hjá búðinni} \dicExampleCS{zaparkovat auto u~obchodu}  \textbf{5.} \dicFlx{dat} \dicDirectTranslationCS{vyřadit, vyřazovat} \dicExampleIS{leggja gömlu kaffivélinni} \dicExampleCS{vyřadit starý kávovar}  \textbf{6.} \dicFlx{acc} \dicDirectTranslationCS{porazit, porážet (protivníka ap.)}  \textbf{7.} \dicFlx{acc} \dicDirectTranslationCS{(za)útočit, napadnout, napadat}  \textbf{8.} \dicPhraseIS{e‑ð leggur} \dicFlx{impers} {\textbf{a.}} \dicDirectTranslationCS{(co) se šíří, (co) se rozšiřuje (vůně ap.)};  {\textbf{b.}} \dicDirectTranslationCS{(co) zamrzá};  \dicIdiom{leggja}[að]{ \dicPhraseIS{leggja hart að sér}} \dicDirectTranslationCS{tvrdě pracovat, vyvinout veškeré úsilí}; { \dicPhraseIS{leggja fast að e‑m}} \dicDirectTranslationCS{vyvíjet tlak na (koho)}; { \dicPhraseIS{leggja e‑ð að veði}} \dicDirectTranslationCS{dát (co) v~sázku, vsadit (co)};  \dicIdiom{leggja}[af]{ \dicPhraseIS{leggja af}} \dicDirectTranslationCS{(z)hubnout} \dicExampleIS{Hún lagði af vegna þess að hún var veik.} \dicExampleCS{Zhubla, protože byla nemocná.};  \dicIdiom{leggja}[aftur]{ \dicPhraseIS{leggja aftur e‑ð}} \dicSynonym{loka\smash{\textsuperscript{2}}} \dicDirectTranslationCS{zavřít (co), zamknout (co)} \dicExampleIS{leggja aftur augun} \dicExampleCS{zavřít oči};  \dicIdiom{leggja}[á]{ \dicPhraseIS{leggja á borð}} \dicDirectTranslationCS{prostřít (stůl)}; { \dicPhraseIS{leggja ást á e‑n}} \dicDirectTranslationCS{zamilovat se do (koho), milovat (koho)}; { \dicPhraseIS{leggja áherslu á e‑ð}} \dicDirectTranslationCS{zdůraznit\,/\addthin zdůrazňovat (co), klást na (co) důraz}; { \dicPhraseIS{leggja mikið á sig}} \dicSynonym*{beita sig hörku} \dicDirectTranslationCS{opřít se do toho, vynasnažit se}; { \dicPhraseIS{leggja á}} \dicDirectTranslationCS{zavěsit, položit (telefon)} \dicExampleIS{Hann lagði símann á.} \dicExampleCS{Položil telefon.}; { \dicPhraseIS{leggja e‑ð á minnið, leggja e‑ð í minni}} \dicDirectTranslationCS{uložit (co) do paměti, zapamatovat si (co)}; { \dicPhraseIS{leggja skatt á e‑ð}} \dicDirectTranslationCS{(z)danit (co), uvalit daň na (co)}; { \dicPhraseIS{leggja hatur á e‑n}} \dicDirectTranslationCS{nenávidět (koho)}; { \dicPhraseIS{leggja dul á e‑ð}} \dicDirectTranslationCS{(u)tajit (co)}; { \dicPhraseIS{leggja stund á e‑ð}} \dicDirectTranslationCS{věnovat se (čemu)};  \dicIdiom{leggja}[fram]{ \dicPhraseIS{leggja sig allan fram}} \dicDirectTranslationCS{vynasnažit se, vyvinout úsilí}; { \dicPhraseIS{leggja fram e‑ð}} \dicDirectTranslationCS{předložit\,/\addthin předkládat (co) (k~diskuzi ap.)};  \dicIdiom{leggja}[fyrir]{ \dicPhraseIS{leggja e‑ð fyrir e‑n}} \dicDirectTranslationCS{předložit\,/\addthin předkládat (komu co) (otázku ap.)}; { \dicPhraseIS{leggja fyrir sig e‑ð}} \dicDirectTranslationCS{zabývat se (čím), být zapojen do (čeho)}; { \dicPhraseIS{leggja fyrir}} \dicDirectTranslationCS{odkládat, šetřit};  \dicIdiom{leggja}[inn]{ \dicPhraseIS{leggja e‑ð inn}} \dicDirectTranslationCS{uložit\,/\addthin ukládat (co)} \dicExampleIS{Ég legg peninga inn í bankann.} \dicExampleCS{Ukládám si peníze do banky.};  \dicIdiom{leggja}[í]{ \dicPhraseIS{leggja í e‑ð}} \dicDirectTranslationCS{odvážit se na (co)}; { \dicPhraseIS{leggja í hann}} \dicDirectTranslationCS{vydat se na cestu, vyrazit};  \dicIdiom{leggja}[niður]{ \dicPhraseIS{leggja niður e‑ð}} \dicDirectTranslationCS{skončit s~(čím), skoncovat s~(čím)} \dicExampleIS{leggja niður starf} \dicExampleCS{skončit s~prací}; { \dicPhraseIS{leggja niður vopn}} \dicDirectTranslationCS{složit zbraně}; { \dicPhraseIS{leggja sig niður við e‑ð}} \dicDirectTranslationCS{snížit se k~(čemu)};  \dicIdiom{leggja}[saman]{ \dicPhraseIS{leggja saman e‑ð}} \dicFieldCat{mat.} \dicDirectTranslationCS{sčítat (co), přičítat (co)} \dicExampleIS{Barnið leggur saman tvo og þrjá og fær fimm.} \dicExampleCS{Dítě sečte dva plus tři a~vyjde mu pět.}; { \dicPhraseIS{leggja saman tvo og tvo}} \dicLangCat{přen.} \dicDirectTranslationCS{dát si dvě a~dvě dohromady};  \dicIdiom{leggja}[til]{ \dicPhraseIS{leggja e‑m til e‑ð}} \dicDirectTranslationCS{poskytnout\,/\addthin poskytovat (komu co)}; { \dicPhraseIS{leggja e‑ð til}} \dicDirectTranslationCS{navrhnout\,/\addthin navrhovat (co), předložit návrh o~(čem)} \dicExampleIS{Ég legg til að þessi plaköt verði lagfærð.} \dicExampleCS{Navrhuji, aby tyto plakáty byly upraveny.};  \dicPhraseIS{leggja sér e‑ð til munns} \dicDirectTranslationCS{jíst (co), pojídat (co)};  \dicIdiom{leggja}[undir]{ \dicPhraseIS{leggja undir sig e‑ð}} \dicDirectTranslationCS{podmanit\,/\addthin podmaňovat si (co), podrobit\,/\addthin podrobovat si (co)};  \dicIdiom{leggja}[upp]{ \dicPhraseIS{leggja upp}} \dicDirectTranslationCS{vyrazit, vydat se (na cestu ap.)} \dicExampleIS{leggja upp í ferðina} \dicExampleCS{vydat se na cestu};  \dicIdiom{leggja}[upp úr]{ \dicPhraseIS{leggja mikið\,/\addthin lítið upp úr e‑u}} \dicDirectTranslationCS{pokládat (co) za hodně\,/\addthin málo důležité};  \dicIdiom{leggja}[út]{ \dicPhraseIS{leggja út e‑ð}} {\textbf{a.}} \dicSynonym{borga} \dicDirectTranslationCS{vyplatit (co), přispět (čím) (částkou ap.)};  {\textbf{b.}} \dicSynonym{túlka} \dicDirectTranslationCS{přeložit (co), tlumočit (co)};  \dicIdiom{leggja}[við]{ \dicPhraseIS{leggja rækt við e‑ð}} \dicDirectTranslationCS{kultivovat (co), rozvíjet (co)}; { \dicPhraseIS{leggja e‑ð við e‑ð}} \dicDirectTranslationCS{přidat (co) k~(čemu)} \dicExampleIS{leggja tvo við þrjá} \dicExampleCS{přidat dvě ke třem};  \dicIdiom{leggjast}{ \dicPhraseIS{leggjast}} \dicFlx{refl} \dicDirectTranslationCS{lehnout si, natáhnout se};  \dicIdiom{leggjast}[að]{ \dicPhraseIS{leggjast að bryggju}} \dicFlx{refl} \dicDirectTranslationCS{přistát u~mola};  \dicIdiom{leggjast}[af]{ \dicPhraseIS{leggjast af}} \dicFlx{refl} \dicDirectTranslationCS{vymizet, zaniknout};  \dicIdiom{leggjast}[á]{ \dicPhraseIS{leggjast á eitt}} \dicFlx{refl} \dicDirectTranslationCS{spojit síly, táhnout za jeden provaz}; { \dicPhraseIS{e‑að leggst þungt á e‑n}} \dicFlx{refl} \dicLangCat{přen.} \dicDirectTranslationCS{(co koho) tíží (obavy ap.)};  \dicIdiom{leggjast}[gegn]{ \dicPhraseIS{leggjast gegn e‑u}} \dicFlx{refl} \dicDirectTranslationCS{postavit\,/\addthin stavět se proti (čemu)};  \dicIdiom{leggjast}[í]{ \dicPhraseIS{leggjast í drykkju}} \dicFlx{refl} \dicDirectTranslationCS{oddávat se pití (alkoholu)}; { \dicPhraseIS{leggjast í þunglyndi}} \dicFlx{refl} \dicDirectTranslationCS{upadnout do deprese};  \dicIdiom{leggjast}[með]{ \dicPhraseIS{leggjast með e‑m}} \dicFlx{refl} \dicDirectTranslationCS{vyspat se s~(kým), spát s~(kým)};  \dicIdiom{leggjast}[niður]{ \dicPhraseIS{leggjast niður}} \dicFlx{refl} \dicDirectTranslationCS{být přerušen, přestat};  \dicIdiom{leggjast}[yfir]{ \dicPhraseIS{e‑að leggst yfir e‑ð}} \dicFlx{refl} \dicDirectTranslationCS{(co) se usazuje nad (čím) (mlha nad městem ap.)}; { \dicPhraseIS{leggjast yfir e‑ð}} \dicFlx{refl} \dicDirectTranslationCS{pročíst (co), (důkladně) přečíst (co)}
\dicEntry[leggur] \dicTerm{legg|ur} \dicsymFrequent\  \dicIPA{{l}{\textepsilon}{\r{g}}{\textlengthmark}{\textscy}{\textsubring{r}}} \dicPos{m}[9] \dicFlx{(‑jar\,/\addthin ‑s, ‑ir)}[26] \textbf{1.} \dicSynonym{bein} \dicDirectTranslationCS{kost}  \textbf{2.} \dicSynonym{fótleggur} \dicDirectTranslationCS{noha, dolní končetina} \dicExampleIS{Ég er tilfinningalaus í leggnum.} \dicExampleCS{Nemám v~noze žádný cit.}  \textbf{3.} \dicFieldCat{bot.} \dicSynonym{stilkur} \dicDirectTranslationCS{stopka, stonek};  \dicIdiom{leggur}{ \dicPhraseIS{komast á legg}} \dicFlx{refl} \dicLangCat{přen.} \dicDirectTranslationCS{postavit se na vlastní nohy}
\dicEntry[leggöng] \dicTerm{leg··göng} \dicIPA{{l}{\textepsilon}{\textbabygamma}{\r{g}}{\oe i}{\ng}{\r{g}}} \dicPos{n}[2] \dicFlx{pl}[9] \dicFieldCat{anat.} \dicDirectTranslationCS{vagina, pochva}
\dicEntry[legið] \dicTerm{legið} \dicIPA{{l}{ei}{\textsci}{\texttheta}} \dicPos{v} \dicFlx{supin} \dicLink{liggja}
\dicEntry[legir] \dicTerm{legir} \dicIPA{{l}{ei}{\textsci}{\textsubring{r}}} \dicPos{m} \dicFlx{pl nom} \dicLink{lögur}
\dicEntry[legkaka] \dicTerm{leg··|kaka} \dicIPA{{l}{\textepsilon}{\textbabygamma}{k\smash{\textsuperscript{h}}}{a}{\r{g}}{a}} \dicPos{f}[1] \dicFlx{(‑köku, ‑kökur)}[20] \dicFieldCat{anat.} \dicDirectTranslationCS{placenta, plodový obal}
\dicEntry[legstaður] \dicTerm{leg··stað|ur} \dicIPA{{l}{\textepsilon}{x}{s}{\textsubring{d}}{a}{ð}{\textscy}{\textsubring{r}}} \dicPos{m}[10] \dicFlx{(‑ar, ‑ir)}[14] \dicSynonym{gröf} \dicDirectTranslationCS{hrob(ka)}
\dicEntry[legsteinn] \dicTerm{leg··stein|n} \dicIPA{{l}{\textepsilon}{x}{s}{\textsubring{d}}{ei}{\textsubring{d}}{\textsubring{n}}} \dicPos{m}[6] \dicFlx{(‑s, ‑ar)}[42] \dicDirectTranslationCS{náhrobní kámen}
\dicEntry[legubekkur] \dicTerm{legu··bekk|ur} \dicIPA{{l}{\textepsilon}{\textlengthmark}{\textbabygamma}{\textscy}{\textsubring{b}}{\textepsilon}{h}{\r{g}}{\textscy}{\textsubring{r}}} \dicPos{m}[9] \dicFlx{(‑s\,/\addthin ‑jar, ‑ir)}[26] \dicDirectTranslationCS{pohovka, lehátko}
\dicEntry[legudagur] \dicTerm{legu··dag|ur} \dicIPA{{l}{\textepsilon}{\textlengthmark}{\textbabygamma}{\textscy}{\textsubring{d}}{a}{\textbabygamma}{\textscy}{\textsubring{r}}} \dicPos{m}[6] \dicFlx{(‑s, ‑ar)}[62] \textbf{1.} \dicDirectTranslationCS{den na lůžku (v~nemocnici ap.)}  \textbf{2.} \dicDirectTranslationCS{den (lodi ap.) v~přístavu}
\dicEntry[legugjald] \dicTerm{legu··|gjald} \dicIPA{{l}{\textepsilon}{\textlengthmark}{\textbabygamma}{\textscy}{\r{\textObardotlessj}}{a}{l}{\textsubring{d}}} \dicPos{n}[2] \dicFlx{(‑gjalds, ‑gjöld)}[8] \dicFieldCat{nám.} \dicDirectTranslationCS{kotevné} \dicIndirectTranslationCS{(poplatek za kotvení)}
\dicEntry[legusár] \dicTerm{legu··sár} \dicIPA{{l}{\textepsilon}{\textlengthmark}{\textbabygamma}{\textscy}{s}{au}{\textsubring{r}}} \dicPos{n}[2] \dicFlx{(‑s, ‑)}[5] \dicFieldCat{med.} \dicDirectTranslationCS{proleženina}
\dicEntry[leið] \dicTerm{leið\smash{\textsuperscript{1}}} \dicsymFrequent\  \dicIPA{{l}{ei}{\textlengthmark}{\texttheta}} \dicPos{f}[7] \dicFlx{(‑ar, ‑ir)}[1] \textbf{1.} \dicSynonym{vegur} \dicDirectTranslationCS{cesta, stezka} \dicExampleIS{bein leið} \dicExampleCS{přímá cesta};  \dicPhraseIS{eiga leið hjá} \dicDirectTranslationCS{mít cestu kolem};  \dicPhraseIS{vera á leið(inni) (þangað)} \dicDirectTranslationCS{být na cestě (tam)}  \textbf{2.} \dicSynonym{háttur\smash{\textsuperscript{1}}} \dicDirectTranslationCS{způsob, postup};  \dicPhraseIS{það er engin leið að \dots{}} \dicDirectTranslationCS{není žádný způsob, jak\dots{}}  \textbf{3.} \dicSynonym{vegalengd} \dicDirectTranslationCS{cesta, vzdálenost} \dicExampleIS{dagleið} \dicExampleCS{jednodenní cesta}  \textbf{4.} \dicSynonym{áætlunarleið} \dicDirectTranslationCS{(dopravní) linka};  \dicIdiom{leið}{ \dicPhraseIS{á miðri leið}} \dicFlx{adv} \dicDirectTranslationCS{v~půli cesty}; { \dicPhraseIS{heim á leið}} \dicFlx{adv} \dicDirectTranslationCS{domů}; { \dicPhraseIS{leggja leið sína, leggja leiðir sínar}} \dicDirectTranslationCS{vydat se (do cizí země ap.)} \dicExampleIS{leggja leið sína til Íslands} \dicExampleCS{vydat se na Island};  \dicPhraseIS{um leið} \dicFlx{adv} \dicDirectTranslationCS{zároveň, současně, naráz}; { \dicPhraseIS{um leið og}} \dicFlx{conj} \dicSynonym*{strax og} \dicDirectTranslationCS{jakmile, hned jak} \dicExampleIS{Hann fór að sofa um leið og hann kom heim.} \dicExampleCS{Jakmile přišel domů, šel si lehnout.}; { \dicPhraseIS{vera komin (fimm) mánuði á leið}} \dicDirectTranslationCS{být v~(pátém) měsíci (těhotenství)}
\dicEntry[leið] \dicTerm{leið\smash{\textsuperscript{2}}} \dicIPA{{l}{ei}{\textlengthmark}{\texttheta}} \dicPos{v} \dicFlx{ind pf sg 1 pers} \dicLink{líða}
\dicEntry[leiða] \dicTerm{lei|ða} \dicsymFrequent\  \dicIPA{{l}{ei}{\textlengthmark}{ð}{a}} \dicPos{v}[2] \dicFlx{(‑ddi, ‑tt)}[169] \dicFlx{acc} \textbf{1.} \dicDirectTranslationCS{(při)vést, přivádět, vodit} \dicExampleIS{leiða e‑n þangað} \dicExampleCS{přivést (koho) tam}  \textbf{2.} \dicDirectTranslationCS{(při)vést, přivádět (elektřinu ap.)} \dicExampleIS{leiða rafmagn í hús} \dicExampleCS{přivádět elektřinu do domu};  \dicIdiom{leiða}[að]{ \dicPhraseIS{leiða hugann að e‑u}} \dicDirectTranslationCS{zavést myšlenku na (co)}; { \dicPhraseIS{leiða tal að e‑u}} \dicDirectTranslationCS{zavést řeč na (co), nadhodit (co)};  \dicIdiom{leiða}[af]{ \dicPhraseIS{leiða e‑ð af sér}} \dicDirectTranslationCS{mít (co) za následek, vést k~(čemu), skončit se (čím)};  \dicIdiom{leiða}[fyrir]{ \dicPhraseIS{leiða e‑ð e‑m fyrir sjónir}} \dicDirectTranslationCS{ukázat (komu co), vysvětlit (komu co)};  \dicIdiom{leiða}[hjá]{ \dicPhraseIS{leiða e‑ð hjá sér}} \dicDirectTranslationCS{nevšímat si (čeho), nebrat na vědomí (co)} \dicExampleIS{leiða orð hans hjá sér} \dicExampleCS{nevšímat si jeho slov};  \dicIdiom{leiða}[í]{ \dicPhraseIS{leiða e‑u í ljós}} \dicDirectTranslationCS{(po)odhalit (co)};  \dicIdiom{leiða}[til]{ \dicPhraseIS{leiða til e‑s}} \dicDirectTranslationCS{vést k~(čemu), vyústit v~(co)};  \dicIdiom{leiðast}{ \dicPhraseIS{leiðast}} \dicFlx{refl} \dicDirectTranslationCS{vést se za ruce} \dicExampleIS{Þau leiddust niður götuna.} \dicExampleCS{Šli po ulici a~vedli se za ruce.}; { \dicPhraseIS{leiðast út í drykkju}} \dicFlx{refl} \dicDirectTranslationCS{začít pít (alkohol)}; { \dicPhraseIS{láta til leiðast}} \dicDirectTranslationCS{nechat se přesvědčit};  \dicPhraseIS{e‑m leiðist} \dicFlx{refl impers} \dicDirectTranslationCS{(kdo) se nudí, (komu) je dlouhá chvíle} \dicExampleIS{Mér leiðist.} \dicExampleCS{Nudím se.}; { \dicPhraseIS{e‑m leiðist e‑að}} \dicFlx{refl impers} \dicDirectTranslationCS{(koho) nudí (co), (komu) přijde nudné (co)}
\dicEntry[leiðabók] \dicTerm{leiða··|bók} \dicIPA{{l}{ei}{\textlengthmark}{ð}{a}{\textsubring{b}}{ou}{\r{g}}} \dicPos{f}[8] \dicFlx{(‑bókar, ‑bækur)}[5] \dicDirectTranslationCS{jízdní řád}
\dicEntry[leiðangur] \dicTerm{leið··ang|ur} \dicsymFrequent\  \dicIPA{{l}{ei}{\textlengthmark}{ð}{au}{\ng}{\r{g}}{\textscy}{\textsubring{r}}} \dicPos{m}[5] \dicFlx{(‑urs, ‑rar)}[3] \dicSynonym{könnunarferð} \dicDirectTranslationCS{výprava, expedice, mise} \dicExampleIS{Leiðangurinn tekur tvær vikur.} \dicExampleCS{Expedice potrvá dva týdny.}
\dicEntry[leiðarendi] \dicTerm{leiðar··end|i} \dicIPA{{l}{ei}{\textlengthmark}{ð}{a}{r}{\textepsilon}{n}{\textsubring{d}}{\textsci}} \dicPos{m}[1] \dicFlx{(‑a, ‑ar)}[1] \dicSynonym{ákvörðunarstaður} \dicDirectTranslationCS{cíl cesty, destinace} \dicExampleIS{komast á leiðarenda} \dicExampleCS{dorazit do cíle cesty}
\dicEntry[leiðari] \dicTerm{leið··ar|i} \dicIPA{{l}{ei}{\textlengthmark}{ð}{a}{r}{\textsci}} \dicPos{m}[1] \dicFlx{(‑a, ‑ar)}[13] \textbf{1.} \dicSynonym{grein} \dicDirectTranslationCS{úvodník} \dicIndirectTranslationCS{(článek umístěný na první stránce novin)}  \textbf{2.} \dicFieldCat{fyz.} \dicDirectTranslationCS{vodič (elektřiny ap.)}
\dicEntry[leiðarljós] \dicTerm{leiðar··ljós} \dicIPA{{l}{ei}{\textlengthmark}{ð}{a}{r}{l}{j}{ou}{s}} \dicPos{n}[2] \dicFlx{(‑s, ‑)}[5] \dicLangCat{přen.} \dicDirectTranslationCS{záchytný bod, vodítko};  \dicPhraseIS{hafa e‑ð að leiðarljósi} \dicLangCat{přen.} \dicDirectTranslationCS{mít (co) jako záchytný bod}
\dicEntry[leiðarlok] \dicTerm{leiðar··lok} \dicIPA{{l}{ei}{\textlengthmark}{ð}{a}{r}{l}{\textopeno}{\r{g}}} \dicPos{n}[2] \dicFlx{pl}[1] \dicDirectTranslationCS{závěr\,/\addthin konec cesty (silnice i~putování)}
\dicEntry[leiðarlýsing] \dicTerm{leiðar··lýs·ing} \dicIPA{{l}{ei}{\textlengthmark}{ð}{a}{r}{l}{i}{s}{i}{\ng}{\r{g}}} \dicPos{f}[4] \dicFlx{(‑ar, ‑ar)}[5] \dicDirectTranslationCS{itinerář, cestovní plán}
\dicEntry[leiðarminni] \dicTerm{leiðar··minni} \dicIPA{{l}{ei}{\textlengthmark}{ð}{a}{r}{m}{\textsci}{n}{\textsci}} \dicPos{n}[2] \dicFlx{(‑s, ‑)}[14] \dicDirectTranslationCS{příznačný motiv, leitmotiv}
\dicEntry[leiðarvísir] \dicTerm{leiðar··vís|ir} \dicIPA{{l}{ei}{\textlengthmark}{ð}{a}{r}{v}{i}{s}{\textsci}{\textsubring{r}}} \dicPos{m}[7] \dicFlx{(‑is, ‑ar)}[1] \dicDirectTranslationCS{příručka, manuál, návod} \dicExampleIS{leiðarvísir um vél} \dicExampleCS{manuál k~zařízení}
\dicEntry[leiðbeina] \dicTerm{leið··bein|a} \dicIPA{{l}{ei}{ð}{\textsubring{b}}{ei}{n}{a}} \dicPos{v}[2] \dicFlx{(‑di, ‑t)}[152] \dicFlx{dat} \dicSynonym{kenna} \dicDirectTranslationCS{učit, vyučovat, cvičit, školit} \dicExampleIS{leiðbeina e‑m í e‑u} \dicExampleCS{školit (koho) v~(čem)}
\dicEntry[leiðbeinandi] \dicTerm{leið··bein·|andi} \dicIPA{{l}{ei}{ð}{\textsubring{b}}{ei}{n}{a}{n}{\textsubring{d}}{\textsci}} \dicPos{m}[2] \dicFlx{(‑anda, ‑endur)}[1] \textbf{1.} \dicDirectTranslationCS{školitel(ka), instruktor(ka), cvičitel(ka)}  \textbf{2.} \dicFieldCat{škol.} \dicDirectTranslationCS{nekvalifikovaný učitel, nekvalifikovaná učitelka}
\dicEntry[leiðbeining] \dicTerm{leið··bein·ing} \dicIPA{{l}{ei}{ð}{\textsubring{b}}{ei}{n}{i}{\ng}{\r{g}}} \dicPos{f}[4] \dicFlx{(‑ar, ‑ar)}[5] \textbf{1.} \dicSynonym{leiðsögn} \dicDirectTranslationCS{pokyn, instrukce}  \textbf{2.} \dicPhraseIS{leiðbeiningar} \dicFlx{pl} \dicSynonym{leiðarvísir} \dicDirectTranslationCS{pokyny, návod, manuál, instrukce} \dicExampleIS{leiðbeiningar um notkun} \dicExampleCS{návod k~použití}
\dicEntry[leiði] \dicTerm{leið|i\smash{\textsuperscript{1}}} \dicIPA{{l}{ei}{\textlengthmark}{ð}{\textsci}} \dicPos{m}[1] \dicFlx{(‑a)}[3] \dicSynonym{leiðindi} \dicDirectTranslationCS{nuda, dlouhá chvíle} \dicExampleIS{fá leiða á leikjum} \dicExampleCS{nudit se u~hry}
\dicEntry[leiði] \dicTerm{leiði\smash{\textsuperscript{2}}} \dicIPA{{l}{ei}{\textlengthmark}{ð}{\textsci}} \dicPos{n}[2] \dicFlx{(‑s, ‑)}[14] \textbf{1.} \dicSynonym{gröf} \dicDirectTranslationCS{náhrobek, hrobka} \dicExampleIS{setja kross á leiðið} \dicExampleCS{umístit kříž na náhrobek}  \textbf{2.} \dicSynonym{meðvindur} \dicDirectTranslationCS{příznivý vítr}  \textbf{3.} \dicSynonym{færð} \dicDirectTranslationCS{sjízdnost}
\dicEntry[leiðindamál] \dicTerm{leiðinda··mál} \dicIPA{{l}{ei}{\textlengthmark}{ð}{\textsci}{n}{\textsubring{d}}{a}{m}{au}{\textsubring{l}}} \dicPos{n}[2] \dicFlx{(‑s, ‑)}[5] \dicDirectTranslationCS{nepříjemnost, nepříjemná\,/\addthin otravná věc}
\dicEntry[leiðindi] \dicTerm{leið··indi} \dicIPA{{l}{ei}{\textlengthmark}{ð}{\textsci}{n}{\textsubring{d}}{\textsci}} \dicPos{n}[2] \dicFlx{pl}[19] \textbf{1.} \dicSynonym{leiði\smash{\textsuperscript{1}}} \dicDirectTranslationCS{nuda, nudnost, dlouhá chvíle}  \textbf{2.} \dicSynonym{óánægja} \dicDirectTranslationCS{protivnost, špatná nálada}
\dicEntry[leiðinlegur] \dicTerm{leiðin··legur} \dicsymFrequent\  \dicIPA{{l}{ei}{\textlengthmark}{ð}{\textsci}{n}{l}{\textepsilon}{\textbabygamma}{\textscy}{\textsubring{r}}} \dicPos{adj}[1]\dicFlx{}[-8] \textbf{1.} \dicDirectTranslationCS{nudný, fádní} \dicExampleIS{leiðinlegur kennari} \dicExampleCS{nudný učitel}  \textbf{2.} \dicSynonym{dapurlegur} \dicDirectTranslationCS{smutný, tristní}  \textbf{3.} \dicDirectTranslationCS{protivný, nepříjemný (počasí ap.)}
\dicEntry[leiðitamur] \textls[15]{\dicTerm{leiði··|tamur} \dicIPA{{l}{ei}{\textlengthmark}{ð}{\textsci}{t\smash{\textsuperscript{h}}}{a}{m}{\textscy}{\textsubring{r}}} \dicPos{adj}[1] \dicFlx{(f ‑töm)}[2] \dicSynonym{eftirlátur}} \dicDirectTranslationCS{svolný, povolný}
\dicEntry[leiðni] \dicTerm{leiðn|i} \dicIPA{{l}{ei}{ð}{n}{\textsci}} \dicPos{f}[3] \dicFlx{(‑i)}[3] \dicFieldCat{fyz.} \dicDirectTranslationCS{vodivost}
\dicEntry[leiðrétta] \dicTerm{leið··rétt|a} \dicIPA{{l}{ei}{ð}{r}{j}{\textepsilon}{h}{\textsubring{d}}{a}} \dicPos{v}[2] \dicFlx{(‑i, ‑)}[1] \dicFlx{acc} \dicDirectTranslationCS{opravit, opravovat, (z)korigovat (chyby ap.)} \dicExampleIS{leiðrétta verkefnið} \dicExampleCS{opravit cvičení}
\dicEntry[leiðrétting] \dicTerm{leið··rétt·ing} \dicIPA{{l}{ei}{ð}{r}{j}{\textepsilon}{h}{\textsubring{d}}{i}{\ng}{\r{g}}} \dicPos{f}[4] \dicFlx{(‑ar, ‑ar)}[5] \dicDirectTranslationCS{oprava, opravení, korekce, korektura} \dicExampleIS{leiðrétting  á mistökum} \dicExampleCS{oprava chyb}
\dicEntry[leiðsla] \dicTerm{leiðsl|a} \dicIPA{{l}{ei}{ð}{s}{\textsubring{d}}{l}{a}} \dicPos{f}[1] \dicFlx{(‑u, ‑ur)}[19] \textbf{1.} \dicSynonym{leiðsögn} \dicDirectTranslationCS{pokyn, instrukce}  \textbf{2.} \dicSynonym{rör} \dicDirectTranslationCS{roura, trubka, potrubí}  \textbf{3.} \dicSynonym{rafleiðsla} \dicDirectTranslationCS{(elektrické) vedení}  \textbf{4.} \dicSynonym*{ofurhrifning} \dicDirectTranslationCS{vytržení, trans} \dicExampleIS{vera í leiðslu} \dicExampleCS{být v~transu}
\dicEntry[leiðsögn] \dicTerm{leið··|sögn} \dicIPA{{l}{ei}{ð}{s}{\oe}{\r{g}}{\textsubring{n}}} \dicPos{f}[7] \dicFlx{(‑sagnar)}[19] \dicDirectTranslationCS{vedení, navádění, provádění}
\dicEntry[leiðsögumaður] \dicTerm{leið·sögu··|maður} \dicIPA{{l}{ei}{ð}{s}{\oe}{\textbabygamma}{\textscy}{m}{a}{ð}{\textscy}{\textsubring{r}}} \dicPos{m}[13] \dicFlx{(‑manns, ‑menn)}[2] \dicDirectTranslationCS{průvodce, průvodkyně} \dicExampleIS{leiðsögumaður ferðamanna} \dicExampleCS{průvodce turistů}
\dicEntry[leiðtogi] \dicTerm{leið··tog|i} \dicIPA{{l}{ei}{ð}{t\smash{\textsuperscript{h}}}{\textopeno i}{j}{\textsci}} \dicPos{m}[1] \dicFlx{(‑a, ‑ar)}[1] \dicSynonym{foringi} \dicDirectTranslationCS{vedoucí představitel(ka), lídr(yně)} \dicExampleIS{Flokkurinn hefur kosið sér leiðtoga.} \dicExampleCS{Strana si zvolila vedoucího představitele.}
\dicEntry[leiður] \dicTerm{leiður} \dicsymFrequent\  \dicIPA{{l}{ei}{\textlengthmark}{ð}{\textscy}{\textsubring{r}}} \dicPos{adj}[2]\dicFlx{}[-6] \textbf{1.} \dicSynonym{sorgmæddur} \dicDirectTranslationCS{zarmoucený, cítící lítost} \dicExampleIS{leiður út af e‑u} \dicExampleCS{zarmoucený kvůli čemu}  \textbf{2.} \dicSynonym{hvimleiður} \dicDirectTranslationCS{znuděný, unavený (z~poslouchání nářků ap.)} \dicExampleIS{vera orðinn leiður á verkinu} \dicExampleCS{být už unavený z~práce}
\dicEntry[leif] \dicTerm{leif} \dicsymFrequent\  \dicIPA{{l}{ei}{\textlengthmark}{f}} \dicPos{f}[4] \dicFlx{(‑ar, ‑ar)}[1] \textbf{1.} \dicLangCat{básn.} \dicSynonym{arfur} \dicDirectTranslationCS{dědictví, pozůstalost}  \textbf{2.} \dicPhraseIS{leifar} \dicFlx{pl} \dicDirectTranslationCS{zbytky, ostatky} \dicExampleIS{matarleifar} \dicExampleCS{zbytky jídla}  \textbf{3.} \dicFieldCat{mat.} \dicSynonym{afgangur} \dicDirectTranslationCS{zbytek}
\dicEntry[leifa] \dicTerm{leif|a} \dicIPA{{l}{ei}{\textlengthmark}{v}{a}} \dicPos{v}[2] \dicFlx{(‑ði, ‑t)}[99] \dicFlx{dat} \dicDirectTranslationCS{(za)nechat (jídlo nedojedené ap.)} \dicExampleIS{leifa matnum} \dicExampleCS{nechat jídlo}
\dicEntry[leiftra] \dicTerm{leiftr|a} \dicIPA{{l}{ei}{f}{\textsubring{d}}{r}{a}} \dicPos{v}[1] \dicFlx{(‑aði)}[1] \dicSynonym{glitra} \dicDirectTranslationCS{blýskat se, blyštět se, třpytit se} \dicExampleIS{Stjörnurnar leiftra á himninum.} \dicExampleCS{Hvězdy se blyští na obloze.}
\dicEntry[leiftur] \dicTerm{leift|ur\smash{\textsuperscript{1}}} \dicIPA{{l}{ei}{f}{\textsubring{d}}{\textscy}{\textsubring{r}}} \dicPos{m}[5] \dicFlx{(‑urs, ‑rar)}[1] \dicFieldCat{zool.} \dicDirectTranslationCS{plískavice běloboká} \textit{(l.~{\textLA{Lagenorhynchus acutus}})}  \dicsymPhoto\ 
\dicFigure{ds_image_leiftur_1_1.jpg}{Leiftur}{Leiftur - Twp, CC BY-SA 3.0}
\dicEntry[leiftur] \dicTerm{leiftur\smash{\textsuperscript{2}}} \dicIPA{{l}{ei}{f}{\textsubring{d}}{\textscy}{\textsubring{r}}} \dicPos{n}[2] \dicFlx{(‑s, ‑)}[25] \dicSynonym{blossi} \dicDirectTranslationCS{(za)blýsknutí, zá\-blesk}
\dicEntry[leifturhraði] \dicTerm{leiftur··hrað|i} \dicIPA{{l}{ei}{f}{\textsubring{d}}{\textscy}{\textsubring{r}}{a}{ð}{\textsci}} \dicPos{m}[1] \dicFlx{(‑a)}[3] \dicLangCat{přen.} \dicDirectTranslationCS{rychlost blesku, blesková rychlost}
\dicEntry[leiftursókn] \dicTerm{leiftur··sókn} \dicIPA{{l}{ei}{f}{\textsubring{d}}{\textscy}{\textsubring{r}}{s}{ou}{h}{\r{g}}{\textsubring{n}}} \dicPos{f}[7] \dicFlx{(‑ar, ‑ir)}[1] \dicDirectTranslationCS{bleskový útok}
\dicEntry[leiga] \dicTerm{leig|a} \dicsymFrequent\  \dicIPA{{l}{ei}{\textlengthmark}{\textbabygamma}{a}} \dicPos{f}[1] \dicFlx{(‑u, ‑ur)}[19] \textbf{1.} \dicSynonym*{það að leigja} \dicDirectTranslationCS{(pro)nájem} \dicExampleIS{leiga á íbúð} \dicExampleCS{pronájem bytu}  \textbf{2.} \dicSynonym*{leigugjald} \dicDirectTranslationCS{nájem, nájemné} \dicExampleIS{borga leiguna} \dicExampleCS{zaplatit nájem}
\dicEntry[leigja] \dicTerm{leig|ja} \dicsymFrequent\  \dicIPA{{l}{ei}{j}{\textlengthmark}{a}} \dicPos{v}[2] \dicFlx{(‑ði, ‑t)}[89] \dicFlx{acc} \dicDirectTranslationCS{(pro)najmout, (pro)najímat} \dicExampleIS{leigja íbúð af e‑m} \dicExampleCS{pronajmout si od (koho) byt}
\dicEntry[leigjandi] \dicTerm{leigj··|andi} \dicIPA{{l}{ei}{j}{\textlengthmark}{a}{n}{\textsubring{d}}{\textsci}} \dicPos{m}[2] \dicFlx{(‑anda, ‑endur)}[1] \dicDirectTranslationCS{(pod)nájemník, (pod)nájemnice}
\dicEntry[leigubifreið] \dicTerm{leigu··bif·reið} \dicIPA{{l}{ei}{\textlengthmark}{\textbabygamma}{\textscy}{\textsubring{b}}{\textsci}{v}{r}{ei}{\texttheta}} \dicPos{f}[4] \dicFlx{(‑ar, ‑ar\,/\addthin ‑ir)}[30] \dicLangCat{form.} \dicDirectTranslationCS{taxi, taxislužba}
\dicEntry[leigubílastöð] \textls[10]{\dicTerm{leigu·bíla··stöð} \dicIPA{{l}{ei}{\textlengthmark}{\textbabygamma}{\textscy}{\textsubring{b}}{i}{l}{a}{s}{\textsubring{d}}{\oe}{\texttheta}} \dicPos{f}[6] \dicFlx{(‑var, ‑var)}[1] \textbf{1.} \dicDirectTranslationCS{taxi\-služba}} \dicIndirectTranslationCS{(firma)}  \textbf{2.} \dicDirectTranslationCS{stanoviště taxi}
\dicEntry[leigubíll] \dicTerm{leigu··bíl|l} \dicIPA{{l}{ei}{\textlengthmark}{\textbabygamma}{\textscy}{\textsubring{b}}{i}{\textsubring{d}}{\textsubring{l}}} \dicPos{m}[6] \dicFlx{(‑s, ‑ar)}[48] \dicDirectTranslationCS{taxi, taxík}
\dicEntry[leigubílstjóri] \dicTerm{leigu··bíl·stjór|i} \dicIPA{{l}{ei}{\textlengthmark}{\textbabygamma}{\textscy}{\textsubring{b}}{i}{l}{s}{\textsubring{d}}{j}{ou}{r}{\textsci}} \dicPos{m}[1] \dicFlx{(‑a, ‑ar)}[1] \dicDirectTranslationCS{taxikář(ka), řidič(ka) taxi}
\dicEntry[leiguflug] \dicTerm{leigu··flug} \dicIPA{{l}{ei}{\textlengthmark}{\textbabygamma}{\textscy}{f}{l}{\textscy}{x}} \dicPos{n}[2] \dicFlx{(‑s, ‑)}[5] \dicDirectTranslationCS{charterový let}
\dicEntry[leiguíbúð] \dicTerm{leigu··í·búð} \dicIPA{{l}{ei}{\textlengthmark}{\textbabygamma}{\textscy}{i}{\textsubring{b}}{u}{\texttheta}} \dicPos{f}[7] \dicFlx{(‑ar, ‑ir)}[1] \dicDirectTranslationCS{pronajatý byt}
\dicEntry[leiguliði] \dicTerm{leigu··lið|i} \dicIPA{{l}{ei}{\textlengthmark}{\textbabygamma}{\textscy}{l}{\textsci}{ð}{\textsci}} \dicPos{m}[1] \dicFlx{(‑a, ‑ar)}[1] \dicDirectTranslationCS{nájemce, nájemkyně}
\dicEntry[leigumiðlari] \dicTerm{leigu··miðl·ar|i} \dicIPA{{l}{ei}{\textlengthmark}{\textbabygamma}{\textscy}{m}{\textsci}{ð}{l}{a}{r}{\textsci}} \dicPos{m}[1] \dicFlx{(‑a, ‑ar)}[13] \dicDirectTranslationCS{zprostředkovatel(ka) pronájmu}
\dicEntry[leigumorð] \dicTerm{leigu··morð} \dicIPA{{l}{ei}{\textlengthmark}{\textbabygamma}{\textscy}{m}{\textopeno}{r}{\texttheta}} \dicPos{n}[2] \dicFlx{(‑s, ‑)}[5] \dicDirectTranslationCS{nájemná vražda}
\dicEntry[leigumorðingi] \dicTerm{leigu··morð·ing|i} \dicIPA{{l}{ei}{\textlengthmark}{\textbabygamma}{\textscy}{m}{\textopeno}{r}{ð}{i}{\textltailn}{\r{\textObardotlessj}}{\textsci}} \dicPos{m}[1] \dicFlx{(‑ja, ‑jar)}[14] \dicDirectTranslationCS{nájemný vrah, nájemná vražedkyně}
\dicEntry[leigusali] \dicTerm{leigu··sal|i} \dicIPA{{l}{ei}{\textlengthmark}{\textbabygamma}{\textscy}{s}{a}{l}{\textsci}} \dicPos{m}[1] \dicFlx{(‑a, ‑ar)}[8] \dicDirectTranslationCS{pronajímatel(ka)}
\dicEntry[leigusamningur] \dicTerm{leigu··samn·ing|ur} \dicIPA{{l}{ei}{\textlengthmark}{\textbabygamma}{\textscy}{s}{a}{m}{n}{i}{\ng}{\r{g}}{\textscy}{\textsubring{r}}} \dicPos{m}[6] \dicFlx{(‑s, ‑ar)}[8] \dicDirectTranslationCS{nájem\-ní smlouva}
\dicEntry[leigutaki] \dicTerm{leigu··tak|i} \dicIPA{{l}{ei}{\textlengthmark}{\textbabygamma}{\textscy}{t\smash{\textsuperscript{h}}}{a}{\r{\textObardotlessj}}{\textsci}} \dicPos{m}[1] \dicFlx{(‑a, ‑ar)}[8] \dicDirectTranslationCS{nájemce, nájemkyně}
\dicEntry[leik] \dicTerm{leik} \dicIPA{{l}{ei}{\textlengthmark}{\r{g}}} \dicPos{v} \dicFlx{ind praes sg 1 pers} \dicLink{leika}
\dicEntry[leika] \dicTerm{leika} \dicsymFrequent\  \dicIPA{{l}{ei}{\textlengthmark}{\r{g}}{a}} \dicPos{v}[6] \dicFlx{(leik, lék, lékum, léki, leikið)}[63] \dicFlx{acc} \textbf{1.} \dicDirectTranslationCS{(za)hrát} \dicIndirectTranslationCS{(bavit se pro zábavu)};  \dicPhraseIS{leika sér} \dicDirectTranslationCS{hrát si} \dicExampleIS{Börnin leika sér saman.} \dicExampleCS{Děti si spolu hrají.};  \dicPhraseIS{leika sér að e‑u} \dicDirectTranslationCS{hrát si s~(čím)} \dicExampleIS{leika sér að dúkkum} \dicExampleCS{hrát si s~panenkami}  \textbf{2.} \dicDirectTranslationCS{(za)hrát (si)} \dicIndirectTranslationCS{(provozovat jako sport)} \dicExampleIS{leika knattspyrnu} \dicExampleCS{hrát fotbal}  \textbf{3.} \dicDirectTranslationCS{(za)hrát} \dicIndirectTranslationCS{(provozovat hudbu)};  \dicPhraseIS{leika á hljóðfæri} \dicDirectTranslationCS{hrát na hudební nástroj} \dicExampleIS{Hún lék á fiðlu sína.} \dicExampleCS{Hrála na housle.}  \textbf{4.} \dicDirectTranslationCS{(za)hrát} \dicIndirectTranslationCS{(předvádět divadelní hru nebo osobu v~ní)} \dicExampleIS{leika Hamlet} \dicExampleCS{hrát Hamleta}  \textbf{5.} \dicPhraseIS{leika e‑n grátt\,/\addthin hart\,/\addthin illa} \dicDirectTranslationCS{zatočit s~(kým), nemazlit se s~(kým)};  \dicIdiom{leika}[af]{ \dicPhraseIS{leika af sér}} \dicDirectTranslationCS{udělat špatný tah (v~šachu ap.)};  \dicIdiom{leika}[á]{ \dicPhraseIS{leika á e‑n}} \dicDirectTranslationCS{vyzrát na (koho), přelstít (koho)}; { \dicPhraseIS{leika á als oddi, leika á alsoddi}} \dicLangCat{přen.} \dicDirectTranslationCS{mít výbornou náladu, být v~dobrém rozmaru}; { \dicPhraseIS{það leikur ekki á tveim tungum}} \dicFlx{impers} \dicLangCat{přen.} \dicDirectTranslationCS{o~tom není pochyb};  \dicIdiom{leika}[eftir]{ \dicPhraseIS{leika e‑ð eftir e‑m}} \dicDirectTranslationCS{zopakovat (co) po (kom)};  \dicIdiom{leika}[í]{ \dicPhraseIS{leika í e‑u}} \dicDirectTranslationCS{hrát v~(čem) (ve filmu ap.)};  \dicIdiom{leika}[undir]{ \dicPhraseIS{leika undir (e‑ð)}} \dicFieldCat{hud.} \dicDirectTranslationCS{doprovodit\,/\addthin doprovázet ((co))};  \dicIdiom{leika}[við]{ \dicPhraseIS{leika við e‑n}} \dicDirectTranslationCS{hrát si s~(kým)} \dicExampleIS{Lilja leikur við mömmu sína.} \dicExampleCS{Lilja si hraje s~maminkou.}; { \dicPhraseIS{leika við hvern sinn fingur}} \dicLangCat{přen.} \dicDirectTranslationCS{být dobře naladěn, mít dobrou náladu}
\dicEntry[leikandi] \dicTerm{leik··|andi\smash{\textsuperscript{1}}} \dicIPA{{l}{ei}{\textlengthmark}{\r{g}}{a}{n}{\textsubring{d}}{\textsci}} \dicPos{m}[2] \dicFlx{(‑anda, ‑endur)}[1] \textbf{1.} \dicDirectTranslationCS{hráč(ka) (ve fotbale ap.)}  \textbf{2.} \dicDirectTranslationCS{herec, herečka}
\dicEntry[leikandi] \dicTerm{leik··andi\smash{\textsuperscript{2}}} \dicIPA{{l}{ei}{\textlengthmark}{\r{g}}{a}{n}{\textsubring{d}}{\textsci}} \dicPos{adv} \dicDirectTranslationCS{hravě, s~lehkostí} \dicExampleIS{stökkva leikandi yfir skurðinn} \dicExampleCS{přeskočit hravě (přes) příkop}
\dicEntry[leikaraskapur] \dicTerm{leikara··skap|ur} \dicIPA{{l}{ei}{\textlengthmark}{\r{g}}{a}{r}{a}{s}{\r{g}}{a}{\textsubring{b}}{\textscy}{\textsubring{r}}} \dicPos{m}[10] \dicFlx{(‑ar)}[15] \textbf{1.} \dicSynonym{fíflalæti} \dicDirectTranslationCS{šaškoviny, komediantství}  \textbf{2.} \dicSynonym{látalæti} \dicDirectTranslationCS{předstírání, hraní}
\dicEntry[leikari] \dicTerm{leik··ar|i} \dicsymFrequent\  \dicIPA{{l}{ei}{\textlengthmark}{\r{g}}{a}{r}{\textsci}} \dicPos{m}[1] \dicFlx{(‑a, ‑ar)}[13] \dicDirectTranslationCS{herec, herečka} \dicExampleIS{Leikarinn hlaut verðlaun.} \dicExampleCS{Herec obdržel ocenění.}
\dicEntry[leikbrúða] \dicTerm{leik··brúð|a} \dicIPA{{l}{ei}{\textlengthmark}{\r{g}}{\textsubring{b}}{r}{u}{ð}{a}} \dicPos{f}[1] \dicFlx{(‑u, ‑ur)}[19] \textbf{1.} \dicSynonym{brúða} \dicDirectTranslationCS{loutka} \dicIndirectTranslationCS{(v~loutkovém divadle)}  \textbf{2.} \dicSynonym{handbendi} \dicDirectTranslationCS{loutka} \dicIndirectTranslationCS{(zmanipulovaný člověk)}
\dicEntry[leikdómur] \dicTerm{leik··dóm|ur} \dicIPA{{l}{ei}{\textlengthmark}{\r{g}}{\textsubring{d}}{ou}{m}{\textscy}{\textsubring{r}}} \dicPos{m}[6] \dicFlx{(‑s, ‑ar)}[10] \dicDirectTranslationCS{divadelní kritika}
\dicEntry[leikfang] \dicTerm{leik··|fang} \dicIPA{{l}{ei}{\textlengthmark}{\r{g}}{f}{au}{\ng}{\r{g}}} \dicPos{n}[2] \dicFlx{(‑fangs, ‑föng)}[8] \dicDirectTranslationCS{hračka (na hraní)}
\dicEntry[leikfangabúð] \dicTerm{leik·fanga··búð} \dicIPA{{l}{ei}{\textlengthmark}{\r{g}}{f}{au}{\ng}{\r{g}}{a}{\textsubring{b}}{u}{\texttheta}} \dicPos{f}[7] \dicFlx{(‑ar, ‑ir)}[1] \dicDirectTranslationCS{hračkářství}
\dicEntry[leikfélag] \dicTerm{leik··fé·|lag} \dicIPA{{l}{ei}{\textlengthmark}{\r{g}}{f}{j}{\textepsilon}{l}{a}{x}} \dicPos{n}[2] \dicFlx{(‑lags, ‑lög)}[8] \dicDirectTranslationCS{divadelní společnost}
\dicEntry[leikfélagi] \dicTerm{leik··fé·lag|i} \dicIPA{{l}{ei}{\textlengthmark}{\r{g}}{f}{j}{\textepsilon}{l}{a}{i}{j}{\textsci}} \dicPos{m}[1] \dicFlx{(‑a, ‑ar)}[8] \dicDirectTranslationCS{kamarád(ka) z~dětství}
\dicEntry[leikfimi] \dicTerm{leik··fim|i} \dicIPA{{l}{ei}{\textlengthmark}{\r{g}}{f}{\textsci}{m}{\textsci}} \dicPos{f}[3] \dicFlx{(‑i)}[3] \dicDirectTranslationCS{tělocvik, (tělesné) cvičení}
\dicEntry[leikfimisalur] \dicTerm{leik·fimi··sal|ur} \dicIPA{{l}{ei}{\textlengthmark}{\r{g}}{f}{\textsci}{m}{\textsci}{s}{a}{l}{\textscy}{\textsubring{r}}} \dicPos{m}[10] \dicFlx{(‑ar, ‑ir)}[14] \dicDirectTranslationCS{tělocvična}
\dicEntry[leikfimiskennari] \dicTerm{leik·fimis··kenn·ar|i} \dicIPA{{l}\-{ei}\-{\textlengthmark}\-{\r{g}}\-{f}\-{\textsci}\-{m}\-{\textsci}\-{s}\-{c\smash{\textsuperscript{h}}}\-{\textepsilon}\-{n}\-{a}\-{r}\-{\textsci}\-} \dicPos{m}[1] \dicFlx{(‑a, ‑ar)}[13] \dicDirectTranslationCS{učitel(ka) tělesné výchovy, tělocvikář(ka)}
\dicEntry[leikgagnrýnandi] \dicTerm{leik··gagn·rýn·|andi} \dicIPA{{l}{ei}{\r{g}}{\textlengthmark}{a}{\r{g}}{n}{r}{i}{n}{a}{n}{\textsubring{d}}{\textsci}} \dicPos{m}[2] \dicFlx{(‑anda, ‑endur)}[1] \dicDirectTranslationCS{divadelní kritik\,/\addthin kritička}
\dicEntry[leikgerð] \dicTerm{leik··gerð} \dicIPA{{l}{ei}{\textlengthmark}{\r{g}}{\r{\textObardotlessj}}{\textepsilon}{r}{\texttheta}} \dicPos{f}[7] \dicFlx{(‑ar, ‑ir)}[1] \dicFieldCat{lit.} \dicDirectTranslationCS{dramatizace} \dicExampleIS{leikgerð Sölku Völku} \dicExampleCS{dramatizace Salky Valky}
\dicEntry[leikgrind] \dicTerm{leik··grind} \dicIPA{{l}{ei}{\r{g}}{\textlengthmark}{r}{\textsci}{n}{\textsubring{d}}} \dicPos{f}[8] \dicFlx{(‑ar, ‑ur)}[1] \dicDirectTranslationCS{(dětská) ohrádka}
\dicEntry[leikhús] \dicTerm{leik··hús} \dicIPA{{l}{ei}{\textlengthmark}{\r{g}}{h}{u}{s}} \dicPos{n}[2] \dicFlx{(‑s, ‑)}[5] \dicDirectTranslationCS{divadlo} \dicExampleIS{sýning í leikhúsinu} \dicExampleCS{představení v~divadle}
\dicEntry[leikhússtjóri] \dicTerm{leik·hús··stjór|i} \dicIPA{{l}{ei}{\textlengthmark}{\r{g}}{h}{u}{s}{\textsubring{d}}{j}{ou}{r}{\textsci}} \dicPos{m}[1] \dicFlx{(‑a, ‑ar)}[1] \dicDirectTranslationCS{divadelní ředitel(ka), ředitel(ka) divadla}
\dicEntry[leikið] \dicTerm{leikið} \dicIPA{{l}{ei}{\textlengthmark}{\r{\textObardotlessj}}{\textsci}{\texttheta}} \dicPos{v} \dicFlx{supin} \dicLink{leika}
\dicEntry[leikinn] \dicTerm{leikinn} \dicIPA{{l}{ei}{\textlengthmark}{\r{\textObardotlessj}}{\textsci}{\textsubring{n}}} \dicPos{adj}[6]\dicFlx{}[-2] \dicSynonym{snjall} \dicDirectTranslationCS{obratný, šikovný, dovedný} \dicExampleIS{vera leikinn í e‑u} \dicExampleCS{být šikovný v~(čem)};  \dicPhraseIS{vera grátt\,/\addthin illa leikinn} \dicDirectTranslationCS{nést známky špatného zacházení}
\dicEntry[leikjanámskeið] \dicTerm{leikja··nám·skeið} \dicIPA{{l}{ei}{\textlengthmark}{\r{\textObardotlessj}}{a}{n}{au}{m}{s}{\r{\textObardotlessj}}{ei}{\texttheta}} \dicPos{n}[2] \dicFlx{(‑s, ‑)}[5] \dicDirectTranslationCS{tábor, (dětský) prázdninový kurz}
\dicEntry[leikjatölva] \dicTerm{leikja··tölv|a} \dicIPA{{l}{ei}{\textlengthmark}{\r{\textObardotlessj}}{a}{t\smash{\textsuperscript{h}}}{\oe}{l}{v}{a}} \dicPos{f}[1] \dicFlx{(‑u, ‑ur)}[7] \dicFieldCat{poč.} \dicDirectTranslationCS{(hrací) konzole}
\dicEntry[leikkerahjól] \dicTerm{leik·kera··hjól} \dicIPA{{l}{ei}{\textlengthmark}{\r{g}}{c\smash{\textsuperscript{h}}}{\textepsilon}{r}{a}{\c{c}}{ou}{\textsubring{l}}} \dicPos{n}[2] \dicFlx{(‑s, ‑)}[5] \dicDirectTranslationCS{hrnčířské kolo}
\dicEntry[leikkerasmiður] \dicTerm{leik·kera··smið|ur} \dicIPA{{l}{ei}{\textlengthmark}{\r{g}}{c\smash{\textsuperscript{h}}}{\textepsilon}{r}{a}{s}{m}{\textsci}{ð}{\textscy}{\textsubring{r}}} \dicPos{m}[9] \dicFlx{(‑s, ‑ir)}[11] \dicDirectTranslationCS{hrnčíř(ka), keramik, keramička}
\dicEntry[leikkona] \dicTerm{leik··kon|a} \dicIPA{{l}{ei}{\textlengthmark}{\r{g}}{k\smash{\textsuperscript{h}}}{\textopeno}{n}{a}} \dicPos{f}[1] \dicFlx{(‑u, ‑ur)}[27] \dicDirectTranslationCS{herečka}
\dicEntry[leiklist] \dicTerm{leik··list} \dicIPA{{l}{ei}{\textlengthmark}{\r{g}}{l}{\textsci}{s}{\textsubring{d}}} \dicPos{f}[7] \dicFlx{(‑ar)}[3] \dicDirectTranslationCS{divadlo, divadelní umění}
\dicEntry[leiklistarskóli] \dicTerm{leik·listar··skól|i} \dicIPA{{l}{ei}{\textlengthmark}{\r{g}}{l}{\textsci}{s}{\textsubring{d}}{a}{\textsubring{r}}{s}{\r{g}}{ou}{l}{\textsci}} \dicPos{m}[1] \dicFlx{(‑a, ‑ar)}[1] \dicFieldCat{škol.} \dicDirectTranslationCS{divadelní škola}
\dicEntry[leikmaður] \dicTerm{leik··|maður} \dicIPA{{l}{ei}{\textlengthmark}{\r{g}}{m}{a}{ð}{\textscy}{\textsubring{r}}} \dicPos{m}[13] \dicFlx{(‑manns, ‑menn)}[2] \textbf{1.} \dicDirectTranslationCS{laik, laička, neodborník, neodbornice}  \textbf{2.} \dicDirectTranslationCS{hráč(ka)}  \textbf{3.} \dicSynonym{leikari} \dicDirectTranslationCS{herec, herečka}
\dicEntry[leikmunur] \dicTerm{leik··mun|ur} \dicIPA{{l}{ei}{\textlengthmark}{\r{g}}{m}{\textscy}{n}{\textscy}{\textsubring{r}}} \dicPos{m}[10] \dicFlx{(‑ar, ‑ir)}[11] \dicDirectTranslationCS{rekvizita}
\dicEntry[leikmynd] \dicTerm{leik··mynd} \dicIPA{{l}{ei}{\textlengthmark}{\r{g}}{m}{\textsci}{n}{\textsubring{d}}} \dicPos{f}[7] \dicFlx{(‑ar, ‑ir)}[1] \dicDirectTranslationCS{scénografie, scéna}
\dicEntry[leikmyndahönnuður] \dicTerm{leik·mynda··hönn·uð|ur} \dicIPA{{l}\-{ei}\-{\textlengthmark}\-{\r{g}}\-{m}\-{\textsci}\-{n}\-{\textsubring{d}}\-{a}\-{h}\-{\oe}\-{n}\-{\textscy}\-{ð}\-{\textscy}\-{\textsubring{r}}\-} \dicPos{m}[10] \dicFlx{(‑ar, ‑ir)}[4] \dicDirectTranslationCS{jevištní výtvarník\,/\addthin výtvarnice, scénograf(ka)}
\dicEntry[leikni] \dicTerm{leikn|i} \dicIPA{{l}{ei}{h}{\r{g}}{n}{\textsci}} \dicPos{f}[3] \dicFlx{(‑i)}[3] \dicDirectTranslationCS{šikovnost, obratnost, zručnost, dovednost} \dicExampleIS{leikni í íþróttum} \dicExampleCS{dovednost ve sportu}
\dicEntry[leikregla] \dicTerm{leik··regl|a} \dicIPA{{l}{ei}{\textlengthmark}{\r{g}}{r}{\textepsilon}{\r{g}}{l}{a}} \dicPos{f}[1] \dicFlx{(‑u, ‑ur)}[19] \dicDirectTranslationCS{pravidlo hry}
\dicEntry[leikrit] \dicTerm{leik··rit} \dicIPA{{l}{ei}{\textlengthmark}{\r{g}}{r}{\textsci}{\textsubring{d}}} \dicPos{n}[2] \dicFlx{(‑s, ‑)}[5] \dicDirectTranslationCS{divadelní hra}
\dicEntry[leikritahöfundur] \dicTerm{leik·rita··höf·und|ur} \dicIPA{{l}\-{ei}\-{\textlengthmark}\-{\r{g}}\-{r}\-{\textsci}\-{\textsubring{d}}\-{a}\-{h}\-{\oe}\-{v}\-{\textscy}\-{n}\-{\textsubring{d}}\-{\textscy}\-{\textsubring{r}}\-} \dicPos{m}[6] \dicFlx{(‑ar, ‑ar)}[56] \dicDirectTranslationCS{(divadelní) dramatik, dramatička} \dicExampleIS{frægur tékkneskur leikritahöfundur} \dicExampleCS{slavný český dramatik}
\dicEntry[leikrænn] \dicTerm{leik··rænn} \dicIPA{{l}{ei}{\textlengthmark}{\r{g}}{r}{a}{i}{\textsubring{d}}{\textsubring{n}}} \dicPos{adj}[7]\dicFlx{}[-1] \dicDirectTranslationCS{divadelní}
\dicEntry[leikskáld] \dicTerm{leik··skáld} \dicIPA{{l}{ei}{\textlengthmark}{\r{g}}{s}{\r{g}}{au}{l}{\textsubring{d}}} \dicPos{n}[2] \dicFlx{(‑s, ‑)}[5] \dicDirectTranslationCS{dramatik, dramatička, autor(ka) divadelních her}
\dicEntry[leikskóladeild] \dicTerm{leik·skóla··deild} \dicIPA{{l}{ei}{\textlengthmark}{\r{g}}{s}{\r{g}}{ou}{l}{a}{\textsubring{d}}{ei}{l}{\textsubring{d}}} \dicPos{f}[7] \dicFlx{(‑ar, ‑ir)}[1] \dicDirectTranslationCS{oddělení mateřské školky} \dicIndirectTranslationCS{(v~rámci školy ap.)}
\dicEntry[leikskóladvöl] \dicTerm{leik·skóla··|dvöl} \dicIPA{{l}{ei}{\textlengthmark}{\r{g}}{s}{\r{g}}{ou}{l}{a}{\textsubring{d}}{v}{\oe}{\textsubring{l}}} \dicPos{f}[7] \dicFlx{(‑dvalar, ‑dvalir)}[16] \dicDirectTranslationCS{pobyt ve školce}
\dicEntry[leikskólakennari] \dicTerm{leik·skóla··kenn·ar|i} \dicIPA{{l}\-{ei}\-{\textlengthmark}\-{\r{g}}\-{s}\-{\r{g}}\-{ou}\-{l}\-{a}\-{c\smash{\textsuperscript{h}}}\-{\textepsilon}\-{n}\-{a}\-{r}\-{\textsci}\-} \dicPos{m}[1] \dicFlx{(‑a, ‑ar)}[13] \dicDirectTranslationCS{učitel(ka) v~mateřské školce}
\dicEntry[leikskóli] \dicTerm{leik··skól|i} \dicIPA{{l}{ei}{\textlengthmark}{\r{g}}{s}{\r{g}}{ou}{l}{\textsci}} \dicPos{m}[1] \dicFlx{(‑a, ‑ar)}[1] \dicDirectTranslationCS{(mateřská) školka}
\dicEntry[leikskrá] \dicTerm{leik··skrá} \dicIPA{{l}{ei}{\textlengthmark}{\r{g}}{s}{\r{g}}{r}{au}} \dicPos{f}[4] \dicFlx{(‑r\,/\addthin ‑ar, ‑r)}[21] \dicDirectTranslationCS{divadelní program}
\dicEntry[leikslok] \dicTerm{leiks··lok} \dicIPA{{l}{ei}{\textlengthmark}{\r{g}}{s}{l}{\textopeno}{\r{g}}} \dicPos{n}[2] \dicFlx{pl}[1] \textbf{1.} \dicSynonym{niðurstaða} \dicDirectTranslationCS{závěr, rozhodnutí}  \textbf{2.} \dicDirectTranslationCS{výsledek (zápasu, utkání ap.)}
\dicEntry[leiksoppur] \dicTerm{leik··sopp|ur} \dicIPA{{l}{ei}{\textlengthmark}{\r{g}}{s}{\textopeno}{h}{\textsubring{b}}{\textscy}{\textsubring{r}}} \dicPos{m}[6] \dicFlx{(‑s, ‑ar)}[22] \dicDirectTranslationCS{loutka} \dicIndirectTranslationCS{(člověk vnitřně prázdný)}
\dicEntry[leikstjóri] \dicTerm{leik··stjór|i} \dicIPA{{l}{ei}{\textlengthmark}{\r{g}}{s}{\textsubring{d}}{j}{ou}{r}{\textsci}} \dicPos{m}[1] \dicFlx{(‑a, ‑ar)}[1] \textbf{1.} \dicDirectTranslationCS{režisér(ka)}  \textbf{2.} \dicFieldCat{sport.} \dicDirectTranslationCS{rozhodčí}
\dicEntry[leikstjórn] \dicTerm{leik··stjórn} \dicIPA{{l}{ei}{\textlengthmark}{\r{g}}{s}{\textsubring{d}}{j}{ou}{r}{\textsubring{d}}{\textsubring{n}}} \dicPos{f}[7] \dicFlx{(‑ar)}[3] \dicDirectTranslationCS{režie, režírování}
\dicEntry[leikstýra] \dicTerm{leik··stýr|a} \dicIPA{{l}{ei}{\textlengthmark}{\r{g}}{s}{\textsubring{d}}{i}{r}{a}} \dicPos{v}[2] \dicFlx{(‑ði, ‑t)}[105] \dicFlx{dat} \dicDirectTranslationCS{režírovat}
\dicEntry[leiksvið] \dicTerm{leik··svið} \dicIPA{{l}{ei}{\textlengthmark}{\r{g}}{s}{v}{\textsci}{\texttheta}} \dicPos{n}[2] \dicFlx{(‑s, ‑)}[5] \dicDirectTranslationCS{jeviště, scéna}
\dicEntry[leiksýning] \dicTerm{leik··sýn·ing} \dicIPA{{l}{ei}{\textlengthmark}{\r{g}}{s}{i}{n}{i}{\ng}{\r{g}}} \dicPos{f}[4] \dicFlx{(‑ar, ‑ar)}[5] \dicDirectTranslationCS{(divadelní) představení}
\dicEntry[leiktíð] \dicTerm{leik··tíð} \dicIPA{{l}{ei}{\textlengthmark}{\r{g}}{t\smash{\textsuperscript{h}}}{i}{\texttheta}} \dicPos{f}[7] \dicFlx{(‑ar, ‑ir)}[1] \dicFieldCat{sport.} \dicDirectTranslationCS{sezóna (fotbalová ap.)}
\dicEntry[leikur] \dicTerm{leik|ur} \dicsymFrequent\  \dicIPA{{l}{ei}{\textlengthmark}{\r{g}}{\textscy}{\textsubring{r}}} \dicPos{m}[6] \dicFlx{(‑s, ‑ir\,/\addthin ‑ar)}[67] \textbf{1.} \dicFlx{pl (-ir)} \dicSynonym{barnaleikur} \dicDirectTranslationCS{hra, hraní} \dicExampleIS{leikur barna} \dicExampleCS{hra dětí};  \dicPhraseIS{bregða á leik} \dicDirectTranslationCS{hrát si, skotačit}  \textbf{2.} \dicFlx{pl (-ar)} \dicSynonym{viðureign} \dicDirectTranslationCS{boj, střetnutí, utkání}  \textbf{3.} \dicFlx{pl (-ar)} \dicSynonym*{íþróttakeppni} \dicDirectTranslationCS{zápas, utkání, hra} \dicExampleIS{Ólympíuleikar} \dicExampleCS{olympijské hry};  \dicPhraseIS{fara á leikinn} \dicDirectTranslationCS{jít na zápas}  \textbf{4.} \dicSynonym{leikrit} \dicDirectTranslationCS{hra (divadelní ap.), představení}  \textbf{5.} \dicDirectTranslationCS{tah} \dicIndirectTranslationCS{(posunutí šachové figury ap.)};  \dicPhraseIS{eiga leik} \dicDirectTranslationCS{být na tahu};  \dicIdiom{leikur}{ \dicPhraseIS{á nýjan leik}} \dicFlx{adv} \dicDirectTranslationCS{znovu, opět}; { \dicPhraseIS{e‑að er leikur einn}} \dicLangCat{přen.} \dicDirectTranslationCS{(co) je hračka}; { \dicPhraseIS{með leik}} \dicFlx{adv} \dicDirectTranslationCS{snadno, lehce}; { \dicPhraseIS{vera úr leik}} \dicLangCat{přen.} \dicDirectTranslationCS{být mimo hru}; { \dicPhraseIS{við illan leik}} \dicFlx{adv} \dicDirectTranslationCS{horko těžko}
\dicEntry[leikvangur] \dicTerm{leik··vang|ur} \dicIPA{{l}{ei}{\textlengthmark}{\r{g}}{v}{au}{\ng}{\r{g}}{\textscy}{\textsubring{r}}} \dicPos{m}[6] \dicFlx{(‑s, ‑ar)}[25] \dicSynonym{íþróttavöllur} \dicDirectTranslationCS{stadión}
\dicEntry[leikvöllur] \dicTerm{leik··|völlur} \dicIPA{{l}{ei}{\textlengthmark}{\r{g}}{v}{\oe}{\textsubring{d}}{l}{\textscy}{\textsubring{r}}} \dicPos{m}[11] \dicFlx{(‑vallar, ‑vellir)}[5] \textbf{1.} \dicFieldCat{sport.} \dicSynonym{í\-þrótta\-völlur} \dicDirectTranslationCS{(sportovní) hřiště}  \textbf{2.} \dicSynonym*{barnaleikvöllur} \dicDirectTranslationCS{dětské hřiště}
\dicEntry[leir] \dicTerm{leir} \dicsymFrequent\  \dicIPA{{l}{ei}{\textlengthmark}{\textsubring{r}}} \dicPos{m}[4] \dicFlx{(‑s)}[18] \dicDirectTranslationCS{jíl, hlína} \dicExampleIS{hvít englabörn úr leir} \dicExampleCS{malí andělíčci z~hlíny}
\dicEntry[leira] \dicTerm{leir|a\smash{\textsuperscript{1}}} \dicIPA{{l}{ei}{\textlengthmark}{r}{a}} \dicPos{f}[1] \dicFlx{(‑u, ‑ur)}[19] \dicDirectTranslationCS{jílovitá půda}
\dicEntry[Leira] \dicTerm{Leir|a\smash{\textsuperscript{2}}} \dicIPA{{l}{ei}{\textlengthmark}{r}{a}} \dicPos{f}[1] \dicFlx{(‑u)}[6] \dicFieldCat{geog.} \dicDirectTranslationCS{Loira} \dicIndirectTranslationCS{(řeka ve Francii)}
\dicEntry[leirburður] \dicTerm{leir··burð|ur} \dicIPA{{l}{ei}{r}{\textsubring{b}}{\textscy}{r}{ð}{\textscy}{\textsubring{r}}} \dicPos{m}[10] \dicFlx{(‑ar)}[7] \dicDirectTranslationCS{špatná poezie}
\dicEntry[leirker] \dicTerm{leir··ker} \dicIPA{{l}{ei}{\textsubring{r}}{c\smash{\textsuperscript{h}}}{\textepsilon}{\textsubring{r}}} \dicPos{n}[2] \dicFlx{(‑s, ‑)}[6] \dicDirectTranslationCS{keramika} \dicIndirectTranslationCS{(především z~archeologických nálezů)}
\dicEntry[leirlist] \dicTerm{leir··list} \dicIPA{{l}{ei}{r}{l}{\textsci}{s}{\textsubring{d}}} \dicPos{f}[7] \dicFlx{(‑ar)}[3] \dicDirectTranslationCS{keramika, keramické umění}
\dicEntry[leirmunur] \dicTerm{leir··mun|ur} \dicIPA{{l}{ei}{r}{m}{\textscy}{n}{\textscy}{\textsubring{r}}} \dicPos{m}[10] \dicFlx{(‑ar, ‑ir)}[11] \dicDirectTranslationCS{keramika, keramický výrobek}
\dicEntry[leit] \dicTerm{leit\smash{\textsuperscript{1}}} \dicsymFrequent\  \dicIPA{{l}{ei}{\textlengthmark}{\textsubring{d}}} \dicPos{f}[7] \dicFlx{(‑ar, ‑ir)}[1] \textbf{1.} \dicDirectTranslationCS{hledání, pátrání, vyhledávání} \dicExampleIS{Leitin er mjög umfangsmikil.} \dicExampleCS{Pátrání je velmi rozsáhlé.}  \textbf{2.} \dicPhraseIS{leitir} \dicFlx{pl} \dicSynonym{smölun} \dicDirectTranslationCS{shánění ovcí};  \dicPhraseIS{fara í leitir} \dicDirectTranslationCS{jet shánět ovce}
\dicEntry[leit] \dicTerm{leit\smash{\textsuperscript{2}}} \dicIPA{{l}{ei}{\textlengthmark}{\textsubring{d}}} \dicPos{v} \dicFlx{ind pf sg 1 pers} \dicLink{líta}
\dicEntry[leita] \dicTerm{leit|a} \dicsymFrequent\  \dicIPA{{l}{ei}{\textlengthmark}{\textsubring{d}}{a}} \dicPos{v}[1] \dicFlx{(‑aði)}[1] \dicFlx{gen} \textbf{1.} \dicDirectTranslationCS{(vy)hledat, (vy)pátrat} \dicIndirectTranslationCS{(snažit se nalézt)};  \dicPhraseIS{leita e‑s} \dicDirectTranslationCS{hledat (co)} \dicExampleIS{leita bátsins} \dicExampleCS{hledat člun};  \dicPhraseIS{leita að e‑u\,/\addthin e‑m} \dicDirectTranslationCS{hledat (co\,/\addthin koho)} \dicExampleIS{leita að lyklunum} \dicExampleCS{hledat klíče}  \textbf{2.} \dicDirectTranslationCS{vyhledat, navštívit (lékaře ap.)};  \dicPhraseIS{leita læknis} \dicDirectTranslationCS{vyhledat doktora};  \dicIdiom{leita}[á]{ \dicPhraseIS{leita á e‑m}} \dicDirectTranslationCS{prohledat (koho), prošacovat (koho)};  \dicIdiom{leita}[eftir]{ \dicPhraseIS{leita eftir e‑u við e‑n}} \dicDirectTranslationCS{požádat (koho) o~(co)};  \dicIdiom{leita}[fyrir]{ \dicPhraseIS{leita fyrir sér um e‑ð}} \dicDirectTranslationCS{shánět se po (čem), shánět (co) (práci ap.)};  \dicIdiom{leita}[til]{ \dicPhraseIS{leita til e‑rs}} \dicDirectTranslationCS{obrátit se na (koho)} \dicExampleIS{leita til hennar með vandamál sín} \dicExampleCS{obrátit se na ni se svými problémy};  \dicIdiom{leita}[uppi]{ \dicPhraseIS{leita e‑n uppi}} \dicDirectTranslationCS{vyhledat (koho), vypátrat (koho)};  \dicIdiom{leitast}[við]{ \dicPhraseIS{leitast við}} \dicFlx{refl} \dicDirectTranslationCS{pokoušet se, snažit se}
\dicEntry[leitandi] \dicTerm{leit··andi} \dicIPA{{l}{ei}{\textlengthmark}{\textsubring{d}}{a}{n}{\textsubring{d}}{\textsci}} \dicPos{adj}[13] \dicFlx{indecl}[1] \dicDirectTranslationCS{hledající, pátrající}
\dicEntry[leitarflokkur] \dicTerm{leitar··flokk|ur} \dicIPA{{l}{ei}{\textlengthmark}{\textsubring{d}}{a}{\textsubring{r}}{f}{l}{\textopeno}{h}{\r{g}}{\textscy}{\textsubring{r}}} \dicPos{m}[6] \dicFlx{(‑s, ‑ar)}[8] \dicDirectTranslationCS{pátrací skupina\,/\addthin četa}
\dicEntry[leitarvél] \dicTerm{leitar··vél} \dicIPA{{l}{ei}{\textlengthmark}{\textsubring{d}}{a}{r}{v}{j}{\textepsilon}{\textsubring{l}}} \dicPos{f}[4] \dicFlx{(‑ar, ‑ar)}[1] \dicFieldCat{poč.} \dicDirectTranslationCS{(internetový) vyhledávač}
\dicEntry[leiti] \dicTerm{leiti} \dicIPA{{l}{ei}{\textlengthmark}{\textsubring{d}}{\textsci}} \dicPos{n}[2] \dicFlx{(‑s, ‑)}[14] \dicSynonym{hæð} \dicDirectTranslationCS{vrch, vyvýšenina, kopec};  \dicPhraseIS{á næsta leiti} \dicFlx{adv} \dicDirectTranslationCS{blízko, nedaleko (časově), za dveřmi\,/\addthin rohem} \dicExampleIS{Hátíðin er á næsta leiti.} \dicExampleCS{Svátky jsou za dveřmi.}
\dicEntry[leitun] \dicTerm{leit|un} \dicIPA{{l}{ei}{\textlengthmark}{\textsubring{d}}{\textscy}{\textsubring{n}}} \dicPos{f}[7] \dicFlx{(‑unar)}[9] \dicDirectTranslationCS{hledání, vyhledávání, pátrání};  \dicPhraseIS{leitun að e‑u} \dicDirectTranslationCS{hledání (čeho)};  \dicPhraseIS{það er leitun á\,/\addthin að e‑u} \dicDirectTranslationCS{je vzácné\,/\addthin těžké nalézt (co)}
\dicEntry[lek] \dicTerm{lek} \dicIPA{{l}{\textepsilon}{\textlengthmark}{\r{g}}} \dicPos{v} \dicFlx{ind praes sg 1 pers} \dicLink{leka}
\dicEntry[leka] \dicTerm{leka} \dicsymFrequent\  \dicIPA{{l}{\textepsilon}{\textlengthmark}{\r{g}}{a}} \dicPos{v}[6] \dicFlx{(lek, lak, lákum, læki, lekið)}[17] \textbf{1.} \dicSynonym*{falla í dropum} \dicDirectTranslationCS{téct, stékat, kapat} \dicExampleIS{Vatnið lekur niður á gólfið.} \dicExampleCS{Voda stéká dolů na podlahu.}  \textbf{2.} \dicSynonym*{vera lekur} \dicDirectTranslationCS{téct, prosakovat} \dicIndirectTranslationCS{(propouštět kapalinu)}  \textbf{3.} \dicLangCat{přen.} \dicDirectTranslationCS{uniknout (na veřejnost), prosáknout (na veřejnost) (zpráva ap.)}
\dicEntry[lekandi] \dicTerm{lek··|andi} \dicIPA{{l}{\textepsilon}{\textlengthmark}{\r{g}}{a}{n}{\textsubring{d}}{\textsci}} \dicPos{m}[2] \dicFlx{(‑anda)}[2] \dicFieldCat{med.} \dicDirectTranslationCS{kapavka}
\dicEntry[leki] \dicTerm{lek|i} \dicIPA{{l}{\textepsilon}{\textlengthmark}{\r{\textObardotlessj}}{\textsci}} \dicPos{m}[1] \dicFlx{(‑a, ‑ar)}[1] \textbf{1.} \dicDirectTranslationCS{kapání, odkapávání} \dicExampleIS{leki úr þaki} \dicExampleCS{kapání ze střechy}  \textbf{2.} \dicDirectTranslationCS{prosakování, zatékání} \dicExampleIS{Það kom leki að bátnum.} \dicExampleCS{Loď začala zatékat.}  \textbf{3.} \dicDirectTranslationCS{únik, unikání (informací ap.)}
\dicEntry[lekið] \dicTerm{lekið} \dicIPA{{l}{\textepsilon}{\textlengthmark}{\r{\textObardotlessj}}{\textsci}{\texttheta}} \dicPos{v} \dicFlx{supin} \dicLink{leka}
\dicEntry[lektor] \dicTerm{lektor} \dicIPA{{l}{\textepsilon}{x}{\textsubring{d}}{\textopeno}{\textsubring{r}}} \dicPos{m}[4] \dicFlx{(‑s, ‑ar)}[14] \dicFieldCat{škol.} \dicDirectTranslationCS{docent(ka)} \dicIndirectTranslationCS{(nejvyšší stupeň v~islandském školním systému, pouze na vysokých školách)}
\dicEntry[lekur] \dicTerm{lekur} \dicIPA{{l}{\textepsilon}{\textlengthmark}{\r{g}}{\textscy}{\textsubring{r}}} \dicPos{adj}[1]\dicFlx{}[-1] \dicDirectTranslationCS{zatékající, prosakující} \dicExampleIS{lekur bátur} \dicExampleCS{zatékající člun}
\dicEntry[lem] \dicTerm{lem} \dicIPA{{l}{\textepsilon}{\textlengthmark}{\textsubring{m}}} \dicPos{v} \dicFlx{ind praes sg 1 pers} \dicLink{lemja}
\dicEntry[lemdi] \dicTerm{lemdi} \dicIPA{{l}{\textepsilon}{m}{\textsubring{d}}{\textsci}} \dicPos{v} \dicFlx{con pf sg 1 pers} \dicLink{lemja}
\dicEntry[lemja] \dicTerm{lemja} \dicsymFrequent\  \dicIPA{{l}{\textepsilon}{m}{j}{a}} \dicPos{v}[4] \dicFlx{(lem, lamdi, lömdum, lemdi, lamið)}[3] \dicFlx{acc} \dicSynonym{slá\smash{\textsuperscript{3}}} \dicDirectTranslationCS{uhodit, udeřit, tlouct, mlátit} \dicExampleIS{lemja e‑n með barefli} \dicExampleCS{uhodit (koho) pálkou}
\dicEntry[lemstra] \dicTerm{lemstr|a} \dicIPA{{l}{\textepsilon}{m}{s}{\textsubring{d}}{r}{a}} \dicPos{v}[1] \dicFlx{(‑aði)}[1] \dicFlx{acc} \dicSynonym{limlesta} \dicDirectTranslationCS{zmrzačit}
\dicEntry[lend] \dicTerm{lend} \dicIPA{{l}{\textepsilon}{n}{\textsubring{d}}} \dicPos{f}[4] \dicFlx{(‑ar, ‑ar)}[1] \textbf{1.} \dicDirectTranslationCS{zadek (koně ap.)}  \textbf{2.} \dicPhraseIS{lendar} \dicFlx{pl} \dicSynonym{mjaðmir} \dicDirectTranslationCS{boky, bedra}
\dicEntry[lenda] \dicTerm{len|da} \dicsymFrequent\  \dicIPA{{l}{\textepsilon}{n}{\textsubring{d}}{a}} \dicPos{v}[2] \dicFlx{(‑ti, ‑t)}[41] \dicFlx{dat} \textbf{1.} \dicSynonym*{koma að landi} \dicDirectTranslationCS{přistát, přistávat, dosednout, dosedat} \dicExampleIS{lenda flugvélinni} \dicExampleCS{přistát s~letadlem}  \textbf{2.} \dicDirectTranslationCS{přistát, dopadnout (na zadek ap.)};  \dicIdiom{lenda}[í]{ \dicPhraseIS{lenda í e‑u}} \dicSynonym*{rata í e‑ð} \dicDirectTranslationCS{spadnout do (čeho), dostat se do (čeho)} \dicExampleIS{lenda í vandræðum} \dicExampleCS{dostat se do potíží};  \dicIdiom{lenda}[saman]{ \dicPhraseIS{e‑jum lendir saman}} \dicFlx{impers} \dicLangCat{přen.} \dicDirectTranslationCS{(kdo) se dostává do křížku} \dicExampleIS{Þeim lenti saman.} \dicExampleCS{Dostali se do křížku.}
\dicEntry[lending] \dicTerm{lend··ing} \dicIPA{{l}{\textepsilon}{n}{\textsubring{d}}{i}{\ng}{\r{g}}} \dicPos{f}[4] \dicFlx{(‑ar, ‑ar)}[5] \dicDirectTranslationCS{přistání, přistávání, dosednutí} \dicExampleIS{mjúk lending} \dicExampleCS{měkké přistání}
\dicEntry[lengd] \dicTerm{lengd} \dicsymFrequent\  \dicIPA{{l}{\textepsilon}{\ng}{\textsubring{d}}} \dicPos{f}[7] \dicFlx{(‑ar, ‑ir)}[1] \textbf{1.} \dicDirectTranslationCS{délka} \dicExampleIS{í fullri lengd} \dicExampleCS{v~plné délce}  \textbf{2.} \dicFieldCat{geog.} \dicDirectTranslationCS{zeměpisná délka};  \dicIdiom{lengd}{ \dicPhraseIS{til lengdar}} \dicFlx{adv} \dicDirectTranslationCS{dlouhodobě, v~dlouhém časovém období}
\dicEntry[lengdarbaugur] \dicTerm{lengdar··baug|ur} \dicIPA{{l}{\textepsilon}{\ng}{\textsubring{d}}{a}{r}{\textsubring{b}}{\oe i}{\textbabygamma}{\textscy}{\textsubring{r}}} \dicPos{m}[6] \dicFlx{(‑s, ‑ar)}[24] \dicFieldCat{geog.} \dicDirectTranslationCS{poledník}
\dicEntry[lengi] \dicTerm{lengi} \dicsymFrequent\  \dicIPA{{l}{ei}{\textltailn}{\r{\textObardotlessj}}{\textsci}} \dicPos{adv} \dicFlx{(comp lengur, sup lengst)} \dicSynonym*{langan tíma} \dicDirectTranslationCS{dlouho, dlouze} \dicExampleIS{lengi vetrar} \dicExampleCS{po většinu zimy};  \dicPhraseIS{lengi vel} \dicFlx{adv} \dicDirectTranslationCS{dlouho, dlouhou dobu}
\dicEntry[lenging] \dicTerm{leng··ing} \dicIPA{{l}{ei}{\textltailn}{\r{\textObardotlessj}}{i}{\ng}{\r{g}}} \dicPos{f}[4] \dicFlx{(‑ar, ‑ar)}[5] \textbf{1.} \dicSynonym*{það að lengja e‑ð} \dicDirectTranslationCS{prodloužení, prodlužování} \dicExampleIS{lenging á jólafríinu} \dicExampleCS{prodloužení vánočních prázdnin}  \textbf{2.} \dicSynonym{viðbót} \dicDirectTranslationCS{vylepšení, vylepšování} \dicExampleIS{lenging á bátnum} \dicExampleCS{vylepšení člunu}
\dicEntry[lengja] \dicTerm{lengj|a\smash{\textsuperscript{1}}} \dicIPA{{l}{ei}{\textltailn}{\r{\textObardotlessj}}{a}} \dicPos{f}[1] \dicFlx{(‑u, ‑ur)}[7] \textbf{1.} \dicSynonym{ræma} \dicDirectTranslationCS{pruh, proužek}  \textbf{2.} \dicDirectTranslationCS{karton (cigaret)}
\dicEntry[lengja] \dicTerm{leng|ja\smash{\textsuperscript{2}}} \dicIPA{{l}{ei}{\textltailn}{\r{\textObardotlessj}}{a}} \dicPos{v}[2] \dicFlx{(‑di, ‑t)}[130] \dicFlx{acc} \textbf{1.} \dicDirectTranslationCS{prodloužit, prodlužovat} \dicExampleIS{lengja vinnutímann} \dicExampleCS{prodloužit pracovní dobu}  \textbf{2.} \dicPhraseIS{e‑ð lengir} \dicFlx{impers} \dicDirectTranslationCS{(co) se prodlužuje} \dicExampleIS{Daginn lengir.} \dicExampleCS{Den se prodlužuje.};  \dicIdiom{lengja}[eftir]{ \dicPhraseIS{e‑n lengir eftir e‑m\,/\addthin e‑u}} \dicFlx{impers} \dicDirectTranslationCS{(kdo) touží po (kom\,/\addthin čem)} \dicExampleIS{Mig lengir eftir henni.} \dicExampleCS{Toužím po ní.};  \dicIdiom{lengjast}{ \dicPhraseIS{lengjast}} \dicFlx{refl} \dicDirectTranslationCS{stávat se delší, prodlužovat se} \dicExampleIS{Dagurinn lengist.} \dicExampleCS{Den se prodlužuje.}
\dicEntry[lengra] \dicTerm{lengra} \dicIPA{{l}{\textepsilon}{\ng}{\r{g}}{r}{a}} \dicPos{adv} \dicFlx{comp} \dicLink{langt}
\dicEntry[lengri] \dicTerm{lengri} \dicIPA{{l}{\textepsilon}{\ng}{\r{g}}{r}{\textsci}} \dicPos{adj} \dicFlx{comp m} \dicLink{langur\smash{\textsuperscript{2}}}
\dicEntry[lengst] \dicTerm{lengst} \dicIPA{{l}{\textepsilon}{\ng}{s}{\textsubring{d}}} \dicPos{adv} \dicFlx{sup} \textbf{1.} \dicLink{langt}  \textbf{2.} \dicLink{lengi}
\dicEntry[lengstur] \dicTerm{lengstur} \dicIPA{{l}{\textepsilon}{\ng}{s}{\textsubring{d}}{\textscy}{\textsubring{r}}} \dicPos{adj} \dicFlx{m sg nom sup} \dicLink{langur\smash{\textsuperscript{2}}}
\dicEntry[lengur] \dicTerm{lengur} \dicIPA{{l}{ei}{\ng}{\r{g}}{\textscy}{\textsubring{r}}} \dicPos{adv} \dicFlx{comp} \dicLink{lengi}
\dicEntry[lens] \dicTerm{lens} \dicIPA{{l}{\textepsilon}{n}{s}} \dicPos{n}[2] \dicFlx{(‑)}[24] \dicFieldCat{nám.} \dicSynonym{byr} \dicDirectTranslationCS{zadní vítr}
\dicEntry[lenska] \dicTerm{lensk|a} \dicIPA{{l}{\textepsilon}{n}{s}{\r{g}}{a}} \dicPos{f}[1] \dicFlx{(‑u)}[5] \dicSynonym*{landsvenja} \dicDirectTranslationCS{zvyk, obyčej, zvyklost}
\dicEntry[lenskur] \dicTerm{lenskur} \dicIPA{{l}{\textepsilon}{n}{s}{\r{g}}{\textscy}{\textsubring{r}}} \dicPos{adj}[1]\dicFlx{}[-1] \dicPhraseIS{hvers lenskur} \dicDirectTranslationCS{jaké národnosti} \dicExampleIS{Ég vissi ekki hvers lenskur hann var.} \dicExampleCS{Nevěděl jsem, jaké je národnosti.}
\dicEntry[lep] \dicTerm{lep} \dicIPA{{l}{\textepsilon}{\textlengthmark}{\textsubring{b}}} \dicPos{v} \dicFlx{ind praes sg 1 pers} \dicLink{lepja}
\dicEntry[lepja] \dicTerm{lepja} \dicIPA{{l}{\textepsilon}{\textlengthmark}{\textsubring{b}}{j}{a}} \dicPos{v}[4] \dicFlx{(lep, lapti, löptum, lepti, lapið)}[47] \dicFlx{acc} \textbf{1.} \dicDirectTranslationCS{(s)lízat, vylízat} \dicIndirectTranslationCS{(o~zvířatech)} \dicExampleIS{lepja úr skálinni} \dicExampleCS{lízat z~misky}  \textbf{2.} \dicDirectTranslationCS{papouškovat, opakovat (bez porozumění ap.)}
\dicEntry[leppa] \dicTerm{lepp|a} \dicIPA{{l}{\textepsilon}{h}{\textsubring{b}}{a}} \dicPos{v}[1] \dicFlx{(‑aði)}[1] \dicFlx{acc} \textbf{1.} \dicPhraseIS{leppa sig} \dicSynonym*{klæða sig} \dicDirectTranslationCS{obléct se, oblékat se}  \textbf{2.} \dicSynonym{bæta} \dicDirectTranslationCS{opravit, vyspravit} \dicExampleIS{leppa upp á e‑ð} \dicExampleCS{vyspravit (co)}  \textbf{3.} \dicDirectTranslationCS{vázat} \dicIndirectTranslationCS{(v~šachách)}
\dicEntry[Leppalúði] \dicTerm{Leppa··lúð|i} \dicIPA{{l}{\textepsilon}{h}{\textsubring{b}}{a}{l}{u}{ð}{\textsci}} \dicPos{m}[1] \dicFlx{(‑a)}[5] \dicFlx{prop} \dicFieldCat{pov.} \dicDirectTranslationCS{Leppalúði} \dicIndirectTranslationCS{(nesmírně líný obr, otec 13 vánočních skřítků)}
\dicEntry[leppríki] \dicTerm{lepp··ríki} \dicIPA{{l}{\textepsilon}{h}{\textsubring{b}}{r}{i}{\r{\textObardotlessj}}{\textsci}} \dicPos{n}[2] \dicFlx{(‑s, ‑)}[16] \dicFieldCat{pol.} \dicDirectTranslationCS{loutkový\,/\addthin satelitní stát}
\dicEntry[leppstjórn] \dicTerm{lepp··stjórn} \dicIPA{{l}{\textepsilon}{h}{\textsubring{b}}{s}{\textsubring{d}}{j}{ou}{r}{\textsubring{d}}{\textsubring{n}}} \dicPos{f}[7] \dicFlx{(‑ar, ‑ir)}[1] \dicDirectTranslationCS{loutková vláda}
\dicEntry[leppur] \dicTerm{lepp|ur} \dicIPA{{l}{\textepsilon}{h}{\textsubring{b}}{\textscy}{\textsubring{r}}} \dicPos{m}[6] \dicFlx{(‑s, ‑ar)}[22] \textbf{1.} \dicSynonym{bót} \dicDirectTranslationCS{záplata}  \textbf{2.} \dicDirectTranslationCS{vložka do bot}  \textbf{3.} \dicPhraseIS{leppar} \dicFlx{pl} \dicSynonym*{tötrar} \dicDirectTranslationCS{hadry} \dicIndirectTranslationCS{(o~šatech)}  \textbf{4.} \dicSynonym{handbendi} \dicDirectTranslationCS{loutka, nohsled}  \textbf{5.} \dicDirectTranslationCS{klapka (na oku ap.)}  \textbf{6.} \dicDirectTranslationCS{vazba} \dicIndirectTranslationCS{(v~šachách)}
\dicEntry[lepti] \dicTerm{lepti} \dicIPA{{l}{\textepsilon}{f}{\textsubring{d}}{\textsci}} \dicPos{v} \dicFlx{con pf sg 1 pers} \dicLink{lepja}
\dicEntry[lerki] \dicTerm{lerki} \dicIPA{{l}{\textepsilon}{\textsubring{r}}{\r{\textObardotlessj}}{\textsci}} \dicPos{n}[2] \dicFlx{(‑s)}[20] \dicFieldCat{bot.} \dicDirectTranslationCS{modřín} \textit{(l.~{\textLA{Larix}})}  \dicsymPhoto\ 
\dicFigure{286.jpg}{Lerki}{Lerki - Zicha Ondřej, Biolib, Copyright/CC-BY-NC}
\dicEntry[les] \dicTerm{les} \dicIPA{{l}{\textepsilon}{\textlengthmark}{s}} \dicPos{v} \dicFlx{ind praes sg 1 pers} \dicLink{lesa}
\dicEntry[lesa] \dicTerm{lesa} \dicsymFrequent\  \dicIPA{{l}{\textepsilon}{\textlengthmark}{s}{a}} \dicPos{v}[6] \dicFlx{(les, las, lásum, læsi, lesið)}[16] \dicFlx{acc} \textbf{1.} \dicDirectTranslationCS{(pře)číst} \dicExampleIS{lesa söguna upphátt} \dicExampleCS{číst příběh nahlas}  \textbf{2.} \dicSynonym{læra} \dicDirectTranslationCS{studovat, učit se} \dicExampleIS{Hann les sagnfræði í háskóla.} \dicExampleCS{Studuje historii na vysoké škole.}  \textbf{3.} \dicLangCat{zast.} \dicDirectTranslationCS{sbírat (květy ap.)} \dicExampleIS{lesa aldin af trjánum} \dicExampleCS{sbírat plody ze stromů};  \dicIdiom{lesa}[fyrir]{ \dicPhraseIS{lesa fyrir}} \dicDirectTranslationCS{diktovat}; { \dicPhraseIS{lesa fyrir e‑m e‑ð}} \dicDirectTranslationCS{(na)diktovat (komu co)} \dicExampleIS{lesa henni bréfið fyrir} \dicExampleCS{diktovat jí dopis};  \dicIdiom{lesa}[í]{ \dicPhraseIS{lesa í lófa}} \dicLangCat{přen.} \dicDirectTranslationCS{číst z~dlaně};  \dicIdiom{lesa}[saman]{ \dicPhraseIS{lesa saman e‑ð og e‑ð}} \dicDirectTranslationCS{srovnávat\,/\addthin porovnávat (co) s~(čím)} \dicExampleIS{lesa frumritið og afritið saman} \dicExampleCS{srovnávat původní text s~opisem};  \dicIdiom{lesa}[til]{ \dicPhraseIS{lesa sér til e‑ð}} \dicDirectTranslationCS{seznámit se s~(čím), dozvědět se (co) nového (čtením)};  \dicIdiom{lesa}[undir]{ \dicPhraseIS{lesa undir próf}} \dicDirectTranslationCS{učit se na zkoušku, studovat na zkoušku} \dicExampleIS{lesa undir skóla} \dicExampleCS{učit se do školy};  \dicIdiom{lesa}[upp]{ \dicPhraseIS{lesa upp e‑ð}} \dicDirectTranslationCS{recitovat (co), přednášet (co)} \dicExampleIS{lesa upp ljóð} \dicExampleCS{recitovat báseň}; { \dicPhraseIS{lesa sig upp}} \dicSynonym{klifra} \dicDirectTranslationCS{šplhat, vyšplhat (se)};  \dicIdiom{lesa}[yfir]{ \dicPhraseIS{lesa e‑ð yfir}} \dicDirectTranslationCS{pročíst si (co)}
\dicEntry[lesandi] \dicTerm{les··|andi} \dicsymFrequent\  \dicIPA{{l}{\textepsilon}{\textlengthmark}{s}{a}{n}{\textsubring{d}}{\textsci}} \dicPos{m}[2] \dicFlx{(‑anda, ‑endur)}[1] \dicDirectTranslationCS{čtenář(ka)} \dicExampleIS{Lesendurnir eru ánægðir með bókina.} \dicExampleCS{Čtenářům se knížka líbí.}
\dicEntry[lesbía] \dicTerm{lesbí|a} \dicIPA{{l}{\textepsilon}{s}{\textsubring{b}}{i}{j}{a}} \dicPos{f}[1] \dicFlx{(‑u, ‑ur)}[7] \dicDirectTranslationCS{lesba, lesbička}
\dicEntry[lesblinda] \dicTerm{les··blind|a} \dicIPA{{l}{\textepsilon}{\textlengthmark}{s}{\textsubring{b}}{l}{\textsci}{n}{\textsubring{d}}{a}} \dicPos{f}[1] \dicFlx{(‑u)}[5] \dicFieldCat{med.} \dicDirectTranslationCS{dyslexie}
\dicEntry[lesbók] \dicTerm{les··|bók} \dicIPA{{l}{\textepsilon}{\textlengthmark}{s}{\textsubring{b}}{ou}{\r{g}}} \dicPos{f}[8] \dicFlx{(‑bókar, ‑bækur)}[5] \dicDirectTranslationCS{čítanka}
\dicEntry[lesefni] \dicTerm{les··efni} \dicIPA{{l}{\textepsilon}{\textlengthmark}{s}{\textepsilon}{\textsubring{b}}{n}{\textsci}} \dicPos{n}[2] \dicFlx{(‑s, ‑)}[14] \dicDirectTranslationCS{čtení, četba, čtivo}
\dicEntry[lesgleraugu] \dicTerm{les··gler·augu} \dicIPA{{l}{\textepsilon}{\textlengthmark}{s}{\r{g}}{l}{\textepsilon}{r}{\oe i}{\textbabygamma}{\textscy}} \dicPos{n}[1] \dicFlx{pl}[3] \dicDirectTranslationCS{brýle na čtení}
\dicEntry[lesið] \dicTerm{lesið} \dicIPA{{l}{\textepsilon}{\textlengthmark}{s}{\textsci}{\texttheta}} \dicPos{v} \dicFlx{supin} \dicLink{lesa}
\dicEntry[lesinn] \dicTerm{lesinn} \dicIPA{{l}{\textepsilon}{\textlengthmark}{s}{\textsci}{\textsubring{n}}} \dicPos{adj}[6]\dicFlx{}[-2] \dicDirectTranslationCS{sečtělý}
\dicEntry[lesminni] \dicTerm{les··minni} \dicIPA{{l}{\textepsilon}{\textlengthmark}{s}{m}{\textsci}{n}{\textsci}} \dicPos{n}[2] \dicFlx{(‑s, ‑)}[14] \dicFieldCat{poč.} \dicDirectTranslationCS{ROM} \dicIndirectTranslationCS{(typ elektronické paměti používaný pro uložení firmware v~elektronických přístrojích)} \dicAntonym{vinnsluminni}
\dicEntry[lesning] \dicTerm{les··ning} \dicIPA{{l}{\textepsilon}{s}{\textsubring{d}}{n}{i}{\ng}{\r{g}}} \dicPos{f}[4] \dicFlx{(‑ar, ‑ar)}[5] \dicDirectTranslationCS{četba, čtení}
\dicEntry[Lesótó] \dicTerm{Lesótó} \dicIPA{{l}{\textepsilon}{\textlengthmark}{s}{ou}{\textsubring{d}}{ou}} \dicPos{n}[4] \dicFlx{indecl}[2] \dicFieldCat{geog.} \dicDirectTranslationCS{Lesotho}
\dicEntry[Lesótói] \dicTerm{Lesótó|i} \dicIPA{{l}{\textepsilon}{\textlengthmark}{s}{ou}{\textsubring{d}}{ou}{\textsci}} \dicPos{m}[1] \dicFlx{(‑a, ‑ar)}[1] \dicLink{Lesótómaður}
\dicEntry[Lesótómaður] \dicTerm{Lesótó··|maður}\dicTerm{, Lesótói} \dicIPA{{l}\-{\textepsilon}\-{\textlengthmark}\-{s}\-{ou}\-{\textsubring{d}}\-{ou}\-{m}\-{a}\-{ð}\-{\textscy}\-{\textsubring{r}}\-} \dicPos{m}[13] \dicFlx{(‑manns, ‑menn)}[2] \dicDirectTranslationCS{Lesothan(ka)}
\dicEntry[lesótóskur] \dicTerm{lesótóskur} \dicIPA{{l}{\textepsilon}{\textlengthmark}{s}{ou}{\textsubring{d}}{ou}{s}{\r{g}}{\textscy}{\textsubring{r}}} \dicPos{adj}[1]\dicFlx{}[-6] \dicDirectTranslationCS{lesothský}
\dicEntry[lesskilningur] \dicTerm{les··skiln·ing|ur} \dicIPA{{l}{\textepsilon}{\textlengthmark}{s}{\r{\textObardotlessj}}{\textsci}{l}{n}{i}{\ng}{\r{g}}{\textscy}{\textsubring{r}}} \dicPos{m}[6] \dicFlx{(‑s)}[9] \dicDirectTranslationCS{porozumění (psanému) textu}
\dicEntry[lest] \dicTerm{lest} \dicsymFrequent\  \dicIPA{{l}{\textepsilon}{s}{\textsubring{d}}} \dicPos{f}[7] \dicFlx{(‑ar, ‑ir)}[1] \textbf{1.} \dicDirectTranslationCS{vlak, vláček} \dicExampleIS{fara með lest} \dicExampleCS{jet vlakem};  \dicPhraseIS{missa af lestinni} {\textbf{a.}} \dicDirectTranslationCS{zmeškat vlak};  {\textbf{b.}} \dicLangCat{přen.} \dicDirectTranslationCS{nechat si ujet vlak} \dicIndirectTranslationCS{(promeškat příležitost)}  \textbf{2.} \dicFieldCat{nám.} \dicSynonym{farmur} \dicDirectTranslationCS{náklad}  \textbf{3.} \dicSynonym{tonn} \dicDirectTranslationCS{(lodní) tuna}  \textbf{4.} \dicFieldCat{nám.} \dicSynonym*{lestarrúm} \dicDirectTranslationCS{nákladní prostor}  \textbf{5.} \dicSynonym{röð} \dicDirectTranslationCS{řada, karavana} \dicExampleIS{löng lest af bílum} \dicExampleCS{dlouhá řada aut}
\dicEntry[lesta] \dicTerm{lest|a} \dicIPA{{l}{\textepsilon}{s}{\textsubring{d}}{a}} \dicPos{v}[1] \dicFlx{(‑aði)}[1] \dicSynonym{hlaða\smash{\textsuperscript{2}}} \dicDirectTranslationCS{naložit, nakládat} \dicExampleIS{lesta skipið} \dicExampleCS{nakládat loď}
\dicEntry[lestarstjóri] \dicTerm{lestar··stjór|i} \dicIPA{{l}{\textepsilon}{s}{\textsubring{d}}{a}{\textsubring{r}}{s}{\textsubring{d}}{j}{ou}{r}{\textsci}} \dicPos{m}[1] \dicFlx{(‑a, ‑ar)}[1] \dicDirectTranslationCS{strojvedoucí, strojvůdce, strojvůdkyně}
\dicEntry[lestir] \dicTerm{lestir} \dicIPA{{l}{\textepsilon}{s}{\textsubring{d}}{\textsci}{\textsubring{r}}} \dicPos{m} \dicFlx{pl nom} \dicLink{löstur}
\dicEntry[lestrarbók] \dicTerm{lestrar··|bók} \dicIPA{{l}{\textepsilon}{s}{\textsubring{d}}{r}{a}{r}{\textsubring{b}}{ou}{\r{g}}} \dicPos{f}[8] \dicFlx{(‑bókar, ‑bækur)}[5] \dicDirectTranslationCS{čítanka}
\dicEntry[lestrarsalur] \dicTerm{lestrar··sal|ur} \dicIPA{{l}{\textepsilon}{s}{\textsubring{d}}{r}{a}{\textsubring{r}}{s}{a}{l}{\textscy}{\textsubring{r}}} \dicPos{m}[10] \dicFlx{(‑ar, ‑ir)}[14] \dicDirectTranslationCS{čítárna}
\dicEntry[lestur] \dicTerm{lest|ur} \dicsymFrequent\  \dicIPA{{l}{\textepsilon}{s}{\textsubring{d}}{\textscy}{\textsubring{r}}} \dicPos{m}[5] \dicFlx{(‑rar\,/\addthin ‑urs, ‑rar)}[9] \textbf{1.} \dicSynonym*{það að lesa} \dicDirectTranslationCS{čtení, četba} \dicExampleIS{skemmtilegur lestur} \dicExampleCS{zábavné čtení}  \textbf{2.} \dicSynonym{nám} \dicDirectTranslationCS{studium, učení se}  \textbf{3.} \dicSynonym{klifur} \dicDirectTranslationCS{šplhání}
\dicEntry[let] \dicTerm{let} \dicIPA{{l}{\textepsilon}{\textlengthmark}{\textsubring{d}}} \dicPos{v} \dicFlx{ind praes sg 1 pers} \dicLink{letja}
\dicEntry[leti] \dicTerm{let|i} \dicIPA{{l}{\textepsilon}{\textlengthmark}{\textsubring{d}}{\textsci}} \dicPos{f}[3] \dicFlx{(‑i)}[3] \dicDirectTranslationCS{lenost, lenivost, zahálčivost}
\dicEntry[letidýr] \dicTerm{leti··dýr} \dicIPA{{l}{\textepsilon}{\textlengthmark}{\textsubring{d}}{\textsci}{\textsubring{d}}{i}{\textsubring{r}}} \dicPos{n}[2] \dicFlx{(‑s, ‑)}[5] \dicFieldCat{zool.} \dicDirectTranslationCS{lenochod} \textit{(l.~{\textLA{Bradypus}})}  \dicsymPhoto\ 
\dicFigure{ds_image_letidyr_0_1.jpg}{Letidýr}{Letidýr - Stefan Laube, PD}
\dicEntry[letihaugur] \dicTerm{leti··haug|ur} \dicIPA{{l}{\textepsilon}{\textlengthmark}{\textsubring{d}}{\textsci}{h}{\oe i}{\textbabygamma}{\textscy}{\textsubring{r}}} \dicPos{m}[6] \dicFlx{(‑s, ‑ar)}[22] \dicDirectTranslationCS{lenoch, lenoška, povaleč(ka)}
\dicEntry[letilegur] \dicTerm{leti··legur} \dicIPA{{l}{\textepsilon}{\textlengthmark}{\textsubring{d}}{\textsci}{l}{\textepsilon}{\textbabygamma}{\textscy}{\textsubring{r}}} \dicPos{adj}[1]\dicFlx{}[-8] \dicDirectTranslationCS{líný, lenivý}
\dicEntry[letingi] \dicTerm{let··ing|i} \dicIPA{{l}{\textepsilon}{\textlengthmark}{\textsubring{d}}{i}{\textltailn}{\r{\textObardotlessj}}{\textsci}} \dicPos{m}[1] \dicFlx{(‑ja, ‑jar)}[14] \dicDirectTranslationCS{lenoch, lenoška, povaleč(ka)}
\dicEntry[letja] \dicTerm{letja} \dicIPA{{l}{\textepsilon}{\textlengthmark}{\textsubring{d}}{j}{a}} \dicPos{v}[4] \dicFlx{(let, latti, löttum, letti, latt)}[48] \dicFlx{acc} \dicSynonym*{telja úr} \dicDirectTranslationCS{odradit, odrazovat};  \dicPhraseIS{letja e‑n e‑s} \dicDirectTranslationCS{odrazovat (koho) od (čeho)} \dicExampleIS{letja e‑n fararinnar} \dicExampleCS{odrazovat (koho) od cesty}
\dicEntry[letra] \dicTerm{letr|a} \dicIPA{{l}{\textepsilon}{\textlengthmark}{\textsubring{d}}{r}{a}} \dicPos{v}[1] \dicFlx{(‑aði)}[1] \dicFlx{acc} \textbf{1.} \dicSynonym{skrifa} \dicDirectTranslationCS{psát, napsat}  \textbf{2.} \dicSynonym*{höggva stafi í stein} \dicDirectTranslationCS{vysázet\,/\addthin vyrýt písmem}
\dicEntry[Letti] \dicTerm{Lett|i\smash{\textsuperscript{1}}} \dicIPA{{l}{\textepsilon}{h}{\textsubring{d}}{\textsci}} \dicPos{m}[1] \dicFlx{(‑a, ‑ar)}[1] \dicDirectTranslationCS{Lotyš(ka)}
\dicEntry[letti] \dicTerm{letti\smash{\textsuperscript{2}}} \dicIPA{{l}{\textepsilon}{h}{\textsubring{d}}{\textsci}} \dicPos{v} \dicFlx{con pf sg 1 pers} \dicLink{letja}
\dicEntry[Lettland] \dicTerm{Lett··land} \dicIPA{{l}{\textepsilon}{h}{\textsubring{d}}{l}{a}{n}{\textsubring{d}}} \dicPos{n}[2] \dicFlx{(‑s)}[4] \dicFieldCat{geog.} \dicDirectTranslationCS{Lotyšsko}
\dicEntry[lettneska] \dicTerm{lett··nesk|a} \dicIPA{{l}{\textepsilon}{h}{\textsubring{d}}{n}{\textepsilon}{s}{\r{g}}{a}} \dicPos{f}[1] \dicFlx{(‑u)}[5] \dicDirectTranslationCS{lotyština}
\dicEntry[lettneskur] \dicTerm{lett··neskur} \dicIPA{{l}{\textepsilon}{h}{\textsubring{d}}{n}{\textepsilon}{s}{\r{g}}{\textscy}{\textsubring{r}}} \dicPos{adj}[1]\dicFlx{}[-6] \dicDirectTranslationCS{lotyšský}
\dicEntry[letur] \dicTerm{letur} \dicIPA{{l}{\textepsilon}{\textlengthmark}{\textsubring{d}}{\textscy}{\textsubring{r}}} \dicPos{n}[2] \dicFlx{(‑s, ‑)}[25] \textbf{1.} \dicSynonym*{stafategund} \dicDirectTranslationCS{písmo, druh\,/\addthin typ písma, font}  \textbf{2.} \dicSynonym{áletrun} \dicDirectTranslationCS{nápis, epigraf} \dicExampleIS{letur á steini} \dicExampleCS{nápis na kameni}
\dicEntry[leturbreyting] \dicTerm{letur··breyt·ing} \dicIPA{{l}{\textepsilon}{\textlengthmark}{\textsubring{d}}{\textscy}{r}{\textsubring{b}}{r}{ei}{\textsubring{d}}{i}{\ng}{\r{g}}} \dicPos{f}[4] \dicFlx{(‑ar, ‑ar)}[5] \dicDirectTranslationCS{změna písma} \dicIndirectTranslationCS{(z~tučného písma na normální ap.)}
\dicEntry[leturgerð] \dicTerm{letur··gerð} \dicIPA{{l}{\textepsilon}{\textlengthmark}{\textsubring{d}}{\textscy}{r}{\r{\textObardotlessj}}{\textepsilon}{r}{\texttheta}} \dicPos{f}[7] \dicFlx{(‑ar, ‑ir)}[1] \dicSynonym*{leturtegund} \dicDirectTranslationCS{font, znaková sada}
\dicEntry[leturgrafari] \dicTerm{letur··graf·ar|i} \dicIPA{{l}{\textepsilon}{\textlengthmark}{\textsubring{d}}{\textscy}{r}{\r{g}}{r}{a}{v}{a}{r}{\textsci}} \dicPos{m}[1] \dicFlx{(‑a, ‑ar)}[10] \dicDirectTranslationCS{rytec\,/\addthin rytečka (písma)}
\dicEntry[leturstærð] \dicTerm{letur··stærð} \dicIPA{{l}{\textepsilon}{\textlengthmark}{\textsubring{d}}{\textscy}{\textsubring{r}}{s}{\textsubring{d}}{a}{i}{r}{\texttheta}} \dicPos{f}[7] \dicFlx{(‑ar, ‑ir)}[1] \dicDirectTranslationCS{velikost písma}
\dicEntry[lexía] \dicTerm{lexí|a} \dicIPA{{l}{\textepsilon}{x}{s}{i}{j}{a}} \dicPos{f}[1] \dicFlx{(‑u, ‑ur)}[7] \textbf{1.} \dicSynonym{námsefni} \dicDirectTranslationCS{lekce} \dicIndirectTranslationCS{(úsek učební látky)};  \dicPhraseIS{lesa lexíuna sína} \dicDirectTranslationCS{naučit se, udělat domácí úkoly}  \textbf{2.} \dicSynonym{áminning} \dicDirectTranslationCS{lekce, (nepříjemné) poučení}
\dicEntry[leyfa] \dicTerm{leyf|a} \dicsymFrequent\  \dicIPA{{l}{ei}{\textlengthmark}{v}{a}} \dicPos{v}[2] \dicFlx{(‑ði, ‑t)}[106] \dicFlx{dat + acc} \dicDirectTranslationCS{dovolit, dovolovat, povolit, povolovat, nechat, nechávat};  \dicPhraseIS{leyfa e‑m e‑ð} \dicDirectTranslationCS{dovolit (co komu)} \dicExampleIS{leyfa henni að halda á hestinum} \dicExampleCS{dovolit jí vést koně};  \dicIdiom{leyfast}{ \dicPhraseIS{e‑m leyfist að (gera e‑ð)}} \dicFlx{refl impers} \dicDirectTranslationCS{(komu) je dovoleno (udělat (co))} \dicExampleIS{ef mér leyfist að spyrja} \dicExampleCS{pokud se můžu zeptat}
\dicEntry[leyfður] \dicTerm{leyfður} \dicIPA{{l}{ei}{v}{ð}{\textscy}{\textsubring{r}}} \dicPos{adj}[2]\dicFlx{}[-1] \dicDirectTranslationCS{dovolený, povolený} \dicExampleIS{leyfður hraði} \dicExampleCS{povolená rychlost}
\dicEntry[leyfi] \dicTerm{leyfi} \dicsymFrequent\  \dicIPA{{l}{ei}{\textlengthmark}{v}{\textsci}} \dicPos{n}[2] \dicFlx{(‑s, ‑)}[14] \textbf{1.} \dicSynonym{heimild} \dicDirectTranslationCS{licence, oprávnění}  \textbf{2.} \dicDirectTranslationCS{povolení, dovolení, souhlas, svolení} \dicExampleIS{án leyfis} \dicExampleCS{bez povolení};  \dicPhraseIS{með leyfi} \dicFlx{adv} \dicDirectTranslationCS{s~dovolením}  \textbf{3.} \dicSynonym{frí} \dicDirectTranslationCS{dovolená, volno}
\dicEntry[leyfilegur] \dicTerm{leyfi··legur} \dicIPA{{l}{ei}{\textlengthmark}{v}{\textsci}{l}{\textepsilon}{\textbabygamma}{\textscy}{\textsubring{r}}} \dicPos{adj}[1]\dicFlx{}[-8] \dicSynonym{heimill} \dicDirectTranslationCS{povolený, dovolený, přípustný}
\dicEntry[leyfisleysi] \dicTerm{leyfis··leysi} \dicIPA{{l}{ei}{\textlengthmark}{v}{\textsci}{s}{l}{ei}{s}{\textsci}} \dicPos{n}[2] \dicFlx{(‑s)}[20] \dicPhraseIS{gera e‑ð í leyfisleysi} \dicDirectTranslationCS{udělat (co) bez povolení\,/\addthin dovolení\,/\addthin svolení}
\dicEntry[leyfisveiting] \dicTerm{leyfis··veit·ing} \dicIPA{{l}{ei}{\textlengthmark}{v}{\textsci}{s}{v}{ei}{\textsubring{d}}{i}{\ng}{\r{g}}} \dicPos{f}[4] \dicFlx{(‑ar, ‑ar)}[5] \dicDirectTranslationCS{udělení\,/\addthin vydání oprávnění\,/\addthin povolení}
\dicEntry[leyna] \dicTerm{leyn|a} \dicsymFrequent\  \dicIPA{{l}{ei}{\textlengthmark}{n}{a}} \dicPos{v}[2] \dicFlx{(‑di, ‑t)}[141] \dicFlx{dat} \dicDirectTranslationCS{ukrýt, ukrývat, skrýt, skrývat, (u)tajit, zatajovat};  \dicPhraseIS{leyna e‑n e‑u, leyna e‑u fyrir e‑m} \dicDirectTranslationCS{skrýt (co) před (kým), zatajit (co) před (kým), zatajit (komu co)} \dicExampleIS{leyna hana vitneskju um málið} \dicExampleCS{zatajit jí informace o~té záležitosti};  \dicIdiom{leyna}[á]{ \dicPhraseIS{leyna á sér}} \dicDirectTranslationCS{nezdát se} \dicIndirectTranslationCS{(neukazovat pravou skutečnost)};  \dicIdiom{leynast}{ \dicPhraseIS{leynast}} \dicFlx{refl} \dicDirectTranslationCS{skrýt se, skrývat se}
\dicEntry[leynd] \dicTerm{leynd} \dicIPA{{l}{ei}{n}{\textsubring{d}}} \dicPos{f}[7] \dicFlx{(‑ar, ‑ir)}[1] \textbf{1.} \dicSynonym{launung} \dicDirectTranslationCS{tajnost};  \dicPhraseIS{með leynd} \dicFlx{adv} \dicDirectTranslationCS{v~tajnosti, potají}  \textbf{2.} \dicSynonym*{leynistaður} \dicDirectTranslationCS{(místo) utajení}
\begin{xtolerant}{}{1pt}
\dicEntry[leyndardómsfullur] \dicTerm{leyndar·dóms··|fullur} \dicsymFrequent\addthinS\textls[-10]{\dicIPA{{l}\-{ei}\-{n}\-{\textsubring{d}}\-{a}\-{r}\-{\textsubring{d}}\-{ou}\-{m}\-{s}\-{f}\-{\textscy}\-{\textsubring{d}}\-{l}\-{\textscy}\-{\textsubring{r}}}} \dicPos{adj}[10]\addthinS\dicFlx{(comp ‑fyllri, sup ‑fyllstur)}[7] \dicDirectTranslationCS{tajuplný, tajemný, záhadný} \dicExampleIS{leyndardómsfullur á svip} \dicExampleCS{s~tajuplným výrazem ve tváři}
\end{xtolerant}
\dicEntry[leyndardómur] \dicTerm{leyndar··dóm|ur} \dicIPA{{l}{ei}{n}{\textsubring{d}}{a}{r}{\textsubring{d}}{ou}{m}{\textscy}{\textsubring{r}}} \dicPos{m}[6] \dicFlx{(‑s, ‑ar)}[10] \dicDirectTranslationCS{tajemství, tajemnost, záhada} \dicExampleIS{leyndardómur lífsins} \dicExampleCS{tajemství života}
\dicEntry[leyndarmál] \dicTerm{leyndar··mál} \dicsymFrequent\  \dicIPA{{l}{ei}{n}{\textsubring{d}}{a}{r}{m}{au}{\textsubring{l}}} \dicPos{n}[2] \dicFlx{(‑s, ‑)}[5] \dicDirectTranslationCS{tajemství} \dicExampleIS{opinbert leyndarmál} \dicExampleCS{veřejné tajemství}
\dicEntry[leyndur] \dicTerm{leyndur} \dicsymFrequent\  \dicIPA{{l}{ei}{n}{\textsubring{d}}{\textscy}{\textsubring{r}}} \dicPos{adj}[2]\dicFlx{}[-14] \textbf{1.} \dicDirectTranslationCS{skrytý, schovaný} \dicExampleIS{leyndir hæfileikar} \dicExampleCS{skryté schopnosti}  \textbf{2.} \dicDirectTranslationCS{tajný, utajený};  \dicPhraseIS{fara leynt með e‑ð} \dicDirectTranslationCS{tajit (co), držet (co) v~tajnosti};  \dicPhraseIS{halda e‑ð leyndum} \dicDirectTranslationCS{tajit (co), držet (co) v~tajnosti}
\dicEntry[leyni] \dicTerm{leyni} \dicIPA{{l}{ei}{\textlengthmark}{n}{\textsci}} \dicPos{n}[2] \dicFlx{(‑s, ‑)}[14] \dicDirectTranslationCS{skrýš, úkryt};  \dicPhraseIS{í leyni} \dicFlx{adv} \dicDirectTranslationCS{tajně, potají};  \dicPhraseIS{liggja í leyni} \dicDirectTranslationCS{skrývat se, ukrývat se}
\dicEntry[leynifélag] \dicTerm{leyni··fé·|lag} \dicIPA{{l}{ei}{\textlengthmark}{n}{\textsci}{f}{j}{\textepsilon}{l}{a}{x}} \dicPos{n}[2] \dicFlx{(‑lags, ‑lög)}[8] \dicDirectTranslationCS{tajné společenství, tajná společnost}
\dicEntry[leynilegur] \dicTerm{leyni··legur} \dicIPA{{l}{ei}{\textlengthmark}{n}{\textsci}{l}{\textepsilon}{\textbabygamma}{\textscy}{\textsubring{r}}} \dicPos{adj}[1]\dicFlx{}[-8] \dicSynonym{dulinn} \dicDirectTranslationCS{tajný, utajený}
\dicEntry[leynilögregla] \dicTerm{leyni··lög·regl|a} \dicIPA{{l}{ei}{\textlengthmark}{n}{\textsci}{l}{\oe}{\textbabygamma}{r}{\textepsilon}{\r{g}}{l}{a}} \dicPos{f}[1] \dicFlx{(‑u)}[5] \dicDirectTranslationCS{tajná policie}

\dicEntry[leynilögreglumaður] \textls[23]{\dicTerm{leyni··lög·reglu·|maður} \dicIPA{{l}\-{ei}\-{\textlengthmark}\-{n}\-{\textsci}\-{l}\-{\oe}\-{\textbabygamma}\-{r}\-{\textepsilon}\-{\r{g}}\-{l}\-{\textscy}\-{m}\-{a}\-{ð}\-{\textscy}\-{\textsubring{r}}} \dicPos{m}[13] \dicFlx{(‑manns, ‑menn)}[2]} \dicDirectTranslationCS{detektiv, tajný policista, tajná policistka}

\dicEntry[leynilögreglusaga] \dicTerm{leyni·lög·reglu··|saga} \dicIPA{{l}\-{ei}\-{\textlengthmark}\-{n}\-{\textsci}\-{l}\-{\oe}\-{\textbabygamma}\-{r}\-{\textepsilon}\-{\r{g}}\-{l}\-{\textscy}\-{s}\-{a}\-{\textbabygamma}\-{a}\-} \dicPos{f}[1] \dicFlx{(‑sögu, ‑sög\-ur)}[14] \dicDirectTranslationCS{detektivka, detektivní příběh}
\dicEntry[leyniskytta] \dicTerm{leyni··skytt|a} \dicIPA{{l}{ei}{\textlengthmark}{n}{\textsci}{s}{\r{\textObardotlessj}}{\textsci}{h}{\textsubring{d}}{a}} \dicPos{f}[1] \dicFlx{(‑u, ‑ur)}[19] \dicDirectTranslationCS{odstřelovač(ka), snajper(ka)}
\dicEntry[leyniþjónusta] \textls[15]{\dicTerm{leyni··þjón·ust|a} \dicIPA{{l}{ei}{\textlengthmark}{n}{\textsci}{\texttheta}{j}{ou}{n}{\textscy}{s}{\textsubring{d}}{a}} \dicPos{f}[1] \dicFlx{(‑u)}[5] \dicDirectTranslationCS{tajná\,/\addthin vý\-zvěd\-ná služba, rozvědka}}
\dicEntry[leysa] \dicTerm{leys|a} \dicsymFrequent\  \dicIPA{{l}{ei}{\textlengthmark}{s}{a}} \dicPos{v}[2] \dicFlx{(‑ti, ‑t)}[64] \dicFlx{acc} \textbf{1.} \dicSynonym{losa} \dicDirectTranslationCS{rozvázat, rozvazovat, rozmotat, rozmotávat, uvolnit, uvolňovat} \dicExampleIS{leysa bandið} \dicExampleCS{rozvázat provázek}  \textbf{2.} \dicSynonym*{finna lausn} \dicDirectTranslationCS{vyřešit, rozřešit, rozuzlit}  \textbf{3.} \dicPhraseIS{e‑ð leysir} \dicFlx{impers} \dicDirectTranslationCS{(co) taje, (co) roztává} \dicExampleIS{Ísa leysir.} \dicExampleCS{Ledy tají.};  \dicIdiom{leysa}[af]{ \dicPhraseIS{leysa e‑n af}} \dicDirectTranslationCS{zastoupit (koho)}; { \dicPhraseIS{leysa e‑ð af hólmi}} \dicDirectTranslationCS{nahradit (co), zastoupit (co)};  \dicIdiom{leysa}[niður]{ \dicPhraseIS{leysa niður um sig}} \dicDirectTranslationCS{stáhnout si kalhoty};  \dicIdiom{leysa}[undan]{ \dicPhraseIS{leysa e‑n undan e‑u}} \dicDirectTranslationCS{oprostit (koho) od (čeho), osvobodit (koho) od (čeho)};  \dicIdiom{leysa}[upp]{ \dicPhraseIS{leysa e‑ð upp}} \dicDirectTranslationCS{rozpustit\,/\addthin rozpouštět (co) (sůl ve vodě ap.)};  \dicIdiom{leysa}[úr]{ \dicPhraseIS{leysa úr e‑u}} \dicDirectTranslationCS{vyřešit (co), nalézt řešení (čeho)} \dicExampleIS{leysa úr vanda} \dicExampleCS{vyřešit problém};  \dicIdiom{leysa}[út]{ \dicPhraseIS{leysa e‑ð út}} \dicDirectTranslationCS{zaplatit za (co) (za objednané zboží)};  \dicPhraseIS{leysa e‑n út með e‑ð} \dicDirectTranslationCS{obdarovat (koho čím) na odchodu};  \dicIdiom{leysast}{ \dicPhraseIS{leysast}} \dicFlx{refl} {\textbf{a.}} \dicDirectTranslationCS{rozvázat se, rozmotat se};  {\textbf{b.}} \dicDirectTranslationCS{vyřešit se, rozuzlit se} \dicExampleIS{Deilan leystist með samkomulagi.} \dicExampleCS{Spor se vyřešil dohodou.}
\dicEntry[leysanlegur] \dicTerm{leysan··legur} \dicIPA{{l}{ei}{\textlengthmark}{s}{a}{n}{l}{\textepsilon}{\textbabygamma}{\textscy}{\textsubring{r}}} \dicPos{adj}[1]\dicFlx{}[-8] \textbf{1.} \dicFieldCat{chem.} \dicDirectTranslationCS{rozpustný}  \textbf{2.} \dicDirectTranslationCS{(vy)řešitelný}
\dicEntry[leysiefni] \dicTerm{leysi··efni} \dicIPA{{l}{ei}{\textlengthmark}{s}{\textsci}{\textepsilon}{\textsubring{b}}{n}{\textsci}} \dicPos{n}[2] \dicFlx{(‑s, ‑)}[14] \dicFieldCat{chem.} \dicDirectTranslationCS{rozpouštědlo}
\dicEntry[leysigeisli] \dicTerm{leysi··geisl|i} \dicIPA{{l}{ei}{\textlengthmark}{s}{\textsci}{\r{\textObardotlessj}}{ei}{s}{\textsubring{d}}{l}{\textsci}} \dicPos{m}[1] \dicFlx{(‑a, ‑ar)}[1] \dicFieldCat{fyz.} \dicDirectTranslationCS{laserový paprsek}
\dicEntry[leysing] \dicTerm{leys··ing} \dicIPA{{l}{ei}{\textlengthmark}{s}{i}{\ng}{\r{g}}} \dicPos{f}[4] \dicFlx{(‑ar, ‑ar)}[5] \textbf{1.} \dicSynonym{lausn} \dicDirectTranslationCS{(vy)řešení, rozuzlení}  \textbf{2.} \dicSynonym*{snjóbráð} \dicDirectTranslationCS{tání, obleva}  \textbf{3.} \dicSynonym{upplausn} \dicDirectTranslationCS{rozpuštění, rozpouštění (látky ve vodě ap.)} \dicExampleIS{leysing sykurs í vatni} \dicExampleCS{rozpuštění cukru ve vodě}
\dicEntry[leysingi] \dicTerm{leys··ing|i} \dicIPA{{l}{ei}{\textlengthmark}{s}{i}{\textltailn}{\r{\textObardotlessj}}{\textsci}} \dicPos{m}[1] \dicFlx{(‑ja, ‑jar)}[14] \dicFieldCat{hist.} \dicDirectTranslationCS{osvobozený otrok, osvobozená otrokyně}
\dicEntry[leysinn] \dicTerm{leysinn} \dicIPA{{l}{ei}{\textlengthmark}{s}{\textsci}{\textsubring{n}}} \dicPos{adj}[6]\dicFlx{}[-6] \dicDirectTranslationCS{rozpustný}
\dicEntry[leysiprentari] \dicTerm{leysi··prent·ar|i} \dicIPA{{l}{ei}{\textlengthmark}{s}{\textsci}{p\smash{\textsuperscript{h}}}{r}{\textepsilon}{\textsubring{n}}{\textsubring{d}}{a}{r}{\textsci}} \dicPos{m}[1] \dicFlx{(‑a, ‑ar)}[13] \dicFieldCat{poč.} \dicDirectTranslationCS{laserová tiskárna}
\dicEntry[leysir] \dicTerm{leys|ir} \dicIPA{{l}{ei}{\textlengthmark}{s}{\textsci}{\textsubring{r}}} \dicPos{m}[7] \dicFlx{(‑is, ‑ar)}[1] \dicFieldCat{fyz.} \dicDirectTranslationCS{laser}
\dicEntry[leysni] \dicTerm{leysn|i} \dicIPA{{l}{ei}{s}{\textsubring{d}}{n}{\textsci}} \dicPos{f}[3] \dicFlx{(‑i)}[3] \dicFieldCat{chem.} \dicDirectTranslationCS{rozpustnost (materiálu ap.)}
\dicEntry[leyti] \dicTerm{leyti} \dicsymFrequent\  \dicIPA{{l}{ei}{\textlengthmark}{\textsubring{d}}{\textsci}} \dicPos{n}[2] \dicFlx{(‑s)}[20] \textbf{1.} \dicSynonym{tillit} \dicDirectTranslationCS{ohled, zřetel, míra};  \dicPhraseIS{að mestu leyti} \dicFlx{adv} \dicDirectTranslationCS{převážně, většinou};  \dicPhraseIS{að miklu leyti} \dicFlx{adv} \dicDirectTranslationCS{do velké míry, z~velké části};  \dicPhraseIS{að mörgu leyti} \dicFlx{adv} \dicDirectTranslationCS{do značné míry};  \dicPhraseIS{að nokkru leyti} \dicFlx{adv} \dicDirectTranslationCS{do určité míry, částečně};  \dicPhraseIS{að vissu leyti} \dicFlx{adv} \dicDirectTranslationCS{do jisté míry};  \dicPhraseIS{að öðru leyti} \dicFlx{adv} \dicDirectTranslationCS{jinak};  \dicPhraseIS{að þessu leyti} \dicFlx{adv} \dicDirectTranslationCS{v~tomto ohledu\,/\addthin směru}  \textbf{2.} \dicSynonym{tími} \dicDirectTranslationCS{doba, čas} \dicExampleIS{um sjöleytið} \dicExampleCS{kolem sedmé hodiny};  \dicPhraseIS{um hvaða leyti} \dicDirectTranslationCS{v~kolik (hodin)};  \dicPhraseIS{um sama leyti} \dicFlx{adv} \dicDirectTranslationCS{(přibližně) ve stejný čas};  \dicPhraseIS{um það leyti} \dicFlx{adv} \dicDirectTranslationCS{tehdy, tenkrát}
\dicEntry[léð] \dicTerm{léð} \dicIPA{{l}{j}{\textepsilon}{\textlengthmark}{\texttheta}} \dicPos{v} \dicFlx{supin} \dicLink{ljá}
\dicEntry[léði] \dicTerm{léði} \dicIPA{{l}{j}{\textepsilon}{\textlengthmark}{ð}{\textsci}} \dicPos{v} \dicFlx{ind\,/\addthin con pf sg 1 pers} \dicLink{ljá}
\dicEntry[léðum] \dicTerm{léðum} \dicIPA{{l}{j}{\textepsilon}{\textlengthmark}{ð}{\textscy}{\textsubring{m}}} \dicPos{v} \dicFlx{ind pf pl 1 pers} \dicLink{ljá}
\dicEntry[lék] \dicTerm{lék} \dicIPA{{l}{j}{\textepsilon}{\textlengthmark}{\r{g}}} \dicPos{v} \dicFlx{ind pf sg 1 pers} \dicLink{leika}
\dicEntry[léki] \dicTerm{léki} \dicIPA{{l}{j}{\textepsilon}{\textlengthmark}{\r{\textObardotlessj}}{\textsci}} \dicPos{v} \dicFlx{con pf sg 1 pers} \dicLink{leika}
\dicEntry[lékum] \dicTerm{lékum} \dicIPA{{l}{j}{\textepsilon}{\textlengthmark}{\r{g}}{\textscy}{\textsubring{m}}} \dicPos{v} \dicFlx{ind pf pl 1 pers} \dicLink{leika}
\dicEntry[lélega] \dicTerm{lé··lega} \dicIPA{{l}{j}{\textepsilon}{\textlengthmark}{l}{\textepsilon}{\textbabygamma}{a}} \dicPos{adv} \dicDirectTranslationCS{špatně, bídně, mizerně}
\dicEntry[lélegur] \dicTerm{lé··legur} \dicsymFrequent\  \dicIPA{{l}{j}{\textepsilon}{\textlengthmark}{l}{\textepsilon}{\textbabygamma}{\textscy}{\textsubring{r}}} \dicPos{adj}[1]\dicFlx{}[-8] \dicSynonym{slæmur} \dicDirectTranslationCS{špatný, bídný, mizerný, chabý} \dicExampleIS{benda á lélegan árangur} \dicExampleCS{upozornit na špatný výsledek}
\dicEntry[lén] \dicTerm{lén} \dicIPA{{l}{j}{\textepsilon}{\textlengthmark}{\textsubring{n}}} \dicPos{n}[2] \dicFlx{(‑s, ‑)}[5] \textbf{1.} \dicSynonym{ríki} \dicDirectTranslationCS{léno}  \textbf{2.} \dicFieldCat{poč.} \dicSynonym{umdæmi} \dicDirectTranslationCS{doména}  \textbf{3.} \dicFieldCat{biol.} \dicDirectTranslationCS{doména}
\dicEntry[lénskur] \dicTerm{lénskur} \dicIPA{{l}{j}{\textepsilon}{n}{s}{\r{g}}{\textscy}{\textsubring{r}}} \dicPos{adj}[1]\dicFlx{}[-1] \dicDirectTranslationCS{feudální}
\dicEntry[lénsmaður] \dicTerm{léns··|maður} \dicIPA{{l}{j}{\textepsilon}{n}{s}{m}{a}{ð}{\textscy}{\textsubring{r}}} \dicPos{m}[13] \dicFlx{(‑manns, ‑menn)}[2] \dicDirectTranslationCS{leník, vazal(ka)}
\dicEntry[léreft] \dicTerm{léreft} \dicIPA{{l}{j}{\textepsilon}{\textlengthmark}{r}{\textepsilon}{f}{\textsubring{d}}} \dicPos{n}[2] \dicFlx{(‑s, ‑)}[5] \dicDirectTranslationCS{(lněné) plátno}
\dicEntry[lét] \dicTerm{lét} \dicIPA{{l}{j}{\textepsilon}{\textlengthmark}{\textsubring{d}}} \dicPos{v} \dicFlx{ind pf sg 1 pers} \dicLink{láta}
\dicEntry[léti] \dicTerm{léti} \dicIPA{{l}{j}{\textepsilon}{\textlengthmark}{\textsubring{d}}{\textsci}} \dicPos{v} \dicFlx{con pf sg 1 pers} \dicLink{láta}
\dicEntry[létt] \dicTerm{létt} \dicsymFrequent\  \dicIPA{{l}{j}{\textepsilon}{h}{\textsubring{d}}} \dicPos{adv} \dicFlx{(comp ‑ar, sup ‑ast)} \dicDirectTranslationCS{lehce, mírně, jemně, zlehka} \dicExampleIS{Hann klappaði létt á bakið á mér.} \dicExampleCS{Lehce mě poklepal po zádech.}
\dicEntry[létta] \dicTerm{létt|a} \dicsymFrequent\  \dicIPA{{l}{j}{\textepsilon}{h}{\textsubring{d}}{a}} \dicPos{v}[2] \dicFlx{(‑i, ‑)}[2] \dicFlx{acc\,/\addthin dat} \textbf{1.} \dicDirectTranslationCS{odlehčit, odlehčovat, ulehčit, ulehčovat, (u)činit lehčím} \dicExampleIS{létta byrðina} \dicExampleCS{odlehčit břemeno}  \textbf{2.} \dicPhraseIS{e‑m léttir} \dicFlx{impers} \dicDirectTranslationCS{(komu) se ulehčuje, (komu) se ulevuje}  \textbf{3.} \dicPhraseIS{e‑u léttir} \dicFlx{impers} \dicDirectTranslationCS{(čeho) ubývá (mlhy ap.), (co) mizí};  \dicIdiom{létta}[af]{ \dicPhraseIS{létta af e‑m e‑u}} \dicDirectTranslationCS{odlehčit (komu) od (čeho)}; { \dicPhraseIS{það léttir af e‑m e‑u}} \dicFlx{impers} \dicDirectTranslationCS{(komu) se ulevuje od (čeho)};  \dicIdiom{létta}[á]{ \dicPhraseIS{létta á sér}} \dicDirectTranslationCS{ulevit si} \dicIndirectTranslationCS{(vykonat tělesnou potřebu)};  \dicIdiom{létta}[sér]{ \dicPhraseIS{létta sér lífið}} \dicDirectTranslationCS{pobavit se, povyrazit si};  \dicIdiom{létta}[til]{ \dicPhraseIS{létta til}} \dicDirectTranslationCS{vyčasit se, vyjasnit se, vyjasňovat se} \dicExampleIS{Það er að létta til.} \dicExampleCS{Začíná se vyjasňovat.};  \dicIdiom{létta}[undir]{ \dicPhraseIS{létta undir með e‑m}} \dicDirectTranslationCS{ulehčit (komu), učinit to (komu) lehčím};  \dicIdiom{léttast}{ \dicPhraseIS{léttast}} \dicFlx{refl} \dicDirectTranslationCS{(z)hubnout};  \dicIdiom{létta}{ \dicPhraseIS{létta akkerum}} \dicFieldCat{nám.} \dicDirectTranslationCS{zvednout kotvu}
\dicEntry[léttasótt] \dicTerm{létta··sótt} \dicIPA{{l}{j}{\textepsilon}{h}{\textsubring{d}}{a}{s}{ou}{h}{\textsubring{d}}} \dicPos{f}[7] \dicFlx{(‑ar)}[3] \dicDirectTranslationCS{porodní bolesti, kontrakce} \dicExampleIS{taka léttasóttina} \dicExampleCS{mít porodní bolesti}
\dicEntry[létteind] \dicTerm{létt··eind} \dicIPA{{l}{j}{\textepsilon}{h}{\textsubring{d}}{ei}{n}{\textsubring{d}}} \dicPos{f}[7] \dicFlx{(‑ar, ‑ir)}[1] \dicFieldCat{fyz.} \dicDirectTranslationCS{lepton}
\dicEntry[létti] \dicTerm{létt|i} \dicIPA{{l}{j}{\textepsilon}{h}{\textsubring{d}}{\textsci}} \dicPos{m}[1] \dicFlx{(‑a, ‑ar)}[1] \textbf{1.} \dicSynonym{léttir} \dicDirectTranslationCS{ulehčení, úleva}  \textbf{2.} \dicSynonym{axlaband} \dicDirectTranslationCS{popruh, šle}  \textbf{3.} \dicSynonym*{fúsleiki} \dicDirectTranslationCS{ochota};  \dicPhraseIS{segja allt af létta} \dicDirectTranslationCS{ochotně všechno povědět}
\dicEntry[léttir] \dicTerm{létt|ir} \dicIPA{{l}{j}{\textepsilon}{h}{\textsubring{d}}{\textsci}{\textsubring{r}}} \dicPos{m}[7] \dicFlx{(‑is, ‑ar)}[1] \textbf{1.} \dicSynonym{huggun} \dicDirectTranslationCS{útěcha, úleva}  \textbf{2.} \dicSynonym{hjálp} \dicDirectTranslationCS{pomoc};  \dicPhraseIS{vera e‑m til léttis} \dicDirectTranslationCS{pomoci (komu), ulehčit (komu)}  \textbf{3.} \dicFieldCat{zool.} \dicDirectTranslationCS{delfín obecný} \textit{(l.~{\textLA{Delphinus delphis}})}  \dicsymPhoto\ 
\dicFigure{26462.jpg}{Léttir}{Léttir - Shane Anderson, Biolib, PD}
\dicEntry[léttklæddur] \dicTerm{létt··klæddur} \dicIPA{{l}{j}{\textepsilon}{h}{\textsubring{d}}{k\smash{\textsuperscript{h}}}{l}{a}{i}{\textsubring{d}}{\textscy}{\textsubring{r}}} \dicPos{adj}[2]\dicFlx{}[-18] \dicDirectTranslationCS{lehce oblečený}
\dicEntry[léttleiki] \dicTerm{létt··leik|i} \dicIPA{{l}{j}{\textepsilon}{h}{\textsubring{d}}{l}{ei}{\r{\textObardotlessj}}{\textsci}} \dicPos{m}[1] \dicFlx{(‑a)}[3] \dicDirectTranslationCS{lehkost} \dicExampleIS{óbærilegur léttleiki tilverunnar} \dicExampleCS{nesnesitelná lehkost bytí}
\dicEntry[léttlyndi] \dicTerm{létt··lyndi} \dicIPA{{l}{j}{\textepsilon}{h}{\textsubring{d}}{l}{\textsci}{n}{\textsubring{d}}{\textsci}} \dicPos{n}[2] \dicFlx{(‑s)}[20] \dicDirectTranslationCS{dobrá nálada, uvolněnost, bezstarostnost}
\dicEntry[léttlyndur] \dicTerm{létt··lyndur} \dicIPA{{l}{j}{\textepsilon}{h}{\textsubring{d}}{l}{\textsci}{n}{\textsubring{d}}{\textscy}{\textsubring{r}}} \dicPos{adj}[2]\dicFlx{}[-14] \dicDirectTranslationCS{uvolněný, bezstarostný}
\dicEntry[léttmjólk] \dicTerm{létt··mjólk} \dicIPA{{l}{j}{\textepsilon}{h}{\textsubring{d}}{m}{j}{ou}{\textsubring{l}}{\r{g}}} \dicPos{f}[10] \dicFlx{(‑ur)}[2] \dicDirectTranslationCS{nízkotučné mléko}
\dicEntry[léttskýjaður] \dicTerm{létt··skýj·|aður} \dicIPA{{l}{j}{\textepsilon}{h}{\textsubring{d}}{s}{\r{\textObardotlessj}}{i}{j}{a}{ð}{\textscy}{\textsubring{r}}} \dicPos{adj}[3] \dicFlx{(f ‑uð)}[3] \dicPhraseIS{það er léttskýjað} \dicDirectTranslationCS{je polojasno}
\dicEntry[léttur] \dicTerm{léttur} \dicsymFrequent\  \dicIPA{{l}{j}{\textepsilon}{h}{\textsubring{d}}{\textscy}{\textsubring{r}}} \dicPos{adj}[1]\dicFlx{}[-10] \textbf{1.} \dicDirectTranslationCS{lehký, snadný} \dicExampleIS{létt verk} \dicExampleCS{lehká práce} \dicAntonym{erfiður}  \textbf{2.} \dicDirectTranslationCS{lehký} \dicIndirectTranslationCS{(vážící málo)} \dicAntonym{þungur}  \textbf{3.} \dicSynonym{glaður} \dicDirectTranslationCS{uvolněný, klidný, bezstarostný} \dicExampleIS{Hún er létt í lund.} \dicExampleCS{Je bezstarostná.}  \textbf{4.} \dicDirectTranslationCS{dietní, lehký (strava ap.)} \dicExampleIS{léttar veitingar} \dicExampleCS{lehké občerstvení}  \textbf{5.} \dicDirectTranslationCS{lehký, slabý (alkohol ap.)} \dicExampleIS{létt vín} \dicExampleCS{slabý alkohol}
\dicEntry[léttúð] \dicTerm{létt··úð} \dicIPA{{l}{j}{\textepsilon}{h}{\textsubring{d}}{u}{\texttheta}} \dicPos{f}[4] \dicFlx{(‑ar)}[3] \textbf{1.} \dicSynonym{alvöruleysi} \dicDirectTranslationCS{lehkovážnost}  \textbf{2.} \dicSynonym{lauslæti} \dicDirectTranslationCS{rozpustilost, frivolnost}
\dicEntry[léttúðugur] \dicTerm{létt·úð··ugur} \dicIPA{{l}{j}{\textepsilon}{h}{\textsubring{d}}{u}{ð}{\textscy}{\textbabygamma}{\textscy}{\textsubring{r}}} \dicPos{adj}[1]\dicFlx{}[-8] \dicSynonym{lauslátur} \dicDirectTranslationCS{frivolní, prostopášný}
\dicEntry[léttvín] \dicTerm{létt··vín} \dicIPA{{l}{j}{\textepsilon}{h}{\textsubring{d}}{v}{i}{\textsubring{n}}} \dicPos{n}[2] \dicFlx{(‑s, ‑)}[5] \dicDirectTranslationCS{víno}
\dicEntry[léttvægur] \dicTerm{létt··vægur} \dicIPA{{l}{j}{\textepsilon}{h}{\textsubring{d}}{v}{a}{i}{\textbabygamma}{\textscy}{\textsubring{r}}} \dicPos{adj}[1]\dicFlx{}[-1] \dicDirectTranslationCS{nedůležitý, nepodstat\-ný, nevýznamný}
\dicEntry[létum] \dicTerm{létum} \dicIPA{{l}{j}{\textepsilon}{\textlengthmark}{\textsubring{d}}{\textscy}{\textsubring{m}}} \dicPos{v} \dicFlx{ind pf pl 1 pers} \dicLink{láta}
\dicEntry[lh. nt.] \dicTerm{lh. nt.} \dicPos{zkr} \dicPhraseIS{lýsingarháttur nútíðar} \dicFieldCat{jaz.} \dicDirectTranslationCS{příčestí přítomné}
\dicEntry[lh. þt.] \dicTerm{lh. þt.} \dicPos{zkr} \dicPhraseIS{lýsingarháttur þátíðar} \dicFieldCat{jaz.} \dicDirectTranslationCS{příčestí minulé}
\dicEntry[lið] \dicTerm{lið} \dicsymFrequent\  \dicIPA{{l}{\textsci}{\textlengthmark}{\texttheta}} \dicPos{n}[2] \dicFlx{(‑s, ‑)}[5] \textbf{1.} \dicSynonym{hópur} \dicDirectTranslationCS{družstvo, oddíl, četa};  \dicPhraseIS{ganga í lið við e‑n} \dicDirectTranslationCS{spojit síly s~(kým)};  \dicPhraseIS{safna liði} \dicDirectTranslationCS{shromažďovat podporu}  \textbf{2.} \dicFieldCat{sport.} \dicDirectTranslationCS{družina, tým, mužstvo}  \textbf{3.} \dicSynonym{aðstoð} \dicDirectTranslationCS{výpomoc, pomocná ruka};  \dicPhraseIS{koma til liðs við e‑n} \dicDirectTranslationCS{vypomoct (komu), přijít (komu) na pomoc};  \dicPhraseIS{verða e‑m að liði} \dicDirectTranslationCS{vypomoct (komu), asistovat (komu)}  \textbf{4.} \dicLangCat{hovor.} \dicSynonym{fólk} \dicDirectTranslationCS{lidi}
\dicEntry[liða] \dicTerm{lið|a} \dicIPA{{l}{\textsci}{\textlengthmark}{ð}{a}} \dicPos{v}[1] \dicFlx{(‑aði)}[1] \dicFlx{acc} \textbf{1.} \dicSynonym{skipta} \dicDirectTranslationCS{(roz)dělit (do skupin)}  \textbf{2.} \dicSynonym*{gera hrokkið} \dicDirectTranslationCS{natočit, natáčet, (na)kadeřit (vlasy ap.)} \dicExampleIS{liða hár} \dicExampleCS{natáčet vlasy};  \dicIdiom{liðast}{ \dicPhraseIS{liðast}} \dicFlx{refl} \dicDirectTranslationCS{vlnit se, kroutit se, kadeřit se}
\dicEntry[liðaður] \dicTerm{lið··|aður} \dicIPA{{l}{\textsci}{\textlengthmark}{ð}{a}{ð}{\textscy}{\textsubring{r}}} \dicPos{adj}[3] \dicFlx{(f ‑uð)}[3] \dicSynonym*{bylgjaður} \dicDirectTranslationCS{natočený, nakadeřený, navlněný}
\dicEntry[liðagigt] \dicTerm{liða··gigt} \dicIPA{{l}{\textsci}{\textlengthmark}{ð}{a}{\r{\textObardotlessj}}{\textsci}{x}{\textsubring{d}}} \dicPos{f}[4] \dicFlx{(‑ar)}[3] \dicFieldCat{med.} \dicDirectTranslationCS{revmatoidní artritida}
\dicEntry[liðamót] \dicTerm{liða··mót} \dicIPA{{l}{\textsci}{\textlengthmark}{ð}{a}{m}{ou}{\textsubring{d}}} \dicPos{n}[2] \dicFlx{pl}[1] \dicFieldCat{anat.} \dicDirectTranslationCS{kloub}
\dicEntry[liðband] \dicTerm{lið··|band} \dicIPA{{l}{\textsci}{ð}{\textsubring{b}}{a}{n}{\textsubring{d}}} \dicPos{n}[2] \dicFlx{(‑bands, ‑bönd)}[8] \dicFieldCat{anat.} \dicDirectTranslationCS{vazivo}
\dicEntry[liðdýr] \dicTerm{lið··dýr} \dicIPA{{l}{\textsci}{ð}{\textsubring{d}}{i}{\textsubring{r}}} \dicPos{n}[2] \dicFlx{(‑s, ‑)}[5] \dicFieldCat{zool.} \dicDirectTranslationCS{členovec} \textit{(l.~{\textLA{Arthropoda}})}
\dicEntry[liðfall] \dicTerm{lið··|fall} \dicIPA{{l}{\textsci}{ð}{f}{a}{\textsubring{d}}{\textsubring{l}}} \dicPos{n}[2] \dicFlx{(‑falls, ‑föll)}[8] \dicFieldCat{jaz.} \dicSynonym{brottfall} \dicDirectTranslationCS{elipsa, výpustka}
\dicEntry[liðfelldur] \dicTerm{lið··felldur} \dicIPA{{l}{\textsci}{ð}{f}{\textepsilon}{l}{\textsubring{d}}{\textscy}{\textsubring{r}}} \dicPos{adj}[2]\dicFlx{}[-14] \dicFieldCat{jaz.} \dicDirectTranslationCS{eliptický} \dicExampleIS{liðfellt orðtak} \dicExampleCS{eliptický výraz}
\dicEntry[liðhlaupi] \dicTerm{lið··hlaup|i} \dicIPA{{l}{\textsci}{ð}{\textsubring{l}}{\oe i}{\textsubring{b}}{\textsci}} \dicPos{m}[1] \dicFlx{(‑a, ‑ar)}[1] \dicFieldCat{voj.} \dicDirectTranslationCS{zběh, dezertér(ka)}
\dicEntry[liði] \dicTerm{lið|i\smash{\textsuperscript{1}}} \dicIPA{{l}{\textsci}{\textlengthmark}{ð}{\textsci}} \dicPos{m}[1] \dicFlx{(‑a, ‑ar)}[1] \dicDirectTranslationCS{člen(ka) týmu (vojenského, zdravotního ap.)} \dicIndirectTranslationCS{(vyskytuje se jako druhá část složeného slova)} \dicExampleIS{skæruliði} \dicExampleCS{partyzán}
\dicEntry[liði] \dicTerm{liði\smash{\textsuperscript{2}}} \dicIPA{{l}{\textsci}{\textlengthmark}{ð}{\textsci}} \dicPos{v} \dicFlx{con pf sg 1 pers} \dicLink{líða}
\dicEntry[liðið] \dicTerm{liðið} \dicIPA{{l}{\textsci}{\textlengthmark}{ð}{\textsci}{\texttheta}} \dicPos{v} \dicFlx{supin} \dicLink{líða}
\dicEntry[liðinn] \dicTerm{liðinn} \dicsymFrequent\  \dicIPA{{l}{\textsci}{\textlengthmark}{ð}{\textsci}{\textsubring{n}}} \dicPos{adj}[6]\dicFlx{}[-6] \textbf{1.} \dicDirectTranslationCS{minulý, uplynulý} \dicExampleIS{í liðnum mánuði} \dicExampleCS{v~minulém měsíci}  \textbf{2.} \dicDirectTranslationCS{zesnulý};  \dicPhraseIS{vera vel liðinn} \dicDirectTranslationCS{být populární\,/\addthin oblíbený}
\dicEntry[liðka] \dicTerm{liðk|a} \dicIPA{{l}{\textsci}{\texttheta}{\r{g}}{a}} \dicPos{v}[1] \dicFlx{(‑aði)}[1] \dicFlx{acc} \dicDirectTranslationCS{rozhýbat, rozpohybovat} \dicExampleIS{liðka handlegginn} \dicExampleCS{rozhýbat ruku};  \dicPhraseIS{liðka sig} \dicDirectTranslationCS{rozhýbat se};  \dicIdiom{liðka}[til]{ \dicPhraseIS{liðka e‑ð til}} \dicDirectTranslationCS{usnadnit (co)};  \dicIdiom{liðkast}{ \dicPhraseIS{liðkast}} \dicFlx{refl} \dicDirectTranslationCS{rozhýbat se}
\dicEntry[liðlangur] \dicTerm{lið··|langur} \dicIPA{{l}{\textsci}{ð}{l}{au}{\ng}{\r{g}}{\textscy}{\textsubring{r}}} \dicPos{adj}[10] \dicFlx{(f ‑löng, comp ‑lengri, sup ‑leng\-st\-ur)}[2] \dicPhraseIS{allan liðlangan daginn} \dicFlx{adv} \dicDirectTranslationCS{(po) celý celičký den}
\dicEntry[liðlegur] \dicTerm{lið··legur} \dicIPA{{l}{\textsci}{ð}{l}{\textepsilon}{\textbabygamma}{\textscy}{\textsubring{r}}} \dicPos{adj}[1]\dicFlx{}[-8] \textbf{1.} \dicSynonym{fimur} \dicDirectTranslationCS{obratný}  \textbf{2.} \dicSynonym{hjálpsamur} \dicDirectTranslationCS{vstřícný, úslužný, ochotný pomoci}
\dicEntry[liðleiki] \dicTerm{lið··leik|i} \dicIPA{{l}{\textsci}{ð}{l}{ei}{\r{\textObardotlessj}}{\textsci}} \dicPos{m}[1] \dicFlx{(‑a)}[3] \textbf{1.} \dicDirectTranslationCS{pohyblivost, obratnost}  \textbf{2.} \dicDirectTranslationCS{vstřícnost, ochota pomoci}
\dicEntry[liðsauki] \dicTerm{liðs··auk|i} \dicIPA{{l}{\textsci}{ð}{s}{\oe i}{\r{\textObardotlessj}}{\textsci}} \dicPos{m}[1] \dicFlx{(‑a)}[3] \dicDirectTranslationCS{posila, posílení\,/\addthin rozšíření řad}
\dicEntry[liðsemd] \dicTerm{lið··semd} \dicIPA{{l}{\textsci}{ð}{s}{\textepsilon}{m}{\textsubring{d}}} \dicPos{f}[7] \dicFlx{(‑ar)}[3] \dicSynonym{aðstoð} \dicDirectTranslationCS{výpomoc, asistence}
\dicEntry[liðsforingi] \dicTerm{liðs··for·ing|i} \dicIPA{{l}{\textsci}{ð}{s}{f}{\textopeno}{r}{i}{\textltailn}{\r{\textObardotlessj}}{\textsci}} \dicPos{m}[1] \dicFlx{(‑ja, ‑jar)}[14] \dicFieldCat{voj.} \dicDirectTranslationCS{poručík, poručice}
\dicEntry[liðsinna] \dicTerm{lið··sinn|a} \dicIPA{{l}{\textsci}{ð}{s}{\textsci}{n}{a}} \dicPos{v}[2] \dicFlx{(‑ti, ‑t)}[78] \dicFlx{dat} \dicSynonym{hjálpa} \dicDirectTranslationCS{asistovat, (vy)pomáhat} \dicExampleIS{liðsinna e‑m við e‑ð} \dicExampleCS{asistovat (komu) při (čem)}
\dicEntry[liðsinni] \dicTerm{lið··sinni} \dicIPA{{l}{\textsci}{ð}{s}{\textsci}{n}{\textsci}} \dicPos{n}[2] \dicFlx{(‑s)}[20] \dicSynonym{aðstoð} \dicDirectTranslationCS{asistence, (vý)pomoc}
\dicEntry[liðsmaður] \dicTerm{liðs··|maður} \dicIPA{{l}{\textsci}{ð}{s}{m}{a}{ð}{\textscy}{\textsubring{r}}} \dicPos{m}[13] \dicFlx{(‑manns, ‑menn)}[2] \textbf{1.} \dicSynonym{hermaður} \dicDirectTranslationCS{voják(yně)}  \textbf{2.} \dicSynonym{stuðningsmaður} \dicDirectTranslationCS{přívrženec, přívrženkyně, stoupenec, stoupenkyně}  \textbf{3.} \dicDirectTranslationCS{(spolu)hráč(ka), člen(ka) družiny\,/\addthin týmu}
\dicEntry[liðtækur] \dicTerm{lið··tækur} \dicIPA{{l}{\textsci}{ð}{t\smash{\textsuperscript{h}}}{a}{i}{\r{g}}{\textscy}{\textsubring{r}}} \dicPos{adj}[1]\dicFlx{}[-6] \dicSynonym{hæfur} \dicDirectTranslationCS{kompetentní, způsobilý}
\dicEntry[liðugur] \dicTerm{lið··ugur} \dicIPA{{l}{\textsci}{\textlengthmark}{ð}{\textscy}{\textbabygamma}{\textscy}{\textsubring{r}}} \dicPos{adj}[1]\dicFlx{}[-8] \dicSynonym{fimur} \dicDirectTranslationCS{hbitý, mrštný} \dicExampleIS{vera liðugur í hreyfingum} \dicExampleCS{pohybovat se hbitě};  \dicPhraseIS{vera laus og liðugur} \dicDirectTranslationCS{být volný, užívat si svobody}
\dicEntry[liðum] \dicTerm{liðum} \dicIPA{{l}{\textsci}{\textlengthmark}{ð}{\textscy}{\textsubring{m}}} \dicPos{v} \dicFlx{ind pf pl 1 pers} \dicLink{líða}
\dicEntry[liður] \dicTerm{lið|ur} \dicsymFrequent\  \dicIPA{{l}{\textsci}{\textlengthmark}{ð}{\textscy}{\textsubring{r}}} \dicPos{m}[10] \dicFlx{(‑ar\,/\addthin ‑s, ‑ir)}[37] \textbf{1.} \dicFieldCat{anat.} \dicSynonym{liðamót} \dicDirectTranslationCS{kloub} \dicExampleIS{axlarliður} \dicExampleCS{ramenní kloub};  \dicPhraseIS{geta hvorki hrært legg né lið} \dicLangCat{přen.} \dicDirectTranslationCS{nemoci se vůbec pohnout}  \textbf{2.} \dicSynonym{bylgja} \dicDirectTranslationCS{kudrna, vlnka (ve vlasech)} \dicExampleIS{liður í hári} \dicExampleCS{kudrna ve vlasech}  \textbf{3.} \dicSynonym{atriði} \dicDirectTranslationCS{položka, bod} \dicExampleIS{Þessi liður er rangur.} \dicExampleCS{Tato položka je špatně.}  \textbf{4.} \dicSynonym{ættliður} \dicDirectTranslationCS{koleno, pokolení, generace}  \textbf{5.} \dicFieldCat{mat.} \dicDirectTranslationCS{člen}
\dicEntry[liðveisla] \dicTerm{lið··veisl|a} \dicIPA{{l}{\textsci}{ð}{v}{ei}{s}{\textsubring{d}}{l}{a}} \dicPos{f}[1] \dicFlx{(‑u, ‑ur)}[13] \dicDirectTranslationCS{asistence, (vý)pomoc}
\dicEntry[liðþjálfi] \dicTerm{lið··þjálf|i} \dicIPA{{l}{\textsci}{ð}{\texttheta}{j}{au}{l}{v}{\textsci}} \dicPos{m}[1] \dicFlx{(‑a, ‑ar)}[1] \dicFieldCat{voj.} \dicDirectTranslationCS{seržant(ka)}
\dicEntry[Liechtenstein] \dicTerm{Liechtenstein} \dicIPA{{l}{i}{x}{\textsubring{d}}{\textepsilon}{n}{s}{\textsubring{d}}{a}{i}{\textsubring{n}}} \dicPos{n}[4] \dicFlx{indecl}[2] \dicFieldCat{geog.} \dicDirectTranslationCS{Lichtenštejnsko}
\dicEntry[Liechtensteini] \dicTerm{Liechtenstein|i} \dicIPA{{l}{i}{x}{\textsubring{d}}{\textepsilon}{n}{s}{\textsubring{d}}{a}{i}{n}{\textsci}} \dicPos{m}[1] \dicFlx{(‑a, ‑ar)}[1] \dicDirectTranslationCS{Lichtenštejňan(ka), Lichtenštejnec, Lichtenštejnka}
\dicEntry[liechtensteinskur] \dicTerm{liechtensteinskur} \dicIPA{{l}{i}{x}{\textsubring{d}}{\textepsilon}{n}{s}{\textsubring{d}}{a}{i}{n}{s}{\r{g}}{\textscy}{\textsubring{r}}} \dicPos{adj}[1]\dicFlx{}[-6] \dicDirectTranslationCS{lichtenštejnský}
\dicEntry[lifa] \dicTerm{lif|a} \dicsymFrequent\  \dicIPA{{l}{\textsci}{\textlengthmark}{v}{a}} \dicPos{v}[2] \dicFlx{(‑ði, ‑að)}[124] \dicFlx{acc} \textbf{1.} \dicDirectTranslationCS{žít} \dicExampleIS{lifa langa ævi} \dicExampleCS{žít dlouhý život}  \textbf{2.} \dicDirectTranslationCS{existovat}  \textbf{3.} \dicDirectTranslationCS{zažít, prožít} \dicExampleIS{lifa þessa atburði} \dicExampleCS{zažít tyto události}  \textbf{4.} \dicSynonym*{vera logandi} \dicDirectTranslationCS{planout, svítit};  \dicIdiom{lifa}[af]{ \dicPhraseIS{lifa af}} \dicSynonym{endast} \dicDirectTranslationCS{přežít}; { \dicPhraseIS{lifa af e‑u}} \dicDirectTranslationCS{vyžít z~(čeho)};  \dicIdiom{lifa}[á]{ \dicPhraseIS{lifa á e‑u}} \dicDirectTranslationCS{živit se (čím)};  \dicIdiom{lifa}[eftir]{ \dicPhraseIS{lifa eftir e‑u}} \dicDirectTranslationCS{žít podle (čeho), žít v~souladu s~(čím)};  \dicIdiom{lifa}[fyrir]{ \dicPhraseIS{lifa fyrir e‑n}} \dicDirectTranslationCS{žít pro (koho)};  \dicIdiom{lifa}[inn í]{ \dicPhraseIS{lifa sig inn í e‑ð}} \dicDirectTranslationCS{vžít se do (čeho)} \dicExampleIS{lifa sig inn í starfið} \dicExampleCS{vžít se do práce};  \dicIdiom{lifa}[við]{ \dicPhraseIS{lifa við e‑ð}} \dicDirectTranslationCS{žít s~(čím) (pocitem viny ap.)}
\dicEntry[lifandi] \dicTerm{lif··|andi\smash{\textsuperscript{1}}} \dicIPA{{l}{\textsci}{\textlengthmark}{v}{a}{n}{\textsubring{d}}{\textsci}} \dicPos{m}[2] \dicFlx{(‑anda, ‑endur)}[1] \dicDirectTranslationCS{živý člověk, živá bytost} \dicIndirectTranslationCS{(zvláště v~pl)}
\dicEntry[lifandi] \dicTerm{lif··andi\smash{\textsuperscript{2}}} \dicsymFrequent\  \dicIPA{{l}{\textsci}{\textlengthmark}{v}{a}{n}{\textsubring{d}}{\textsci}} \dicPos{adj}[13] \dicFlx{indecl}[1] \textbf{1.} \dicSynonym*{á lífi} \dicDirectTranslationCS{živý, (jsoucí) naživu} \dicExampleIS{ná birninum lifandi} \dicExampleCS{chytit medvěda živého}  \textbf{2.} \dicSynonym{líflegur} \dicDirectTranslationCS{živý, čilý};  \dicIdiom{lifandi}{ \dicPhraseIS{ekki nokkur\,/\addthin enginn lifandi maður}} \dicLangCat{přen.} \dicDirectTranslationCS{ani živáček};  \dicPhraseIS{lifandi tónlist} \dicDirectTranslationCS{živá hudba}
\dicEntry[lifna] \dicTerm{lifn|a} \dicsymFrequent\  \dicIPA{{l}{\textsci}{\textsubring{b}}{n}{a}} \dicPos{v}[1] \dicFlx{(‑aði)}[1] \textbf{1.} \dicDirectTranslationCS{oživit se, rozhořet se}  \textbf{2.} \dicPhraseIS{lifna við} \dicSynonym*{vakna til lífs} \dicDirectTranslationCS{oživnout, ožít} \dicExampleIS{Fjöllin lifnuðu við eftir fimmtíu ár.} \dicExampleCS{Hory ožily po padesáti letech.}  \textbf{3.} \dicPhraseIS{það lifnar yfir e‑m} \dicFlx{impers} \dicLangCat{přen.} \dicDirectTranslationCS{(kdo) ožívá, (kdo) se stává živějším}
\dicEntry[lifnaður] \dicTerm{lif··nað|ur} \dicIPA{{l}{\textsci}{\textsubring{b}}{n}{a}{ð}{\textscy}{\textsubring{r}}} \dicPos{m}[10] \dicFlx{(‑ar)}[7] \dicSynonym{líferni} \dicDirectTranslationCS{způsob života, životní styl}
\dicEntry[lifrarkæfa] \dicTerm{lifrar··kæf|a} \dicIPA{{l}{\textsci}{v}{r}{a}{\textsubring{r}}{c\smash{\textsuperscript{h}}}{a}{i}{v}{a}} \dicPos{f}[1] \dicFlx{(‑u)}[5] \dicFieldCat{kulin.} \dicDirectTranslationCS{játrová paštika, játrovka}
\dicEntry[lifrarpylsa] \dicTerm{lifrar··pyls|a} \dicIPA{{l}{\textsci}{v}{r}{a}{\textsubring{r}}{p\smash{\textsuperscript{h}}}{\textsci}{l}{s}{a}} \dicPos{f}[1] \dicFlx{(‑u, ‑ur)}[13] \dicFieldCat{kulin.} \dicDirectTranslationCS{jitrnice}
\dicEntry[lifur] \dicTerm{lif|ur} \dicIPA{{l}{\textsci}{\textlengthmark}{v}{\textscy}{\textsubring{r}}} \dicPos{f}[6] \dicFlx{(‑rar, ‑rar)}[4] \dicFieldCat{anat.} \dicDirectTranslationCS{játra}
\dicEntry[ligg] \dicTerm{ligg} \dicIPA{{l}{\textsci}{\r{g}}{\textlengthmark}} \dicPos{v} \dicFlx{ind praes sg 1 pers} \dicLink{liggja}
\dicEntry[liggja] \dicTerm{liggja} \dicsymFrequent\  \dicIPA{{l}{\textsci}{\r{\textObardotlessj}}{\textlengthmark}{a}} \dicPos{v}[7] \dicFlx{(ligg, lá, lágum, lægi, legið)}[9] \textbf{1.} \dicDirectTranslationCS{ležet} \dicIndirectTranslationCS{(spočívat na něčem podélně)} \dicExampleIS{Blaðið liggur á gólfinu.} \dicExampleCS{Noviny leží na podlaze.}  \textbf{2.} \dicDirectTranslationCS{ležet} \dicIndirectTranslationCS{(být uložen ke spánku)}  \textbf{3.} \dicDirectTranslationCS{ležet, rozkládat se, prostírat se} \dicExampleIS{Vegurinn liggur meðfram ánni.} \dicExampleCS{Cesta vede podél řeky.};  \dicIdiom{liggja}[að]{ \dicPhraseIS{liggja að e‑u}} \dicDirectTranslationCS{sousedit s~(čím)};  \dicIdiom{liggja}[á]{ \dicPhraseIS{ekkert liggur á}} \dicDirectTranslationCS{všechno popořadě, vše ve správný čas}; { \dicPhraseIS{e‑m liggur á}} \dicFlx{impers} \dicDirectTranslationCS{(kdo) má naspěch, (kdo) pospíchá, (kdo) chvátá} \dicExampleIS{Honum lá á.} \dicExampleCS{Měl naspěch.}; { \dicPhraseIS{e‑m liggur á e‑u}} \dicFlx{impers} \dicDirectTranslationCS{(kdo) má naspěch s~(čím), (komu) záleží na (čem)} \dicExampleIS{Honum liggur mikið á sendingunni.} \dicExampleCS{Na té zásilce mu moc záleží.}; { \dicPhraseIS{liggja á eggjum}} \dicDirectTranslationCS{sedět na vejcích}; { \dicPhraseIS{það liggur vel á e‑m}} \dicFlx{impers} \dicDirectTranslationCS{(kdo) se dobře cítí, (kdo) je v~dobrém rozpoložení};  \dicIdiom{liggja}[fyrir]{ \dicPhraseIS{e‑að á fyrir e‑m að liggja}} \dicDirectTranslationCS{(co) na (koho) čeká, (co) má (kdo) před sebou}; { \dicPhraseIS{e‑að liggur fyrir e‑m}} \dicDirectTranslationCS{(co) na (koho) čeká, (co) leží před (kým) (rozhodnutí ap.)}; { \dicPhraseIS{liggja fyrir}} \dicDirectTranslationCS{ležet v~posteli};  \dicIdiom{liggja}[í]{ \dicPhraseIS{liggja í e‑u}} \dicDirectTranslationCS{ležet s~(čím) (chřipkou ap.)} \dicExampleIS{liggja í flensu} \dicExampleCS{ležet s~chřipkou}; { \dicPhraseIS{e‑að liggur í loftinu}} \dicLangCat{přen.} \dicDirectTranslationCS{(co) visí ve vzduchu, (co) se vznáší ve vzduchu (napětí ap.)}; { \dicPhraseIS{liggja í því}} \dicDirectTranslationCS{prohrát, dostat nakládačku};  \dicIdiom{liggja}[með]{ \dicPhraseIS{liggja með e‑m}} \dicDirectTranslationCS{spát s~(kým)};  \dicIdiom{liggja}[nærri]{ \dicPhraseIS{það lá nærri e‑u}} \dicFlx{impers} \dicDirectTranslationCS{schylovalo se k~(čemu), (co) bylo na spadnutí};  \dicIdiom{liggja}[undir]{ \dicPhraseIS{liggja undir grun}} \dicDirectTranslationCS{čelit podezření, být v~podezření};  \dicIdiom{liggja}[úti]{ \dicPhraseIS{liggja úti}} \dicDirectTranslationCS{spát venku, spát pod širým nebem};  \dicIdiom{liggja}[við]{ \dicPhraseIS{e‑m liggur við e‑u}} \dicFlx{impers} \dicDirectTranslationCS{(kdo) málem (co) (upadl ap.)}; { \dicPhraseIS{e‑að liggur beint við}} \dicDirectTranslationCS{(co) je přirozené}; { \dicPhraseIS{ef\,/\addthin þegar mikið liggur við}} \dicDirectTranslationCS{pokud to je nutné}; { \dicPhraseIS{það lá við e‑u}} \dicFlx{impers} \dicDirectTranslationCS{schylovalo se k~(čemu)};  \dicIdiom{liggja}[yfir]{ \dicPhraseIS{liggja yfir e‑u}} \dicDirectTranslationCS{ležet nad (čím), ležet v~(čem) (v~učení ap.)};  \dicIdiom{liggja}{ \dicPhraseIS{liggja í augum uppi}} \dicDirectTranslationCS{být jasný\,/\addthin zřejmý\,/\addthin očividný, bít do očí}
\dicEntry[lilja] \dicTerm{lilj|a} \dicIPA{{l}{\textsci}{l}{j}{a}} \dicPos{f}[1] \dicFlx{(‑u, ‑ur)}[7] \dicFieldCat{bot.} \dicDirectTranslationCS{lilie} \textit{(l.~{\textLA{Lilium}})}  \dicsymPhoto\ 
\dicFigure{1249.jpg}{Lilja}{Lilja - Zicha Ondřej, Biolib, Copyright/CC-BY-NC}
\dicEntry[lim] \dicTerm{lim} \dicIPA{{l}{\textsci}{\textlengthmark}{\textsubring{m}}} \dicPos{f}[4] \dicFlx{(‑ar, ‑ar)}[1] \dicDirectTranslationCS{listoví}
\dicEntry[limaburður] \dicTerm{lima··burð|ur} \dicIPA{{l}{\textsci}{\textlengthmark}{m}{a}{\textsubring{b}}{\textscy}{r}{ð}{\textscy}{\textsubring{r}}} \dicPos{m}[10] \dicFlx{(‑ar)}[7] \dicDirectTranslationCS{postoj, póza} \dicIndirectTranslationCS{(držení těla při stání)}
\dicEntry[limgerði] \dicTerm{lim··gerði} \dicIPA{{l}{\textsci}{m}{\r{\textObardotlessj}}{\textepsilon}{r}{ð}{\textsci}} \dicPos{n}[2] \dicFlx{(‑s, ‑)}[14] \dicDirectTranslationCS{živý plot}
\dicEntry[limlesta] \dicTerm{lim··lest|a} \dicIPA{{l}{\textsci}{m}{l}{\textepsilon}{s}{\textsubring{d}}{a}} \dicPos{v}[2] \dicFlx{(‑i, ‑)}[12] \dicFlx{acc} \dicDirectTranslationCS{zmrzačit, polámat kosti}
\dicEntry[limra] \dicTerm{limr|a} \dicIPA{{l}{\textsci}{m}{r}{a}} \dicPos{f}[1] \dicFlx{(‑u, ‑ur)}[7] \dicFieldCat{lit.} \dicDirectTranslationCS{limerik} \dicIndirectTranslationCS{(zábavný poetický útvar)}
\dicEntry[limur] \dicTerm{lim|ur} \dicIPA{{l}{\textsci}{\textlengthmark}{m}{\textscy}{\textsubring{r}}} \dicPos{m}[9] \dicFlx{(‑s, ‑ir)}[8] \textbf{1.} \dicSynonym{útlimur} \dicDirectTranslationCS{úd, končetina} \dicExampleIS{limir líkamans} \dicExampleCS{údy těla}  \textbf{2.} \dicSynonym{getnaðarlimur} \dicDirectTranslationCS{pyj}
\dicEntry[lina] \dicTerm{lin|a} \dicIPA{{l}{\textsci}{\textlengthmark}{n}{a}} \dicPos{v}[1] \dicFlx{(‑aði)}[1] \dicFlx{acc} \dicDirectTranslationCS{(z)mírnit, (z)tlumit, (z)měkčit} \dicExampleIS{Kærleikurinn linar sorg hennar.} \dicExampleCS{Láska mírní její žal.};  \dicIdiom{linast}{ \dicPhraseIS{linast}} \dicFlx{refl} {\textbf{a.}} \dicSynonym*{mýkjast} \dicDirectTranslationCS{(z)měknout} \dicIndirectTranslationCS{(stát\,/\addthin stávat se měkkým)};  {\textbf{b.}} \dicSynonym*{slá af fyrri skoðun} \dicDirectTranslationCS{obměkčit se, ustoupit};  {\textbf{c.}} \dicSynonym*{missa krafta} \dicDirectTranslationCS{(ze)slábnout, ztratit\,/\addthin ztrácet sílu}
\dicEntry[lind] \dicTerm{lind} \dicsymFrequent\  \dicIPA{{l}{\textsci}{n}{\textsubring{d}}} \dicPos{f}[7] \dicFlx{(‑ar, ‑ir)}[1] \textbf{1.} \dicSynonym{uppspretta} \dicDirectTranslationCS{zřídlo, pramen} \dicExampleIS{Hún vísaði þyrstum hermönnum á lind.} \dicExampleCS{Ukázala žíznivým vojákům cestu k~pramenu.}  \textbf{2.} \dicLangCat{přen.} \dicSynonym{upptök} \dicDirectTranslationCS{zřídlo, původ}  \textbf{3.} \dicFieldCat{bot.} \dicSynonym{linditré} \dicDirectTranslationCS{lípa} \textit{(l.~{\textLA{Tilia}})}  \dicsymPhoto\ 
\dicFigure{16367.jpg}{Lind}{Lind - Hanzlík Václav, Biolib, Copyright/CC-BY-NC}
\dicEntry[lindá] \dicTerm{lind··á} \dicIPA{{l}{\textsci}{n}{\textsubring{d}}{au}} \dicPos{f}[4] \dicFlx{(‑r, ‑r)}[18] \dicFieldCat{geol.} \dicDirectTranslationCS{řeka napájená zřídly}
\dicEntry[lindi] \dicTerm{lind|i} \dicIPA{{l}{\textsci}{n}{\textsubring{d}}{\textsci}} \dicPos{m}[1] \dicFlx{(‑a, ‑ar)}[1] \dicSynonym{band} \dicDirectTranslationCS{opasek, pás}
\dicEntry[linditré] \dicTerm{lindi··tré} \dicIPA{{l}{\textsci}{n}{\textsubring{d}}{\textsci}{t\smash{\textsuperscript{h}}}{r}{j}{\textepsilon}} \dicPos{n}[2] \dicFlx{(‑s, ‑)}[36] \dicFieldCat{bot.} \dicSynonym{lind} \dicDirectTranslationCS{lípa} \textit{(l.~{\textLA{Tilia}})}
\dicEntry[lindýr] \dicTerm{lin··dýr} \dicIPA{{l}{\textsci}{n}{\textsubring{d}}{i}{\textsubring{r}}} \dicPos{n}[2] \dicFlx{(‑s, ‑)}[5] \dicFieldCat{zool.} \dicDirectTranslationCS{měkkýš} \textit{(l.~{\textLA{Mollusca}})}
\dicEntry[lingerður] \dicTerm{lin··gerður} \dicIPA{{l}{\textsci}{n}{\r{\textObardotlessj}}{\textepsilon}{r}{ð}{\textscy}{\textsubring{r}}} \dicPos{adj}[2]\dicFlx{}[-1] \textbf{1.} \dicSynonym*{kveifarlegur} \dicDirectTranslationCS{změkčilý}  \textbf{2.} \dicSynonym{veikburða} \dicDirectTranslationCS{zesláblý, neduživý}
\dicEntry[linka] \dicTerm{link|a} \dicIPA{{l}{i}{\r{\ng}}{\r{g}}{a}} \dicPos{f}[1] \dicFlx{(‑u)}[5] \textbf{1.} \dicSynonym{lasleiki} \dicDirectTranslationCS{churavost, neduživost}  \textbf{2.} \dicSynonym{dugleysi} \dicDirectTranslationCS{neschopnost, nezpůsobilost}
\dicEntry[linmæli] \dicTerm{lin··mæli} \dicIPA{{l}{\textsci}{n}{m}{a}{i}{l}{\textsci}} \dicPos{n}[2] \dicFlx{(‑s)}[20] \dicFieldCat{jaz.} \dicIndirectTranslationCS{druh výslovnosti -- používání neaspirované plozivy ([\textsubring{b}], [\textsubring{d}], [\r{g}], [\r{\textObardotlessj}]) uprostřed slova po dlouhé samohlásce (standardní výslovnost)} \dicAntonym{harðmæli}
\dicEntry[linna] \dicTerm{linn|a} \dicIPA{{l}{\textsci}{n}{\textlengthmark}{a}} \dicPos{v}[2] \dicFlx{(‑ti, ‑t)}[80] \dicPhraseIS{e‑u linnir} \dicFlx{impers} \dicDirectTranslationCS{(co) přestává\,/\addthin ustává\,/\addthin končí} \dicExampleIS{Óveðrinu er tekið að linna.} \dicExampleCS{Bouře pomalu ustává.};  \dicPhraseIS{linna ekki látum} \dicDirectTranslationCS{nepřestat, neustat (s~prosbami ap.)}
\dicEntry[linnulaus] \dicTerm{linnu··laus} \dicIPA{{l}{\textsci}{n}{\textlengthmark}{\textscy}{l}{\oe i}{s}} \dicPos{adj}[5]\dicFlx{}[-1] \dicSynonym{sífelldur} \dicDirectTranslationCS{nepřetržitý, neustálý, trvalý} \dicExampleIS{linnulausar árásir} \dicExampleCS{nepřetržité útoky}
\dicEntry[linsa] \dicTerm{lins|a} \dicIPA{{l}{\textsci}{n}{s}{a}} \dicPos{f}[1] \dicFlx{(‑u, ‑ur)}[7] \textbf{1.} \dicFieldCat{fyz.} \dicDirectTranslationCS{čočka}  \textbf{2.} \dicSynonym*{augnlinsa} \dicDirectTranslationCS{(kontaktní) čočka}
\dicEntry[linsoðinn] \dicTerm{lin··soðinn} \dicIPA{{l}{\textsci}{n}{s}{\textopeno}{ð}{\textsci}{\textsubring{n}}} \dicPos{adj}[6]\dicFlx{}[-2] \dicDirectTranslationCS{(jsoucí) naměkko} \dicExampleIS{linsoðið egg} \dicExampleCS{vejce uvařené naměkko}
\dicEntry[linsubaun] \dicTerm{linsu··baun} \dicIPA{{l}{\textsci}{n}{s}{\textscy}{\textsubring{b}}{\oe i}{\textsubring{n}}} \dicPos{f}[7] \dicFlx{(‑ar, ‑ir)}[1] \dicFieldCat{bot.} \dicDirectTranslationCS{čočka} \textit{(l.~{\textLA{Lens culinaris}})}
\dicEntry[linun] \dicTerm{lin|un} \dicIPA{{l}{\textsci}{\textlengthmark}{n}{\textscy}{\textsubring{n}}} \dicPos{f}[7] \dicFlx{(‑unar)}[9] \textbf{1.} \dicSynonym{fró} \dicDirectTranslationCS{uklidnění, úleva}  \textbf{2.} \dicSynonym{mildun} \dicDirectTranslationCS{zmírnění} \dicExampleIS{linun refsingar} \dicExampleCS{zmírnění trestu}
\dicEntry[linur] \dicTerm{linur} \dicIPA{{l}{\textsci}{\textlengthmark}{n}{\textscy}{\textsubring{r}}} \dicPos{adj}[1]\dicFlx{}[-1] \textbf{1.} \dicSynonym{mjúkur} \dicDirectTranslationCS{hebký, měkký, jemný}  \textbf{2.} \dicSynonym{slakur} \dicDirectTranslationCS{slabý, chabý} \dicExampleIS{linur í fótbolta} \dicExampleCS{slabý ve fotbale}  \textbf{3.} \dicSynonym{slappur} \dicDirectTranslationCS{indisponovaný, slabý}
\dicEntry[lipur] \dicTerm{lipur} \dicIPA{{l}{\textsci}{\textlengthmark}{\textsubring{b}}{\textscy}{\textsubring{r}}} \dicPos{adj}[9] \dicFlx{(f ‑)}[1] \textbf{1.} \dicSynonym*{flinkur} \dicDirectTranslationCS{zručný, šikovný} \dicExampleIS{Hún er lipur í höndunum.} \dicExampleCS{Má šikovné ruce.}  \textbf{2.} \dicSynonym{þægilegur} \dicDirectTranslationCS{vstřícný, přívětivý}  \textbf{3.} \dicSynonym{meðfærilegur} \dicDirectTranslationCS{(uživatelsky) přívětivý}
\dicEntry[lipurð] \dicTerm{lipurð} \dicIPA{{l}{\textsci}{\textlengthmark}{\textsubring{b}}{\textscy}{r}{\texttheta}} \dicPos{f}[4] \dicFlx{(‑ar)}[3] \textbf{1.} \dicSynonym*{fimleiki} \dicDirectTranslationCS{obratnost, zručnost}  \textbf{2.} \dicSynonym*{lipur í viðmóti} \dicDirectTranslationCS{vstřícnost, přívětivost}
\dicEntry[lirfa] \dicTerm{lirf|a} \dicIPA{{l}{\textsci}{r}{v}{a}} \dicPos{f}[1] \dicFlx{(‑u, ‑ur)}[19] \dicDirectTranslationCS{larva}
\dicEntry[list] \dicTerm{list} \dicsymFrequent\  \dicIPA{{l}{\textsci}{s}{\textsubring{d}}} \dicPos{f}[7] \dicFlx{(‑ar, ‑ir)}[1] \textbf{1.} \dicDirectTranslationCS{umění} \dicIndirectTranslationCS{(odvětví lidské tvorby)} \dicExampleIS{unna listum} \dicExampleCS{zbožňovat umění}  \textbf{2.} \dicSynonym{færni} \dicDirectTranslationCS{umění, um, dovednost}
\dicEntry[listaháskóli] \dicTerm{lista··há·skól|i} \dicIPA{{l}{\textsci}{s}{\textsubring{d}}{a}{h}{au}{s}{\r{g}}{ou}{l}{\textsci}} \dicPos{m}[1] \dicFlx{(‑a, ‑ar)}[1] \dicFieldCat{škol.} \dicDirectTranslationCS{akademie umění}
\dicEntry[listahátíð] \dicTerm{lista··há·tíð} \dicIPA{{l}{\textsci}{s}{\textsubring{d}}{a}{h}{au}{t\smash{\textsuperscript{h}}}{i}{\texttheta}} \dicPos{f}[7] \dicFlx{(‑ar, ‑ir)}[1] \dicDirectTranslationCS{umělecký festival}
\dicEntry[listakona] \dicTerm{lista··kon|a} \dicIPA{{l}{\textsci}{s}{\textsubring{d}}{a}{k\smash{\textsuperscript{h}}}{\textopeno}{n}{a}} \dicPos{f}[1] \dicFlx{(‑u, ‑ur)}[27] \dicDirectTranslationCS{umělkyně}
\dicEntry[listamaður] \dicTerm{lista··|maður} \dicsymFrequent\  \dicIPA{{l}{\textsci}{s}{\textsubring{d}}{a}{m}{a}{ð}{\textscy}{\textsubring{r}}} \dicPos{m}[13] \dicFlx{(‑manns, ‑menn)}[2] \dicDirectTranslationCS{umělec, umělkyně} \dicExampleIS{Það er eitt dáðasta verk listamannsins.} \dicExampleCS{Je to jedno z~nejvíce obdivovaných děl umělce.}
\dicEntry[listasafn] \dicTerm{lista··|safn} \dicIPA{{l}{\textsci}{s}{\textsubring{d}}{a}{s}{a}{\textsubring{b}}{\textsubring{n}}} \dicPos{n}[2] \dicFlx{(‑safns, ‑söfn)}[8] \dicDirectTranslationCS{galerie (umění)}
\dicEntry[listasaga] \dicTerm{lista··|saga} \dicIPA{{l}{\textsci}{s}{\textsubring{d}}{a}{s}{a}{\textbabygamma}{a}} \dicPos{f}[1] \dicFlx{(‑sögu)}[2] \dicDirectTranslationCS{dějiny umění}
\dicEntry[listaskóli] \dicTerm{lista··skól|i} \dicIPA{{l}{\textsci}{s}{\textsubring{d}}{a}{s}{\r{g}}{ou}{l}{\textsci}} \dicPos{m}[1] \dicFlx{(‑a, ‑ar)}[1] \dicFieldCat{škol.} \dicDirectTranslationCS{umělecká škola}
\dicEntry[listaverk] \dicTerm{lista··verk} \dicIPA{{l}{\textsci}{s}{\textsubring{d}}{a}{v}{\textepsilon}{\textsubring{r}}{\r{g}}} \dicPos{n}[2] \dicFlx{(‑s, ‑)}[5] \dicDirectTranslationCS{umělecké dílo}
\dicEntry[listdans] \dicTerm{list··dans} \dicIPA{{l}{\textsci}{s}{\textsubring{d}}{a}{n}{s}} \dicPos{m}[4] \dicFlx{(‑, ‑ar)}[29] \dicSynonym{ballett} \dicDirectTranslationCS{umělecký tanec, balet}
\dicEntry[listfengur] \dicTerm{list··fengur} \dicIPA{{l}{\textsci}{s}{\textsubring{d}}{f}{ei}{\ng}{\r{g}}{\textscy}{\textsubring{r}}} \dicPos{adj}[1]\dicFlx{}[-1] \dicDirectTranslationCS{umělecky nadaný}
\dicEntry[listfræðingur] \dicTerm{list·fræð··ing|ur} \dicIPA{{l}{\textsci}{s}{\textsubring{d}}{f}{r}{a}{i}{ð}{i}{\ng}{\r{g}}{\textscy}{\textsubring{r}}} \dicPos{m}[6] \dicFlx{(‑s, ‑ar)}[8] \dicDirectTranslationCS{historik\,/\addthin historička umění, kunsthistorik, kunsthistorička}
\dicEntry[listgáfa] \dicTerm{list··gáf|a} \dicIPA{{l}{\textsci}{s}{\textsubring{d}}{\r{g}}{au}{v}{a}} \dicPos{f}[1] \dicFlx{(‑u, ‑ur)}[13] \dicDirectTranslationCS{umělecký talent}
\dicEntry[listi] \dicTerm{list|i} \dicsymFrequent\  \dicIPA{{l}{\textsci}{s}{\textsubring{d}}{\textsci}} \dicPos{m}[1] \dicFlx{(‑a, ‑ar)}[1] \textbf{1.} \dicSynonym{skrá\smash{\textsuperscript{1}}} \dicDirectTranslationCS{seznam, soupis, listina, rejstřík, žebříček} \dicExampleIS{óskalisti} \dicExampleCS{seznam přání};  \dicPhraseIS{listi yfir e‑n\,/\addthin e‑ð} \dicDirectTranslationCS{seznam (koho\,/\addthin čeho)}  \textbf{2.} \dicSynonym{ræma} \dicDirectTranslationCS{lišta, laťka}
\dicEntry[listiðn] \dicTerm{list··iðn} \dicIPA{{l}{\textsci}{s}{\textsubring{d}}{\textsci}{ð}{\textsubring{n}}} \dicPos{f}[7] \dicFlx{(‑ar, ‑ir)}[1] \dicDirectTranslationCS{umělecké řemeslo}
\dicEntry[listiðnaður] \dicTerm{list··iðn·að|ur} \dicIPA{{l}{\textsci}{s}{\textsubring{d}}{\textsci}{ð}{n}{a}{ð}{\textscy}{\textsubring{r}}} \dicPos{m}[10] \dicFlx{(‑ar)}[9] \dicDirectTranslationCS{užité umění}
\dicEntry[listmálari] \dicTerm{list··mál·ar|i} \dicIPA{{l}{\textsci}{s}{\textsubring{d}}{m}{au}{l}{a}{r}{\textsci}} \dicPos{m}[1] \dicFlx{(‑a, ‑ar)}[13] \dicDirectTranslationCS{malíř(ka)} \dicIndirectTranslationCS{(umělec malující obrazy)}
\dicEntry[listnám] \dicTerm{list··nám} \dicIPA{{l}{\textsci}{s}{\textsubring{d}}{n}{au}{\textsubring{m}}} \dicPos{n}[2] \dicFlx{(‑s)}[2] \dicDirectTranslationCS{studium umění}
\dicEntry[listrænn] \dicTerm{list··rænn} \dicIPA{{l}{\textsci}{s}{\textsubring{d}}{r}{a}{i}{\textsubring{d}}{\textsubring{n}}} \dicPos{adj}[7]\dicFlx{}[-1] \dicDirectTranslationCS{umělecký}
\dicEntry[listskautar] \dicTerm{list··skautar} \dicIPA{{l}{\textsci}{s}{\textsubring{d}}{s}{\r{g}}{\oe i}{\textsubring{d}}{a}{\textsubring{r}}} \dicPos{m}[1] \dicFlx{pl}[2] \dicFieldCat{sport.} \dicDirectTranslationCS{krasobruslení}
\dicEntry[listunnandi] \dicTerm{list··unn·|andi} \dicIPA{{l}{\textsci}{s}{\textsubring{d}}{\textscy}{n}{a}{n}{\textsubring{d}}{\textsci}} \dicPos{m}[2] \dicFlx{(‑anda, ‑endur)}[1] \dicDirectTranslationCS{milovník\,/\addthin milovnice umění}
\dicEntry[lita] \dicTerm{lit|a} \dicsymFrequent\  \dicIPA{{l}{\textsci}{\textlengthmark}{\textsubring{d}}{a}} \dicPos{v}[1] \dicFlx{(‑aði)}[1] \dicFlx{acc} \textbf{1.} \dicSynonym{mála} \dicDirectTranslationCS{(na)barvit, nabarvovat, obarvit, obarvovat} \dicExampleIS{lita fötin} \dicExampleCS{nabarvit oblečení}  \textbf{2.} \dicDirectTranslationCS{zabarvit, zkreslit (skutečnost ap.)} \dicExampleIS{Skáldskapurinn litar frá sér.} \dicExampleCS{Literární dílo zkresluje.};  \dicIdiom{litast}{ \dicPhraseIS{litast}} \dicFlx{refl} \dicDirectTranslationCS{zbarvit se, zabarvit se, obarvit se};  \dicIdiom{litast}[um]{ \dicPhraseIS{litast um}} \dicFlx{refl} \dicDirectTranslationCS{(po)rozhlédnout se};  \dicIdiom{litast}[upp]{ \dicPhraseIS{litast upp}} \dicFlx{refl} \dicSynonym*{tapa lit} \dicDirectTranslationCS{(vy)blednout, ztratit\,/\addthin ztrácet barvu}
\dicEntry[litaraft] \dicTerm{litar··aft} \dicIPA{{l}{\textsci}{\textlengthmark}{\textsubring{d}}{a}{r}{a}{f}{\textsubring{d}}} \dicPos{n}[2] \dicFlx{(‑s)}[2] \dicDirectTranslationCS{barva\,/\addthin zbarvení obličeje}
\dicEntry[litarháttur] \dicTerm{litar··hátt|ur} \dicIPA{{l}{\textsci}{\textlengthmark}{\textsubring{d}}{a}{\textsubring{r}}{h}{au}{h}{\textsubring{d}}{\textscy}{\textsubring{r}}} \dicPos{m}[12] \dicFlx{(‑ar)}[8] \dicDirectTranslationCS{barva pleti} \dicExampleIS{fordómar vegna litarháttar} \dicExampleCS{předsudky k~barvě pleti}
\dicEntry[litasamsetning] \dicTerm{lita··sam·set·ning} \dicIPA{{l}{\textsci}{\textlengthmark}{\textsubring{d}}{a}{s}{a}{m}{s}{\textepsilon}{h}{\textsubring{d}}{n}{i}{\ng}{\r{g}}} \dicPos{f}[4] \dicFlx{(‑ar, ‑ar)}[5] \dicDirectTranslationCS{barevná kombinace, barevné schéma}
\dicEntry[litaspjald] \dicTerm{lita··|spjald} \dicIPA{{l}{\textsci}{\textlengthmark}{\textsubring{d}}{a}{s}{\textsubring{b}}{j}{a}{l}{\textsubring{d}}} \dicPos{n}[2] \dicFlx{(‑spjalds, ‑spjöld)}[8] \textbf{1.} \dicDirectTranslationCS{paleta} \dicIndirectTranslationCS{(malířská deska k~míchání a~roztírání barev)}  \textbf{2.} \dicFieldCat{poč.} \dicDirectTranslationCS{paleta} \dicIndirectTranslationCS{(na výběr barev)}
\dicEntry[litblinda] \dicTerm{lit··blind|a} \dicIPA{{l}{\textsci}{\textlengthmark}{\textsubring{d}}{\textsubring{b}}{l}{\textsci}{n}{\textsubring{d}}{a}} \dicPos{f}[1] \dicFlx{(‑u)}[5] \dicDirectTranslationCS{barvoslepost}
\dicEntry[litblindur] \dicTerm{lit··blindur} \dicIPA{{l}{\textsci}{\textlengthmark}{\textsubring{d}}{\textsubring{b}}{l}{\textsci}{n}{\textsubring{d}}{\textscy}{\textsubring{r}}} \dicPos{adj}[2]\dicFlx{}[-14] \dicFieldCat{med.} \dicDirectTranslationCS{barvoslepý}
\dicEntry[litbrigði] \dicTerm{lit··brigði} \dicIPA{{l}{\textsci}{\textlengthmark}{\textsubring{d}}{\textsubring{b}}{r}{\textsci}{\textbabygamma}{ð}{\textsci}} \dicPos{n}[2] \dicFlx{(‑s, ‑)}[14] \dicDirectTranslationCS{odstín (barvy)}
\dicEntry[litgreining] \dicTerm{lit··grein·ing} \dicIPA{{l}{\textsci}{\textlengthmark}{\textsubring{d}}{\r{g}}{r}{ei}{n}{i}{\ng}{\r{g}}} \dicPos{f}[4] \dicFlx{(‑ar)}[7] \textbf{1.} \dicFieldCat{polygr.} \dicDirectTranslationCS{separace barev}  \textbf{2.} \dicDirectTranslationCS{barevná analýza} \dicIndirectTranslationCS{(osobní kombinace barev)}
\dicEntry[Litháen] \dicTerm{Litháen} \dicIPA{{l}{\textsci}{\textlengthmark}{t\smash{\textsuperscript{h}}}{au}{\textepsilon}{\textsubring{n}}} \dicPos{n}[2] \dicFlx{(‑s)}[32] \dicFieldCat{geog.} \dicDirectTranslationCS{Litva}
\dicEntry[Lithái] \dicTerm{Lithá|i} \dicIPA{{l}{\textsci}{\textlengthmark}{t\smash{\textsuperscript{h}}}{au}{\textsci}} \dicPos{m}[1] \dicFlx{(‑a, ‑ar)}[1] \dicDirectTranslationCS{Litevec, Litevka}
\dicEntry[litháíska] \dicTerm{litháísk|a} \dicIPA{{l}{\textsci}{\textlengthmark}{t\smash{\textsuperscript{h}}}{au}{i}{s}{\r{g}}{a}} \dicPos{f}[1] \dicFlx{(‑u)}[5] \dicLink{litháska}
\dicEntry[litháískur] \dicTerm{litháískur} \dicIPA{{l}{\textsci}{\textlengthmark}{t\smash{\textsuperscript{h}}}{au}{i}{s}{\r{g}}{\textscy}{\textsubring{r}}} \dicPos{adj}[1]\dicFlx{}[-6] \dicLink{litháskur}
\dicEntry[litháska] \dicTerm{lithásk|a}\dicTerm{, litháíska} \dicIPA{{l}\-{\textsci}\-{\textlengthmark}\-{t\smash{\textsuperscript{h}}}\-{au}\-{s}\-{\r{g}}\-{a}\-} \dicPos{f}[1] \dicFlx{(‑u)}[5] \dicDirectTranslationCS{litevština}
\dicEntry[litháskur] \dicTerm{litháskur}\dicTerm{, litháískur} \dicIPA{{l}\-{\textsci}\-{\textlengthmark}\-{t\smash{\textsuperscript{h}}}\-{au}\-{s}\-{\r{g}}\-{\textscy}\-{\textsubring{r}}\-} \dicPos{adj}[1]\dicFlx{}[-6] \dicDirectTranslationCS{litevský}
\dicEntry[lithimna] \dicTerm{lit··himn|a} \dicIPA{{l}{\textsci}{\textlengthmark}{\textsubring{d}}{h}{\textsci}{m}{n}{a}} \dicPos{f}[1] \dicFlx{(‑u, ‑ur)}[7] \dicFieldCat{anat.} \dicDirectTranslationCS{duhovka}
\dicEntry[liti] \dicTerm{liti} \dicIPA{{l}{\textsci}{\textlengthmark}{\textsubring{d}}{\textsci}} \dicPos{v} \dicFlx{con pf sg 1 pers} \dicLink{líta}
\dicEntry[litið] \dicTerm{litið} \dicIPA{{l}{\textsci}{\textlengthmark}{\textsubring{d}}{\textsci}{\texttheta}} \dicPos{v} \dicFlx{supin} \dicLink{líta}
\dicEntry[litlafingurs] \dicTerm{litla··fingurs} \dicIPA{{l}{\textsci}{h}{\textsubring{d}}{l}{a}{f}{\textsci}{\ng}{\r{g}}{\textscy}{\textsubring{r}}{s}} \dicPos{m} \dicFlx{sg gen} \dicLink{litlifingur}
\dicEntry[litlatá] \dicTerm{litla··tá} \dicIPA{{l}{\textsci}{h}{\textsubring{d}}{l}{a}{t\smash{\textsuperscript{h}}}{au}} \dicPos{f}[9] \dicFlx{(litlutáar, litlutær)}[2] \dicDirectTranslationCS{malík\,/\addthin malíček (u~nohy)}
\dicEntry[litlaus] \dicTerm{lit··laus} \dicIPA{{l}{\textsci}{\textlengthmark}{\textsubring{d}}{l}{\oe i}{s}} \dicPos{adj}[5]\dicFlx{}[-1] \textbf{1.} \dicSynonym*{ekki í lit} \dicDirectTranslationCS{bezbarvý}  \textbf{2.} \dicSynonym{dauflegur} \dicDirectTranslationCS{mdlý, fádní, bezbarvý} \dicExampleIS{litlaus frásögn} \dicExampleCS{bezbarvé vyprávění}
\dicEntry[litlifingur] \dicTerm{litli··fingur} \dicIPA{{l}{\textsci}{h}{\textsubring{d}}{l}{\textsci}{f}{i}{\ng}{\r{g}}{\textscy}{\textsubring{r}}} \dicPos{m}[5] \dicFlx{(litlafingurs, litlufingur)}[14] \dicDirectTranslationCS{malík\,/\addthin malíček (u~ruky)}
\dicEntry[litlufingur] \dicTerm{litlu··fingur} \dicIPA{{l}{\textsci}{h}{\textsubring{d}}{l}{\textscy}{f}{\textsci}{\ng}{\r{g}}{\textscy}{\textsubring{r}}} \dicPos{m} \dicFlx{pl nom} \dicLink{litlifingur}
\dicEntry[litmynd] \dicTerm{lit··mynd} \dicIPA{{l}{\textsci}{\textlengthmark}{\textsubring{d}}{m}{\textsci}{n}{\textsubring{d}}} \dicPos{f}[7] \dicFlx{(‑ar, ‑ir)}[1] \dicDirectTranslationCS{barevný obraz, barevná fotografie}
\dicEntry[litningur] \dicTerm{litn··ing|ur} \dicIPA{{l}{\textsci}{h}{\textsubring{d}}{n}{i}{\ng}{\r{g}}{\textscy}{\textsubring{r}}} \dicPos{m}[6] \dicFlx{(‑s, ‑ar)}[8] \dicFieldCat{biol.} \dicDirectTranslationCS{chromozom}
\dicEntry[litprentun] \dicTerm{lit··prent|un} \dicIPA{{l}{\textsci}{\textlengthmark}{\textsubring{d}}{p\smash{\textsuperscript{h}}}{r}{\textepsilon}{\textsubring{n}}{\textsubring{d}}{\textscy}{\textsubring{n}}} \dicPos{f}[7] \dicFlx{(‑unar)}[9] \dicFieldCat{polygr.} \dicDirectTranslationCS{barevný tisk}
\dicEntry[litríkur] \dicTerm{lit··ríkur} \dicIPA{{l}{\textsci}{\textlengthmark}{\textsubring{d}}{r}{i}{\r{g}}{\textscy}{\textsubring{r}}} \dicPos{adj}[1]\dicFlx{}[-1] \textbf{1.} \dicDirectTranslationCS{pestrobarevný, pestrý} \dicExampleIS{litrík föt} \dicExampleCS{pestrobarevné oblečení}  \textbf{2.} \dicDirectTranslationCS{pestrý, rozmanitý} \dicExampleIS{Söngkonan á litríkan feril að baki.} \dicExampleCS{Zpěvačka má za sebou pestrou kariéru.}
\dicEntry[litróf] \dicTerm{lit··róf} \dicIPA{{l}{\textsci}{\textlengthmark}{\textsubring{d}}{r}{ou}{f}} \dicPos{n}[2] \dicFlx{(‑s, ‑)}[5] \textbf{1.} \dicFieldCat{fyz.} \dicDirectTranslationCS{barevné spektrum}  \textbf{2.} \dicLangCat{přen.} \dicDirectTranslationCS{spektrum (názorové ap.)}
\dicEntry[litskrúð] \dicTerm{lit··skrúð} \dicIPA{{l}{\textsci}{\textlengthmark}{\textsubring{d}}{s}{\r{g}}{r}{u}{\texttheta}} \dicPos{n}[2] \dicFlx{(‑s)}[2] \dicDirectTranslationCS{pestrobarevnost}
\dicEntry[litskyggna] \dicTerm{lit··skyggn|a} \dicIPA{{l}{\textsci}{\textlengthmark}{\textsubring{d}}{s}{\r{\textObardotlessj}}{\textsci}{\r{g}}{n}{a}} \dicPos{f}[1] \dicFlx{(‑u, ‑ur)}[7] \dicDirectTranslationCS{barevný diapozitiv}
\dicEntry[litum] \dicTerm{litum} \dicIPA{{l}{\textsci}{\textlengthmark}{\textsubring{d}}{\textscy}{\textsubring{m}}} \dicPos{v} \dicFlx{ind pf pl 1 pers} \dicLink{líta}
\dicEntry[litun] \dicTerm{lit|un} \dicIPA{{l}{\textsci}{\textlengthmark}{\textsubring{d}}{\textscy}{\textsubring{n}}} \dicPos{f}[7] \dicFlx{(‑unar)}[9] \dicDirectTranslationCS{(na)barvení}
\dicEntry[litur] \dicTerm{lit|ur} \dicsymFrequent\  \dicIPA{{l}{\textsci}{\textlengthmark}{\textsubring{d}}{\textscy}{\textsubring{r}}} \dicPos{m}[10] \dicFlx{(‑ar, ‑ir)}[13] \textbf{1.} \dicDirectTranslationCS{barva, barevný odstín} \dicExampleIS{Hvernig er það á litinn?} \dicExampleCS{Jakou to má barvu?}  \textbf{2.} \dicSynonym*{litarefni} \dicDirectTranslationCS{barva, nátěr}  \textbf{3.} \dicSynonym*{tegund spila} \dicDirectTranslationCS{barva} \dicIndirectTranslationCS{(v~kartách)}
\dicEntry[Líbani] \dicTerm{Líban|i} \dicIPA{{l}{i}{\textlengthmark}{\textsubring{b}}{a}{n}{\textsci}} \dicPos{m}[3] \dicFlx{(‑a, ‑ar\,/\addthin ‑ir)}[3] \dicDirectTranslationCS{Libanonec, Libanonka}
\dicEntry[Líbanon] \dicTerm{Líbanon} \dicIPA{{l}{i}{\textlengthmark}{\textsubring{b}}{a}{n}{\textopeno}{\textsubring{n}}} \dicPos{n}[2] \dicFlx{(‑s)}[32] \dicFieldCat{geog.} \dicDirectTranslationCS{Libanon}
\dicEntry[líbanskur] \dicTerm{líbanskur} \dicIPA{{l}{i}{\textlengthmark}{\textsubring{b}}{a}{n}{s}{\r{g}}{\textscy}{\textsubring{r}}} \dicPos{adj}[1] \dicFlx{(f líbönsk)}[3] \dicDirectTranslationCS{libanonský}
\dicEntry[Líbería] \dicTerm{Líberí|a} \dicIPA{{l}{i}{\textlengthmark}{\textsubring{b}}{\textepsilon}{r}{i}{j}{a}} \dicPos{f}[1] \dicFlx{(‑u)}[6] \dicFieldCat{geog.} \dicDirectTranslationCS{Libérie}
\dicEntry[líberískur] \dicTerm{líberískur} \dicIPA{{l}{i}{\textlengthmark}{\textsubring{b}}{\textepsilon}{r}{i}{s}{\r{g}}{\textscy}{\textsubring{r}}} \dicPos{adj}[1]\dicFlx{}[-6] \dicDirectTranslationCS{liberijský}
\dicEntry[Líberíumaður] \dicTerm{Líberíu··|maður} \dicIPA{{l}{i}{\textlengthmark}{\textsubring{b}}{\textepsilon}{r}{i}{j}{\textscy}{m}{a}{ð}{\textscy}{\textsubring{r}}} \dicPos{m}[13] \dicFlx{(‑manns, ‑menn)}[2] \dicDirectTranslationCS{Liberijec, Liberijka}
\dicEntry[Líbía] \dicTerm{Líbí|a}\dicTerm{, Líbýa} \dicIPA{{l}\-{i}\-{\textlengthmark}\-{\textsubring{b}}\-{i}\-{j}\-{a}\-} \dicPos{f}[1] \dicFlx{(‑u)}[6] \dicFieldCat{geog.} \dicDirectTranslationCS{Libye}
\dicEntry[líbískur] \dicTerm{líbískur}\dicTerm{, líbýskur} \dicIPA{{l}\-{i}\-{\textlengthmark}\-{\textsubring{b}}\-{i}\-{s}\-{\r{g}}\-{\textscy}\-{\textsubring{r}}\-} \dicPos{adj}[1]\dicFlx{}[-6] \dicDirectTranslationCS{libyjský}
\dicEntry[Líbíumaður] \dicTerm{Líbíu··|maður} \dicIPA{{l}{i}{\textlengthmark}{\textsubring{b}}{i}{j}{\textscy}{m}{a}{ð}{\textscy}{\textsubring{r}}} \dicPos{m}[13] \dicFlx{(‑manns, ‑menn)}[2] \dicDirectTranslationCS{Libyjec, Libyjka}
\dicEntry[Líbýa] \dicTerm{Líbý|a} \dicIPA{{l}{i}{\textlengthmark}{\textsubring{b}}{i}{j}{a}} \dicPos{f}[1] \dicFlx{(‑u)}[6] \dicLink{Líbía}
\dicEntry[líbýskur] \dicTerm{líbýskur} \dicIPA{{l}{i}{\textlengthmark}{\textsubring{b}}{i}{s}{\r{g}}{\textscy}{\textsubring{r}}} \dicPos{adj}[1]\dicFlx{}[-6] \dicLink{líbískur}
\dicEntry[líð] \dicTerm{líð} \dicIPA{{l}{i}{\textlengthmark}{\texttheta}} \dicPos{v} \dicFlx{ind praes sg 1 pers} \dicLink{líða}
\dicEntry[líða] \dicTerm{líða} \dicsymFrequent\  \dicIPA{{l}{i}{\textlengthmark}{ð}{a}} \dicPos{v}[6] \dicFlx{(líð, leið, liðum, liði, liðið)}[66] \dicFlx{acc\,/\addthin dat} \textbf{1.} \dicSynonym*{um tíma} \dicDirectTranslationCS{plynout, ubíhat, utíkat} \dicIndirectTranslationCS{(o~čase)} \dicExampleIS{Árið líður hratt.} \dicExampleCS{Rok rychle ubíhá.}  \textbf{2.} \dicFlx{dat} \dicDirectTranslationCS{pokračovat, postupovat} \dicExampleIS{Hvernig líður verkinu?} \dicExampleCS{Jak pokračuje práce?}  \textbf{3.} \dicFlx{acc} \dicSynonym{þola} \dicDirectTranslationCS{snést, snášet, tolerovat} \dicIndirectTranslationCS{(často se záporem)} \dicExampleIS{líða þetta ekki} \dicExampleCS{nesnést to}  \textbf{4.} \dicPhraseIS{e‑m líður} \dicFlx{impers} \dicDirectTranslationCS{(kdo) se cítí (dobře ap.)} \dicExampleIS{Mér líður vel.} \dicExampleCS{Cítím se dobře.\,/\addthin Je mi dobře.};  \dicPhraseIS{hvernig líður þér?} \dicDirectTranslationCS{jak se máš?\,/\addthin jak se cítíš?};  \dicIdiom{líða}[að]{ \dicPhraseIS{það líður að e‑u}} \dicFlx{impers} \dicDirectTranslationCS{(co) se blíží (svátky ap.)};  \dicIdiom{líða}[hjá]{ \dicPhraseIS{e‑að líður hjá}} \dicDirectTranslationCS{(co) míjí, (co) pomine};  \dicIdiom{líða}[í]{ \dicPhraseIS{líða í ómegin}} \dicDirectTranslationCS{omdlít, upadnout do mdlob};  \dicIdiom{líða}[undir]{ \dicPhraseIS{e‑að líður undir lok}} \dicDirectTranslationCS{(co) se schyluje\,/\addthin chýlí ke konci};  \dicIdiom{líða}[út af]{ \dicPhraseIS{líða út af}} {\textbf{a.}} \dicDirectTranslationCS{omdlít};  {\textbf{b.}} \dicDirectTranslationCS{usnout} \dicExampleIS{líða út af í hægindastólnum} \dicExampleCS{usnout v~křesle};  \dicIdiom{líða}[yfir]{ \dicPhraseIS{það líður yfir e‑n}} \dicFlx{impers} \dicDirectTranslationCS{(kdo) omdlívá, (kdo) ztrácí vědomí} \dicExampleIS{Það leið yfir mig.} \dicExampleCS{Omdlel jsem.};  \dicIdiom{líðast}{ \dicPhraseIS{e‑m líðst e‑að}} \dicFlx{refl impers} \dicDirectTranslationCS{(co) se (komu) toleruje} \dicIndirectTranslationCS{(často se záporem)};  \dicIdiom{líða}{ \dicPhraseIS{hvað sem öðru líður}} \dicDirectTranslationCS{beztak, když se to vezme kolem a~kolem}
\dicEntry[líðan] \dicTerm{líðan} \dicIPA{{l}{i}{\textlengthmark}{ð}{a}{\textsubring{n}}} \dicPos{f}[4] \dicFlx{(‑ar)}[3] \dicDirectTranslationCS{(zdravotní) stav} \dicExampleIS{Líðan sjúklingsins fer batnandi.} \dicExampleCS{Zdravotní stav pacienta se zlepšuje.}
\dicEntry[líðandi] \dicTerm{líð··andi} \dicIPA{{l}{i}{\textlengthmark}{ð}{a}{n}{\textsubring{d}}{\textsci}} \dicPos{adj}[13] \dicFlx{indecl}[1] \dicPhraseIS{á líðandi stund} \dicFlx{adv} \dicSynonym{núna} \dicDirectTranslationCS{teď, v~současné chvíli}
\dicEntry[líf] \dicTerm{líf} \dicsymFrequent\  \dicIPA{{l}{i}{\textlengthmark}{f}} \dicPos{n}[2] \dicFlx{(‑s, ‑)}[5] \textbf{1.} \dicSynonym*{það að lifa} \dicDirectTranslationCS{život, žití, bytí};  \dicPhraseIS{vera á lífi} \dicDirectTranslationCS{být naživu}  \textbf{2.} \dicSynonym*{lífverur} \dicDirectTranslationCS{život, existence (organismu)} \dicExampleIS{lífið á jörðinni} \dicExampleCS{život na Zemi}  \textbf{3.} \dicSynonym*{dagleg tilvera} \dicDirectTranslationCS{život (všední ap.)}  \textbf{4.} \dicSynonym{fjör} \dicDirectTranslationCS{živo, rušno};  \dicIdiom{líf}{ \dicPhraseIS{af lífi og sál}} \dicFlx{adv} \dicDirectTranslationCS{z~celých sil}; { \dicPhraseIS{blása lífið í e‑ð}} \dicLangCat{přen.} \dicDirectTranslationCS{vdechnout život (čemu)}; { \dicPhraseIS{elska e‑n út af lífinu}} \dicDirectTranslationCS{milovat (koho) z~celé duše};  \dicPhraseIS{láta lífið} \dicDirectTranslationCS{přijít o~život}; { \dicPhraseIS{svipta sig lífi}} \dicDirectTranslationCS{vzít si život}; { \dicPhraseIS{taka e‑n af lífi}} \dicDirectTranslationCS{vzít (komu) život}; { \dicPhraseIS{vekja til lífsins}} \dicLangCat{přen.} \dicDirectTranslationCS{probudit k~životu}
\dicEntry[lífbátur] \dicTerm{líf··bát|ur} \dicIPA{{l}{i}{v}{\textsubring{b}}{au}{\textsubring{d}}{\textscy}{\textsubring{r}}} \dicPos{m}[6] \dicFlx{(‑s, ‑ar)}[22] \dicDirectTranslationCS{záchranný člun}
\dicEntry[lífbein] \dicTerm{líf··bein} \dicIPA{{l}{i}{v}{\textsubring{b}}{ei}{\textsubring{n}}} \dicPos{n}[2] \dicFlx{(‑s, ‑)}[5] \dicFieldCat{anat.} \dicDirectTranslationCS{stydká kost}
\dicEntry[lífeðlisfræði] \dicTerm{líf··eðlis·fræð|i} \dicIPA{{l}{i}{\textlengthmark}{v}{\textepsilon}{ð}{l}{\textsci}{s}{f}{r}{a}{i}{ð}{\textsci}} \dicPos{f}[3] \dicFlx{(‑i)}[3] \dicDirectTranslationCS{fyziologie}
\dicEntry[lífeðlisfræðingur] \dicTerm{líf·eðlis·fræð··ing|ur} \dicIPA{{l}\-{i}\-{\textlengthmark}\-{v}\-{\textepsilon}\-{ð}\-{l}\-{\textsci}\-{s}\-{f}\-{r}\-{a}\-{i}\-{ð}\-{i}\-{\ng}\-{\r{g}}\-{\textscy}\-{\textsubring{r}}\-} \dicPos{m}[6] \dicFlx{(‑s, ‑ar)}[8] \dicDirectTranslationCS{fyziolog, fyzioložka}
\dicEntry[lífefnafræði] \dicTerm{líf··efna·fræð|i} \dicIPA{{l}{i}{\textlengthmark}{v}{\textepsilon}{\textsubring{b}}{n}{a}{f}{r}{a}{i}{ð}{\textsci}} \dicPos{f}[3] \dicFlx{(‑i)}[3] \dicDirectTranslationCS{biochemie}
\dicEntry[lífefnafræðingur] \dicTerm{líf·efna·fræð··ing|ur} \dicIPA{{l}\-{i}\-{\textlengthmark}\-{v}\-{\textepsilon}\-{\textsubring{b}}\-{n}\-{a}\-{f}\-{r}\-{a}\-{i}\-{ð}\-{i}\-{\ng}\-{\r{g}}\-{\textscy}\-{\textsubring{r}}\-} \dicPos{m}[6] \dicFlx{(‑s, ‑ar)}[8] \dicDirectTranslationCS{biochemik, biochemička}
\dicEntry[líferni] \dicTerm{líf··erni} \dicIPA{{l}{i}{\textlengthmark}{v}{\textepsilon}{r}{\textsubring{d}}{n}{\textsci}} \dicPos{n}[2] \dicFlx{(‑s)}[20] \dicDirectTranslationCS{životospráva, životní styl}
\dicEntry[lífeyrir] \dicTerm{líf··eyr|ir} \dicIPA{{l}{i}{\textlengthmark}{v}{ei}{r}{\textsci}{\textsubring{r}}} \dicPos{m}[7] \dicFlx{(‑is)}[2] \dicDirectTranslationCS{důchod, penze} \dicIndirectTranslationCS{(pravidelný příjem osob plynoucí z~pojištění)} \dicExampleIS{taka lífeyris} \dicExampleCS{příjem důchodu}
\dicEntry[lífeyrisgreiðsla] \dicTerm{líf·eyris··greiðsl|a} \dicIPA{{l}{i}{\textlengthmark}{v}{ei}{r}{\textsci}{s}{\r{g}}{r}{ei}{ð}{s}{\textsubring{d}}{l}{a}} \dicPos{f}[1] \dicFlx{(‑u, ‑ur)}[13] \dicDirectTranslationCS{výplata důchodu}
\dicEntry[lífeyrisréttindi] \dicTerm{líf·eyris··rétt·indi} \dicIPA{{l}{i}{\textlengthmark}{v}{ei}{r}{\textsci}{s}{r}{j}{\textepsilon}{h}{\textsubring{d}}{\textsci}{n}{\textsubring{d}}{\textsci}} \dicPos{n}[2] \dicFlx{pl}[19] \dicDirectTranslationCS{nárok na důchod}
\dicEntry[lífeyrissjóðstekjur] \dicTerm{líf·eyris·sjóðs··tekjur} \dicIPA{{l}\-{i}\-{\textlengthmark}\-{v}\-{ei}\-{r}\-{\textsci}\-{s}\-{j}\-{ou}\-{ð}\-{s}\-{t\smash{\textsuperscript{h}}}\-{\textepsilon}\-{\r{\textObardotlessj}}\-{\textscy}\-{\textsubring{r}}\-} \dicPos{f}[1] \dicFlx{pl}[18] \dicDirectTranslationCS{příjmy z~penzijního fondu}
\dicEntry[lífeyrissjóður] \dicTerm{líf·eyris··sjóð|ur} \dicIPA{{l}{i}{\textlengthmark}{v}{ei}{r}{\textsci}{s}{j}{ou}{ð}{\textscy}{\textsubring{r}}} \dicPos{m}[9] \dicFlx{(‑s, ‑ir)}[6] \dicFieldCat{ekon.} \dicDirectTranslationCS{penzijní fond}
\dicEntry[lífeyrissparnaður] \dicTerm{líf·eyris··spar·nað|ur} \dicIPA{{l}\-{i}\-{\textlengthmark}\-{v}\-{ei}\-{r}\-{\textsci}\-{s}\-{\textsubring{b}}\-{a}\-{r}\-{\textsubring{d}}\-{n}\-{a}\-{ð}\-{\textscy}\-{\textsubring{r}}\-} \dicPos{m}[10] \dicFlx{(‑ar)}[9] \dicDirectTranslationCS{důchodové spoření}
\dicEntry[lífeyristryggingar] \dicTerm{líf··eyris··trygg·ingar} \dicIPA{{l}\-{i}\-{\textlengthmark}\-{v}\-{ei}\-{r}\-{\textsci}\-{s}\-{t\smash{\textsuperscript{h}}}\-{r}\-{\textsci}\-{\r{\textObardotlessj}}\-{i}\-{\ng}\-{\r{g}}\-{a}\-{\textsubring{r}}\-} \dicPos{f}[4] \dicFlx{pl}[6] \dicDirectTranslationCS{důchodové pojištění}
\dicEntry[lífeyrisþegi] \dicTerm{líf·eyris··þeg|i} \dicIPA{{l}{i}{\textlengthmark}{v}{ei}{r}{\textsci}{s}{\texttheta}{ei}{\textsci}} \dicPos{m}[1] \dicFlx{(‑a, ‑ar)}[1] \dicDirectTranslationCS{důchodce, důchodkyně, penzista, penzistka}
\dicEntry[líffræði] \dicTerm{líf··fræð|i} \dicIPA{{l}{i}{f}{\textlengthmark}{r}{a}{i}{ð}{\textsci}} \dicPos{f}[3] \dicFlx{(‑i)}[3] \dicDirectTranslationCS{biologie}
\dicEntry[líffræðingur] \dicTerm{líf·fræð··ing|ur} \dicIPA{{l}{i}{f}{\textlengthmark}{r}{a}{i}{ð}{i}{\ng}{\r{g}}{\textscy}{\textsubring{r}}} \dicPos{m}[6] \dicFlx{(‑s, ‑ar)}[8] \dicDirectTranslationCS{biolog, bioložka}
\dicEntry[líffærafræði] \dicTerm{líf·færa··fræð|i} \dicIPA{{l}{i}{f}{\textlengthmark}{a}{i}{r}{a}{f}{r}{a}{i}{ð}{\textsci}} \dicPos{f}[3] \dicFlx{(‑i)}[3] \dicDirectTranslationCS{anatomie}
\dicEntry[líffæri] \dicTerm{líf··færi} \dicIPA{{l}{i}{f}{\textlengthmark}{a}{i}{r}{\textsci}} \dicPos{n}[2] \dicFlx{(‑s, ‑)}[14] \dicFieldCat{anat.} \dicDirectTranslationCS{orgán, ústrojí}
\dicEntry[lífga] \dicTerm{lífg|a} \dicIPA{{l}{i}{v}{\r{g}}{a}} \dicPos{v}[1] \dicFlx{(‑aði)}[1] \dicFlx{acc} \textbf{1.} \dicSynonym{fjörga} \dicDirectTranslationCS{oživit, oživovat, vzkřísit} \dicIndirectTranslationCS{(činit živějším)} \dicExampleIS{lífga eldinn} \dicExampleCS{oživit oheň};  \dicPhraseIS{lífga upp á e‑ð} \dicDirectTranslationCS{oživit\,/\addthin osvěžit (co) (pokoj barvami ap.)}  \textbf{2.} \dicSynonym*{vekja til lífs} \dicDirectTranslationCS{oživit, oživovat, vzkřísit, resuscitovat} \dicIndirectTranslationCS{(probudit člověka k~životu)};  \dicPhraseIS{lífga e‑n úr dauðadái} \dicDirectTranslationCS{probudit (koho) z~kómatu};  \dicPhraseIS{lífga e‑n við} \dicDirectTranslationCS{oživit (koho), probudit (koho) k~životu}
\dicEntry[lífgjafi] \dicTerm{líf··gjaf|i} \dicIPA{{l}{i}{v}{\r{\textObardotlessj}}{a}{v}{\textsci}} \dicPos{m}[1] \dicFlx{(‑a, ‑ar)}[8] \dicDirectTranslationCS{zachránce\,/\addthin zachránkyně (života)}
\dicEntry[lífgjöf] \dicTerm{líf··|gjöf} \dicIPA{{l}{i}{v}{\r{\textObardotlessj}}{\oe}{f}} \dicPos{f}[7] \dicFlx{(‑gjafar)}[19] \dicDirectTranslationCS{záchrana života}
\dicEntry[lífgun] \dicTerm{lífg|un} \dicIPA{{l}{i}{v}{\r{g}}{\textscy}{\textsubring{n}}} \dicPos{f}[7] \dicFlx{(‑unar, ‑anir)}[8] \dicDirectTranslationCS{resuscitace, reanimace, oživování}
\dicEntry[lífhimna] \dicTerm{líf··himn|a} \dicIPA{{l}{i}{\textlengthmark}{f}{h}{\textsci}{m}{n}{a}} \dicPos{f}[1] \dicFlx{(‑u, ‑ur)}[7] \dicFieldCat{anat.} \dicDirectTranslationCS{pobřišnice}
\dicEntry[lífklukka] \dicTerm{líf··klukk|a} \dicIPA{{l}{i}{v}{k\smash{\textsuperscript{h}}}{l}{\textscy}{h}{\r{g}}{a}} \dicPos{f}[1] \dicFlx{(‑u, ‑ur)}[13] \dicDirectTranslationCS{biologické hodiny}
\dicEntry[líflaus] \dicTerm{líf··laus} \dicIPA{{l}{i}{v}{l}{\oe i}{s}} \dicPos{adj}[5]\dicFlx{}[-1] \textbf{1.} \dicDirectTranslationCS{(jsoucí) bez života, mrtvý}  \textbf{2.} \dicFieldCat{biol.} \dicLangCat{zast.} \dicSynonym{ólífrænn} \dicDirectTranslationCS{neživý}
\dicEntry[líflát] \dicTerm{líf··lát} \dicIPA{{l}{i}{v}{l}{au}{\textsubring{d}}} \dicPos{n}[2] \dicFlx{(‑s, ‑)}[5] \dicSynonym{aftaka\smash{\textsuperscript{1}}} \dicDirectTranslationCS{poprava} \dicExampleIS{hóta e‑m  líflát} \dicExampleCS{hrozit (komu) popravou}
\dicEntry[lífláta] \dicTerm{líf··|láta} \dicIPA{{l}{i}{v}{l}{au}{\textsubring{d}}{a}} \dicPos{v}[6] \dicFlx{(‑læt, ‑lét, ‑létum, ‑léti, ‑látið)}[87] \dicFlx{acc} \dicDirectTranslationCS{popravit} \dicExampleIS{lífláta e‑n} \dicExampleCS{popravit (koho)}
\dicEntry[líflátið] \dicTerm{líf··látið} \dicIPA{{l}{i}{v}{l}{au}{\textsubring{d}}{\textsci}{\texttheta}} \dicPos{v} \dicFlx{supin} \dicLink{lífláta}
\dicEntry[líflegur] \dicTerm{líf··legur} \dicIPA{{l}{i}{v}{l}{\textepsilon}{\textbabygamma}{\textscy}{\textsubring{r}}} \dicPos{adj}[1]\dicFlx{}[-8] \dicSynonym{fjörugur} \dicDirectTranslationCS{živý, svěží, plný života}
\dicEntry[líflét] \dicTerm{líf··lét} \dicIPA{{l}{i}{v}{l}{j}{\textepsilon}{\textsubring{d}}} \dicPos{v} \dicFlx{ind pf sg 1 pers} \dicLink{lífláta}
\dicEntry[lífléti] \dicTerm{líf··léti} \dicIPA{{l}{i}{v}{l}{j}{\textepsilon}{\textsubring{d}}{\textsci}} \dicPos{v} \dicFlx{con pf sg 1 pers} \dicLink{lífláta}
\dicEntry[líflétum] \dicTerm{líf··létum} \dicIPA{{l}{i}{v}{l}{j}{\textepsilon}{\textsubring{d}}{\textscy}{\textsubring{m}}} \dicPos{v} \dicFlx{ind pf pl 1 pers} \dicLink{lífláta}
\dicEntry[líflæt] \dicTerm{líf··læt} \dicIPA{{l}{i}{v}{l}{a}{i}{\textsubring{d}}} \dicPos{v} \dicFlx{ind praes sg 1 pers} \dicLink{lífláta}
\dicEntry[lífríki] \dicTerm{líf··ríki} \dicIPA{{l}{i}{v}{r}{i}{\r{\textObardotlessj}}{\textsci}} \dicPos{n}[2] \dicFlx{(‑s, ‑)}[16] \dicDirectTranslationCS{biosystém, biologický systém}
\dicEntry[lífrænn] \dicTerm{líf··rænn} \dicIPA{{l}{i}{v}{r}{a}{i}{\textsubring{d}}{\textsubring{n}}} \dicPos{adj}[7]\dicFlx{}[-1] \textbf{1.} \dicDirectTranslationCS{živý, živoucí, rušný} \dicExampleIS{lífrænt starf} \dicExampleCS{živá práce}  \textbf{2.} \dicSynonym{náttúrulegur} \dicDirectTranslationCS{organický, ekologický} \dicExampleIS{lífrænn búskapur} \dicExampleCS{organické zemědělství} \dicAntonym{ólífrænn}
\dicEntry[lífsbarátta] \dicTerm{lífs··bar·átt|a} \dicIPA{{l}{i}{f}{s}{\textsubring{b}}{a}{r}{au}{h}{\textsubring{d}}{a}} \dicPos{f}[1] \dicFlx{(‑u)}[5] \dicDirectTranslationCS{existenční\,/\addthin životní boj}
\dicEntry[lífseigur] \dicTerm{líf··seigur} \dicIPA{{l}{i}{f}{s}{ei}{\textbabygamma}{\textscy}{\textsubring{r}}} \dicPos{adj}[1]\dicFlx{}[-1] \dicDirectTranslationCS{houževnatý, (jsoucí) s~tuhým kořínkem}
\dicEntry[lífsflótti] \dicTerm{lífs··flótt|i} \dicIPA{{l}{i}{f}{s}{f}{l}{ou}{h}{\textsubring{d}}{\textsci}} \dicPos{m}[1] \dicFlx{(‑a)}[3] \dicDirectTranslationCS{útěk před realitou, eskapismus}
\dicEntry[lífsglaður] \dicTerm{lífs··|glaður} \dicIPA{{l}{i}{f}{s}{\r{g}}{l}{a}{ð}{\textscy}{\textsubring{r}}} \dicPos{adj}[2] \dicFlx{(f ‑glöð)}[7] \dicSynonym{glaðlyndur} \dicDirectTranslationCS{radostný, šťastný, radující se ze života}
\dicEntry[lífsgæði] \dicTerm{lífs··gæði} \dicIPA{{l}{i}{f}{s}{\r{\textObardotlessj}}{a}{i}{ð}{\textsci}} \dicPos{n}[2] \dicFlx{pl}[19] \dicDirectTranslationCS{životní úroveň, kvalita života}
\dicEntry[lífsháski] \dicTerm{lífs··hásk|i} \dicIPA{{l}{i}{f}{s}{h}{au}{s}{\r{\textObardotlessj}}{\textsci}} \dicPos{m}[1] \dicFlx{(‑a)}[3] \dicDirectTranslationCS{ohrožení života}
\dicEntry[lífshætta] \dicTerm{lífs··hætt|a} \dicIPA{{l}{i}{f}{s}{h}{a}{i}{h}{\textsubring{d}}{a}} \dicPos{f}[1] \dicFlx{(‑u)}[5] \dicDirectTranslationCS{nebezpečí života}
\dicEntry[lífshættulegur] \dicTerm{lífs··hættu·legur} \dicIPA{{l}{i}{f}{s}{h}{a}{i}{h}{\textsubring{d}}{\textscy}{l}{\textepsilon}{\textbabygamma}{\textscy}{\textsubring{r}}} \dicPos{adj}[1]\dicFlx{}[-8] \dicDirectTranslationCS{životu nebezpečný}
\dicEntry[lífskjör] \dicTerm{lífs··kjör} \dicIPA{{l}{i}{f}{s}{c\smash{\textsuperscript{h}}}{\oe}{\textsubring{r}}} \dicPos{n}[2] \dicFlx{pl}[9] \dicDirectTranslationCS{životní podmínky}
\dicEntry[lífsleiði] \dicTerm{lífs··leið|i} \dicIPA{{l}{i}{f}{s}{l}{ei}{ð}{\textsci}} \dicPos{m}[1] \dicFlx{(‑a)}[3] \dicDirectTranslationCS{životní nuda, omrzelost životem}
\dicEntry[lífsleikni] \dicTerm{lífs··leikn|i} \dicIPA{{l}{i}{f}{s}{l}{ei}{h}{\r{g}}{n}{\textsci}} \dicPos{f}[3] \dicFlx{(‑i)}[3] \dicFieldCat{škol.} \dicDirectTranslationCS{praktická výchova} \dicIndirectTranslationCS{(předmět ve škole)}
\dicEntry[lífsnauðsyn] \dicTerm{lífs··nauð·syn} \dicIPA{{l}{i}{f}{s}{n}{\oe i}{ð}{s}{\textsci}{\textsubring{n}}} \dicPos{f}[4] \dicFlx{(‑jar, ‑jar)}[9] \dicDirectTranslationCS{životní nutnost}
\dicEntry[lífsorka] \dicTerm{lífs··ork|a} \dicIPA{{l}{i}{f}{s}{\textopeno}{\textsubring{r}}{\r{g}}{a}} \dicPos{f}[1] \dicFlx{(‑u)}[5] \dicDirectTranslationCS{vitalita, životní energie}
\dicEntry[lífsregla] \dicTerm{lífs··regl|a} \dicIPA{{l}{i}{f}{s}{r}{\textepsilon}{\r{g}}{l}{a}} \dicPos{f}[1] \dicFlx{(‑u, ‑ur)}[19] \dicDirectTranslationCS{životní zásada}
\dicEntry[lífsreynsla] \dicTerm{lífs··reynsl|a} \dicIPA{{l}{i}{f}{s}{r}{ei}{n}{s}{\textsubring{d}}{l}{a}} \dicPos{f}[1] \dicFlx{(‑u)}[5] \dicDirectTranslationCS{životní zkušenost}
\dicEntry[lífsskilyrði] \dicTerm{lífs··skil·yrði} \dicIPA{{l}{i}{f}{s}{\r{\textObardotlessj}}{\textsci}{l}{\textsci}{r}{ð}{\textsci}} \dicPos{n}[2] \dicFlx{pl}[19] \dicDirectTranslationCS{životní podmínky}
\dicEntry[lífsskoðun] \dicTerm{lífs··skoð|un} \dicIPA{{l}{i}{f}{s}{\r{g}}{\textopeno}{ð}{\textscy}{\textsubring{n}}} \dicPos{f}[7] \dicFlx{(‑unar, ‑anir)}[8] \dicDirectTranslationCS{životní postoj\,/\addthin názor}
\dicEntry[lífsspeki] \dicTerm{lífs··spek|i} \dicIPA{{l}{i}{f}{s}{\textsubring{b}}{\textepsilon}{\r{\textObardotlessj}}{\textsci}} \dicPos{f}[3] \dicFlx{(‑i)}[3] \dicDirectTranslationCS{životní moudrost}
\dicEntry[lífsstarf] \dicTerm{lífs··|starf} \dicIPA{{l}{i}{f}{s}{\textsubring{d}}{a}{r}{f}} \dicPos{n}[2] \dicFlx{(‑starfs, ‑störf)}[8] \dicDirectTranslationCS{životní poslání}
\dicEntry[lífsstíll] \dicTerm{lífs··stíl|l} \dicIPA{{l}{i}{f}{s}{\textsubring{d}}{i}{\textsubring{d}}{\textsubring{l}}} \dicPos{m}[6] \dicFlx{(‑s)}[50] \dicDirectTranslationCS{životní styl}
\dicEntry[lífstykki] \dicTerm{líf··stykki} \dicIPA{{l}{i}{f}{s}{\textsubring{d}}{\textsci}{h}{\r{\textObardotlessj}}{\textsci}} \dicPos{n}[2] \dicFlx{(‑s, ‑)}[16] \dicDirectTranslationCS{korzet, šněrovačka}
\dicEntry[lífsveig] \dicTerm{lífs··veig} \dicIPA{{l}{i}{f}{s}{v}{ei}{x}} \dicPos{f}[4] \dicFlx{(‑ar, ‑ar)}[1] \dicDirectTranslationCS{elixír života}
\dicEntry[lífsviðurværi] \dicTerm{lífs··viður·væri} \dicIPA{{l}{i}{f}{s}{v}{\textsci}{ð}{\textscy}{r}{v}{a}{i}{r}{\textsci}} \dicPos{n}[2] \dicFlx{(‑s)}[20] \dicDirectTranslationCS{obživa (jídlo ap.)}
\dicEntry[lífsþægindi] \dicTerm{lífs··þæg·indi} \dicIPA{{l}{i}{f}{s}{\texttheta}{a}{i}{j}{\textsci}{n}{\textsubring{d}}{\textsci}} \dicPos{n}[2] \dicFlx{pl}[19] \dicDirectTranslationCS{životní komfort\,/\addthin pohodlí}
\dicEntry[líftóra] \dicTerm{líf··tór|a} \dicIPA{{l}{i}{f}{t\smash{\textsuperscript{h}}}{ou}{r}{a}} \dicPos{f}[1] \dicFlx{(‑u)}[5] \dicDirectTranslationCS{jiskra života};  \dicPhraseIS{hræða úr e‑m líftóruna} \dicLangCat{přen.} \dicDirectTranslationCS{vyděsit (koho) k~smrti}
\dicEntry[líftrygging] \dicTerm{líf··trygg·ing} \dicIPA{{l}{i}{f}{t\smash{\textsuperscript{h}}}{r}{\textsci}{\r{\textObardotlessj}}{i}{\ng}{\r{g}}} \dicPos{f}[4] \dicFlx{(‑ar, ‑ar)}[5] \dicDirectTranslationCS{životní pojištění}
\dicEntry[líftryggja] \dicTerm{líf··trygg|ja} \dicIPA{{l}{i}{f}{t\smash{\textsuperscript{h}}}{r}{\textsci}{\r{\textObardotlessj}}{a}} \dicPos{v}[2] \dicFlx{(‑ði, ‑t)}[92] \dicFlx{acc} \dicDirectTranslationCS{uzavřít životní pojištění}
\dicEntry[lífvera] \dicTerm{líf··ver|a} \dicIPA{{l}{i}{v}{\textlengthmark}{\textepsilon}{r}{a}} \dicPos{f}[1] \dicFlx{(‑u, ‑ur)}[7] \dicDirectTranslationCS{živá bytost, živý organismus}
\dicEntry[lífvænlegur] \dicTerm{líf··væn·legur} \dicIPA{{l}{i}{v}{\textlengthmark}{a}{i}{n}{l}{\textepsilon}{\textbabygamma}{\textscy}{\textsubring{r}}} \dicPos{adj}[1]\dicFlx{}[-8] \textbf{1.} \dicDirectTranslationCS{rentabilní} \dicExampleIS{lífvænleg atvinna} \dicExampleCS{rentabilní zaměstnání}  \textbf{2.} \dicDirectTranslationCS{příznivý pro život} \dicExampleIS{lífvænlegur staður} \dicExampleCS{místo příznivé pro život}
\begin{xtolerant}{}{1pt}
\dicEntry[lífvörður] \dicTerm{líf··|vörður}\addthinS\dicIPA{{l}{i}{v}{\textlengthmark}{\oe}{r}{ð}{\textscy}{\textsubring{r}}}\addthinS\dicPos{m}[11]\addthinS\dicFlx{(‑varðar, ‑verðir)}[5] \dicDirectTranslationCS{osobní strážce\,/\addthin strážkyně, tělesná stráž, bodyguard}
\end{xtolerant}
\dicEntry[lík] \dicTerm{lík} \dicsymFrequent\  \dicIPA{{l}{i}{\textlengthmark}{\r{g}}} \dicPos{n}[2] \dicFlx{(‑s, ‑)}[5] \textbf{1.} \dicSynonym*{dauður líkami} \dicDirectTranslationCS{mrtvola} \dicExampleIS{líkið af e‑m} \dicExampleCS{(čí) mrtvola}  \textbf{2.} \dicLangCat{hovor.} \dicSynonym*{tóm vínflaska} \dicDirectTranslationCS{prázdná láhev (po alkoholu)}
\dicEntry[líka] \dicTerm{lík|a\smash{\textsuperscript{1}}} \dicsymFrequent\  \dicIPA{{l}{i}{\textlengthmark}{\r{g}}{a}} \dicPos{v}[1] \dicFlx{(‑aði)}[82] \dicFlx{impers} \dicPhraseIS{e‑m líkar e‑að} \dicDirectTranslationCS{(co) se (komu) líbí, (co) má (kdo) rád} \dicExampleIS{Mér líkar það ekki.} \dicExampleCS{Mně se to nelíbí.};  \dicPhraseIS{e‑m líkar vel við e‑n} \dicDirectTranslationCS{(kdo) má (koho) rád}
\dicEntry[líka] \dicTerm{líka\smash{\textsuperscript{2}}} \dicsymFrequent\  \dicIPA{{l}{i}{\textlengthmark}{\r{g}}{a}} \dicPos{adv} \dicSynonym{einnig} \dicDirectTranslationCS{také, taky, rovněž} \dicExampleIS{Dagblaðið er líka á netinu.} \dicExampleCS{Deník je také na internetu.}
\dicEntry[líkami] \dicTerm{líkam|i} \dicsymFrequent\  \dicIPA{{l}{i}{\textlengthmark}{\r{g}}{a}{m}{\textsci}} \dicPos{m}[1] \dicFlx{(‑a, ‑ar)}[8] \dicDirectTranslationCS{tělo} \dicExampleIS{æfingar fyrir líkama} \dicExampleCS{cvičení pro tělo}
\dicEntry[líkamlegur] \dicTerm{líkam··legur} \dicsymFrequent\  \dicIPA{{l}{i}{\r{g}}{a}{m}{l}{\textepsilon}{\textbabygamma}{\textscy}{\textsubring{r}}} \dicPos{adj}[1]\dicFlx{}[-8] \dicDirectTranslationCS{tělesný, fyzický} \dicExampleIS{líkamleg atvinna} \dicExampleCS{fyzická práce}
\dicEntry[líkamsárás] \dicTerm{líkams··á·rás} \dicIPA{{l}{i}{\textlengthmark}{\r{g}}{a}{m}{s}{au}{r}{au}{s}} \dicPos{f}[7] \dicFlx{(‑ar, ‑ir)}[1] \dicDirectTranslationCS{tělesný útok, fyzické napadení}
\dicEntry[líkamsástand] \dicTerm{líkams··á·stand} \dicIPA{{l}{i}{\textlengthmark}{\r{g}}{a}{m}{s}{au}{s}{\textsubring{d}}{a}{n}{\textsubring{d}}} \dicPos{n}[2] \dicFlx{(‑s)}[2] \dicDirectTranslationCS{tělesný stav}
\dicEntry[líkamsburðir] \dicTerm{líkams··burðir} \dicIPA{{l}{i}{\textlengthmark}{\r{g}}{a}{m}{s}{\textsubring{b}}{\textscy}{r}{ð}{\textsci}{\textsubring{r}}} \dicPos{m}[10] \dicFlx{pl}[1] \dicDirectTranslationCS{tělesná\,/\addthin fyzická síla}
\dicEntry[líkamshiti] \dicTerm{líkams··hit|i} \dicIPA{{l}{i}{\textlengthmark}{\r{g}}{a}{m}{s}{h}{\textsci}{\textsubring{d}}{\textsci}} \dicPos{m}[1] \dicFlx{(‑a)}[3] \dicDirectTranslationCS{tělesná teplota}
\dicEntry[líkamsrækt] \dicTerm{líkams··rækt} \dicIPA{{l}{i}{\textlengthmark}{\r{g}}{a}{m}{s}{r}{a}{i}{x}{\textsubring{d}}} \dicPos{f}[4] \dicFlx{(‑ar)}[3] \dicDirectTranslationCS{tělesné cvičení, posilování, fitness}
\dicEntry[líkamsræktarstöð] \dicTerm{líkams·ræktar··stöð} \dicIPA{{l}{i}{\textlengthmark}{\r{g}}{a}{m}{s}{r}{a}{i}{x}{\textsubring{d}}{a}{\textsubring{r}}{s}{\textsubring{d}}{\oe}{\texttheta}} \dicPos{f}[6] \dicFlx{(‑var, ‑var)}[1] \dicDirectTranslationCS{posilovna, fitness centrum}
\dicEntry[líkamsstærð] \dicTerm{líkams··stærð} \dicIPA{{l}{i}{\textlengthmark}{\r{g}}{a}{m}{s}{\textsubring{d}}{a}{i}{r}{\texttheta}} \dicPos{f}[7] \dicFlx{(‑ar)}[3] \dicDirectTranslationCS{životní velikost} \dicExampleIS{í fullri líkamsstærð} \dicExampleCS{v~životní velikosti}
\dicEntry[líkan] \dicTerm{lík|an} \dicIPA{{l}{i}{\textlengthmark}{\r{g}}{a}{\textsubring{n}}} \dicPos{n}[2] \dicFlx{(‑ans, ‑ön)}[8] \dicDirectTranslationCS{model, maketa} \dicIndirectTranslationCS{(zmenšené provedení předmětu)} \dicExampleIS{líkan af húsi} \dicExampleCS{model domu}
\dicEntry[líkbrennsla] \dicTerm{lík··brennsl|a} \dicIPA{{l}{i}{\textlengthmark}{\r{g}}{\textsubring{b}}{r}{\textepsilon}{n}{s}{\textsubring{d}}{l}{a}} \dicPos{f}[1] \dicFlx{(‑u, ‑ur)}[19] \dicSynonym{bálför} \dicDirectTranslationCS{kremace}
\dicEntry[líkbörur] \dicTerm{lík··börur} \dicIPA{{l}{i}{\textlengthmark}{\r{g}}{\textsubring{b}}{\oe}{r}{\textscy}{\textsubring{r}}} \dicPos{f}[12] \dicFlx{pl}[4] \dicDirectTranslationCS{máry}
\dicEntry[líkfylgd] \dicTerm{lík··fylgd} \dicIPA{{l}{i}{\textlengthmark}{\r{g}}{f}{\textsci}{l}{\textsubring{d}}} \dicPos{f}[7] \dicFlx{(‑ar, ‑ir)}[1] \dicDirectTranslationCS{pohřební průvod}
\dicEntry[líkhús] \dicTerm{lík··hús} \dicIPA{{l}{i}{\textlengthmark}{\r{g}}{h}{u}{s}} \dicPos{n}[2] \dicFlx{(‑s, ‑)}[5] \dicDirectTranslationCS{márnice}
\dicEntry[líki] \dicTerm{lík|i\smash{\textsuperscript{1}}} \dicIPA{{l}{i}{\textlengthmark}{\r{\textObardotlessj}}{\textsci}} \dicPos{m}[1] \dicFlx{(‑a, ‑ar)}[1] \dicSynonym{jafningi} \dicDirectTranslationCS{sobě rovný, podobný (člověk)} \dicExampleIS{hann og hans líkar} \dicExampleCS{on a~jemu podobní};  \dicPhraseIS{eiga engan sinn líka} \dicDirectTranslationCS{nemít sobě rovného}
\dicEntry[líki] \dicTerm{líki\smash{\textsuperscript{2}}} \dicIPA{{l}{i}{\textlengthmark}{\r{\textObardotlessj}}{\textsci}} \dicPos{n}[2] \dicFlx{(‑s, ‑)}[16] \dicSynonym{sköpulag} \dicDirectTranslationCS{podoba, forma};  \dicPhraseIS{í heilu líki} \dicFlx{adj} \dicDirectTranslationCS{celý, netknutý}
\dicEntry[líkindi] \dicTerm{lík··indi} \dicsymFrequent\  \dicIPA{{l}{i}{\textlengthmark}{\r{\textObardotlessj}}{\textsci}{n}{\textsubring{d}}{\textsci}} \dicPos{n}[2] \dicFlx{pl}[19] \dicDirectTranslationCS{pravděpodobnost} \dicExampleIS{Það eru mikil líkindi til þess að \dots{}} \dicExampleCS{Je velká pravděpodobnost, že\dots{}};  \dicPhraseIS{að öllum líkindum} \dicFlx{adv} \dicDirectTranslationCS{s~největší pravděpodobností}
\dicEntry[líking] \dicTerm{lík··ing} \dicIPA{{l}{i}{\textlengthmark}{\r{\textObardotlessj}}{i}{\ng}{\r{g}}} \dicPos{f}[4] \dicFlx{(‑ar, ‑ar)}[5] \textbf{1.} \dicSynonym{líki\smash{\textsuperscript{2}}} \dicDirectTranslationCS{podoba, podobnost, obdoba, analogie}  \textbf{2.} \dicDirectTranslationCS{přirovnání, podobenství} \dicExampleIS{tala í líkingum} \dicExampleCS{mluvit v~přirovnáních}  \textbf{3.} \dicFieldCat{lit.} \dicSynonym{samlíking} \dicDirectTranslationCS{metafora} \dicExampleIS{líkingar í skáldskap} \dicExampleCS{metafora v~literatuře}  \textbf{4.} \dicFieldCat{mat.} \dicSynonym{jafna\smash{\textsuperscript{1}}} \dicDirectTranslationCS{rovnost}
\dicEntry[líkja] \dicTerm{lík|ja} \dicsymFrequent\  \dicIPA{{l}{i}{\textlengthmark}{\r{\textObardotlessj}}{a}} \dicPos{v}[2] \dicFlx{(‑ti, ‑t)}[25] \dicFlx{dat} \textbf{1.} \dicDirectTranslationCS{přirovnat, přirovnávat};  \dicPhraseIS{líkja e‑u saman} \dicDirectTranslationCS{srovnávat (co)};  \dicPhraseIS{líkja e‑u við e‑ð} \dicDirectTranslationCS{přirovnávat (co) k~(čemu)} \dicExampleIS{líkja ævi mannsins við blóm} \dicExampleCS{přirovnávat život člověka ke květině}  \textbf{2.} \dicDirectTranslationCS{napodobit, napodobovat};  \dicPhraseIS{líkja eftir e‑u} \dicDirectTranslationCS{napodobit (co), imitovat (co)} \dicExampleIS{líkja eftir hljóði fuglsins} \dicExampleCS{napodobovat zvuk ptáčka};  \dicIdiom{líkjast}{ \dicPhraseIS{líkjast e‑m}} \dicFlx{refl} \dicDirectTranslationCS{připomínat (koho), podobat se (komu)} \dicExampleIS{Hverjum líkist ég?} \dicExampleCS{Komu se podobám?}
\dicEntry[líkjör] \dicTerm{líkjör} \dicIPA{{l}{i}{\textlengthmark}{\r{\textObardotlessj}}{\oe}{\textsubring{r}}} \dicPos{m}[4] \dicFlx{(‑s, ‑ar)}[14] \dicDirectTranslationCS{likér}
\dicEntry[líkkista] \dicTerm{lík··kist|a} \dicIPA{{l}{i}{\textlengthmark}{\r{g}}{c\smash{\textsuperscript{h}}}{\textsci}{s}{\textsubring{d}}{a}} \dicPos{f}[1] \dicFlx{(‑u, ‑ur)}[19] \dicDirectTranslationCS{rakev}
\dicEntry[líkklæði] \dicTerm{lík··klæði} \dicIPA{{l}{i}{\textlengthmark}{\r{g}}{k\smash{\textsuperscript{h}}}{l}{a}{i}{ð}{\textsci}} \dicPos{n}[2] \dicFlx{pl}[19] \dicDirectTranslationCS{pohřební roucho}
\dicEntry[líkl.] \dicTerm{líkl.} \dicPos{zkr} \dicPhraseIS{líklega} \dicFlx{adv} \dicDirectTranslationCS{pravděpodobně}
\dicEntry[líklega] \dicTerm{lík··leg|a} \dicsymFrequent\  \dicIPA{{l}{i}{\textlengthmark}{\r{g}}{l}{\textepsilon}{\textbabygamma}{a}} \dicPos{adv} \dicFlx{(comp ‑ar, sup ‑ast)} \dicSynonym{sennilega} \dicDirectTranslationCS{pravděpodobně, nejspíš, asi} \dicExampleIS{Þau segja líklega nei.} \dicExampleCS{Pravděpodobně řeknou ne.}
\dicEntry[líklegur] \dicTerm{lík··legur} \dicsymFrequent\  \dicIPA{{l}{i}{\textlengthmark}{\r{g}}{l}{\textepsilon}{\textbabygamma}{\textscy}{\textsubring{r}}} \dicPos{adj}[1]\dicFlx{}[-8] \textbf{1.} \dicSynonym{efnilegur} \dicDirectTranslationCS{slibný, nadějný}  \textbf{2.} \dicSynonym{sennilegur} \dicDirectTranslationCS{pravděpodobný, předpokládaný} \dicExampleIS{Þeir eru líklegir til að feta í sömu fótspor.} \dicExampleCS{Pravděpodobně budou kráčet ve stejných stopách.}
\dicEntry[líkn] \dicTerm{líkn} \dicIPA{{l}{i}{h}{\r{g}}{\textsubring{n}}} \dicPos{f}[7] \dicFlx{(‑ar, ‑ir)}[1] \textbf{1.} \dicSynonym{miskunn} \dicDirectTranslationCS{milost, smilování}  \textbf{2.} \dicSynonym{hjúkrun} \dicDirectTranslationCS{opatrování, péče}
\dicEntry[líkna] \dicTerm{líkn|a} \dicIPA{{l}{i}{h}{\r{g}}{n}{a}} \dicPos{v}[1] \dicFlx{(‑aði)}[1] \dicFlx{dat} \dicSynonym{hjúkra} \dicDirectTranslationCS{pečovat, poskytnout péči, opatrovat} \dicExampleIS{líkna e‑m} \dicExampleCS{pečovat o~(koho)}
\dicEntry[líknarbelgur] \dicTerm{líknar··belg|ur} \dicIPA{{l}{i}{h}{\r{g}}{n}{a}{r}{\textsubring{b}}{\textepsilon}{l}{\r{g}}{\textscy}{\textsubring{r}}} \dicPos{m}[9] \dicFlx{(‑s\,/\addthin ‑jar, ‑ir)}[26] \textbf{1.} \dicFieldCat{anat.} \dicSynonym*{fósturfylgja} \dicDirectTranslationCS{amnion, vnitřní plodový obal}  \textbf{2.} \dicSynonym*{loftpúði í bíl} \dicDirectTranslationCS{airbag} \dicIndirectTranslationCS{(zařízení pasivní bezpečnosti v~autě)}
\dicEntry[líknardeild] \dicTerm{líknar··deild} \dicIPA{{l}{i}{h}{\r{g}}{n}{a}{r}{\textsubring{d}}{ei}{l}{\textsubring{d}}} \dicPos{f}[7] \dicFlx{(‑ar, ‑ir)}[1] \dicDirectTranslationCS{hospic}
\dicEntry[líknardráp] \dicTerm{líknar··dráp} \dicIPA{{l}{i}{h}{\r{g}}{n}{a}{r}{\textsubring{d}}{r}{au}{\textsubring{b}}} \dicPos{n}[2] \dicFlx{(‑s, ‑)}[5] \dicDirectTranslationCS{eutanazie}
\dicEntry[líknarmál] \dicTerm{líknar··mál} \dicIPA{{l}{i}{h}{\r{g}}{n}{a}{r}{m}{au}{\textsubring{l}}} \dicPos{n}[2] \dicFlx{(‑s, ‑)}[5] \dicDirectTranslationCS{charita, dobročinnost}
\dicEntry[líknarstofnun] \dicTerm{líknar··stofn|un} \dicIPA{{l}{i}{h}{\r{g}}{n}{a}{\textsubring{r}}{s}{\textsubring{d}}{\textopeno}{\textsubring{b}}{n}{\textscy}{\textsubring{n}}} \dicPos{f}[7] \dicFlx{(‑unar, ‑anir)}[8] \dicSynonym{góðgerðastofnun} \dicDirectTranslationCS{dobročinná\,/\addthin charitativní organizace}
\dicEntry[líkneski] \dicTerm{lík··neski} \dicIPA{{l}{i}{h}{\r{g}}{n}{\textepsilon}{s}{\r{\textObardotlessj}}{\textsci}} \dicPos{n}[2] \dicFlx{(‑s, ‑)}[16] \dicSynonym{höggmynd} \dicDirectTranslationCS{podobizna}
\dicEntry[líknsamur] \dicTerm{líkn··|samur} \dicIPA{{l}{i}{h}{\r{g}}{\textsubring{n}}{s}{a}{m}{\textscy}{\textsubring{r}}} \dicPos{adj}[1] \dicFlx{(f ‑söm)}[2] \dicDirectTranslationCS{milosrdný, milostivý, soucitný}
\dicEntry[líkræða] \dicTerm{lík··ræð|a} \dicIPA{{l}{i}{\textlengthmark}{\r{g}}{r}{a}{i}{ð}{a}} \dicPos{f}[1] \dicFlx{(‑u, ‑ur)}[19] \dicDirectTranslationCS{smuteční proslov}
\dicEntry[líkskoðun] \dicTerm{lík··skoð|un} \dicIPA{{l}{i}{\textlengthmark}{\r{g}}{s}{\r{g}}{\textopeno}{ð}{\textscy}{\textsubring{n}}} \dicPos{f}[7] \dicFlx{(‑unar)}[9] \dicDirectTranslationCS{ohledání mrtvoly}
\dicEntry[líkskurður] \dicTerm{lík··skurð|ur} \dicIPA{{l}{i}{\textlengthmark}{\r{g}}{s}{\r{g}}{\textscy}{r}{ð}{\textscy}{\textsubring{r}}} \dicPos{m}[10] \dicFlx{(‑ar, ‑ir)}[4] \dicDirectTranslationCS{pitva}
\dicEntry[líkur] \dicTerm{líkur\smash{\textsuperscript{1}}} \dicsymFrequent\  \dicIPA{{l}{i}{\textlengthmark}{\r{g}}{\textscy}{\textsubring{r}}} \dicPos{f}[12] \dicFlx{pl}[5] \dicSynonym{líkindi} \dicDirectTranslationCS{šance, pravděpodobnost} \dicExampleIS{líkur á e‑u\,/\addthin líkur til e‑s} \dicExampleCS{šance na (co)};  \dicPhraseIS{ef að líkum lætur} \dicFlx{adv} \dicDirectTranslationCS{pravděpodobně, patrně}
\dicEntry[líkur] \dicTerm{líkur\smash{\textsuperscript{2}}} \dicsymFrequent\  \dicIPA{{l}{i}{\textlengthmark}{\r{g}}{\textscy}{\textsubring{r}}} \dicPos{adj}[1]\dicFlx{}[-1] \textbf{1.} \dicDirectTranslationCS{podobný} \dicExampleIS{lík vandamál} \dicExampleCS{podobné problémy};  \dicPhraseIS{vera líkur e‑m} \dicDirectTranslationCS{být (komu) podobný, podobat se (komu)}  \textbf{2.} \dicDirectTranslationCS{obdobný, analogický};  \dicIdiom{líkur}{ \dicPhraseIS{líkt og}} \dicFlx{conj} \dicDirectTranslationCS{jak, jako} \dicExampleIS{líta út líkt og Íslendingur} \dicExampleCS{vypadat jako Islanďan}
\dicEntry[líkþorn] \dicTerm{lík··þorn} \dicIPA{{l}{i}{\textlengthmark}{\r{g}}{\texttheta}{\textopeno}{r}{\textsubring{d}}{\textsubring{n}}} \dicPos{n}[2] \dicFlx{(‑s, ‑)}[5] \dicDirectTranslationCS{kuří oko}
\dicEntry[líkþrá] \dicTerm{lík··þrá} \dicIPA{{l}{i}{\textlengthmark}{\r{g}}{\texttheta}{r}{au}} \dicPos{f}[4] \dicFlx{(‑r)}[19] \dicFieldCat{med.} \dicDirectTranslationCS{lepra, malomocenství}
\dicEntry[líkþrár] \dicTerm{lík··þrár} \dicIPA{{l}{i}{\textlengthmark}{\r{g}}{\texttheta}{r}{au}{\textsubring{r}}} \dicPos{adj}[4]\dicFlx{}[-3] \dicDirectTranslationCS{malomocný}
\dicEntry[lím] \dicTerm{lím} \dicIPA{{l}{i}{\textlengthmark}{\textsubring{m}}} \dicPos{n}[2] \dicFlx{(‑s, ‑)}[5] \dicDirectTranslationCS{lepidlo, lep, klih}
\dicEntry[líma] \dicTerm{lím|a} \dicsymFrequent\  \dicIPA{{l}{i}{\textlengthmark}{m}{a}} \dicPos{v}[2] \dicFlx{(‑di, ‑t)}[141] \dicFlx{acc} \dicDirectTranslationCS{(na)lepit, nalepovat, přilepit, přilepovat} \dicExampleIS{líma mynd á blað} \dicExampleCS{nalepit obrázek na papír};  \dicIdiom{límast}{ \dicPhraseIS{límast}} \dicFlx{refl} \dicDirectTranslationCS{(při)lepit se}
\dicEntry[límband] \dicTerm{lím··|band} \dicIPA{{l}{i}{m}{\textsubring{b}}{a}{n}{\textsubring{d}}} \dicPos{n}[2] \dicFlx{(‑bands, ‑bönd)}[8] \dicDirectTranslationCS{lepicí páska, izolepa}
\dicEntry[límmiði] \dicTerm{lím··mið|i} \dicIPA{{l}{i}{m}{\textlengthmark}{\textsci}{ð}{\textsci}} \dicPos{m}[1] \dicFlx{(‑a, ‑ar)}[1] \dicDirectTranslationCS{nálepka, štítek, etiketa}
\dicEntry[lín] \dicTerm{lín} \dicIPA{{l}{i}{\textlengthmark}{\textsubring{n}}} \dicPos{n}[2] \dicFlx{(‑s)}[2] \textbf{1.} \dicFieldCat{bot.} \dicSynonym{hör} \dicDirectTranslationCS{len setý} \textit{(l.~{\textLA{Linum usitatissimum}})}  \dicsymPhoto\   \textbf{2.} \dicSynonym{léreft} \dicDirectTranslationCS{len, lněné plátno} \dicExampleIS{spinna lín} \dicExampleCS{tkát len}
\dicFigure{ds_image_lin_0_2.jpg}{Lín}{Lín - Franz Eugen Kohler, PD}
\dicEntry[lína] \dicTerm{lín|a} \dicsymFrequent\  \dicIPA{{l}{i}{\textlengthmark}{n}{a}} \dicPos{f}[1] \dicFlx{(‑u, ‑ur)}[7] \textbf{1.} \dicSynonym{strik} \dicDirectTranslationCS{čára, linka, linie}  \textbf{2.} \dicSynonym*{textalína} \dicDirectTranslationCS{řádek, řádka} \dicExampleIS{nokkrar línur úr textanum} \dicExampleCS{několik řádek z~textu}  \textbf{3.} \dicSynonym{band} \dicDirectTranslationCS{lano, šňůra, provázek}  \textbf{4.} \dicSynonym{lóð\smash{\textsuperscript{2}}} \dicDirectTranslationCS{lanová síť}  \textbf{5.} \dicSynonym{stefna\smash{\textsuperscript{1}}} \dicDirectTranslationCS{politický směr, politická linie}  \textbf{6.} \dicFieldCat{mat.} \dicDirectTranslationCS{přímka};  \dicIdiom{lína}{ \dicPhraseIS{lesa milli línanna}} \dicLangCat{přen.} \dicDirectTranslationCS{číst mezi řádky} \dicIndirectTranslationCS{(porozumět více, než je napsáno)}; { \dicPhraseIS{passa upp á línurnar}} \dicLangCat{přen.} \dicDirectTranslationCS{dávat pozor na linii} \dicIndirectTranslationCS{(snažit se neztloustnout)}
\dicEntry[línolía] \dicTerm{lín··olí|a} \dicIPA{{l}{i}{\textlengthmark}{n}{\textopeno}{l}{i}{j}{a}} \dicPos{f}[1] \dicFlx{(‑u)}[5] \dicDirectTranslationCS{lněný olej}
\dicEntry[línubil] \dicTerm{línu··bil} \dicIPA{{l}{i}{\textlengthmark}{n}{\textscy}{\textsubring{b}}{\textsci}{\textsubring{l}}} \dicPos{n}[2] \dicFlx{(‑s, ‑)}[5] \dicFieldCat{poč.} \dicDirectTranslationCS{řádkování}
\dicEntry[línumaður] \dicTerm{línu··|maður} \dicIPA{{l}{i}{\textlengthmark}{n}{\textscy}{m}{a}{ð}{\textscy}{\textsubring{r}}} \dicPos{m}[13] \dicFlx{(‑manns, ‑menn)}[2] \dicFieldCat{sport.} \dicDirectTranslationCS{krajní útočník\,/\addthin útočnice, křídlo}
\dicEntry[línurit] \dicTerm{línu··rit} \dicIPA{{l}{i}{\textlengthmark}{n}{\textscy}{r}{\textsci}{\textsubring{d}}} \dicPos{n}[2] \dicFlx{(‑s, ‑)}[5] \dicDirectTranslationCS{graf, diagram} \dicExampleIS{töflur og línurit} \dicExampleCS{tabulky a~grafy}
\dicEntry[línuskauti] \dicTerm{línu··skaut|i} \dicIPA{{l}{i}{\textlengthmark}{n}{\textscy}{s}{\r{g}}{\oe i}{\textsubring{d}}{\textsci}} \dicPos{m}[1] \dicFlx{(‑a, ‑ar)}[1] \dicPhraseIS{línuskautar} \dicFlx{pl} \dicFieldCat{sport.} \dicDirectTranslationCS{kolečkové brusle}
\dicEntry[línuskil] \dicTerm{línu··skil} \dicIPA{{l}{i}{\textlengthmark}{n}{\textscy}{s}{\r{\textObardotlessj}}{\textsci}{\textsubring{l}}} \dicPos{n}[2] \dicFlx{pl}[1] \dicFieldCat{poč.} \dicDirectTranslationCS{zalomení řádku}
\dicEntry[líra] \dicTerm{lír|a} \dicIPA{{l}{i}{\textlengthmark}{r}{a}} \dicPos{f}[1] \dicFlx{(‑u, ‑ur)}[7] \textbf{1.} \dicDirectTranslationCS{lira} \dicIndirectTranslationCS{(bývalá italská měna)}  \textbf{2.} \dicFieldCat{hud.} \dicDirectTranslationCS{lyra}
\dicEntry[lít] \dicTerm{lít} \dicIPA{{l}{i}{\textlengthmark}{\textsubring{d}}} \dicPos{v} \dicFlx{ind praes sg 1 pers} \dicLink{líta}
\dicEntry[líta] \dicTerm{líta} \dicsymFrequent\  \dicIPA{{l}{i}{\textlengthmark}{\textsubring{d}}{a}} \dicPos{v}[6] \dicFlx{(lít, leit, litum, liti, litið)}[76] \dicFlx{acc} \dicDirectTranslationCS{(po)dívat se, hledět, pohlédnout, pohlížet} \dicExampleIS{líta á klukkuna} \dicExampleCS{podívat se na hodiny};  \dicPhraseIS{líta e‑n} \dicDirectTranslationCS{pohlédnout na (koho)};  \dicPhraseIS{líta á e‑n} \dicDirectTranslationCS{podívat se na (koho)} \dicExampleIS{líta spyrjandi á hana} \dicExampleCS{podívat se na ni tázavě};  \dicIdiom{líta}[af]{ \dicPhraseIS{líta af e‑u\,/\addthin e‑m}} \dicDirectTranslationCS{moct spustit oči z~(čeho\,/\addthin koho)} \dicIndirectTranslationCS{(většinou v~záporu)};  \dicIdiom{líta}[á]{ \dicPhraseIS{líta á e‑ð}} \dicDirectTranslationCS{podívat se na (co), kouknout na (co), prozkoumat (co)} \dicExampleIS{líta á vélina} \dicExampleCS{kouknout na motor};  \dicPhraseIS{líta stórt á sig} \dicDirectTranslationCS{mít o~sobě vysoké mínění};  \dicIdiom{líta}[eftir]{ \dicPhraseIS{líta eftir e‑m}} \dicDirectTranslationCS{dohlédnout\,/\addthin dohlížet na (koho)} \dicExampleIS{líta eftir barninu} \dicExampleCS{dohlížet na dítě};  \dicIdiom{líta}[fram hjá]{ \dicPhraseIS{líta fram hjá e‑u}} \dicDirectTranslationCS{přehlédnout\,/\addthin přehlížet (co)} \dicExampleIS{líta fram hjá mikilvægum staðreyndum} \dicExampleCS{přehlížet důležitá fakta};  \dicIdiom{líta}[inn]{ \dicPhraseIS{líta inn hjá e‑m}} \dicDirectTranslationCS{(na)kouknout ke (komu), (za)stavit se u~(koho) (na návštěvu ap.)};  \dicIdiom{líta}[í]{ \dicPhraseIS{líta í e‑ð}} \dicDirectTranslationCS{podívat se do (čeho), nahlédnout do (čeho) (novin ap.)};  \dicIdiom{líta}[niður]{ \dicPhraseIS{líta niður á e‑n}} \dicDirectTranslationCS{dívat se na (koho) spatra, shlížet na (koho) povýšeně};  \dicIdiom{líta}[til]{ \dicPhraseIS{líta til e‑rs}} \dicDirectTranslationCS{zastavit se u~(koho), podívat se ke (komu)};  \dicIdiom{líta}[undan]{ \dicPhraseIS{líta undan}} \dicDirectTranslationCS{sklopit zrak, podívat se dolů};  \dicIdiom{líta}[upp]{ \dicPhraseIS{líta upp til e‑rs}} \dicDirectTranslationCS{obdivovat (koho), vzhlížet ke (komu) (s~úctou ap.)};  \dicIdiom{líta}[út]{ \dicPhraseIS{líta út}} \dicDirectTranslationCS{vypadat, vyhlížet} \dicExampleIS{líta vel út} \dicExampleCS{vypadat dobře}; { \dicPhraseIS{líta út fyrir að (vera ríkur)}} \dicDirectTranslationCS{vypadat na (boháče), vypadat jako (bohatý)};  \dicIdiom{líta}[við]{ \dicPhraseIS{líta ekki við e‑u}} \dicDirectTranslationCS{netknout se (čeho), nejevit zájem o~(co)}; { \dicPhraseIS{líta við (þar)}} \dicDirectTranslationCS{stavit se (tam) na skok, zaskočit (tam)};  \dicIdiom{líta}[yfir]{ \dicPhraseIS{líta yfir e‑ð}} \dicDirectTranslationCS{projít (co), prohlédnout (co)} \dicExampleIS{líta yfir textann} \dicExampleCS{projít text};  \dicIdiom{lítast}[á]{ \dicPhraseIS{e‑m líst vel á e‑ð}} \dicFlx{refl impers} \dicDirectTranslationCS{(co) se (komu) líbí}
\dicEntry[lítið] \dicTerm{lítið} \dicsymFrequent\  \dicIPA{{l}{i}{\textlengthmark}{\textsubring{d}}{\textsci}{\texttheta}} \dicPos{adv} \dicFlx{(comp minna, sup minnst)} \dicDirectTranslationCS{málo, trochu} \dicExampleIS{vera lítið áberandi} \dicExampleCS{být trochu nápadný}
\dicEntry[lítilfjörlegur] \dicTerm{lítil··fjör·legur} \dicIPA{{l}{i}{\textlengthmark}{\textsubring{d}}{\textsci}{l}{f}{j}{\oe}{r}{l}{\textepsilon}{\textbabygamma}{\textscy}{\textsubring{r}}} \dicPos{adj}[1]\dicFlx{}[-8] \textbf{1.} \dicSynonym{ómerkilegur} \dicDirectTranslationCS{nezajímavý, nezáživný, všední}  \textbf{2.} \dicSynonym{lasinn} \dicDirectTranslationCS{indisponovaný, slabý}
\dicEntry[lítill] \dicTerm{lítill} \dicsymFrequent\  \dicIPA{{l}{i}{\textlengthmark}{\textsubring{d}}{\textsci}{\textsubring{d}}{\textsubring{l}}} \dicPos{adj}[11] \dicFlx{(comp minni, sup minnstur)}[10] \textbf{1.} \dicSynonym{smár} \dicDirectTranslationCS{malý, drobný} \dicExampleIS{lítill maður} \dicExampleCS{malý člověk} \dicAntonym{stór}  \textbf{2.} \dicSynonym*{þann yngri til aðgreiningar frá þeim eldri} \dicDirectTranslationCS{malý, mladší} \dicExampleIS{Sölvi litli} \dicExampleCS{malý Sölvi}  \textbf{3.} \dicSynonym{óverulegur} \dicDirectTranslationCS{malý, zanedbatelný, nepatrný}  \textbf{4.} \dicSynonym{óduglegur} \dicDirectTranslationCS{zahálčivý, málo snaživý} \dicExampleIS{lítill íþróttamaður} \dicExampleCS{malý sportovec};  \dicIdiom{lítill}{ \dicPhraseIS{vera lítill í sér}} \dicDirectTranslationCS{být zbabělý}
\dicEntry[lítillátur] \dicTerm{lítil··látur} \dicIPA{{l}{i}{\textlengthmark}{\textsubring{d}}{\textsci}{l}{au}{\textsubring{d}}{\textscy}{\textsubring{r}}} \dicPos{adj}[1]\dicFlx{}[-1] \textbf{1.} \dicSynonym*{hrokalaus} \dicDirectTranslationCS{pokorný}  \textbf{2.} \dicSynonym{auðmjúkur} \dicDirectTranslationCS{nenáročný, skromný}
\dicEntry[lítillækka] \dicTerm{lítil··lækk|a} \dicIPA{{l}{i}{\textlengthmark}{\textsubring{d}}{\textsci}{l}{a}{i}{h}{\r{g}}{a}} \dicPos{v}[1] \dicFlx{(‑aði)}[1] \dicFlx{acc} \dicDirectTranslationCS{pokořit, pokořovat, ponížit, ponižovat} \dicExampleIS{lítillækka sig} \dicExampleCS{ponížit se}
\dicEntry[lítillækkun] \dicTerm{lítil··lækk|un} \dicIPA{{l}{i}{\textlengthmark}{\textsubring{d}}{\textsci}{l}{a}{i}{h}{\r{g}}{\textscy}{\textsubring{n}}} \dicPos{f}[7] \dicFlx{(‑unar)}[9] \dicDirectTranslationCS{pokoření, ponížení}
\dicEntry[lítillæti] \dicTerm{lítil··læti} \dicIPA{{l}{i}{\textlengthmark}{\textsubring{d}}{\textsci}{l}{a}{i}{\textsubring{d}}{\textsci}} \dicPos{n}[2] \dicFlx{(‑s)}[20] \textbf{1.} \dicSynonym{yfirlætisleysi} \dicDirectTranslationCS{pokora}  \textbf{2.} \dicSynonym{auðmýkt} \dicDirectTranslationCS{skromnost}
\dicEntry[lítilmagni] \dicTerm{lítil··magn|i} \dicIPA{{l}{i}{\textsubring{d}}{\textsci}{l}{m}{a}{\r{g}}{n}{\textsci}} \dicPos{m}[1] \dicFlx{(‑a, ‑ar)}[8] \dicDirectTranslationCS{padavka, outsider(ka)}
\dicEntry[lítilmenni] \dicTerm{lítil··menni} \dicIPA{{l}{i}{\textlengthmark}{\textsubring{d}}{\textsci}{l}{m}{\textepsilon}{n}{\textsci}} \dicPos{n}[2] \dicFlx{(‑s, ‑)}[14] \textbf{1.} \dicDirectTranslationCS{malý člověk (vzrůstem)}  \textbf{2.} \dicDirectTranslationCS{malicherný\,/\addthin nanicovatý člověk}
\dicEntry[lítilmótlegur] \dicTerm{lítil··mót·legur} \dicIPA{{l}{i}{\textlengthmark}{\textsubring{d}}{\textsci}{l}{m}{ou}{\textsubring{d}}{l}{\textepsilon}{\textbabygamma}{\textscy}{\textsubring{r}}} \dicPos{adj}[1]\dicFlx{}[-8] \dicSynonym{ómerkilegur} \dicDirectTranslationCS{ubohý, mizerný, mrzký}
\dicEntry[lítilræði] \dicTerm{lítil··ræði} \dicIPA{{l}{i}{\textlengthmark}{\textsubring{d}}{\textsci}{l}{r}{a}{i}{ð}{\textsci}} \dicPos{n}[2] \dicFlx{(‑s)}[20] \dicDirectTranslationCS{maličkost, maličko}
\dicEntry[lítilsháttar] \dicTerm{lítils··háttar\smash{\textsuperscript{1}}} \dicIPA{{l}{i}{\textlengthmark}{\textsubring{d}}{\textsci}{l}{s}{h}{au}{h}{\textsubring{d}}{a}{\textsubring{r}}} \dicPos{adj}[13] \dicFlx{indecl}[1] \dicDirectTranslationCS{malý, drobný}
\dicEntry[lítilsháttar] \dicTerm{lítils··háttar\smash{\textsuperscript{2}}} \dicIPA{{l}{i}{\textlengthmark}{\textsubring{d}}{\textsci}{l}{s}{h}{au}{h}{\textsubring{d}}{a}{\textsubring{r}}} \dicPos{adv} \dicDirectTranslationCS{málo, trochu}
\dicEntry[lítilsigldur] \dicTerm{lítil··sigldur} \dicIPA{{l}{i}{\textlengthmark}{\textsubring{d}}{\textsci}{l}{s}{\textsci}{l}{\textsubring{d}}{\textscy}{\textsubring{r}}} \dicPos{adj}[2]\dicFlx{}[-14] \dicSynonym{lítilfjörlegur} \dicDirectTranslationCS{nevýznamný, ubohý, nicotný}
\dicEntry[lítilsvirða] \dicTerm{lítils··vir|ða} \dicIPA{{l}{i}{\textlengthmark}{\textsubring{d}}{\textsci}{l}{s}{v}{\textsci}{r}{ð}{a}} \dicPos{v}[2] \dicFlx{(‑ti, ‑t)}[52] \dicFlx{acc} \dicDirectTranslationCS{pohrdat, opovrhovat}
\dicEntry[lítilsvirðing] \dicTerm{lítils··virð·ing} \dicIPA{{l}{i}{\textlengthmark}{\textsubring{d}}{\textsci}{l}{s}{v}{\textsci}{r}{ð}{i}{\ng}{\r{g}}} \dicPos{f}[4] \dicFlx{(‑ar)}[7] \dicSynonym{óvirðing} \dicDirectTranslationCS{opovržení, pohrdání}
\dicEntry[lítilvægur] \dicTerm{lítil··vægur} \dicIPA{{l}{i}{\textlengthmark}{\textsubring{d}}{\textsci}{l}{v}{a}{i}{\textbabygamma}{\textscy}{\textsubring{r}}} \dicPos{adj}[1]\dicFlx{}[-1] \dicSynonym*{þýðingarlítill} \dicDirectTranslationCS{nedůležitý, nepodstatný, bezvýznamný}
\dicEntry[lítilþægur] \dicTerm{lítil··þægur} \dicIPA{{l}{i}{\textlengthmark}{\textsubring{d}}{\textsci}{l}{\texttheta}{a}{i}{\textbabygamma}{\textscy}{\textsubring{r}}} \dicPos{adj}[1]\dicFlx{}[-1] \dicDirectTranslationCS{skromný, nenáročný} \dicIndirectTranslationCS{(často používané v~záporu)}
\dicEntry[lítri] \dicTerm{lítr|i} \dicIPA{{l}{i}{\textlengthmark}{\textsubring{d}}{r}{\textsci}} \dicPos{m}[1] \dicFlx{(‑a, ‑ar)}[1] \dicDirectTranslationCS{litr}
\dicEntry[lítt] \dicTerm{lítt} \dicsymFrequent\  \dicIPA{{l}{i}{h}{\textsubring{d}}} \dicPos{adv} \dicFlx{(comp miður, sup minnst)} \textbf{1.} \dicSynonym{lítið} \dicDirectTranslationCS{trochu, málo} \dicExampleIS{lítt þekktur ráðherra} \dicExampleCS{málo známý ministr}  \textbf{2.} \dicSynonym{illa} \dicDirectTranslationCS{špatně}
\dicEntry[ljá] \dicTerm{ljá} \dicIPA{{l}{j}{au}{\textlengthmark}} \dicPos{v}[5] \dicFlx{(ljæ, léði, léðum, léði, léð)}[15] \dicFlx{dat + acc} \textbf{1.} \dicLangCat{zast.} \dicDirectTranslationCS{(vy)půjčit} \dicExampleIS{ljá e‑m e‑ð} \dicExampleCS{půjčit (komu co)}  \textbf{2.} \dicPhraseIS{ljá e‑m eyra} \dicLangCat{přen.} \dicDirectTranslationCS{vyslyšet (koho), dopřát (komu) sluchu}  \textbf{3.} \dicDirectTranslationCS{vnést, vnášet, přidat, přidávat} \dicIndirectTranslationCS{(způsobit uplatnění, začlenění něčeho)}  \textbf{4.} \dicPhraseIS{ljá máls á e‑u} \dicDirectTranslationCS{zavést téma na (co)}
\dicEntry[ljár] \dicTerm{ljá|r} \dicIPA{{l}{j}{au}{\textlengthmark}{\textsubring{r}}} \dicPos{m}[9] \dicFlx{(‑s, ‑ir)}[12] \dicDirectTranslationCS{kosa} \dicIndirectTranslationCS{(ruční nástroj)};  \dicPhraseIS{slá með orfi og ljá} \dicDirectTranslationCS{sekat kosou};  \dicPhraseIS{vera e‑m óþægur ljár í þúfu} \dicLangCat{přen.} \dicDirectTranslationCS{být (komu) trnem v~oku}
\dicEntry[ljóð] \dicTerm{ljóð} \dicsymFrequent\  \dicIPA{{l}{j}{ou}{\textlengthmark}{\texttheta}} \dicPos{n}[2] \dicFlx{(‑s, ‑)}[5] \dicDirectTranslationCS{báseň} \dicExampleIS{ljóð eftir e‑n} \dicExampleCS{(čí) báseň}
\dicEntry[ljóðabók] \dicTerm{ljóða··|bók} \dicIPA{{l}{j}{ou}{\textlengthmark}{ð}{a}{\textsubring{b}}{ou}{\r{g}}} \dicPos{f}[8] \dicFlx{(‑bókar, ‑bækur)}[5] \dicDirectTranslationCS{sbírka\,/\addthin kniha poezie}
\dicEntry[ljóðaháttur] \dicTerm{ljóða··|háttur} \dicIPA{{l}{j}{ou}{\textlengthmark}{ð}{a}{h}{au}{h}{\textsubring{d}}{\textscy}{\textsubring{r}}} \dicPos{m}[12] \dicFlx{(‑háttar, ‑hættir)}[7] \dicFieldCat{lit.} \dicDirectTranslationCS{eddické metrum (šestiverší)}
\dicEntry[ljóðlist] \dicTerm{ljóð··list} \dicIPA{{l}{j}{ou}{ð}{l}{\textsci}{s}{\textsubring{d}}} \dicPos{f}[7] \dicFlx{(‑ar)}[3] \dicDirectTranslationCS{básnictví, básnické umění}
\dicEntry[ljóðmæli] \dicTerm{ljóð··mæli} \dicIPA{{l}{j}{ou}{ð}{m}{a}{i}{l}{\textsci}} \dicPos{n}[2] \dicFlx{pl}[19] \dicDirectTranslationCS{sbírka poezie}
\dicEntry[ljóðræna] \dicTerm{ljóð··ræn|a} \dicIPA{{l}{j}{ou}{ð}{r}{a}{i}{n}{a}} \dicPos{f}[1] \dicFlx{(‑u)}[5] \dicFieldCat{lit.} \dicDirectTranslationCS{lyrika}
\dicEntry[ljóðrænn] \dicTerm{ljóð··rænn} \dicIPA{{l}{j}{ou}{ð}{r}{a}{i}{\textsubring{d}}{\textsubring{n}}} \dicPos{adj}[7]\dicFlx{}[-1] \dicDirectTranslationCS{poetický, lyrický}
\dicEntry[ljóðskáld] \dicTerm{ljóð··skáld} \dicIPA{{l}{j}{ou}{ð}{s}{\r{g}}{au}{l}{\textsubring{d}}} \dicPos{n}[2] \dicFlx{(‑s, ‑)}[5] \dicDirectTranslationCS{básník, básnířka, poeta, poetka}
\dicEntry[ljóður] \dicTerm{ljóð|ur} \dicIPA{{l}{j}{ou}{\textlengthmark}{ð}{\textscy}{\textsubring{r}}} \dicPos{m}[6] \dicFlx{(‑s)}[26] \dicSynonym{lýti} \dicDirectTranslationCS{kaz, vada}
\dicEntry[ljóma] \dicTerm{ljóm|a} \dicIPA{{l}{j}{ou}{\textlengthmark}{m}{a}} \dicPos{v}[1] \dicFlx{(‑aði)}[1] \dicDirectTranslationCS{zářit, vyzařovat} \dicExampleIS{Sólin ljómar á himninum.} \dicExampleCS{Slunce září na obloze.};  \dicPhraseIS{ljóma af gleði} \dicLangCat{přen.} \dicDirectTranslationCS{zářit štěstím}
\dicEntry[ljómandi] \dicTerm{ljóm··andi\smash{\textsuperscript{1}}} \dicIPA{{l}{j}{ou}{\textlengthmark}{m}{a}{n}{\textsubring{d}}{\textsci}} \dicPos{adj}[13] \dicFlx{indecl}[1] \dicDirectTranslationCS{zářivý, zářící} \dicExampleIS{ljómandi sól} \dicExampleCS{zářící slunce}
\dicEntry[ljómandi] \dicTerm{ljóm··andi\smash{\textsuperscript{2}}} \dicIPA{{l}{j}{ou}{\textlengthmark}{m}{a}{n}{\textsubring{d}}{\textsci}} \dicPos{adv} \dicDirectTranslationCS{úžasně, fantasticky} \dicExampleIS{ljómandi vel} \dicExampleCS{fantasticky dobře}
\dicEntry[ljómi] \dicTerm{ljóm|i} \dicIPA{{l}{j}{ou}{\textlengthmark}{m}{\textsci}} \dicPos{m}[1] \dicFlx{(‑a)}[3] \textbf{1.} \dicSynonym{birta\smash{\textsuperscript{1}}} \dicDirectTranslationCS{zář(e), svit}  \textbf{2.} \dicFieldCat{fyz.} \dicDirectTranslationCS{jas}
\dicEntry[ljón] \dicTerm{ljón} \dicsymFrequent\  \dicIPA{{l}{j}{ou}{\textlengthmark}{\textsubring{n}}} \dicPos{n}[2] \dicFlx{(‑s, ‑)}[5] \textbf{1.} \dicFieldCat{zool.} \dicDirectTranslationCS{lev} \textit{(l.~{\textLA{Panthera leo}})}  \dicsymPhoto\  \dicExampleIS{Ljón hleypur í átt að manni.} \dicExampleCS{Směrem k~člověku běží lev.}  \textbf{2.} \dicDirectTranslationCS{Lev} \dicIndirectTranslationCS{(znamení zvěrokruhu)} \textit{(l.~{\textLA{Leo}})}
\dicFigure{20786.jpg}{Ljón}{Ljón - Stansell, Ken, Biolib, PD}
\dicEntry[ljónatemjari] \dicTerm{ljóna··temj·ar|i} \dicIPA{{l}{j}{ou}{\textlengthmark}{n}{a}{t\smash{\textsuperscript{h}}}{\textepsilon}{m}{j}{a}{r}{\textsci}} \dicPos{m}[1] \dicFlx{(‑a, ‑ar)}[13] \dicDirectTranslationCS{krotitel(ka) lvů\,/\addthin šelem\,/\addthin dravé zvěře}
\dicEntry[ljónslappi] \dicTerm{ljóns··lapp|i} \dicIPA{{l}{j}{ou}{n}{s}{l}{a}{h}{\textsubring{b}}{\textsci}} \dicPos{m}[1] \dicFlx{(‑a)}[3] \dicFieldCat{bot.} \dicDirectTranslationCS{kontryhel alpinský} \textit{(l.~{\textLA{Alchemilla alpina}})}  \dicsymPhoto\ 
\dicFigure{88632.jpg}{Ljónslappi}{Ljónslappi - Pleva František, Biolib, PD}
\dicEntry[ljónynja] \dicTerm{ljón··ynj|a} \dicIPA{{l}{j}{ou}{\textlengthmark}{n}{\textsci}{n}{j}{a}} \dicPos{f}[1] \dicFlx{(‑u, ‑ur)}[7] \dicDirectTranslationCS{lvice}
\dicEntry[ljóri] \dicTerm{ljór|i} \dicIPA{{l}{j}{ou}{\textlengthmark}{r}{\textsci}} \dicPos{m}[1] \dicFlx{(‑a, ‑ar)}[1] \dicDirectTranslationCS{světlík}
\dicEntry[ljós] \dicTerm{ljós\smash{\textsuperscript{1}}} \dicsymFrequent\  \dicIPA{{l}{j}{ou}{\textlengthmark}{s}} \dicPos{n}[2] \dicFlx{(‑s, ‑)}[5] \textbf{1.} \dicSynonym{birta\smash{\textsuperscript{1}}} \dicDirectTranslationCS{světlo, záře, jas, svit} \dicExampleIS{ljósið frá lampanum} \dicExampleCS{světlo lampy};  \dicPhraseIS{kveikja ljósið} \dicDirectTranslationCS{rozsvítit, zapnout\,/\addthin zapálit světlo};  \dicPhraseIS{slökkva ljósið} \dicDirectTranslationCS{zhasnout, vypnout světlo}  \textbf{2.} \dicSynonym*{þægt barn} \dicDirectTranslationCS{zlatíčko, andílek} \dicIndirectTranslationCS{(o~poslušném dítěti)};  \dicIdiom{ljós}{ \dicPhraseIS{í nýju ljósi}} \dicFlx{adv} \dicLangCat{přen.} \dicDirectTranslationCS{v~novém světle}; { \dicPhraseIS{það kom í ljós að}} \dicDirectTranslationCS{ukázalo se, že; vyšlo najevo, že}
\dicEntry[ljós] \dicTerm{ljós\smash{\textsuperscript{2}}} \dicsymFrequent\  \dicIPA{{l}{j}{ou}{\textlengthmark}{s}} \dicPos{adj}[5]\dicFlx{}[-1] \dicDirectTranslationCS{jasný, světlý} \dicExampleIS{vera ljós á húð} \dicExampleCS{mít světlou pokožku};  \dicPhraseIS{gera sér e‑ð ljóst} \dicDirectTranslationCS{uvědomit si (co)};  \dicPhraseIS{e‑m er ljóst e‑að} \dicFlx{impers} \dicDirectTranslationCS{(co) je (komu) jasné};  \dicPhraseIS{e‑að er deginum ljósara} \dicLangCat{přen.} \dicDirectTranslationCS{(co) je nad slunce jasnější, (co) je jasné jak facka}
\dicEntry[ljósakróna] \dicTerm{ljósa··krón|a} \dicIPA{{l}{j}{ou}{\textlengthmark}{s}{a}{k\smash{\textsuperscript{h}}}{r}{ou}{n}{a}} \dicPos{f}[1] \dicFlx{(‑u, ‑ur)}[7] \dicDirectTranslationCS{(mnohoramenný) lustr}
\dicEntry[ljósapera] \dicTerm{ljósa··per|a} \dicIPA{{l}{j}{ou}{\textlengthmark}{s}{a}{p\smash{\textsuperscript{h}}}{\textepsilon}{r}{a}} \dicPos{f}[1] \dicFlx{(‑u, ‑ur)}[7] \dicFieldCat{elek.} \dicDirectTranslationCS{žárovka}
\dicEntry[ljósaskilti] \dicTerm{ljósa··skilti} \dicIPA{{l}{j}{ou}{\textlengthmark}{s}{a}{s}{\r{\textObardotlessj}}{\textsci}{\textsubring{l}}{\textsubring{d}}{\textsci}} \dicPos{n}[2] \dicFlx{(‑s, ‑)}[14] \dicDirectTranslationCS{neonová reklama}
\dicEntry[ljósaskipti] \dicTerm{ljósa··skipti} \dicIPA{{l}{j}{ou}{\textlengthmark}{s}{a}{s}{\r{\textObardotlessj}}{\textsci}{f}{\textsubring{d}}{\textsci}} \dicPos{n}[2] \dicFlx{pl}[19] \dicSynonym{rökkur} \dicDirectTranslationCS{soumrak}
\dicEntry[ljósastaur] \dicTerm{ljósa··staur} \dicIPA{{l}{j}{ou}{\textlengthmark}{s}{a}{s}{\textsubring{d}}{\oe i}{\textsubring{r}}} \dicPos{m}[4] \dicFlx{(‑s, ‑ar)}[14] \dicDirectTranslationCS{kandelábr, stojan veřejného osvětlení}
\dicEntry[ljósastæði] \dicTerm{ljósa··stæði} \dicIPA{{l}{j}{ou}{\textlengthmark}{s}{a}{s}{\textsubring{d}}{a}{i}{ð}{\textsci}} \dicPos{n}[2] \dicFlx{(‑s, ‑)}[14] \dicDirectTranslationCS{přípojka na světlo}
\dicEntry[Ljósálfheimur] \dicTerm{Ljós··álf·heim|ur}\dicTerm{, Álfheimur} \dicIPA{{l}\-{j}\-{ou}\-{\textlengthmark}\-{s}\-{au}\-{l}\-{f}\-{h}\-{ei}\-{m}\-{\textscy}\-{\textsubring{r}}\-} \dicPos{m}[6] \dicFlx{(‑s)}[3] \dicFieldCat{myt.} \dicDirectTranslationCS{Álfheim} \dicIndirectTranslationCS{(říše světla, světlých elfů)} \dicAntonym{Svartálfheimur}
\dicEntry[ljósár] \dicTerm{ljós··ár} \dicIPA{{l}{j}{ou}{\textlengthmark}{s}{au}{\textsubring{r}}} \dicPos{n}[2] \dicFlx{(‑s, ‑)}[5] \dicFieldCat{astro.} \dicDirectTranslationCS{světelný rok}
\dicEntry[ljósberi] \dicTerm{ljós··ber|i} \dicIPA{{l}{j}{ou}{s}{\textsubring{b}}{\textepsilon}{r}{\textsci}} \dicPos{m}[1] \dicFlx{(‑a, ‑ar)}[1] \dicFieldCat{bot.} \dicDirectTranslationCS{kohoutek alpský} \textit{(l.~{\textLA{Lychnis alpina}})}  \dicsymPhoto\ 
\dicFigure{72890.jpg}{Ljósberi}{Ljósberi - Kesl Michael, Biolib, Copyright/CC-BY-NC}
\dicEntry[ljóseind] \dicTerm{ljós··eind} \dicIPA{{l}{j}{ou}{\textlengthmark}{s}{ei}{n}{\textsubring{d}}} \dicPos{f}[7] \dicFlx{(‑ar, ‑ir)}[1] \dicFieldCat{fyz.} \dicDirectTranslationCS{foton}
\dicEntry[ljósfræði] \dicTerm{ljós··fræð|i} \dicIPA{{l}{j}{ou}{s}{f}{r}{a}{i}{ð}{\textsci}} \dicPos{f}[3] \dicFlx{(‑i)}[3] \dicFieldCat{fyz.} \dicDirectTranslationCS{optika}
\dicEntry[ljósgeisli] \dicTerm{ljós··geisl|i} \dicIPA{{l}{j}{ou}{s}{\r{\textObardotlessj}}{ei}{s}{\textsubring{d}}{l}{\textsci}} \dicPos{m}[1] \dicFlx{(‑a, ‑ar)}[1] \dicDirectTranslationCS{světelný paprsek, paprsek světla}
\dicEntry[ljósgulur] \dicTerm{ljós··gulur} \dicIPA{{l}{j}{ou}{s}{\r{g}}{\textscy}{l}{\textscy}{\textsubring{r}}} \dicPos{adj}[1]\dicFlx{}[-1] \dicDirectTranslationCS{světle žlutý, nažloutlý}
\dicEntry[ljóshraði] \dicTerm{ljós··hrað|i} \dicIPA{{l}{j}{ou}{\textlengthmark}{s}{\textsubring{r}}{a}{ð}{\textsci}} \dicPos{m}[1] \dicFlx{(‑a)}[3] \dicFieldCat{fyz.} \dicDirectTranslationCS{rychlost světla}
\dicEntry[ljóshærður] \dicTerm{ljós··hærður} \dicsymFrequent\  \dicIPA{{l}{j}{ou}{\textlengthmark}{s}{h}{a}{i}{r}{ð}{\textscy}{\textsubring{r}}} \dicPos{adj}[2]\dicFlx{}[-1] \dicDirectTranslationCS{blonďatý, světlovlasý} \dicExampleIS{Hann er ljóshærður.} \dicExampleCS{Má světlé vlasy.}
\dicEntry[ljóshöfðaönd] \dicTerm{ljós·höfða··|önd} \dicIPA{{l}{j}{ou}{\textlengthmark}{s}{h}{\oe}{v}{ð}{a}{\oe}{n}{\textsubring{d}}} \dicPos{f}[8] \dicFlx{(‑andar, ‑endur)}[2] \dicFieldCat{zool.} \dicDirectTranslationCS{hvízdák americký} \textit{(l.~{\textLA{Anas americana}})}  \dicsymPhoto\ 
\dicFigure{ds_image_ljoshofdaond_0_2.jpg}{Ljóshöfðaönd}{Ljóshöfðaönd - Donna Dewhurst, USFWS, PD}
\dicEntry[ljóska] \dicTerm{ljósk|a} \dicIPA{{l}{j}{ou}{s}{\r{g}}{a}} \dicPos{f}[1] \dicFlx{(‑u, ‑ur)}[19] \dicDirectTranslationCS{blondýna, blondýnka}
\dicEntry[ljóskastari] \dicTerm{ljós··kast·ar|i} \dicIPA{{l}{j}{ou}{s}{k\smash{\textsuperscript{h}}}{a}{s}{\textsubring{d}}{a}{r}{\textsci}} \dicPos{m}[1] \dicFlx{(‑a, ‑ar)}[10] \dicDirectTranslationCS{světlomet, reflektor}
\dicEntry[ljósker] \dicTerm{ljós··ker} \dicIPA{{l}{j}{ou}{s}{c\smash{\textsuperscript{h}}}{\textepsilon}{\textsubring{r}}} \dicPos{n}[2] \dicFlx{(‑s, ‑)}[6] \dicDirectTranslationCS{lampa, svítilna, lucerna}
\dicEntry[ljóslifandi] \dicTerm{ljós··lif·andi} \dicIPA{{l}{j}{ou}{s}{l}{\textsci}{v}{a}{n}{\textsubring{d}}{\textsci}} \dicPos{adj}[13] \dicFlx{indecl}[1] \dicDirectTranslationCS{živý, barvitý} \dicExampleIS{e‑að stendur e‑m ljóslifandi fyrir hugskotssjónum} \dicExampleCS{(co) má (kdo) v~živé paměti}
\dicEntry[ljósmerki] \dicTerm{ljós··merki} \dicIPA{{l}{j}{ou}{s}{m}{\textepsilon}{\textsubring{r}}{\r{\textObardotlessj}}{\textsci}} \dicPos{n}[2] \dicFlx{(‑s, ‑)}[16] \dicDirectTranslationCS{světelný signál, světelné znamení}
\dicEntry[ljósmóðir] \dicTerm{ljós··|móðir} \dicIPA{{l}{j}{ou}{s}{m}{ou}{ð}{\textsci}{\textsubring{r}}} \dicPos{f}[11] \dicFlx{(‑móður, ‑mæður)}[1] \dicDirectTranslationCS{porodní asistentka}
\dicEntry[ljósmynd] \dicTerm{ljós··mynd} \dicsymFrequent\  \dicIPA{{l}{j}{ou}{s}{m}{\textsci}{n}{\textsubring{d}}} \dicPos{f}[7] \dicFlx{(‑ar, ‑ir)}[1] \dicDirectTranslationCS{fotografie, fotka} \dicExampleIS{svarthvít ljósmynd af ungri konu} \dicExampleCS{černobílá fotografie mladé ženy}
\dicEntry[ljósmynda] \dicTerm{ljós··mynd|a} \dicIPA{{l}{j}{ou}{s}{m}{\textsci}{n}{\textsubring{d}}{a}} \dicPos{v}[1] \dicFlx{(‑aði)}[1] \dicDirectTranslationCS{(vy)fotit, (vy)fotografovat, (u)dělat fotografii}
\dicEntry[ljósmyndari] \dicTerm{ljós··mynd·ar|i} \dicIPA{{l}{j}{ou}{s}{m}{\textsci}{n}{\textsubring{d}}{a}{r}{\textsci}} \dicPos{m}[1] \dicFlx{(‑a, ‑ar)}[13] \dicDirectTranslationCS{fotograf(ka)}
\dicEntry[ljósmyndavél] \dicTerm{ljós··mynda·vél} \dicIPA{{l}{j}{ou}{s}{m}{\textsci}{n}{\textsubring{d}}{a}{v}{j}{\textepsilon}{\textsubring{l}}} \dicPos{f}[4] \dicFlx{(‑ar, ‑ar)}[1] \dicDirectTranslationCS{fotoaparát}
\dicEntry[ljósmyndun] \dicTerm{ljós··mynd|un} \dicIPA{{l}{j}{ou}{s}{m}{\textsci}{n}{\textsubring{d}}{\textscy}{\textsubring{n}}} \dicPos{f}[7] \dicFlx{(‑unar, ‑anir)}[8] \textbf{1.} \dicDirectTranslationCS{fotografování}  \textbf{2.} \dicDirectTranslationCS{fotografie} \dicIndirectTranslationCS{(umělecký obor)}
\dicEntry[ljósmælir] \dicTerm{ljós··mæl|ir} \dicIPA{{l}{j}{ou}{s}{m}{a}{i}{l}{\textsci}{\textsubring{r}}} \dicPos{m}[7] \dicFlx{(‑is, ‑ar)}[1] \dicFieldCat{fyz.} \dicDirectTranslationCS{expozimetr}
\dicEntry[ljósnemi] \dicTerm{ljós··nem|i} \dicIPA{{l}{j}{ou}{s}{n}{\textepsilon}{m}{\textsci}} \dicPos{m}[1] \dicFlx{(‑a, ‑ar)}[1] \dicFieldCat{fyz.} \dicDirectTranslationCS{fotočlánek}
\dicEntry[ljósnæmur] \dicTerm{ljós··næmur} \dicIPA{{l}{j}{ou}{s}{n}{a}{i}{m}{\textscy}{\textsubring{r}}} \dicPos{adj}[1]\dicFlx{}[-1] \dicDirectTranslationCS{citlivý na světlo, fotosenzitivní}
\dicEntry[ljósrit] \dicTerm{ljós··rit} \dicIPA{{l}{j}{ou}{s}{r}{\textsci}{\textsubring{d}}} \dicPos{n}[2] \dicFlx{(‑s, ‑)}[5] \dicDirectTranslationCS{fotokopie}
\dicEntry[ljósrita] \dicTerm{ljós··rit|a} \dicIPA{{l}{j}{ou}{s}{r}{\textsci}{\textsubring{d}}{a}} \dicPos{v}[1] \dicFlx{(‑aði)}[1] \dicFlx{acc} \dicDirectTranslationCS{(o)kopírovat (pomocí fotokopírky)}
\dicEntry[ljósritunarvél] \dicTerm{ljós·ritunar··vél} \dicIPA{{l}{j}{ou}{s}{r}{\textsci}{\textsubring{d}}{\textscy}{n}{a}{r}{v}{j}{\textepsilon}{\textsubring{l}}} \dicPos{f}[4] \dicFlx{(‑ar, ‑ar)}[1] \dicDirectTranslationCS{(foto)kopírka}
\dicEntry[ljósta] \dicTerm{ljósta} \dicsymFrequent\  \dicIPA{{l}{j}{ou}{s}{\textsubring{d}}{a}} \dicPos{v}[6] \dicFlx{(lýst, laust, lustum, lysti, lostið)}[97] \dicFlx{acc} \dicLangCat{zast.} \dicDirectTranslationCS{udeřit, uhodit} \dicExampleIS{ljósta e‑n í andlitið með e‑u} \dicExampleCS{uhodit (koho) do tváře (čím)};  \dicIdiom{ljósta}[niður]{ \dicPhraseIS{e‑u lýstur niður}} \dicFlx{impers} \dicDirectTranslationCS{(co) udeří, (co) zasáhne (blesk ap.)} \dicExampleIS{Eldingu laust niður í húsið.} \dicExampleCS{Blesk udeřil do domu.};  \dicIdiom{ljósta}[saman]{ \dicPhraseIS{e‑m lýstur saman}} \dicFlx{impers} \dicDirectTranslationCS{(kdo) se střetne} \dicExampleIS{Herjunum laust saman.} \dicExampleCS{Armády se střetly.};  \dicIdiom{ljósta}[upp]{ \dicPhraseIS{ljósta upp ópi}} \dicDirectTranslationCS{vykřiknout}
\dicEntry[ljóstillífun] \dicTerm{ljós··til·líf|un} \dicIPA{{l}{j}{ou}{s}{t\smash{\textsuperscript{h}}}{\textsci}{l}{i}{v}{\textscy}{\textsubring{n}}} \dicPos{f}[7] \dicFlx{(‑unar)}[9] \dicFieldCat{biol.} \dicDirectTranslationCS{fotosyntéza}
\dicEntry[ljóstra] \dicTerm{ljóstr|a} \dicIPA{{l}{j}{ou}{s}{\textsubring{d}}{r}{a}} \dicPos{v}[1] \dicFlx{(‑aði)}[1] \dicFlx{dat} \dicPhraseIS{ljóstra e‑u upp} \dicDirectTranslationCS{odhalit\,/\addthin odhalovat (co), odkrýt\,/\addthin odkrývat (co)} \dicExampleIS{ljóstra upp leyndarmáli} \dicExampleCS{odkrýt tajemství}
\dicEntry[ljósvaki] \dicTerm{ljós··vak|i} \dicIPA{{l}{j}{ou}{\textlengthmark}{s}{v}{a}{\r{\textObardotlessj}}{\textsci}} \dicPos{m}[1] \dicFlx{(‑a, ‑ar)}[8] \dicDirectTranslationCS{éter} \dicIndirectTranslationCS{(podle starých představ látka zaplňující vesmírný prostor)};  \dicPhraseIS{á öldum ljósvakans} \dicFlx{adv} \dicDirectTranslationCS{na vlnách éteru}
\dicEntry[ljótleiki] \dicTerm{ljót··leik|i} \dicIPA{{l}{j}{ou}{\textlengthmark}{\textsubring{d}}{l}{ei}{\r{\textObardotlessj}}{\textsci}} \dicPos{m}[1] \dicFlx{(‑a)}[3] \dicSynonym{lýti} \dicDirectTranslationCS{šerednost, ohyzdnost (vzhled ap.)}
\dicEntry[ljótur] \dicTerm{ljótur} \dicsymFrequent\  \dicIPA{{l}{j}{ou}{\textlengthmark}{\textsubring{d}}{\textscy}{\textsubring{r}}} \dicPos{adj}[1]\dicFlx{}[-1] \dicDirectTranslationCS{šeredný, ošklivý, škaredý} \dicExampleIS{ljótar athugasemdir} \dicExampleCS{ošklivé poznámky}
\dicEntry[Ljubljana] \dicTerm{Ljubljana} \dicIPA{{l}{j}{\textscy}{\textsubring{b}}{l}{j}{a}{n}{a}} \dicPos{subs} \dicFlx{indecl} \dicFieldCat{geog.} \dicDirectTranslationCS{Lublaň} \dicIndirectTranslationCS{(hlavní město Slovinska)}
\dicEntry[ljúffengur] \dicTerm{ljúf··fengur} \dicIPA{{l}{j}{u}{f}{\textlengthmark}{ei}{\ng}{\r{g}}{\textscy}{\textsubring{r}}} \dicPos{adj}[1]\dicFlx{}[-1] \dicDirectTranslationCS{chutný, lahodný, delikátní} \dicIndirectTranslationCS{(o~jídle)}
\dicEntry[ljúflingur] \dicTerm{ljúf··ling|ur} \dicIPA{{l}{j}{u}{v}{l}{i}{\ng}{\r{g}}{\textscy}{\textsubring{r}}} \dicPos{m}[6] \dicFlx{(‑s, ‑ar)}[8] \textbf{1.} \dicSynonym{huldumaður} \dicDirectTranslationCS{elf}  \textbf{2.} \dicSynonym{eftirlæti} \dicDirectTranslationCS{miláček, zlato}
\dicEntry[ljúflyndi] \dicTerm{ljúf··lyndi} \dicIPA{{l}{j}{u}{v}{l}{\textsci}{n}{\textsubring{d}}{\textsci}} \dicPos{n}[2] \dicFlx{(‑s)}[20] \dicDirectTranslationCS{přívětivost, laskavost}
\dicEntry[ljúfmenni] \dicTerm{ljúf··menni} \dicIPA{{l}{j}{u}{v}{m}{\textepsilon}{n}{\textsci}} \dicPos{n}[2] \dicFlx{(‑s, ‑)}[14] \dicSynonym{góðmenni} \dicDirectTranslationCS{milý\,/\addthin přívětivý člověk}
\dicEntry[ljúfmennska] \dicTerm{ljúf··mennsk|a} \dicIPA{{l}{j}{u}{v}{m}{\textepsilon}{n}{s}{\r{g}}{a}} \dicPos{f}[1] \dicFlx{(‑u)}[5] \dicDirectTranslationCS{dobrota, přívětivost, vlídnost}
\dicEntry[ljúfur] \dicTerm{ljúf|ur\smash{\textsuperscript{1}}} \dicIPA{{l}{j}{u}{\textlengthmark}{v}{\textscy}{\textsubring{r}}} \dicPos{m}[6] \dicFlx{(‑s, ‑ar)}[24] \dicDirectTranslationCS{poklad, drahoušek, miláček} \dicExampleIS{ljúfur(inn) (minn)} \dicExampleCS{ty můj poklade}
\dicEntry[ljúfur] \dicTerm{ljúfur\smash{\textsuperscript{2}}} \dicsymFrequent\  \dicIPA{{l}{j}{u}{\textlengthmark}{v}{\textscy}{\textsubring{r}}} \dicPos{adj}[1]\dicFlx{}[-1] \textbf{1.} \dicSynonym{blíður} \dicDirectTranslationCS{laskavý, přívětivý, vlídný}  \textbf{2.} \dicSynonym{þægilegur} \dicDirectTranslationCS{příjemný, milý} \dicExampleIS{ljúfar minningar} \dicExampleCS{příjemné vzpomínky};  \dicPhraseIS{e‑m er ljúft að (gera e‑ð)} \dicFlx{impers} \dicDirectTranslationCS{(komu) je příjemné (udělat (co))}
\dicEntry[ljúga] \dicTerm{ljúga} \dicsymFrequent\  \dicIPA{{l}{j}{u}{\textlengthmark}{a}} \dicPos{v}[6] \dicFlx{(lýg, laug, lugum, lygi, logið)}[52] \dicFlx{dat} \dicDirectTranslationCS{lhát, zalhat} \dicExampleIS{ljúga e‑u} \dicExampleCS{zalhat o~(čem)};  \dicPhraseIS{ljúga að e‑m} \dicDirectTranslationCS{lhát (komu)};  \dicPhraseIS{ljúga til um e‑ð} \dicDirectTranslationCS{lhát o~(čem)} \dicExampleIS{ljúga til um aldur} \dicExampleCS{lhát o~(svém) věku}
\dicEntry[ljúgvitni] \dicTerm{ljúg··vitni} \dicIPA{{l}{j}{u}{\textlengthmark}{v}{\textsci}{h}{\textsubring{d}}{n}{\textsci}} \dicPos{n}[2] \dicFlx{(‑s, ‑)}[14] \dicDirectTranslationCS{křivá přísaha, křivé doznání};  \dicPhraseIS{bera ljúgvitni gegn e‑m} \dicDirectTranslationCS{křivě svědčit proti (komu)}
\dicEntry[ljúka] \dicTerm{ljúka} \dicsymFrequent\  \dicIPA{{l}{j}{u}{\textlengthmark}{\r{g}}{a}} \dicPos{v}[6] \dicFlx{(lýk, lauk, lukum, lyki, lokið)}[47] \dicFlx{dat} \textbf{1.} \dicSynonym{klára} \dicDirectTranslationCS{(u)končit, ukončovat, zakončit, zakončovat, dokončit, dokončovat};  \dicPhraseIS{ljúka e‑u} \dicDirectTranslationCS{ukončit (co)};  \dicPhraseIS{ljúka við e‑ð} \dicDirectTranslationCS{ukončit (co), (s)končit s~(čím)}  \textbf{2.} \dicPhraseIS{e‑u lýkur} \dicFlx{impers} \dicDirectTranslationCS{(co) končí} \dicExampleIS{Skólanum lýkur í júní.} \dicExampleCS{Škola končí v~červnu.};  \dicIdiom{ljúka}[af]{ \dicPhraseIS{ljúka e‑u af}} \dicDirectTranslationCS{ukončit (co), skončit (co)};  \dicIdiom{ljúka}[upp]{ \dicPhraseIS{ljúka upp e‑u}} \dicLangCat{zast.} \dicSynonym{opna\smash{\textsuperscript{2}}} \dicDirectTranslationCS{otevřít (co)} \dicExampleIS{ljúka dyrunum upp} \dicExampleCS{otevřít dveře};  \dicIdiom{ljúka}[yfir]{ \dicPhraseIS{þar til yfir lýkur}} \dicFlx{adv} \dicDirectTranslationCS{až do samého konce}; { \dicPhraseIS{áður en yfir lýkur}} \dicFlx{adv} \dicDirectTranslationCS{dříve než bude po všem};  \dicIdiom{ljúkast}[saman]{ \dicPhraseIS{ljúkast saman}} \dicFlx{refl} \dicLangCat{zast.} \dicDirectTranslationCS{zavřít se};  \dicIdiom{ljúkast}[upp]{ \dicPhraseIS{ljúkast upp}} \dicFlx{refl} \dicLangCat{zast.} \dicDirectTranslationCS{otevřít se, rozevřít se}; { \dicPhraseIS{e‑að lýkst upp fyrir e‑m}} \dicFlx{refl} \dicLangCat{přen.} \dicDirectTranslationCS{(co) se otevírá před (kým) (nová myšlenka ap.)}
\dicEntry[ljæ] \dicTerm{ljæ} \dicIPA{{l}{j}{a}{i}{\textlengthmark}} \dicPos{v} \dicFlx{ind praes sg 1 pers} \dicLink{ljá}
\dicEntry[lo.] \dicTerm{lo.} \dicPos{zkr} \dicPhraseIS{lýsingarorð} \dicFieldCat{jaz.} \dicDirectTranslationCS{přídavné jméno}
\dicEntry[lobbíisti] \dicTerm{lobbí··ist|i} \dicIPA{{l}{\textopeno}{\textsubring{b}}{\textlengthmark}{i}{j}{\textsci}{s}{\textsubring{d}}{\textsci}} \dicPos{m}[1] \dicFlx{(‑a, ‑ar)}[1] \dicDirectTranslationCS{lobbista, lobbistka}
\dicEntry[loddari] \dicTerm{lodd··ar|i} \dicIPA{{l}{\textopeno}{\textsubring{d}}{\textlengthmark}{a}{r}{\textsci}} \dicPos{m}[1] \dicFlx{(‑a, ‑ar)}[13] \textbf{1.} \dicSynonym{svikari} \dicDirectTranslationCS{šarlatán(ka)}  \textbf{2.} \dicSynonym{trúður} \dicDirectTranslationCS{šašek, klaun(ka)}
\dicEntry[loða] \dicTerm{lo|ða} \dicIPA{{l}{\textopeno}{\textlengthmark}{ð}{a}} \dicPos{v}[2] \dicFlx{(‑ddi, ‑ðað)}[182] \dicSynonym{tolla\smash{\textsuperscript{2}}} \dicDirectTranslationCS{přiléhat, lnout, přimknout se, přimykat};  \dicIdiom{loða}[saman]{ \dicPhraseIS{loða saman}} \dicDirectTranslationCS{přilnout k~sobě};  \dicIdiom{loða}[við]{ \dicPhraseIS{loða við e‑ð}} \dicDirectTranslationCS{přilnout k~(čemu)} \dicExampleIS{Málningin loðir við vegginn.} \dicExampleCS{Nátěr přilne ke zdi.}
\dicEntry[loðdýr] \dicTerm{loð··dýr} \dicIPA{{l}{\textopeno}{ð}{\textsubring{d}}{i}{\textsubring{r}}} \dicPos{n}[2] \dicFlx{(‑s, ‑)}[5] \dicDirectTranslationCS{kožešinové zvíře}
\dicEntry[loðfeldur] \dicTerm{loð··feld|ur} \dicIPA{{l}{\textopeno}{ð}{f}{\textepsilon}{l}{\textsubring{d}}{\textscy}{\textsubring{r}}} \dicPos{m}[10] \dicFlx{(‑ar, ‑ir)}[4] \dicSynonym{loðskinn} \dicDirectTranslationCS{kožešina}
\dicEntry[loðfíll] \dicTerm{loð··fíl|l} \dicIPA{{l}{\textopeno}{ð}{f}{i}{\textsubring{d}}{\textsubring{l}}} \dicPos{m}[6] \dicFlx{(‑s, ‑ar)}[48] \dicFieldCat{zool.} \dicDirectTranslationCS{mamut} \textit{(l.~{\textLA{Mammuthus primigenius}})}
\dicEntry[loðhúfa] \dicTerm{loð··húf|a} \dicIPA{{l}{\textopeno}{\textlengthmark}{\texttheta}{h}{u}{v}{a}} \dicPos{f}[1] \dicFlx{(‑u, ‑ur)}[19] \dicDirectTranslationCS{kožešinová čepice, beranice}
\dicEntry[loðinn] \dicTerm{loðinn} \dicIPA{{l}{\textopeno}{\textlengthmark}{ð}{\textsci}{\textsubring{n}}} \dicPos{adj}[6]\dicFlx{}[-2] \textbf{1.} \dicDirectTranslationCS{chlupatý, srstnatý, huňatý}  \textbf{2.} \dicSynonym{óhreinn} \dicDirectTranslationCS{vyhýbavý, mlhavý (odpověď ap.)}
\dicEntry[loðkápa] \dicTerm{loð··káp|a} \dicIPA{{l}{\textopeno}{ð}{k\smash{\textsuperscript{h}}}{au}{\textsubring{b}}{a}} \dicPos{f}[1] \dicFlx{(‑u, ‑ur)}[19] \dicDirectTranslationCS{kožich}
\dicEntry[loðna] \dicTerm{loðn|a} \dicIPA{{l}{\textopeno}{ð}{n}{a}} \dicPos{f}[1] \dicFlx{(‑u, ‑ur)}[7] \textbf{1.} \dicDirectTranslationCS{ochlupení}  \textbf{2.} \dicFieldCat{zool.} \dicDirectTranslationCS{huňáček severní} \textit{(l.~{\textLA{Mallotus villosus}})}
\dicEntry[loðskinn] \dicTerm{loð··skinn} \dicIPA{{l}{\textopeno}{ð}{s}{\r{\textObardotlessj}}{\textsci}{\textsubring{n}}} \dicPos{n}[2] \dicFlx{(‑s, ‑)}[5] \dicDirectTranslationCS{kožešina, kožka}
\dicEntry[loðvíðir] \dicTerm{loð··víð|ir} \dicIPA{{l}{\textopeno}{ð}{v}{i}{ð}{\textsci}{\textsubring{r}}} \dicPos{m}[7] \dicFlx{(‑is)}[2] \dicFieldCat{bot.} \dicDirectTranslationCS{vrba vlnatá} \textit{(l.~{\textLA{Salix lanata}})}  \dicsymPhoto\ 
\dicFigure{66869.jpg}{Loðvíðir}{Loðvíðir - Hroneš Michal, Biolib, Copyright/CC-BY-NC}
\dicEntry[lof] \dicTerm{lof} \dicIPA{{l}{\textopeno}{\textlengthmark}{f}} \dicPos{n}[2] \dicFlx{(‑s)}[2] \dicSynonym{hrós} \dicDirectTranslationCS{chvála, pochvala};  \dicPhraseIS{bera lof á e‑n} \dicDirectTranslationCS{chválit (koho)};  \dicPhraseIS{guði sé lof} \dicDirectTranslationCS{díky bohu};  \dicPhraseIS{klappa lof í lófa} \dicDirectTranslationCS{tleskat, aplaudovat}
\dicEntry[lofa] \dicTerm{lof|a} \dicsymFrequent\  \dicIPA{{l}{\textopeno}{\textlengthmark}{v}{a}} \dicPos{v}[1] \dicFlx{(‑aði)}[1] \dicFlx{acc\,/\addthin dat} \textbf{1.} \dicFlx{acc} \dicSynonym{hrósa} \dicDirectTranslationCS{(po)chválit, vychvalovat};  \dicPhraseIS{lofa e‑n\,/\addthin e‑ð} \dicDirectTranslationCS{chválit (koho\,/\addthin co)} \dicExampleIS{lofa listaverkið} \dicExampleCS{pochválit umělecké dílo}  \textbf{2.} \dicFlx{dat + dat} \dicSynonym{heita} \dicDirectTranslationCS{(při)slíbit, slibovat};  \dicPhraseIS{lofa e‑m e‑u} \dicDirectTranslationCS{slíbit (komu co)}  \textbf{3.} \dicFlx{dat + acc} \dicSynonym{leyfa} \dicDirectTranslationCS{dovolit, dovolovat, povolit, povolovat};  \dicPhraseIS{lofa e‑m e‑ð} \dicDirectTranslationCS{dovolit (komu co)} \dicExampleIS{Ég lofaði henni að dansa.} \dicExampleCS{Dovolil jsem jí tančit.};  \dicIdiom{lofast}{ \dicPhraseIS{lofast e‑m}} \dicFlx{refl} \dicLangCat{zast.} \dicDirectTranslationCS{zaslíbit se (komu)}
\dicEntry[lofgerð] \dicTerm{lof··gerð} \dicIPA{{l}{\textopeno}{v}{\r{\textObardotlessj}}{\textepsilon}{r}{\texttheta}} \dicPos{f}[7] \dicFlx{(‑ar, ‑ir)}[1] \dicSynonym{lof} \dicDirectTranslationCS{chvála, pochvala}
\dicEntry[loforð] \dicTerm{lof··orð} \dicIPA{{l}{\textopeno}{\textlengthmark}{v}{\textopeno}{r}{\texttheta}} \dicPos{n}[2] \dicFlx{(‑s, ‑)}[5] \textbf{1.} \dicDirectTranslationCS{slib, slovo};  \dicPhraseIS{standa við loforð} \dicDirectTranslationCS{držet slovo}  \textbf{2.} \dicDirectTranslationCS{přísaha}
\dicEntry[lofsamlegur] \dicTerm{lof·sam··legur} \dicIPA{{l}{\textopeno}{f}{s}{a}{m}{l}{\textepsilon}{\textbabygamma}{\textscy}{\textsubring{r}}} \dicPos{adj}[1]\dicFlx{}[-8] \dicDirectTranslationCS{pochvalný}
\dicEntry[lofsorð] \dicTerm{lofs··orð} \dicIPA{{l}{\textopeno}{f}{s}{\textopeno}{r}{\texttheta}} \dicPos{n}[2] \dicFlx{(‑s, ‑)}[5] \dicPhraseIS{ljúka lofsorði á e‑ð} \dicDirectTranslationCS{chválit (co), vyjádřit se o~(čem) pochvalně}
\dicEntry[lofsverður] \dicTerm{lofs··verður} \dicIPA{{l}{\textopeno}{f}{s}{v}{\textepsilon}{r}{ð}{\textscy}{\textsubring{r}}} \dicPos{adj}[2]\dicFlx{}[-1] \dicDirectTranslationCS{chvályhodný, záslužný} \dicExampleIS{lofsverð viðleitni} \dicExampleCS{chvályhodné úsilí}
\dicEntry[lofsöngur] \dicTerm{lof··söng|ur} \dicIPA{{l}{\textopeno}{f}{s}{\oe i}{\ng}{\r{g}}{\textscy}{\textsubring{r}}} \dicPos{m}[8] \dicFlx{(‑s, ‑var)}[2] \dicDirectTranslationCS{chvalozpěv}
\dicEntry[loft] \dicTerm{loft} \dicsymFrequent\  \dicIPA{{l}{\textopeno}{f}{\textsubring{d}}} \dicPos{n}[2] \dicFlx{(‑s, ‑)}[5] \textbf{1.} \dicSynonym{andrúmsloft} \dicDirectTranslationCS{vzduch, povětří, ovzduší} \dicExampleIS{ferskt loft} \dicExampleCS{čerstvý vzduch}  \textbf{2.} \dicSynonym{himinn} \dicDirectTranslationCS{vzduch, obloha} \dicExampleIS{kasta e‑u upp í loftið} \dicExampleCS{vyhodit (co) do vzduchu}  \textbf{3.} \dicDirectTranslationCS{poschodí, podlaží};  \dicPhraseIS{uppi á lofti} \dicFlx{adv} \dicDirectTranslationCS{v~horním patře}  \textbf{4.} \dicSynonym{háaloft} \dicDirectTranslationCS{podkroví};  \dicIdiom{loft}{ \dicPhraseIS{það er allt upp í loft}} \dicLangCat{přen.} \dicDirectTranslationCS{všechno je vzhůru nohama};  \dicPhraseIS{e‑að er úr lausu lofti gripið} \dicLangCat{přen.} \dicDirectTranslationCS{(co) je neopodstatněné}
\dicEntry[loftárás] \dicTerm{loft··á·rás} \dicIPA{{l}{\textopeno}{f}{\textsubring{d}}{au}{r}{au}{s}} \dicPos{f}[7] \dicFlx{(‑ar, ‑ir)}[1] \dicDirectTranslationCS{nálet, letecký útok}
\dicEntry[loftbelgur] \dicTerm{loft··belg|ur} \dicIPA{{l}{\textopeno}{f}{\textsubring{d}}{\textsubring{b}}{\textepsilon}{l}{\r{g}}{\textscy}{\textsubring{r}}} \dicPos{m}[9] \dicFlx{(‑s\,/\addthin ‑jar, ‑ir)}[26] \dicFieldCat{let.} \dicDirectTranslationCS{(horkovzdušný) balón}
\dicEntry[loftbor] \dicTerm{loft··bor} \dicIPA{{l}{\textopeno}{f}{\textsubring{d}}{\textsubring{b}}{\textopeno}{\textsubring{r}}} \dicPos{m}[4] \dicFlx{(‑s, ‑ar)}[14] \dicDirectTranslationCS{sbíječka, sbíjecí kladivo}
\dicEntry[loftbréf] \dicTerm{loft··bréf} \dicIPA{{l}{\textopeno}{f}{\textsubring{d}}{\textsubring{b}}{r}{j}{\textepsilon}{f}} \dicPos{n}[2] \dicFlx{(‑s, ‑)}[5] \dicDirectTranslationCS{letecké psaní}
\dicEntry[loftbrú] \dicTerm{loft··|brú} \dicIPA{{l}{\textopeno}{f}{\textsubring{d}}{\textsubring{b}}{r}{u}} \dicPos{f}[9] \dicFlx{(‑brúar, ‑brýr)}[4] \dicDirectTranslationCS{letecký most}
\dicEntry[loftfar] \dicTerm{loft··|far} \dicIPA{{l}{\textopeno}{f}{\textsubring{d}}{f}{a}{\textsubring{r}}} \dicPos{n}[2] \dicFlx{(‑fars, ‑för)}[8] \dicDirectTranslationCS{létající prostředek (letadlo, balón, vzducholoď ap.)}
\dicEntry[loftfimleikar] \dicTerm{loft··fim·leikar} \dicIPA{{l}{\textopeno}{f}{\textsubring{d}}{f}{\textsci}{m}{l}{ei}{\r{g}}{a}{\textsubring{r}}} \dicPos{m}[6] \dicFlx{pl}[2] \dicDirectTranslationCS{letecká akrobacie}
\dicEntry[lofthræddur] \dicTerm{loft··hræddur} \dicIPA{{l}{\textopeno}{f}{\textsubring{d}}{\textsubring{r}}{a}{i}{\textsubring{d}}{\textscy}{\textsubring{r}}} \dicPos{adj}[2]\dicFlx{}[-18] \dicDirectTranslationCS{mající strach z~výšek}
\dicEntry[lofthræðsla] \dicTerm{loft··hræðsl|a} \dicIPA{{l}{\textopeno}{f}{\textsubring{d}}{\textsubring{r}}{a}{i}{ð}{s}{\textsubring{d}}{l}{a}} \dicPos{f}[1] \dicFlx{(‑u)}[5] \dicDirectTranslationCS{strach z~výšek}
\dicEntry[loftkastali] \dicTerm{loft··kastal|i} \dicIPA{{l}{\textopeno}{f}{\textsubring{d}}{k\smash{\textsuperscript{h}}}{a}{s}{\textsubring{d}}{a}{l}{\textsci}} \dicPos{m}[1] \dicFlx{(‑a, ‑ar)}[12] \dicDirectTranslationCS{vzdušný zámek};  \dicPhraseIS{byggja loftkastala} \dicLangCat{přen.} \dicDirectTranslationCS{stavět vzdušné zámky}
\dicEntry[loftkenndur] \dicTerm{loft··kenndur} \dicIPA{{l}{\textopeno}{f}{\textsubring{d}}{c\smash{\textsuperscript{h}}}{\textepsilon}{n}{\textsubring{d}}{\textscy}{\textsubring{r}}} \dicPos{adj}[2]\dicFlx{}[-14] \textbf{1.} \dicDirectTranslationCS{vzdušný}  \textbf{2.} \dicDirectTranslationCS{plynný} \dicExampleIS{loftkennd gosefni} \dicExampleCS{plynné magma}
\dicEntry[loftkældur] \dicTerm{loft··kældur} \dicIPA{{l}{\textopeno}{f}{\textsubring{d}}{c\smash{\textsuperscript{h}}}{a}{i}{l}{\textsubring{d}}{\textscy}{\textsubring{r}}} \dicPos{adj}[2]\dicFlx{}[-17] \dicDirectTranslationCS{klimatizovaný} \dicExampleIS{loftkæld skrifstofa} \dicExampleCS{klimatizovaná kancelář}
\dicEntry[loftkæling] \dicTerm{loft··kæl·ing} \dicIPA{{l}{\textopeno}{f}{\textsubring{d}}{c\smash{\textsuperscript{h}}}{a}{i}{l}{i}{\ng}{\r{g}}} \dicPos{f}[4] \dicFlx{(‑ar, ‑ar)}[5] \dicDirectTranslationCS{klimatizace}
\begin{xtolerant}{}{1pt}
\dicEntry[loftmengun] \dicTerm{loft··meng|un}\addthinS\dicIPA{{l}{\textopeno}{f}{\textsubring{d}}{m}{ei}{\ng}{\r{g}}{\textscy}{\textsubring{n}}}\addthinS\dicPos{f}[7]\addthinS\dicFlx{(‑unar)}[9] \dicDirectTranslationCS{znečištění ovzduší}
\end{xtolerant}
\dicEntry[loftnet] \dicTerm{loft··net} \dicIPA{{l}{\textopeno}{f}{\textsubring{d}}{n}{\textepsilon}{\textsubring{d}}} \dicPos{n}[2] \dicFlx{(‑s, ‑)}[5] \dicDirectTranslationCS{anténa}
\dicEntry[loftræsta] \dicTerm{loft··ræst|a} \dicIPA{{l}{\textopeno}{f}{\textsubring{d}}{r}{a}{i}{s}{\textsubring{d}}{a}} \dicPos{v}[2] \dicFlx{(‑i, ‑)}[16] \dicFlx{acc} \dicDirectTranslationCS{(vy)větrat, (vy)ventilovat} \dicExampleIS{loftræsta sal} \dicExampleCS{vyvětrat sál}
\dicEntry[loftræsting] \dicTerm{loft··ræst·ing} \dicIPA{{l}{\textopeno}{f}{\textsubring{d}}{r}{a}{i}{s}{\textsubring{d}}{i}{\ng}{\r{g}}} \dicPos{f}[4] \dicFlx{(‑ar, ‑ar)}[5] \dicDirectTranslationCS{větrání, ventilace}
\begin{xtolerant}{}{1pt}
\dicEntry[loftskeytamaður] \dicTerm{loft·skeyta··|maður}\addthinSS\dicIPA{{l}{\textopeno}{f}{\textsubring{d}}{s}{c\smash{\textsuperscript{h}}}{ei}{\textsubring{d}}{a}{m}{a}{ð}{\textscy}{\textsubring{r}}}\addthinSS\dicPos{m}[1\hspace{-0.10667em}3]\addthinS\dicFlx{(‑manns,\addthinS‑menn)}[2] \dicDirectTranslationCS{radista, radistka, rádiový operátor, rádiová operátorka}
\end{xtolerant}
\dicEntry[loftskip] \dicTerm{loft··skip} \dicIPA{{l}{\textopeno}{f}{\textsubring{d}}{s}{\r{\textObardotlessj}}{\textsci}{\textsubring{b}}} \dicPos{n}[2] \dicFlx{(‑s, ‑)}[5] \dicFieldCat{let.} \dicDirectTranslationCS{vzducholoď}
\dicEntry[loftslag] \dicTerm{lofts··lag} \dicIPA{{l}{\textopeno}{f}{s}{l}{a}{x}} \dicPos{n}[2] \dicFlx{(‑s)}[2] \dicSynonym{veðurfar} \dicDirectTranslationCS{podnebí, klima}
\dicEntry[loftslagsbelti] \dicTerm{lofts·lags··belti} \dicIPA{{l}{\textopeno}{f}{s}{l}{a}{x}{s}{\textsubring{b}}{\textepsilon}{\textsubring{l}}{\textsubring{d}}{\textsci}} \dicPos{n}[2] \dicFlx{(‑s, ‑)}[14] \dicFieldCat{meteo.} \dicDirectTranslationCS{podnebný pás}
\dicEntry[loftslagsfræði] \dicTerm{lofts·lags··fræð|i} \dicIPA{{l}{\textopeno}{f}{s}{l}{a}{x}{s}{f}{r}{a}{i}{ð}{\textsci}} \dicPos{f}[3] \dicFlx{(‑i)}[3] \dicSynonym{veðurfarsfræði} \dicDirectTranslationCS{klimatologie}
\dicEntry[loftsteinn] \dicTerm{loft··stein|n} \dicIPA{{l}{\textopeno}{f}{\textsubring{d}}{s}{\textsubring{d}}{ei}{\textsubring{d}}{\textsubring{n}}} \dicPos{m}[6] \dicFlx{(‑s, ‑ar)}[42] \dicFieldCat{astro.} \dicDirectTranslationCS{meteorit}
\dicEntry[lofttegund] \dicTerm{loft··tegund} \dicIPA{{l}{\textopeno}{f}{t\smash{\textsuperscript{h}}}{\textepsilon}{\textbabygamma}{\textscy}{n}{\textsubring{d}}} \dicPos{f}[7] \dicFlx{(‑ar, ‑ir)}[1] \dicFieldCat{chem.} \dicDirectTranslationCS{plyn}
\dicEntry[lofttóm] \textls[15]{\dicTerm{loft··tóm} \dicIPA{{l}{\textopeno}{f}{t\smash{\textsuperscript{h}}}{ou}{\textsubring{m}}} \dicPos{n}[2] \dicFlx{(‑s)}[2] \dicFieldCat{fyz.} \dicDirectTranslationCS{vakuum, vzduchoprázdno}}
\dicEntry[loftvarnir] \dicTerm{loft··varnir} \dicIPA{{l}{\textopeno}{f}{\textsubring{d}}{v}{a}{r}{\textsubring{d}}{n}{\textsci}{\textsubring{r}}} \dicPos{f}[7] \dicFlx{pl}[18] \dicFieldCat{voj.} \dicDirectTranslationCS{protiletecká\,/\addthin protiletadlová obrana}
\dicEntry[loftveiki] \dicTerm{loft··veik|i} \dicIPA{{l}{\textopeno}{f}{\textsubring{d}}{v}{ei}{\r{\textObardotlessj}}{\textsci}} \dicPos{f}[3] \dicFlx{(‑i)}[3] \dicFieldCat{med.} \dicDirectTranslationCS{výšková\,/\addthin horská nemoc}
\dicEntry[loftvog] \dicTerm{loft··vog} \dicIPA{{l}{\textopeno}{f}{\textsubring{d}}{v}{\textopeno}{x}} \dicPos{f}[7] \dicFlx{(‑ar, ‑ir)}[1] \dicDirectTranslationCS{barometr}
\dicEntry[loftþéttur] \dicTerm{loft··þéttur} \dicIPA{{l}{\textopeno}{f}{\textsubring{d}}{\texttheta}{j}{\textepsilon}{h}{\textsubring{d}}{\textscy}{\textsubring{r}}} \dicPos{adj}[1]\dicFlx{}[-10] \dicDirectTranslationCS{vzduchotěsný, hermetický}
\dicEntry[loftþrýstingur] \dicTerm{loft··þrýst·ing|ur} \dicIPA{{l}{\textopeno}{f}{\textsubring{d}}{\texttheta}{r}{i}{s}{\textsubring{d}}{i}{\ng}{\r{g}}{\textscy}{\textsubring{r}}} \dicPos{m}[6] \dicFlx{(‑s)}[9] \dicDirectTranslationCS{tlak vzduchu, atmosférický tlak}
\dicEntry[loga] \dicTerm{log|a} \dicsymFrequent\  \dicIPA{{l}{\textopeno}{\textlengthmark}{\textbabygamma}{a}} \dicPos{v}[1] \dicFlx{(‑aði)}[44] \textbf{1.} \dicDirectTranslationCS{hořet, planout};  \dicPhraseIS{það logar á e‑u} \dicFlx{impers} \dicDirectTranslationCS{(co) hoří} \dicExampleIS{Það logar vel á kertinu.} \dicExampleCS{Svíčka dobře hoří.}  \textbf{2.} \dicDirectTranslationCS{hořet, zuřit (město ap.)} \dicIndirectTranslationCS{(o~nepokojích nebo válce)};  \dicPhraseIS{e‑að logar í e‑u} \dicDirectTranslationCS{v~(čem) zuří (co) (ve městě nepokoje ap.)}
\dicEntry[logandi] \dicTerm{log··andi\smash{\textsuperscript{1}}} \dicIPA{{l}{\textopeno}{\textlengthmark}{\textbabygamma}{a}{n}{\textsubring{d}}{\textsci}} \dicPos{adj}[13] \dicFlx{indecl}[1] \dicDirectTranslationCS{hořící, planoucí} \dicExampleIS{logandi kerti} \dicExampleCS{hořící svíce}
\dicEntry[logandi] \dicTerm{log··andi\smash{\textsuperscript{2}}} \dicIPA{{l}{\textopeno}{\textlengthmark}{\textbabygamma}{a}{n}{\textsubring{d}}{\textsci}} \dicPos{adv} \dicDirectTranslationCS{hrozně, strašně} \dicIndirectTranslationCS{(k~zdůraznění)} \dicExampleIS{Hann var logandi hræddur.} \dicExampleCS{Strašně se bál.};  \dicPhraseIS{hvert þó í logandi!} \dicDirectTranslationCS{panebože!, svatá dobroto!}
\dicEntry[logga] \dicTerm{logg|a} \dicIPA{{l}{\textopeno}{\r{g}}{\textlengthmark}{a}} \dicPos{v}[1] \dicFlx{(‑aði)}[1] \dicFieldCat{poč.} \dicDirectTranslationCS{logovat, zaznamenat do souboru};  \dicIdiom{logga}[inn]{ \dicPhraseIS{logga sig inn}} \dicFieldCat{poč.} \dicDirectTranslationCS{(za)logovat se, nalogovat se, přihlásit se};  \dicIdiom{logga}[út]{ \dicPhraseIS{logga sig út}} \dicFieldCat{poč.} \dicDirectTranslationCS{vylogovat se, odlogovat se, odhlásit se}
\dicEntry[logi] \dicTerm{log|i} \dicsymFrequent\  \dicIPA{{l}{\textopeno i}{j}{\textlengthmark}{\textsci}} \dicPos{m}[1] \dicFlx{(‑a, ‑ar)}[1] \dicDirectTranslationCS{plamen, planutí} \dicExampleIS{Loginn hefur dofnað.} \dicExampleCS{Plamen zeslábl.};  \dicPhraseIS{fara eins og logi yfir akur} \dicLangCat{přen.} \dicDirectTranslationCS{přehnat se jako bouře}
\dicEntry[logið] \dicTerm{logið} \dicIPA{{l}{\textopeno i}{j}{\textlengthmark}{\textsci}{\texttheta}} \dicPos{v} \dicFlx{supin} \dicLink{ljúga}
\dicEntry[logn] \dicTerm{logn} \dicsymFrequent\  \dicIPA{{l}{\textopeno}{\r{g}}{\textsubring{n}}} \dicPos{n}[2] \dicFlx{(‑s, ‑)}[5] \dicSynonym*{vindleysa} \dicDirectTranslationCS{bezvětří} \dicIndirectTranslationCS{(Beaufort 0) (0--0,2~m/s)} \dicExampleIS{Það er logn í dag.} \dicExampleCS{Dnes je bezvětří.}
\dicEntry[lognmolla] \dicTerm{logn··moll|a} \dicIPA{{l}{\textopeno}{\r{g}}{n}{m}{\textopeno}{\textsubring{d}}{l}{a}} \dicPos{f}[1] \dicFlx{(‑u)}[5] \textbf{1.} \dicSynonym{andleysi} \dicDirectTranslationCS{nečinnost, neaktivita}  \textbf{2.} \dicDirectTranslationCS{poklidné počasí}
\dicEntry[logsauð] \dicTerm{log··sauð} \dicIPA{{l}{\textopeno}{x}{s}{\oe i}{\texttheta}} \dicPos{v} \dicFlx{ind pf sg 1 pers} \dicLink{logsjóða}
\dicEntry[logsjóða] \dicTerm{log··|sjóða} \dicIPA{{l}{\textopeno}{x}{s}{j}{ou}{ð}{a}} \dicPos{v}[6] \dicFlx{(‑sýð, ‑sauð, ‑suðum, ‑syði, ‑soðið)}[104] \dicFlx{acc} \dicFieldCat{techn.} \dicDirectTranslationCS{svářet, svařovat}
\dicEntry[logsoðið] \dicTerm{log··soðið} \dicIPA{{l}{\textopeno}{x}{s}{\textopeno}{ð}{\textsci}{\texttheta}} \dicPos{v} \dicFlx{supin} \dicLink{logsjóða}
\dicEntry[logsuða] \dicTerm{log··suð|a} \dicIPA{{l}{\textopeno}{x}{s}{\textscy}{ð}{a}} \dicPos{f}[1] \dicFlx{(‑u)}[5] \dicSynonym*{það að logsjóða} \dicDirectTranslationCS{sváření, svařování}
\dicEntry[logsuðum] \dicTerm{log··suðum} \dicIPA{{l}{\textopeno}{x}{s}{\textscy}{ð}{\textscy}{\textsubring{m}}} \dicPos{v} \dicFlx{ind pf pl 1 pers} \dicLink{logsjóða}
\dicEntry[logsyði] \dicTerm{log··syði} \dicIPA{{l}{\textopeno}{x}{s}{\textsci}{ð}{\textsci}} \dicPos{v} \dicFlx{con pf sg 1 pers} \dicLink{logsjóða}
\dicEntry[logsýð] \dicTerm{log··sýð} \dicIPA{{l}{\textopeno}{x}{s}{i}{\texttheta}} \dicPos{v} \dicFlx{ind praes sg 1 pers} \dicLink{logsjóða}
\dicEntry[lok] \dicTerm{lok} \dicsymFrequent\  \dicIPA{{l}{\textopeno}{\textlengthmark}{\r{g}}} \dicPos{n}[2] \dicFlx{(‑s, ‑)}[5] \textbf{1.} \dicDirectTranslationCS{uzávěr, víčko, víko, poklička, poklice} \dicExampleIS{lokið á dósinni} \dicExampleCS{víčko plechovky}  \textbf{2.} \dicPhraseIS{lok} \dicFlx{pl} \dicSynonym{endir} \dicDirectTranslationCS{konec, závěr};  \dicPhraseIS{að lokum} \dicFlx{adv} \dicDirectTranslationCS{nakonec, na závěr};  \dicPhraseIS{e‑að líður undir lok} \dicDirectTranslationCS{(co) se chýlí k~závěru};  \dicPhraseIS{í lokin} \dicFlx{adv} \dicDirectTranslationCS{nakonec}
\dicEntry[loka] \dicTerm{lok|a\smash{\textsuperscript{1}}} \dicIPA{{l}{\textopeno}{\textlengthmark}{\r{g}}{a}} \dicPos{f}[1] \dicFlx{(‑u, ‑ur)}[19] \textbf{1.} \dicSynonym{hespa\smash{\textsuperscript{1}}} \dicDirectTranslationCS{západka, petlice, závora}  \textbf{2.} \dicFieldCat{anat.} \dicDirectTranslationCS{chlopeň}
\dicEntry[loka] \dicTerm{lok|a\smash{\textsuperscript{2}}} \dicsymFrequent\  \dicIPA{{l}{\textopeno}{\textlengthmark}{\r{g}}{a}} \dicPos{v}[1] \dicFlx{(‑aði)}[1] \dicFlx{dat} \dicDirectTranslationCS{(u)zavřít, zavírat} \dicExampleIS{loka glugganum} \dicExampleCS{zavřít okno};  \dicIdiom{loka}[að]{ \dicPhraseIS{loka að sér}} \dicDirectTranslationCS{zavřít za sebou (dveře)};  \dicIdiom{loka}[fyrir]{ \dicPhraseIS{loka fyrir e‑ð}} \dicDirectTranslationCS{zavřít (co), vypnout (co) (elektřinu ap.)};  \dicIdiom{loka}[inni]{ \dicPhraseIS{loka e‑n inni}} \dicDirectTranslationCS{zavřít (koho) (do blázince ap.)};  \dicIdiom{loka}[úti]{ \dicPhraseIS{loka e‑n úti}} \dicDirectTranslationCS{vyloučit (koho) (z~party ap.), nevpustit (koho) dovnitř};  \dicIdiom{lokast}{ \dicPhraseIS{lokast}} \dicFlx{refl} \dicDirectTranslationCS{zavřít se, zavírat se}
\dicEntry[lokaáfangi] \dicTerm{loka··á·fang|i} \dicIPA{{l}{\textopeno}{\textlengthmark}{\r{g}}{a}{au}{f}{au}{\textltailn}{\r{\textObardotlessj}}{\textsci}} \dicPos{m}[1] \dicFlx{(‑a, ‑ar)}[8] \dicDirectTranslationCS{poslední etapa}
\dicEntry[lokaður] \dicTerm{lok··|aður} \dicsymFrequent\  \dicIPA{{l}{\textopeno}{\textlengthmark}{\r{g}}{a}{ð}{\textscy}{\textsubring{r}}} \dicPos{adj}[3] \dicFlx{(f ‑uð)}[3] \textbf{1.} \dicSynonym*{ekki opinn} \dicDirectTranslationCS{zavřený, uzavřený} \dicExampleIS{lokaður vegur} \dicExampleCS{uzavřená cesta}  \textbf{2.} \dicLangCat{přen.} \dicDirectTranslationCS{uzavřený, odměřený, zdrženlivý, rezervovaný}
\dicEntry[lokafrestur] \dicTerm{loka··frest|ur} \dicIPA{{l}{\textopeno}{\textlengthmark}{\r{g}}{a}{f}{r}{\textepsilon}{s}{\textsubring{d}}{\textscy}{\textsubring{r}}} \dicPos{m}[6] \dicFlx{(‑s)}[7] \dicDirectTranslationCS{uzávěrka, (konečný) termín, deadline}
\dicEntry[lokaorð] \dicTerm{loka··orð} \dicIPA{{l}{\textopeno}{\textlengthmark}{\r{g}}{a}{\textopeno}{r}{\texttheta}} \dicPos{n}[2] \dicFlx{(‑s, ‑)}[5] \dicDirectTranslationCS{konečné\,/\addthin závěrečné slovo, doslov};  \dicPhraseIS{eiga lokaorðið} \dicLangCat{přen.} \dicDirectTranslationCS{mít poslední slovo}
\dicEntry[lokapróf] \dicTerm{loka··próf} \dicIPA{{l}{\textopeno}{\textlengthmark}{\r{g}}{a}{p\smash{\textsuperscript{h}}}{r}{ou}{f}} \dicPos{n}[2] \dicFlx{(‑s, ‑)}[5] \dicFieldCat{škol.} \dicDirectTranslationCS{závěrečná zkouška}
\dicEntry[lokaritgerð] \dicTerm{loka··rit·gerð} \dicIPA{{l}{\textopeno}{\textlengthmark}{\r{g}}{a}{r}{\textsci}{\textsubring{d}}{\r{\textObardotlessj}}{\textepsilon}{r}{\texttheta}} \dicPos{f}[7] \dicFlx{(‑ar, ‑ir)}[1] \dicFieldCat{škol.} \dicDirectTranslationCS{závěrečná práce}
\dicEntry[lokasjóður] \dicTerm{loka··sjóð|ur} \dicIPA{{l}{\textopeno}{\textlengthmark}{\r{g}}{a}{s}{j}{ou}{ð}{\textscy}{\textsubring{r}}} \dicPos{m}[9] \dicFlx{(‑s, ‑ir)}[6] \dicFieldCat{bot.} \dicDirectTranslationCS{kokrhel menší} \textit{(l.~{\textLA{Rhinanthus minor}})}  \dicsymPhoto\ 
\dicFigure{71849.jpg}{Lokasjóður}{Lokasjóður - ŠARŽÍK František, Biolib, Copyright/PD}
\dicEntry[lokaæfing] \dicTerm{loka··æf·ing} \dicIPA{{l}{\textopeno}{\textlengthmark}{\r{g}}{a}{a}{i}{v}{i}{\ng}{\r{g}}} \dicPos{f}[4] \dicFlx{(‑ar, ‑ar)}[5] \dicDirectTranslationCS{generální zkouška, generálka}
\dicEntry[lokhljóð] \dicTerm{lok··hljóð} \dicIPA{{l}{\textopeno}{\textlengthmark}{\r{g}}{\textsubring{l}}{j}{ou}{\texttheta}} \dicPos{n}[2] \dicFlx{(‑s, ‑)}[5] \dicFieldCat{jaz.} \dicDirectTranslationCS{ražená souhláska, ploziva}
\dicEntry[loki] \dicTerm{lok|i} \dicIPA{{l}{\textopeno}{\textlengthmark}{\r{\textObardotlessj}}{\textsci}} \dicPos{m}[1] \dicFlx{(‑a, ‑ar)}[1] \textbf{1.} \dicSynonym{hnútur} \dicDirectTranslationCS{uzel, zauzlení}  \textbf{2.} \dicFieldCat{techn.} \dicSynonym*{öryggisloki} \dicDirectTranslationCS{ventil, ventilek}
\dicEntry[lokið] \dicTerm{lokið} \dicIPA{{l}{\textopeno}{\textlengthmark}{\r{\textObardotlessj}}{\textsci}{\texttheta}} \dicPos{v} \dicFlx{supin} \dicLink{ljúka}
\dicEntry[lokinn] \dicTerm{lokinn} \dicsymFrequent\  \dicIPA{{l}{\textopeno}{\textlengthmark}{\r{\textObardotlessj}}{\textsci}{\textsubring{n}}} \dicPos{adj}[6]\dicFlx{}[-6] \dicSynonym{búinn} \dicDirectTranslationCS{skončený, uzavřený} \dicExampleIS{Rannsókn er lokið.} \dicExampleCS{Vyšetřování je uzavřené.};  \dicPhraseIS{e‑m er öllum lokið} \dicFlx{impers} \dicLangCat{přen.} \dicDirectTranslationCS{(kdo) toho má dost, (kdo) je hotový}
\dicEntry[lokka] \dicTerm{lokk|a} \dicIPA{{l}{\textopeno}{h}{\r{g}}{a}} \dicPos{v}[1] \dicFlx{(‑aði)}[1] \dicFlx{acc} \textbf{1.} \dicSynonym{ginna} \dicDirectTranslationCS{(na)lákat, zlákat} \dicExampleIS{lokka e‑n til sín} \dicExampleCS{nalákat (koho) k~sobě}  \textbf{2.} \dicDirectTranslationCS{vylákat, vymámit, vyloudit} \dicExampleIS{lokka e‑ð út úr e‑m} \dicExampleCS{vylákat (co) od (koho)}
\dicEntry[lokkur] \dicTerm{lokk|ur} \dicIPA{{l}{\textopeno}{h}{\r{g}}{\textscy}{\textsubring{r}}} \dicPos{m}[6] \dicFlx{(‑s, ‑ar)}[22] \dicDirectTranslationCS{lokna, kudrna}
\dicEntry[lokræsi] \dicTerm{lok··ræsi} \dicIPA{{l}{\textopeno}{\textlengthmark}{\r{g}}{r}{a}{i}{s}{\textsci}} \dicPos{n}[2] \dicFlx{(‑s, ‑)}[14] \dicLangCat{zast.} \dicDirectTranslationCS{stoka, odpad}
\dicEntry[loks] \dicTerm{loks} \dicsymFrequent\  \dicIPA{{l}{\textopeno}{\r{g}}{s}} \dicPos{adv} \dicSynonym*{að lokum} \dicDirectTranslationCS{nakonec, konečně} \dicExampleIS{Hurðin var loks opnuð.} \dicExampleCS{Dveře se nakonec otevřely.}
\dicEntry[loksins] \dicTerm{loksins} \dicsymFrequent\  \dicIPA{{l}{\textopeno}{\r{g}}{s}{\textsci}{n}{s}} \dicPos{adv} \dicSynonym*{á endanum} \dicDirectTranslationCS{konečně} \dicExampleIS{Fundur er loksins að hefjast.} \dicExampleCS{Schůze konečně začíná.}
\dicEntry[lokun] \dicTerm{lok|un} \dicIPA{{l}{\textopeno}{\textlengthmark}{\r{g}}{\textscy}{\textsubring{n}}} \dicPos{f}[7] \dicFlx{(‑unar, ‑anir)}[8] \textbf{1.} \dicSynonym*{það að loka} \dicDirectTranslationCS{zavírání, (u)zavření} \dicExampleIS{lokun búða} \dicExampleCS{zavření obchodu}  \textbf{2.} \dicSynonym*{það að setja hindrun} \dicDirectTranslationCS{uzavírka (cesty ap.)} \dicExampleIS{lokun vegar} \dicExampleCS{uzavírka cesty}
\dicEntry[lokunartími] \textls[15]{\dicTerm{lokunar··tím|i} \dicIPA{{l}{\textopeno}{\r{g}}{\textscy}{n}{a}{\textsubring{r}}{t\smash{\textsuperscript{h}}}{i}{m}{\textsci}} \dicPos{m}[1] \dicFlx{(‑a)}[3] \dicDirectTranslationCS{zavírací doba}} \dicAntonym{opnunartími}
\dicEntry[lopapeysa] \dicTerm{lopa··peys|a} \dicIPA{{l}{\textopeno}{\textlengthmark}{\textsubring{b}}{a}{p\smash{\textsuperscript{h}}}{ei}{s}{a}} \dicPos{f}[1] \dicFlx{(‑u, ‑ur)}[7] \dicIndirectTranslationCS{pletený vlněný svetr s~typickými islandskými vzory}
\dicEntry[lopi] \dicTerm{lop|i} \dicIPA{{l}{\textopeno}{\textlengthmark}{\textsubring{b}}{\textsci}} \dicPos{m}[1] \dicFlx{(‑a, ‑ar)}[1] \textbf{1.} \dicSynonym*{ullarþráður} \dicDirectTranslationCS{mykaná vlna}  \textbf{2.} \dicFieldCat{med.} \dicSynonym{bjúgur} \dicDirectTranslationCS{edém, otok};  \dicIdiom{lopi}{ \dicPhraseIS{spinna\,/\addthin teygja lopann}} \dicLangCat{přen.} \dicDirectTranslationCS{pustit si pusu na špacír}
\dicEntry[loppa] \dicTerm{lopp|a} \dicIPA{{l}{\textopeno}{h}{\textsubring{b}}{a}} \dicPos{f}[1] \dicFlx{(‑u, ‑ur)}[19] \dicDirectTranslationCS{tlapa, tlapka}
\dicEntry[los] \dicTerm{los} \dicIPA{{l}{\textopeno}{\textlengthmark}{s}} \dicPos{n}[2] \dicFlx{(‑s, ‑)}[5] \dicDirectTranslationCS{nestabilita, nestabilnost, nesoudržnost} \dicExampleIS{það er los á e‑u} \dicExampleCS{(co) je nestabilní}
\dicEntry[losa] \dicTerm{los|a} \dicsymFrequent\  \dicIPA{{l}{\textopeno}{\textlengthmark}{s}{a}} \dicPos{v}[1] \dicFlx{(‑aði)}[1] \dicFlx{acc} \textbf{1.} \dicDirectTranslationCS{uvolnit, uvolňovat, povolit, povolovat (šroubek ap.)} \dicExampleIS{Hann losar hjólið af bílnum.} \dicExampleCS{Uvolňuje kolo u~auta.}  \textbf{2.} \dicDirectTranslationCS{uvolnit, uvolňovat (pokoj ap.)} \dicExampleIS{losa hótelherbergi} \dicExampleCS{uvolnit hotelový pokoj}  \textbf{3.} \dicDirectTranslationCS{vyprázdnit, vyprazdňovat} \dicExampleIS{losa flöskuna} \dicExampleCS{vyprázdnit láhev}  \textbf{4.} \dicPhraseIS{e‑að losar (20 tonn)} \dicDirectTranslationCS{(co) má něco málo přes (20 tun), (co) přesahuje (20 tun)};  \dicIdiom{losa}[um]{ \dicPhraseIS{losa um e‑ð}} \dicDirectTranslationCS{uvolnit (co), zmírnit (co) (omezení ap.)};  \dicIdiom{losa}[úr]{ \dicPhraseIS{losa e‑n úr e‑u}} \dicDirectTranslationCS{uvolnit (koho) z~(čeho), vymanit (koho) z~(čeho)};  \dicIdiom{losa}[við]{ \dicPhraseIS{losa e‑n við e‑ð}} \dicDirectTranslationCS{pomoct (komu) s~(čím)}; { \dicPhraseIS{losa sig við e‑ð}} \dicDirectTranslationCS{zbavit se (čeho)}
\dicEntry[losaralegur] \dicTerm{losara··legur} \dicIPA{{l}{\textopeno}{\textlengthmark}{s}{a}{r}{a}{l}{\textepsilon}{\textbabygamma}{\textscy}{\textsubring{r}}} \dicPos{adj}[1]\dicFlx{}[-8] \dicSynonym{sundurlaus} \dicDirectTranslationCS{náhod\-ný, nahodilý, neplánovaný}
\dicEntry[losna] \dicTerm{losn|a} \dicsymFrequent\  \dicIPA{{l}{\textopeno}{s}{\textsubring{d}}{n}{a}} \dicPos{v}[1] \dicFlx{(‑aði)}[1] \textbf{1.} \dicDirectTranslationCS{uvolnit se, uprázdnit se (místo ap.)} \dicExampleIS{Starfið losnar bráðlega.} \dicExampleCS{Zaměstnání se brzy uvolní.}  \textbf{2.} \dicDirectTranslationCS{uvolnit\,/\addthin uvolňovat se, povolit, povolovat (šroubek ap.)} \dicExampleIS{Það hefur losnað hjól af bílnum.} \dicExampleCS{U~auta se uvolnilo kolo.}  \textbf{3.} \dicDirectTranslationCS{uvolnit\,/\addthin uvolňovat se (z~práce ap.)} \dicExampleIS{losna úr vinnunni} \dicExampleCS{uvolnit se z~práce};  \dicIdiom{losna}[undan]{ \dicPhraseIS{losna undan e‑u}} \dicDirectTranslationCS{vyhnout\,/\addthin vyhýbat se (čemu), uniknout\,/\addthin unikat (čemu)} \dicExampleIS{losna undan mörgum skyldum} \dicExampleCS{vyhýbat se mnoha povinnostem};  \dicIdiom{losna}[við]{ \dicPhraseIS{losna við e‑ð}} \dicSynonym*{koma e‑u af sér} \dicDirectTranslationCS{zbavit se (čeho)}
\dicEntry[lost] \dicTerm{lost} \dicIPA{{l}{\textopeno}{s}{\textsubring{d}}} \dicPos{n}[2] \dicFlx{(‑s, ‑)}[5] \textbf{1.} \dicFieldCat{med.} \dicDirectTranslationCS{šok}  \textbf{2.} \dicLangCat{hovor.} \dicDirectTranslationCS{šok, překvapení} \dicExampleIS{Ég var alveg í losti á eftir.} \dicExampleCS{Byl jsem po tom v~šoku.}
\dicEntry[lostafullur] \dicTerm{losta··|fullur} \dicIPA{{l}{\textopeno}{s}{\textsubring{d}}{a}{f}{\textscy}{\textsubring{d}}{l}{\textscy}{\textsubring{r}}} \dicPos{adj}[10] \dicFlx{(comp ‑fyllri, sup ‑fyllstur)}[7] \dicSynonym*{munaðargjarn} \dicDirectTranslationCS{chlípný, žádostivý, chtivý, vilný}
\dicEntry[losti] \dicTerm{lost|i} \dicIPA{{l}{\textopeno}{s}{\textsubring{d}}{\textsci}} \dicPos{m}[1] \dicFlx{(‑a)}[3] \dicSynonym{girnd} \dicDirectTranslationCS{chlípnost, žádostivost, chtíč, vilnost}
\dicEntry[lostið] \dicTerm{lostið} \dicIPA{{l}{\textopeno}{s}{\textsubring{d}}{\textsci}{\texttheta}} \dicPos{v} \dicFlx{supin} \dicLink{ljósta}
\dicEntry[lostinn] \dicTerm{lostinn} \dicIPA{{l}{\textopeno}{s}{\textsubring{d}}{\textsci}{\textsubring{n}}} \dicPos{adj}[6]\dicFlx{}[-6] \dicPhraseIS{skelfingu lostinn} \dicFlx{adj} \dicDirectTranslationCS{vyděšený, vystrašený};  \dicPhraseIS{vera furðu lostinn yfir e‑u} \dicDirectTranslationCS{oněmět úžasem nad (čím)};  \dicPhraseIS{vera þrumu lostinn yfir e‑u} \dicDirectTranslationCS{být šokován (čím)}
\dicEntry[lostæti] \dicTerm{lost··æti} \dicIPA{{l}{\textopeno}{s}{\textsubring{d}}{a}{i}{\textsubring{d}}{\textsci}} \dicPos{n}[2] \dicFlx{(‑s, ‑)}[14] \dicSynonym{góðgæti} \dicDirectTranslationCS{pochoutka, lahůdka, delikatesa}
\dicEntry[losun] \dicTerm{los|un} \dicIPA{{l}{\textopeno}{\textlengthmark}{s}{\textscy}{\textsubring{n}}} \dicPos{f}[7] \dicFlx{(‑unar)}[9] \textbf{1.} \dicSynonym*{losun sorps o.fl.} \dicDirectTranslationCS{vypouštění, vyprazdňování (odpadu, odpadků ap.)}  \textbf{2.} \dicSynonym*{afferming} \dicDirectTranslationCS{vykládka, vyložení} \dicExampleIS{losun á farminum} \dicExampleCS{vyložení nákladu}
\dicEntry[lota] \dicTerm{lot|a} \dicIPA{{l}{\textopeno}{\textlengthmark}{\textsubring{d}}{a}} \dicPos{f}[1] \dicFlx{(‑u, ‑ur)}[19] \dicDirectTranslationCS{období, cyklus, perioda, (časový) úsek} \dicExampleIS{í einni lotu} \dicExampleCS{na jeden zátah}
\dicEntry[lotið] \dicTerm{lotið} \dicIPA{{l}{\textopeno}{\textlengthmark}{\textsubring{d}}{\textsci}{\texttheta}} \dicPos{v} \dicFlx{supin} \dicLink{lúta}
\dicEntry[lotinn] \dicTerm{lotinn} \dicIPA{{l}{\textopeno}{\textlengthmark}{\textsubring{d}}{\textsci}{\textsubring{n}}} \dicPos{adj}[6]\dicFlx{}[-2] \dicSynonym{boginn} \dicDirectTranslationCS{shrbený} \dicExampleIS{lotinn í herðum} \dicExampleCS{shrbený v~zádech}
\dicEntry[lotning] \dicTerm{lot··ning} \dicIPA{{l}{\textopeno}{h}{\textsubring{d}}{n}{i}{\ng}{\r{g}}} \dicPos{f}[4] \dicFlx{(‑ar)}[7] \dicSynonym{virðing} \dicDirectTranslationCS{úcta, uctívání} \dicExampleIS{sýna e‑m lotningu} \dicExampleCS{projevovat ke (komu) úctu}
\begin{xtolerant}{}{1pt}
\dicEntry[lotningarhreimur] \dicTerm{lotningar··hreim|ur}\addthinS\dicIPA{{l}{\textopeno}{h}{\textsubring{d}}{n}{i}{\ng}{\r{g}}{a}{\textsubring{r}}{ei}{m}{\textscy}{\textsubring{r}}}\addthinS\dicPos{m}[6]\addthinS\dicFlx{(‑s, ‑ar)}[22] \dicDirectTranslationCS{pochlebovačný tón}
\end{xtolerant}
\dicEntry[lottó] \dicTerm{lottó} \dicIPA{{l}{\textopeno}{h}{\textsubring{d}}{ou}} \dicPos{n}[2] \dicFlx{(‑s, ‑)}[34] \dicDirectTranslationCS{loterie, sportka}
\dicEntry[lotugræðgi] \dicTerm{lotu··græðg|i} \dicIPA{{l}{\textopeno}{\textlengthmark}{\textsubring{d}}{\textscy}{\r{g}}{r}{a}{i}{ð}{\r{g}}{\textsci}} \dicPos{f}[3] \dicFlx{(‑i)}[3] \dicFieldCat{med.} \dicDirectTranslationCS{bulimie}
\dicEntry[lotukerfi] \dicTerm{lotu··kerfi} \dicIPA{{l}{\textopeno}{\textlengthmark}{\textsubring{d}}{\textscy}{c\smash{\textsuperscript{h}}}{\textepsilon}{r}{v}{\textsci}} \dicPos{n}[2] \dicFlx{(‑s)}[20] \dicFieldCat{chem.} \dicDirectTranslationCS{periodická tabulka\,/\addthin soustava prvků}
\dicEntry[lotutúlkun] \dicTerm{lotu··túlk|un} \dicIPA{{l}{\textopeno}{\textlengthmark}{\textsubring{d}}{\textscy}{t\smash{\textsuperscript{h}}}{u}{\textsubring{l}}{\r{g}}{\textscy}{\textsubring{n}}} \dicPos{f}[7] \dicFlx{(‑unar, ‑anir)}[8] \dicDirectTranslationCS{konsekutivní tlumočení}
\dicEntry[lóa] \dicTerm{ló|a} \dicIPA{{l}{ou}{\textlengthmark}{a}} \dicPos{f}[1] \dicFlx{(‑u, ‑ur)}[7] \dicFieldCat{zool.} \dicSynonym{heiðlóa} \dicDirectTranslationCS{kulík, kulík zlatý} \textit{(l.~{\textLA{Pluvialis apricaria}})}  \dicsymPhoto\ 
\dicFigure{21284.jpg}{Lóa}{Lóa - Karney, Lee, Biolib, PD}
\dicEntry[lóð] \dicTerm{lóð\smash{\textsuperscript{1}}} \dicIPA{{l}{ou}{\textlengthmark}{\texttheta}} \dicPos{f}[7] \dicFlx{(‑ar, ‑ir)}[1] \dicSynonym*{landspilda} \dicDirectTranslationCS{parcela, pozemek} \dicExampleIS{Húsið stendur á lóðinni.} \dicExampleCS{Dům stojí na pozemku.}
\dicEntry[lóð] \dicTerm{lóð\smash{\textsuperscript{2}}} \dicIPA{{l}{ou}{\textlengthmark}{\texttheta}} \dicPos{f}[7] \dicFlx{(‑ar, ‑ir)}[1] \dicFieldCat{nám.} \dicSynonym{fiskilína} \dicDirectTranslationCS{síť s~mnoha háčky};  \dicPhraseIS{leggja lóðirnar} {\textbf{a.}} \dicDirectTranslationCS{položit síť} \dicIndirectTranslationCS{(síť s~mnoha háčky)};  {\textbf{b.}} \dicSynonym*{selja upp af sjóveiki} \dicDirectTranslationCS{zvracet} \dicIndirectTranslationCS{(kvůli mořské nemoci)}
\dicEntry[lóð] \dicTerm{lóð\smash{\textsuperscript{3}}} \dicIPA{{l}{ou}{\textlengthmark}{\texttheta}} \dicPos{n}[2] \dicFlx{(‑s, ‑)}[5] \dicDirectTranslationCS{závaží} \dicExampleIS{lóð á vog} \dicExampleCS{závaží na váze};  \dicPhraseIS{það er lóðið} \dicLangCat{přen.} \dicDirectTranslationCS{to je jádro věci}
\dicEntry[lóða] \dicTerm{lóð|a} \dicIPA{{l}{ou}{\textlengthmark}{ð}{a}} \dicPos{v}[1] \dicFlx{(‑aði)}[1] \dicFlx{acc} \textbf{1.} \dicFieldCat{techn.} \dicDirectTranslationCS{letovat, pájet} \dicExampleIS{lóða e‑ð saman} \dicExampleCS{sletovat (co)}  \textbf{2.} \dicSynonym*{mæla dýpt} \dicDirectTranslationCS{sondovat (dno nebo moře pro výskyt ryb)};  \dicPhraseIS{lóða á fisk(i)} \dicFieldCat{nám.} \dicDirectTranslationCS{hledat ryby sonarem}
\dicEntry[lóðbolti] \dicTerm{lóð··bolt|i} \dicIPA{{l}{ou}{ð}{\textsubring{b}}{\textopeno}{\textsubring{l}}{\textsubring{d}}{\textsci}} \dicPos{m}[1] \dicFlx{(‑a, ‑ar)}[1] \dicFieldCat{techn.} \dicDirectTranslationCS{páječka, pájedlo}
\dicEntry[lóðréttur] \dicTerm{lóð··réttur} \dicIPA{{l}{ou}{ð}{r}{j}{\textepsilon}{h}{\textsubring{d}}{\textscy}{\textsubring{r}}} \dicPos{adj}[1]\dicFlx{}[-10] \dicDirectTranslationCS{svislý, vertikální, kolmý}
\dicEntry[lóðun] \dicTerm{lóð|un} \dicIPA{{l}{ou}{\textlengthmark}{ð}{\textscy}{\textsubring{n}}} \dicPos{f}[7] \dicFlx{(‑unar)}[9] \dicFieldCat{techn.} \dicSynonym*{það að lóða} \dicDirectTranslationCS{letování, pájení}
\dicEntry[lófaklapp] \dicTerm{lófa··klapp} \dicIPA{{l}{ou}{\textlengthmark}{v}{a}{k\smash{\textsuperscript{h}}}{l}{a}{h}{\textsubring{b}}} \dicPos{n}[2] \dicFlx{(‑s)}[2] \dicDirectTranslationCS{potlesk, aplaus}
\dicEntry[lófatak] \dicTerm{lófa··|tak} \dicIPA{{l}{ou}{\textlengthmark}{v}{a}{t\smash{\textsuperscript{h}}}{a}{\r{g}}} \dicPos{n}[2] \dicFlx{(‑taks, ‑tök)}[8] \dicSynonym{lófaklapp} \dicDirectTranslationCS{potlesk, aplaus}
\dicEntry[lófatölva] \dicTerm{lófa··tölv|a} \dicIPA{{l}{ou}{\textlengthmark}{v}{a}{t\smash{\textsuperscript{h}}}{\oe}{l}{v}{a}} \dicPos{f}[1] \dicFlx{(‑u, ‑ur)}[7] \dicFieldCat{poč.} \dicDirectTranslationCS{palmtop, PDA, kapesní počítač}
\dicEntry[lófi] \dicTerm{lóf|i} \dicsymFrequent\  \dicIPA{{l}{ou}{\textlengthmark}{v}{\textsci}} \dicPos{m}[1] \dicFlx{(‑a, ‑ar)}[1] \dicDirectTranslationCS{dlaň} \dicExampleIS{geyma hring í lófa sínum} \dicExampleCS{držet prsten ve své dlani}
\dicEntry[lófótur] \textls[15]{\dicTerm{ló··|fótur} \dicIPA{{l}{ou}{\textlengthmark}{f}{ou}{\textsubring{d}}{\textscy}{\textsubring{r}}} \dicPos{m}[13] \dicFlx{(‑fótar, ‑fætur)}[5] \dicFieldCat{bot.} \dicDirectTranslationCS{prustka obecná}} \textit{(l.~{\textLA{Hippuris vulgaris}})}  \dicsymPhoto\ 
\dicFigure{3847.jpg}{Lófótur}{Lófótur - Zicha Ondřej, Biolib, Copyright/CC-BY-NC}
\dicEntry[lóga] \dicTerm{lóg|a} \dicIPA{{l}{ou}{\textlengthmark}{a}} \dicPos{v}[1] \dicFlx{(‑aði)}[34] \dicFlx{dat} \textbf{1.} \dicSynonym{slátra} \dicDirectTranslationCS{porazit, porážet (ovci ap.)} \dicExampleIS{lóga fé} \dicExampleCS{porážet ovce}  \textbf{2.} \dicSynonym*{aflífa} \dicDirectTranslationCS{utratit, uspat (psa ap.)} \dicExampleIS{lóga hundi} \dicExampleCS{uspat psa}
\dicEntry[lógaritmi] \dicTerm{lógaritm|i} \dicIPA{{l}{ou}{\textlengthmark}{\r{g}}{a}{r}{\textsci}{h}{\textsubring{d}}{m}{\textsci}} \dicPos{m}[1] \dicFlx{(‑a, ‑ar)}[1] \dicFieldCat{mat.} \dicDirectTranslationCS{logaritmus}
\dicEntry[lómur] \dicTerm{lóm|ur} \dicIPA{{l}{ou}{\textlengthmark}{m}{\textscy}{\textsubring{r}}} \dicPos{m}[6] \dicFlx{(‑s, ‑ar)}[24] \textbf{1.} \dicFieldCat{zool.} \dicDirectTranslationCS{potáplice malá} \textit{(l.~{\textLA{Gavia stellata}})}  \dicsymPhoto\   \textbf{2.} \dicSynonym{kveinstafir} \dicDirectTranslationCS{nářky, stížnosti};  \dicPhraseIS{berja\,/\addthin lemja lóminn} \dicDirectTranslationCS{stěžovat si, naříkat}
\dicFigure{21071.jpg}{Lómur}{Lómur - Bowman, Tim, Biolib, PD}
\dicEntry[lón] \dicTerm{lón} \dicIPA{{l}{ou}{\textlengthmark}{\textsubring{n}}} \dicPos{n}[2] \dicFlx{(‑s, ‑)}[5] \dicDirectTranslationCS{laguna};  \dicPhraseIS{Bláa lónið} \dicFieldCat{geog.} \dicDirectTranslationCS{Modrá laguna} \dicIndirectTranslationCS{(laguna v~blízkosti Keflavíku)}
\dicEntry[lóuþræll] \dicTerm{lóu··þræl|l} \dicIPA{{l}{ou}{\textlengthmark}{\textscy}{\texttheta}{r}{a}{i}{\textsubring{d}}{\textsubring{l}}} \dicPos{m}[6] \dicFlx{(‑s, ‑ar)}[51] \dicFieldCat{zool.} \dicDirectTranslationCS{jespák, jespák obecný} \textit{(l.~{\textLA{Calidris alpina}})}  \dicsymPhoto\ 
\dicFigure{21101.jpg}{Lóuþræll}{Lóuþræll - Bowman, Tim, Biolib, PD}
\dicEntry[luggarasegl] \dicTerm{luggara··segl} \dicIPA{{l}{\textscy}{\r{g}}{\textlengthmark}{a}{r}{a}{s}{\textepsilon}{\r{g}}{\textsubring{l}}} \dicPos{n}[2] \dicFlx{(‑s, ‑)}[5] \dicFieldCat{nám.} \dicDirectTranslationCS{lugrová plachta}
\dicEntry[lugt] \dicTerm{lugt} \dicIPA{{l}{\textscy}{x}{\textsubring{d}}} \dicPos{f}[7] \dicFlx{(‑ar, ‑ir)}[1] \dicLink{lukt}
\dicEntry[lugum] \dicTerm{lugum} \dicIPA{{l}{\textscy}{\textlengthmark}{\textbabygamma}{\textscy}{\textsubring{m}}} \dicPos{v} \dicFlx{ind pf pl 1 pers} \dicLink{ljúga}
\dicEntry[lukið] \dicTerm{lukið} \dicIPA{{l}{\textscy}{\textlengthmark}{\r{\textObardotlessj}}{\textsci}{\texttheta}} \dicPos{v} \dicFlx{supin} \dicLink{lykja}
\dicEntry[lukka] \dicTerm{lukk|a} \dicIPA{{l}{\textscy}{h}{\r{g}}{a}} \dicPos{f}[1] \dicFlx{(‑u)}[5] \dicSynonym{hamingja} \dicDirectTranslationCS{štěstí, zdar} \dicExampleIS{óska e‑m til lukku með afmælið} \dicExampleCS{popřát (komu) štěstí k~narozeninám}
\dicEntry[lukkast] \dicTerm{lukk|ast} \dicIPA{{l}{\textscy}{h}{\r{g}}{a}{s}{\textsubring{d}}} \dicPos{v}[1] \dicFlx{(‑aðist)}[97] \dicFlx{refl} \dicSynonym{heppnast} \dicDirectTranslationCS{(po)dařit se, zdařit se} \dicExampleIS{Hjónabandið lukkast vel.} \dicExampleCS{Manželství se daří.}
\dicEntry[lukkulegur] \dicTerm{lukku··legur} \dicIPA{{l}{\textscy}{h}{\r{g}}{\textscy}{l}{\textepsilon}{\textbabygamma}{\textscy}{\textsubring{r}}} \dicPos{adj}[1]\dicFlx{}[-8] \dicSynonym{ánægður} \dicDirectTranslationCS{šťastný, spokojený, potěšený}
\dicEntry[lukt] \dicTerm{lukt}\dicTerm{, lugt} \dicIPA{{l}\-{\textscy}\-{x}\-{\textsubring{d}}\-} \dicPos{f}[7] \dicFlx{(‑ar, ‑ir)}[1] \textbf{1.} \dicDirectTranslationCS{lucerna, svítilna} \dicExampleIS{vasalukt} \dicExampleCS{kapesní svítilna}  \textbf{2.} \dicDirectTranslationCS{přední světlo (u~auta, kola ap.)}
\dicEntry[lukti] \dicTerm{lukti} \dicIPA{{l}{\textscy}{x}{\textsubring{d}}{\textsci}} \dicPos{v} \dicFlx{ind pf sg 1 pers} \dicLink{lykja}
\dicEntry[luktum] \dicTerm{luktum} \dicIPA{{l}{\textscy}{x}{\textsubring{d}}{\textscy}{\textsubring{m}}} \dicPos{v} \dicFlx{ind pf pl 1 pers} \dicLink{lykja}
\dicEntry[luktur] \dicTerm{luktur} \dicIPA{{l}{\textscy}{x}{\textsubring{d}}{\textscy}{\textsubring{r}}} \dicPos{adj}[1]\dicFlx{}[-10] \dicDirectTranslationCS{(u)zavřený, obklopený}
\dicEntry[lukum] \dicTerm{lukum} \dicIPA{{l}{\textscy}{\textlengthmark}{\r{g}}{\textscy}{\textsubring{m}}} \dicPos{v} \dicFlx{ind pf pl 1 pers} \dicLink{ljúka}
\dicEntry[luma] \dicTerm{lum|a} \dicIPA{{l}{\textscy}{\textlengthmark}{m}{a}} \dicPos{v}[1] \dicFlx{(‑aði)}[1] \dicPhraseIS{luma á e‑u} \dicDirectTranslationCS{ukrýt si (co), schovat si (co), ulít si (co)} \dicExampleIS{Hann lumar á peningum.} \dicExampleCS{Ukrývá si peníze.}
\dicEntry[lumbra] \dicTerm{lumbr|a} \dicIPA{{l}{\textscy}{m}{\textsubring{b}}{r}{a}} \dicPos{v}[1] \dicFlx{(‑aði)}[1] \dicDirectTranslationCS{zašantročit, zastrčit} \dicExampleIS{lumbra e‑u e‑s staðar} \dicExampleCS{zašantročit (co kde)};  \dicIdiom{lumbra}[á]{ \dicPhraseIS{lumbra á e‑m}} \dicSynonym{berja} \dicDirectTranslationCS{(z)bít (koho), (z)mlátit (koho)}
\dicEntry[lumma] \dicTerm{lumm|a} \dicIPA{{l}{\textscy}{m}{\textlengthmark}{a}} \dicPos{f}[1] \dicFlx{(‑u, ‑ur)}[7] \dicFieldCat{kulin.} \dicDirectTranslationCS{lívanec, lívaneček};  \dicPhraseIS{e‑að selst eins og heitar lummur} \dicFlx{refl} \dicLangCat{přen.} \dicDirectTranslationCS{(co) jde na dračku};  \dicPhraseIS{það er gömul lumma} \dicLangCat{přen.} \dicDirectTranslationCS{to je ohraná písnička}
\dicEntry[lummulegur] \dicTerm{lummu··legur} \dicIPA{{l}{\textscy}{m}{\textlengthmark}{\textscy}{l}{\textepsilon}{\textbabygamma}{\textscy}{\textsubring{r}}} \dicPos{adj}[1]\dicFlx{}[-8] \dicLangCat{hovor.} \dicSynonym{tilkomulítill} \dicDirectTranslationCS{nijaký, nezáživný, nudný}
\dicEntry[lund] \dicTerm{lund} \dicsymFrequent\  \dicIPA{{l}{\textscy}{n}{\textsubring{d}}} \dicPos{f}[7] \dicFlx{(‑ar, ‑ir)}[1] \textbf{1.} \dicSynonym{geð} \dicDirectTranslationCS{povaha, temperament} \dicExampleIS{vera léttur í lund} \dicExampleCS{být šťastný}  \textbf{2.} \dicSynonym{háttur\smash{\textsuperscript{1}}} \dicDirectTranslationCS{způsob}
\dicEntry[Lundey] \dicTerm{Lund··ey} \dicIPA{{l}{\textscy}{n}{\textsubring{d}}{ei}} \dicPos{f}[4] \dicFlx{(‑jar)}[17] \dicFieldCat{geog.} \dicDirectTranslationCS{Lundey} \dicIndirectTranslationCS{(ostrov severně od Islandu)}
\dicEntry[lundi] \dicTerm{lund|i} \dicIPA{{l}{\textscy}{n}{\textsubring{d}}{\textsci}} \dicPos{m}[1] \dicFlx{(‑a, ‑ar)}[1] \dicFieldCat{zool.} \dicDirectTranslationCS{papuchálek, papuchalk, papuchalk ploskozobý\,/\addthin bělobradý} \textit{(l.~{\textLA{Fratercula arctica}})}  \dicsymPhoto\ 
\dicFigure{ds_image_lundi_0_1.jpg}{Lundi}{Lundi - Tgo2002, CC BY-SA 2.5}
\dicEntry[lundir] \dicTerm{lundir} \dicIPA{{l}{\textscy}{n}{\textsubring{d}}{\textsci}{\textsubring{r}}} \dicPos{f}[7] \dicFlx{pl}[2] \dicDirectTranslationCS{jemné filety, svíčková}
\dicEntry[lundur] \dicTerm{lund|ur} \dicIPA{{l}{\textscy}{n}{\textsubring{d}}{\textscy}{\textsubring{r}}} \dicPos{m}[10] \dicFlx{(‑ar, ‑ir)}[4] \dicDirectTranslationCS{háj, lesík}
\dicEntry[Lundúnir] \dicTerm{Lundúnir} \dicIPA{{l}{\textscy}{n}{\textsubring{d}}{u}{n}{\textsci}{\textsubring{r}}} \dicPos{f}[12] \dicFlx{pl}[6] \dicFieldCat{geog.} \dicDirectTranslationCS{Londýn} \dicIndirectTranslationCS{(hlavní město Spojeného království)}
\dicEntry[lunga] \dicTerm{lung|a} \dicIPA{{l}{u}{\ng}{\r{g}}{a}} \dicPos{n}[1] \dicFlx{(‑a, ‑u)}[2] \dicFieldCat{anat.} \dicDirectTranslationCS{plíce}
\dicEntry[lungnabólga] \dicTerm{lungna··bólg|a} \dicIPA{{l}{u}{\ng}{n}{a}{\textsubring{b}}{ou}{l}{\r{g}}{a}} \dicPos{f}[1] \dicFlx{(‑u)}[5] \dicFieldCat{med.} \dicDirectTranslationCS{zápal plic, pneumonie}
\dicEntry[lungnakvef] \dicTerm{lungna··kvef} \dicIPA{{l}{u}{\ng}{n}{a}{k\smash{\textsuperscript{h}}}{v}{\textepsilon}{f}} \dicPos{n}[2] \dicFlx{(‑s)}[2] \dicFieldCat{med.} \dicDirectTranslationCS{zánět průdušek, bronchitida}
\dicEntry[lungnapípa] \dicTerm{lungna··píp|a} \dicIPA{{l}{u}{\ng}{n}{a}{p\smash{\textsuperscript{h}}}{i}{\textsubring{b}}{a}} \dicPos{f}[1] \dicFlx{(‑u, ‑ur)}[19] \dicFieldCat{anat.} \dicDirectTranslationCS{průduška}
\dicEntry[luntalegur] \dicTerm{lunta··legur} \dicIPA{{l}{\textscy}{\textsubring{n}}{\textsubring{d}}{a}{l}{\textepsilon}{\textbabygamma}{\textscy}{\textsubring{r}}} \dicPos{adj}[1]\dicFlx{}[-8] \dicSynonym{ólundarlegur} \dicDirectTranslationCS{mrzutý, nevrlý}
\dicEntry[luralegur] \dicTerm{lura··legur} \dicIPA{{l}{\textscy}{\textlengthmark}{r}{a}{l}{\textepsilon}{\textbabygamma}{\textscy}{\textsubring{r}}} \dicPos{adj}[1]\dicFlx{}[-8] \dicSynonym*{þyngslalegur} \dicDirectTranslationCS{nemotorný, neohrabaný}
\dicEntry[lustum] \dicTerm{lustum} \dicIPA{{l}{\textscy}{s}{\textsubring{d}}{\textscy}{\textsubring{m}}} \dicPos{v} \dicFlx{ind pf pl 1 pers} \dicLink{ljósta}
\dicEntry[lutum] \dicTerm{lutum} \dicIPA{{l}{\textscy}{\textlengthmark}{\textsubring{d}}{\textscy}{\textsubring{m}}} \dicPos{v} \dicFlx{ind pf pl 1 pers} \dicLink{lúta}
\dicEntry[lúalegur] \dicTerm{lúa··legur} \dicIPA{{l}{u}{\textlengthmark}{a}{l}{\textepsilon}{\textbabygamma}{\textscy}{\textsubring{r}}} \dicPos{adj}[1]\dicFlx{}[-8] \textbf{1.} \dicSynonym{þreytulegur} \dicDirectTranslationCS{vyčerpaný, unavený}  \textbf{2.} \dicSynonym{ódrengilegur} \dicDirectTranslationCS{hanebný, sprostý} \dicExampleIS{lúalegar getsakir} \dicExampleCS{hanebné narážky}
\dicEntry[lúð] \dicTerm{lúð} \dicIPA{{l}{u}{\textlengthmark}{\texttheta}} \dicPos{v} \dicFlx{supin} \dicLink{lýja}
\dicEntry[lúða] \dicTerm{lúð|a} \dicIPA{{l}{u}{\textlengthmark}{ð}{a}} \dicPos{f}[1] \dicFlx{(‑u, ‑ur)}[19] \dicFieldCat{zool.} \dicSynonym{flyðra} \dicDirectTranslationCS{platýs, platýs obecný} \textit{(l.~{\textLA{Hippoglossus hippoglossus}})}
\dicEntry[lúðalegur] \dicTerm{lúða··legur} \dicIPA{{l}{u}{\textlengthmark}{ð}{a}{l}{\textepsilon}{\textbabygamma}{\textscy}{\textsubring{r}}} \dicPos{adj}[1]\dicFlx{}[-8] \dicDirectTranslationCS{slizký, odporný}
\dicEntry[lúði] \dicTerm{lúð|i\smash{\textsuperscript{1}}} \dicIPA{{l}{u}{\textlengthmark}{ð}{\textsci}} \dicPos{m}[1] \dicFlx{(‑a, ‑ar)}[1] \dicDirectTranslationCS{křupan, balík}
\dicEntry[lúði] \dicTerm{lúði\smash{\textsuperscript{2}}} \dicIPA{{l}{u}{\textlengthmark}{ð}{\textsci}} \dicPos{v} \dicFlx{ind pf sg 1 pers} \dicLink{lýja}
\dicEntry[lúðrasveit] \dicTerm{lúðra··sveit} \dicIPA{{l}{u}{ð}{r}{a}{s}{v}{ei}{\textsubring{d}}} \dicPos{f}[7] \dicFlx{(‑ar, ‑ir)}[1] \dicFieldCat{hud.} \dicDirectTranslationCS{dechová kapela, dechovka}
\dicEntry[lúðum] \dicTerm{lúðum} \dicIPA{{l}{u}{\textlengthmark}{ð}{\textscy}{\textsubring{m}}} \dicPos{v} \dicFlx{ind pf pl 1 pers} \dicLink{lýja}
\dicEntry[lúður] \dicTerm{lúð|ur} \dicIPA{{l}{u}{\textlengthmark}{ð}{\textscy}{\textsubring{r}}} \dicPos{m}[5] \dicFlx{(‑urs, ‑rar)}[1] \dicFieldCat{hud.} \dicDirectTranslationCS{trumpeta} \dicExampleIS{blása í lúður} \dicExampleCS{foukat na trumpetu}
\dicEntry[lúðurþeytari] \dicTerm{lúður··þeyt·ar|i} \dicIPA{{l}{u}{\textlengthmark}{ð}{\textscy}{\textsubring{r}}{\texttheta}{ei}{\textsubring{d}}{a}{r}{\textsci}} \dicPos{m}[1] \dicFlx{(‑a, ‑ar)}[13] \dicDirectTranslationCS{trubač(ka)}
\dicEntry[lúga] \dicTerm{lúg|a} \dicIPA{{l}{u}{\textlengthmark}{a}} \dicPos{f}[1] \dicFlx{(‑u, ‑ur)}[7] \dicDirectTranslationCS{poklop, příklop}
\dicEntry[lúi] \dicTerm{lú|i} \dicIPA{{l}{u}{\textlengthmark}{\textsci}} \dicPos{m}[1] \dicFlx{(‑a)}[3] \dicSynonym{þreyta\smash{\textsuperscript{1}}} \dicDirectTranslationCS{znavení, vyčerpanost, vysílení} \dicExampleIS{finna fyrir lúa eftir ferðina} \dicExampleCS{cítit vysílení po cestě}
\dicEntry[lúið] \dicTerm{lúið} \dicIPA{{l}{u}{\textlengthmark}{\textsci}{\texttheta}} \dicPos{v} \dicFlx{supin} \dicLink{lýja}
\dicEntry[lúinn] \dicTerm{lúinn} \dicIPA{{l}{u}{\textlengthmark}{\textsci}{\textsubring{n}}} \dicPos{adj}[6]\dicFlx{}[-2] \textbf{1.} \dicSynonym{þreyttur} \dicDirectTranslationCS{znavený, vyčerpaný, vysílený}  \textbf{2.} \dicSynonym{slitinn} \dicDirectTranslationCS{opotřebovaný, obnošený}
\dicEntry[lúka] \dicTerm{lúk|a} \dicIPA{{l}{u}{\textlengthmark}{\r{g}}{a}} \dicPos{f}[1] \dicFlx{(‑u, ‑ur)}[19] \textbf{1.} \dicSynonym{lófi} \dicDirectTranslationCS{dlaň};  \dicPhraseIS{vera með lífið í lúkunum} \dicLangCat{přen.} \dicDirectTranslationCS{mít srdce v~kalhotách}  \textbf{2.} \dicSynonym{handfylli} \dicDirectTranslationCS{hrst} \dicExampleIS{lúka af salti} \dicExampleCS{hrst soli}
\dicEntry[lúkar] \dicTerm{lúkar} \dicIPA{{l}{u}{\textlengthmark}{\r{g}}{a}{\textsubring{r}}} \dicPos{m}[4] \dicFlx{(‑s, ‑ar)}[15] \dicFieldCat{nám.} \dicSynonym{káeta} \dicDirectTranslationCS{důstojnická kabina\,/\addthin kajuta}
\dicEntry[lúlla] \dicTerm{lúll|a} \dicIPA{{l}{u}{l}{\textlengthmark}{a}} \dicPos{v}[1] \dicFlx{(‑aði)}[44] \dicLangCat{dět.} \dicSynonym{sofa} \dicDirectTranslationCS{spinkat, hajat} \dicExampleIS{Farðu nú að lúlla.} \dicExampleCS{Jdi spinkat.}
\dicEntry[lúmskur] \dicTerm{lúmskur} \dicIPA{{l}{u}{m}{s}{\r{g}}{\textscy}{\textsubring{r}}} \dicPos{adj}[1]\dicFlx{}[-1] \dicSynonym{undirförull} \dicDirectTranslationCS{podlý, zákeřný};  \dicPhraseIS{hafa lúmskt gaman af að (gera e‑ð)} \dicDirectTranslationCS{mít škodolibou radost z~(dělání (čeho))}
\dicEntry[lúpína] \dicTerm{lúpín|a} \dicIPA{{l}{u}{\textlengthmark}{p\smash{\textsuperscript{h}}}{i}{n}{a}} \dicPos{f}[1] \dicFlx{(‑u, ‑ur)}[7] \dicFieldCat{bot.} \dicDirectTranslationCS{lupina nutkajská, vlčí bob} \textit{(l.~{\textLA{Lupinus nootkatensis}})}  \dicsymPhoto\ 
\dicFigure{ds_image_lupina_0_1.jpg}{Lúpína}{Lúpína - Jutta234, GFDL}
\dicEntry[lúpulegur] \dicTerm{lúpu··legur} \dicIPA{{l}{u}{\textlengthmark}{\textsubring{b}}{\textscy}{l}{\textepsilon}{\textbabygamma}{\textscy}{\textsubring{r}}}\addthinS\dicPos{adj}[1]\dicFlx{}[-8] \dicSynonym{skömmustulegur} \dicDirectTranslationCS{zahanbený, (jsoucí) v~rozpacích}
\dicEntry[lúr] \dicTerm{lúr} \dicIPA{{l}{u}{\textlengthmark}{\textsubring{r}}} \dicPos{m}[4] \dicFlx{(‑s, ‑ar)}[14] \dicSynonym{blundur} \dicDirectTranslationCS{zdřímnutí, šlofík} \dicExampleIS{fá sér svolítinn lúr} \dicExampleCS{trochu si zdřímnout}
\dicEntry[lúra] \dicTerm{lúr|a} \dicIPA{{l}{u}{\textlengthmark}{r}{a}} \dicPos{v}[2] \dicFlx{(‑ði, ‑að\,/\addthin ‑t)}[184] \dicSynonym{sofa} \dicDirectTranslationCS{zdřímnout si, dřímat, podřimovat};  \dicIdiom{lúra}[á]{ \dicPhraseIS{lúra á e‑u}} \dicDirectTranslationCS{schovávat (co), ukrýt (si) (co)} \dicExampleIS{lúra á góðgæti} \dicExampleCS{schovávat dobroty}
\dicEntry[lús] \dicTerm{lús} \dicIPA{{l}{u}{\textlengthmark}{s}} \dicPos{f}[9] \dicFlx{(‑ar, lýs)}[6] \dicFieldCat{zool.} \dicDirectTranslationCS{veš} \textit{(l.~{\textLA{Anoplura}})};  \dicPhraseIS{læðast eins og lús með saum(i)} \dicFlx{refl} \dicLangCat{přen.} \dicDirectTranslationCS{loudat se jako hlemýžď};  \dicPhraseIS{lús kviknaði milli þeirra} \dicLangCat{přen.} \dicDirectTranslationCS{dostali se do křížku};  \dicPhraseIS{nú detta mér allar dauðar lýs úr höfði} \dicLangCat{přen.} \dicDirectTranslationCS{to jsem z~toho jelen}
\dicEntry[lúsarlegur] \dicTerm{lúsar··legur} \dicIPA{{l}{u}{\textlengthmark}{s}{a}{r}{l}{\textepsilon}{\textbabygamma}{\textscy}{\textsubring{r}}} \dicPos{adj}[1]\dicFlx{}[-8] \dicSynonym{smásálarlegur} \dicDirectTranslationCS{ubohý, tristní, mizerný (plat ap.)}
\dicEntry[lúta] \dicTerm{lúta} \dicsymFrequent\  \dicIPA{{l}{u}{\textlengthmark}{\textsubring{d}}{a}} \dicPos{v}[6] \dicFlx{(lýt, laut, lutum, lyti, lotið)}[106] \dicFlx{dat} \textbf{1.} \dicSynonym{beygja\smash{\textsuperscript{2}}} \dicDirectTranslationCS{sklonit, sklánět, sehnout, shýbat};  \dicPhraseIS{lúta höfði} \dicDirectTranslationCS{sklonit hlavu}  \textbf{2.} \dicDirectTranslationCS{sklonit se, podrobit se};  \dicPhraseIS{lúta e‑m} \dicDirectTranslationCS{sklonit se před (kým), podrobit se (komu)};  \dicIdiom{lúta}[að]{ \dicPhraseIS{lúta að e‑u}} \dicDirectTranslationCS{vztahovat se na (co), týkat se (čeho)};  \dicIdiom{lúta}{ \dicPhraseIS{lúta (svo) lágt}} \dicLangCat{přen.} \dicDirectTranslationCS{snížit se, klesnout} \dicIndirectTranslationCS{(propůjčit se k~něčemu špatnému)}; { \dicPhraseIS{lúta í lægra haldi fyrir e‑m}} \dicLangCat{přen.} \dicDirectTranslationCS{sklonit se před (kým)} \dicIndirectTranslationCS{(přijmout prohru)}
\dicEntry[lúterskur] \dicTerm{lúterskur} \dicIPA{{l}{u}{\textlengthmark}{t\smash{\textsuperscript{h}}}{e}{\textsubring{r}}{s}{\r{g}}{\textscy}{\textsubring{r}}} \dicPos{adj}[1]\dicFlx{}[-1] \dicFieldCat{náb.} \dicDirectTranslationCS{luteránský}
\dicEntry[lúterstrú] \dicTerm{lúters··trú} \dicIPA{{l}{u}{\textlengthmark}{t\smash{\textsuperscript{h}}}{\textepsilon}{\textsubring{r}}{s}{t\smash{\textsuperscript{h}}}{r}{u}} \dicPos{f}[4] \dicFlx{(‑ar)}[27] \dicFieldCat{náb.} \dicDirectTranslationCS{luteránství}
\dicEntry[lútur] \dicTerm{lút|ur} \dicIPA{{l}{u}{\textlengthmark}{\textsubring{d}}{\textscy}{\textsubring{r}}} \dicPos{m}[6] \dicFlx{(‑s\,/\addthin ‑ar, ‑ar)}[65] \dicFieldCat{chem.} \dicDirectTranslationCS{louh}
\dicEntry[Lúxemborg] \dicTerm{Lúxem··borg} \dicIPA{{l}{u}{x}{s}{\textepsilon}{m}{\textsubring{b}}{\textopeno}{r}{\r{g}}} \dicPos{f}[7] \dicFlx{(‑ar)}[4] \textbf{1.} \dicFieldCat{geog.} \dicDirectTranslationCS{Lucembursko}  \textbf{2.} \dicFieldCat{geog.} \dicDirectTranslationCS{Lucemburk} \dicIndirectTranslationCS{(hlavní město Lucemburska)}
\dicEntry[Lúxemborgari] \dicTerm{Lúxem··borg·ar|i} \dicIPA{{l}{u}{x}{s}{\textepsilon}{m}{\textsubring{b}}{\textopeno}{r}{\r{g}}{a}{r}{\textsci}} \dicPos{m}[1] \dicFlx{(‑a, ‑ar)}[13] \dicDirectTranslationCS{Lucemburčan(ka)}
\dicEntry[lúxemborgíska] \dicTerm{lúxem··borgísk|a} \dicIPA{{l}{u}{x}{s}{\textepsilon}{m}{\textsubring{b}}{\textopeno}{r}{\r{\textObardotlessj}}{i}{s}{\r{g}}{a}} \dicPos{f}[1] \dicFlx{(‑u)}[5] \dicDirectTranslationCS{lucemburština}
\dicEntry[lúxemborgskur] \dicTerm{lúxem··borgskur} \dicIPA{{l}{u}{x}{s}{\textepsilon}{m}{\textsubring{b}}{\textopeno}{r}{\r{g}}{s}{\r{g}}{\textscy}{\textsubring{r}}} \dicPos{adj}[1]\dicFlx{}[-6] \dicDirectTranslationCS{lucemburský}
\dicEntry[lúxus] \dicTerm{lúxus} \dicIPA{{l}{u}{x}{s}{\textscy}{s}} \dicPos{m}[4] \dicFlx{(‑s)}[21] \dicDirectTranslationCS{luxus, přepych}
\dicEntry[lyf] \dicTerm{lyf} \dicsymFrequent\  \dicIPA{{l}{\textsci}{\textlengthmark}{f}} \dicPos{n}[2] \dicFlx{(‑s, ‑)}[13] \dicDirectTranslationCS{lék} \dicExampleIS{lyf gegn lyfseðli} \dicExampleCS{lék na předpis};  \dicPhraseIS{lyf gegn e‑u} \dicDirectTranslationCS{lék proti (čemu), lék na (co)} \dicExampleIS{lyf gegn kvefi} \dicExampleCS{lék na rýmu};  \dicPhraseIS{lyf við e‑ð} \dicDirectTranslationCS{lék na (co)};  \dicPhraseIS{taka inn lyf} \dicDirectTranslationCS{brát léky};  \dicPhraseIS{verja á lyfjum} \dicDirectTranslationCS{být na prášcích}
\dicEntry[lyfjabúð] \dicTerm{lyfja··búð} \dicIPA{{l}{\textsci}{v}{j}{a}{\textsubring{b}}{u}{\texttheta}} \dicPos{f}[7] \dicFlx{(‑ar, ‑ir)}[1] \dicDirectTranslationCS{lékárna}
\dicEntry[lyfjafræði] \dicTerm{lyfja··fræð|i} \dicIPA{{l}{\textsci}{v}{j}{a}{f}{r}{a}{i}{ð}{\textsci}} \dicPos{f}[3] \dicFlx{(‑i)}[3] \textbf{1.} \dicDirectTranslationCS{lékárnictví, farmacie}  \textbf{2.} \dicDirectTranslationCS{farmakologie}
\dicEntry[lyfjafræðingur] \dicTerm{lyfja·fræð··ing|ur} \dicIPA{{l}{\textsci}{v}{j}{a}{f}{r}{a}{i}{ð}{i}{\ng}{\r{g}}{\textscy}{\textsubring{r}}} \dicPos{m}[6] \dicFlx{(‑s, ‑ar)}[8] \textbf{1.} \dicDirectTranslationCS{lékárník, lékárnice, farmaceut(ka)}  \textbf{2.} \dicDirectTranslationCS{farmakolog, farmakoložka}
\dicEntry[lyfjagras] \dicTerm{lyfja··|gras} \dicIPA{{l}{\textsci}{v}{j}{a}{\r{g}}{r}{a}{s}} \dicPos{n}[2] \dicFlx{(‑grass, ‑grös)}[8] \dicFieldCat{bot.} \dicDirectTranslationCS{tučnice obecná} \textit{(l.~{\textLA{Pinguicula vulgaris}})}  \dicsymPhoto\ 
\dicFigure{31254.jpg}{Lyfjagras}{Lyfjagras - Dvořák Václav, Biolib, Copyright/CC-BY-NC}
\dicEntry[lyfjakort] \dicTerm{lyfja··kort} \dicIPA{{l}{\textsci}{v}{j}{a}{k\smash{\textsuperscript{h}}}{\textopeno}{\textsubring{r}}{\textsubring{d}}} \dicPos{n}[2] \dicFlx{(‑s, ‑)}[5] \dicIndirectTranslationCS{karta pacienta s~informacemi o~užívaných lécích}
\dicEntry[lyfjakostnaður] \dicTerm{lyfja··kost·nað|ur} \dicIPA{{l}{\textsci}{v}{j}{a}{k\smash{\textsuperscript{h}}}{\textopeno}{s}{\textsubring{d}}{n}{a}{ð}{\textscy}{\textsubring{r}}} \dicPos{m}[10] \dicFlx{(‑ar)}[9] \dicDirectTranslationCS{náklady na léky}
\dicEntry[lyfjaskírteini] \dicTerm{lyfja··skír·teini} \dicIPA{{l}{\textsci}{v}{j}{a}{s}{\r{\textObardotlessj}}{i}{\textsubring{r}}{t\smash{\textsuperscript{h}}}{ei}{n}{\textsci}} \dicPos{n}[2] \dicFlx{(‑s, ‑)}[14] \dicIndirectTranslationCS{průkaz opravňující ke slevě na léky}
\dicEntry[lyfjaverð] \dicTerm{lyfja··verð} \dicIPA{{l}{\textsci}{v}{j}{a}{v}{\textepsilon}{r}{\texttheta}} \dicPos{n}[2] \dicFlx{(‑s)}[2] \dicDirectTranslationCS{cena léku}
\dicEntry[lyfjaverslun] \dicTerm{lyfja··versl|un} \dicIPA{{l}{\textsci}{v}{j}{a}{v}{\textepsilon}{\textsubring{r}}{s}{\textsubring{d}}{l}{\textscy}{\textsubring{n}}} \dicPos{f}[7] \dicFlx{(‑unar, ‑anir)}[8] \dicSynonym{lyfjabúð} \dicDirectTranslationCS{lékárna}
\dicEntry[lyflækningar] \dicTerm{lyf··lækn·ingar} \dicIPA{{l}{\textsci}{v}{l}{a}{i}{h}{\r{g}}{n}{i}{\ng}{\r{g}}{a}{\textsubring{r}}} \dicPos{f}[4] \dicFlx{pl}[6] \dicFieldCat{med.} \dicDirectTranslationCS{klinická medicína}
\dicEntry[lyfsali] \dicTerm{lyf··sal|i} \dicIPA{{l}{\textsci}{f}{s}{a}{l}{\textsci}} \dicPos{m}[1] \dicFlx{(‑a, ‑ar)}[8] \dicDirectTranslationCS{lékárník, lékárnice}
\dicEntry[lyfseðill] \dicTerm{lyf··seð|ill} \dicIPA{{l}{\textsci}{f}{s}{\textepsilon}{ð}{\textsci}{\textsubring{d}}{\textsubring{l}}} \dicPos{m}[6] \dicFlx{(‑ils, ‑lar)}[35] \dicDirectTranslationCS{(lékařský) předpis}
\dicEntry[lyfta] \dicTerm{lyft|a\smash{\textsuperscript{1}}} \dicIPA{{l}{\textsci}{f}{\textsubring{d}}{a}} \dicPos{f}[1] \dicFlx{(‑u, ‑ur)}[7] \textbf{1.} \dicDirectTranslationCS{výtah}  \textbf{2.} \dicDirectTranslationCS{zdviž}  \textbf{3.} \dicSynonym{skíðalyfta} \dicDirectTranslationCS{vlek (lyžařský ap.)}
\dicEntry[lyfta] \dicTerm{lyft|a\smash{\textsuperscript{2}}} \dicsymFrequent\  \dicIPA{{l}{\textsci}{f}{\textsubring{d}}{a}} \dicPos{v}[2] \dicFlx{(‑i, ‑)}[14] \dicFlx{dat} \dicSynonym{hefja} \dicDirectTranslationCS{zvednout, zvedat, zdvihat, nadzvednout, nadzdvihnout} \dicExampleIS{lyfta steini} \dicExampleCS{zvednout kámen};  \dicPhraseIS{lyfta brúnum} \dicDirectTranslationCS{zvednout obočí};  \dicIdiom{lyfta}[sér]{ \dicPhraseIS{lyfta sér}} \dicDirectTranslationCS{(na)kynout (těsto ap.)};  \dicIdiom{lyfta}[upp]{ \dicPhraseIS{lyfta sér upp}} \dicSynonym*{skemmta sér} \dicDirectTranslationCS{povyrazit si, odvázat se};  \dicIdiom{lyftast}{ \dicPhraseIS{lyftast}} \dicFlx{refl} \dicDirectTranslationCS{zvednout\,/\addthin zvedat se (letadlo ap.)}
\dicEntry[lyftari] \dicTerm{lyft··ar|i} \dicIPA{{l}{\textsci}{f}{\textsubring{d}}{a}{r}{\textsci}} \dicPos{m}[1] \dicFlx{(‑a, ‑ar)}[13] \dicFieldCat{techn.} \dicDirectTranslationCS{vysokozdvižný vozík}
\dicEntry[lyftiduft] \dicTerm{lyfti··duft} \dicIPA{{l}{\textsci}{f}{\textsubring{d}}{\textsci}{\textsubring{d}}{\textscy}{f}{\textsubring{d}}} \dicPos{n}[2] \dicFlx{(‑s)}[2] \dicDirectTranslationCS{prášek do pečiva}
\dicEntry[lyfting] \dicTerm{lyft··ing} \dicIPA{{l}{\textsci}{f}{\textsubring{d}}{i}{\ng}{\r{g}}} \dicPos{f}[4] \dicFlx{(‑ar, ‑ar)}[5] \textbf{1.} \dicSynonym*{það að lyfta} \dicDirectTranslationCS{zvednutí, zvedání, (vy)zdvižení, nadzvednutí}  \textbf{2.} \dicSynonym*{fjörgun} \dicDirectTranslationCS{pozvednutí, povznesení (na duchu ap.)} \dicExampleIS{lyfting andans} \dicExampleCS{povznesení na duchu}  \textbf{3.} \dicPhraseIS{lyftingar} \dicFlx{pl} \dicFieldCat{sport.} \dicDirectTranslationCS{vzpírání}
\begin{xtolerant}{}{1pt}
\dicEntry[lyftingamaður] \dicTerm{lyftinga··|maður}\addthinS\dicIPA{{l}{\textsci}{f}{\textsubring{d}}{i}{\ng}{\r{g}}{a}{m}{a}{ð}{\textscy}{\textsubring{r}}}\addthinS\dicPos{m}[13]\addthinS\dicFlx{(‑manns, ‑menn)}[2] \dicDirectTranslationCS{vzpě\-rač(ka)}
\end{xtolerant}
\dicEntry[lygamælir] \dicTerm{lyga··mæl|ir} \dicIPA{{l}{\textsci}{\textlengthmark}{\textbabygamma}{a}{m}{a}{i}{l}{\textsci}{\textsubring{r}}} \dicPos{m}[7] \dicFlx{(‑is, ‑ar)}[1] \dicDirectTranslationCS{detektor lži}
\dicEntry[lygar] \dicTerm{lygar} \dicIPA{{l}{\textsci}{\textlengthmark}{\textbabygamma}{a}{\textsubring{r}}} \dicPos{f} \dicFlx{pl nom} \dicLink{lygi\smash{\textsuperscript{1}}}
\dicEntry[lygari] \dicTerm{lyg··ar|i} \dicIPA{{l}{\textsci}{\textlengthmark}{\textbabygamma}{a}{r}{\textsci}} \dicPos{m}[1] \dicFlx{(‑a, ‑ar)}[13] \dicDirectTranslationCS{lhář(ka)}
\dicEntry[lygasaga] \dicTerm{lyga··|saga} \dicIPA{{l}{\textsci}{\textlengthmark}{\textbabygamma}{a}{s}{a}{\textbabygamma}{a}} \dicPos{f}[1] \dicFlx{(‑sögu, ‑sögur)}[14] \dicDirectTranslationCS{báchorka, pohádka}
\dicEntry[lygi] \dicTerm{lyg|i\smash{\textsuperscript{1}}} \dicsymFrequent\  \dicIPA{{l}{i}{j}{\textlengthmark}{\textsci}} \dicPos{f}[2] \dicFlx{(‑i, ‑ar)}[1] \dicDirectTranslationCS{lež, nepravda} \dicExampleIS{Bæjarstjórinn segir lygar.} \dicExampleCS{Starosta lže.};  \dicPhraseIS{fara með lygar} \dicDirectTranslationCS{lhát}
\dicEntry[lygi] \dicTerm{lygi\smash{\textsuperscript{2}}} \dicIPA{{l}{i}{j}{\textlengthmark}{\textsci}} \dicPos{v} \dicFlx{con pf sg 1 pers} \dicLink{ljúga}
\dicEntry[lygilegur] \textls[15]{\dicTerm{lygi··legur} \dicIPA{{l}{i}{j}{\textlengthmark}{\textsci}{l}{\textepsilon}{\textbabygamma}{\textscy}{\textsubring{r}}} \dicPos{adj}[1]\dicFlx{}[-8] \textbf{1.} \dicSynonym{ótrúlegur} \dicDirectTranslationCS{nevěrohodný, nepravděpodobný}}  \textbf{2.} \dicSynonym{ósannur} \dicDirectTranslationCS{nepravdivý, lživý}
\dicEntry[lyginn] \dicTerm{lyginn} \dicIPA{{l}{i}{j}{\textlengthmark}{\textsci}{\textsubring{n}}} \dicPos{adj}[6]\dicFlx{}[-2] \dicDirectTranslationCS{ulhaný, prolhaný}
\dicEntry[lygn] \dicTerm{lygn} \dicIPA{{l}{\textsci}{\r{g}}{\textsubring{n}}} \dicPos{adj}[5]\dicFlx{}[-5] \textbf{1.} \dicSynonym*{vindlaus} \dicDirectTranslationCS{bezvětrný, klidný} \dicIndirectTranslationCS{(o~počasí)}  \textbf{2.} \dicSynonym*{straumlaus} \dicDirectTranslationCS{nehybný, nezčeřený, klidný} \dicIndirectTranslationCS{(o~vodě)}
\dicEntry[lygna] \dicTerm{lygn|a\smash{\textsuperscript{1}}} \dicIPA{{l}{\textsci}{\r{g}}{n}{a}} \dicPos{f}[1] \dicFlx{(‑u, ‑ur)}[7] \textbf{1.} \dicSynonym*{kyrrt vatn} \dicDirectTranslationCS{klidná voda}  \textbf{2.} \dicSynonym*{kyrrt loft} \dicDirectTranslationCS{bezvětří}
\dicEntry[lygna] \dicTerm{lygn|a\smash{\textsuperscript{2}}} \dicIPA{{l}{\textsci}{\r{g}}{n}{a}} \dicPos{v}[2] \dicFlx{(‑di, ‑t)}[150] \dicFlx{dat} \textbf{1.} \dicPhraseIS{það lygnir} \dicFlx{impers} \dicSynonym*{kyrrast} \dicDirectTranslationCS{utišuje se (vítr), přestává foukat}  \textbf{2.} \dicPhraseIS{lygna aftur augunum} \dicDirectTranslationCS{zavřít\,/\addthin přivřít oči}
\dicEntry[lyk] \dicTerm{lyk} \dicIPA{{l}{\textsci}{\textlengthmark}{\r{g}}} \dicPos{v} \dicFlx{ind praes sg 1 pers} \dicLink{lykja}
\dicEntry[lyki] \dicTerm{lyki} \dicIPA{{l}{\textsci}{\textlengthmark}{\r{\textObardotlessj}}{\textsci}} \dicPos{v} \dicFlx{con pf sg 1 pers} \dicLink{ljúka}
\dicEntry[lykilaðstaða] \dicTerm{lykil··að·|staða} \dicIPA{{l}{\textsci}{\textlengthmark}{\r{\textObardotlessj}}{\textsci}{l}{a}{ð}{s}{\textsubring{d}}{a}{ð}{a}} \dicPos{f}[1] \dicFlx{(‑stöðu)}[2] \dicDirectTranslationCS{klíčové postavení}
\dicEntry[lykilatriði] \dicTerm{lykil··at·riði} \dicIPA{{l}{\textsci}{\textlengthmark}{\r{\textObardotlessj}}{\textsci}{l}{a}{\textsubring{d}}{r}{\textsci}{ð}{\textsci}} \dicPos{n}[2] \dicFlx{(‑s, ‑)}[14] \dicDirectTranslationCS{klíčový bod, klíčová otázka}
\dicEntry[lykill] \dicTerm{lyk|ill} \dicsymFrequent\  \dicIPA{{l}{\textsci}{\textlengthmark}{\r{\textObardotlessj}}{\textsci}{\textsubring{d}}{\textsubring{l}}} \dicPos{m}[6] \dicFlx{(‑ils, ‑lar)}[35] \textbf{1.} \dicDirectTranslationCS{klíč (od dveří ap.)};  \dicPhraseIS{lykill að e‑u} \dicDirectTranslationCS{klíč od (čeho)} \dicExampleIS{lykill að húsinu} \dicExampleCS{klíč od domu}  \textbf{2.} \dicFieldCat{techn.} \dicDirectTranslationCS{klíč} \dicIndirectTranslationCS{(ruční nástroj)}  \textbf{3.} \dicFieldCat{hud.} \dicDirectTranslationCS{klíč} \dicIndirectTranslationCS{(hudební značka)} \dicExampleIS{nótnalykill} \dicExampleCS{hudební klíč}  \textbf{4.} \dicSynonym{skýring} \dicDirectTranslationCS{klíč, nápověda} \dicIndirectTranslationCS{(pomůcka podstatná pro vysvětlení něčeho)} \dicExampleIS{orðalykill} \dicExampleCS{konkordance}
\dicEntry[lykilmaður] \dicTerm{lykil··|maður} \dicIPA{{l}{\textsci}{\textlengthmark}{\r{\textObardotlessj}}{\textsci}{l}{m}{a}{ð}{\textscy}{\textsubring{r}}} \dicPos{m}[13] \dicFlx{(‑manns, ‑menn)}[2] \dicDirectTranslationCS{klíčová postava}
\dicEntry[lykilorð] \dicTerm{lykil··orð} \dicIPA{{l}{\textsci}{\textlengthmark}{\r{\textObardotlessj}}{\textsci}{l}{\textopeno}{r}{\texttheta}} \dicPos{n}[2] \dicFlx{(‑s, ‑)}[5] \textbf{1.} \dicSynonym{aðalatriði} \dicDirectTranslationCS{klíčové slovo}  \textbf{2.} \dicFieldCat{poč.} \dicSynonym*{aðgangsorð} \dicDirectTranslationCS{(přístupové) heslo}
\dicEntry[lykja] \dicTerm{lykja} \dicIPA{{l}{\textsci}{\textlengthmark}{\r{\textObardotlessj}}{a}} \dicPos{v}[4] \dicFlx{(lyk, lukti, luktum, lykti, lukið)}[61] \dicDirectTranslationCS{obklopit, obklopovat, obstoupit, obstupovat} \dicExampleIS{Vatn lukti um hana.} \dicExampleCS{Voda ji obklopila.};  \dicIdiom{lykjast}{ \dicPhraseIS{lykjast}} \dicFlx{refl} \dicDirectTranslationCS{obklopit, obklopovat, obstoupit, obstupovat}
\dicEntry[lykkja] \dicTerm{lykkj|a} \dicIPA{{l}{\textsci}{h}{\r{\textObardotlessj}}{a}} \dicPos{f}[1] \dicFlx{(‑u, ‑ur)}[25] \textbf{1.} \dicSynonym{hanki} \dicDirectTranslationCS{smyčka, oko, klička (na provázku ap.)} \dicExampleIS{lykkja á bandi} \dicExampleCS{smyčka na provázku}  \textbf{2.} \dicDirectTranslationCS{oko} \dicIndirectTranslationCS{(v~pletení)}  \textbf{3.} \dicSynonym{getnaðarvörn} \dicDirectTranslationCS{spirála} \dicIndirectTranslationCS{(druh antikoncepce)}
\dicEntry[lykkjufall] \dicTerm{lykkju··|fall} \dicIPA{{l}{\textsci}{h}{\r{\textObardotlessj}}{\textscy}{f}{a}{\textsubring{d}}{\textsubring{l}}} \dicPos{n}[2] \dicFlx{(‑falls, ‑föll)}[8] \dicDirectTranslationCS{puštěné oko} \dicIndirectTranslationCS{(v~pletení)} \dicExampleIS{lykkjufall á sokkabuxurnar} \dicExampleCS{oko na punčoše}
\dicEntry[lyklaborð] \dicTerm{lykla··borð} \dicsymFrequent\  \dicIPA{{l}{\textsci}{h}{\r{g}}{l}{a}{\textsubring{b}}{\textopeno}{r}{\texttheta}} \dicPos{n}[2] \dicFlx{(‑s, ‑)}[5] \dicFieldCat{poč.} \dicDirectTranslationCS{klávesnice} \dicExampleIS{íslenskt lyklaborð} \dicExampleCS{islandská klávesnice}
\dicEntry[lyklakippa] \dicTerm{lykla··kipp|a} \dicIPA{{l}{\textsci}{h}{\r{g}}{l}{a}{c\smash{\textsuperscript{h}}}{\textsci}{h}{\textsubring{b}}{a}} \dicPos{f}[1] \dicFlx{(‑u, ‑ur)}[19] \dicDirectTranslationCS{svazek klíčů}
\dicEntry[lykt] \dicTerm{lykt\smash{\textsuperscript{1}}} \dicIPA{{l}{\textsci}{x}{\textsubring{d}}} \dicPos{f}[7] \dicFlx{(‑ar, ‑ir)}[1] \dicSynonym{endir} \dicDirectTranslationCS{konec, závěr};  \dicPhraseIS{að lyktum} \dicFlx{adv} \dicSynonym*{að lokum} \dicDirectTranslationCS{nakonec, na závěr};  \dicPhraseIS{leiða e‑ð til lykta} \dicSynonym*{ljúka e‑u} \dicDirectTranslationCS{dovést (co) do konce, dokončit (co)};  \dicPhraseIS{gera lyktir á e‑u} \dicSynonym*{ljúka við e‑ð} \dicDirectTranslationCS{dokončit\,/\addthin dokončovat (co)}
\dicEntry[lykt] \dicTerm{lykt\smash{\textsuperscript{2}}} \dicsymFrequent\  \dicIPA{{l}{\textsci}{x}{\textsubring{d}}} \dicPos{f}[7] \dicFlx{(‑ar, ‑ir)}[1] \dicSynonym{ilmur} \dicDirectTranslationCS{(zá)pach, vůně, aroma} \dicExampleIS{lykt af blómunum} \dicExampleCS{vůně květin}
\dicEntry[lykta] \dicTerm{lykt|a\smash{\textsuperscript{1}}} \dicIPA{{l}{\textsci}{x}{\textsubring{d}}{a}} \dicPos{v}[1] \dicFlx{(‑aði)}[5] \dicPhraseIS{e‑u lyktar} \dicFlx{impers} \dicDirectTranslationCS{(co) končí} \dicExampleIS{Leiknum lyktaði með jafntefli.} \dicExampleCS{Zápas skončil remízou.}
\dicEntry[lykta] \dicTerm{lykt|a\smash{\textsuperscript{2}}} \dicIPA{{l}{\textsci}{x}{\textsubring{d}}{a}} \dicPos{v}[1] \dicFlx{(‑aði)}[34] \textbf{1.} \dicSynonym{ilma} \dicDirectTranslationCS{páchnout, být cítit, vonět, vydávat vůni} \dicExampleIS{lykta vel} \dicExampleCS{vonět dobře};  \dicPhraseIS{lykta af e‑u} \dicDirectTranslationCS{páchnout (čím), být cítit (čím), vonět (čím)} \dicExampleIS{lykta af ilmvatni} \dicExampleCS{vonět parfémem}  \textbf{2.} \dicSynonym{þefa} \dicDirectTranslationCS{přivonět (si), přičichnout (si)};  \dicPhraseIS{lykta af e‑u} \dicSynonym{þefa} \dicDirectTranslationCS{přivonět (si) k~(čemu)} \dicExampleIS{lykta af blóminu} \dicExampleCS{přivonět si ke květině}
\dicEntry[lyktandi] \dicTerm{lykt··andi} \dicIPA{{l}{\textsci}{x}{\textsubring{d}}{a}{n}{\textsubring{d}}{\textsci}} \dicPos{adj}[13] \dicFlx{indecl}[1] \dicDirectTranslationCS{vonící, šířící vůni};  \dicPhraseIS{e‑að er illa lyktandi} \dicDirectTranslationCS{(co) smrdí\,/\addthin páchne}
\dicEntry[lykti] \dicTerm{lykti} \dicIPA{{l}{\textsci}{x}{\textsubring{d}}{\textsci}} \dicPos{v} \dicFlx{con pf sg 1 pers} \dicLink{lykja}
\dicEntry[lyktnæmur] \dicTerm{lykt··næmur} \dicIPA{{l}{\textsci}{x}{\textsubring{d}}{n}{a}{i}{m}{\textscy}{\textsubring{r}}} \dicPos{adj}[1]\dicFlx{}[-1] \dicDirectTranslationCS{mající dobrý čich}
\dicEntry[lymska] \dicTerm{lymsk|a} \dicIPA{{l}{\textsci}{m}{s}{\r{g}}{a}} \dicPos{f}[1] \dicFlx{(‑u)}[5] \dicSynonym*{lævísi} \dicDirectTranslationCS{záludnost, zákeřnost, proradnost}
\dicEntry[lymskur] \dicTerm{lymskur} \dicIPA{{l}{\textsci}{m}{s}{\r{g}}{\textscy}{\textsubring{r}}} \dicPos{adj}[1]\dicFlx{}[-1] \dicSynonym{lævís} \dicDirectTranslationCS{záludný, zákeřný, proradný}
\dicEntry[lynda] \dicTerm{lyn|da} \dicIPA{{l}{\textsci}{n}{\textsubring{d}}{a}} \dicPos{v}[2] \dicFlx{(‑ti, ‑t)}[44] \dicPhraseIS{e‑m lyndir við e‑n} \dicFlx{impers} \dicDirectTranslationCS{(kdo) si rozumí s~(kým), (kdo) má pochopení pro (koho)} \dicExampleIS{Mér lyndir ekki við hana.} \dicExampleCS{Nerozumím si s~ní.};  \dicPhraseIS{e‑jum lyndir} \dicFlx{impers} \dicDirectTranslationCS{(kdo) si rozumí, (kdo) má pochopení jeden pro druhého};  \dicPhraseIS{láta sér e‑ð lynda} \dicDirectTranslationCS{spokojit se s~(čím)}
\dicEntry[lyndi] \dicTerm{lyndi} \dicIPA{{l}{\textsci}{n}{\textsubring{d}}{\textsci}} \dicPos{n}[2] \dicFlx{(‑s)}[20] \dicSynonym{skap} \dicDirectTranslationCS{nálada, rozpoložení};  \dicPhraseIS{e‑að leikur í lyndi} \dicDirectTranslationCS{(co) klape, (co) hraje do karet};  \dicPhraseIS{e‑m leikur allt í lyndi} \dicFlx{impers} \dicDirectTranslationCS{(komu) vše vychází, (komu) vše hraje do noty}
\dicEntry[lyndiseinkunn] \dicTerm{lyndis··ein·kunn} \dicIPA{{l}{\textsci}{n}{\textsubring{d}}{\textsci}{s}{ei}{\r{\ng}}{\r{g}}{\textscy}{\textsubring{n}}} \dicPos{f}[7] \dicFlx{(‑ar, ‑ir)}[1] \dicSynonym*{skaphöfn} \dicDirectTranslationCS{charakter}
\dicEntry[lyng] \dicTerm{lyng} \dicIPA{{l}{i}{\ng}{\r{g}}} \dicPos{n}[2] \dicFlx{(‑s)}[2] \dicFieldCat{bot.} \dicDirectTranslationCS{vřesovec} \textit{(l.~{\textLA{Erica}})}  \dicsymPhoto\ 
\dicFigure{74636.jpg}{Lyng}{Lyng - Kesl Michael, Biolib, Copyright/CC-BY-NC}
\dicEntry[lyst] \dicTerm{lyst} \dicIPA{{l}{\textsci}{s}{\textsubring{d}}} \dicPos{f}[7] \dicFlx{(‑ar, ‑ir)}[1] \textbf{1.} \dicSynonym{matarlyst} \dicDirectTranslationCS{chuť (k~jídlu), apetit};  \dicPhraseIS{hafa lyst á e‑u} \dicDirectTranslationCS{mít chuť na (co)} \dicExampleIS{hafa góða lyst á matnum} \dicExampleCS{mít dobrý apetit}  \textbf{2.} \dicSynonym{löngun} \dicDirectTranslationCS{chuť, přání};  \dicPhraseIS{af hjartans lyst} \dicFlx{adv} \dicDirectTranslationCS{z~celého srdce}
\dicEntry[lystarlaus] \dicTerm{lystar··laus} \dicIPA{{l}{\textsci}{s}{\textsubring{d}}{a}{r}{l}{\oe i}{s}} \dicPos{adj}[5]\dicFlx{}[-1] \dicDirectTranslationCS{(jsoucí) bez apetitu, nemající chuť (na jídlo)}
\dicEntry[lystarleysi] \dicTerm{lystar··leysi} \dicIPA{{l}{\textsci}{s}{\textsubring{d}}{a}{r}{l}{ei}{s}{\textsci}} \dicPos{n}[2] \dicFlx{(‑s)}[20] \dicDirectTranslationCS{nechutenství}
\dicEntry[lystarstol] \dicTerm{lystar··stol} \dicIPA{{l}{\textsci}{s}{\textsubring{d}}{a}{\textsubring{r}}{s}{\textsubring{d}}{\textopeno}{\textsubring{l}}} \dicPos{n}[2] \dicFlx{(‑s)}[2] \dicFieldCat{med.} \dicDirectTranslationCS{(mentální) anorexie}
\dicEntry[lysti] \dicTerm{lysti} \dicIPA{{l}{\textsci}{s}{\textsubring{d}}{\textsci}} \dicPos{v} \dicFlx{con pf sg 1 pers} \dicLink{ljósta}
\dicEntry[lystibátur] \dicTerm{lysti··bát|ur} \dicIPA{{l}{\textsci}{s}{\textsubring{d}}{\textsci}{\textsubring{b}}{au}{\textsubring{d}}{\textscy}{\textsubring{r}}} \dicPos{m}[6] \dicFlx{(‑s, ‑ar)}[22] \dicFieldCat{nám.} \dicDirectTranslationCS{jachta}
\dicEntry[lystigarður] \dicTerm{lysti··garð|ur} \dicIPA{{l}{\textsci}{s}{\textsubring{d}}{\textsci}{\r{g}}{a}{r}{ð}{\textscy}{\textsubring{r}}} \dicPos{m}[6] \dicFlx{(‑s, ‑ar)}[5] \dicSynonym{skemmtigarður} \dicDirectTranslationCS{(zábavní) park}
\dicEntry[lystiskip] \dicTerm{lysti··skip} \dicIPA{{l}{\textsci}{s}{\textsubring{d}}{\textsci}{s}{\r{\textObardotlessj}}{\textsci}{\textsubring{b}}} \dicPos{n}[2] \dicFlx{(‑s, ‑)}[5] \dicDirectTranslationCS{jachta, výletní loď}
\dicEntry[lystugur] \dicTerm{lyst··ugur} \dicIPA{{l}{\textsci}{s}{\textsubring{d}}{\textscy}{\textbabygamma}{\textscy}{\textsubring{r}}} \dicPos{adj}[1]\dicFlx{}[-8] \textbf{1.} \dicSynonym{sólginn} \dicDirectTranslationCS{mající dobrý apetit} \dicExampleIS{vera lystugur á matinn} \dicExampleCS{mít dobrý apetit}  \textbf{2.} \dicSynonym*{sem vekur lyst} \dicDirectTranslationCS{chutný, vábný} \dicIndirectTranslationCS{(o~jídle)}
\dicEntry[lyti] \dicTerm{lyti} \dicIPA{{l}{\textsci}{\textlengthmark}{\textsubring{d}}{\textsci}} \dicPos{v} \dicFlx{con pf sg 1 pers} \dicLink{lúta}
\dicEntry[lýðfrjáls] \dicTerm{lýð··frjáls} \dicIPA{{l}{i}{ð}{f}{r}{j}{au}{l}{s}} \dicPos{adj}[5]\dicFlx{}[-1] \dicDirectTranslationCS{demokratický, svobodný} \dicExampleIS{lýðfrjálst land} \dicExampleCS{demokratická země}
\dicEntry[lýðháskóli] \dicTerm{lýð··há·skól|i} \dicIPA{{l}{i}{\textlengthmark}{\texttheta}{h}{au}{s}{\r{g}}{ou}{l}{\textsci}} \dicPos{m}[1] \dicFlx{(‑a, ‑ar)}[1] \dicFieldCat{škol.} \dicDirectTranslationCS{veřejná vysoká škola}
\dicEntry[lýðhylli] \dicTerm{lýð··hyll|i} \dicIPA{{l}{i}{\textlengthmark}{\texttheta}{h}{\textsci}{\textsubring{d}}{l}{\textsci}} \dicPos{f}[3] \dicFlx{(‑i)}[3] \dicDirectTranslationCS{přízeň lidu, popularita}
\dicEntry[lýði] \dicTerm{lýði} \dicIPA{{l}{i}{\textlengthmark}{ð}{\textsci}} \dicPos{v} \dicFlx{con pf sg 1 pers} \dicLink{lýja}
\dicEntry[lýðréttindi] \dicTerm{lýð··rétt·indi} \dicIPA{{l}{i}{ð}{r}{j}{\textepsilon}{h}{\textsubring{d}}{\textsci}{n}{\textsubring{d}}{\textsci}} \dicPos{n}[2] \dicFlx{pl}[19] \dicDirectTranslationCS{občanská práva}
\dicEntry[lýðræði] \dicTerm{lýð··ræði} \dicIPA{{l}{i}{ð}{r}{a}{i}{ð}{\textsci}} \dicPos{n}[2] \dicFlx{(‑s)}[20] \dicDirectTranslationCS{demokracie}
\dicEntry[lýðræðislegur] \dicTerm{lýð·ræðis··legur} \dicIPA{{l}{i}{ð}{r}{a}{i}{ð}{\textsci}{s}{l}{\textepsilon}{\textbabygamma}{\textscy}{\textsubring{r}}} \dicPos{adj}[1]\dicFlx{}[-8] \dicDirectTranslationCS{demokratický}
\dicEntry[lýðræðissinnaður] \dicTerm{lýð·ræðis··sinn·|aður} \dicIPA{{l}\-{i}\-{ð}\-{r}\-{a}\-{i}\-{ð}\-{\textsci}\-{s}\-{\textsci}\-{n}\-{a}\-{ð}\-{\textscy}\-{\textsubring{r}}\-} \dicPos{adj}[3] \dicFlx{(f ‑uð)}[3] \dicDirectTranslationCS{(pro)demokratický}
\dicEntry[lýðræðissinni] \dicTerm{lýð·ræðis··sinn|i} \dicIPA{{l}{i}{ð}{r}{a}{i}{ð}{\textsci}{s}{\textsci}{n}{\textsci}} \dicPos{m}[1] \dicFlx{(‑a, ‑ar)}[1] \dicDirectTranslationCS{demokrat(ka)}
\dicEntry[lýðskrum] \dicTerm{lýð··skrum} \dicIPA{{l}{i}{ð}{s}{\r{g}}{r}{\textscy}{\textsubring{m}}} \dicPos{n}[2] \dicFlx{(‑s)}[2] \dicDirectTranslationCS{demagogie, populismus}
\dicEntry[lýðskrumari] \dicTerm{lýð··skrum·ar|i} \dicIPA{{l}{i}{ð}{s}{\r{g}}{r}{\textscy}{m}{a}{r}{\textsci}} \dicPos{m}[1] \dicFlx{(‑a, ‑ar)}[13] \dicDirectTranslationCS{demagog, demagožka, populista, populistka}
\dicEntry[lýður] \dicTerm{lýð|ur} \dicIPA{{l}{i}{\textlengthmark}{ð}{\textscy}{\textsubring{r}}} \dicPos{m}[9] \dicFlx{(‑s, ‑ir)}[8] \textbf{1.} \dicSynonym{almenningur} \dicDirectTranslationCS{lid(é), občané}  \textbf{2.} \dicSynonym{skríll} \dicDirectTranslationCS{lůza, holota};  \dicPhraseIS{e‑að er enn við lýði} \dicDirectTranslationCS{(co) stále existuje, (co) se stále vyskytuje (zvyk ap.)}
\dicEntry[lýðveldi] \dicTerm{lýð··veldi} \dicIPA{{l}{i}{ð}{v}{\textepsilon}{l}{\textsubring{d}}{\textsci}} \dicPos{n}[2] \dicFlx{(‑s, ‑)}[14] \dicDirectTranslationCS{republika}
\dicEntry[lýg] \dicTerm{lýg} \dicIPA{{l}{i}{\textlengthmark}{x}} \dicPos{v} \dicFlx{ind praes sg 1 pers} \dicLink{ljúga}
\dicEntry[lýi] \dicTerm{lýi} \dicIPA{{l}{i}{j}{\textsci}} \dicPos{v} \dicFlx{ind praes sg 1 pers} \dicLink{lýja}
\dicEntry[lýja] \dicTerm{lýja} \dicIPA{{l}{i}{j}{\textlengthmark}{a}} \dicPos{v}[4] \dicFlx{(lýi, lúði, lúðum, lýði, lúið\,/\addthin lúð)}[54] \dicFlx{acc} \textbf{1.} \dicDirectTranslationCS{kovat, kout} \dicExampleIS{lýja járn} \dicExampleCS{kovat železo}  \textbf{2.} \dicSynonym{þreyta\smash{\textsuperscript{2}}} \dicDirectTranslationCS{unavit, utahat} \dicExampleIS{Þetta lýir mann.} \dicExampleCS{To člověka unaví.};  \dicIdiom{lýjast}{ \dicPhraseIS{lýjast}} \dicFlx{refl} \dicSynonym*{þreytast} \dicDirectTranslationCS{vyčerpat se, unavit se} \dicExampleIS{Strákurinn er farinn að lýjast.} \dicExampleCS{Kluk se začíná unavovat.}
\dicEntry[lýjandi] \dicTerm{lýj··andi} \dicIPA{{l}{i}{j}{\textlengthmark}{a}{n}{\textsubring{d}}{\textsci}} \dicPos{adj}[13] \dicFlx{indecl}[1] \dicDirectTranslationCS{vyčerpávající, vysilující} \dicExampleIS{lýjandi starf} \dicExampleCS{vyčerpávající práce}
\dicEntry[lýk] \dicTerm{lýk} \dicIPA{{l}{i}{\textlengthmark}{\r{g}}} \dicPos{v} \dicFlx{ind praes sg 1 pers} \dicLink{ljúka}
\dicEntry[lýrik] \dicTerm{lýrik} \dicIPA{{l}{i}{\textlengthmark}{r}{\textsci}{\r{g}}} \dicPos{f}[10] \dicFlx{(‑ur)}[2] \dicFieldCat{lit.} \dicSynonym{ljóðræna} \dicDirectTranslationCS{lyrika}
\dicEntry[lýrískur] \dicTerm{lýrískur} \dicIPA{{l}{i}{\textlengthmark}{r}{i}{s}{\r{g}}{\textscy}{\textsubring{r}}} \dicPos{adj}[1]\dicFlx{}[-1] \dicFieldCat{lit.} \dicSynonym{ljóðrænn} \dicDirectTranslationCS{lyrický}
\dicEntry[lýs] \dicTerm{lýs} \dicIPA{{l}{i}{\textlengthmark}{s}} \dicPos{f} \dicFlx{pl nom} \dicLink{lús}
\dicEntry[lýsa] \dicTerm{lýs|a\smash{\textsuperscript{1}}} \dicIPA{{l}{i}{\textlengthmark}{s}{a}} \dicPos{f}[1] \dicFlx{(‑u, ‑ur)}[7] \textbf{1.} \dicSynonym{ljómi} \dicDirectTranslationCS{světlo, paprsek} \dicExampleIS{fyrsta lýsa morgunsins} \dicExampleCS{první ranní paprsek}  \textbf{2.} \dicFieldCat{zool.} \dicDirectTranslationCS{treska bezvousá} \textit{(l.~{\textLA{Merlangius merlangus}})}  \dicsymPhoto\ 
\dicFigure{ds_image_lysa_1_2.jpg}{Lýsa}{Lýsa - Georges Jansoone (JoJan), CC BY 3.0}
\dicEntry[lýsa] \dicTerm{lýs|a\smash{\textsuperscript{2}}} \dicsymFrequent\  \dicIPA{{l}{i}{\textlengthmark}{s}{a}} \dicPos{v}[2] \dicFlx{(‑ti, ‑t)}[67] \dicFlx{dat} \textbf{1.} \dicSynonym*{bera birtu} \dicDirectTranslationCS{osvětlit, osvětlovat, osvítit, ozářit, ozařovat} \dicExampleIS{Tunglið lýsir mér.} \dicExampleCS{Měsíc mě osvětluje.};  \dicPhraseIS{það lýsir (af degi)} \dicFlx{impers} \dicSynonym*{það birtir} \dicDirectTranslationCS{rozednívá se}  \textbf{2.} \dicSynonym*{lýsa mynd} \dicDirectTranslationCS{vyvolat, vyvolávat (film ap.)} \dicExampleIS{lýsa myndina} \dicExampleCS{vyvolat film}  \textbf{3.} \dicSynonym{sýna} \dicDirectTranslationCS{projevit, projevovat, vyjádřit, vyjadřovat}  \textbf{4.} \dicSynonym*{gefa lýsingu} \dicDirectTranslationCS{popsat, popisovat, (vy)líčit, vykreslit, vykreslovat} \dicExampleIS{lýsa reynslu sinni} \dicExampleCS{popsat svou zkušenost}  \textbf{5.} \dicSynonym{tilkynna} \dicDirectTranslationCS{prohlásit, prohlašovat, deklarovat};  \dicIdiom{lýsa}[eftir]{ \dicPhraseIS{lýsa eftir e‑m}} \dicDirectTranslationCS{vyhlásit pátrání po (kom)};  \dicIdiom{lýsa}[upp]{ \dicPhraseIS{lýsa upp e‑ð}} \dicDirectTranslationCS{osvětlit (co) (místnost ap.)};  \dicIdiom{lýsa}[yfir]{ \dicPhraseIS{lýsa yfir e‑u}} \dicDirectTranslationCS{ohlásit (co), oznámit (co)}
\dicEntry[lýsi] \dicTerm{lýsi} \dicIPA{{l}{i}{\textlengthmark}{s}{\textsci}} \dicPos{n}[2] \dicFlx{(‑s, ‑)}[14] \dicDirectTranslationCS{rybí tuk} \dicIndirectTranslationCS{(medicína)}
\dicEntry[lýsing] \dicTerm{lýs··ing} \dicsymFrequent\  \dicIPA{{l}{i}{\textlengthmark}{s}{i}{\ng}{\r{g}}} \dicPos{f}[4] \dicFlx{(‑ar, ‑ar)}[5] \textbf{1.} \dicDirectTranslationCS{osvětlení, nasvícení, iluminace} \dicExampleIS{lýsing í eldhúsinu} \dicExampleCS{osvětlení v~kuchyni}  \textbf{2.} \dicSynonym{frásögn} \dicDirectTranslationCS{popis, vylíčení, vykreslení} \dicExampleIS{lýsing á ferðinni} \dicExampleCS{vylíčení cesty}  \textbf{3.} \dicDirectTranslationCS{osvit, expozice}  \textbf{4.} \dicSynonym{skreyting} \dicDirectTranslationCS{ilustrace, ilustrování (rukopisu, knihy ap.)}
\dicEntry[lýsingarháttur] \dicTerm{lýsingar··|háttur} \dicIPA{{l}{i}{\textlengthmark}{s}{i}{\ng}{\r{g}}{a}{\textsubring{r}}{h}{au}{h}{\textsubring{d}}{\textscy}{\textsubring{r}}} \dicPos{m}[12] \dicFlx{(‑háttar, ‑hættir)}[7] \dicFieldCat{jaz.} \dicDirectTranslationCS{příčestí};  \dicPhraseIS{lýsingarháttur nútíðar} \dicFieldCat{jaz.} \dicDirectTranslationCS{příčestí přítomné};  \dicPhraseIS{lýsingarháttur þátíðar} \dicFieldCat{jaz.} \dicDirectTranslationCS{příčestí minulé}
\dicEntry[lýsingarorð] \dicTerm{lýsingar··orð} \dicIPA{{l}{i}{\textlengthmark}{s}{i}{\ng}{\r{g}}{a}{r}{\textopeno}{r}{\texttheta}} \dicPos{n}[2] \dicFlx{(‑s, ‑)}[5] \dicFieldCat{jaz.} \dicDirectTranslationCS{přídavné jméno, adjektivum} \textit{(l.~{\textLA{nomen adiectivum}})}
\dicEntry[lýst] \dicTerm{lýst} \dicIPA{{l}{i}{s}{\textsubring{d}}} \dicPos{v} \dicFlx{ind praes sg 1 pers} \dicLink{ljósta}
\dicEntry[lýt] \dicTerm{lýt} \dicIPA{{l}{i}{\textlengthmark}{\textsubring{d}}} \dicPos{v} \dicFlx{ind praes sg 1 pers} \dicLink{lúta}
\dicEntry[lýtalaus] \textls[15]{\dicTerm{lýta··laus} \dicIPA{{l}{i}{\textlengthmark}{\textsubring{d}}{a}{l}{\oe i}{s}} \dicPos{adj}[5]\dicFlx{}[-1] \dicSynonym{gallalaus} \dicDirectTranslationCS{bezchybný, (jsoucí) bez poskvrnky}}
\dicEntry[lýtalækningar] \dicTerm{lýta··lækn·ingar} \dicIPA{{l}{i}{\textlengthmark}{\textsubring{d}}{a}{l}{a}{i}{h}{\r{g}}{n}{i}{\ng}{\r{g}}{a}{\textsubring{r}}} \dicPos{f}[4] \dicFlx{pl}[6] \dicFieldCat{med.} \dicDirectTranslationCS{plastická chirurgie}
\dicEntry[lýtalæknir] \dicTerm{lýta··lækn|ir} \dicIPA{{l}{i}{\textlengthmark}{\textsubring{d}}{a}{l}{a}{i}{h}{\r{g}}{n}{\textsci}{\textsubring{r}}} \dicPos{m}[7] \dicFlx{(‑is, ‑ar)}[1] \dicDirectTranslationCS{plastický chirurg, plastická chirurgyně\,/\addthin chiruržka}
\dicEntry[lýti] \dicTerm{lýti} \dicIPA{{l}{i}{\textlengthmark}{\textsubring{d}}{\textsci}} \dicPos{n}[2] \dicFlx{(‑s, ‑)}[14] \textbf{1.} \dicSynonym{ljóður} \dicDirectTranslationCS{poskvrna, kaz, vada} \dicExampleIS{lýti á bókinni} \dicExampleCS{vada na knížce}  \textbf{2.} \dicSynonym{ljótleiki} \dicDirectTranslationCS{šerednost, ošklivost (vzhled člověka ap.)}
\dicEntry[læða] \dicTerm{læð|a\smash{\textsuperscript{1}}} \dicIPA{{l}{a}{i}{\textlengthmark}{ð}{a}} \dicPos{f}[1] \dicFlx{(‑u, ‑ur)}[19] \dicDirectTranslationCS{kočka} \dicIndirectTranslationCS{(samice)}
\dicEntry[læða] \dicTerm{læ|ða\smash{\textsuperscript{2}}} \dicsymFrequent\  \dicIPA{{l}{a}{i}{\textlengthmark}{ð}{a}} \dicPos{v}[2] \dicFlx{(‑ddi, ‑tt)}[172] \dicFlx{dat} \dicPhraseIS{læða e‑u að e‑m} \dicDirectTranslationCS{propašovat (komu co), vstrčit (komu co) skrytě} \dicExampleIS{læða þessari frétt að henni} \dicExampleCS{nenápadně jí sdělit zprávu};  \dicIdiom{læðast}{ \dicPhraseIS{læðast}} \dicFlx{refl} \dicDirectTranslationCS{(v)plížit se, plazit se, krást se} \dicExampleIS{læðast burt} \dicExampleCS{odplížit se pryč}
\dicEntry[læðupokast] \dicTerm{læðu··pok|ast} \dicIPA{{l}{a}{i}{\textlengthmark}{ð}{\textscy}{p\smash{\textsuperscript{h}}}{\textopeno}{\r{g}}{a}{s}{\textsubring{d}}} \dicPos{v}[1] \dicFlx{(‑aðist)}[91] \dicFlx{refl} \dicDirectTranslationCS{(při)krást se, plížit se}
\dicEntry[lægð] \dicTerm{lægð} \dicIPA{{l}{a}{i}{\textbabygamma}{\texttheta}} \dicPos{f}[7] \dicFlx{(‑ar, ‑ir)}[1] \textbf{1.} \dicSynonym{dæld} \dicDirectTranslationCS{prohlubeň, proláklina} \dicExampleIS{Það myndast djúp lægð í götuna.} \dicExampleCS{Na ulici se tvoří hluboká prohlubeň.}  \textbf{2.} \dicFieldCat{meteo.} \dicSynonym{lágþrýstisvæði} \dicDirectTranslationCS{tlaková níže} \dicAntonym{hægð}  \textbf{3.} \dicSynonym{deyfð} \dicDirectTranslationCS{útlum};  \dicPhraseIS{vera í lægð} \dicDirectTranslationCS{být v~útlumu}
\dicEntry[lægi] \dicTerm{lægi\smash{\textsuperscript{1}}} \dicIPA{{l}{a}{i}{j}{\textlengthmark}{\textsci}} \dicPos{n}[2] \dicFlx{(‑s, ‑)}[16] \dicSynonym{skipalægi} \dicDirectTranslationCS{kotviště}
\dicEntry[lægi] \dicTerm{lægi\smash{\textsuperscript{2}}} \dicIPA{{l}{a}{i}{j}{\textlengthmark}{\textsci}} \dicPos{v} \dicFlx{con pf sg 1 pers} \dicLink{liggja}
\dicEntry[lægja] \dicTerm{læg|ja} \dicIPA{{l}{a}{i}{j}{\textlengthmark}{a}} \dicPos{v}[2] \dicFlx{(‑ði, ‑t)}[90] \dicFlx{acc} \dicSynonym{auðmýkja} \dicDirectTranslationCS{zklidnit, zklidňovat, zmírnit, zmírňovat} \dicExampleIS{lægja æsing} \dicExampleCS{zklidnit vztek};  \dicPhraseIS{lægja rostann í e‑m} \dicDirectTranslationCS{zklidnit (koho)};  \dicPhraseIS{e‑ð lægir} \dicFlx{impers} {\textbf{a.}} \dicDirectTranslationCS{(co) klesá, (co) zapadá (slunce ap.)} \dicExampleIS{Sólina lægir.} \dicExampleCS{Slunce klesá.};  {\textbf{b.}} \dicDirectTranslationCS{(co) ustává (o~větru ap.), (co) se uklidňuje} \dicExampleIS{Veðrið lægir.} \dicExampleCS{Počasí se uklidňuje.}
\dicEntry[lægra] \dicTerm{lægra} \dicIPA{{l}{a}{i}{\textbabygamma}{r}{a}} \dicPos{adv} \dicFlx{comp} \dicLink{lágt}
\dicEntry[lægri] \dicTerm{lægri} \dicIPA{{l}{a}{i}{\textbabygamma}{r}{\textsci}} \dicPos{adj} \dicFlx{comp m} \dicLink{lágur}
\dicEntry[lægst] \dicTerm{lægst} \dicIPA{{l}{a}{i}{x}{s}{\textsubring{d}}} \dicPos{adv} \dicFlx{sup} \dicLink{lágt}
\dicEntry[lægstur] \dicTerm{lægstur} \dicIPA{{l}{a}{i}{x}{s}{\textsubring{d}}{\textscy}{\textsubring{r}}} \dicPos{adj} \dicFlx{m sg nom sup} \dicLink{lágur}
\dicEntry[læki] \dicTerm{læki} \dicIPA{{l}{a}{i}{\textlengthmark}{\r{\textObardotlessj}}{\textsci}} \dicPos{v} \dicFlx{con pf sg 1 pers} \dicLink{leka}
\dicEntry[lækka] \dicTerm{lækk|a} \dicsymFrequent\  \dicIPA{{l}{a}{i}{h}{\r{g}}{a}} \dicPos{v}[1] \dicFlx{(‑aði)}[48] \dicFlx{acc} \textbf{1.} \dicSynonym{minnka} \dicDirectTranslationCS{snížit, snižovat, (z)redukovat} \dicExampleIS{lækka skatta} \dicExampleCS{snížit daně};  \dicPhraseIS{lækka í e‑u} \dicDirectTranslationCS{zeslabit (co), ztišit (co), ztlumit (co), zmírnit (co)} \dicExampleIS{lækka í útvarpinu} \dicExampleCS{ztišit rádio};  \dicPhraseIS{lækka röddina\,/\addthin róminn} \dicDirectTranslationCS{snížit hlas}  \textbf{2.} \dicDirectTranslationCS{klesnout, klesat, sestoupit, sestupovat, snášet se} \dicExampleIS{lækka flugið} \dicExampleCS{klesat s~letadlem};  \dicPhraseIS{e‑að lækkar} \dicDirectTranslationCS{(co) klesá, (co) ustupuje (voda ap.)};  \dicPhraseIS{það lækkar í e‑u} \dicFlx{impers} \dicDirectTranslationCS{(co) klesá, (co) opadává} \dicExampleIS{Það lækkar í ánni.} \dicExampleCS{Voda v~řece opadává.}
\dicEntry[lækkun] \dicTerm{lækk|un} \dicIPA{{l}{a}{i}{h}{\r{g}}{\textscy}{\textsubring{n}}} \dicPos{f}[7] \dicFlx{(‑unar)}[9] \textbf{1.} \dicSynonym*{það að lækka} \dicDirectTranslationCS{snížení, snižování, redukce, redukování} \dicExampleIS{lækkun skatta} \dicExampleCS{snížení daní}  \textbf{2.} \dicSynonym*{það að e‑að lækkar} \dicDirectTranslationCS{klesání, opadání, ubývání} \dicExampleIS{lækkun á yfirborði sjávar} \dicExampleCS{klesání hladiny moře}
\dicEntry[lækna] \dicTerm{lækn|a} \dicsymFrequent\  \dicIPA{{l}{a}{i}{h}{\r{g}}{n}{a}} \dicPos{v}[1] \dicFlx{(‑aði)}[1] \dicFlx{acc} \dicDirectTranslationCS{(vy)léčit};  \dicPhraseIS{lækna e‑n af e‑u} \dicDirectTranslationCS{léčit (koho) z~(čeho)} \dicExampleIS{lækna e‑n af berklum} \dicExampleCS{léčit (koho) z~tuberkulózy}
\dicEntry[læknadeild] \dicTerm{lækna··deild} \dicIPA{{l}{a}{i}{h}{\r{g}}{n}{a}{\textsubring{d}}{ei}{l}{\textsubring{d}}} \dicPos{f}[7] \dicFlx{(‑ar, ‑ir)}[1] \dicFieldCat{škol.} \dicDirectTranslationCS{lékařská fakulta}
\dicEntry[læknandi] \dicTerm{lækn··andi} \dicIPA{{l}{a}{i}{h}{\r{g}}{n}{a}{n}{\textsubring{d}}{\textsci}} \dicPos{adj}[13] \dicFlx{indecl}[1] \dicDirectTranslationCS{léčivý, léčebný}
\dicEntry[læknanemi] \dicTerm{lækna··nem|i} \dicIPA{{l}{a}{i}{h}{\r{g}}{n}{a}{n}{\textepsilon}{m}{\textsci}} \dicPos{m}[1] \dicFlx{(‑a, ‑ar)}[1] \dicDirectTranslationCS{student(ka) medicíny\,/\addthin lékařství}
\dicEntry[læknanlegur] \dicTerm{læknan··legur} \dicIPA{{l}{a}{i}{h}{\r{g}}{n}{a}{n}{l}{\textepsilon}{\textbabygamma}{\textscy}{\textsubring{r}}} \dicPos{adj}[1]\dicFlx{}[-8] \dicDirectTranslationCS{(vy)léčitelný}
\dicEntry[læknastofa] \dicTerm{lækna··stof|a} \dicIPA{{l}{a}{i}{h}{\r{g}}{n}{a}{s}{\textsubring{d}}{\textopeno}{v}{a}} \dicPos{f}[1] \dicFlx{(‑u, ‑ur)}[7] \dicDirectTranslationCS{ordinace}
\dicEntry[lækning] \dicTerm{lækn··ing} \dicsymFrequent\  \dicIPA{{l}{a}{i}{h}{\r{g}}{n}{i}{\ng}{\r{g}}} \dicPos{f}[4] \dicFlx{(‑ar, ‑ar)}[5] \dicDirectTranslationCS{léčení, léčba};  \dicPhraseIS{lækning við e‑u} \dicDirectTranslationCS{léčba (čeho)} \dicExampleIS{lækning við sjúkdómi} \dicExampleCS{léčba nemoci}
\dicEntry[lækningastofa] \dicTerm{lækninga··stof|a} \dicIPA{{l}{a}{i}{h}{\r{g}}{n}{i}{\ng}{\r{g}}{a}{s}{\textsubring{d}}{\textopeno}{v}{a}} \dicPos{f}[1] \dicFlx{(‑u, ‑ur)}[7] \dicDirectTranslationCS{ordinace}
\dicEntry[læknir] \dicTerm{lækn|ir} \dicsymFrequent\  \dicIPA{{l}{a}{i}{h}{\r{g}}{n}{\textsci}{\textsubring{r}}} \dicPos{m}[7] \dicFlx{(‑is, ‑ar)}[1] \dicDirectTranslationCS{doktor(ka), lékař(ka)} \dicExampleIS{panta tíma hjá lækni} \dicExampleCS{objednat se k~lékaři}
\dicEntry[læknisfræði] \dicTerm{læknis··fræð|i} \dicIPA{{l}{a}{i}{h}{\r{g}}{n}{\textsci}{s}{f}{r}{a}{i}{ð}{\textsci}} \dicPos{f}[3] \dicFlx{(‑i)}[3] \dicDirectTranslationCS{lékařství, medicína} \dicIndirectTranslationCS{(nauka)}
\dicEntry[læknisskoðun] \dicTerm{læknis··skoð|un} \dicIPA{{l}{a}{i}{h}{\r{g}}{n}{\textsci}{s}{\r{g}}{\textopeno}{ð}{\textscy}{\textsubring{n}}} \dicPos{f}[7] \dicFlx{(‑unar)}[9] \dicDirectTranslationCS{zdravotní\,/\addthin lékařská prohlídka}
\dicEntry[læknisvottorð] \dicTerm{læknis··vott·orð} \dicIPA{{l}{a}{i}{h}{\r{g}}{n}{\textsci}{s}{v}{\textopeno}{h}{\textsubring{d}}{\textopeno}{r}{\texttheta}} \dicPos{n}[2] \dicFlx{(‑s, ‑)}[5] \dicDirectTranslationCS{lékařská zpráva}
\dicEntry[læknisþjónusta] \dicTerm{læknis··þjón·ust|a} \dicIPA{{l}{a}{i}{h}{\r{g}}{n}{\textsci}{s}{\texttheta}{j}{ou}{n}{\textscy}{s}{\textsubring{d}}{a}} \dicPos{f}[1] \dicFlx{(‑u)}[5] \dicDirectTranslationCS{zdravot\-ní\,/\addthin lékařská služba}
\dicEntry[lækur] \dicTerm{læk|ur} \dicsymFrequent\  \dicIPA{{l}{a}{i}{\textlengthmark}{\r{g}}{\textscy}{\textsubring{r}}} \dicPos{m}[10] \dicFlx{(‑jar, ‑ir)}[23] \textbf{1.} \dicDirectTranslationCS{potok, říčka} \dicExampleIS{ár og lækir} \dicExampleCS{řeky a~potoky}  \textbf{2.} \dicFieldCat{anat.} \dicSynonym{naflastrengur} \dicDirectTranslationCS{pupeční šňůra}
\dicEntry[læmingi] \dicTerm{læm··ing|i} \dicIPA{{l}{a}{i}{\textlengthmark}{m}{i}{\textltailn}{\r{\textObardotlessj}}{\textsci}} \dicPos{m}[1] \dicFlx{(‑ja, ‑jar)}[14] \dicFieldCat{zool.} \dicDirectTranslationCS{lumík norský} \textit{(l.~{\textLA{Lemmus lemmus}})}  \dicsymPhoto\ 
\dicFigure{ds_image_laemingi_0_1.jpg}{Læmingi}{Læmingi - Argus fin, PD}
\dicEntry[læra] \dicTerm{lær|a} \dicsymFrequent\  \dicIPA{{l}{a}{i}{\textlengthmark}{r}{a}} \dicPos{v}[2] \dicFlx{(‑ði, ‑t)}[103] \dicFlx{acc} \dicDirectTranslationCS{(na)učit se, studovat} \dicExampleIS{læra íslensku} \dicExampleCS{učit se islandštinu};  \dicPhraseIS{læra e‑ð utanbókar} \dicDirectTranslationCS{učit se (co) zpaměti, memorovat (co)} \dicExampleIS{læra utanbókar ljóð} \dicExampleCS{učit se básničku nazpaměť};  \dicIdiom{læra}[af]{ \dicPhraseIS{læra af e‑m}} \dicDirectTranslationCS{učit se od (koho)};  \dicIdiom{læra}[á]{ \dicPhraseIS{læra á e‑ð}} \dicDirectTranslationCS{učit se na (co) (na hudební nástroj ap.)} \dicExampleIS{Ég lærði á píanó.} \dicExampleCS{Učil jsem se na piano.};  \dicIdiom{læra}[fyrir]{ \dicPhraseIS{læra fyrir e‑ð}} \dicDirectTranslationCS{učit se na (co)} \dicExampleIS{læra fyrir próf} \dicExampleCS{učit se na zkoušku};  \dicIdiom{læra}[undir]{ \dicPhraseIS{læra undir e‑ð}} \dicDirectTranslationCS{učit se na (co) (na zkoušku ap.)} \dicExampleIS{læra undir skóla} \dicExampleCS{učit se do školy};  \dicIdiom{læra}[utan að]{ \dicPhraseIS{læra utan að}} \dicDirectTranslationCS{učit se zpaměti, memorovat};  \dicIdiom{lærast}{ \dicPhraseIS{e‑að lærist vel}} \dicFlx{refl} \dicDirectTranslationCS{(co) se učí dobře}; { \dicPhraseIS{e‑m lærist að (gera e‑ð)}} \dicFlx{refl impers} \dicDirectTranslationCS{(kdo) se učí (dělat (co)), (kdo) si osvojuje (dělat (co))};  \dicProverb\  \dicPhraseIS{Svo lengi lærir sem lifir.} \dicLangCat{přís.} \dicDirectTranslationCS{Člověk se učí po celý život.}
\dicEntry[lærbein] \dicTerm{lær··bein} \dicIPA{{l}{a}{i}{r}{\textsubring{b}}{ei}{\textsubring{n}}} \dicPos{n}[2] \dicFlx{(‑s, ‑)}[5] \dicFieldCat{anat.} \dicDirectTranslationCS{stehenní kost}
\dicEntry[lærdómur] \dicTerm{lær··dóm|ur} \dicIPA{{l}{a}{i}{r}{\textsubring{d}}{ou}{m}{\textscy}{\textsubring{r}}} \dicPos{m}[6] \dicFlx{(‑s, ‑ar)}[10] \dicSynonym{þekking} \dicDirectTranslationCS{učení, učenost, vzdělání};  \dicPhraseIS{draga lærdóma af e‑u} \dicDirectTranslationCS{poučit se z~(čeho)}
\dicEntry[lærður] \dicTerm{lærður} \dicsymFrequent\  \dicIPA{{l}{a}{i}{r}{ð}{\textscy}{\textsubring{r}}} \dicPos{adj}[2]\dicFlx{}[-1] \dicDirectTranslationCS{učený, vzdělaný} \dicExampleIS{mjög lærður maður} \dicExampleCS{velmi vzdělaný člověk}
\dicEntry[læri] \dicTerm{læri} \dicsymFrequent\  \dicIPA{{l}{a}{i}{\textlengthmark}{r}{\textsci}} \dicPos{n}[2] \dicFlx{(‑s, ‑)}[14] \textbf{1.} \dicFieldCat{anat.} \dicDirectTranslationCS{stehno}  \textbf{2.} \dicDirectTranslationCS{kýta} \dicExampleIS{læri af lambi} \dicExampleCS{jehněčí kýta}  \textbf{3.} \dicSynonym{nám} \dicDirectTranslationCS{učení} \dicIndirectTranslationCS{(učňovská léta)};  \dicPhraseIS{vera í læri hjá e‑m} \dicDirectTranslationCS{být u~(koho) v~učení}
\dicEntry[lærifaðir] \dicTerm{læri··|faðir} \dicIPA{{l}{a}{i}{\textlengthmark}{r}{\textsci}{f}{a}{ð}{\textsci}{\textsubring{r}}} \dicPos{m}[13] \dicFlx{(‑föður, ‑feður)}[3] \dicSynonym{kennari} \dicDirectTranslationCS{duchovní otec, mistr, guru}
\dicEntry[lærisveinn] \dicTerm{læri··svein|n} \dicIPA{{l}{a}{i}{\textlengthmark}{r}{\textsci}{s}{v}{ei}{\textsubring{d}}{\textsubring{n}}} \dicPos{m}[6] \dicFlx{(‑s, ‑ar)}[42] \dicSynonym{nemandi} \dicDirectTranslationCS{učeň(ka), učnice, učedník, učednice, žák(yně), žačka} \dicExampleIS{lærisveinar Jesú} \dicExampleCS{učedníci Ježíše}
\dicEntry[lærleggur] \dicTerm{lær··legg|ur} \dicIPA{{l}{a}{i}{r}{l}{\textepsilon}{\r{g}}{\textscy}{\textsubring{r}}} \dicPos{m}[9] \dicFlx{(‑jar\,/\addthin ‑s, ‑ir)}[26] \dicFieldCat{anat.} \dicDirectTranslationCS{stehenní kost}
\dicEntry[lærlingur] \dicTerm{lær··ling|ur} \dicIPA{{l}{a}{i}{r}{l}{i}{\ng}{\r{g}}{\textscy}{\textsubring{r}}} \dicPos{m}[6] \dicFlx{(‑s, ‑ar)}[8] \dicSynonym{nemandi} \dicDirectTranslationCS{učedník, učednice, učeň(ka), učnice}
\dicEntry[læs] \dicTerm{læs} \dicIPA{{l}{a}{i}{\textlengthmark}{s}} \dicPos{adj}[5]\dicFlx{}[-1] \dicDirectTranslationCS{gramotný, umějící číst}
\dicEntry[læsa] \dicTerm{læs|a} \dicsymFrequent\  \dicIPA{{l}{a}{i}{\textlengthmark}{s}{a}} \dicPos{v}[2] \dicFlx{(‑ti, ‑t)}[63] \dicFlx{dat\,/\addthin acc} \textbf{1.} \dicFlx{dat} \dicSynonym*{loka með lás} \dicDirectTranslationCS{(u)zamknout, (u)zamykat, (u)zavřít, (u)zavírat} \dicExampleIS{Dyrunum var læst.} \dicExampleCS{Dveře byly zamčeny.};  \dicPhraseIS{læsa að sér} \dicDirectTranslationCS{zamknout (za sebou) dveře};  \dicPhraseIS{læsa á eftir sér} \dicDirectTranslationCS{zavřít za sebou (dveře)};  \dicPhraseIS{læsa e‑n inni} \dicDirectTranslationCS{zavřít (koho), dát (koho) pod zámek (do vězení ap.)}  \textbf{2.} \dicFlx{dat} \dicSynonym{innsigla} \dicDirectTranslationCS{zalepit, zalepovat, zapečetit, zapečeťovat (obálku ap.)}  \textbf{3.} \dicFlx{acc} \dicSynonym*{breiðast út} \dicDirectTranslationCS{(roz)šířit se, rozšiřovat se};  \dicPhraseIS{læsa sig um e‑ð} \dicDirectTranslationCS{rozšířit se po (čem)}  \textbf{4.} \dicFlx{dat} \dicDirectTranslationCS{sevřít, svírat (v~drápech\,/\addthin zubech)};  \dicIdiom{læsast}{ \dicPhraseIS{læsast}} \dicFlx{refl} {\textbf{a.}} \dicDirectTranslationCS{zavřít se (klíče v~domě ap.)};  {\textbf{b.}} \dicDirectTranslationCS{sevřít, chopit se}
\dicEntry[læsi] \dicTerm{læsi\smash{\textsuperscript{1}}} \dicIPA{{l}{a}{i}{\textlengthmark}{s}{\textsci}} \dicPos{n}[2] \dicFlx{(‑s)}[20] \dicDirectTranslationCS{gramotnost, dovednost čtení}
\dicEntry[læsi] \dicTerm{læsi\smash{\textsuperscript{2}}} \dicIPA{{l}{a}{i}{\textlengthmark}{s}{\textsci}} \dicPos{v} \dicFlx{con pf sg 1 pers} \dicLink{lesa}
\dicEntry[læsilegur] \dicTerm{læsi··legur} \dicIPA{{l}{a}{i}{\textlengthmark}{s}{\textsci}{l}{\textepsilon}{\textbabygamma}{\textscy}{\textsubring{r}}} \dicPos{adj}[1]\dicFlx{}[-8] \textbf{1.} \dicDirectTranslationCS{čitelný} \dicExampleIS{læsileg skrift} \dicExampleCS{čitelné písmo}  \textbf{2.} \dicDirectTranslationCS{čtivý} \dicExampleIS{læsileg bók} \dicExampleCS{čtivá knížka}
\dicEntry[læsing] \dicTerm{læs··ing} \dicIPA{{l}{a}{i}{\textlengthmark}{s}{i}{\ng}{\r{g}}} \dicPos{f}[4] \dicFlx{(‑ar, ‑ar)}[5] \textbf{1.} \dicSynonym{lás} \dicDirectTranslationCS{zámek, zamykání (u~dveří ap.)} \dicExampleIS{læsing á dyrunum} \dicExampleCS{zámek u~dveří}  \textbf{2.} \dicSynonym{lokun} \dicDirectTranslationCS{(u)zamčení, (u)zamknutí}
\dicEntry[læstur] \dicTerm{læstur} \dicsymFrequent\  \dicIPA{{l}{a}{i}{s}{\textsubring{d}}{\textscy}{\textsubring{r}}} \dicPos{adj}[1]\dicFlx{}[-10] \dicSynonym*{aflokaður} \dicDirectTranslationCS{(u)zamčený, (u)zamknutý} \dicExampleIS{læstar dyr} \dicExampleCS{zamknuté dveře}
\dicEntry[læt] \dicTerm{læt} \dicIPA{{l}{a}{i}{\textlengthmark}{\textsubring{d}}} \dicPos{v} \dicFlx{ind praes sg 1 pers} \dicLink{láta}
\dicEntry[læti] \dicTerm{læti} \dicsymFrequent\  \dicIPA{{l}{a}{i}{\textlengthmark}{\textsubring{d}}{\textsci}} \dicPos{n}[2] \dicFlx{pl}[19] \dicDirectTranslationCS{rozruch, povyk} \dicExampleIS{vera með læti} \dicExampleCS{dělat povyk};  \dicPhraseIS{kunna sér ekki læti} \dicLangCat{přen.} \dicDirectTranslationCS{být štěstím bez sebe}
\dicEntry[lævirki] \dicTerm{læ··virk|i} \dicIPA{{l}{a}{i}{\textlengthmark}{v}{\textsci}{\textsubring{r}}{\r{\textObardotlessj}}{\textsci}} \dicPos{m}[1] \dicFlx{(‑ja, ‑jar)}[14] \dicFieldCat{zool.} \dicDirectTranslationCS{skřivan} \textit{(l.~{\textLA{Alauda}})}  \dicsymPhoto\ 
\dicFigure{ds_image_laevirki_0_2.jpg}{Lævirki}{Lævirki - Daniel Pettersson, CC BY-SA 2.5}
\dicEntry[lævís] \dicTerm{læ··vís} \dicIPA{{l}{a}{i}{\textlengthmark}{v}{i}{s}} \dicPos{adj}[5]\dicFlx{}[-1] \dicSynonym{slóttugur} \dicDirectTranslationCS{záludný, zákeřný}
\dicEntry[löðrungur] \dicTerm{löðr··ung|ur} \dicIPA{{l}{\oe}{ð}{r}{u}{\ng}{\r{g}}{\textscy}{\textsubring{r}}} \dicPos{m}[6] \dicFlx{(‑s, ‑ar)}[8] \dicSynonym{kinnhestur} \dicDirectTranslationCS{facka, políček} \dicExampleIS{gefa e‑m löðrungur} \dicExampleCS{dát (komu) facku}
\dicEntry[löður] \dicTerm{löður} \dicIPA{{l}{\oe}{\textlengthmark}{ð}{\textscy}{\textsubring{r}}} \dicPos{n}[2] \dicFlx{(‑s)}[28] \dicSynonym{froða} \dicDirectTranslationCS{(vodní) pěna}
\dicEntry[lög] \dicTerm{lög} \dicIPA{{l}{\oe}{\textlengthmark}{x}} \dicPos{n}[2] \dicFlx{pl}[9] \dicFieldCat{práv.} \dicDirectTranslationCS{zákon};  \dicPhraseIS{leiða e‑ð í lög} \dicDirectTranslationCS{uzákonit (co)};  \dicPhraseIS{setja lög} \dicDirectTranslationCS{vydat zákon}
\dicEntry[lögaldur] \dicTerm{lög··ald|ur} \dicIPA{{l}{\oe}{\textlengthmark}{\textbabygamma}{a}{l}{\textsubring{d}}{\textscy}{\textsubring{r}}} \dicPos{m}[5] \dicFlx{(‑urs, ‑rar)}[3] \dicFieldCat{práv.} \dicDirectTranslationCS{plnoletost, zletilost, zákonný věk};  \dicPhraseIS{ná lögaldri} \dicDirectTranslationCS{dosáhnout plnoletosti}
\dicEntry[lögbann] \dicTerm{lög··bann} \dicIPA{{l}{\oe}{\textbabygamma}{\textsubring{b}}{a}{\textsubring{n}}} \dicPos{n}[2] \dicFlx{(‑s)}[2] \dicFieldCat{práv.} \dicDirectTranslationCS{soudní zákaz}
\dicEntry[lögbrot] \dicTerm{lög··brot} \dicIPA{{l}{\oe}{\textbabygamma}{\textsubring{b}}{r}{\textopeno}{\textsubring{d}}} \dicPos{n}[2] \dicFlx{(‑s, ‑)}[5] \dicFieldCat{práv.} \dicDirectTranslationCS{porušení zákona, delikt}
\dicEntry[lögfesta] \dicTerm{lög··fest|a} \dicIPA{{l}{\oe}{x}{f}{\textepsilon}{s}{\textsubring{d}}{a}} \dicPos{v}[2] \dicFlx{(‑i, ‑)}[12] \dicFlx{acc} \dicFieldCat{práv.} \dicDirectTranslationCS{uzákonit, legalizovat, kodifikovat}
\dicEntry[lögformlegur] \dicTerm{lög··form·legur} \dicIPA{{l}{\oe}{x}{f}{\textopeno}{r}{m}{l}{\textepsilon}{\textbabygamma}{\textscy}{\textsubring{r}}} \dicPos{adj}[1]\dicFlx{}[-8] \dicFieldCat{práv.} \dicDirectTranslationCS{zákonný}
\dicEntry[lögfræði] \dicTerm{lög··fræð|i} \dicIPA{{l}{\oe}{x}{f}{r}{a}{i}{ð}{\textsci}} \dicPos{f}[3] \dicFlx{(‑i)}[3] \dicDirectTranslationCS{právní věda, práva}
\dicEntry[lögfræðideild] \dicTerm{lög·fræði··deild} \dicIPA{{l}{\oe}{x}{f}{r}{a}{i}{ð}{\textsci}{\textsubring{d}}{ei}{l}{\textsubring{d}}} \dicPos{f}[7] \dicFlx{(‑ar, ‑ir)}[1] \dicFieldCat{škol.} \dicDirectTranslationCS{právnická fakulta}
\dicEntry[lögfræðilegur] \dicTerm{lög·fræði··legur} \dicIPA{{l}{\oe}{x}{f}{r}{a}{i}{ð}{\textsci}{l}{\textepsilon}{\textbabygamma}{\textscy}{\textsubring{r}}} \dicPos{adj}[1]\dicFlx{}[-8] \dicDirectTranslationCS{právní} \dicExampleIS{lögfræðilegur ráðgjafi} \dicExampleCS{právní poradce}
\dicEntry[lögfræðingur] \dicTerm{lög·fræð··ing|ur} \dicsymFrequent\  \dicIPA{{l}{\oe}{x}{f}{r}{a}{i}{ð}{i}{\ng}{\r{g}}{\textscy}{\textsubring{r}}} \dicPos{m}[6] \dicFlx{(‑s, ‑ar)}[8] \dicDirectTranslationCS{právník, právnička} \dicExampleIS{aðstoð lögfræðings} \dicExampleCS{pomoc právníka}
\dicEntry[lögg] \dicTerm{lögg} \dicIPA{{l}{\oe}{\r{g}}{\textlengthmark}} \dicPos{f}[7] \dicFlx{(laggar\,/\addthin ‑var, laggir\,/\addthin ‑ir\,/\addthin ‑var)}[30] \dicDirectTranslationCS{kapka, doušek (v~láhvi ap.)} \dicExampleIS{kaffilögg} \dicExampleCS{doušek kávy};  \dicPhraseIS{setja e‑ð á laggirnar} \dicDirectTranslationCS{založit (co), zřídit (co)} \dicExampleIS{setja fyrirtæki á laggirnar} \dicExampleCS{zřídit společnost};  \dicPhraseIS{koma e‑u á laggirnar} \dicDirectTranslationCS{založit (co), zřídit (co)}
\dicEntry[lögga] \dicTerm{lögg|a} \dicsymFrequent\  \dicIPA{{l}{\oe}{\r{g}}{\textlengthmark}{a}} \dicPos{f}[1] \dicFlx{(‑u, ‑ur)}[7] \dicLangCat{hovor.} \dicSynonym{lögreglumaður} \dicDirectTranslationCS{polda, policajt(ka)} \dicExampleIS{löggur í kvikmyndum} \dicExampleCS{poldové ve filmech}
\dicEntry[löggilda] \dicTerm{lög··gil|da} \dicIPA{{l}{\oe}{\textbabygamma}{\r{\textObardotlessj}}{\textsci}{l}{\textsubring{d}}{a}} \dicPos{v}[2] \dicFlx{(‑ti, ‑t)}[37] \dicFlx{acc} \dicFieldCat{práv.} \dicDirectTranslationCS{oprávnit, zmocnit, autorizovat}
\dicEntry[löggilding] \dicTerm{lög··gild·ing} \dicIPA{{l}{\oe}{\textbabygamma}{\r{\textObardotlessj}}{\textsci}{l}{\textsubring{d}}{i}{\ng}{\r{g}}} \dicPos{f}[4] \dicFlx{(‑ar, ‑ar)}[5] \dicFieldCat{práv.} \dicDirectTranslationCS{oprávnění, zmocnění, autorizace}
\dicEntry[löggiltur] \dicTerm{lög··giltur} \dicIPA{{l}{\oe}{\textbabygamma}{\r{\textObardotlessj}}{\textsci}{\textsubring{l}}{\textsubring{d}}{\textscy}{\textsubring{r}}} \dicPos{adj}[1]\dicFlx{}[-13] \dicFieldCat{práv.} \dicDirectTranslationCS{oprávněný, autorizovaný};  \dicPhraseIS{löggiltur skjalaþýðandi} \dicDirectTranslationCS{úřední překladatel(ka)}
\dicEntry[löggjafarvald] \dicTerm{lög·gjafar··vald} \dicIPA{{l}{\oe}{\textbabygamma}{\r{\textObardotlessj}}{a}{v}{a}{r}{v}{a}{l}{\textsubring{d}}} \dicPos{n}[2] \dicFlx{(‑s)}[2] \dicDirectTranslationCS{zákonodárná moc}
\dicEntry[löggjafarþing] \dicTerm{lög·gjafar··þing} \dicIPA{{l}{\oe}{\textbabygamma}{\r{\textObardotlessj}}{a}{f}{a}{\textsubring{r}}{\texttheta}{i}{\ng}{\r{g}}} \dicPos{n}[2] \dicFlx{(‑s, ‑)}[5] \dicFieldCat{práv.} \dicDirectTranslationCS{zákonodárný sbor}
\dicEntry[löggjafi] \dicTerm{lög··gjaf|i} \dicIPA{{l}{\oe}{\textbabygamma}{\r{\textObardotlessj}}{a}{v}{\textsci}} \dicPos{m}[1] \dicFlx{(‑a, ‑ar)}[8] \dicFieldCat{pol.} \dicDirectTranslationCS{zákonodárce, zákonodárkyně}
\dicEntry[löggjöf] \dicTerm{lög··|gjöf} \dicIPA{{l}{\oe}{\textbabygamma}{\r{\textObardotlessj}}{\oe}{f}} \dicPos{f}[7] \dicFlx{(‑gjafar)}[19] \dicFieldCat{práv.} \dicDirectTranslationCS{legislativa, zákonodárství}
\dicEntry[löggæsla] \dicTerm{lög··gæsl|a} \dicIPA{{l}{\oe}{\textbabygamma}{\r{\textObardotlessj}}{a}{i}{s}{\textsubring{d}}{l}{a}} \dicPos{f}[1] \dicFlx{(‑u)}[5] \dicDirectTranslationCS{prosazování práva}
\dicEntry[löghald] \dicTerm{lög··hald} \dicIPA{{l}{\oe}{\textlengthmark}{x}{h}{a}{l}{\textsubring{d}}} \dicPos{n}[2] \dicFlx{(‑s)}[2] \dicFieldCat{práv.} \dicDirectTranslationCS{(úřední) zabavení, konfiskace};  \dicPhraseIS{leggja löghald á e‑u} \dicDirectTranslationCS{zkonfiskovat (co)}
\dicEntry[lögheimili] \dicTerm{lög··heimili} \dicIPA{{l}{\oe}{\textlengthmark}{x}{h}{ei}{m}{\textsci}{l}{\textsci}} \dicPos{n}[2] \dicFlx{(‑s, ‑)}[14] \dicFieldCat{práv.} \dicDirectTranslationCS{trvalé bydliště}
\dicEntry[löghlýðinn] \dicTerm{lög··hlýðinn} \dicIPA{{l}{\oe}{x}{\textsubring{l}}{i}{ð}{\textsci}{\textsubring{n}}} \dicPos{adj}[6]\dicFlx{}[-2] \dicDirectTranslationCS{dodržující zákony}
\dicEntry[löghyggja] \dicTerm{lög··hyggj|a} \dicIPA{{l}{\oe}{\textlengthmark}{x}{h}{\textsci}{\r{\textObardotlessj}}{a}} \dicPos{f}[1] \dicFlx{(‑u)}[5] \dicFieldCat{filos.} \dicDirectTranslationCS{determinismus}
\dicEntry[löglegur] \dicTerm{lög··legur} \dicIPA{{l}{\oe}{\textbabygamma}{l}{\textepsilon}{\textbabygamma}{\textscy}{\textsubring{r}}} \dicPos{adj}[1]\dicFlx{}[-8] \dicDirectTranslationCS{legální, právoplatný}
\dicEntry[lögmaður] \dicTerm{lög··|maður} \dicsymFrequent\  \dicIPA{{l}{\oe}{\textbabygamma}{m}{a}{ð}{\textscy}{\textsubring{r}}} \dicPos{m}[13] \dicFlx{(‑manns, ‑menn)}[2] \dicDirectTranslationCS{advokát(ka)} \dicExampleIS{Vantar þig lögmann?} \dicExampleCS{Potřebuješ advokáta?}
\dicEntry[lögmál] \dicTerm{lög··mál} \dicIPA{{l}{\oe}{\textbabygamma}{m}{au}{\textsubring{l}}} \dicPos{n}[2] \dicFlx{(‑s, ‑)}[5] \textbf{1.} \dicSynonym{lög} \dicDirectTranslationCS{zákon}  \textbf{2.} \dicDirectTranslationCS{zákon, řád (přírody ap.)} \dicExampleIS{lögmál lífsins} \dicExampleCS{zákon života}  \textbf{3.} \dicFieldCat{filos.} \dicSynonym{grundvallarregla} \dicDirectTranslationCS{princip, zákonitost}
\dicEntry[lögmæti] \dicTerm{lög··mæti} \dicIPA{{l}{\oe}{\textbabygamma}{m}{a}{i}{\textsubring{d}}{\textsci}} \dicPos{n}[2] \dicFlx{(‑s)}[20] \dicDirectTranslationCS{legálnost, zákonnost}
\dicEntry[lögmætur] \dicTerm{lög··mætur} \dicIPA{{l}{\oe}{\textbabygamma}{m}{a}{i}{\textsubring{d}}{\textscy}{\textsubring{r}}} \dicPos{adj}[1]\dicFlx{}[-1] \dicSynonym{gildur} \dicDirectTranslationCS{legální, zákonný}
\dicEntry[lögn] \dicTerm{lögn} \dicIPA{{l}{\oe}{\r{g}}{\textsubring{n}}} \dicPos{f}[7] \dicFlx{(lagnar, lagnir)}[16] \textbf{1.} \dicSynonym{leiðsla} \dicDirectTranslationCS{vedení, instalace} \dicExampleIS{raflögn} \dicExampleCS{elektrické vedení}  \textbf{2.} \dicSynonym*{það að leggja net} \dicDirectTranslationCS{pokládání sítě}
\dicEntry[lögregla] \dicTerm{lög··regl|a} \dicsymFrequent\  \dicIPA{{l}{\oe}{\textbabygamma}{r}{\textepsilon}{\r{g}}{l}{a}} \dicPos{f}[1] \dicFlx{(‑u, ‑ur)}[19] \textbf{1.} \dicSynonym*{stofnun til löggæslu} \dicDirectTranslationCS{policie} \dicExampleIS{Lögreglan leitar að vitnum að árekstri.} \dicExampleCS{Policie hledá svědky srážky vozidel.}  \textbf{2.} \dicSynonym{lögreglumaður} \dicDirectTranslationCS{policista, policistka}
\dicEntry[lögreglubíll] \dicTerm{lög·reglu··bíl|l} \dicsymFrequent\  \dicIPA{{l}{\oe}{\textbabygamma}{r}{\textepsilon}{\r{g}}{l}{\textscy}{\textsubring{b}}{i}{\textsubring{d}}{\textsubring{l}}} \dicPos{m}[6] \dicFlx{(‑s, ‑ar)}[48] \dicDirectTranslationCS{policejní auto\,/\addthin vůz} \dicExampleIS{Lögreglubílar lentu í árekstri.} \dicExampleCS{Policejní auta se srazila.}
\dicEntry[lögreglumaður] \dicTerm{lög·reglu··|maður} \dicsymFrequent\  \dicIPA{{l}{\oe}{\textbabygamma}{r}{\textepsilon}{\r{g}}{l}{\textscy}{m}{a}{ð}{\textscy}{\textsubring{r}}} \dicPos{m}[13] \dicFlx{(‑manns, ‑menn)}[2] \dicDirectTranslationCS{policista, policistka} \dicExampleIS{Lögreglumenn réðust inn í húsið.} \dicExampleCS{Policisté vtrhli do domu.}
\dicEntry[lögreglusamþykkt] \dicTerm{lög·reglu··sam·þykkt} \dicIPA{{l}{\oe}{\textbabygamma}{r}{\textepsilon}{\r{g}}{l}{\textscy}{s}{a}{m}{\texttheta}{\textsci}{x}{\textsubring{d}}} \dicPos{f}[7] \dicFlx{(‑ar, ‑ir)}[1] \dicDirectTranslationCS{policejní nařízení}
\dicEntry[lögreglustjóri] \dicTerm{lög·reglu··stjór|i} \dicIPA{{l}{\oe}{\textbabygamma}{r}{\textepsilon}{\r{g}}{l}{\textscy}{s}{\textsubring{d}}{j}{ou}{r}{\textsci}} \dicPos{m}[1] \dicFlx{(‑a, ‑ar)}[1] \dicDirectTranslationCS{policejní prezident(ka)}
\dicEntry[lögreglustöð] \dicTerm{lög·reglu··stöð} \dicIPA{{l}{\oe}{\textbabygamma}{r}{\textepsilon}{\r{g}}{l}{\textscy}{s}{\textsubring{d}}{\oe}{\texttheta}} \dicPos{f}[6] \dicFlx{(‑var, ‑var)}[1] \dicDirectTranslationCS{policejní stanice}
\dicEntry[lögregluvarðstjóri] \dicTerm{lög·reglu··varð·stjór|i} \dicIPA{{l}\-{\oe}\-{\textbabygamma}\-{r}\-{\textepsilon}\-{\r{g}}\-{l}\-{\textscy}\-{v}\-{a}\-{r}\-{ð}\-{s}\-{\textsubring{d}}\-{j}\-{ou}\-{r}\-{\textsci}\-} \dicPos{m}[1] \dicFlx{(‑a, ‑ar)}[1] \dicDirectTranslationCS{policejní komisař(ka)}
\dicEntry[lögregluvarðstofa] \dicTerm{lög·reglu··varð·stof|a} \dicIPA{{l}\-{\oe}\-{\textbabygamma}\-{r}\-{\textepsilon}\-{\r{g}}\-{l}\-{\textscy}\-{v}\-{a}\-{r}\-{ð}\-{s}\-{\textsubring{d}}\-{\textopeno}\-{v}\-{a}\-} \dicPos{f}[1] \dicFlx{(‑u, ‑ur)}[7] \dicDirectTranslationCS{okrsková služebna, policejní stanice}
\dicEntry[lögregluþjónn] \dicTerm{lög·reglu··þjón|n} \dicsymFrequent\  \dicIPA{{l}{\oe}{\textbabygamma}{r}{\textepsilon}{\r{g}}{l}{\textscy}{\texttheta}{j}{ou}{\textsubring{d}}{\textsubring{n}}} \dicPos{m}[6] \dicFlx{(‑s, ‑ar)}[42] \dicDirectTranslationCS{policista, policistka} \dicExampleIS{lögregluþjónar á næturvakt} \dicExampleCS{policisté na noční službě}
\dicEntry[Lögrétta] \dicTerm{Lög··rétt|a} \dicIPA{{l}{\oe}{\textbabygamma}{r}{j}{\textepsilon}{h}{\textsubring{d}}{a}} \dicPos{f}[1] \dicFlx{(‑u)}[6] \dicFieldCat{geog., hist.} \dicDirectTranslationCS{Lögrétta} \dicIndirectTranslationCS{(zákonodárný a~soudní sněm nacházející se v~Altingu)}
\dicEntry[lögræði] \dicTerm{lög··ræði} \dicIPA{{l}{\oe}{\textbabygamma}{r}{a}{i}{ð}{\textsci}} \dicPos{n}[2] \dicFlx{(‑s)}[20] \dicFieldCat{práv.} \dicDirectTranslationCS{svéprávnost};  \dicPhraseIS{svipta e‑n lögræði} \dicDirectTranslationCS{zbavit (koho) svéprávnosti}
\dicEntry[lögsótt] \dicTerm{lög··sótt} \dicIPA{{l}{\oe}{x}{s}{ou}{h}{\textsubring{d}}} \dicPos{v} \dicFlx{supin} \dicLink{lögsækja}
\dicEntry[lögsótti] \dicTerm{lög··sótti} \dicIPA{{l}{\oe}{x}{s}{ou}{h}{\textsubring{d}}{\textsci}} \dicPos{v} \dicFlx{ind pf sg 1 pers} \dicLink{lögsækja}
\dicEntry[lögsóttum] \dicTerm{lög··sóttum} \dicIPA{{l}{\oe}{x}{s}{ou}{h}{\textsubring{d}}{\textscy}{\textsubring{m}}} \dicPos{v} \dicFlx{ind pf pl 1 pers} \dicLink{lögsækja}
\dicEntry[lögsæki] \dicTerm{lög··sæki} \dicIPA{{l}{\oe}{x}{s}{a}{i}{\r{\textObardotlessj}}{\textsci}} \dicPos{v} \dicFlx{ind praes sg 1 pers} \dicLink{lögsækja}
\dicEntry[lögsækja] \dicTerm{lög··|sækja} \dicIPA{{l}{\oe}{x}{s}{a}{i}{\r{\textObardotlessj}}{a}} \dicPos{v}[5] \dicFlx{(‑sæki, ‑sótti, ‑sóttum, ‑sækti, ‑sótt)}[7] \dicFlx{acc} \dicFieldCat{práv.} \dicDirectTranslationCS{žalovat, podat\,/\addthin podávat žalobu} \dicExampleIS{lögsækja e‑n fyrir e‑ð} \dicExampleCS{žalovat (koho) za (co)}
\dicEntry[lögsækti] \dicTerm{lög··sækti} \dicIPA{{l}{\oe}{x}{s}{a}{i}{x}{\textsubring{d}}{\textsci}} \dicPos{v} \dicFlx{con pf sg 1 pers} \dicLink{lögsækja}
\dicEntry[lögtak] \dicTerm{lög··|tak} \dicIPA{{l}{\oe}{x}{t\smash{\textsuperscript{h}}}{a}{\r{g}}} \dicPos{n}[2] \dicFlx{(‑taks, ‑tök)}[8] \dicFieldCat{práv.} \dicDirectTranslationCS{exekuce}
\dicEntry[lögun] \dicTerm{lögun} \dicsymFrequent\  \dicIPA{{l}{\oe}{\textlengthmark}{\textbabygamma}{\textscy}{\textsubring{n}}} \dicPos{f}[7] \dicFlx{(‑ar, laganir)}[16] \textbf{1.} \dicSynonym{form} \dicDirectTranslationCS{tvar, podoba} \dicExampleIS{breyta löguninni} \dicExampleCS{změnit podobu}  \textbf{2.} \dicSynonym*{kaffilögun o.þ.h.} \dicDirectTranslationCS{příprava (kávy ap.)} \dicExampleIS{lögun á kaffi} \dicExampleCS{příprava kávy}
\dicEntry[lögur] \dicTerm{lögur} \dicIPA{{l}{\oe}{\textlengthmark}{\textbabygamma}{\textscy}{\textsubring{r}}} \dicPos{m}[11] \dicFlx{(lagar, legir)}[5] \textbf{1.} \dicDirectTranslationCS{kapalina, tekutina}  \textbf{2.} \dicSynonym{vatn} \dicDirectTranslationCS{voda, vodní plocha} \dicExampleIS{á láði og legi} \dicExampleCS{na souši a~na vodě}
\dicEntry[lök] \dicTerm{lök\smash{\textsuperscript{1}}} \dicIPA{{l}{\oe}{\textlengthmark}{\r{g}}} \dicPos{n} \dicFlx{pl nom} \dicLink{lak\smash{\textsuperscript{1}}}
\dicEntry[lök] \dicTerm{lök\smash{\textsuperscript{2}}} \dicIPA{{l}{\oe}{\textlengthmark}{\r{g}}} \dicPos{adj} \dicFlx{f sg nom pos} \dicLink{lakur}
\dicEntry[lökk] \dicTerm{lökk} \dicIPA{{l}{\oe}{h}{\r{g}}} \dicPos{n} \dicFlx{pl nom} \dicLink{lakk}
\dicEntry[lömb] \dicTerm{lömb} \dicIPA{{l}{\oe}{m}{\textsubring{b}}} \dicPos{n} \dicFlx{pl nom} \dicLink{lamb}
\dicEntry[lömdum] \dicTerm{lömdum} \dicIPA{{l}{\oe}{m}{\textsubring{d}}{\textscy}{\textsubring{m}}} \dicPos{v} \dicFlx{ind pf pl 1 pers} \dicLink{lemja}
\dicEntry[lömuð] \dicTerm{lömuð} \dicIPA{{l}{\oe}{\textlengthmark}{m}{\textscy}{\texttheta}} \dicPos{adj} \dicFlx{f sg nom pos} \dicLink{lamaður}
\dicEntry[lömun] \dicTerm{löm|un} \dicIPA{{l}{\oe}{\textlengthmark}{m}{\textscy}{\textsubring{n}}} \dicPos{f}[7] \dicFlx{(‑unar)}[12] \dicFieldCat{med.} \dicDirectTranslationCS{ochrnutí, paralýza} \dicExampleIS{fá lömun í fótinn} \dicExampleCS{ochrnout na nohu}
\dicEntry[lömunarveiki] \dicTerm{lömunar··veik|i} \dicIPA{{l}{\oe}{\textlengthmark}{m}{\textscy}{n}{a}{r}{v}{ei}{\r{\textObardotlessj}}{\textsci}} \dicPos{f}[3] \dicFlx{(‑i)}[3] \dicFieldCat{med.} \dicDirectTranslationCS{obrna}
\dicEntry[lönd] \dicTerm{lönd} \dicIPA{{l}{\oe}{n}{\textsubring{d}}} \dicPos{n} \dicFlx{pl nom} \dicLink{land}
\dicEntry[löndun] \dicTerm{löndun} \dicIPA{{l}{\oe}{n}{\textsubring{d}}{\textscy}{\textsubring{n}}} \dicPos{f}[7] \dicFlx{(‑ar, landanir)}[16] \dicSynonym*{uppskipun} \dicDirectTranslationCS{vykládka, vylodění}
\dicEntry[löng] \dicTerm{löng} \dicIPA{{l}{\oe i}{\ng}{\r{g}}} \dicPos{adj} \dicFlx{f sg nom pos} \dicLink{langur\smash{\textsuperscript{2}}}
\dicEntry[löngu] \dicTerm{löngu\smash{\textsuperscript{1}}} \dicIPA{{l}{\oe i}{\ng}{\r{g}}{\textscy}} \dicPos{f} \dicFlx{sg gen} \dicLink{langa\smash{\textsuperscript{1}}}
\dicEntry[löngu] \dicTerm{löngu\smash{\textsuperscript{2}}} \dicIPA{{l}{\oe i}{\ng}{\r{g}}{\textscy}} \dicPos{adv} \dicDirectTranslationCS{dlouho, dávno};  \dicPhraseIS{endur fyrir löngu} \dicFlx{adv} \dicDirectTranslationCS{dávno, před dávnými dobami};  \dicPhraseIS{fyrir (langa\,/\addthin margt) löngu} \dicFlx{adv} \dicDirectTranslationCS{dávno, před dávnou dobou};  \dicPhraseIS{fyrir lifandi löngu} \dicFlx{adv} \dicDirectTranslationCS{už\,/\addthin již (velmi) dávno};  \dicPhraseIS{löngu áður} \dicFlx{adv} \dicDirectTranslationCS{dlouho předtím};  \dicPhraseIS{löngu fyrr} \dicFlx{adv} \dicDirectTranslationCS{mnohem dříve, dlouho předtím};  \dicPhraseIS{löngu síðar\,/\addthin seinna} \dicFlx{adv} \dicDirectTranslationCS{dlouho poté, o~dost později}
\dicEntry[löngum] \dicTerm{löngum} \dicsymFrequent\  \dicIPA{{l}{\oe i}{\ng}{\r{g}}{\textscy}{\textsubring{m}}} \dicPos{adv} \dicDirectTranslationCS{dlouho} \dicExampleIS{Það hefur löngum verið vitað að sólböð eru hættuleg.} \dicExampleCS{Už se dlouho ví, že opalování je nebezpečné.}
\dicEntry[löngun] \dicTerm{löngun} \dicsymFrequent\  \dicIPA{{l}{\oe i}{\ng}{\r{g}}{\textscy}{\textsubring{n}}} \dicPos{f}[7] \dicFlx{(‑ar, langanir)}[16] \dicDirectTranslationCS{touha, toužení, tužba} \dicExampleIS{löngun í nýjan bíl} \dicExampleCS{touha po novém autě}
\dicEntry[löngutangar] \dicTerm{löngu··tangar} \dicIPA{{l}{\oe i}{\ng}{\r{g}}{\textscy}{t\smash{\textsuperscript{h}}}{au}{\ng}{\r{g}}{a}{\textsubring{r}}} \dicPos{f} \dicFlx{sg gen} \dicLink{langatöng}
\dicEntry[löngutengur] \dicTerm{löngu··tengur} \dicIPA{{l}{\oe i}{\ng}{\r{g}}{\textscy}{t\smash{\textsuperscript{h}}}{\textepsilon}{i}{\ng}{\r{g}}{\textscy}{\textsubring{r}}} \dicPos{f} \dicFlx{pl nom} \dicLink{langatöng}
\dicEntry[löngutöng] \dicTerm{löngu··|töng} \dicIPA{{l}{\oe i}{\ng}{\r{g}}{\textscy}{t\smash{\textsuperscript{h}}}{\oe i}{\ng}{\r{g}}} \dicPos{f}[8] \dicFlx{(‑tangar, ‑tengur\,/\addthin ‑tangir)}[9] \dicLink{langatöng}
\dicEntry[löpp] \dicTerm{löpp} \dicsymFrequent\  \dicIPA{{l}{\oe}{h}{\textsubring{b}}} \dicPos{f}[7] \dicFlx{(lappar, lappir)}[16] \dicDirectTranslationCS{noha, chodidlo} \dicExampleIS{vöðvar í löppum} \dicExampleCS{svaly na nohou};  \dicPhraseIS{koma sér á lappir} \dicDirectTranslationCS{vstát z~postele}
\dicEntry[löptum] \dicTerm{löptum} \dicIPA{{l}{\oe}{f}{\textsubring{d}}{\textscy}{\textsubring{m}}} \dicPos{v} \dicFlx{ind pf pl 1 pers} \dicLink{lepja}
\dicEntry[löstur] \dicTerm{löstur} \dicIPA{{l}{\oe}{s}{\textsubring{d}}{\textscy}{\textsubring{r}}} \dicPos{m}[11] \dicFlx{(lastar, lestir)}[5] \dicDirectTranslationCS{zlozvyk, neřest} \dicExampleIS{drykkjuskapur og aðrir lestir} \dicExampleCS{opilství a~jiné neřesti}
\dicEntry[löt] \dicTerm{löt} \dicIPA{{l}{\oe}{\textlengthmark}{\textsubring{d}}} \dicPos{adj} \dicFlx{f sg nom pos} \dicLink{latur}
\dicEntry[lötra] \dicTerm{lötr|a} \dicIPA{{l}{\oe}{\textlengthmark}{\textsubring{d}}{r}{a}} \dicPos{v}[1] \dicFlx{(‑aði)}[44] \dicDirectTranslationCS{loudat se, šinout se, courat se} \dicExampleIS{Hann lötraði á eftir hópnum.} \dicExampleCS{Coural se za skupinou.}
\dicEntry[löttum] \dicTerm{löttum} \dicIPA{{l}{\oe}{h}{\textsubring{d}}{\textscy}{\textsubring{m}}} \dicPos{v} \dicFlx{ind pf pl 1 pers} \dicLink{letja}
}

\input{letters}
%\input{letters_problematic}

\restoregeometry
\pagestyle{plain}

\fi

% ================================================================ PHONETICS =

\ifinputphon

\cleardoublepage

\chapter{Seznam islandských fonémů}              \label{sec:phon_phonems}

\begin{table}[h]
\begin{tabular}{lllll} \toprule

\textbf{foném} & \textbf{vysvětlivka} & \textbf{příklad} & \textbf{fonetický přepis} & \textbf{překlad} \\
\midrule
{\textipa{[{a}]}} & otevřená přední nezaokrouhlená samohl. & rata & {\textipa{[{r}{a}{\textlengthmark}{\textsubring{d}}{a}]}} & trefit \\ 
{\textipa{[{\textepsilon}]}} & polootevřená přední nezaokrouhlená samohl. & meta & {\textipa{[{m}{\textepsilon}{\textlengthmark}{\textsubring{d}}{a}]}} & ocenit \\ 
{\textipa{[{\textsci}]}} & téměř zavřená téměř přední nezaokrouhlená samohl. & vinna & {\textipa{[{v}{\textsci}{n}{\textlengthmark}{a}]}} & pracovat \\ 
{\textipa{[{i}]}} & zavřená přední nezaokrouhlená samohl. & fínn & {\textipa{[{f}{i}{\textsubring{d}}{\textsubring{n}}]}} & hezký \\ 
{\textipa{[{\textopeno}]}} & polootevřená zadní zaokrouhlená samohl. & lofa & {\textipa{[{l}{\textopeno}{\textlengthmark}{v}{a}]}} & slibovat \\ 
{\textipa{[{ou}]}} & dvojhláska & bóndi & {\textipa{[{\textsubring{b}}{ou}{n}{\textsubring{d}}{\textsci}]}} & farmář \\ 
{\textipa{[{\textscy}]}} & téměř zavřená téměř přední zaokrouhlená samohl. & hundur & {\textipa{[{h}{\textscy}{n}{\textsubring{d}}{\textscy}{\textsubring{r}}]}} & pes \\ 
{\textipa{[{u}]}} & zavřená zadní zaokrouhlená samohl. & súkkulaði & {\textipa{[{s}{u}{h}{\r{g}}{\textscy}{l}{a}{ð}{\textsci}]}} & čokoláda \\ 
{\textipa{[{\oe}]}} & téměř otevřená přední nezaokrouhlená samohl. & lönd & {\textipa{[{l}{\oe}{n}{\textsubring{d}}]}} & země \\ 
{\textipa{[{ei}]}} & dvojhláska & heiður & {\textipa{[{h}{ei}{\textlengthmark}{ð}{\textscy}{\textsubring{r}}]}} & čest \\ 
{\textipa{[{\textsubring{b}}]}} & neznělá bilabiální ploziva & bara & {\textipa{[{\textsubring{b}}{a}{\textlengthmark}{r}{a}]}} & jenom \\ 
{\textipa{[{p\textsuperscript{h}}]}} & neznělá bilabiální ploziva s přídechem & prestur & {\textipa{[{p\textsuperscript{h}}{r}{\textepsilon}{s}{\textsubring{d}}{\textscy}{\textsubring{r}}]}} & kněz \\ 
{\textipa{[{\textsubring{d}}]}} & neznělá alveolární ploziva & dæmi & {\textipa{[{\textsubring{d}}{a}{i}{\textlengthmark}{m}{\textsci}]}} & příklad \\ 
{\textipa{[{t\textsuperscript{h}}]}} & neznělá alveolární ploziva s přídechem & tómur & {\textipa{[{t\textsuperscript{h}}{ou}{\textlengthmark}{m}{\textscy}{\textsubring{r}}]}} & prázdný \\ 
{\textipa{[{\r{g}}]}} & neznělá velární ploziva & gata & {\textipa{[{\r{g}}{a}{\textlengthmark}{\textsubring{d}}{a}]}} & ulice \\ 
{\textipa{[{k\textsuperscript{h}}]}} & neznělá velární ploziva s přídechem & koma & {\textipa{[{k\textsuperscript{h}}{\textopeno}{\textlengthmark}{m}{a}]}} & přijít \\
{\textipa{[{\r{\textObardotlessj}}]}} & neznělá palatální ploziva & gefa & {\textipa{[{\r{\textObardotlessj}}{\textepsilon}{\textlengthmark}{v}{a}]}} & dát \\ 
{\textipa{[{c\textsuperscript{h}}]}} & neznělá palatální ploziva s přídechem & kenna & {\textipa{[{c\textsuperscript{h}}{\textepsilon}{n}{\textlengthmark}{a}]}} & učit \\
{\textipa{[{f}]}} & neznělá labiodentální frikativa & finna & {\textipa{[{f}{\textsci}{n}{\textlengthmark}{a}]}} & najít \\ 
{\textipa{[{v}]}} & znělá labiodentální frikativa & hafa & {\textipa{[{h}{a}{\textlengthmark}{v}{a}]}} & mít \\ 
{\textipa{[{ð}]}} & znělá dentální frikativa & borða & {\textipa{[{\textsubring{b}}{\textopeno}{r}{ð}{a}]}} & jíst \\ 
{\textipa{[{\texttheta}]}} & neznělá dentální frikativa & þola & {\textipa{[{\texttheta}{\textopeno}{\textlengthmark}{l}{a}]}} & vydržet \\ 
{\textipa{[{\textbabygamma}]}} & znělá velární frikativa & sagði & {\textipa{[{s}{a}{\textbabygamma}{ð}{\textsci}]}} & řekl \\ 
{\textipa{[{x}]}} & neznělá velární frikativa & lag & {\textipa{[{l}{a}{\textlengthmark}{x}]}} & píseň \\
{\textipa{[{h}]}} & neznělá glotální frikativa & heima & {\textipa{[{h}{ei}{\textlengthmark}{m}{a}]}} & doma \\ 
{\textipa{[{\c{c}}]}} & neznělá palatální frikativa & hjóla & {\textipa{[{\c{c}}{ou}{\textlengthmark}{l}{a}]}} & jet na kole \\ 
{\textipa{[{s}]}} & neznělá alveolární frikativa & systir & {\textipa{[{s}{\textsci}{s}{\textsubring{d}}{\textsci}{\textsubring{r}}]}} & sestra \\ 
{\textipa{[{r}]}} & alveolární vibranta & reynsla & {\textipa{[{r}{ei}{n}{s}{\textsubring{d}}{l}{a}]}} & zkušenost \\ 
{\textipa{[{\textsubring{r}}]}} & neznělá alveolární vibranta & hraun & {\textipa{[{\textsubring{r}}{\oe i}{\textlengthmark}{\textsubring{n}}]}} & láva \\ 
{\textipa{[{l}]}} & alveolární laterální aproximanta & lofa & {\textipa{[{l}{\textopeno}{\textlengthmark}{v}{a}]}} & slíbit \\ 
{\textipa{[{\textsubring{l}}]}} & neznělá alveolární laterální aproximanta & hlutur & {\textipa{[{\textsubring{l}}{\textscy}{\textlengthmark}{\textsubring{d}}{\textscy}{\textsubring{r}}]}} & věc \\  
{\textipa{[{j}]}} & palatální aproximanta & jól & {\textipa{[{j}{ou}{\textlengthmark}{\textsubring{l}}]}} & Vánoce \\ 
{\textipa{[{m}]}} & bilabiální nazála & mýri & {\textipa{[{m}{i}{\textlengthmark}{r}{\textsci}]}} & slatina \\ 
{\textipa{[{\textsubring{m}}]}} & neznělá bilabiální nazála & þyrmt & {\textipa{[{\texttheta}{\textsci}{\textsubring{r}}{\textsubring{m}}{\textsubring{d}}]}} & {\textit{$\shortrightarrow$ þyrma}} \\  
{\textipa{[{n}]}} & alveolární nazála & nemandi & {\textipa{[{n}{\textepsilon}{\textlengthmark}{m}{a}{n}{\textsubring{d}}{\textsci}]}} & žák \\ 
{\textipa{[{\textsubring{n}}]}} & neznělá alveolární nazála & mennt & {\textipa{[{m}{\textepsilon}{\textsubring{n}}{\textsubring{d}}]}} & vzdělání \\ 
{\textipa{[{\textltailn}]}} & palatální nazála & angi & {\textipa{[{au}{\textltailn}{\r{\textObardotlessj}}{\textsci}]}} & drobek \\ 
{\textipa{[{\r{\textltailn}}]}} & neznělá palatální nazála & banki & {\textipa{[{\textsubring{b}}{au}{\r{\textltailn}}{\r{\textObardotlessj}}{\textsci}]}} & banka \\
{\textipa{[{\ng}]}} & velární nazála & skyggndur & {\textipa{[{s}{\r{\textObardotlessj}}{\textsci}{\ng}{\textsubring{d}}{\textscy}{\textsubring{r}}]}} & {\textit{$\shortrightarrow$ skyggna}} \\ 
{\textipa{[{\r{\ng}}]}} & neznělá velární nazála & skyggnt & {\textipa{[{s}{\r{\textObardotlessj}}{\textsci}{\r{\ng}}{\textsubring{d}}]}} & {\textit{$\shortrightarrow$ skyggna}} \\ 
\bottomrule
\end{tabular}
\end{table}

\clearpage

%\section{Výslovnost islandských samohlásek}      \label{sec:phon_vowels}
%\LTXtable{\columnwidth}{phon/vowels}
%\clearpage

%\section{Výslovnost islandských souhlásek}       \label{sec:phon_consonants}
%\LTXtable{\columnwidth}{phon/consonants}
%\clearpage
\fi

% =============================================================== MORPHOLOGY =

\ifinputmorpho

\cleardoublepage

\chapter{Morfologický klíč}                      \label{sec:morpho}

\section{Podstatné jméno, rod mužský}            \label{sec:morpho_m}
{\small\LTXfw{morpho/m}}
\clearpage

\section{Podstatné jméno, rod ženský}            \label{sec:morpho_f}
{\small\LTXfw{morpho/f}}
\clearpage

\section{Podstatné jméno, rod střední}           \label{sec:morpho_n}
{\small\LTXfw{morpho/n}}
\clearpage

\section{Přídavné jméno}                         \label{sec:morpho_adj}
{\small\LTXfw{morpho/adj}}
\clearpage

\section{Zájmeno}                                \label{sec:morpho_pron}
{\small\LTXfw{morpho/pron}}
 \begin{center}
\begin{minipage}[t]{.45\textwidth}
\begin{tabular}{l>{\footnotesize\itshape}l>{\small}l>{\small}l}
 {\textbf{\textit{třída}}} & {\textit{pád}}   & \textit{sg} & \textit{pl}  \\ 
 \hline
\multirow{3}{*}{{{\textbf{pron} \Large{\textbf{17}}}}}  &  nom & ég & við   \\*
 & acc &  mig  & okkur  \\*
 & dat & mér & okkur   \\*
 & gen & mín  & okkar  \\
\hline

\multirow{3}{*}{{{\textbf{pron} \Large{\textbf{17}}}}}  &  nom & hann & þeir   \\*
 & acc &  hann  & þá  \\*
 & dat & honum & þeim   \\*
 & gen & hans  & þeirra  \\
\hline

\multirow{3}{*}{{{\textbf{pron} \Large{\textbf{17}}}}}  &  nom & hún & þær   \\*
 & acc &  hana  & þær  \\*
 & dat & henni & þeim   \\*
 & gen & hennar  & þeirra  \\
\hline

\multirow{3}{*}{{{\textbf{pron} \Large{\textbf{17}}}}}  &  nom &  & vér   \\*
 & acc &    & oss  \\*
 & dat &  & oss   \\*
 & gen &   & vor  \\
\hline

\multirow{3}{*}{{{\textbf{pron} \Large{\textbf{17}}}}}  &  nom &  & þér   \\*
 & acc &    & yður  \\*
 & dat &  & yður   \\*
 & gen &   & yðar  \\
\hline

\multirow{3}{*}{{{\textbf{pron} \Large{\textbf{17}}}}}  &  nom & þú & þið   \\*
 & acc &  þig  & ykkur  \\*
 & dat & þér & ykkur   \\*
 & gen & þín  & ykkar  \\
\hline

\multirow{3}{*}{{{\textbf{pron} \Large{\textbf{17}}}}}  &  nom & það & þau   \\*
 & acc &  það  & þau  \\*
 & dat & því & þeim   \\*
 & gen & þess  & þeirra  \\
\hline
\end{tabular}
\end{minipage}\hfil
\begin{minipage}[t]{.45\textwidth}
\begin{tabular}{l>{\footnotesize\itshape}l>{\small}ll}
\hline
\multirow{3}{*}{{{\textbf{pron} \Large{\textbf{18}}}}}  &  nom &  &    \\*
 & acc &  sig  &   \\*
 & dat & sér &    \\*
 & gen & sín  &   \\
 \hline
\end{tabular}
\end{minipage}
\end{center}

\clearpage

\section{Číslovka}                               \label{sec:morpho_num}
 {\small\LTXfw{morpho/num}}
\clearpage

\begin{landscape}
\section{Sloveso}                                \label{sec:morpho_v}
{\small\LTXfw{morpho/v}}
\end{landscape}

\clearpage

\fi

% ============================================================== BACK MATTER =

\ifmakebackmatter

% ------------------------------------------------ Photographs authors index -

\cleardoublepage
%\pagestyle{empty}
%\pagestyle{indexstyle}
%\phantomsection
%\addcontentsline{toc}{chapter}
%  {Seznam autorů fotografií a ilustrací}        
\label{sec:photo}
\printindex[figures]

% ------------------------------------------------------------- Bibliography -
\cleardoublepage
\chapter{Bibliografie}
%\twocolumn
%\phantomsection
%\addcontentsline{toc}{chapter}{Bibliografie}
\begin{multicols}{2}
\nocite{*}\printbibliography[heading=none]
\end{multicols}

% ------------------------------------------------------------------ Apendix -
\cleardoublepage
\onecolumn
\ifPDF
\else
\chapter{Apendix: Islandská gramatika}
\clearpage
%\includepdf[pages={1,2,3,4,5,6}, turn=false]{gram.pdf}
%\includepdf[]{gram.pdf}
\covergeometry  
\makeatletter
\includepdf[
  pages={2,1,3,4,5,6,7,8,9,10,11,12,13,14,15,16,17,18,19,20,21,22,23,24,25,26,27,28,29,30,31,32,33,34,35,36,37,38,39,40,41,42,43,44,45,46,47,48,49,50,51,52,53,54,55,56,57,58,59,60,61,62,63,64,65,66,67,68,69,70,71,72,73},
  turn=false,
  width=\Gm@layoutwidth,
  height=\Gm@layoutheight,
  offset={\dimexpr(\Gm@layoutwidth-\paperwidth)/1+\Gm@layouthoffset\relax}
  {\dimexpr(\paperheight-\Gm@layoutheight)/2-\Gm@layoutvoffset\relax}
   ]{gram.pdf}
\makeatother
\restoregeometry  
\fi

% ------------------------------------------------------------------ Barcode -
% \EANisbn[SC5b] \EANisbn
\pagenumbering{gobble}
\ifPDF
\else
\cleardoublepage\null\clearpage % clear to verso
\vspace*\fill
\EANisbn[SC3]
\fi
\fi

% --------------------------------------------------------------- Back cover -

\ifmakecovers
  \cleardoublepage\null\clearpage % inner side of the cover
\covergeometry  
  \makecoverwith\backcoverimages
\restoregeometry  
\fi

\end{document}
